% --- Metadados (se preferir, mova estes \newcommand para o preâmbulo) ---
\newcommand{\autorLivro}{Arthur de Sá Ferreira}
\newcommand{\tituloLivro}{Ciência com R}
\newcommand{\localPublicacao}{Rio de Janeiro}
\newcommand{\editora}{Editora}
\newcommand{\anoPublicacao}{2025}
\newcommand{\isbn}{XXX-XX-XXXX-XXX-X}

\newcommand{\palavraschavea}{Estatística aplicada}
\newcommand{\palavraschaveb}{Metodologia científica}
\newcommand{\palavraschavec}{Análise de dados}
\newcommand{\palavraschaved}{R (Linguagem de programação)}
\newcommand{\palavraschavee}{Pesquisa científica}
\newcommand{\palavraschavef}{Modelagem estatística}
\newcommand{\palavraschaveg}{Reprodutibilidade científica}
\newcommand{\palavraschaveh}{Boas práticas em pesquisa}

% --- Posiciona a ficha no rodapé da página ---
\vspace*{\stretch{1}}

\hrule
\begin{center}
\begin{minipage}[c]{12.5cm}

% corpo da ficha
\autorLivro

\hspace{0.5cm} \tituloLivro. -- \localPublicacao: \editora, \anoPublicacao.

\hspace{0.5cm} \pageref{LastPage} p. : il. (alguma cor).

\hspace{0.5cm} ISBN \isbn

\hspace{0.5cm}
\begin{inparaenum}[1.]
  \item \palavraschavea.
  \item \palavraschaveb.
  \item \palavraschavec.
  \item \palavraschaved.
  \item \palavraschavee.
  \item \palavraschavef.
  \item \palavraschaveg.
  \item \palavraschaveh.
\end{inparaenum}

\begin{inparaenum}[I.]
  \item Título.
  \item Educação.
  \item Tecnologia.
\end{inparaenum}

\end{minipage}
\end{center}
\hrule

% pequeno respiro mantendo no pé da página
\vspace*{\stretch{0.1}}
