% Ícones
\usepackage{fontawesome5}

% Fonte do sistema (XeLaTeX/LuaLaTeX)
\usepackage{fontspec}
\setmainfont{Times New Roman}
\usepackage{amsmath}

% Figuras e tabelas
\usepackage{float}
\usepackage{booktabs}
\usepackage{longtable}
\usepackage{graphicx}
\usepackage[notransparent]{svg}

\usepackage[table]{xcolor}
\definecolor{lightred}{HTML}{FFCDD2}
\definecolor{lightyellow}{HTML}{FFF9C4}
\definecolor{lightgreen}{HTML}{E8F5E9}

% Ficha / utilidades
\usepackage{lastpage}
\usepackage{paralist}
\usepackage{microtype}
\usepackage{letltxmacro}
\usepackage{pdfpages}

% Cores (com 'orange')
\usepackage[dvipsnames]{xcolor}
\definecolor{lightgray}{gray}{0.95}

% Caixas coloridas
\usepackage{tcolorbox}

% Bibliografia no sumário e personalização de sumários
\usepackage{tocbibind}
\usepackage[titles]{tocloft}

% URLs e hyperlinks (hyperref por último)
\usepackage[spaces,obeyspaces,hyphens]{url}
\makeatletter
\g@addto@macro{\UrlBreaks}{\UrlOrds}
\makeatother
\usepackage[hidelinks,linktoc=all]{hyperref}

% Redefine \href para colocar URLs em notas de rodapé
\renewcommand{\href}[2]{#2\footnote{\url{#1}}}

% Caixa preta customizada
\newtcolorbox{blackbox}{
  colback=lightgray,
  colframe=orange,
  coltext=black,
  boxsep=5pt,
  arc=4pt}

\newenvironment{infobox}[1]{%
  \begin{list}{}{%
    \setlength{\leftmargin}{4em}   % igual ao padding-left do CSS
    \setlength{\labelsep}{1em}     % distância entre ícone e texto
    \setlength{\labelwidth}{3em}   % largura reservada ao ícone
    \setlength{\itemindent}{0pt}   % sem indent extra
    \setlength{\listparindent}{0pt}
    \setlength{\rightmargin}{0pt}
  }
  \renewcommand{\makelabel}[1]{%
    \raisebox{-.7\height}[0pt][0pt]{%
      {\setkeys{Gin}{width=3em,keepaspectratio}\includegraphics{#1}}}}
  \setlength{\fboxsep}{1em}
  \begin{blackbox}
  \item
}{%
  \end{blackbox}
  \end{list}
}

% Atalhos para (des)ativar notas de rodapé
\LetLtxMacro\Oldfootnote\footnote
\newcommand{\EnableFootNotes}{\LetLtxMacro\footnote\Oldfootnote}
\newcommand{\DisableFootNotes}{\renewcommand{\footnote}[2][]{\relax}}

% Atalhos para (des)ativar o título
\AtBeginDocument{\let\maketitle\relax}

% Define landscape environment to use with large PDF tables
\usepackage{tabularray}
\usepackage{pdflscape}
\newcommand{\blandscape}{\begin{landscape}}
\newcommand{\elandscape}{\end{landscape}}
% to Portuguese in long tables
\UseTblrLibrary{booktabs}
\DefTblrTemplate{contfoot-text}{default}{Continuado na próxima página}
\DefTblrTemplate{conthead-text}{default}{(Continuado)}

% Capítulos não numerados e paginação em algarismos romanos
\frontmatter
