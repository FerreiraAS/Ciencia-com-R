% Options for packages loaded elsewhere
\PassOptionsToPackage{unicode}{hyperref}
\PassOptionsToPackage{hyphens}{url}
\documentclass[
  brazilian,
  a4paper,
  twoside,
  openright]{book}
\usepackage{xcolor}
\usepackage[inner=30mm,outer=22mm,top=25mm,bottom=30mm,includehead,includefoot]{geometry}
\usepackage{amsmath,amssymb}
\setcounter{secnumdepth}{5}
\usepackage{iftex}
\ifPDFTeX
  \usepackage[T1]{fontenc}
  \usepackage[utf8]{inputenc}
  \usepackage{textcomp} % provide euro and other symbols
\else % if luatex or xetex
  \usepackage{unicode-math} % this also loads fontspec
  \defaultfontfeatures{Scale=MatchLowercase}
  \defaultfontfeatures[\rmfamily]{Ligatures=TeX,Scale=1}
\fi
\usepackage{lmodern}
\ifPDFTeX\else
  % xetex/luatex font selection
\fi
% Use upquote if available, for straight quotes in verbatim environments
\IfFileExists{upquote.sty}{\usepackage{upquote}}{}
\IfFileExists{microtype.sty}{% use microtype if available
  \usepackage[]{microtype}
  \UseMicrotypeSet[protrusion]{basicmath} % disable protrusion for tt fonts
}{}
\makeatletter
\@ifundefined{KOMAClassName}{% if non-KOMA class
  \IfFileExists{parskip.sty}{%
    \usepackage{parskip}
  }{% else
    \setlength{\parindent}{0pt}
    \setlength{\parskip}{6pt plus 2pt minus 1pt}}
}{% if KOMA class
  \KOMAoptions{parskip=half}}
\makeatother
\usepackage{color}
\usepackage{fancyvrb}
\newcommand{\VerbBar}{|}
\newcommand{\VERB}{\Verb[commandchars=\\\{\}]}
\DefineVerbatimEnvironment{Highlighting}{Verbatim}{commandchars=\\\{\}}
% Add ',fontsize=\small' for more characters per line
\usepackage{framed}
\definecolor{shadecolor}{RGB}{248,248,248}
\newenvironment{Shaded}{\begin{snugshade}}{\end{snugshade}}
\newcommand{\AlertTok}[1]{\textcolor[rgb]{0.94,0.16,0.16}{#1}}
\newcommand{\AnnotationTok}[1]{\textcolor[rgb]{0.56,0.35,0.01}{\textbf{\textit{#1}}}}
\newcommand{\AttributeTok}[1]{\textcolor[rgb]{0.13,0.29,0.53}{#1}}
\newcommand{\BaseNTok}[1]{\textcolor[rgb]{0.00,0.00,0.81}{#1}}
\newcommand{\BuiltInTok}[1]{#1}
\newcommand{\CharTok}[1]{\textcolor[rgb]{0.31,0.60,0.02}{#1}}
\newcommand{\CommentTok}[1]{\textcolor[rgb]{0.56,0.35,0.01}{\textit{#1}}}
\newcommand{\CommentVarTok}[1]{\textcolor[rgb]{0.56,0.35,0.01}{\textbf{\textit{#1}}}}
\newcommand{\ConstantTok}[1]{\textcolor[rgb]{0.56,0.35,0.01}{#1}}
\newcommand{\ControlFlowTok}[1]{\textcolor[rgb]{0.13,0.29,0.53}{\textbf{#1}}}
\newcommand{\DataTypeTok}[1]{\textcolor[rgb]{0.13,0.29,0.53}{#1}}
\newcommand{\DecValTok}[1]{\textcolor[rgb]{0.00,0.00,0.81}{#1}}
\newcommand{\DocumentationTok}[1]{\textcolor[rgb]{0.56,0.35,0.01}{\textbf{\textit{#1}}}}
\newcommand{\ErrorTok}[1]{\textcolor[rgb]{0.64,0.00,0.00}{\textbf{#1}}}
\newcommand{\ExtensionTok}[1]{#1}
\newcommand{\FloatTok}[1]{\textcolor[rgb]{0.00,0.00,0.81}{#1}}
\newcommand{\FunctionTok}[1]{\textcolor[rgb]{0.13,0.29,0.53}{\textbf{#1}}}
\newcommand{\ImportTok}[1]{#1}
\newcommand{\InformationTok}[1]{\textcolor[rgb]{0.56,0.35,0.01}{\textbf{\textit{#1}}}}
\newcommand{\KeywordTok}[1]{\textcolor[rgb]{0.13,0.29,0.53}{\textbf{#1}}}
\newcommand{\NormalTok}[1]{#1}
\newcommand{\OperatorTok}[1]{\textcolor[rgb]{0.81,0.36,0.00}{\textbf{#1}}}
\newcommand{\OtherTok}[1]{\textcolor[rgb]{0.56,0.35,0.01}{#1}}
\newcommand{\PreprocessorTok}[1]{\textcolor[rgb]{0.56,0.35,0.01}{\textit{#1}}}
\newcommand{\RegionMarkerTok}[1]{#1}
\newcommand{\SpecialCharTok}[1]{\textcolor[rgb]{0.81,0.36,0.00}{\textbf{#1}}}
\newcommand{\SpecialStringTok}[1]{\textcolor[rgb]{0.31,0.60,0.02}{#1}}
\newcommand{\StringTok}[1]{\textcolor[rgb]{0.31,0.60,0.02}{#1}}
\newcommand{\VariableTok}[1]{\textcolor[rgb]{0.00,0.00,0.00}{#1}}
\newcommand{\VerbatimStringTok}[1]{\textcolor[rgb]{0.31,0.60,0.02}{#1}}
\newcommand{\WarningTok}[1]{\textcolor[rgb]{0.56,0.35,0.01}{\textbf{\textit{#1}}}}
\usepackage{longtable,booktabs,array}
\usepackage{calc} % for calculating minipage widths
% Correct order of tables after \paragraph or \subparagraph
\usepackage{etoolbox}
\makeatletter
\patchcmd\longtable{\par}{\if@noskipsec\mbox{}\fi\par}{}{}
\makeatother
% Allow footnotes in longtable head/foot
\IfFileExists{footnotehyper.sty}{\usepackage{footnotehyper}}{\usepackage{footnote}}
\makesavenoteenv{longtable}
\usepackage{graphicx}
\makeatletter
\newsavebox\pandoc@box
\newcommand*\pandocbounded[1]{% scales image to fit in text height/width
  \sbox\pandoc@box{#1}%
  \Gscale@div\@tempa{\textheight}{\dimexpr\ht\pandoc@box+\dp\pandoc@box\relax}%
  \Gscale@div\@tempb{\linewidth}{\wd\pandoc@box}%
  \ifdim\@tempb\p@<\@tempa\p@\let\@tempa\@tempb\fi% select the smaller of both
  \ifdim\@tempa\p@<\p@\scalebox{\@tempa}{\usebox\pandoc@box}%
  \else\usebox{\pandoc@box}%
  \fi%
}
% Set default figure placement to htbp
\def\fps@figure{htbp}
\makeatother
% definitions for citeproc citations
\NewDocumentCommand\citeproctext{}{}
\NewDocumentCommand\citeproc{mm}{%
  \begingroup\def\citeproctext{#2}\cite{#1}\endgroup}
\makeatletter
 % allow citations to break across lines
 \let\@cite@ofmt\@firstofone
 % avoid brackets around text for \cite:
 \def\@biblabel#1{}
 \def\@cite#1#2{{#1\if@tempswa , #2\fi}}
\makeatother
\newlength{\cslhangindent}
\setlength{\cslhangindent}{1.5em}
\newlength{\csllabelwidth}
\setlength{\csllabelwidth}{3em}
\newenvironment{CSLReferences}[2] % #1 hanging-indent, #2 entry-spacing
 {\begin{list}{}{%
  \setlength{\itemindent}{0pt}
  \setlength{\leftmargin}{0pt}
  \setlength{\parsep}{0pt}
  % turn on hanging indent if param 1 is 1
  \ifodd #1
   \setlength{\leftmargin}{\cslhangindent}
   \setlength{\itemindent}{-1\cslhangindent}
  \fi
  % set entry spacing
  \setlength{\itemsep}{#2\baselineskip}}}
 {\end{list}}
\usepackage{calc}
\newcommand{\CSLBlock}[1]{\hfill\break\parbox[t]{\linewidth}{\strut\ignorespaces#1\strut}}
\newcommand{\CSLLeftMargin}[1]{\parbox[t]{\csllabelwidth}{\strut#1\strut}}
\newcommand{\CSLRightInline}[1]{\parbox[t]{\linewidth - \csllabelwidth}{\strut#1\strut}}
\newcommand{\CSLIndent}[1]{\hspace{\cslhangindent}#1}
\ifLuaTeX
\usepackage[bidi=basic]{babel}
\else
\usepackage[bidi=default]{babel}
\fi
% get rid of language-specific shorthands (see #6817):
\let\LanguageShortHands\languageshorthands
\def\languageshorthands#1{}
\setlength{\emergencystretch}{3em} % prevent overfull lines
\providecommand{\tightlist}{%
  \setlength{\itemsep}{0pt}\setlength{\parskip}{0pt}}
% Ícones
\usepackage{fontawesome5}

% Fonte do sistema (XeLaTeX/LuaLaTeX)
\usepackage{fontspec}
\setmainfont{Times New Roman}
\usepackage{amsmath}

% Figuras e tabelas
\usepackage{float}
\usepackage{booktabs}
\usepackage{longtable}
\usepackage{graphicx}
\usepackage[notransparent]{svg}
\DeclareGraphicsExtensions{.pdf,.png,.jpg}

\usepackage[table]{xcolor}
\definecolor{lightred}{HTML}{FFCDD2}
\definecolor{lightyellow}{HTML}{FFF9C4}
\definecolor{lightgreen}{HTML}{E8F5E9}

% Ficha / utilidades
\usepackage{lastpage}
\usepackage{paralist}
\usepackage{microtype}
\usepackage{letltxmacro}
\usepackage{pdfpages}

% Cores (com 'orange')
\usepackage[dvipsnames]{xcolor}
\definecolor{lightgray}{gray}{0.95}

% Caixas coloridas
\usepackage{tcolorbox}

% Bibliografia no sumário e personalização de sumários
\usepackage{tocbibind}
\usepackage[titles]{tocloft}

% URLs e hyperlinks (hyperref por último)
\usepackage[spaces,obeyspaces,hyphens]{url}
\makeatletter
\g@addto@macro{\UrlBreaks}{\UrlOrds}
\makeatother
\usepackage[hidelinks,linktoc=all]{hyperref}

% Redefine \href para colocar URLs em notas de rodapé
\renewcommand{\href}[2]{#2\footnote{\url{#1}}}

% Caixa preta customizada
\newtcolorbox{blackbox}{
  colback=lightgray,
  colframe=orange,
  coltext=black,
  boxsep=5pt,
  arc=4pt}

\newenvironment{infobox}[1]{%
  \begin{list}{}{%
    \setlength{\leftmargin}{4em}   % igual ao padding-left do CSS
    \setlength{\labelsep}{1em}     % distância entre ícone e texto
    \setlength{\labelwidth}{3em}   % largura reservada ao ícone
    \setlength{\itemindent}{0pt}   % sem indent extra
    \setlength{\listparindent}{0pt}
    \setlength{\rightmargin}{0pt}
  }
  \renewcommand{\makelabel}[1]{%
    \raisebox{-.7\height}[0pt][0pt]{%
      {\setkeys{Gin}{width=3em,keepaspectratio}\includegraphics{#1}}}}
  \setlength{\fboxsep}{1em}
  \begin{blackbox}
  \item
}{%
  \end{blackbox}
  \end{list}
}

% Atalhos para (des)ativar notas de rodapé
\LetLtxMacro\Oldfootnote\footnote
\newcommand{\EnableFootNotes}{\LetLtxMacro\footnote\Oldfootnote}
\newcommand{\DisableFootNotes}{\renewcommand{\footnote}[2][]{\relax}}

% Atalhos para (des)ativar o título
\AtBeginDocument{\let\maketitle\relax}

% Define landscape environment to use with large PDF tables
\usepackage{tabularray}
\usepackage{pdflscape}
\newcommand{\blandscape}{\begin{landscape}}
\newcommand{\elandscape}{\end{landscape}}
% to Portuguese in long tables
\UseTblrLibrary{booktabs}
\DefTblrTemplate{contfoot-text}{default}{Continuado na próxima página}
\DefTblrTemplate{conthead-text}{default}{(Continuado)}

% Capítulos não numerados e paginação em algarismos romanos
\frontmatter
% Marca d'água
\usepackage{draftwatermark}
\SetWatermarkLightness{0.9}
\SetWatermarkText{RASCUNHO}
\SetWatermarkScale{0.9}
\usepackage{booktabs}
\usepackage{longtable}
\usepackage{array}
\usepackage{multirow}
\usepackage{wrapfig}
\usepackage{float}
\usepackage{colortbl}
\usepackage{pdflscape}
\usepackage{tabu}
\usepackage{threeparttable}
\usepackage{threeparttablex}
\usepackage[normalem]{ulem}
\usepackage{makecell}
\usepackage{xcolor}
\usepackage{caption}
\usepackage{anyfontsize}
\usepackage{tabularray}
\usepackage[normalem]{ulem}
\usepackage{graphicx}
\usepackage{rotating}
\UseTblrLibrary{siunitx}
\NewTableCommand{\tinytableDefineColor}[3]{\definecolor{#1}{#2}{#3}}
\newcommand{\tinytableTabularrayUnderline}[1]{\underline{#1}}
\newcommand{\tinytableTabularrayStrikeout}[1]{\sout{#1}}
\usepackage{bookmark}
\IfFileExists{xurl.sty}{\usepackage{xurl}}{} % add URL line breaks if available
\urlstyle{same}
\hypersetup{
  pdftitle={Ciência com R},
  pdflang={pt-BR},
  hidelinks,
  pdfcreator={LaTeX via pandoc}}

\title{\textbf{Ciência com R}}
\usepackage{etoolbox}
\makeatletter
\providecommand{\subtitle}[1]{% add subtitle to \maketitle
  \apptocmd{\@title}{\par {\large #1 \par}}{}{}
}
\makeatother
\subtitle{Perguntas e respostas para pesquisadores e analistas de dados}
\author{© 2023--2026 by Arthur de Sá Ferreira
https://orcid.org/0000-0001-7014-2002
@cienciacomr}
\date{}

\begin{document}
\maketitle

\includepdf[pages={1},noautoscale=false, fitpaper=true, width=8.27in, height=11.69in, frame=false, trim=0mm 0mm 0mm 0mm, clip, offset=0mm 0mm]{covers/Cover_1.pdf}
\newpage

\includepdf[pages={1},noautoscale=false, fitpaper=true, width=8.27in, height=11.69in, frame=false, trim=0mm 0mm 0mm 0mm, clip, offset=0mm 0mm]{covers/Cover_2.pdf}
\newpage

\newcommand{\copyrightnotice}{
  \begin{center}
    \small
    Copyright © 2023--\the\year\ Arthur de Sá Ferreira

    \vspace{1em}

    Publicado em Niterói, Brasil \\
    Edição do autor / Publicação independente

    \vspace{1em}

    Todos os direitos reservados. Nenhuma parte deste livro pode ser reproduzida
    ou transmitida por qualquer meio eletrônico, mecânico, fotocópia, gravação
    ou outro, sem a permissão prévia por escrito do autor, exceto nos casos
    previstos pela lei de direitos autorais.

    \vspace{1em}

    Nenhuma garantia é dada em relação ao conteúdo desta obra, incluindo
    adequação a finalidades específicas. O uso é de responsabilidade
    exclusiva do leitor.

    \vspace{1em}

    As opiniões expressas nesta obra são de responsabilidade exclusiva do autor.

    \vspace{1em}

    Para solicitar permissões, entre em contato: \texttt{cienciacomr@gmail.com}

    \vspace{2em}

    1ª edição — \the\year \\
    Capa dura: ISBN \\
    Brochura: ISBN \\
    E-book: ISBN \\

    \vspace{1em}
    Depósito legal: Biblioteca Nacional, Brasil
  \end{center}
}

\copyrightnotice

\newpage


% --- Metadados (se preferir, mova estes \newcommand para o preâmbulo) ---
\newcommand{\autorLivro}{Arthur de Sá Ferreira}
\newcommand{\tituloLivro}{Ciência com R}
\newcommand{\localPublicacao}{Rio de Janeiro}
\newcommand{\editora}{Editora}
\newcommand{\anoPublicacao}{2026}
\newcommand{\isbn}{XXX-XX-XXXX-XXX-X}

\newcommand{\palavraschavea}{Estatística aplicada}
\newcommand{\palavraschaveb}{Metodologia científica}
\newcommand{\palavraschavec}{Análise de dados}
\newcommand{\palavraschaved}{R (Linguagem de programação)}
\newcommand{\palavraschavee}{Pesquisa científica}
\newcommand{\palavraschavef}{Modelagem estatística}
\newcommand{\palavraschaveg}{Reprodutibilidade científica}
\newcommand{\palavraschaveh}{Boas práticas em pesquisa}

% --- Posiciona a ficha no rodapé da página ---
\vspace*{\stretch{1}}

\hrule
\begin{center}
\begin{minipage}[c]{12.5cm}

% corpo da ficha
\autorLivro

\hspace{0.5cm} \tituloLivro. -- \localPublicacao: \editora, \anoPublicacao.

\hspace{0.5cm} \pageref{LastPage} p. : il. (alguma cor).

\hspace{0.5cm} ISBN \isbn

\hspace{0.5cm}
\begin{inparaenum}[1.]
  \item \palavraschavea.
  \item \palavraschaveb.
  \item \palavraschavec.
  \item \palavraschaved.
  \item \palavraschavee.
  \item \palavraschavef.
  \item \palavraschaveg.
  \item \palavraschaveh.
\end{inparaenum}

\begin{inparaenum}[I.]
  \item Título.
  \item Educação.
  \item Tecnologia.
\end{inparaenum}

\end{minipage}
\end{center}
\hrule

% pequeno respiro mantendo no pé da página
\vspace*{\stretch{0.1}}

% --- Fim da ficha catalográfica ---

{
\setcounter{tocdepth}{1}
\tableofcontents
}
\listoffigures
\listoftables
\DisableFootNotes

\clearpage
\markboth{}{}

\vspace*{\fill}

\chapter*{\texorpdfstring{\textbf{Dedicatória}}{Dedicatória}}\label{dedicatoria}
\addcontentsline{toc}{chapter}{\textbf{Dedicatória}}

\markboth{}{}

Esta obra é dedicada a todos que, em princípio, buscam conhecimento para melhorar a qualidade da pesquisa científica --- seja a sua própria, a de colegas ou a de desconhecidos --- mas, em última análise, desejam mesmo prover melhores condições de saúde e desenvolvimento da sociedade.

Dedico também ao leitor eventual que chegou aqui por acaso.

\chapter*{\texorpdfstring{\textbf{Agradecimentos}}{Agradecimentos}}\label{agradecimentos}
\addcontentsline{toc}{chapter}{\textbf{Agradecimentos}}

\markboth{}{}

Este trabalho não seria possível sem o apoio e suporte da minha esposa Daniele, minha irmã Mônica, meu pai José Victorino, minha mãe Angela (\emph{in memoriam}) e meus filhos Giovanna, Victor e Lucas.

\chapter*{\texorpdfstring{\textbf{Apresentação}}{Apresentação}}\label{apresentacao}
\addcontentsline{toc}{chapter}{\textbf{Apresentação}}

\markboth{}{}

No âmbito da análise estatística de dados, os processos envolvidos são marcados por uma série de escolhas críticas. Estas decisões abrangem considerações metodológicas e ações operacionais que moldam toda a jornada analítica. Deve-se selecionar, cuidadosamente, um delineamento de estudo para enfrentar os desafios únicos colocados por um projeto de pesquisa. Além disso, a escolha de métodos estatísticos adequados para lidar com os dados gerados pelo delineamento escolhido tem um peso importante. Estas decisões necessitam de uma base construída sobre as evidências mais convincentes da literatura existente e na adesão a práticas sólidas de investigação.

Interpretar os resultados destas análises não é uma tarefa simples. Confiar apenas na formação educacional convencional, no bom senso e na intuição para decifrar tabelas e gráficos pode revelar-se inadequado. Interpretações errôneas podem gerar consequências indesejáveis, incluindo a utilização de testes diagnósticos imprecisos ou o endosso de tratamentos ineficazes.

Este livro emerge do reconhecimento desses desafios.

A proposta gira em torno da organização de um compêndio abrangente de métodos e técnicas de ponta, para análise estatística de dados em pesquisa científica, apresentados em formato de perguntas e respostas. Esse formato promove um diálogo direto e objetivo com o leitor, respondendo a dúvidas comumente colocadas por alunos de graduação, pós-graduação lato sensu, pós-graduação stricto sensu (mestrado e doutorado), bem como por pesquisadores.

O objetivo geral de cada capítulo é elucidar as questões metodológicas fundamentais: \emph{``O que é?''}, \emph{``Por que usar?''}, \emph{``Quando usar?''}, \emph{``Quando não usar?''} e \emph{``Como fazer?''}. Em cada capítulo, diversas questões específicas são propostas e respondidas sistematicamente, permitindo ao leitor uma melhor elaboração do conteúdo e resultado do seu trabalho. Todos eles com citações de fontes confiáveis referências, que podem ser consultadas para aprofundamento e verificação das informações apresentadas --- um total de 512 referências foram incluídas para apoiar as informações e recomendações apresentadas.

Os capítulos foram organizados para seguir uma progressão de conceitos e aplicações. Embora sejam fragmentados para maior clareza instrucional, as referências cruzadas ajudam a mitigar a fragmentação do conteúdo e reforçar a interconexão dos tópicos.

O público-alvo compreende pesquisadores, professores, analistas de dados, profissionais e estudantes que regularmente lidam com a tomada de decisões em pesquisa. Os estudantes de pós-graduação encontrarão aqui uma obra repleta de exemplos para adaptar na análise dos dados de seus projetos de pesquisa. Professores de graduação e pós-graduação terão acesso a uma obra didática de referência, direcionada para seus alunos. Pesquisadores e analistas de dados iniciantes descobrirão um valioso acervo de informações e referências para a construção de projetos e manuscritos. Pesquisadores e os cientistas mais experientes podem recorrer às referências e esclarecimentos mais atuais sobre vieses, paradoxos, mitos e mal práticas em pesquisa. E mesmo os leitores não familiarizados ainda com as técnicas de análise de dados em pesquisa terão a oportunidade de apreciar o papel fundamental de colocar e responder suas perguntas na busca do conhecimento científico.

Arthur de Sá Ferreira

\chapter*{\texorpdfstring{\textbf{Sobre o autor}}{Sobre o autor}}\label{author}
\addcontentsline{toc}{chapter}{\textbf{Sobre o autor}}

\markboth{}{}

\pandocbounded{\includegraphics[keepaspectratio]{images/ASF.png}}

\textbf{Arthur de Sá Ferreira}

Obtive minha Graduação em Fisioterapia pela Universidade Federal do Rio de Janeiro (1999), Formação em Acupuntura pela Academia Brasileira de Arte e Ciência Oriental (2001), Mestrado em Engenharia Biomédica pela Universidade Federal do Rio de Janeiro (2002) e Doutorado em Engenharia Biomédica pela Universidade Federal do Rio de Janeiro (2006).

Tenho experiência em docência no ensino superior, atuei com professor da graduação em cursos de Fisioterapia, Enfermagem e Odontologia, entre outros (2001-2018); pós-graduação \emph{lato sensu} em Fisioterapia (2001-atual) e \emph{stricto sensu} níveis mestrado e doutorado (2010-atual).

Como pesquisador, sou Professor Adjunto do Centro Universitário Augusto Motta (\href{https://www.unisuam.edu.br}{UNISUAM}), atuando nos Programas de Pós-graduação em Ciências da Reabilitação (\href{https://www.unisuam.edu.br/programa-pos-graduacao-ciencias-da-reabilitacao}{PPGCR}; 2009-atual) e Desenvolvimento Local (\href{https://www.unisuam.edu.br/programa-pos-graduacao-desenvolvimento-local}{PPGDL}; 2018-atual). Também sou pesquisador do Instituto D'Or de Pesquisa e Ensino (\href{https://www.rededorsaoluiz.com.br/instituto/idor}{IDOR}; 2024-atual). Fundei o Laboratório de Simulação Computacional e Modelagem em Reabilitação (LSCMR) em 2012, onde desenvolvo projetos de pesquisa principalmente nos seguintes temas: Bioestatística, Modelagem e simulação computacional, Processamento de sinais biomédicos, Movimento funcional humano, Medicina tradicional (chinesa), Distúrbios musculoesqueléticos, Doenças cardiovasculares e Doenças respiratórias.

Dentre os editais públicos que obtive financiamento, destaco os Programas Jovem Cientista do Nosso Estado (JCNE; 2012-2015; 2015-2017) e Cientista do Nosso Estado (2021-atual) da Fundação Carlos Chagas Filho de Amparo à Pesquisa do Estado do Rio de Janeiro (\href{https://www.faperj.br}{FAPERJ}; e Bolsista Produtividade em Pesquisa pelo Conselho Nacional de Desenvolvimento Científico e Tecnológico (\href{http://www.cnpq.br}{CNPq}; 2021-atual).

Como gestor, estou na Coordenação do Programa de Pós-Graduação \emph{stricto sensu} em Ciências da Reabilitação (\href{https://www.unisuam.edu.br/programa-pos-graduacao-ciencias-da-reabilitacao}{PPGCR}; 2016-atual). Coordeno o Curso Superior de Tecnologia em Radiologia da Faculdade IDOR de Ciências Médicas (\href{https://www.rededorsaoluiz.com.br/instituto/idor/cursos/vestibular-agendado-curso-superior-de-tecnologia-em-radiologia/}{IDOR}; 2024-atual). Atuei como coordenador do Comitê de Ética em Pesquisa (CEP) do Centro Universitário Augusto Motta (\href{https://www.unisuam.edu.br}{UNISUAM}; 2020-2024) e como Coordenador do Curso de Graduação em Fisioterapia da Universidade Salgado de Oliveira (\href{https://universo.edu.br}{UNIVERSO}; 2004-2009).

Sou membro da Associação Brasileira de Pesquisa e Pós-Graduação em Fisioterapia (\href{https://abrapg-ft.org.br/portal/}{ABRAPG-FT}) (2007-atual), Consórcio Acadêmico Brasileiro de Saúde Integrativa (\href{https://cabsin.org.br}{CABSIN}) (2019-atual), e Royal Statistical Society (\href{https://rss.org.uk}{RSS}) (2021-atual). Fui membro do Committee on Publication Ethics (\href{https://publicationethics.org}{COPE}) (2018-2024).

Componho o corpo editorial e de revisores de periódicos nacionais e internacionais como \href{https://www.nature.com/srep/about/editors}{Scientific Reports}, \href{https://www.frontiersin.org/research-topics/26395/systemic-effects-and-disabilities-in-long-covid-syndrome-current-approaches-and-clinical-challenges}{Frontiers in Rehabilitation Sciences}, \href{https://onlinelibrary.wiley.com/journal/17517176}{The Journal of Clinical Hypertension}, \href{https://www.springer.com/journal/11655/editors}{Chinese Journal of Integrative Medicine}, \href{https://www.journals.elsevier.com/journal-of-integrative-medicine/editorial-board}{Journal of Integrative Medicine}, \href{https://www.sciencedirect.com/journal/brazilian-journal-of-physical-therapy}{Brazilian Journal of Physical Therapy}, \href{https://www.scielo.br/journal/fp/about/\#editors}{Fisioterapia e Pesquisa}.

% normal chapter numbering and arabic page numbering
\mainmatter

\cftaddtitleline{toc}{chapter}{\rule{\textwidth}{0.4pt}}{}

\chapter*{\texorpdfstring{\emph{PARTE 1: PENSAMENTO CIENTÍFICO}}{PARTE 1: PENSAMENTO CIENTÍFICO}}\label{parte-1}
\addcontentsline{toc}{chapter}{\emph{PARTE 1: PENSAMENTO CIENTÍFICO}}

\par\noindent\rule{\textwidth}{0.05in}

\section*{Conceitos essenciais para pensar cientificamente e evitar armadilhas comuns}\label{conceitos-essenciais-para-pensar-cientificamente-e-evitar-armadilhas-comuns}

\markboth{}{}

\EnableFootNotes

\chapter{\texorpdfstring{\textbf{Pensamento probabilístico}}{Pensamento probabilístico}}\label{pensamento-probabilistico}

\section{Experimento}\label{experimento}

\subsection{O que é um experimento?}\label{o-que-uxe9-um-experimento}

\begin{itemize}
\item
  Um experimento é um processo de simulação ou medição cujo resultado é chamado de desfecho.\textsuperscript{\citeproc{ref-grami2023}{1}}
\item
  Tentativa se refere a uma repetição de um experimento.\textsuperscript{\citeproc{ref-grami2023}{1}}
\end{itemize}

\subsection{O que é um experimento aleatório?}\label{o-que-uxe9-um-experimento-aleatuxf3rio}

\begin{itemize}
\tightlist
\item
  Em um experimento aleatório, o desfecho de cada tentativa é imprevisível.\textsuperscript{\citeproc{ref-grami2023}{1}}
\end{itemize}

\section{Espaço amostral e eventos discretos}\label{espauxe7o-amostral-e-eventos-discretos}

\subsection{O que é espaço amostral discreto?}\label{o-que-uxe9-espauxe7o-amostral-discreto}

\begin{itemize}
\item
  O espaço amostral \(S\) de um experimento aleatório é definido como o conjunto de todos os desfechos possíveis de um experimento.\textsuperscript{\citeproc{ref-grami2023}{1}}
\item
  Em probabilidade discreta, o espaço amostral \(S\) pode ser enumerado e contado.\textsuperscript{\citeproc{ref-grami2023}{1}}
\end{itemize}

\begin{figure}

{\centering \includegraphics{Ciencia-com-R_files/figure-latex/espaco-amostral-dado-1} 

}

\caption{Exemplos de espaço amostral discreto. Superior: Todas as faces de uma moeda. Inferior: Todas as faces de um dado.}\label{fig:espaco-amostral-dado}
\end{figure}

\subsection{O que é evento discreto?}\label{o-que-uxe9-evento-discreto}

\begin{itemize}
\item
  Um evento \(E\) é um único desfecho ou uma coleção de desfechos.\textsuperscript{\citeproc{ref-grami2023}{1}}
\item
  Um evento \(E\) é um subconjunto do espaço amostral \(S\) de um experimento.\textsuperscript{\citeproc{ref-grami2023}{1}}
\end{itemize}

\begin{figure}

{\centering \includegraphics{Ciencia-com-R_files/figure-latex/evento-dado-1} 

}

\caption{Exemplos de evento de experimento. Superior: 1 lançamento de 1 moeda. Inferior: 1 lançamento de 1 dado.}\label{fig:evento-dado}
\end{figure}

\subsection{O que é espaço de eventos discretos?}\label{o-que-uxe9-espauxe7o-de-eventos-discretos}

\begin{itemize}
\item
  Um espaço de eventos \(E_{s}\) também é um subconjunto do espaço amostral \(S\) de um experimento.\textsuperscript{\citeproc{ref-grami2023}{1}}
\item
  A união de dois eventos \(E_{1} \cup E_{2}\) é o conjunto de todos os desfechos que estão em \(E_{1}\), em \(E_{2}\) ou em ambos.\textsuperscript{\citeproc{ref-grami2023}{1}}
\item
  A intersecção de dois eventos \(E_{1} \cap E_{2}\), ou evento conjunto, é o conjunto de todos os desfechos que estão em ambos os eventos.\textsuperscript{\citeproc{ref-grami2023}{1}}
\item
  O complemento de um evento \(E^C\) consiste em todos os desfechos que não estão incluídos no evento \(E\).\textsuperscript{\citeproc{ref-grami2023}{1}}
\end{itemize}

\begin{figure}

{\centering \includegraphics{Ciencia-com-R_files/figure-latex/espaco-eventos-dado-1} 

}

\caption{Espaço de eventos: União dos eventos face = 3 e face = 4 de um dado.}\label{fig:espaco-eventos-dado}
\end{figure}

\section{Espaço amostral e eventos contínuos}\label{espauxe7o-amostral-e-eventos-contuxednuos}

\subsection{O que é espaço amostral contínuo?}\label{o-que-uxe9-espauxe7o-amostral-contuxednuo}

\begin{itemize}
\tightlist
\item
  .\textsuperscript{\citeproc{ref-REF}{\textbf{REF?}}}
\end{itemize}

\subsection{O que é evento contínuo?}\label{o-que-uxe9-evento-contuxednuo}

\begin{itemize}
\tightlist
\item
  .\textsuperscript{\citeproc{ref-REF}{\textbf{REF?}}}
\end{itemize}

\subsection{O que é espaço de eventos contínuo?}\label{o-que-uxe9-espauxe7o-de-eventos-contuxednuo}

\begin{itemize}
\tightlist
\item
  .\textsuperscript{\citeproc{ref-REF}{\textbf{REF?}}}
\end{itemize}

\section{Probabilidade}\label{probabilidade}

\subsection{O que é probabilidade?}\label{o-que-uxe9-probabilidade}

\begin{itemize}
\tightlist
\item
  Com um espaço amostral \(S\) finito e não vazio de desfechos igualmente prováveis, a probabilidade \(P\) de um evento \(E\) é a razão entre o número de desfechos no evento \(E\) e o número de desfechos no espaço amostral \(S\) \eqref{eq:p}.\textsuperscript{\citeproc{ref-grami2023}{1}}
\end{itemize}

\begin{equation}
\label{eq:p}
P(E) = \frac{\text{número de desfechos em } E}{\text{número de desfechos em } S}
\end{equation}

\begin{itemize}
\item
  Um evento \(E\) impossível não contém um desfecho e, portanto, nunca ocorre: \(P(E)=0\).\textsuperscript{\citeproc{ref-grami2023}{1},\citeproc{ref-Viti2015}{2}}
\item
  Um evento \(E\) é certo consiste em qualquer um dos desfechos possíveis e, portanto, sempre ocorre: \(P(E)=1\).\textsuperscript{\citeproc{ref-grami2023}{1}}
\end{itemize}

\subsection{Quais são os axiomas da probabilidade?}\label{quais-suxe3o-os-axiomas-da-probabilidade}

\begin{itemize}
\item
  A probabilidade de um evento é um número real que satisfaz os seguintes axiomas descritos por Andrei Nikolaevich Kolmogorov em 1950\textsuperscript{\citeproc{ref-grami2023}{1},\citeproc{ref-Viti2015}{2}}
\item
  Axioma I. Probabilidades de um evento \(E\) são números não-negativos: \(P(E) \geq 0\).\textsuperscript{\citeproc{ref-grami2023}{1},\citeproc{ref-Viti2015}{2}}
\item
  Axioma II. Probabilidade de todos os eventos do espaço amostral \(A\) ocorrerem é 100\%: \(P(S)=1\).\textsuperscript{\citeproc{ref-grami2023}{1},\citeproc{ref-Viti2015}{2}}
\item
  Axioma III. A probabilidade de um conjunto \emph{k} de eventos mutuamente exclusivos é igual a soma da probabilidade de cada evento: \(P(E_{1} \cup E_{2} \cup ... E_{k}) = P(E_{1}) + P(E_{2}) + ... + P(E_{k})\).\textsuperscript{\citeproc{ref-grami2023}{1},\citeproc{ref-Viti2015}{2}}
\end{itemize}

\subsection{Quais as consequências dos axiomas da probabilidade?}\label{quais-as-consequuxeancias-dos-axiomas-da-probabilidade}

\begin{itemize}
\item
  A soma da probabilidade de dois eventos que dividem o espaço amostral é \(100\%\): \(P(E)+P(E)^C=1\).\textsuperscript{\citeproc{ref-grami2023}{1}}
\item
  O valor máximo de probabilidade de um evento é 100\%: \(P(S) \leq 1\).\textsuperscript{\citeproc{ref-grami2023}{1}}
\item
  A probabilidade é uma função não decrescente do número de desfechos de um evento.\textsuperscript{\citeproc{ref-grami2023}{1}}
\end{itemize}

\section{Independência e probabilidade}\label{independuxeancia-e-probabilidade}

\subsection{O que é independência em estatística?}\label{o-que-uxe9-independuxeancia-em-estatuxedstica}

\begin{itemize}
\item
  Em experimentos aleatórios, é comum assumir que os eventos de tentativas separadas são independentes devido a independência física de eventos e experimentos.\textsuperscript{\citeproc{ref-grami2023}{1}}
\item
  Se a ocorrência do evento \(E_{1}\) não tiver efeito na ocorrência do evento \(E_{2}\), os eventos \(E_{1}\) e \(E_{2}\) são considerados estatisticamente independentes.\textsuperscript{\citeproc{ref-grami2023}{1}}
\item
  Eventos são mutuamente exclusivos, ou disjuntos, se a ocorrência de um exclui a ocorrência dos outros.\textsuperscript{\citeproc{ref-grami2023}{1}}
\item
  Se dois eventos \(E_{1}\) e \(E_{2}\) são mutuamente exclusivos, então os eventos \(E_{1}\) e \(E_{2}\) não podem ocorrer ao mesmo tempo e, portanto, são eventos independentes.\textsuperscript{\citeproc{ref-grami2023}{1}}
\end{itemize}

\begin{figure}

{\centering \includegraphics{Ciencia-com-R_files/figure-latex/independencia-venn-1} 

}

\caption{Superior: Eventos independentes. Inferior: Eventos dependentes.}\label{fig:independencia-venn}
\end{figure}

\begin{itemize}
\tightlist
\item
  Em experimentos independentes, o desfecho de uma tentativa é independente dos desfechos de outras tentativas, passadas e/ou futuras. Uma tentativa em um experimento aleatório é independente se a probabilidade de cada desfecho possível não mudar de tentativa para tentativa.\textsuperscript{\citeproc{ref-grami2023}{1}}
\end{itemize}

\begin{figure}

{\centering \includegraphics{Ciencia-com-R_files/figure-latex/independencia-dado-1} 

}

\caption{Esquerda: Evento (face = 4). Direita: Experimentos de 1 lançamento de 1 dado (superior), 3 lançamentos de 1 dado (central), 10 lançamentos de 1 dado (inferior).}\label{fig:independencia-dado}
\end{figure}

\subsection{O que é probabilidade marginal?}\label{o-que-uxe9-probabilidade-marginal}

\begin{itemize}
\tightlist
\item
  Probabilidade marginal é a probabilidade de ocorrência de um evento \(E\) independentemente da(s) probabilidade(s) de outro(s) evento(s).\textsuperscript{\citeproc{ref-grami2023}{1}}
\end{itemize}

\subsection{O que é probabilidade conjunta?}\label{o-que-uxe9-probabilidade-conjunta}

\begin{itemize}
\item
  Probabilidade conjunta é a probabilidade de ocorrência de dois ou mais eventos independentes \(E_{1}\), \(E_{2}\), \ldots, \(E_{k}\), independentemente da(s) probabilidade(s) de outro(s) evento(s).\textsuperscript{\citeproc{ref-grami2023}{1}}
\item
  Se a probabilidade conjunta dos eventos é nula (\(E_{1} \cup E_{2} = 0\)), esses dois eventos \(E_{1}\) e \(E_{2}\) são mutuamente exclusivos ou disjuntos.\textsuperscript{\citeproc{ref-grami2023}{1}}
\end{itemize}

\subsection{O que é probabilidade condicional?}\label{o-que-uxe9-probabilidade-condicional}

\begin{itemize}
\item
  Probabilidade condicional é a probabilidade de ocorrência do evento \(E_{2}\) quando se sabe que o evento \(E_{1}\) já ocorreu \(P(E_{2} | E_{1})\).\textsuperscript{\citeproc{ref-grami2023}{1}}
\item
  A probabilidade condicional \(P(E_{2} | E_{1})\) representa que a ocorrência do evento \(E_{1}\) fornece informação sobre a ocorrência do evento \(E_{2}\).\textsuperscript{\citeproc{ref-grami2023}{1}}
\item
  Se a ocorrência do evento \(E_{1}\) tiver alguma influência na ocorrência do evento \(E_{2}\), então a probabilidade condicional do evento \(E_{2}\) dado o evento \(E_{1}\) pode ser maior ou menor do que a probabilidade marginal.\textsuperscript{\citeproc{ref-grami2023}{1}}
\end{itemize}

\section{Leis dos números anômalos}\label{leis-dos-nuxfameros-anuxf4malos}

\subsection{O que é a lei dos números anômalos?}\label{o-que-uxe9-a-lei-dos-nuxfameros-anuxf4malos}

\begin{itemize}
\tightlist
\item
  A lei dos números anômalos --- lei de Benford --- é uma distribuição de probabilidade que descreve a frequência de ocorrência do primeiro dígito em muitos conjuntos de dados do mundo real.\textsuperscript{\citeproc{ref-Benford1938}{3}}
\end{itemize}

\section{Leis dos pequenos números}\label{leis-dos-pequenos-nuxfameros}

\subsection{O que é a lei dos pequenos números?}\label{o-que-uxe9-a-lei-dos-pequenos-nuxfameros}

\begin{itemize}
\item
  A crença exagerada na probabilidade de replicar com sucesso os achados de um estudo, pela tendência de se considerar uma amostra como representativa da população.\textsuperscript{\citeproc{ref-tversky1971}{4}}
\item
  A crença na lei dos pequenos números se refere à tendência de superestimar a estabilidade das estimativas provenientes de estudos com amostras pequenas.\textsuperscript{\citeproc{ref-bishop2022}{5}}
\item
  Quando se percebe um padrão, pode não ser possível identificar se tal padrão é real.\textsuperscript{\citeproc{ref-guy1988}{6}}
\end{itemize}

\subsection{Quais são as versões da lei dos pequenos números?}\label{quais-suxe3o-as-versuxf5es-da-lei-dos-pequenos-nuxfameros}

\begin{itemize}
\item
  1a Lei Forte dos Pequenos Números: ``Não há pequenos números suficientes para atender às muitas demandas que lhes são feitas''.\textsuperscript{\citeproc{ref-guy1988}{6}}
\item
  2a Lei Forte dos Pequenos Números: ``Quando dois números parecem iguais, não são necessariamente assim''.\textsuperscript{\citeproc{ref-guy1990}{7}}
\end{itemize}

\section{Leis dos grandes números}\label{leis-dos-grandes-nuxfameros}

\subsection{O que é a lei dos grandes números?}\label{o-que-uxe9-a-lei-dos-grandes-nuxfameros}

\begin{itemize}
\item
  A lei dos grandes números descreve que, ao realizar o mesmo experimento \(E\) um grande número de vezes (\(n\)), a média \(\mu\) dos resultados obtidos tende a se aproximar do valor esperado \(E[\bar{X}]\) à medida que mais experimentos forem realizados (\(n \to \infty\)).\textsuperscript{\citeproc{ref-REF}{\textbf{REF?}}}
\item
  De acordo com a lei dos grandes números, a média amostral converge para a média populacional à medida que o tamanho da amostra aumenta.\textsuperscript{\citeproc{ref-REF}{\textbf{REF?}}}
\end{itemize}

\subsection{Quais são as versões da lei dos grandes números?}\label{quais-suxe3o-as-versuxf5es-da-lei-dos-grandes-nuxfameros}

\begin{itemize}
\item
  Lei Fraca dos Grandes Números (de Poisson): ``\,``.\textsuperscript{\citeproc{ref-REF}{\textbf{REF?}}}
\item
  Lei Fraca dos Grandes Números (de Bernoulli): ``\,``.\textsuperscript{\citeproc{ref-REF}{\textbf{REF?}}}
\item
  Lei Forte dos Grandes Números: ``\,``.\textsuperscript{\citeproc{ref-REF}{\textbf{REF?}}}
\end{itemize}

\section{Teorema central do limite}\label{teorema-central-do-limite}

\subsection{O que é teorema central do limite?}\label{o-que-uxe9-teorema-central-do-limite}

\begin{itemize}
\tightlist
\item
  O teorema central do limite \eqref{eq:central-limit-theorem} afirma que, para uma amostra aleatória de tamanho \(n\) de uma população com valor esperado igual à média \(E[\bar{X_{i}}] = \mu\) e variância \(Var[\bar{X_{i}}]\) igual a \(\sigma^{2}\), a distribuição amostral da média de uma variável \(\bar{X}\) se aproxima de uma distribuição normal \(N\) com média \(\mu\) e variância \(\sigma^{2}/n\) à medida que \(n\) aumenta (\(n \to \infty\)):\textsuperscript{\citeproc{ref-kwak2017}{8}}
\end{itemize}

\begin{equation}
\label{eq:central-limit-theorem}
\sqrt{n}(\bar{X} - \mu) \xrightarrow{n \to \infty} N(0, \sigma^2)
\end{equation}

\begin{itemize}
\item
  O teorema central do limite demonstra que se o tamanho da amostra \(n\) for suficientemente grande, a distribuição amostral das médias obtidas utilizando reamostragem com substituição será aproximadamente normal, com média \(\mu\) e variância \(\sigma^{2}/n\), independentemente da distribuição da população.\textsuperscript{\citeproc{ref-kwak2017}{8}}
\item
  No exemplo abaixo, uma variável aleatória numérica com distribuição uniforme no espaço amostral \(S=[18;65]\) tem média \(\mu\) = 38.53 e variância \(\sigma^{2}\) = 172.433. A distribuição amostral da média de 100 amostras de tamanho 5, 50, 500 e 5000 tomadas da população com reposição e igual probabilidade se aproxima de uma distribuição normal com média \(\mu\) = 38.493 e variância \(\sigma^{2}\) = 0.038, independentemente da distribuição da população:
\end{itemize}

\begin{figure}

{\centering \includegraphics{Ciencia-com-R_files/figure-latex/teorema-central-limite-continua-plot-1} 

}

\caption{Esquerda: Histogramas de uma variável aleatória com distribuição uniforme (N = 100). Direita: Histogramas da média de 100 amostras de tamanhos 5, 50, 500 e 5000 tomadas da população com reposição e igual probabilidade.}\label{fig:teorema-central-limite-continua-plot}
\end{figure}

\begin{itemize}
\tightlist
\item
  Em outro exemplo, o lançamento de um dado com distribuição uniforme no espaço amostral \(S=\{1,2,3,4,5,6\}\) tem média \(\mu\) = 3.77 e variância \(\sigma^{2}\) = 3.169. A distribuição amostral da média de 100 amostras de tamanho 5, 50, 500 e 5000 tomadas da população com reposição e igual probabilidade se aproxima de uma distribuição normal com média \(\mu\) = 3.767 e variância \(\sigma^{2}\) = 0.001, independentemente da distribuição da população:
\end{itemize}

\begin{figure}

{\centering \includegraphics{Ciencia-com-R_files/figure-latex/teorema-central-limite-dado-plot-1} 

}

\caption{Esquerda: Histogramas de lançamento de 1 dado com distribuição uniforme (N = 100). Direita: Histogramas da média de 100 amostras de tamanhos 5, 50, 500 e 5000 tomadas da população com reposição e igual probabilidade.}\label{fig:teorema-central-limite-dado-plot}
\end{figure}

\begin{itemize}
\tightlist
\item
  Mais um exemplo, o lançamento de uma moeda com distribuição uniforme no espaço amostral \(S=\{0,1\}\) --- codificado para \(sucesso = 1\) e \(insucesso = 0\) --- tem média \(\mu\) = 0.57 e variância \(\sigma^{2}\) = 0.248. A distribuição amostral da média de 100 amostras de tamanho 5, 50, 500 e 5000 tomadas da população com reposição e igual probabilidade se aproxima de uma distribuição normal com média \(\mu\) = 0.57 e variância \(\sigma^{2}\) = 0, independentemente da distribuição da população:
\end{itemize}

\begin{figure}

{\centering \includegraphics{Ciencia-com-R_files/figure-latex/teorema-central-limite-moeda-plot-1} 

}

\caption{Esquerda: Histogramas de lançamento de 1 moeda com distribuição uniforme (N = 100). Direita: Histogramas da média de 100 amostras de tamanhos 5, 50, 500 e 5000 tomadas da população com reposição e igual probabilidade.}\label{fig:teorema-central-limite-moeda-plot}
\end{figure}

\subsection{Quais as condições de validade do teorema central do limite?}\label{quais-as-condiuxe7uxf5es-de-validade-do-teorema-central-do-limite}

\begin{itemize}
\item
  As condições de validade do teorema central do limite são:\textsuperscript{\citeproc{ref-kwak2017}{8}}

  \begin{itemize}
  \item
    As variáveis aleatórias devem ser independentes e identicamente distribuídas (\emph{independent and identically distributed} ou i.i.d.);
  \item
    As variáveis aleatórias devem ter média \(\mu\) e variância \(\sigma^{2}\) finitas;
  \item
    O tamanho da amostra deve ser suficientemente grande (geralmente, \(n \geq 30\)).
  \end{itemize}
\end{itemize}

\subsection{Qual a relação entre a lei dos grandes números e o teorema central do limite?}\label{qual-a-relauxe7uxe3o-entre-a-lei-dos-grandes-nuxfameros-e-o-teorema-central-do-limite}

\begin{itemize}
\tightlist
\item
  A lei dos grandes números é um precursor do teorema central do limite, pois estabelece que a média da amostra se torna cada vez mais próxima da média populacional (isto é, mais representativa) à medida que o tamanho da amostra aumenta, e o teorema central do limite demonstra que o a distribuição da soma das variáveis aleatórias se aproxima de uma distribuição normal também à medida que o tamanho da amostra aumenta.\textsuperscript{\citeproc{ref-REF}{\textbf{REF?}}}
\end{itemize}

\subsection{Qual a relevância do teorema central do limite para a análise estatística?}\label{qual-a-relevuxe2ncia-do-teorema-central-do-limite-para-a-anuxe1lise-estatuxedstica}

\begin{itemize}
\item
  O teorema central do limite explica porque os testes paramétricos têm maior poder estatístico do que os testes não paramétricos, os quais não requerem suposições de distribuição de probabilidade.\textsuperscript{\citeproc{ref-kwak2017}{8}}
\item
  O teorema central do limite implica que os métodos estatísticos que se aplicam a distribuições normais podem ser aplicados a outras distribuições quando suas suposições são satisfeitas.\textsuperscript{\citeproc{ref-kwak2017}{8}}
\item
  Como o teorema central do limite determina a distribuição amostral \(Z\) \eqref{eq:central-limit-theorem-z} das médias com tamanho amostral suficientemente grande, a média pode ser padronizada para uma distribuição normal com média 0 e variância 1, \(N(0,1)\):\textsuperscript{\citeproc{ref-kwak2017}{8}}
\end{itemize}

\begin{equation}
\label{eq:central-limit-theorem-z}
Z = \frac{\bar{X} - \mu}{\sigma / \sqrt{n}}
\end{equation}

\begin{itemize}
\tightlist
\item
  Para amostras com \(n \geq 30\), a distribuição amostral Student-\emph{t} se aproxima da distribuição normal padrão \(Z\) e, portanto, as suposições sobre a distribuição populacional não são mais necessárias de acordo com o teorema central do limite. Neste cenário, a suposição de distribuição normal pode ser usada para a distribuição de probabilidade.\textsuperscript{\citeproc{ref-kwak2017}{8}}
\end{itemize}

\section{Regressão para a média}\label{regressuxe3o-para-a-muxe9dia}

\subsection{O que é regressão para a média?}\label{o-que-uxe9-regressuxe3o-para-a-muxe9dia}

\begin{itemize}
\item
  Regressão para a média\textsuperscript{\citeproc{ref-galton1886}{9}} é um fenômeno estatístico que ocorre quando uma variável aleatória \(X\) é medida na mesma unidade de análise em dois ou mais momentos diferentes, \(X_{1}\), \(X_{2}\), \ldots, \(X_{t}\) e \(X_{t}\) é mais próximo da média populacional do que \(X_{1}\), ou seja, \(E(X_{t})\) é mais próxima de \(E(X)\) do que \(E(X_{1})\) é de \(E(X)\).\textsuperscript{\citeproc{ref-barnett2004}{10}}
\item
  O valor real --- sem erros aleatório ou sistemático --- em geral não é conhecido, mas pode ser estimado pela média de várias observações.\textsuperscript{\citeproc{ref-barnett2004}{10}}
\item
  Regressão para a média pode ocorrer em qualquer pesquisa cujo delineamento envolva medidas repetidas.\textsuperscript{\citeproc{ref-senn2011}{11}}
\item
  Em medidas repetidas, a média de várias observações é mais próxima da média verdadeira do que qualquer observação individual, pois o erro aleatório é reduzido pela média.\textsuperscript{\citeproc{ref-barnett2004}{10}}
\item
  Valores extremos --- em direção ao mínimo ou máximo --- em uma medição inicial tendem a ser seguidos por valores mais próximos da média (valor real) na medição subsequente.\textsuperscript{\citeproc{ref-barnett2004}{10}}
\item
  No exemplo abaixo, a 2a medida (dado 2 = 121) é mais próxima da média (valor real = 120) do que a 1a medida (dado 1 = 118):
\end{itemize}

\begin{figure}

{\centering \includegraphics{Ciencia-com-R_files/figure-latex/regressao-media-plot-1} 

}

\caption{Representação gráfica da regressão para a média em medidas repetidas. A segunda medida (dado 2) é mais próxima da média (valor real) do que a primeira medida (dado 1).}\label{fig:regressao-media-plot}
\end{figure}

\subsection{Qual a causa da regressão para a média?}\label{qual-a-causa-da-regressuxe3o-para-a-muxe9dia}

\begin{itemize}
\item
  A regressão para a média pode ser atribuída ao erro aleatório, que é a variação não sistemática nos valores observados em torno de uma média verdadeira (por exemplo, erro de medição aleatório ou variações aleatórias em um participante).\textsuperscript{\citeproc{ref-barnett2004}{10}}
\item
  Regressão para a média é uma consequência da observação de que dados extremos não se repetem com frequência.\textsuperscript{\citeproc{ref-senn2011}{11}}
\item
  Deve-se assumir que a regressão para a média ocorreu até que os dados mostrem o contrário.\textsuperscript{\citeproc{ref-barnett2004}{10}}
\end{itemize}

\subsection{Por que detectar o fenômeno de regressão para a média?}\label{por-que-detectar-o-fenuxf4meno-de-regressuxe3o-para-a-muxe9dia}

\begin{itemize}
\tightlist
\item
  A regressão para a média pode levar a conclusões errôneas sobre a eficácia de uma intervenção, pois a mudança observada pode ser devida ao erro aleatório e não ao tratamento.\textsuperscript{\citeproc{ref-senn2011}{11}}
\end{itemize}

\subsection{Como detectar o fenômeno de regressão para a média?}\label{como-detectar-o-fenuxf4meno-de-regressuxe3o-para-a-muxe9dia}

\begin{itemize}
\tightlist
\item
  O fenômeno de regressão para a média pode ser detectado por meio de gráfico de dispersão da diferença (estudos transversais) ou mudança (estudos longitudinais) versus os valores da 1a medida.\textsuperscript{\citeproc{ref-barnett2004}{10}}
\end{itemize}

\begin{infobox}{images/Rlogo}
O pacote \emph{regtomean}\textsuperscript{\citeproc{ref-regtomean}{12}} fornece as funções \href{https://www.rdocumentation.org/packages/regtomean/versions/1.1/topics/cordata}{\emph{cordata}} para calcular a correlação entre medidas tipo antes-e-depois e \href{https://www.rdocumentation.org/packages/regtomean/versions/1.1/topics/meechua_reg}{\emph{meechua\_reg}} para ajustar modelos lineares de regressão.

\end{infobox}

\subsection{Como o fenômeno de regressão para a média pode ser evitado?}\label{como-o-fenuxf4meno-de-regressuxe3o-para-a-muxe9dia-pode-ser-evitado}

\begin{itemize}
\item
  Aloque os participantes de modo aleatório nos grupos de tratamento e controle para reduzir o fenômeno de regressão para a média.\textsuperscript{\citeproc{ref-barnett2004}{10}}
\item
  Selecione participantes com base em medidas repetidas ao invés de medidas únicas.\textsuperscript{\citeproc{ref-barnett2004}{10}}
\end{itemize}

\chapter{\texorpdfstring{\textbf{Pensamento estatístico}}{Pensamento estatístico}}\label{pensamento-estatistico}

\section{Unidade de análise}\label{unidade-de-anuxe1lise}

\subsection{O que é unidade de análise?}\label{o-que-uxe9-unidade-de-anuxe1lise}

\begin{itemize}
\item
  A unidade de análise (ou unidade experimental) de pesquisas na área de saúde geralmente é o indivíduo.\textsuperscript{\citeproc{ref-Altman1997}{13}}
\item
  A unidade de análise também pode ser a instituição em estudos multicêntricos (ex.: hospitais, clínicas) ou um estudo publicado em meta-análise (ex.: ensaios clínicos).\textsuperscript{\citeproc{ref-Altman1997}{13}}
\end{itemize}

\subsection{Por que identificar a unidade de análise de um estudo?}\label{por-que-identificar-a-unidade-de-anuxe1lise-de-um-estudo}

\begin{itemize}
\tightlist
\item
  É fundamental identificar corretamente a unidade de análise para evitar inflação do tamanho da amostra (ex.: medidas bilaterais resultando em o dobro de participantes), violações de suposições dos testes de hipótese (ex.: independência entre medidas e/ou unidade de análise) e resultados espúrios em testes de hipótese (ex.: P-valores menores que aqueles observados se a amostra não estivesse inflada).\textsuperscript{\citeproc{ref-Altman1997}{13},\citeproc{ref-Matthews1990}{14}}
\end{itemize}

\subsection{Que medidas podem ser obtidas da unidade de análise de um estudo?}\label{que-medidas-podem-ser-obtidas-da-unidade-de-anuxe1lise-de-um-estudo}

\begin{itemize}
\tightlist
\item
  Da unidade de análise podem ser coletadas informações em medidas únicas, repetidas, seriadas ou múltiplas.
\end{itemize}

\section{População}\label{populauxe7uxe3o}

\subsection{O que é população?}\label{o-que-uxe9-populauxe7uxe3o}

\begin{itemize}
\item
  População --- ou população-alvo --- refere-se ao conjunto completo sobre o qual se pretende obter informações.\textsuperscript{\citeproc{ref-Banerjee2010}{15},\citeproc{ref-martinez-mesa2016}{16}}
\item
  População é metodologicamente delimitada pelos critérios de inclusão e exclusão do estudo.\textsuperscript{\citeproc{ref-Banerjee2010}{15}}
\item
  Em estudos observacionais, inicialmente as características geográficas e/ou demográficas, por exemplo, definem a população a ser estudada.\textsuperscript{\citeproc{ref-Banerjee2010}{15}}
\item
  Em estudos analíticos, a população é inicialmente definida pelos objetivos da pesquisa e, posteriormente, as observações são realizadas na amostra.\textsuperscript{\citeproc{ref-Banerjee2010}{15}}
\end{itemize}

\subsection{O que é representatividade e por que ela importa?}\label{o-que-uxe9-representatividade-e-por-que-ela-importa}

\begin{itemize}
\item
  Representatividade refere-se ao grau em que uma amostra reflete com fidelidade as características da população de referência.\textsuperscript{\citeproc{ref-martinez-mesa2016}{16}}
\item
  Quando a amostra contém menos indivíduos do que o número mínimo necessário, mas mantém a representatividade, a inferência estatística ainda é possível, embora possa haver redução da precisão e/ou do poder estatístico para detectar os efeitos.\textsuperscript{\citeproc{ref-martinez-mesa2016}{16}}
\item
  Amostras não representativas comprometem a validade da inferência estatística, mesmo quando o tamanho da amostra atende aos requisitos de poder da análise.\textsuperscript{\citeproc{ref-martinez-mesa2016}{16}}
\end{itemize}

\section{Amostra}\label{amostra}

\subsection{O que é amostra?}\label{o-que-uxe9-amostra}

\begin{itemize}
\item
  Amostra é uma parte finita da população do estudo.\textsuperscript{\citeproc{ref-Banerjee2010}{15},\citeproc{ref-martinez-mesa2016}{16}}
\item
  Em pesquisa científica, utilizam-se dados de uma amostra de participantes (ou outras unidades de análise) para realizar inferências sobre a população.\textsuperscript{\citeproc{ref-Bland2015}{17}}
\end{itemize}

\subsection{Por que usar dados de amostras?}\label{por-que-usar-dados-de-amostras}

\begin{itemize}
\item
  Estudos com amostras, em vez de censos, são preferíveis por diversas razões, dentre elas: questões éticas; limitações orçamentárias; desafios logísticos; restrição de tempo; e tamanho populacional desconhecido.\textsuperscript{\citeproc{ref-martinez-mesa2016}{16}}
\item
  Dados de uma amostra de tamanho suficiente e características representativas podem ser utilizados para inferência sobre uma população.\textsuperscript{\citeproc{ref-kwak2017}{8}}
\item
  Em geral, amostras de tamanhos maiores possuem médias mais próximas da média populacional e menores variâncias.\textsuperscript{\citeproc{ref-kwak2017}{8}}
\end{itemize}

\section{Amostragem}\label{amostragem}

\subsection{O que é amostragem?}\label{o-que-uxe9-amostragem}

\begin{itemize}
\tightlist
\item
  Amostragem é o processo pelo qual se seleciona uma parte de uma população para constituir a amostra que será efetivamente estudada.\textsuperscript{\citeproc{ref-martinez-mesa2016}{16}}
\end{itemize}

\begin{figure}

{\centering \includegraphics{Ciencia-com-R_files/figure-latex/populacao-amostra-1} 

}

\caption{Representação esquemática da amostragem: seleção de uma população para a amostra.}\label{fig:populacao-amostra}
\end{figure}

\subsection{Quais métodos de amostragem são usados para obter uma amostra da população?}\label{quais-muxe9todos-de-amostragem-suxe3o-usados-para-obter-uma-amostra-da-populauxe7uxe3o}

\begin{itemize}
\item
  O método de amostragem é geralmente definido pelas condições de viabilidade do estudo, no que diz respeito a acesso aos participantes, ao tempo de execução e aos custos envolvidos, entre outras.\textsuperscript{\citeproc{ref-Banerjee2010}{15}}
\item
  Não-probabilísticas ou intencionais:\textsuperscript{\citeproc{ref-Banerjee2010}{15},\citeproc{ref-martinez-mesa2016}{16}}

  \begin{itemize}
  \item
    Bola de neve.
  \item
    Conveniência.
  \item
    Participantes encaminhados.
  \item
    Proposital.
  \end{itemize}
\item
  Probabilísticas:\textsuperscript{\citeproc{ref-Banerjee2010}{15},\citeproc{ref-martinez-mesa2016}{16}}

  \begin{itemize}
  \item
    Simples.
  \item
    Sistemática.
  \item
    Multiestágio.
  \item
    Estratificada.
  \item
    Agregada.
  \end{itemize}
\end{itemize}

\subsection{O que é erro de amostragem?}\label{o-que-uxe9-erro-de-amostragem}

\begin{itemize}
\tightlist
\item
  Erro de amostragem é a variação natural entre os resultados obtidos a partir de uma amostra e os resultados que seriam obtidos caso toda a população fosse examinada. Reflete o grau de incerteza inerente à generalização de uma amostra para a população.\textsuperscript{\citeproc{ref-martinez-mesa2016}{16}}
\end{itemize}

\begin{figure}

{\centering \includegraphics{Ciencia-com-R_files/figure-latex/erro-amostragem-esquematico-1} 

}

\caption{Representação esquemática do erro de amostragem: seleção de várias amostras independentes de uma população.}\label{fig:erro-amostragem-esquematico}
\end{figure}

\begin{figure}

{\centering \includegraphics{Ciencia-com-R_files/figure-latex/erro-amostragem-1} 

}

\caption{Representação esquemática da amostragem de uma população para a amostra.}\label{fig:erro-amostragem}
\end{figure}

\section{Reamostragem}\label{reamostragem}

\subsection{O que é reamostragem?}\label{o-que-uxe9-reamostragem}

\begin{itemize}
\item
  Reamostragem é um procedimento que cria vários conjuntos de dados sorteados a partir de um conjunto de dados real - a amostra da população - sem a necessidade de fazer suposições sobre os dados e suas distribuições.\textsuperscript{\citeproc{ref-Bland2015}{17}}
\item
  O procedimento é repetido várias vezes para usar a variabilidade dos resultados para obter um intervalo de confiança do parâmetro no nível de significância \(\alpha\) pré-estabelecido.\textsuperscript{\citeproc{ref-Bland2015}{17}}
\end{itemize}

\subsection{Por que utilizar reamostragem?}\label{por-que-utilizar-reamostragem}

\begin{itemize}
\item
  Quando se dispõe de dados de apenas 1 amostra, as diversas suposições que são feitas podem não ser atingidas.\textsuperscript{\citeproc{ref-Bland2015}{17}}
\item
  Procedimentos de reamostragem produzem um conjunto de observações escolhidas aleatoriamente da amostra, igualmente representativo da população original.\textsuperscript{\citeproc{ref-Bland2015}{17}}
\item
  Procedimentos de reamostragem permitem estimar o erro-padrão e intervalos de confiança sem a necessidade de tais suposições, sendo, portanto, um conjunto de procedimentos não-paramétricos.\textsuperscript{\citeproc{ref-Bland2015}{17}}
\end{itemize}

\subsection{Quais procedimentos de reamostragem podem ser realizados?}\label{quais-procedimentos-de-reamostragem-podem-ser-realizados}

\begin{itemize}
\tightlist
\item
  \emph{Bootstrap}: Cada iteração gera uma amostra \emph{bootstrap} do mesmo tamanho do conjunto de dados original escolhendo aleatoriamente observações reais, uma de cada vez. Cada observação tem chances iguais de ser escolhida a cada vez, portanto, algumas observações serão escolhidas mais de uma vez e outras nem serão escolhidas.\textsuperscript{\citeproc{ref-Bland2015}{17}}
\end{itemize}

\begin{figure}

{\centering \includegraphics{Ciencia-com-R_files/figure-latex/reamostragem-1} 

}

\caption{Representação esquemática da reamostragem de uma amostra.}\label{fig:reamostragem}
\end{figure}

\section{Subamostragem}\label{subamostragem}

\subsection{O que é subamostragem?}\label{o-que-uxe9-subamostragem}

\begin{itemize}
\tightlist
\item
  .\textsuperscript{\citeproc{ref-REF}{\textbf{REF?}}}
\end{itemize}

\begin{figure}

{\centering \includegraphics{Ciencia-com-R_files/figure-latex/subamostragem-1} 

}

\caption{Representação esquemática da subamostragem de uma amostra.}\label{fig:subamostragem}
\end{figure}

\section{Superamostragem}\label{superamostragem}

\subsection{O que é superamostragem?}\label{o-que-uxe9-superamostragem}

\begin{itemize}
\tightlist
\item
  .\textsuperscript{\citeproc{ref-REF}{\textbf{REF?}}}
\end{itemize}

\begin{figure}

{\centering \includegraphics{Ciencia-com-R_files/figure-latex/superamostragem-1} 

}

\caption{Representação esquemática da superamostragem de uma população.}\label{fig:superamostragem}
\end{figure}

\chapter{\texorpdfstring{\textbf{Pensamento metodológico}}{Pensamento metodológico}}\label{pensamento-metodologico}

\section{Metodologia da pesquisa}\label{metodologia-da-pesquisa}

\subsection{O que é metodologia da pesquisa?}\label{o-que-uxe9-metodologia-da-pesquisa}

\begin{itemize}
\tightlist
\item
  A utilização de um vocabulário próprio --- incluindo termos frequentemente usados em metodologia, epidemiologia e estatística --- facilita a discussão na comunidade científica e melhora a compreensão das publicações.\textsuperscript{\citeproc{ref-amatuzzi2006}{18},\citeproc{ref-amatuzzi2006a}{19}}
\end{itemize}

\section{Relação Estatística-Metodologia}\label{relauxe7uxe3o-estatuxedstica-metodologia}

\subsection{Qual a relação entre estatística e metodologia da pesquisa?}\label{qual-a-relauxe7uxe3o-entre-estatuxedstica-e-metodologia-da-pesquisa}

\begin{itemize}
\tightlist
\item
  .\textsuperscript{\citeproc{ref-munafuxf22017}{20}}
\end{itemize}

\begin{figure}

{\centering \includegraphics[width=1\linewidth]{Ciencia-com-R_files/figure-latex/mapa-mental} 

}

\caption{Mapa mental da relação entre o pensamento estatístico e o pensamento metodológico.}\label{fig:mapa-mental}
\end{figure}

\section{Pesquisa quantitativa vs.~qualitativa}\label{pesquisa-quantitativa-vs.-qualitativa}

\subsection{O que significa a distinção entre pesquisa qualitativa e quantitativa?}\label{o-que-significa-a-distinuxe7uxe3o-entre-pesquisa-qualitativa-e-quantitativa}

\begin{itemize}
\item
  A divisão entre quantitativo e qualitativo é amplamente usada, mas é considerada por muitos autores como superficial ou imprecisa. Em geral, associa-se o qualitativo à exploração detalhada de casos e significados, e o quantitativo ao uso de estatística e amostras maiores.\textsuperscript{\citeproc{ref-wood2010}{21}}
\item
  Tais associações ocultam múltiplas dimensões --- por exemplo, análise estatística vs.~não estatística e teste de hipóteses vs.~indução --- que não coincidem perfeitamente.\textsuperscript{\citeproc{ref-wood2010}{21}}
\end{itemize}

\subsection{Por que essa dicotomia pode ser problemática?}\label{por-que-essa-dicotomia-pode-ser-problemuxe1tica}

\begin{itemize}
\item
  Ao assumir apenas duas categorias, deixamos de lado possibilidades metodológicas úteis, como indução estatística (uso de estatística para construir teorias a partir dos dados) e teste de hipóteses não estatístico (avaliação de hipóteses em estudos de caso ou comparações conceituais).\textsuperscript{\citeproc{ref-wood2010}{21}}
\item
  A consequência é restringir artificialmente a variedade de métodos possíveis e criar mal-entendidos sobre o que cada termo implica.\textsuperscript{\citeproc{ref-wood2010}{21}}
\end{itemize}

\subsection{Qual é uma alternativa para pensar o debate?}\label{qual-uxe9-uma-alternativa-para-pensar-o-debate}

\begin{itemize}
\item
  Usar termos mais específicos como ``dados ricos'' (\emph{rich data}), ``abordagem estatística'', ``ilustração de possibilidades'', ``teste de hipóteses'', ``seguimento de paradigma''.\textsuperscript{\citeproc{ref-wood2010}{21}}
\item
  Descrever com clareza como os dados foram coletados, analisados e interpretados, sem recorrer a rótulos amplos que podem confundir ou carregar preconceitos metodológicos.\textsuperscript{\citeproc{ref-wood2010}{21}}
\end{itemize}

\section{Pesquisa de métodos mistos}\label{pesquisa-de-muxe9todos-mistos}

\subsection{O que é pesquisa de métodos mistos?}\label{o-que-uxe9-pesquisa-de-muxe9todos-mistos}

\begin{itemize}
\item
  Método misto é uma metodologia que integra de forma sistemática abordagens quantitativas e qualitativas em um único estudo, com o objetivo de responder a perguntas de pesquisa de maneira mais completa.\textsuperscript{\citeproc{ref-lall2021}{22}}
\item
  Essa integração não é apenas a justaposição de duas técnicas; trata-se de um processo intencional de ``mistura'' de dados e interpretações em etapas como coleta, análise e interpretação, criando uma compreensão mais robusta.\textsuperscript{\citeproc{ref-lall2021}{22}}
\end{itemize}

\subsection{Quais são as principais dimensões do desenho de métodos mistos?}\label{quais-suxe3o-as-principais-dimensuxf5es-do-desenho-de-muxe9todos-mistos}

\begin{itemize}
\item
  O desenho de pesquisa em métodos mistos deve considerar dimensões como propósito do estudo, orientação teórica, tempo (simultâneo ou sequencial), pontos de integração entre componentes, complexidade e se o desenho é planejado ou emergente.\textsuperscript{\citeproc{ref-schoonenboom2017}{23}}
\item
  Entre as razões clássicas para combinar métodos estão: triangulação, complementaridade, desenvolvimento (um método orienta o outro), iniciação (explorar contradições) e expansão (ampliar o alcance da pesquisa).\textsuperscript{\citeproc{ref-schoonenboom2017}{23}}
\end{itemize}

\subsection{Quais são os delineamentos centrais em pesquisa de métodos mistos?}\label{quais-suxe3o-os-delineamentos-centrais-em-pesquisa-de-muxe9todos-mistos}

\begin{itemize}
\item
  Três delineamentos principais são descritos como centrais: convergente, sequencial explanatório e sequencial exploratório.\textsuperscript{\citeproc{ref-lall2021}{22}}
\item
  Convergente: coleta e análise de dados quantitativos e qualitativos em paralelo, com integração na interpretação.\textsuperscript{\citeproc{ref-lall2021}{22}}
\item
  Sequencial explanatório: inicia com dados quantitativos, seguidos por qualitativos para explicar ou expandir os achados.\textsuperscript{\citeproc{ref-lall2021}{22}}
\item
  Sequencial exploratório: inicia com dados qualitativos, seguidos por quantitativos que testam ou generalizam os resultados iniciais.\textsuperscript{\citeproc{ref-lall2021}{22}}
\item
  Tipologias adicionais incluem delineamentos incorporados (\emph{embedded}), transformativos (inspirados em perspectivas críticas, feministas ou de justiça social) e multifásicos, que combinam várias fases ao longo do tempo.\textsuperscript{\citeproc{ref-schoonenboom2017}{23}}
\end{itemize}

\section{Pesquisa exploratória vs.~confirmatória}\label{pesquisa-exploratuxf3ria-vs.-confirmatuxf3ria}

\subsection{O que são pesquisas exploratórias e confirmatórias?}\label{o-que-suxe3o-pesquisas-exploratuxf3rias-e-confirmatuxf3rias}

\begin{itemize}
\item
  Confirmatória: teste planejado \emph{a priori} de hipóteses com plano analítico predefinido (variáveis, modelos, critérios de exclusão, correções para múltiplos testes). Favorece controle de erro tipo I e interpretações diretas.\textsuperscript{\citeproc{ref-rubin2022}{24}}
\item
  Exploratória: testes pós-hoc motivados pelos dados, voltados a descoberta de padrões, geração/refinamento de hipóteses e checagens de plausibilidade. Pode revelar relações não antecipadas e orientar estudos futuros.\textsuperscript{\citeproc{ref-rubin2022}{24}}
\end{itemize}

\subsection{Por que a dicotomia é limitada?}\label{por-que-a-dicotomia-uxe9-limitada}

\begin{itemize}
\item
  Na prática, há um \emph{continuum} entre exploração e confirmação; muitos estudos combinam elementos de ambos em momentos distintos (p.ex., análises principais confirmatórias + análises de sensibilidade/descoberta).\textsuperscript{\citeproc{ref-rubin2022}{24}}
\item
  Análises exploratórias não são inerentemente inferiores: quando bem justificadas e comparando explicações alternativas, podem aumentar a rigorosidade do teste e produzir inferências informativas.\textsuperscript{\citeproc{ref-rubin2022}{24}}
\item
  A discussão contemporânea sobre a distinção entre estudos confirmatórios e exploratórios também enfatiza a necessidade de justificar o tamanho de efeito mínimo de interesse e aplicar testes severos conforme o arcabouço de falsificacionismo metodológico.\textsuperscript{\citeproc{ref-spuxe4th2025}{25}}
\end{itemize}

\subsection{Quais são as boas práticas de transparência?}\label{quais-suxe3o-as-boas-pruxe1ticas-de-transparuxeancia}

\begin{itemize}
\item
  Rotular claramente quais análises são confirmatórias e quais são exploratórias.\textsuperscript{\citeproc{ref-rubin2022}{24}}
\item
  Pré-registrar hipóteses e plano confirmatório; documentar desvios e justificá-los.\textsuperscript{\citeproc{ref-rubin2022}{24}}
\item
  Relatar análises de sensibilidade (modelos alternativos, decisões analíticas razoáveis) para avaliar robustez.\textsuperscript{\citeproc{ref-rubin2022}{24}}
\item
  Disponibilizar dados e código sempre que eticamente possível, distinguindo scripts confirmatórios de scripts exploratórios.\textsuperscript{\citeproc{ref-rubin2022}{24}}
\end{itemize}

\section{Pesquisa translacional}\label{pesquisa-translacional}

\subsection{O que é pesquisa translacional?}\label{o-que-uxe9-pesquisa-translacional}

\begin{itemize}
\tightlist
\item
  .\textsuperscript{\citeproc{ref-REF}{\textbf{REF?}}}
\end{itemize}

\section{Pré-registro}\label{pruxe9-registro}

\subsection{O que é pré-registro?}\label{o-que-uxe9-pruxe9-registro}

\begin{itemize}
\item
  Pré-registro é o ato de registrar publicamente o plano de pesquisa antes da coleta de dados ou análise.\textsuperscript{\citeproc{ref-REF}{\textbf{REF?}}}
\item
  O pré-registro é um elemento central das práticas abertas que aumentam a severidade de testes estatísticos e reduzem a flexibilidade analítica oportunista.\textsuperscript{\citeproc{ref-spuxe4th2025}{25}}
\end{itemize}

\section{Reprodutibilidade}\label{reprodutibilidade}

\subsection{O que é reprodutibilidade?}\label{o-que-uxe9-reprodutibilidade}

\begin{itemize}
\tightlist
\item
  Reprodutibilidade é a habilidade de se obter resultados iguais ou similares quando uma análise ou teste estatístico é repetido.\textsuperscript{\citeproc{ref-resnik2016}{26}--\citeproc{ref-mair2016}{28}}
\end{itemize}

\subsection{Por que reprodutibilidade é importante?}\label{por-que-reprodutibilidade-uxe9-importante}

\begin{itemize}
\item
  Analisar a reprodutibilidade pode fornecer evidências a respeito da objetividade e confiabilidade dos achados, em detrimento de terem sido obtidos devido a vieses ou ao acaso.\textsuperscript{\citeproc{ref-resnik2016}{26}}
\item
  A reprodutibilidade não é apenas uma questão metodológica, mas também ética, uma vez que pode envolver mal práticas científicas como fabricação e/ou falsificação de dados.\textsuperscript{\citeproc{ref-resnik2016}{26}}
\item
  Reprodutibilidade pode ser considerada um padrão mínimo em pesquisa científica.\textsuperscript{\citeproc{ref-hofner2015}{27}}
\end{itemize}

\subsection{Como contribuir para a reprodutibilidade?}\label{como-contribuir-para-a-reprodutibilidade}

\begin{itemize}
\item
  Disponibilize publicamente os bancos de dados, respeitando as considerações éticas vigentes (ex.: autorização dos participantes e do Comitê de Ética em Pesquisa) e internacionalmente.\textsuperscript{\citeproc{ref-mair2016}{28}}
\item
  Produza manuscritos reprodutíveis --- manuscritos executáveis ou relatórios dinâmicos --- que permitem a integração do banco de dados da(s) amostra(s), do(s) script(s) de análise estatística (incluindo comentários para sua interpretação), dos pacotes ou bibliotecas utilizados, das fontes e referências bibliográficas citadas, além dos demais elementos textuais (tabelas, gráficos) - todos gerados dinamicamente.\textsuperscript{\citeproc{ref-hinsen2011}{29}}
\item
  Ao adotar delineamentos confirmatórios e justificar previamente hipóteses e tamanhos de efeito, o pesquisador contribui não apenas para a reprodutibilidade, mas também para a ``severidade'' inferencial.\textsuperscript{\citeproc{ref-spuxe4th2025}{25}}
\end{itemize}

\section{Robustez}\label{robustez}

\subsection{O que é robustez?}\label{o-que-uxe9-robustez}

\begin{itemize}
\tightlist
\item
  .\textsuperscript{\citeproc{ref-REF}{\textbf{REF?}}}
\end{itemize}

\section{Replicabilidade}\label{replicabilidade}

\subsection{O que é replicabilidade?}\label{o-que-uxe9-replicabilidade}

\begin{itemize}
\tightlist
\item
  Replicabilidade é a habilidade de se obter conclusões iguais ou similares quando um experimento é repetido.\textsuperscript{\citeproc{ref-hofner2015}{27},\citeproc{ref-mair2016}{28}}
\end{itemize}

\section{Generalização}\label{generalizauxe7uxe3o}

\subsection{O que é generalização?}\label{o-que-uxe9-generalizauxe7uxe3o}

\begin{itemize}
\tightlist
\item
  Generalização refere-se à extrapolação das conclusões do estudo, observados na amostra, para a população.\textsuperscript{\citeproc{ref-Banerjee2010}{15}}
\end{itemize}

\begin{figure}

{\centering \includegraphics{Ciencia-com-R_files/figure-latex/generalizacao-1} 

}

\caption{Representação esquemática da generalização de uma amostra para a população.}\label{fig:generalizacao}
\end{figure}

\chapter{\texorpdfstring{\textbf{Pensamento computacional}}{Pensamento computacional}}\label{pensamento-computacional}

\section{R}\label{r}

\subsection{O que é R?}\label{o-que-uxe9-r}

\begin{itemize}
\item
  R é um programa de computador de código aberto com linguagem computacional direcionada para análise estatística.\textsuperscript{\citeproc{ref-ihaka1996}{30},\citeproc{ref-introduc2020}{31}}
\item
  R version 4.5.2 (2025-10-31) está disponível gratuitamente em \emph{Comprehensive R Archive Network} (CRAN).\textsuperscript{\citeproc{ref-CRAN}{32}}
\end{itemize}

\subsection{Por que usar R?}\label{por-que-usar-r}

\begin{itemize}
\tightlist
\item
  R é o software de maior abrangência de métodos estatísticos, possui sintaxe que permite análises estatísticas reproduzíveis e está disponível gratuitamente no \emph{Comprehensive R Archive Network} (CRAN).\textsuperscript{\citeproc{ref-mair2016}{28},\citeproc{ref-CRAN}{32}}
\end{itemize}

\subsection{O que é R Markdown?}\label{o-que-uxe9-r-markdown}

\begin{itemize}
\item
  R Markdown\textsuperscript{\citeproc{ref-R-rmarkdown}{33}} é uma ferramenta que permite a integração de texto, código e saída em um único documento.\textsuperscript{\citeproc{ref-REF}{\textbf{REF?}}}
\item
  O R Markdown é uma extensão do Markdown, que é uma linguagem de marcação simples e fácil de aprender, que é usada para formatar texto.\textsuperscript{\citeproc{ref-REF}{\textbf{REF?}}}
\item
  O R Markdown permite a inclusão de blocos de código R, Python, SQL, C++, entre outros, e a saída desses blocos de código é incorporada ao documento final.\textsuperscript{\citeproc{ref-REF}{\textbf{REF?}}}
\item
  O R Markdown é uma ferramenta poderosa para a criação de relatórios dinâmicos, que podem ser facilmente atualizados com novos dados ou análises.\textsuperscript{\citeproc{ref-REF}{\textbf{REF?}}}
\item
  O R Markdown é amplamente utilizado na comunidade científica para a criação de relatórios de pesquisa, artigos científicos, apresentações, livros, entre outros.\textsuperscript{\citeproc{ref-REF}{\textbf{REF?}}}
\item
  O trabalho com RMarkdown\textsuperscript{\citeproc{ref-R-rmarkdown}{33}} permite um fluxo de dados totalmente transparente, desde o conjunto de dados coletados até o manuscrito finalizado. Todos os aspectos do fluxo de dados podem ser incorporados em blocos de R script (\emph{chunk}), exibindo tanto o R script quando o respectivo texto, tabelas e figuras formatadas no estilo científico de interesse.\textsuperscript{\citeproc{ref-holmes2021}{34}}
\item
  O RMarkdown\textsuperscript{\citeproc{ref-R-rmarkdown}{33}} foi projetado especificamente para relatórios dinâmicos onde a análise é realizada em R e oferece uma flexibilidade incrível por meio de uma linguagem de marcação.\textsuperscript{\citeproc{ref-mair2016}{28}}
\end{itemize}

\subsection{Que programas de computador podem ser usados para análise estatística com R?}\label{que-programas-de-computador-podem-ser-usados-para-anuxe1lise-estatuxedstica-com-r}

\begin{itemize}
\item
  \href{https://jasp-stats.org}{JASP}.\textsuperscript{\citeproc{ref-love2019}{35}}
\item
  \href{https://www.jamovi.org}{jamovi}.\textsuperscript{\citeproc{ref-sahin2020}{36}}
\end{itemize}

\begin{infobox}{images/Rlogo}
Os pacotes \emph{jmv}\textsuperscript{\citeproc{ref-jmv}{37}} e \emph{jmvconnect}\textsuperscript{\citeproc{ref-jmvconnect}{38}} fornecem funções para análise descritiva e inferencial com interface com jamovi.

\end{infobox}

\section{RStudio}\label{rstudio}

\subsection{O que é RStudio?}\label{o-que-uxe9-rstudio}

\begin{itemize}
\item
  RStudio é um ambiente de desenvolvimento integrado (\emph{integrated development environment}, IDE) desenvolvido visando a reprodutibilidade e a simplicidade para a criação e disseminação de conhecimento.\textsuperscript{\citeproc{ref-introduc2020}{31},\citeproc{ref-racine2011}{39}}
\item
  O ambiente do RStudio é dividido em paineis:

  \begin{itemize}
  \item
    \emph{Source/Script editor}: para edição de R scripts.\textsuperscript{\citeproc{ref-introduc2020}{31}}
  \item
    \emph{Console}: para execução de códigos simples.\textsuperscript{\citeproc{ref-introduc2020}{31}}
  \item
    \emph{Environments}: para visualização de objetos criados durante a sessão de trabalho.\textsuperscript{\citeproc{ref-introduc2020}{31}}
  \item
    \emph{Output}: para visualização de gráficos criados durante a sessão de trabalho.\textsuperscript{\citeproc{ref-introduc2020}{31}}
  \end{itemize}
\end{itemize}

\begin{figure}

{\centering \includegraphics[width=1\linewidth]{Ciencia-com-R_files/figure-latex/RStudioConsole} 

}

\caption{Interface do RStudio. Fonte: https://docs.posit.co/ide/user/}\label{fig:RStudioConsole}
\end{figure}

\begin{itemize}
\item
  As principais características do RStudio incluem um ambiente de edição com abas para acesso rápido a arquivos, comandos e resultados; histórico de comandos previamente utilizados; ferramentas para visualização de bancos de dados e elaboração de scripts e gráficos e tabelas.\textsuperscript{\citeproc{ref-introduc2020}{31},\citeproc{ref-racine2011}{39}}
\item
  RStudio está disponível gratuitamente em \href{https://posit.co/download/rstudio-desktop/}{Posit}.
\end{itemize}

\begin{infobox}{images/Rlogo}
O pacote \emph{learnr}\textsuperscript{\citeproc{ref-learnr}{40}} fornece tutoriais interativos para RStudio.

\end{infobox}

\section{Scripts}\label{scripts}

\subsection{O que são R scripts?}\label{o-que-suxe3o-r-scripts}

\begin{itemize}
\item
  ``Scripts são dados''.\textsuperscript{\citeproc{ref-hinsen2011}{29}}
\item
  Scripts permitem ao usuário se concentrar nas tarefas mais importantes da computação e utilizar pacotes ou bibliotecas para executar as funções mais básicas com maior eficiência.\textsuperscript{\citeproc{ref-hinsen2011}{29}}
\item
  Um script é um arquivo de texto contendo (quase) os mesmos comandos que você digitaria na linha de comando do R. O ``quase'' refere-se ao fato de que se você estiver usando \emph{sink()} para enviar a saída para um arquivo, você terá que incluir alguns comandos em \emph{print()} para obter a mesma saída da linha de comando.\textsuperscript{\citeproc{ref-REF}{\textbf{REF?}}}
\end{itemize}

\begin{Shaded}
\begin{Highlighting}[]
\CommentTok{\# Exemplo de R script}

\CommentTok{\# Este é um comentário}

\CommentTok{\# Esta é uma variável}
\NormalTok{variavel }\OtherTok{\textless{}{-}} \FloatTok{3.14} \CommentTok{\# Atribui o valor 3.14 à variável}

\CommentTok{\# Esta é uma função}
\NormalTok{f }\OtherTok{\textless{}{-}} \ControlFlowTok{function}\NormalTok{(x) \{}
  \FunctionTok{return}\NormalTok{(x}\SpecialCharTok{\^{}}\DecValTok{2}\NormalTok{) }\CommentTok{\# Retorna o quadrado do valor de x}
\NormalTok{\}}

\CommentTok{\# Esta é uma chamada de função}
\NormalTok{resultado }\OtherTok{\textless{}{-}} \FunctionTok{f}\NormalTok{(variavel) }\CommentTok{\# Chama a função f com a variável como argumento}

\CommentTok{\# Exibe o resultado da função}
\FunctionTok{print}\NormalTok{(resultado) }\CommentTok{\# Exibe o resultado na saída padrão}

\CommentTok{\# Este é um vetor}
\NormalTok{vetor }\OtherTok{\textless{}{-}} \FunctionTok{c}\NormalTok{(}\DecValTok{1}\NormalTok{, }\DecValTok{2}\NormalTok{, }\DecValTok{3}\NormalTok{, }\DecValTok{4}\NormalTok{, }\DecValTok{5}\NormalTok{) }\CommentTok{\# Cria um vetor com os valores de 1 a 5}
\CommentTok{\# Exibe o vetor}
\FunctionTok{print}\NormalTok{(vetor) }\CommentTok{\# Exibe o vetor na saída padrão}

\CommentTok{\# Esta é uma matrix}
\NormalTok{matriz }\OtherTok{\textless{}{-}} \FunctionTok{matrix}\NormalTok{(}\DecValTok{1}\SpecialCharTok{:}\DecValTok{9}\NormalTok{, }\AttributeTok{nrow=}\DecValTok{3}\NormalTok{, }\AttributeTok{ncol=}\DecValTok{3}\NormalTok{) }\CommentTok{\# Cria uma matriz 3x3 com os valores de 1 a 9}
\CommentTok{\# Exibe a matriz}
\FunctionTok{print}\NormalTok{(matriz) }\CommentTok{\# Exibe a matriz na saída padrão}

\CommentTok{\# Esta é uma lista}
\NormalTok{lista }\OtherTok{\textless{}{-}} \FunctionTok{list}\NormalTok{(}\AttributeTok{nome=}\StringTok{"João"}\NormalTok{, }\AttributeTok{idade=}\DecValTok{30}\NormalTok{, }\AttributeTok{altura=}\FloatTok{1.75}\NormalTok{) }\CommentTok{\# Cria uma lista com nome, idade e altura}
\CommentTok{\# Exibe a lista}
\FunctionTok{print}\NormalTok{(lista) }\CommentTok{\# Exibe a lista na saída padrão}

\CommentTok{\# Este é um dataframe}
\NormalTok{dataframe }\OtherTok{\textless{}{-}} \FunctionTok{data.frame}\NormalTok{(}\AttributeTok{nome=}\FunctionTok{c}\NormalTok{(}\StringTok{"João"}\NormalTok{, }\StringTok{"Maria"}\NormalTok{, }\StringTok{"José"}\NormalTok{), }\AttributeTok{idade=}\FunctionTok{c}\NormalTok{(}\DecValTok{30}\NormalTok{, }\DecValTok{25}\NormalTok{, }\DecValTok{40}\NormalTok{), }\AttributeTok{altura=}\FunctionTok{c}\NormalTok{(}\FloatTok{1.75}\NormalTok{, }\FloatTok{1.60}\NormalTok{, }\FloatTok{1.80}\NormalTok{)) }\CommentTok{\# Cria um dataframe com nome, idade e altura}
\CommentTok{\# Exibe o dataframe}
\FunctionTok{print}\NormalTok{(dataframe) }\CommentTok{\# Exibe o dataframe na saída padrão}

\CommentTok{\# Este é um loop for}
\ControlFlowTok{for}\NormalTok{ (i }\ControlFlowTok{in} \DecValTok{1}\SpecialCharTok{:}\DecValTok{5}\NormalTok{) \{}
  \FunctionTok{print}\NormalTok{(i) }\CommentTok{\# Exibe os valores de 1 a 5 na saída padrão}
\NormalTok{\}}

\CommentTok{\# Este é um loop while}
\NormalTok{j }\OtherTok{\textless{}{-}} \DecValTok{1}
\ControlFlowTok{while}\NormalTok{ (j }\SpecialCharTok{\textless{}=} \DecValTok{5}\NormalTok{) \{}
  \FunctionTok{print}\NormalTok{(j) }\CommentTok{\# Exibe os valores de 1 a 5 na saída padrão}
\NormalTok{  j }\OtherTok{\textless{}{-}}\NormalTok{ j }\SpecialCharTok{+} \DecValTok{1} \CommentTok{\# Incrementa o valor de j em 1}
\NormalTok{\}}

\CommentTok{\# Este é um condicional if{-}else}
\NormalTok{k }\OtherTok{\textless{}{-}} \DecValTok{3}
\ControlFlowTok{if}\NormalTok{ (k }\SpecialCharTok{\textgreater{}} \DecValTok{0}\NormalTok{) \{}
  \FunctionTok{print}\NormalTok{(}\StringTok{"k é positivo"}\NormalTok{) }\CommentTok{\# Exibe "k é positivo" se k for maior que 0}
\NormalTok{\} }\ControlFlowTok{else} \ControlFlowTok{if}\NormalTok{ (k }\SpecialCharTok{\textless{}} \DecValTok{0}\NormalTok{) \{}
  \FunctionTok{print}\NormalTok{(}\StringTok{"k é negativo"}\NormalTok{) }\CommentTok{\# Exibe "k é negativo" se k for menor que 0}
\NormalTok{\} }\ControlFlowTok{else}\NormalTok{ \{}
  \FunctionTok{print}\NormalTok{(}\StringTok{"k é zero"}\NormalTok{) }\CommentTok{\# Exibe "k é zero" se k for igual a 0}
\NormalTok{\}}

\CommentTok{\# Fim do exemplo de R script}
\end{Highlighting}
\end{Shaded}

\subsection{Quais são as boas práticas na redação de scripts?}\label{quais-suxe3o-as-boas-pruxe1ticas-na-redauxe7uxe3o-de-scripts}

\begin{itemize}
\item
  Use nomes consistentes para as variáveis.\textsuperscript{\citeproc{ref-SchwabSimon2021}{41}}
\item
  Defina os tipos de variáveis adequadamente no banco de dados.\textsuperscript{\citeproc{ref-SchwabSimon2021}{41}}
\item
  Defina constantes --- isto é, variáveis de valor fixo --- ao invés de digitar valores.\textsuperscript{\citeproc{ref-SchwabSimon2021}{41}}
\item
  Use e cite os pacotes disponíveis para suas análises.\textsuperscript{\citeproc{ref-SchwabSimon2021}{41}}
\item
  Controle as versões do script.\textsuperscript{\citeproc{ref-SchwabSimon2021}{41},\citeproc{ref-Eglen2017}{42}}
\item
  Teste o script antes de sua utilização.\textsuperscript{\citeproc{ref-SchwabSimon2021}{41}}
\item
  Conduza revisão por pares do código durante a redação (digitação em dupla).\textsuperscript{\citeproc{ref-SchwabSimon2021}{41}}
\end{itemize}

\begin{infobox}{images/Rlogo}
O pacote \emph{formatR}\textsuperscript{\citeproc{ref-formatR}{43}} fornece a função \href{https://www.rdocumentation.org/packages/formatR/versions/1.14/topics/tidy_source}{\emph{tidy\_source}} para formatar um R script.

\end{infobox}

\begin{infobox}{images/Rlogo}
O pacote \emph{styler}\textsuperscript{\citeproc{ref-styler}{44}} fornece a função \href{https://www.rdocumentation.org/packages/styler/versions/1.10.1/topics/style_file}{\emph{style\_file}} para formatar um R script.

\end{infobox}

\begin{infobox}{images/Rlogo}
O pacote \emph{lintr}\textsuperscript{\citeproc{ref-lintr}{45}} fornece a função \href{https://www.rdocumentation.org/packages/lintr/versions/3.1.0/topics/lint}{\emph{lint}} para verificar a adesão de um script a um determinado estilo, identificando erros de sintaxe e possíveis problemas semânticos.

\end{infobox}

\section{Pacotes}\label{pacotes}

\subsection{O que são pacotes?}\label{o-que-suxe3o-pacotes}

\begin{itemize}
\item
  Pacotes são conjuntos de scripts programados pela comunidade e compartilhados para uso público.\textsuperscript{\citeproc{ref-introduc2020}{31}}
\item
  Os pacotes ficam armazenados no \emph{Comprehensive R Archive Network} (CRAN) e podem ser instalados diretamente no RStudio.\textsuperscript{\citeproc{ref-introduc2020}{31},\citeproc{ref-CRAN}{32}}
\item
  Na mais recente atualização deste livro, o {[}\emph{Comprehensive R Archive Network} (CRAN) possui 391102 pacotes disponíveis.\textsuperscript{\citeproc{ref-introduc2020}{31},\citeproc{ref-CRAN}{32}}
\item
  Os pacotes disponíveis podem ser encontrados em \emph{R PACKAGES DOCUMENTATION}.\textsuperscript{\citeproc{ref-R_Packages}{46}}
\end{itemize}

\begin{infobox}{images/Rlogo}
O pacote \emph{utils}\textsuperscript{\citeproc{ref-utils}{47}} fornece a função \href{https://www.rdocumentation.org/packages/utils/versions/3.6.2/topics/install.packages}{\emph{install.packages}} para instalar os pacotes no computador.

\end{infobox}

\begin{infobox}{images/Rlogo}
O pacote \emph{utils}\textsuperscript{\citeproc{ref-utils}{47}} fornece a função \href{https://www.rdocumentation.org/packages/utils/versions/3.6.2/topics/library}{\emph{library}} para carregar os pacotes instalados no computador.

\end{infobox}

\begin{infobox}{images/Rlogo}
O pacote \emph{utils}\textsuperscript{\citeproc{ref-utils}{47}} fornece a função \href{https://www.rdocumentation.org/packages/utils/versions/3.6.2/topics/require}{\emph{require}} para indicar se o pacote requisitado está disponível.

\end{infobox}

\begin{infobox}{images/Rlogo}
O pacote \emph{utils}\textsuperscript{\citeproc{ref-utils}{47}} fornece a função \href{https://www.rdocumentation.org/packages/utils/versions/3.6.2/topics/installed.packages}{\emph{installed.packages}} para listar os pacotes instalados no computador.

\end{infobox}

\begin{infobox}{images/Rlogo}
O pacote \emph{utils}\textsuperscript{\citeproc{ref-utils}{47}} fornece a função \href{https://www.rdocumentation.org/packages/utils/versions/3.6.2/topics/update.packages}{\emph{update.packages}} para atualizar os pacotes instalados no computador.

\end{infobox}

\begin{infobox}{images/Rlogo}
O pacote \emph{roxygen2}\textsuperscript{\citeproc{ref-roxygen2}{48}} fornece a função \href{https://cran.r-project.org/web/packages/roxygen2/index.html}{\emph{roxygenize}} para criar arquivos .Rd para documentar pacotes.

\end{infobox}

\section{Aplicativos}\label{aplicativos}

\subsection{O que são Shiny Apps?}\label{o-que-suxe3o-shiny-apps}

\begin{itemize}
\tightlist
\item
  Shiny Apps são aplicativos web interativos que permitem a criação de interfaces gráficas para visualização e análise de dados em tempo real, utilizando o R como backend.\textsuperscript{\citeproc{ref-REF}{\textbf{REF?}}}
\end{itemize}

\section{Manuscritos reproduzíveis}\label{manuscritos-reproduzuxedveis}

\subsection{O que são manuscritos reproduzíveis?}\label{o-que-suxe3o-manuscritos-reproduzuxedveis}

\begin{itemize}
\tightlist
\item
  Manuscritos reproduzíveis --- manuscritos executáveis ou relatórios dinâmicos --- permitem a produção de um manuscrito completo a partir da integração do banco de dados da(s) amostra(s), do(s) script(s) de análise estatística (incluindo comentários para sua interpretação), dos pacotes ou bibliotecas utilizados, das fontes e referências bibliográficas citadas, além dos demais elementos textuais (tabelas, gráficos) - todos gerados dinamicamente.\textsuperscript{\citeproc{ref-hinsen2011}{29}}
\end{itemize}

\subsection{Por que usar manuscritos reproduzíveis?}\label{por-que-usar-manuscritos-reproduzuxedveis}

\begin{itemize}
\item
  No processo tradicional de redação científica há muitas etapas de copiar e colar não reproduzíveis envolvidas. Documentos dinâmicos combinam uma ferramenta de processamento de texto com o R script que produz o texto/tabela/figura a ser incorporado no manuscrito.\textsuperscript{\citeproc{ref-mair2016}{28}}
\item
  Ao trabalhar com relatórios dinâmicos, é possível extrair o mesmo script usado para análise estatística. Os documentos podem ser compilados em vários formatos de saída e salvos como DOCX, PPTX e PDF.\textsuperscript{\citeproc{ref-mair2016}{28}}
\item
  Muitos erros de análise poderiam ser evitados com a adoção de boas práticas de programação em manuscritos reproduzíveis.\textsuperscript{\citeproc{ref-trisovic2022}{49}}
\end{itemize}

\begin{infobox}{images/Rlogo}
O pacote \emph{rmarkdown}\textsuperscript{\citeproc{ref-R-rmarkdown}{33}} fornece as funções \href{https://www.rdocumentation.org/packages/rmarkdown/versions/2.24/topics/render}{\emph{render}} para criar manuscritos reprodutíveis a partir de arquivos .Rmd.

\end{infobox}

\begin{infobox}{images/Rlogo}
O pacote \emph{officedown}\textsuperscript{\citeproc{ref-officedown}{50}} fornece as funções \href{https://www.rdocumentation.org/packages/officedown/versions/0.3.0/topics/rdocx_document}{\emph{rdocx\_document}} e \href{https://www.rdocumentation.org/packages/officedown/versions/0.3.0/topics/rpptx_document}{\emph{rpptx\_document}} para criar arquivos DOCX e PPTX, respectivamente, com o conteúdo criado no manuscrito reprodutível.

\end{infobox}

\begin{infobox}{images/Rlogo}
O pacote \emph{bookdown}\textsuperscript{\citeproc{ref-bookdown}{51}} fornece as funções \href{https://www.rdocumentation.org/packages/bookdown/versions/0.35/topics/gitbook}{\emph{gitbook}}, \href{https://www.rdocumentation.org/packages/bookdown/versions/0.35/topics/pdf_book}{\emph{pdf\_book}}, \href{https://www.rdocumentation.org/packages/bookdown/versions/0.35/topics/epub_book}{\emph{epub\_book}} e \href{https://www.rdocumentation.org/packages/bookdown/versions/0.35/topics/html_document2}{\emph{html\_document2}} para criar documentos reprodutíveis em diversos formatos (Git, PDF, EPUB e HTML, respectivamente).

\end{infobox}

\subsection{Como manuscritos reprodutíveis contribuem para a ciência?}\label{como-manuscritos-reprodutuxedveis-contribuem-para-a-ciuxeancia}

\begin{itemize}
\tightlist
\item
  O compartilhamento de bancos de dados e seus scripts de análise estatística permitem a adoção de práticas reprodutíveis, tais como a reanálise dos dados.\textsuperscript{\citeproc{ref-ioannidis2014}{52}}
\end{itemize}

\begin{infobox}{images/Rlogo}
O pacote \emph{projects}\textsuperscript{\citeproc{ref-projects}{53}} fornece a função \href{https://www.rdocumentation.org/packages/projects/versions/2.1.3/topics/setup_projects}{\emph{setup\_projects}} para criar um projeto com arquivos organizados em diretórios.

\end{infobox}

\begin{infobox}{images/Rlogo}
O pacote \emph{rmarkdown}\textsuperscript{\citeproc{ref-R-rmarkdown}{33}} fornece a função \href{https://www.rdocumentation.org/packages/rmarkdown/versions/2.24/topics/render}{\emph{render}} para criar manuscritos reprodutíveis a partir de arquivos .Rmd.

\end{infobox}

\begin{infobox}{images/Rlogo}
O pacote \emph{bookdown}\textsuperscript{\citeproc{ref-bookdown}{51}} fornece as funções \href{https://www.rdocumentation.org/packages/bookdown/versions/0.35/topics/gitbook}{\emph{gitbook}}, \href{https://www.rdocumentation.org/packages/bookdown/versions/0.35/topics/pdf_book}{\emph{pdf\_book}}, \href{https://www.rdocumentation.org/packages/bookdown/versions/0.35/topics/epub_book}{\emph{epub\_book}} e \href{https://www.rdocumentation.org/packages/bookdown/versions/0.35/topics/html_document2}{\emph{html\_document2}} para criar documentos reprodutíveis em diversos formatos (Git, PDF, EPUB e HTML, respectivamente).

\end{infobox}

\section{Compartilhamento}\label{compartilhamento}

\subsection{Por que compartilhar scripts?}\label{por-que-compartilhar-scripts}

\begin{itemize}
\tightlist
\item
  Compartilhar o script --- principalmente junto aos dados --- pode facilitar a replicação direta do estudo, a detecção de eventuais erros de análise, a detecção de pesquisas fraudulentas.\textsuperscript{\citeproc{ref-schultze2023}{54}}
\end{itemize}

\subsection{O que pode ser compartilhado?}\label{o-que-pode-ser-compartilhado}

\begin{itemize}
\item
  Idealmente, todos os scripts, pacotes/bibliotecas e dados necessários para outros reproduzirem seus dados.\textsuperscript{\citeproc{ref-Eglen2017}{42}}
\item
  Minimamente, partes importantes incluindo implementações de novos algoritmos e dados que permitam reproduzir um resultado importante.\textsuperscript{\citeproc{ref-Eglen2017}{42}}
\end{itemize}

\subsection{Como preparar dados para compartilhamento?}\label{como-preparar-dados-para-compartilhamento}

\begin{itemize}
\tightlist
\item
  .\textsuperscript{\citeproc{ref-REF}{\textbf{REF?}}}
\end{itemize}

\subsection{Como preparar scripts para compartilhamento?}\label{como-preparar-scripts-para-compartilhamento}

\begin{itemize}
\item
  Providencie a documentação sobre seu script (ex.: arquivo README).\textsuperscript{\citeproc{ref-Eglen2017}{42}}
\item
  Inclua a versão dos pacotes usados no seu script por meio de um script inicial para instalação de pacotes (ex.: `instalar.R').\textsuperscript{\citeproc{ref-trisovic2022}{49}}
\item
  Documente em um arquivo README os arquivos disponíveis e os pré-requisitos necessários para executar o código (ex.: pacotes e respectivas versões). Uma lista de configurações (hardware e software) que foram usadas para rodar o código pode ajudar na reprodução dos resultados.\textsuperscript{\citeproc{ref-hofner2015}{27}}
\item
  Use endereços de arquivos relativos.\textsuperscript{\citeproc{ref-trisovic2022}{49}}
\item
  Crie links persistentes para versões do seu script.\textsuperscript{\citeproc{ref-Eglen2017}{42}}
\item
  Defina uma semente para o gerador de números aleatórios em scripts com métodos computacionais que dependem da geração de números pseudoaleatórios.\textsuperscript{\citeproc{ref-hofner2015}{27}}
\end{itemize}

\begin{infobox}{images/Rlogo}
O pacote \emph{base}\textsuperscript{\citeproc{ref-base}{55}} fornece a função \href{https://www.rdocumentation.org/packages/base/versions/3.6.2/topics/Random}{\emph{set.seed}} para especificar uma semente para reprodutibilidade de computações que envolvem números aleatórios.

\end{infobox}

\begin{itemize}
\item
  Escolha uma licença apropriada para garantir os direitos de criação e como outros poderão usar seus scripts.\textsuperscript{\citeproc{ref-Eglen2017}{42}}
\item
  Teste o script em uma nova sessão antes de compartilhar.\textsuperscript{\citeproc{ref-trisovic2022}{49}}
\item
  Cite todos os pacotes relacionados à sua análise.\textsuperscript{\citeproc{ref-Zhao2023}{56}}
\end{itemize}

\begin{infobox}{images/Rlogo}
O pacote \emph{utils}\textsuperscript{\citeproc{ref-utils}{47}} fornece a função \href{https://www.rdocumentation.org/packages/utils/versions/3.6.2/topics/citation}{\emph{citation}} para citar o programa R e os pacotes da sessão atual.

\end{infobox}

\begin{infobox}{images/Rlogo}
O pacote \emph{grateful}\textsuperscript{\citeproc{ref-grateful}{57}} fornece a função \href{https://www.rdocumentation.org/packages/grateful/versions/0.2.0/topics/cite_packages}{\emph{cite\_packages}} para citar os pacotes utilizados em um projeto R.

\end{infobox}

\begin{itemize}
\tightlist
\item
  Inclua a informação da sessão em que os scripts foram rodados.\textsuperscript{\citeproc{ref-trisovic2022}{49}}
\end{itemize}

\begin{infobox}{images/Rlogo}
O pacote \emph{utils}\textsuperscript{\citeproc{ref-utils}{47}} fornece a função \href{https://www.rdocumentation.org/packages/utils/versions/3.6.2/topics/sessionInfo}{\emph{sessionInfo}} para descrever as características do programa, pacotes e plataforma da sessão atual.

\end{infobox}

\subsection{O que incluir no arquivo README?}\label{o-que-incluir-no-arquivo-readme}

\begin{itemize}
\item
  Título do trabalho.\textsuperscript{\citeproc{ref-hofner2015}{27}}
\item
  Autores do trabalho.\textsuperscript{\citeproc{ref-hofner2015}{27}}
\item
  Principais responsáveis pela escrita do script e quaisquer outras pessoas que fizeram contribuições substanciais para o desenvolvimento do script.\textsuperscript{\citeproc{ref-hofner2015}{27}}
\item
  Endereço de e-mail do autor ou contribuidor a quem devem ser direcionadas dúvidas, comentários, sugestões e bugs sobre o script.\textsuperscript{\citeproc{ref-hofner2015}{27}}
\item
  Lista de configurações nas quais o script foi testado, tais com nome e versão do programa, pacotes e plataforma.\textsuperscript{\citeproc{ref-hofner2015}{27}}
\end{itemize}

\chapter{\texorpdfstring{\textbf{Letramento estatístico}}{Letramento estatístico}}\label{letramento-estatistico}

\section{Introdução ao letramento estatístico}\label{introduuxe7uxe3o-ao-letramento-estatuxedstico}

\subsection{O que é letramento estatístico?}\label{o-que-uxe9-letramento-estatuxedstico}

\begin{itemize}
\item
  Letramento em informação: Capacidade de reconhecer quando a informação é necessária e de localizá-la, avaliá-la criticamente (qualidade, validade, relevância, completude, imparcialidade) e usá-la de forma eficaz e ética. Abrange qualquer tipo de informação, em texto ou números.\textsuperscript{\citeproc{ref-shields2005}{58}}
\item
  Letramento em dados: Competência técnica para acessar, manipular, resumir e apresentar dados, utilizando ferramentas e métodos (SQL, planilhas, softwares estatísticos), com foco na preparação e organização de conjuntos de dados para análise e comunicação.\textsuperscript{\citeproc{ref-shields2005}{58}}
\item
  Letramento estatístico é a competência para compreender, interpretar e avaliar informações baseadas em dados, integrando conhecimentos técnicos (linguagem, estatística, matemática) e contextuais com postura crítica, crenças e atitudes que sustentem o uso ético e fundamentado da estatística.\textsuperscript{\citeproc{ref-gal2002a}{59}--\citeproc{ref-hidayati2020}{61}}
\item
  Letramento estatístico é parte essencial do letramento informacional (fornece a capacidade de reconhecer, acessar e avaliar informações) e do letramento em dados (envolve acessar, manipular e apresentar dados de forma adequada).\textsuperscript{\citeproc{ref-shields2005}{58}}
\end{itemize}

\subsection{Por que o letramento estatístico é importante?}\label{por-que-o-letramento-estatuxedstico-uxe9-importante}

\begin{itemize}
\item
  A presença dos dados no cotidiano deixou de ser restrita a decisões políticas ou relatórios técnicos: hoje, todos estamos expostos e interagimos com dados de forma constante, seja por dispositivos móveis, redes sociais ou sistemas automatizados de recomendação.\textsuperscript{\citeproc{ref-gould2017}{62}}
\item
  Ferramentas para coletar e analisar dados estão mais acessíveis e baratas, o que amplia a possibilidade de qualquer pessoa atuar não só como consumidora, mas também como produtora de informações.\textsuperscript{\citeproc{ref-gould2017}{62}}
\end{itemize}

\subsection{Quais são exemplos de armadilhas comuns na interpretação de estatísticas?}\label{quais-suxe3o-exemplos-de-armadilhas-comuns-na-interpretauxe7uxe3o-de-estatuxedsticas}

\begin{itemize}
\item
  Escolha do indicador: usar média ou mediana pode levar a conclusões muito diferentes sobre o mesmo fenômeno (por exemplo, renda média vs.~mediana antes e depois de impostos).\textsuperscript{\citeproc{ref-shields2005}{58}}
\item
  Confusão entre taxas e contagens: comparar números absolutos sem considerar proporções populacionais pode distorcer a realidade.\textsuperscript{\citeproc{ref-shields2005}{58}}
\item
  Fatores de confusão: diferenças observadas podem ser explicadas por variáveis não consideradas, como idade média da população ao comparar taxas de mortalidade.\textsuperscript{\citeproc{ref-shields2005}{58}}
\end{itemize}

\section{Elementos centrais do letramento estatístico}\label{elementos-centrais-do-letramento-estatuxedstico}

\subsection{Quais são os elementos de conhecimento que sustentam o letramento estatístico?}\label{quais-suxe3o-os-elementos-de-conhecimento-que-sustentam-o-letramento-estatuxedstico}

\begin{itemize}
\tightlist
\item
  O modelo de letramento estatístico é composto por cinco elementos de conhecimento e dois elementos disposicionais.\textsuperscript{\citeproc{ref-gal2002a}{59}--\citeproc{ref-hidayati2020}{61}}
\end{itemize}

\subsection{Quais são os cinco elementos de conhecimento que sustentam o letramento estatístico?}\label{quais-suxe3o-os-cinco-elementos-de-conhecimento-que-sustentam-o-letramento-estatuxedstico}

\begin{itemize}
\item
  Competências de letramento, incluindo leitura de textos, gráficos e tabelas.\textsuperscript{\citeproc{ref-gal2002a}{59}}
\item
  Conhecimento estatístico básico, incluindo conceitos, métodos, interpretação de dados e probabilidade.\textsuperscript{\citeproc{ref-gal2002a}{59}}
\item
  Conhecimento matemático sobre percentagens, médias e raciocínio numérico.\textsuperscript{\citeproc{ref-gal2002a}{59}}
\item
  Conhecimento de contexto/mundo, com entendimento do cenário e origem dos dados.\textsuperscript{\citeproc{ref-gal2002a}{59}}
\item
  Questões críticas (lista de \emph{worry questions} para avaliar a validade da informação.\textsuperscript{\citeproc{ref-gal2002a}{59}}
\end{itemize}

\subsection{Quais são os dois elementos de disposição que facilitam a ação estatisticamente letrada?}\label{quais-suxe3o-os-dois-elementos-de-disposiuxe7uxe3o-que-facilitam-a-auxe7uxe3o-estatisticamente-letrada}

\begin{itemize}
\item
  Postura crítica: propensão para questionar e analisar mensagens quantitativas.\textsuperscript{\citeproc{ref-gal2002a}{59}}
\item
  Crenças e atitudes: visão positiva sobre a capacidade de pensar estatisticamente; valorização de dados bem produzidos.\textsuperscript{\citeproc{ref-gal2002a}{59}}
\end{itemize}

\subsection{Que tipo de perguntas críticas devemos fazer ao interpretar informação estatística?}\label{que-tipo-de-perguntas-cruxedticas-devemos-fazer-ao-interpretar-informauxe7uxe3o-estatuxedstica}

\begin{itemize}
\item
  De onde vêm os dados? Que tipo de estudo foi feito?\textsuperscript{\citeproc{ref-gal2002a}{59}}
\item
  A amostra é representativa e suficientemente grande?\textsuperscript{\citeproc{ref-gal2002a}{59}}
\item
  Os instrumentos de medição são confiáveis?\textsuperscript{\citeproc{ref-gal2002a}{59}}
\item
  As estatísticas e gráficos são apropriados e não distorcem?\textsuperscript{\citeproc{ref-gal2002a}{59}}
\item
  Há relação causal ou apenas correlação? Há informação em falta?\textsuperscript{\citeproc{ref-gal2002a}{59}}
\item
  Existem interpretações alternativas plausíveis?\textsuperscript{\citeproc{ref-gal2002a}{59}}
\end{itemize}

\section{Hierarquia de letramento estatístico}\label{hierarquia-de-letramento-estatuxedstico}

\subsection{Quais são os níveis da hierarquia de letramento estatístico?}\label{quais-suxe3o-os-nuxedveis-da-hierarquia-de-letramento-estatuxedstico}

\begin{itemize}
\item
  Nível 6 -- Crítico Matemático: É o nível mais alto. A pessoa questiona e analisa as informações de forma profunda, usando cálculos e raciocínio proporcional (como comparar porcentagens e proporções). Reconhece que previsões sempre envolvem algum grau de incerteza e percebe detalhes sutis na forma como os dados são apresentados.\textsuperscript{\citeproc{ref-callingham2017}{63}}
\item
  Nível 5 -- Crítico: Também envolve uma postura questionadora, mas sem exigir cálculos complexos de proporção. Usa corretamente a linguagem estatística, entende o significado de termos ligados à probabilidade e percebe que os resultados podem variar.\textsuperscript{\citeproc{ref-callingham2017}{63}}
\item
  Nível 4 -- Consistente, mas Não Crítico: Consegue interpretar dados e usar termos estatísticos corretamente, mas não chega a questionar a forma como as informações são apresentadas. Reconhece a variação apenas em situações que envolvem sorte ou acaso, e sabe lidar com conceitos como média, probabilidades simples e leitura de gráficos.\textsuperscript{\citeproc{ref-callingham2017}{63}}
\item
  Nível 3 -- Inconsistente: Analisa partes do problema, mas de forma irregular. Pode identificar conclusões corretas, mas sem explicá-las. Usa ideias estatísticas de maneira mais descritiva do que numérica.\textsuperscript{\citeproc{ref-callingham2017}{63}}
\item
  Nível 2 -- Informal: A interpretação é mais baseada no senso comum do que em conceitos estatísticos. Utiliza apenas alguns termos corretos e consegue fazer cálculos muito simples com tabelas, gráficos ou situações de probabilidade.\textsuperscript{\citeproc{ref-callingham2017}{63}}
\item
  Nível 1 -- Idiossincrático: Responde de forma muito pessoal ou confusa, usando termos de maneira incorreta ou limitada. Realiza apenas contagens diretas e leituras simples de dados.\textsuperscript{\citeproc{ref-callingham2017}{63}}
\end{itemize}

\subsection{Quais são os componentes centrais do letramento estatístico com literacia de dados?}\label{quais-suxe3o-os-componentes-centrais-do-letramento-estatuxedstico-com-literacia-de-dados}

\begin{itemize}
\item
  Compreender quem coleta dados, por que e como essa coleta é feita.\textsuperscript{\citeproc{ref-gould2017}{62}}
\item
  Saber interpretar dados de amostras aleatórias e não aleatórias, avaliando limitações e potencial.\textsuperscript{\citeproc{ref-gould2017}{62}}
\item
  Conhecer e aplicar práticas de proteção de dados e direitos de propriedade sobre informações coletadas.\textsuperscript{\citeproc{ref-gould2017}{62}}
\item
  Produzir representações descritivas (tabelas, gráficos, mapas, \emph{dashboards}) para responder perguntas sobre fenômenos reais.\textsuperscript{\citeproc{ref-gould2017}{62}}
\item
  Reconhecer a importância da proveniência e do armazenamento dos dados, bem como a necessidade de pré-processamento antes da análise.\textsuperscript{\citeproc{ref-gould2017}{62}}
\item
  Entender fundamentos de modelagem preditiva e algoritmos, como árvores de classificação e regressão, especialmente no contexto de dados massivos (\emph{big data}).\textsuperscript{\citeproc{ref-gould2017}{62}}
\end{itemize}

\section{Habilidades de letramento estatístico baseadas no pensamento crítico}\label{habilidades-de-letramento-estatuxedstico-baseadas-no-pensamento-cruxedtico}

\subsection{Quais são as habilidades de letramento estatístico?}\label{quais-suxe3o-as-habilidades-de-letramento-estatuxedstico}

\begin{itemize}
\item
  Identificar: Descobrir qual é a principal afirmação de um texto ou relatório e separar o que é opinião do que é realmente evidência ou dado.\textsuperscript{\citeproc{ref-koga2022}{64}}
\item
  Questionar: Fazer perguntas sobre os dados: de onde vieram, como foram coletados, qual o tamanho da amostra, se houve erros, se os gráficos estão claros e se o questionário foi bem feito.\textsuperscript{\citeproc{ref-koga2022}{64}}
\item
  Julgar: Avaliar se a afirmação é bem sustentada pelos dados ou se está exagerando, por exemplo, dizendo que algo causa quando só foi encontrada uma relação.\textsuperscript{\citeproc{ref-koga2022}{64}}
\item
  Esclarecer: Entender e explicar palavras técnicas e expressões que podem confundir, além de saber como foi feita a pesquisa e a análise.\textsuperscript{\citeproc{ref-koga2022}{64}}
\item
  Avaliar: Decidir se a afirmação é confiável comparando com outras informações disponíveis e verificando se faz sentido.\textsuperscript{\citeproc{ref-koga2022}{64}}
\item
  Investigar mais: Procurar informações que não foram mostradas, como quem fez a pesquisa, por que foi feita, detalhes do processo ou fatores escondidos que podem influenciar os resultados.\textsuperscript{\citeproc{ref-koga2022}{64}}
\item
  Considerar alternativas: Pensar em outras explicações possíveis ou diferentes interpretações para os mesmos dados.\textsuperscript{\citeproc{ref-koga2022}{64}}
\item
  Concluir: Chegar à sua própria conclusão sobre o assunto, usando as informações e o raciocínio de forma clara e bem fundamentada.\textsuperscript{\citeproc{ref-koga2022}{64}}
\end{itemize}

\cftaddtitleline{toc}{chapter}{\rule{\textwidth}{0.4pt}}{}

\chapter*{\texorpdfstring{\emph{PARTE 2: AMEAÇAS À QUALIDADE DA EVIDÊNCIA CIENTÍFICA}}{PARTE 2: AMEAÇAS À QUALIDADE DA EVIDÊNCIA CIENTÍFICA}}\label{parte-2}
\addcontentsline{toc}{chapter}{\emph{PARTE 2: AMEAÇAS À QUALIDADE DA EVIDÊNCIA CIENTÍFICA}}

\par\noindent\rule{\textwidth}{0.05in}

\section*{Vieses Metodológicos, Erros de Interpretação e Condutas Questionáveis}\label{vieses-metodoluxf3gicos-erros-de-interpretauxe7uxe3o-e-condutas-questionuxe1veis}

\markboth{}{}

\chapter{\texorpdfstring{\textbf{Vieses metodológicos}}{Vieses metodológicos}}\label{vieses-metodologicos}

\section{Vieses metodológicos}\label{vieses-metodoluxf3gicos}

\subsection{O que são vieses metodológicos?}\label{o-que-suxe3o-vieses-metodoluxf3gicos}

\begin{itemize}
\tightlist
\item
  .\textsuperscript{\citeproc{ref-REF}{\textbf{REF?}}}
\end{itemize}

\section{Tipos de vieses metodológicos}\label{tipos-de-vieses-metodoluxf3gicos}

\subsection{Quais são os tipos de vieses metodológicos?}\label{quais-suxe3o-os-tipos-de-vieses-metodoluxf3gicos}

\begin{itemize}
\tightlist
\item
  .\textsuperscript{\citeproc{ref-REF}{\textbf{REF?}}}
\end{itemize}

\section{Efeitos relacionados aos vieses metodológicos}\label{efeitos-relacionados-aos-vieses-metodoluxf3gicos}

\subsection{Quais são os efeitos relacionados aos vieses metodológicos?}\label{quais-suxe3o-os-efeitos-relacionados-aos-vieses-metodoluxf3gicos}

\begin{itemize}
\tightlist
\item
  .\textsuperscript{\citeproc{ref-REF}{\textbf{REF?}}}
\end{itemize}

\subsection{O que é efeito placebo?}\label{o-que-uxe9-efeito-placebo}

\begin{itemize}
\tightlist
\item
  .\textsuperscript{\citeproc{ref-REF}{\textbf{REF?}}}
\end{itemize}

\subsection{O que é efeito nocebo?}\label{o-que-uxe9-efeito-nocebo}

\begin{itemize}
\tightlist
\item
  .\textsuperscript{\citeproc{ref-REF}{\textbf{REF?}}}
\end{itemize}

\subsection{O que é efeito Hawthorne?}\label{o-que-uxe9-efeito-hawthorne}

\begin{itemize}
\tightlist
\item
  .\textsuperscript{\citeproc{ref-REF}{\textbf{REF?}}}
\end{itemize}

\subsection{O que é efeito Rosenthal?}\label{o-que-uxe9-efeito-rosenthal}

\begin{itemize}
\tightlist
\item
  .\textsuperscript{\citeproc{ref-REF}{\textbf{REF?}}}
\end{itemize}

\section{Diretrizes para redação}\label{diretrizes-para-redauxe7uxe3o}

\subsection{Quais são as diretrizes para redação de análises de vieses metodológicos?}\label{quais-suxe3o-as-diretrizes-para-redauxe7uxe3o-de-anuxe1lises-de-vieses-metodoluxf3gicos}

\begin{itemize}
\item
  Visite a rede \emph{Enhancing the QUAlity and Transparency Of health Research} (\href{https://www.equator-network.org/}{EQUATOR Network}) para encontrar diretrizes específicas.
\item
  \emph{PROBAST: A Tool to Assess the Risk of Bias and Applicability of Prediction Model Studies}.\textsuperscript{\citeproc{ref-wolff2019}{65}}
\item
  \emph{RoB 2: A Revised Tool for Assessing Risk of Bias in Randomized Trials}.\textsuperscript{\citeproc{ref-sterne2019}{66}}
\item
  \emph{AMSTAR 2: A Critical Appraisal Tool for Systematic Reviews that Include Randomised or Non-Randomised Studies of Healthcare Interventions}\textsuperscript{\citeproc{ref-shea2017}{67}}
\item
  \emph{ROBINS-I: A Tool for Assessing Risk of Bias in Non-randomized Studies of Interventions}.\textsuperscript{\citeproc{ref-sterne2016}{68}}
\item
  \emph{ROBIS: A New Tool to Assess Risk of Bias in Systematic Reviews}\textsuperscript{\citeproc{ref-whiting2016}{69}}
\item
  \emph{QUADAS-2: A Revised Tool for the Quality Assessment of Diagnostic Accuracy Studies}\textsuperscript{\citeproc{ref-whiting2011}{70}}
\end{itemize}

\chapter{\texorpdfstring{\textbf{Falácias estatísticas}}{Falácias estatísticas}}\label{falacias-estatisticas}

\section{Falácias}\label{faluxe1cias}

\subsection{O que são falácias estatísticas?}\label{o-que-suxe3o-faluxe1cias-estatuxedsticas}

\begin{itemize}
\item
  Falácias estatísticas são erros de raciocínio que ocorrem em situações que envolvem dados e estatísticas. Elas podem ocorrer em qualquer etapa do processo de análise de dados, desde a coleta até a interpretação dos resultados.\textsuperscript{\citeproc{ref-REF}{\textbf{REF?}}}
\item
  Falácias podem ser intencionais ou não intencionais, e podem ser usadas para manipular, enganar ou confundir as pessoas.\textsuperscript{\citeproc{ref-REF}{\textbf{REF?}}}
\item
  As falácias estatísticas podem ser difíceis de detectar, pois muitas vezes são sutis e podem parecer plausíveis à primeira vista. No entanto, é importante estar ciente delas e saber como identificá-las para evitar erros de interpretação e tomada de decisão.\textsuperscript{\citeproc{ref-REF}{\textbf{REF?}}}
\end{itemize}

\subsection{O que é a falácia do jogador?}\label{o-que-uxe9-a-faluxe1cia-do-jogador}

\begin{itemize}
\tightlist
\item
  A falácia do jogador é a crença de que eventos independentes têm uma influência sobre eventos futuros. Por exemplo, se uma moeda é lançada várias vezes e cai cara em todas as vezes, a falácia do jogador sugere que a próxima jogada será coroa, pois a moeda ``deve'' se equilibrar. No entanto, cada lançamento da moeda é independente e não afeta o resultado do próximo lançamento.\textsuperscript{\citeproc{ref-polin2023}{71}}
\end{itemize}

\subsection{O que é a falácia da mão quente?}\label{o-que-uxe9-a-faluxe1cia-da-muxe3o-quente}

\begin{itemize}
\tightlist
\item
  A falácia da mão quente é a crença de que um jogador que teve sucesso em um jogo de azar terá mais chances de sucesso no futuro. Por exemplo, se uma moeda é lançada várias vezes e cai cara em todas as vezes, a falácia da mão quente sugere que a próxima jogada será cara, pois o jogador está ``quente''. No entanto, cada lançamento da moeda é independente e não afeta o resultado do próximo lançamento.\textsuperscript{\citeproc{ref-polin2023}{71}}
\end{itemize}

\chapter{\texorpdfstring{\textbf{Paradoxos estatísticos}}{Paradoxos estatísticos}}\label{paradoxos-estatisticos}

\section{Paradoxos}\label{paradoxos}

\subsection{O que são paradoxos estatísticos?}\label{o-que-suxe3o-paradoxos-estatuxedsticos}

\begin{itemize}
\tightlist
\item
  Paradoxos podem originar da incompreensão ou mal informação da nossa intuição a respeito do fenômeno.\textsuperscript{\citeproc{ref-meng2018}{72}}
\end{itemize}

\subsection{O que é o paradoxo de Abelson?}\label{o-que-uxe9-o-paradoxo-de-abelson}

\begin{itemize}
\tightlist
\item
  .\textsuperscript{\citeproc{ref-abelson1985}{73}}
\end{itemize}

\subsection{O que é o paradoxo de Berkson?}\label{o-que-uxe9-o-paradoxo-de-berkson}

\begin{itemize}
\tightlist
\item
  .\textsuperscript{\citeproc{ref-berkson1946}{74}}
\end{itemize}

\subsection{O que é o paradoxo de grandes dados?}\label{o-que-uxe9-o-paradoxo-de-grandes-dados}

\begin{itemize}
\tightlist
\item
  \emph{Big Data}: ``Quanto maior a quantidade de dados, maior a certeza de que vamos nos enganar''.\textsuperscript{\citeproc{ref-meng2018}{72}}
\end{itemize}

\subsection{O que é o paradoxo de Ellsberg?}\label{o-que-uxe9-o-paradoxo-de-ellsberg}

\begin{itemize}
\tightlist
\item
  .\textsuperscript{\citeproc{ref-ellsberg1961}{75}}
\end{itemize}

\subsection{O que é o paradoxo de Freedman?}\label{o-que-uxe9-o-paradoxo-de-freedman}

\begin{itemize}
\tightlist
\item
  .\textsuperscript{\citeproc{ref-freedman1983}{76},\citeproc{ref-freedman1989}{77}}
\end{itemize}

\subsection{O que é o paradoxo de Hand?}\label{o-que-uxe9-o-paradoxo-de-hand}

\begin{itemize}
\tightlist
\item
  .\textsuperscript{\citeproc{ref-hand1992}{78}}
\end{itemize}

\subsection{O que é o paradoxo de Kelley?}\label{o-que-uxe9-o-paradoxo-de-kelley}

\begin{itemize}
\tightlist
\item
  .\textsuperscript{\citeproc{ref-REF}{\textbf{REF?}}}
\end{itemize}

\subsection{O que é o paradoxo de Lindley?}\label{o-que-uxe9-o-paradoxo-de-lindley}

\begin{itemize}
\tightlist
\item
  .\textsuperscript{\citeproc{ref-lindley1957}{79}}
\end{itemize}

\subsection{O que é o paradoxo de Lord?}\label{o-que-uxe9-o-paradoxo-de-lord}

\begin{itemize}
\tightlist
\item
  .\textsuperscript{\citeproc{ref-lord1967}{80},\citeproc{ref-lord1969}{81}}
\end{itemize}

\subsection{O que é o paradoxo de Proebsting?}\label{o-que-uxe9-o-paradoxo-de-proebsting}

\begin{itemize}
\tightlist
\item
  .\textsuperscript{\citeproc{ref-REF}{\textbf{REF?}}}
\end{itemize}

\subsection{O que é o paradoxo de Simpson?}\label{o-que-uxe9-o-paradoxo-de-simpson}

\begin{itemize}
\item
  O paradoxo de Simpson ocorre quando a associação entre duas variáveis \(X\) e \(Y\) desaparece ou mesmo reverte sua direção quando condicionadas em uma terceira variável \(Z\).\textsuperscript{\citeproc{ref-simpson1951}{82},\citeproc{ref-blyth1972}{83}}
\item
  Para decisão do paradoxo de Simpson pode-se utilizar o conceito de `back-door', o qual considera os `caminhos' (isto é, associações) no gráfico acíclio direcionado e assegura que todos as associações espúrias do tratamento \(X\) para o desfecho \(Y\) nesse diagrama causal sejam interceptados pela variável \(Z\).\textsuperscript{\citeproc{ref-pearl2014}{84}}
\item
  Dependendo do contexto em que os dados foram obtidos --- delineamento do estudo, escolha dos instrumentos e dos tipos de variáveis --- a melhor escolha para a análise pode variar entre a análise da população agregada ou da subpopulação desagregada.\textsuperscript{\citeproc{ref-pearl2014}{84}}
\item
  É possível que em alguns contextos nem a análise agregada ou a desagregada podem oferecer a resposta correta, sendo necessário o uso de outras (mais) covariáveis.\textsuperscript{\citeproc{ref-pearl2014}{84}}
\end{itemize}

\begin{figure}

{\centering \includegraphics{Ciencia-com-R_files/figure-latex/simpson-1} 

}

\caption{Paradoxo de Simpson representado com dados simulados. Os pontos no gráfico representam observações individuais e as linhas de tendência representam as regressões lineares ajustadas para os dados desagregados da população e agregados por subpopulação.}\label{fig:simpson}
\end{figure}

\subsection{O que é o paradoxo de Stein?}\label{o-que-uxe9-o-paradoxo-de-stein}

\begin{itemize}
\tightlist
\item
  .\textsuperscript{\citeproc{ref-stein1956}{85}}
\end{itemize}

\subsection{O que é o paradoxo de Okie?}\label{o-que-uxe9-o-paradoxo-de-okie}

\begin{itemize}
\tightlist
\item
  .\textsuperscript{\citeproc{ref-REF}{\textbf{REF?}}}
\end{itemize}

\subsection{O que é o paradoxo da acurácia?}\label{o-que-uxe9-o-paradoxo-da-acuruxe1cia}

\begin{itemize}
\tightlist
\item
  .\textsuperscript{\citeproc{ref-REF}{\textbf{REF?}}}
\end{itemize}

\subsection{O que é o paradoxo do falso positivo?}\label{o-que-uxe9-o-paradoxo-do-falso-positivo}

\begin{itemize}
\tightlist
\item
  .\textsuperscript{\citeproc{ref-REF}{\textbf{REF?}}}
\end{itemize}

\subsection{O que é o paradoxo da caixa de Bertrand?}\label{o-que-uxe9-o-paradoxo-da-caixa-de-bertrand}

\begin{itemize}
\tightlist
\item
  .\textsuperscript{\citeproc{ref-REF}{\textbf{REF?}}}
\end{itemize}

\subsection{O que é o paradoxo do elevador?}\label{o-que-uxe9-o-paradoxo-do-elevador}

\begin{itemize}
\tightlist
\item
  .\textsuperscript{\citeproc{ref-de1996}{86}}
\end{itemize}

\subsection{O que é o paradoxo da amizade?}\label{o-que-uxe9-o-paradoxo-da-amizade}

\begin{itemize}
\tightlist
\item
  .\textsuperscript{\citeproc{ref-feld1991}{87}}
\end{itemize}

\subsection{O que é o paradoxo do menino ou menina?}\label{o-que-uxe9-o-paradoxo-do-menino-ou-menina}

\begin{itemize}
\tightlist
\item
  .\textsuperscript{\citeproc{ref-de1996}{86}}
\end{itemize}

\subsection{O que é o paradoxo do aniversário?}\label{o-que-uxe9-o-paradoxo-do-aniversuxe1rio}

\begin{itemize}
\tightlist
\item
  .\textsuperscript{\citeproc{ref-REF}{\textbf{REF?}}}
\end{itemize}

\subsection{O que é o paradoxo do teste surpresa?}\label{o-que-uxe9-o-paradoxo-do-teste-surpresa}

\begin{itemize}
\tightlist
\item
  .\textsuperscript{\citeproc{ref-REF}{\textbf{REF?}}}
\end{itemize}

\subsection{O que é o paradoxo do nó da gravata?}\label{o-que-uxe9-o-paradoxo-do-nuxf3-da-gravata}

\begin{itemize}
\tightlist
\item
  .\textsuperscript{\citeproc{ref-REF}{\textbf{REF?}}}
\end{itemize}

\subsection{O que é o paradoxo de Monty Hall?}\label{o-que-uxe9-o-paradoxo-de-monty-hall}

\begin{itemize}
\tightlist
\item
  .\textsuperscript{\citeproc{ref-REF}{\textbf{REF?}}}
\end{itemize}

\subsection{O que é o paradoxo da Bela Adormecida?}\label{o-que-uxe9-o-paradoxo-da-bela-adormecida}

\begin{itemize}
\tightlist
\item
  .\textsuperscript{\citeproc{ref-REF}{\textbf{REF?}}}
\end{itemize}

\chapter{\texorpdfstring{\textbf{Práticas questionáveis em pesquisa}}{Práticas questionáveis em pesquisa}}\label{praticas-questionaveis}

\section{Práticas Questionáveis em Pesquisa}\label{pruxe1ticas-questionuxe1veis-em-pesquisa}

\subsection{O que são práticas questionáveis em pesquisa?}\label{o-que-suxe3o-pruxe1ticas-questionuxe1veis-em-pesquisa}

\begin{itemize}
\tightlist
\item
  Práticas questionáveis em pesquisa são más condutas ou comportamentos impróprios, realizados desde o planejamento até a publicação dos resultados.\textsuperscript{\citeproc{ref-john2012}{88},\citeproc{ref-bausell2021}{89}}
\end{itemize}

\subsection{Por que práticas questionáveis em pesquisa devem ser combatidas?}\label{por-que-pruxe1ticas-questionuxe1veis-em-pesquisa-devem-ser-combatidas}

\begin{itemize}
\item
  Práticas questionáveis em pesquisa são prevalentes.\textsuperscript{\citeproc{ref-neoh2023}{90}}
\item
  Práticas questionáveis em pesquisa comprometem a integridade científica, a confiabilidade dos resultados e a confiança do público na ciência.\textsuperscript{\citeproc{ref-john2012}{88},\citeproc{ref-bausell2021}{89}}
\item
  Práticas questionáveis em pesquisa inflam artificialmente o tamanho do efeito e poder estatístico.\textsuperscript{\citeproc{ref-bausell2021}{89}}
\item
  Práticas questionáveis em pesquisa parecem contribuir para a crise da replicação na ciência, onde muitos estudos não conseguem ser replicados ou reproduzidos.\textsuperscript{\citeproc{ref-bausell2021}{89}}
\end{itemize}

\section{Prática não intencional e má conduta}\label{pruxe1tica-nuxe3o-intencional-e-muxe1-conduta}

\subsection{Quais são as categorias de práticas questionáveis em pesquisa?}\label{quais-suxe3o-as-categorias-de-pruxe1ticas-questionuxe1veis-em-pesquisa}

\begin{itemize}
\item
  Práticas questionáveis podem ser classificadas em más condutas e não intencionais.\textsuperscript{\citeproc{ref-Kleinert2009}{91}}
\item
  Más condutas são aquelas que são deliberadamente realizadas com o objetivo de enganar ou manipular os resultados, enquanto práticas não intencionais são aquelas que ocorrem devido a falta de conhecimento, treinamento inadequado ou outras razões.\textsuperscript{\citeproc{ref-REF}{\textbf{REF?}}}
\item
  Práticas na zona cinzenta são aquelas que podem ser interpretadas de diferentes maneiras, dependendo do contexto e da intenção do pesquisador.\textsuperscript{\citeproc{ref-REF}{\textbf{REF?}}}
\end{itemize}

\begin{table}
\centering
\caption{\label{tab:praticas-questionaveis}Classificação das práticas questionáveis em pesquisa segundo sua intencionalidade.}
\centering
\begin{tabu} to \linewidth {>{\centering}X>{\centering}X>{\centering}X}
\toprule
\textbf{Prática} & \textbf{Intencionalidade} & \textbf{Definição}\\
\midrule
\cellcolor{lightred}{\cellcolor{gray!10}{Data fabrication}} & \cellcolor{lightred}{\cellcolor{gray!10}{Má conduta}} & \cellcolor{lightred}{\cellcolor{gray!10}{Inventar dados inexistentes}}\\
\cellcolor{lightred}{Data falsification} & \cellcolor{lightred}{Má conduta} & \cellcolor{lightred}{Alterar ou manipular dados reais}\\
\cellcolor{lightred}{\cellcolor{gray!10}{Fake authorship}} & \cellcolor{lightred}{\cellcolor{gray!10}{Má conduta}} & \cellcolor{lightred}{\cellcolor{gray!10}{Inserir autores fictícios ou inexistentes}}\\
\cellcolor{lightred}{Fake peer review} & \cellcolor{lightred}{Má conduta} & \cellcolor{lightred}{Criar revisões falsas para facilitar publicação}\\
\cellcolor{lightred}{\cellcolor{gray!10}{Honorary authorship}} & \cellcolor{lightred}{\cellcolor{gray!10}{Má conduta}} & \cellcolor{lightred}{\cellcolor{gray!10}{Incluir autores sem contribuição real}}\\
\cellcolor{lightred}{Gold authorship} & \cellcolor{lightred}{Má conduta} & \cellcolor{lightred}{Atribuir autoria como forma de prestígio ou recompensa}\\
\cellcolor{lightred}{\cellcolor{gray!10}{Ghost authorship}} & \cellcolor{lightred}{\cellcolor{gray!10}{Má conduta}} & \cellcolor{lightred}{\cellcolor{gray!10}{Omitir autores que participaram do estudo}}\\
\cellcolor{lightred}{Duplicate publication} & \cellcolor{lightred}{Má conduta} & \cellcolor{lightred}{Publicar o mesmo estudo em mais de uma revista}\\
\cellcolor{lightred}{\cellcolor{gray!10}{Spin (doloso)}} & \cellcolor{lightred}{\cellcolor{gray!10}{Má conduta}} & \cellcolor{lightred}{\cellcolor{gray!10}{Apresentar os resultados de forma a exagerar efeitos positivos}}\\
\cellcolor{lightred}{Data distortion} & \cellcolor{lightred}{Má conduta} & \cellcolor{lightred}{Modificar dados ou gráficos para torná-los mais convincentes}\\
\cellcolor{lightred}{\cellcolor{gray!10}{SPARKing}} & \cellcolor{lightred}{\cellcolor{gray!10}{Má conduta}} & \cellcolor{lightred}{\cellcolor{gray!10}{Ajustar o tamanho da amostra após a coleta dos dados para obter significância estatística}}\\
\cellcolor{lightyellow}{HARKing} & \cellcolor{lightyellow}{Zona cinzenta} & \cellcolor{lightyellow}{Criar hipóteses após ver os dados (sem pré-registro)}\\
\cellcolor{lightyellow}{\cellcolor{gray!10}{Storytelling}} & \cellcolor{lightyellow}{\cellcolor{gray!10}{Zona cinzenta}} & \cellcolor{lightyellow}{\cellcolor{gray!10}{Construir uma narrativa forçada para justificar os achados}}\\
\cellcolor{lightyellow}{Selective reporting} & \cellcolor{lightyellow}{Zona cinzenta} & \cellcolor{lightyellow}{Relatar apenas os resultados favoráveis ou positivos}\\
\cellcolor{lightyellow}{\cellcolor{gray!10}{P-hacking}} & \cellcolor{lightyellow}{\cellcolor{gray!10}{Zona cinzenta}} & \cellcolor{lightyellow}{\cellcolor{gray!10}{Testar múltiplas análises até encontrar p<0.05}}\\
\cellcolor{lightyellow}{Data peeking} & \cellcolor{lightyellow}{Zona cinzenta} & \cellcolor{lightyellow}{Analisar dados antes do término da coleta, parando quando um efeito aparece}\\
\cellcolor{lightyellow}{\cellcolor{gray!10}{Cherry picking}} & \cellcolor{lightyellow}{\cellcolor{gray!10}{Zona cinzenta}} & \cellcolor{lightyellow}{\cellcolor{gray!10}{Selecionar apenas os resultados que apoiam a hipótese}}\\
\cellcolor{lightyellow}{Salami slicing} & \cellcolor{lightyellow}{Zona cinzenta} & \cellcolor{lightyellow}{Dividir artificialmente um estudo em vários artigos para inflar publicações}\\
\cellcolor{lightyellow}{\cellcolor{gray!10}{Beautification}} & \cellcolor{lightyellow}{\cellcolor{gray!10}{Zona cinzenta}} & \cellcolor{lightyellow}{\cellcolor{gray!10}{Embelezar tabelas, gráficos ou resultados para torná-los mais atraentes}}\\
\cellcolor{lightgreen}{P-hacking reverso} & \cellcolor{lightgreen}{Não intencional} & \cellcolor{lightgreen}{Forçar análises para que não haja significância estatística}\\
\cellcolor{lightgreen}{\cellcolor{gray!10}{Fishing expedition}} & \cellcolor{lightgreen}{\cellcolor{gray!10}{Não intencional}} & \cellcolor{lightgreen}{\cellcolor{gray!10}{Procurar achados sem plano prévio}}\\
\cellcolor{lightgreen}{Data dredging} & \cellcolor{lightgreen}{Não intencional} & \cellcolor{lightgreen}{Explorar excessivamente os dados para encontrar associações irrelevantes}\\
\cellcolor{lightgreen}{\cellcolor{gray!10}{File drawer problem}} & \cellcolor{lightgreen}{\cellcolor{gray!10}{Não intencional}} & \cellcolor{lightgreen}{\cellcolor{gray!10}{Não publicar estudos com resultados negativos ou nulos}}\\
\cellcolor{lightgreen}{Publication bias} & \cellcolor{lightgreen}{Não intencional} & \cellcolor{lightgreen}{Tendência geral das revistas em favorecer publicações com resultados positivos}\\
\bottomrule
\end{tabu}
\end{table}

\subsection{Quais práticas questionáveis podem ocorrer durante o planejamento do estudo?}\label{quais-pruxe1ticas-questionuxe1veis-podem-ocorrer-durante-o-planejamento-do-estudo}

\begin{itemize}
\item
  \emph{Hypothesizing After Results are Known} (HARKing) consiste em formular hipóteses após a análise dos dados, o que pode levar a resultados enviesados e não replicáveis.\textsuperscript{\citeproc{ref-Kerr1998}{92}}
\item
  \emph{Storytelling} é a prática de criar narrativas convincentes para justificar os resultados, mesmo que não sejam suportados pelos dados.\textsuperscript{\citeproc{ref-REF}{\textbf{REF?}}}
\end{itemize}

\subsection{Quais práticas questionáveis podem ocorrer durante a coleta de dados?}\label{quais-pruxe1ticas-questionuxe1veis-podem-ocorrer-durante-a-coleta-de-dados}

\begin{itemize}
\item
  \emph{Data falsification} é a prática de manipular ou inventar dados para obter resultados desejados.\textsuperscript{\citeproc{ref-REF}{\textbf{REF?}}}
\item
  \emph{Data fabrication} é a prática de inventar dados ou resultados que nunca foram coletados.\textsuperscript{\citeproc{ref-REF}{\textbf{REF?}}}
\end{itemize}

\subsection{Quais práticas questionáveis podem ocorrer durante a análise dos dados?}\label{quais-pruxe1ticas-questionuxe1veis-podem-ocorrer-durante-a-anuxe1lise-dos-dados}

\begin{itemize}
\item
  \emph{P-hacking} é a prática de manipular os dados ou análises para obter resultados estatisticamente significativos, como realizar múltiplos testes sem correção adequada.\textsuperscript{\citeproc{ref-degroot2014}{93}--\citeproc{ref-stefan2023}{95}}
\item
  \emph{P-hacking} reverso é a prática de manipular os dados ou análises para obter resultados não estatisticamente significativos, como realizar múltiplos testes sem correção adequada, o que pode levar a conclusões enviesadas e enganosas.\textsuperscript{\citeproc{ref-chuard2019}{96}}
\item
  \emph{SPARKing} (\emph{Sample size Planning After the Results are Known}) é uma mal prática que envolve o ajuste do tamanho da amostra após a coleta dos dados, com o objetivo de obter resultados estatisticamente significativos.\textsuperscript{\citeproc{ref-Sasaki2023}{97}}
\item
  \emph{Data peeking} é a prática de analisar os dados repetidamente antes de completar a coleta, visando interromper a coleta quando um resultado desejado é alcançado.\textsuperscript{\citeproc{ref-armitage1969}{98}}
\item
  \emph{Fishing expedition} refere-se à exploração dos dados sem uma hipótese pré-definida, o que pode levar a conclusões enganosas e enviesadas, uma vez que os resultados podem ser meramente acidentais.\textsuperscript{@ \citeproc{ref-andrade2021}{94}}
\item
  \emph{Data dredging} refere-se à exploração excessiva dos dados para encontrar padrões ou relações que não são teoricamente fundamentados, o que pode resultar em conclusões enganosas e enviesadas.\textsuperscript{\citeproc{ref-andrade2021}{94}}
\item
  \emph{Selective reporting} é a prática de relatar apenas os resultados que suportam uma hipótese específica, ignorando aqueles que não a apoiam, o que pode levar a conclusões enganosas e enviesadas.\textsuperscript{\citeproc{ref-hutton2000}{99}}
\end{itemize}

\subsection{Quais práticas questionáveis podem ocorrer durante a apresentação dos resultados?}\label{quais-pruxe1ticas-questionuxe1veis-podem-ocorrer-durante-a-apresentauxe7uxe3o-dos-resultados}

\begin{itemize}
\item
  \emph{Cherry picking} consiste em selecionar apenas os resultados que suportam uma hipótese específica, ignorando aqueles que não a apoiam, o que pode levar a conclusões enganosas e enviesadas.\textsuperscript{\citeproc{ref-andrade2021}{94}}
\item
  \emph{Spin} é a prática de apresentar os resultados de forma a enfatizar aspectos positivos ou minimizar aspectos negativos, o que pode levar a interpretações enganosas e enviesadas dos dados.\textsuperscript{\citeproc{ref-horton1995}{100},\citeproc{ref-chiu2017}{101}}
\item
  \emph{Beautification} é a prática de embelezar visualmente gráficos ou tabelas para aumentar impacto visual.\textsuperscript{\citeproc{ref-REF}{\textbf{REF?}}}
\item
  \emph{Data distortion} é a prática de modificar ou omitir informações nos dados para induzir interpretações específicas.\textsuperscript{\citeproc{ref-REF}{\textbf{REF?}}}
\end{itemize}

\subsection{Quais práticas questionáveis podem ocorrer durante a publicação e revisão por pares?}\label{quais-pruxe1ticas-questionuxe1veis-podem-ocorrer-durante-a-publicauxe7uxe3o-e-revisuxe3o-por-pares}

\begin{itemize}
\item
  \emph{Honorary authorship} refere-se à inclusão de autores que não contribuíram significativamente para o estudo, o que pode distorcer a atribuição de crédito e responsabilidade.\textsuperscript{\citeproc{ref-picano2024}{102}}
\item
  \emph{Ghost authorship} é a prática de não reconhecer autores que contribuíram significativamente para o estudo, o que pode distorcer a atribuição de crédito e responsabilidade.\textsuperscript{\citeproc{ref-picano2024}{102}}
\item
  \emph{Gold authorship} é a prática de atribuir autoria em troca de prestígio, recursos ou favorecimento político, independentemente da contribuição acadêmica.\textsuperscript{\citeproc{ref-picano2024}{102}}
\item
  \emph{Fake authorship} refere-se à inclusão de autores fictícios ou inexistentes em uma publicação.\textsuperscript{\citeproc{ref-picano2024}{102}}
\item
  \emph{Fake peer review} refere-se à prática de criar revisões por pares falsas ou fraudulentas para apoiar a publicação de um estudo, o que compromete a integridade do processo de revisão por pares e pode levar a conclusões enganosas.\textsuperscript{\citeproc{ref-REF}{\textbf{REF?}}}
\item
  \emph{File drawer problem} refere-se à tendência de não publicar estudos com resultados negativos ou não significativos, o que pode levar a uma visão distorcida da literatura científica e dificultar a replicação de estudos.\textsuperscript{\citeproc{ref-REF}{\textbf{REF?}}}
\item
  \emph{Salami slicing} é a prática de dividir os resultados em múltiplas publicações para aumentar o número de publicações, o que pode levar a uma má interpretação dos dados e à fragmentação do conhecimento.\textsuperscript{\citeproc{ref-REF}{\textbf{REF?}}}
\item
  \emph{Publication bias} é a tendência de publicar apenas resultados positivos ou significativos, o que pode levar a uma visão distorcida da literatura científica e dificultar a replicação de estudos.\textsuperscript{\citeproc{ref-montori2000}{103}}
\item
  \emph{Duplicate publication} é a prática de publicar o mesmo estudo ou resultados em mais de uma revista, o que pode levar a uma superestimação da importância dos resultados e à confusão na literatura científica.\textsuperscript{\citeproc{ref-REF}{\textbf{REF?}}}
\end{itemize}

\section{Prevenindo práticas questionáveis em pesquisa}\label{prevenindo-pruxe1ticas-questionuxe1veis-em-pesquisa}

\subsection{Como prevenir práticas questionáveis?}\label{como-prevenir-pruxe1ticas-questionuxe1veis}

\begin{itemize}
\item
  Educação formal em integridade científica e estatística.\textsuperscript{\citeproc{ref-REF}{\textbf{REF?}}}
\item
  Pré-registro do protocolo do estudo de ensaios clínicos (ex.: \href{https://ensaiosclinicos.gov.br}{ReBEC}, \href{https://clinicaltrials.gov}{ClinicalTrials.gov}, revisões sistemáticas (ex.: \href{https://www.crd.york.ac.uk/prospero/}{PROSPERO}), ou outras plataformas (ex.: \href{https://osf.io}{OSF}).\textsuperscript{\citeproc{ref-nosek2018}{104},\citeproc{ref-p.simmons2021}{105}}
\item
  Planos de análise detalhados.\textsuperscript{\citeproc{ref-REF}{\textbf{REF?}}}
\item
  Compartilhamento de dados/scripts (reprodutibilidade).\textsuperscript{\citeproc{ref-REF}{\textbf{REF?}}}

  \begin{itemize}
  \item
    \href{http://datadryad.org/}{Dryad Digital Repository}
  \item
    \href{http://figshare.com/}{figshare}
  \item
    \href{http://dataverse.harvard.edu/}{Harvard Dataverse}
  \item
    \href{https://data.mendeley.com/}{Mendeley Data}
  \item
    \href{https://osf.io/}{Open Science Framework}
  \item
    \href{http://zenodo.org/}{Zenodo}
  \end{itemize}
\item
  Manuscritos reprodutíveis (RMarkdown, bookdown, etc.).\textsuperscript{\citeproc{ref-REF}{\textbf{REF?}}}
\item
  Adoção de diretrizes para redação de manuscritos (CONSORT, STROBE, PRISMA).\textsuperscript{\citeproc{ref-REF}{\textbf{REF?}}}
\end{itemize}

\section{Reações éticas e institucionais práticas questionáveis em pesquisa}\label{reauxe7uxf5es-uxe9ticas-e-institucionais-pruxe1ticas-questionuxe1veis-em-pesquisa}

\begin{itemize}
\item
  \emph{Post-publication peer review} é a prática de revisar e criticar publicações após sua publicação, o que pode levar a uma melhor compreensão dos resultados e à correção de erros, mas também pode ser usada para desacreditar estudos sem justificativa adequada.\textsuperscript{\citeproc{ref-REF}{\textbf{REF?}}}
\item
  \emph{Corrigendum} é uma correção publicada para corrigir erros ou imprecisões em um artigo já publicado, o que pode levar a uma melhor compreensão dos resultados e à correção de erros, mas também pode ser usada para desacreditar estudos sem justificativa adequada.\textsuperscript{\citeproc{ref-REF}{\textbf{REF?}}}
\item
  \emph{Expression of concern} é uma declaração emitida por uma revista científica para alertar os leitores sobre preocupações com a integridade de um estudo, sem necessariamente retirar o artigo.\textsuperscript{\citeproc{ref-REF}{\textbf{REF?}}}
\item
  \emph{Retraction} é a prática de retirar uma publicação devido a erros, fraudes ou práticas questionáveis, o que pode levar a uma melhor compreensão dos resultados e à correção de erros, mas também pode ser usada para desacreditar estudos sem justificativa adequada.\textsuperscript{\citeproc{ref-REF}{\textbf{REF?}}}
\item
  \href{https://retractionwatch.com}{Retraction Watch} é um blog que monitora e relata casos de retratações e preocupações éticas em publicações científicas, fornecendo informações sobre práticas questionáveis e promovendo a transparência na pesquisa.\textsuperscript{\citeproc{ref-REF}{\textbf{REF?}}}
\end{itemize}

\begin{infobox}{images/Rlogo}
O pacote \emph{retractcheck}\textsuperscript{\citeproc{ref-retractcheck}{106}} fornece a função \href{https://rdrr.io/github/chartgerink/retractcheck/man/retractcheck.html}{\emph{retractcheck}} para verificar se um artigo foi retratado usando a \href{http://openretractions.com}{Open Retractions}.

\end{infobox}

\cftaddtitleline{toc}{chapter}{\rule{\textwidth}{0.4pt}}{}

\chapter*{\texorpdfstring{\emph{PARTE 3: DO MUNDO REAL À TABELA}}{PARTE 3: DO MUNDO REAL À TABELA}}\label{parte-3}
\addcontentsline{toc}{chapter}{\emph{PARTE 3: DO MUNDO REAL À TABELA}}

\par\noindent\rule{\textwidth}{0.05in}

\section*{Da coleta à organização: estruturando dados para análises}\label{da-coleta-uxe0-organizauxe7uxe3o-estruturando-dados-para-anuxe1lises}

\markboth{}{}

\chapter{\texorpdfstring{\textbf{Variáveis e fatores}}{Variáveis e fatores}}\label{variaveis-fatores}

\section{Variáveis}\label{variuxe1veis}

\subsection{O que são variáveis?}\label{o-que-suxe3o-variuxe1veis}

\begin{itemize}
\item
  Variáveis são informações que podem variar entre medidas em diferentes indivíduos e/ou repetições.\textsuperscript{\citeproc{ref-Altman1999}{107}}
\item
  Variáveis definem características de uma amostra extraída da população, tipicamente observados por aplicação de métodos de amostragem (isto é, seleção) da população de interesse.\textsuperscript{\citeproc{ref-vetter2017}{108}}
\end{itemize}

\subsection{Como são classificadas as variáveis?}\label{como-suxe3o-classificadas-as-variuxe1veis}

\begin{itemize}
\item
  Quanto à informação:\textsuperscript{\citeproc{ref-vetter2017}{108}--\citeproc{ref-kaliyadan2019}{111}}

  \begin{itemize}
  \item
    Quantitativa
  \item
    Qualitativa
  \end{itemize}
\item
  Quanto ao conteúdo:\textsuperscript{\citeproc{ref-vetter2017}{108}--\citeproc{ref-barkan2015}{112}}

  \begin{itemize}
  \item
    Contínua: representam ordem e magnitude entre valores.

    \begin{itemize}
    \item
      Discreta (valores inteiros ou contáveis) vs.~Contínua (valores em escala contínua).
    \item
      Intervalo (valor `0' é arbitrário) vs.~Razão (valor `0' verdadeiro).
    \end{itemize}
  \item
    Categórica ordinal (numérica ou nominal): representam ordem, mas não magnitude entre valores.
  \item
    Categórica nominal (multinominal ou dicotômica): não representam ordem ou magnitude, apenas categorias.
  \end{itemize}
\item
  Quanto à interpretação:\textsuperscript{\citeproc{ref-vetter2017}{108}--\citeproc{ref-kaliyadan2019}{111}}

  \begin{itemize}
  \item
    Dependente (desfecho)
  \item
    Independente (preditora, covariável, confundidora, controle)
  \item
    Mediadora
  \item
    Moderadora
  \item
    Modificadora
  \item
    Auxiliar
  \item
    Indicadora
  \end{itemize}
\end{itemize}

\begin{infobox}{images/Rlogo}
O pacote \emph{base}\textsuperscript{\citeproc{ref-base}{55}} fornece a função \href{https://www.rdocumentation.org/packages/base/versions/3.6.2/topics/class}{\emph{class}} para identificar qual é o tipo do objeto.

\end{infobox}

\begin{infobox}{images/Rlogo}
O pacote \emph{base}\textsuperscript{\citeproc{ref-base}{55}} fornece as funções \href{https://www.rdocumentation.org/packages/base/versions/3.6.2/topics/numeric}{\emph{as.numeric}} e \href{https://www.rdocumentation.org/packages/base/versions/3.6.2/topics/character}{\emph{as.character}} para criar objetos numéricos e categóricos, respectivamente.

\end{infobox}

\begin{infobox}{images/Rlogo}
O pacote \emph{base}\textsuperscript{\citeproc{ref-base}{55}} fornece as funções \href{https://www.rdocumentation.org/packages/base/versions/3.6.2/topics/as.Date}{\emph{as.Date}} e \href{https://www.rdocumentation.org/packages/base/versions/3.6.2/topics/logical}{\emph{as.logical}} para criar objetos em formato de data e lógicos (VERDADEIRO, FALSO), respectivamente.

\end{infobox}

\section{Transformação de variáveis}\label{transformauxe7uxe3o-de-variuxe1veis}

\subsection{Por que é importante classificar as variáveis?}\label{por-que-uxe9-importante-classificar-as-variuxe1veis}

\begin{itemize}
\tightlist
\item
  Identificar corretamente os tipos de variáveis da pesquisa é uma das etapas da escolha dos métodos estatísticos adequados para as análises e representações no texto, tabelas e gráficos.\textsuperscript{\citeproc{ref-Dettori2018}{110}}
\end{itemize}

\subsection{O que é transformação de variáveis?}\label{o-que-uxe9-transformauxe7uxe3o-de-variuxe1veis}

\begin{itemize}
\item
  Transformação significa aplicar uma função matemática à variável medida em sua unidade original.\textsuperscript{\citeproc{ref-Bland1996}{113}}
\item
  A transformação visa atender aos pressupostos dos modelos estatísticos quanto à distribuição da variável, em geral a distribuição gaussiana.\textsuperscript{\citeproc{ref-vetter2017}{108},\citeproc{ref-Bland1996}{113}}
\item
  A dicotomização pode ser interpretada como um caso particular de agrupamento.\textsuperscript{\citeproc{ref-Fedorov2009}{114}}
\end{itemize}

\subsection{Por que transformar variáveis?}\label{por-que-transformar-variuxe1veis}

\begin{itemize}
\item
  Muitos procedimentos estatísticos supõem que as variáveis --- ou seus termos de erro, mais especificamente --- são normalmente distribuídas. A violação dessa suposição pode aumentar suas chances de cometer um erro do tipo I ou II.\textsuperscript{\citeproc{ref-osborne2010}{115}}
\item
  Mesmo quando se está usando análises consideradas robustas para violações dessas suposições ou testes não paramétricos (que não assumem explicitamente termos de erro normalmente distribuídos), atender a essas questões pode melhorar os resultados das análises (por exemplo, Zimmerman, 1995).\textsuperscript{\citeproc{ref-osborne2010}{115}}
\end{itemize}

\subsection{Quais transformações de variáveis podem ser aplicadas?}\label{quais-transformauxe7uxf5es-de-variuxe1veis-podem-ser-aplicadas}

\begin{itemize}
\tightlist
\item
  Distribuições com assimetria à direita: raiz quadrada, logaritmo natural, logaritmo base 10, transformação inversa.\textsuperscript{\citeproc{ref-osborne2010}{115}}
\end{itemize}

\begin{figure}

{\centering \includegraphics[width=1\linewidth]{Ciencia-com-R_files/figure-latex/assimetria-direita-1} 

}

\caption{Transformações de variáveis com assimetria à direita (Original, Raiz quadrada, Log natural, Log10, Inversa).}\label{fig:assimetria-direita}
\end{figure}

\begin{itemize}
\tightlist
\item
  Distribuições com assimetria à esquerda: reflexão e raiz quadrada, reflexão e logaritmo natural, reflexão e logaritmo base 10, reflexão e transformação inversa.\textsuperscript{\citeproc{ref-osborne2010}{115}}
\end{itemize}

\begin{figure}

{\centering \includegraphics[width=1\linewidth]{Ciencia-com-R_files/figure-latex/assimetria-esquerda-1} 

}

\caption{Transformações de variáveis com assimetria à esquerda (Original, Reflexão + Raiz quadrada, Reflexão + Log natural, Reflexão + Log10, Reflexão + Inversa).}\label{fig:assimetria-esquerda}
\end{figure}

\begin{itemize}
\tightlist
\item
  Transformação \(z\) de Fisher \eqref{eq:fisher-z}.{[}REF{]}
\end{itemize}

\begin{equation}
\label{eq:fisher-z}
Z = \frac{1}{2} \ln\left(\frac{1 + r}{1 - r}\right)
\end{equation}

\begin{itemize}
\tightlist
\item
  Transformação de Box-Cox \eqref{eq:box-cox}.\textsuperscript{\citeproc{ref-box1964}{116}}
\end{itemize}

\begin{equation}
\label{eq:box-cox}
Y(\lambda) =
\begin{cases}
\frac{Y^{\lambda} - 1}{\lambda}, & \text{se } \lambda \neq 0 \\
\ln(Y), & \text{se } \lambda = 0
\end{cases}
\end{equation}

\begin{itemize}
\tightlist
\item
  Transformação arco-seno \eqref{eq:arco-seno}.\textsuperscript{\citeproc{ref-osborne2010}{115}}
\end{itemize}

\begin{equation}
\label{eq:arco-seno}
Y' = \arcsin(\sqrt{Y})
\end{equation}

\begin{itemize}
\item
  Diferenciação.
\item
  Categorização.
\item
  Dicotomização.
\end{itemize}

\begin{infobox}{images/Rlogo}
O pacote \emph{MASS}\textsuperscript{\citeproc{ref-MASS}{117}} fornece a função \href{https://www.rdocumentation.org/packages/MASS/versions/7.3-58.3/topics/boxcox}{\emph{boxcox}} para executar a transformação de Box-Cox.\textsuperscript{\citeproc{ref-box1964}{116}}

\end{infobox}

\section{\texorpdfstring{Centralização de variáveis (\emph{centering})}{Centralização de variáveis (centering)}}\label{centralizauxe7uxe3o-de-variuxe1veis-centering}

\subsection{O que é centralização?}\label{o-que-uxe9-centralizauxe7uxe3o}

\begin{itemize}
\tightlist
\item
  É uma transformação linear em que se subtrai a média da variável de cada observação. O objetivo é recentrar a variável em torno de zero, sem alterar a sua variabilidade.\textsuperscript{\citeproc{ref-REF}{\textbf{REF?}}}
\end{itemize}

\subsection{Por que centralizar?}\label{por-que-centralizar}

\begin{itemize}
\item
  Facilita a interpretação dos coeficientes de regressão, especialmente em modelos com termos de interação.\textsuperscript{\citeproc{ref-REF}{\textbf{REF?}}}
\item
  Reduz a multicolinearidade entre variáveis e seus termos de interação ou polinomiais.\textsuperscript{\citeproc{ref-REF}{\textbf{REF?}}}
\item
  Mantém a escala original (apenas desloca a média).\textsuperscript{\citeproc{ref-REF}{\textbf{REF?}}}
\end{itemize}

\section{Padronização de variáveis}\label{padronizauxe7uxe3o-de-variuxe1veis}

\subsection{O que é padronização?}\label{o-que-uxe9-padronizauxe7uxe3o}

\begin{itemize}
\tightlist
\item
  Padronização é a transformação de uma variável contínua para uma escala comum, permitindo comparações entre variáveis medidas em diferentes unidades ou magnitudes.\textsuperscript{\citeproc{ref-REF}{\textbf{REF?}}}
\end{itemize}

\subsection{Por que padronizar?}\label{por-que-padronizar}

\begin{itemize}
\item
  Facilita a interpretação em análises multivariadas.\textsuperscript{\citeproc{ref-REF}{\textbf{REF?}}}
\item
  Evita que variáveis em escalas maiores dominem os resultados de algoritmos que dependem de distância.\textsuperscript{\citeproc{ref-REF}{\textbf{REF?}}}
\item
  Melhora a comparabilidade entre estudos e bases de dados diferentes.\textsuperscript{\citeproc{ref-REF}{\textbf{REF?}}}
\end{itemize}

\subsection{Quais são os métodos de padronização mais comuns?}\label{quais-suxe3o-os-muxe9todos-de-padronizauxe7uxe3o-mais-comuns}

\begin{itemize}
\tightlist
\item
  Escore-Z (Z-score) \eqref{eq:zscore}: subtrair a média e dividir pelo desvio-padrão.\textsuperscript{\citeproc{ref-REF}{\textbf{REF?}}}
\end{itemize}

\begin{equation}
\label{eq:zscore}
Z = \frac{X - \mu}{\sigma}
\end{equation}

\begin{itemize}
\tightlist
\item
  Escala Min-Max \eqref{eq:minmax}: transformar para o intervalo {[}0,1{]}.\textsuperscript{\citeproc{ref-REF}{\textbf{REF?}}}
\end{itemize}

\begin{equation}
\label{eq:minmax}
X_{norm} = \frac{X - X_{min}}{X_{max} - X_{min}}
\end{equation}

\begin{figure}

{\centering \includegraphics[width=1\linewidth]{Ciencia-com-R_files/figure-latex/padronizacao-1} 

}

\caption{Comparação entre variáveis originais e padronizadas (Z-score e Min-Max).}\label{fig:padronizacao}
\end{figure}

\subsection{Quais são as boas práticas de nomenclatura ao padronizar variáveis?}\label{quais-suxe3o-as-boas-pruxe1ticas-de-nomenclatura-ao-padronizar-variuxe1veis}

\begin{itemize}
\item
  Usar sufixos como \texttt{\_z} ou \texttt{\_std} para indicar padronização (\texttt{altura\_z}, \texttt{peso\_std}).\textsuperscript{\citeproc{ref-REF}{\textbf{REF?}}}
\item
  Documentar no dicionário de dados como cada variável foi transformada.\textsuperscript{\citeproc{ref-REF}{\textbf{REF?}}}
\item
  Evitar substituir a variável original: manter sempre a versão bruta e a padronizada.\textsuperscript{\citeproc{ref-REF}{\textbf{REF?}}}
\end{itemize}

\begin{infobox}{images/Rlogo}
O pacote \emph{base}\textsuperscript{\citeproc{ref-base}{55}} fornece a função \href{https://www.rdocumentation.org/packages/base/versions/3.6.2/topics/scale}{\emph{scale}} para calcular automaticamente a padronização (média = 0, desvio padrão = 1).

\end{infobox}

\section{Categorização de variáveis contínuas}\label{categorizauxe7uxe3o-de-variuxe1veis-contuxednuas}

\subsection{O que é categorização de uma variável?}\label{o-que-uxe9-categorizauxe7uxe3o-de-uma-variuxe1vel}

\begin{itemize}
\tightlist
\item
  .\textsuperscript{\citeproc{ref-REF}{\textbf{REF?}}}
\end{itemize}

\subsection{Por que não é recomendado categorizar variáveis contínuas?}\label{por-que-nuxe3o-uxe9-recomendado-categorizar-variuxe1veis-contuxednuas}

\begin{itemize}
\item
  Nenhum dos argumentos usados para defender a categorização de variáveis se sustenta sob uma análise técnica rigorosa.\textsuperscript{\citeproc{ref-MacCallum2002}{118}}
\item
  Categorizar variáveis não é necessário para conduzir análises estatísticas. Ao invés de categorizar, priorize as variáveis contínuas.\textsuperscript{\citeproc{ref-Altman2006}{119}--\citeproc{ref-Collins2016}{121}}
\item
  Em geral, não existe uma justificativa racional (plausibilidade biológica) para assumir que as categorias artificiais subjacentes existam.\textsuperscript{\citeproc{ref-Altman2006}{119}--\citeproc{ref-Collins2016}{121}}
\item
  Caso exista um ponto de corte ou limiar verdadeiro que discrimine três ou mais grupos independentes, identificar tal ponto de corte ainda é um desafio.\textsuperscript{\citeproc{ref-Prince2017}{122}}
\item
  Categorização de variáveis contínuas aumenta a quantidade de testes de hipótese para comparações pareadas entre os quantis, inflando, portanto, o erro tipo I.\textsuperscript{\citeproc{ref-Bennette2012}{123}}
\item
  Categorização de variáveis contínuas requer uma função teórica que pressupõe a homogeneidade da variável dentro dos grupos, levando tanto a uma perda de poder como a uma estimativa imprecisa.\textsuperscript{\citeproc{ref-Bennette2012}{123}}
\item
  Categorização de variáveis contínuas pode dificultar a comparação de resultados entre estudos devido aos pontos de corte baseados em dados de um banco usados para definir as categorias.\textsuperscript{\citeproc{ref-Bennette2012}{123}}
\end{itemize}

\begin{infobox}{images/Rlogo}
O pacote \emph{questionr}\textsuperscript{\citeproc{ref-questionr}{124}} fornece a função \href{https://www.rdocumentation.org/packages/questionr/versions/0.7.8/topics/irec}{\emph{irec}} para executar uma interface interativa para codificação de variáveis categóricas.

\end{infobox}

\subsection{Quais são as alternativas à categorização de variáveis contínuas?}\label{quais-suxe3o-as-alternativas-uxe0-categorizauxe7uxe3o-de-variuxe1veis-contuxednuas}

\begin{itemize}
\item
  Análise com os dados das variáveis na escala de medida original.\textsuperscript{\citeproc{ref-MacCallum2002}{118}}
\item
  Análise com modelos de regressão com pesos locais (\emph{lowess}) tais como \emph{splines} e polinômios fracionais.\textsuperscript{\citeproc{ref-MacCallum2002}{118}}
\end{itemize}

\section{Dicotomização de variáveis contínuas}\label{dicotomizauxe7uxe3o-de-variuxe1veis-contuxednuas}

\subsection{O que são variáveis dicotômicas?}\label{o-que-suxe3o-variuxe1veis-dicotuxf4micas}

\begin{itemize}
\item
  Variáveis dicotômicas (ou binárias) podem representar categorias naturais tipo ``presente/ausente'', ``sim/não''.\textsuperscript{\citeproc{ref-REF}{\textbf{REF?}}}
\item
  Variáveis dicotômicas podem representar categorias fictícias, criadas a partir de variáveis multinominais, em que cada nível é convertido em uma variável dicotômica indicatoda (\emph{dummy}).\textsuperscript{\citeproc{ref-REF}{\textbf{REF?}}}
\item
  Dicotomização é considerado um artefato da análise de dados, uma vez que é realizada após a coleta de dados.\textsuperscript{\citeproc{ref-aguinis2008}{125}}
\item
  Geralmente são representadas por ``1'' (presente, sucesso) e ``0'' (ausente, falha).\textsuperscript{\citeproc{ref-REF}{\textbf{REF?}}}
\end{itemize}

\subsection{Quais argumentos são usados para defender a categorização ou dicotomização de variáveis contínuas?}\label{quais-argumentos-suxe3o-usados-para-defender-a-categorizauxe7uxe3o-ou-dicotomizauxe7uxe3o-de-variuxe1veis-contuxednuas}

\begin{itemize}
\item
  O argumento principal para dicotomização de variáveis é que tal procedimento facilita e simplifica a apresentação dos resultados, principalmente para o público em geral.\textsuperscript{\citeproc{ref-Fedorov2009}{114}}
\item
  Os pesquisadores não conhecem as consequências estatísticas da dicotomização.\textsuperscript{\citeproc{ref-MacCallum2002}{118}}
\item
  Os pesquisadores não conhecem os métodos adequados de análise não-paramétrica, não-linear e robusta.\textsuperscript{\citeproc{ref-MacCallum2002}{118}}
\item
  As categorias representam características existentes dos participantes da pesquisa, de modo que as análises devam ser feitas por grupos e não por indivíduos.\textsuperscript{\citeproc{ref-MacCallum2002}{118}}
\item
  A confiabilidade da(s) variável(eis) medida(s) é baixa e, portanto, categorizar os participantes resultaria em uma medida mais confiável.\textsuperscript{\citeproc{ref-MacCallum2002}{118}}
\end{itemize}

\subsection{Por que não é recomendado dicotomizar variáveis contínuas?}\label{por-que-nuxe3o-uxe9-recomendado-dicotomizar-variuxe1veis-contuxednuas}

\begin{itemize}
\item
  Nenhum dos argumentos usados para defender a dicotomização de variáveis se sustenta sob uma análise técnica rigorosa.\textsuperscript{\citeproc{ref-MacCallum2002}{118}}
\item
  Dicotomizar variáveis não é necessário para conduzir análises estatísticas. Ao invés de dicotomizar, priorize as variáveis contínuas.\textsuperscript{\citeproc{ref-Altman2006}{119}--\citeproc{ref-Collins2016}{121}}
\item
  Em geral, não existe uma justificativa racional (plausibilidade biológica) para assumir que as categorias artificiais subjacentes existam.\textsuperscript{\citeproc{ref-Altman2006}{119}--\citeproc{ref-Collins2016}{121}}
\item
  Dicotomização causa perda de informação e consequentemente perda de poder estatístico para detectar efeitos.\textsuperscript{\citeproc{ref-MacCallum2002}{118},\citeproc{ref-Altman2006}{119}}
\item
  Dicotomização também classifica indivíduos com valores próximos na variável contínua como indivíduos em pontos opostos e extremos, artificialmente sugerindo que são muito diferentes.\textsuperscript{\citeproc{ref-Altman2006}{119}}
\item
  Dicotomização pode diminuir a variabilidade das variáveis.\textsuperscript{\citeproc{ref-Altman2006}{119}}
\item
  Dicotomização pode ocultar não-linearidades presentes na variável contínua.\textsuperscript{\citeproc{ref-MacCallum2002}{118},\citeproc{ref-Altman2006}{119}}
\item
  A média ou a mediana, embora amplamente utilizadas, não são bons parâmetros para dicotomizar variáveis.\textsuperscript{\citeproc{ref-Fedorov2009}{114},\citeproc{ref-Altman2006}{119}}
\item
  Caso exista um ponto de corte ou limiar verdadeiro que discrimine dois grupos independentes, identificar tal ponto de corte ainda é um desafio.\textsuperscript{\citeproc{ref-Prince2017}{122}}
\end{itemize}

\subsection{Quais cenários legitimam a dicotomização das variáveis contínuas?}\label{quais-cenuxe1rios-legitimam-a-dicotomizauxe7uxe3o-das-variuxe1veis-contuxednuas}

\begin{itemize}
\item
  Quando existem dados e/ou análises que suportem a existência --- não apenas a suposição ou teorização --- de categorias com um ponto de corte claro e com significado entre elas.\textsuperscript{\citeproc{ref-MacCallum2002}{118}}
\item
  Quando a distribuição da variável contínua é muito assimétrica, de modo que uma grande quantidade de observações está em um dos extremos da escala.\textsuperscript{\citeproc{ref-MacCallum2002}{118}}
\end{itemize}

\subsection{Quais métodos são usados para dicotomizar variáveis contínuas?}\label{quais-muxe9todos-suxe3o-usados-para-dicotomizar-variuxe1veis-contuxednuas}

\begin{itemize}
\item
  Em termos de tabelas de contingência 2x2, os seguintes métodos permitem\textsuperscript{\citeproc{ref-Prince2017}{122}} a identificação do limiar verdadeiro:

  \begin{itemize}
  \item
    Youden.\textsuperscript{\citeproc{ref-YOUDEN1950}{126}}
  \item
    Gini Index.\textsuperscript{\citeproc{ref-strobl2007}{127}}
  \item
    Estatística qui-quadrado (\(\chi^2\)).\textsuperscript{\citeproc{ref-pearson1900}{128}}
  \item
    Risco relativo (\(RR\)).\textsuperscript{\citeproc{ref-Greiner2000}{129}}
  \item
    Kappa (\(\kappa\)).\textsuperscript{\citeproc{ref-fleiss1971}{130}}.
  \end{itemize}
\end{itemize}

\section{Representação de variáveis categóricas}\label{representauxe7uxe3o-de-variuxe1veis-categuxf3ricas}

\subsection{O que são variáveis indicadoras (dummy variables)?}\label{o-que-suxe3o-variuxe1veis-indicadoras-dummy-variables}

\begin{itemize}
\item
  Variáveis indicadoras são variáveis dicotômicas criadas a partir dos níveis de um fator.\textsuperscript{\citeproc{ref-REF}{\textbf{REF?}}}
\item
  Cada variável indicadora assume o valor \(1\) quando a observação pertence àquela categoria e \(0\) caso contrário.\textsuperscript{\citeproc{ref-REF}{\textbf{REF?}}}
\item
  Variáveis indicadoras não representam magnitude ou ordem, apenas a presença ou ausência de uma categoria.\textsuperscript{\citeproc{ref-REF}{\textbf{REF?}}}
\end{itemize}

\begin{table}
\centering
\caption{\label{tab:variaveis-indicadoras}Tabela de variáveis indicadoras (dummy variables) criadas a partir de variáveis categóricas Sexo e Grupo.}
\centering
\begin{tabu} to \linewidth {>{\centering}X>{\centering}X>{\centering}X>{\centering}X>{\centering}X>{\centering}X}
\toprule
ID & Sexo & Grupo & Sexo\_Feminino & Grupo\_TratA & Grupo\_TratB\\
\midrule
1 & Masculino & Tratamento A & 1 & 1 & 0\\
2 & Masculino & Controle & 1 & 0 & 0\\
3 & Masculino & Tratamento A & 1 & 1 & 0\\
4 & Feminino & Tratamento B & 0 & 0 & 1\\
5 & Masculino & Controle & 1 & 0 & 0\\
6 & Feminino & Tratamento B & 0 & 0 & 1\\
7 & Feminino & Tratamento B & 0 & 0 & 1\\
8 & Feminino & Controle & 0 & 0 & 0\\
9 & Masculino & Controle & 1 & 0 & 0\\
10 & Masculino & Controle & 1 & 0 & 0\\
11 & Feminino & Controle & 0 & 0 & 0\\
12 & Feminino & Tratamento B & 0 & 0 & 1\\
\bottomrule
\end{tabu}
\end{table}

\begin{infobox}{images/Rlogo}
O pacote \emph{stats}\textsuperscript{\citeproc{ref-stats}{131}} fornece a função \href{https://www.rdocumentation.org/packages/stats/versions/3.5.2/topics/model.matrix}{\emph{model.matrix}} para expandir variáveis categóricas em variáveis indicadoras.

\end{infobox}

\subsection{Por que variáveis indicadoras são importantes?}\label{por-que-variuxe1veis-indicadoras-suxe3o-importantes}

\begin{itemize}
\item
  Permitem a inclusão de fatores em modelos estatísticos.\textsuperscript{\citeproc{ref-REF}{\textbf{REF?}}}
\item
  Tornam explícitas as comparações entre categorias.\textsuperscript{\citeproc{ref-REF}{\textbf{REF?}}}
\item
  Garantem coerência matemática sem perder o significado conceitual das categorias.\textsuperscript{\citeproc{ref-REF}{\textbf{REF?}}}
\end{itemize}

\subsection{Quantas variáveis indicadoras são necessárias para um fator?}\label{quantas-variuxe1veis-indicadoras-suxe3o-necessuxe1rias-para-um-fator}

\begin{itemize}
\item
  Um fator com k níveis é representado por \(k − 1\) variáveis indicadoras.\textsuperscript{\citeproc{ref-REF}{\textbf{REF?}}}
\item
  O nível que não gera uma variável indicadora explícita é chamado de nível de referência.\textsuperscript{\citeproc{ref-REF}{\textbf{REF?}}}
\end{itemize}

\subsection{O que é o nível de referência?}\label{o-que-uxe9-o-nuxedvel-de-referuxeancia}

\begin{itemize}
\item
  O nível de referência é a categoria usada como base de comparação para as demais.\textsuperscript{\citeproc{ref-REF}{\textbf{REF?}}}
\item
  Os coeficientes associados às variáveis indicadoras representam diferenças em relação a esse nível de referência.\textsuperscript{\citeproc{ref-REF}{\textbf{REF?}}}
\end{itemize}

\subsection{\texorpdfstring{Por que não se usam k variáveis indicadoras para \(k\) níveis?}{Por que não se usam k variáveis indicadoras para k níveis?}}\label{por-que-nuxe3o-se-usam-k-variuxe1veis-indicadoras-para-k-nuxedveis}

\begin{itemize}
\item
  Utilizar \(k\) variáveis indicadoras gera redundância perfeita entre as variáveis.\textsuperscript{\citeproc{ref-REF}{\textbf{REF?}}}
\item
  Essa redundância causa problemas de identificabilidade nos modelos, fenômeno conhecido como \emph{dummy trap}.\textsuperscript{\citeproc{ref-REF}{\textbf{REF?}}}
\end{itemize}

\subsection{Variáveis indicadoras são uma forma de dicotomização?}\label{variuxe1veis-indicadoras-suxe3o-uma-forma-de-dicotomizauxe7uxe3o}

\begin{itemize}
\item
  Variáveis indicadoras são dicotômicas, mas não resultam da dicotomização de variáveis contínuas.\textsuperscript{\citeproc{ref-REF}{\textbf{REF?}}}
\item
  Variáveis indicadoras são criadas a partir de variáveis categóricas multinominais, preservando toda a informação original do fator.\textsuperscript{\citeproc{ref-REF}{\textbf{REF?}}}
\item
  Variáveis indicadoras não reduzem informação, enquanto a dicotomização de variáveis contínuas descarta informação por construção.\textsuperscript{\citeproc{ref-REF}{\textbf{REF?}}}
\end{itemize}

\subsection{Variáveis indicadoras alteram os dados originais?}\label{variuxe1veis-indicadoras-alteram-os-dados-originais}

\begin{itemize}
\item
  Não. Variáveis indicadoras apenas representam os níveis do fator de forma numérica.\textsuperscript{\citeproc{ref-REF}{\textbf{REF?}}}
\item
  A variável categórica original permanece intacta no conjunto de dados.\textsuperscript{\citeproc{ref-REF}{\textbf{REF?}}}
\end{itemize}

\section{Fatores}\label{fatores}

\subsection{O que são fatores?}\label{o-que-suxe3o-fatores}

\begin{itemize}
\item
  Fator é um sinônimo de variável categórica.\textsuperscript{\citeproc{ref-REF}{\textbf{REF?}}}
\item
  Na modelagem, fator é sinônimo de variável preditora, em particular quando se refere à modelagem de efeitos fixos e aleatórios -- os fatores (variáveis) são fatores fixos ou fatores aleatórios.\textsuperscript{\citeproc{ref-REF}{\textbf{REF?}}}
\item
  Fatores são variáveis controladas pelos pesquisadores em um experimento para determinar seu efeito na(s) variável(ies) de resposta. Um fator pode assumir apenas um pequeno número de valores, conhecidos como níveis. Os fatores podem ser uma variável categórica ou baseados em uma variável contínua, mas usam apenas um número limitado de valores escolhidos pelos experimentadores.\textsuperscript{\citeproc{ref-REF}{\textbf{REF?}}}
\end{itemize}

\begin{infobox}{images/Rlogo}
O pacote \emph{base}\textsuperscript{\citeproc{ref-base}{55}} fornece a função \href{https://www.rdocumentation.org/packages/base/versions/3.6.2/topics/factor}{\emph{as.factor}} para converter uma variável em fator.

\end{infobox}

\subsection{O que são níveis de um fator?}\label{o-que-suxe3o-nuxedveis-de-um-fator}

\begin{itemize}
\tightlist
\item
  Níveis de um fator são as possíveis categorias que descrevem um fator.\textsuperscript{\citeproc{ref-REF}{\textbf{REF?}}}
\end{itemize}

\begin{infobox}{images/Rlogo}
O pacote \emph{base}\textsuperscript{\citeproc{ref-base}{55}} fornece as funções \href{https://www.rdocumentation.org/packages/base/versions/3.6.2/topics/levels}{\emph{levels}} e \href{https://www.rdocumentation.org/packages/base/versions/3.6.2/topics/nlevels}{\emph{nlevels}} para listar os níveis e a quantidade deles em um fator.

\end{infobox}

\chapter{\texorpdfstring{\textbf{Dados e metadados}}{Dados e metadados}}\label{dados-metadados}

\section{Dados}\label{dados}

\subsection{O que são dados?}\label{o-que-suxe3o-dados}

\begin{itemize}
\item
  ``Tudo são dados''.\textsuperscript{\citeproc{ref-Olson2021}{132}}
\item
  Dados coletados em um estudo geralmente contêm erros de mensuração e/ou classificação, dados perdidos e são agrupados por alguma unidade de análise.\textsuperscript{\citeproc{ref-van2022a}{133}}
\end{itemize}

\subsection{O que são dados estruturados?}\label{o-que-suxe3o-dados-estruturados}

\begin{itemize}
\tightlist
\item
  Dados estruturados são dados organizados em um formato tabular, como planilhas eletrônicas ou bancos de dados relacionais, onde cada linha representa uma observação e cada coluna representa uma variável ou atributo.\textsuperscript{\citeproc{ref-REF}{\textbf{REF?}}}
\end{itemize}

\subsection{O que são dados não estruturados?}\label{o-que-suxe3o-dados-nuxe3o-estruturados}

\begin{itemize}
\tightlist
\item
  Dados não estruturados são dados que não possuem um formato ou organização predefinidos, como textos, imagens, vídeos, áudios e sinais biomédicos, tornando sua análise mais complexa em comparação com dados estruturados.\textsuperscript{\citeproc{ref-REF}{\textbf{REF?}}}
\end{itemize}

\section{Grandes dados}\label{grandes-dados}

\subsection{O que são grandes dados?}\label{o-que-suxe3o-grandes-dados}

\begin{itemize}
\tightlist
\item
  Grandes dados (\emph{big data}) refere-se a bancos de dados muito grandes com um mecanismo ``R'' --- aleatório (\emph{Random}), auto-reportado (\emph{self-Reported}), reportado administrativamente (\emph{administratively Reported}), seletivamente respondido (\emph{selectively Responded}) --- descontrolado ou desconhecido.\textsuperscript{\citeproc{ref-meng2018}{72}}
\end{itemize}

\subsection{Quais são as fontes de dados?}\label{quais-suxe3o-as-fontes-de-dados}

\begin{itemize}
\item
  Experimentos.\textsuperscript{\citeproc{ref-REF}{\textbf{REF?}}}
\item
  Mundo real.\textsuperscript{\citeproc{ref-REF}{\textbf{REF?}}}
\item
  Simulação.\textsuperscript{\citeproc{ref-REF}{\textbf{REF?}}}
\end{itemize}

\subsection{O que são dados primários e secundários?}\label{o-que-suxe3o-dados-primuxe1rios-e-secunduxe1rios}

\begin{itemize}
\item
  Dados primários são dados originais coletados intencionalmente para uma determinada análise exploratória ou inferencial planejada a priori.\textsuperscript{\citeproc{ref-vetter2017}{108}}
\item
  Dados secundários compreendem dados coletados inicialmente para análises de um estudo, e são subsequentemente utilizados para outras análises.\textsuperscript{\citeproc{ref-vetter2017}{108}}
\end{itemize}

\subsection{O que são dados quantitativos e qualitativos?}\label{o-que-suxe3o-dados-quantitativos-e-qualitativos}

\begin{itemize}
\tightlist
\item
  .\textsuperscript{\citeproc{ref-REF}{\textbf{REF?}}}
\end{itemize}

\section{Metadados}\label{metadados}

\subsection{O que são metadados?}\label{o-que-suxe3o-metadados}

\begin{itemize}
\item
  Metadados são informações técnicas relacionadas às variáveis do estudo, tais como rótulos, limites de valores plausíveis, códigos para dados perdidos e unidades de medida.\textsuperscript{\citeproc{ref-Baillie2022}{134}}
\item
  Metadados também são informações relacionadas ao delineamento e/ou protocolo do estudo, recrutamento dos participantes, e métodos para realização das medidas.\textsuperscript{\citeproc{ref-Baillie2022}{134}}
\end{itemize}

\subsection{Quais são as recomendações para os metadados de um banco de dados?}\label{quais-suxe3o-as-recomendauxe7uxf5es-para-os-metadados-de-um-banco-de-dados}

\begin{itemize}
\item
  Utilize rótulos padronizados para variáveis e fatores para facilitar o reuso (reprodutibilidade) do conjuntos de dados e scripts de análise.\textsuperscript{\citeproc{ref-buttliere2021}{135}}
\item
  Crie rótulos de variáveis concisos, claros e mutuamente exclusivos.\textsuperscript{\citeproc{ref-buttliere2021}{135}}
\item
  Evite muitas letras maiúsculas ou outros caracteres especiais que usam a \emph{shift}.\textsuperscript{\citeproc{ref-buttliere2021}{135}}
\item
  Na existência de versões de instrumentos publicadas em diferentes anos, use o ano de publicação das escalas no rótulo.\textsuperscript{\citeproc{ref-buttliere2021}{135}}
\item
  Divida o rótulo da variável ou fator em partes e ordene-as do mais geral para o mais particular geral (ex.: experimento -\textgreater{} repetição -\textgreater{} escala -\textgreater{} item).\textsuperscript{\citeproc{ref-buttliere2021}{135}}
\end{itemize}

\begin{infobox}{images/Rlogo}
O pacote \emph{base}\textsuperscript{\citeproc{ref-base}{55}} fornece a função \href{https://www.rdocumentation.org/packages/base/versions/3.6.2/topics/names}{\emph{names}} para declarar o nome de uma variável.

\end{infobox}

\begin{infobox}{images/Rlogo}
O pacote \emph{base}\textsuperscript{\citeproc{ref-base}{55}} fornece a função \href{https://www.rdocumentation.org/packages/base/versions/3.6.2/topics/labels}{\emph{labels}} para declarar o rótulo de uma variável.

\end{infobox}

\begin{infobox}{images/Rlogo}
O pacote \emph{units}\textsuperscript{\citeproc{ref-units}{136}} fornece a função \href{https://www.rdocumentation.org/packages/units/versions/0.8-3/topics/units}{\emph{units}} para declarar as unidades de medida de uma variável.

\end{infobox}

\begin{infobox}{images/Rlogo}
O pacote \emph{units}\textsuperscript{\citeproc{ref-units}{136}} fornece a função \href{https://www.rdocumentation.org/packages/units/versions/0.8-3/topics/valid_udunits}{\emph{valid\_udunits}} para listar as opções de unidades de medida de uma variável.

\end{infobox}

\begin{infobox}{images/Rlogo}
O pacote \emph{janitor}\textsuperscript{\citeproc{ref-janitor}{137}} fornece a função \href{https://www.rdocumentation.org/packages/janitor/versions/2.2.0/topics/clean_names}{\emph{clean\_names}} para formatar de modo padronizado o nome das variáveis utilizando apenas caracteres, números e o símbolo `\_'.

\end{infobox}

\begin{infobox}{images/Rlogo}
O pacote \emph{Hmisc}\textsuperscript{\citeproc{ref-Hmisc}{138}} fornece a função \href{https://www.rdocumentation.org/packages/Hmisc/versions/5.1-0/topics/contents}{\emph{contents}} para criar um objeto com os metadados (nomes, rótulos, unidades, quantidade e níveis das variáveis categóricas, e quantidade de dados perdidos) de um dataframe.

\end{infobox}

\chapter{\texorpdfstring{\textbf{Medidas e instrumentos}}{Medidas e instrumentos}}\label{medidas-instrumentos}

\section{Escalas}\label{escalas}

\subsection{O que são escalas?}\label{o-que-suxe3o-escalas}

\begin{itemize}
\item
  Uma escala de medição grosseira representa um construto de natureza contínua medido por itens tais que diferentes pontuações são agrupadas na mesma categoria no ato da coleta de dados.\textsuperscript{\citeproc{ref-aguinis2008}{125}}
\item
  Em escalas grosseiras, erros são introduzidos porque as variações contínunas do constructo são colapsadas em uma mesma categorias ou separadas entre categorias próximas.\textsuperscript{\citeproc{ref-aguinis2008}{125}}
\item
  Escalas tipo Likert com 5 categorias tipo ``discordo totalmente'', ``discordo parcialmente'', ``nem concordo nem discordo'', ``concordo parcialmente'', e ``concordo totalmente'' são escalas grosseira porque as diferenças entre as categorias não são iguais. Por exemplo, a diferença entre ``discordo totalmente'' e ``discordo parcialmente'' não é a mesma que a diferença entre ``concordo parcialmente'' e ``concordo totalmente''.\textsuperscript{\citeproc{ref-aguinis2008}{125}}
\end{itemize}

\begin{figure}

{\centering \includegraphics{Ciencia-com-R_files/figure-latex/instrumento-likert-1} 

}

\caption{Exemplo de instrumento com 3 itens tipo Likert com 5 categorias cada.}\label{fig:instrumento-likert}
\end{figure}

\begin{table}
\centering
\caption{\label{tab:instrumento-likert}Descrição dos itens tipo Likert do instrumento.}
\centering
\begin{tabu} to \linewidth {>{}l>{\centering}X>{\centering}X>{\centering}X>{\centering}X>{\centering}X}
\toprule
\textbf{Itens} & \textbf{Discordância} & \textbf{Neutro} & \textbf{Concordância} & \textbf{Média} & \textbf{DP}\\
\midrule
\textbf{Item1} & 40 & 22 & 38 & 2.94 & 1.38\\
\textbf{Item2} & 36 & 20 & 44 & 3.12 & 1.42\\
\textbf{Item3} & 38 & 34 & 28 & 2.82 & 1.32\\
\bottomrule
\end{tabu}
\end{table}

\begin{infobox}{images/Rlogo}
O pacote \emph{likert}\textsuperscript{\citeproc{ref-likert}{139}} fornece a função \href{https://www.rdocumentation.org/packages/likert/versions/1.3.5/topics/likert}{\emph{likert}} para analisar respostas de instrumentos em escala tipo Likert.

\end{infobox}

\begin{infobox}{images/Rlogo}
O pacote \emph{ggstats}\textsuperscript{\citeproc{ref-ggstats}{140}} fornece a função \href{https://www.rdocumentation.org/packages/ggstats/versions/0.10.0/topics/gglikert}{\emph{gglikert}} para gerar um gráfico em escalas tipo Likert.

\end{infobox}

\begin{itemize}
\tightlist
\item
  O erro em escalas grosseiras é considerado sistemático mas não pode ser corrigido em nível da unidade de análise.\textsuperscript{\citeproc{ref-aguinis2008}{125}}
\end{itemize}

\section{Medição e Medidas}\label{mediuxe7uxe3o-e-medidas}

\subsection{O que é medição?}\label{o-que-uxe9-mediuxe7uxe3o}

\begin{itemize}
\item
  Processo empírico, realizado por meio de um instrumento, que estabelece uma correspondência rigorosa e objetiva entre uma observação e uma categoria em um modelo da observação.\textsuperscript{\citeproc{ref-ferris2004}{141}}
\item
  Esse processo tem como objetivo distinguir de maneira substantiva a manifestação observada de outras possíveis manifestações que também possam ser diferenciadas.\textsuperscript{\citeproc{ref-ferris2004}{141}}
\end{itemize}

\subsection{O que são medidas diretas?}\label{o-que-suxe3o-medidas-diretas}

\begin{itemize}
\tightlist
\item
  .\textsuperscript{\citeproc{ref-REF}{\textbf{REF?}}}
\end{itemize}

\subsection{O que são medidas derivadas?}\label{o-que-suxe3o-medidas-derivadas}

\begin{itemize}
\tightlist
\item
  .\textsuperscript{\citeproc{ref-REF}{\textbf{REF?}}}
\end{itemize}

\subsection{O que são medidas por teoria?}\label{o-que-suxe3o-medidas-por-teoria}

\begin{itemize}
\tightlist
\item
  .\textsuperscript{\citeproc{ref-REF}{\textbf{REF?}}}
\end{itemize}

\subsection{O que são medidas únicas?}\label{o-que-suxe3o-medidas-uxfanicas}

\begin{itemize}
\item
  A medida única da pressão arterial sistólica no braço esquerdo resulta em um valor pontual.\textsuperscript{\citeproc{ref-REF}{\textbf{REF?}}}
\item
  Medidas únicas obtidas de diferentes unidades de análise podem ser consideradas independentes se observadas outras condições na coleta de dados.\textsuperscript{\citeproc{ref-REF}{\textbf{REF?}}}
\item
  O valor pontual será considerado representativo da variável para a unidade de análise (ex.: \textbf{120 mmHg} para o participante \textbf{\#9}).
\end{itemize}

\begin{table}
\centering
\caption{\label{tab:medidas-unicas}Dados brutos com medidas únicas.}
\centering
\begin{tabu} to \linewidth {>{\centering}X>{\centering}X}
\toprule
\textbf{Unidade de análise} & \textbf{Pressão arterial, braço esquerdo (mmHg)}\\
\midrule
\cellcolor{gray!10}{1} & \cellcolor{gray!10}{118}\\
2 & 113\\
\cellcolor{gray!10}{3} & \cellcolor{gray!10}{116}\\
4 & 110\\
\cellcolor{gray!10}{5} & \cellcolor{gray!10}{111}\\
6 & 116\\
\cellcolor{gray!10}{7} & \cellcolor{gray!10}{120}\\
8 & 111\\
\cellcolor[HTML]{E6E6E6}{\textbf{\cellcolor{gray!10}{9}}} & \cellcolor[HTML]{E6E6E6}{\textbf{\cellcolor{gray!10}{120}}}\\
10 & 112\\
\bottomrule
\end{tabu}
\end{table}

\subsection{O que são medidas repetidas?}\label{o-que-suxe3o-medidas-repetidas}

\begin{itemize}
\item
  As medidas repetidas podem ser tabuladas separadamente, por exemplo para análise da confiabilidade de obtenção dessa medida.\textsuperscript{\citeproc{ref-REF}{\textbf{REF?}}}
\item
  A medida repetida da pressão arterial no braço esquerdo resulta em um conjunto de valores pontuais (ex.: \textbf{110 mmHg}, \textbf{118 mmHg} e \textbf{116 mmHg} para o participante \textbf{\#5}).
\end{itemize}

\begin{table}
\centering
\caption{\label{tab:medidas-repetidas-separadas}Dados brutos com medidas repetidas.}
\centering
\begin{tabu} to \linewidth {>{\centering}X>{\centering}X>{\centering}X>{\centering}X}
\toprule
\textbf{Unidade de análise} & \textbf{Pressão arterial, braço esquerdo (mmHg) \#1} & \textbf{Pressão arterial, braço esquerdo (mmHg) \#2} & \textbf{Pressão arterial, braço esquerdo (mmHg) \#3}\\
\midrule
\cellcolor{gray!10}{1} & \cellcolor{gray!10}{114} & \cellcolor{gray!10}{112} & \cellcolor{gray!10}{112}\\
2 & 115 & 120 & 113\\
\cellcolor{gray!10}{3} & \cellcolor{gray!10}{115} & \cellcolor{gray!10}{110} & \cellcolor{gray!10}{120}\\
4 & 117 & 116 & 114\\
\cellcolor[HTML]{E6E6E6}{\textbf{\cellcolor{gray!10}{5}}} & \cellcolor[HTML]{E6E6E6}{\textbf{\cellcolor{gray!10}{110}}} & \cellcolor[HTML]{E6E6E6}{\textbf{\cellcolor{gray!10}{118}}} & \cellcolor[HTML]{E6E6E6}{\textbf{\cellcolor{gray!10}{116}}}\\
6 & 110 & 120 & 113\\
\cellcolor{gray!10}{7} & \cellcolor{gray!10}{118} & \cellcolor{gray!10}{114} & \cellcolor{gray!10}{117}\\
8 & 111 & 112 & 119\\
\cellcolor{gray!10}{9} & \cellcolor{gray!10}{120} & \cellcolor{gray!10}{112} & \cellcolor{gray!10}{117}\\
10 & 110 & 115 & 115\\
\bottomrule
\end{tabu}
\end{table}

\begin{itemize}
\item
  As medidas repetidas podem ser agregadas por algum parâmetro --- ex.: média, mediana, máximo, mínimo, entre outros ---, observando-se a relevância biológica, clínica e/ou metodológica desta escolha.\textsuperscript{\citeproc{ref-REF}{\textbf{REF?}}}
\item
  Medidas agregadas obtidas de diferentes unidades de análise podem ser consideradas independentes se observadas outras condições na coleta de dados.\textsuperscript{\citeproc{ref-REF}{\textbf{REF?}}}
\item
  O valor agregado será considerado representativo da variável para a unidade de análise (ex.: média = \textbf{115 mmHg} para o participante \textbf{\#5}).
\end{itemize}

\begin{table}
\centering
\caption{\label{tab:medidas-repetidas-agregadas}Dados brutos com medidas repetidas agregadas.}
\centering
\begin{tabu} to \linewidth {>{\centering}X>{\centering}X}
\toprule
\textbf{Unidade de análise} & \textbf{Pressão arterial, braço esquerdo (mmHg) média}\\
\midrule
\cellcolor{gray!10}{1} & \cellcolor{gray!10}{113}\\
2 & 116\\
\cellcolor{gray!10}{3} & \cellcolor{gray!10}{115}\\
4 & 116\\
\cellcolor[HTML]{E6E6E6}{\textbf{\cellcolor{gray!10}{5}}} & \cellcolor[HTML]{E6E6E6}{\textbf{\cellcolor{gray!10}{115}}}\\
6 & 114\\
\cellcolor{gray!10}{7} & \cellcolor{gray!10}{116}\\
8 & 114\\
\cellcolor{gray!10}{9} & \cellcolor{gray!10}{116}\\
10 & 113\\
\bottomrule
\end{tabu}
\end{table}

\begin{infobox}{images/Rlogo}
O pacote \emph{stats}\textsuperscript{\citeproc{ref-stats}{131}} fornece a função \href{https://www.rdocumentation.org/packages/stats/versions/3.6.2/topics/aggregate}{\emph{aggregate}} para agregar medidas repetidas utilizando uma função personalizada.

\end{infobox}

\subsection{O que são medidas seriadas?}\label{o-que-suxe3o-medidas-seriadas}

\begin{itemize}
\item
  Medidas seriadas são possivelmente relacionadas e, portanto, dependentes na mesma unidade de análise.\textsuperscript{\citeproc{ref-REF}{\textbf{REF?}}}
\item
  Por exemplo, a medida seriada da pressão arterial no braço esquerdo, em intervalos tipicamente regulares (ex.: \textbf{114 mmHg}, \textbf{120 mmHg} e \textbf{110 mmHg} em \textbf{1 min}, \textbf{2 min} e \textbf{3 min}, respectivamente, para o participante \textbf{\#1}).
\end{itemize}

\begin{table}
\centering
\caption{\label{tab:medidas-seriadas-separadas}Dados brutos com medidas seriadas não agregadas.}
\centering
\begin{tabu} to \linewidth {>{\centering}X>{\centering}X>{\centering}X}
\toprule
\textbf{Unidade de análise} & \textbf{Tempo (min)} & \textbf{Pressão arterial, braço esquerdo (mmHg)}\\
\midrule
\cellcolor[HTML]{E6E6E6}{\textbf{\cellcolor{gray!10}{1}}} & \cellcolor[HTML]{E6E6E6}{\textbf{\cellcolor{gray!10}{1}}} & \cellcolor[HTML]{E6E6E6}{\textbf{\cellcolor{gray!10}{114}}}\\
\cellcolor[HTML]{E6E6E6}{\textbf{1}} & \cellcolor[HTML]{E6E6E6}{\textbf{2}} & \cellcolor[HTML]{E6E6E6}{\textbf{120}}\\
\cellcolor[HTML]{E6E6E6}{\textbf{\cellcolor{gray!10}{1}}} & \cellcolor[HTML]{E6E6E6}{\textbf{\cellcolor{gray!10}{3}}} & \cellcolor[HTML]{E6E6E6}{\textbf{\cellcolor{gray!10}{110}}}\\
2 & 1 & 119\\
\cellcolor{gray!10}{2} & \cellcolor{gray!10}{2} & \cellcolor{gray!10}{120}\\
2 & 3 & 114\\
\cellcolor{gray!10}{3} & \cellcolor{gray!10}{1} & \cellcolor{gray!10}{116}\\
3 & 2 & 114\\
\cellcolor{gray!10}{3} & \cellcolor{gray!10}{3} & \cellcolor{gray!10}{116}\\
4 & 1 & 113\\
\bottomrule
\end{tabu}
\end{table}

\begin{itemize}
\tightlist
\item
  Medidas seriadas também agregadas por parâmetros --- ex.: máximo, mínimo, amplitude --- são consideradas representativas da variação temporal ou de uma característica de interesse (ex.: amplitude = \textbf{10 mmHg} para o participante \textbf{\#1}).
\end{itemize}

\begin{table}
\centering
\caption{\label{tab:medidas-seriadas-agregadas}Dados brutos com medidas seriadas não agregadas.}
\centering
\begin{tabu} to \linewidth {>{\centering}X>{\centering}X}
\toprule
\textbf{Unidade de análise} & \textbf{Pressão arterial, braço esquerdo (mmHg) amplitude}\\
\midrule
\cellcolor[HTML]{E6E6E6}{\textbf{\cellcolor{gray!10}{1}}} & \cellcolor[HTML]{E6E6E6}{\textbf{\cellcolor{gray!10}{10}}}\\
2 & 6\\
\cellcolor{gray!10}{3} & \cellcolor{gray!10}{2}\\
4 & 6\\
\cellcolor{gray!10}{5} & \cellcolor{gray!10}{1}\\
6 & 8\\
\cellcolor{gray!10}{7} & \cellcolor{gray!10}{9}\\
8 & 10\\
\cellcolor{gray!10}{9} & \cellcolor{gray!10}{7}\\
10 & 5\\
\bottomrule
\end{tabu}
\end{table}

\begin{infobox}{images/Rlogo}
O pacote \emph{stats}\textsuperscript{\citeproc{ref-stats}{131}} fornece a função \href{https://www.rdocumentation.org/packages/stats/versions/3.6.2/topics/aggregate}{\emph{aggregate}} para agregar medidas repetidas utilizando uma função personalizada.

\end{infobox}

\subsection{O que são medidas múltiplas?}\label{o-que-suxe3o-medidas-muxfaltiplas}

\begin{itemize}
\item
  Medidas múltiplas também são possivelmente relacionadas e, portanto, são dependentes na mesma unidade de análise. Medidas múltiplas podem ser obtidas de modo repetido para análise agregada ou seriada.\textsuperscript{\citeproc{ref-REF}{\textbf{REF?}}}
\item
  A medida de pressão arterial bilateral resulta em um conjunto de valores pontuais (ex.: braço esquerdo = \textbf{114 mmHg}, braço direito = \textbf{118 mmHg} para o participante \textbf{\#8}). Neste caso, ambos os valores pontuais são considerados representativos daquela unidade de análise.
\end{itemize}

\begin{infobox}{images/Rlogo}
O pacote \emph{stats}\textsuperscript{\citeproc{ref-stats}{131}} fornece a função \href{https://www.rdocumentation.org/packages/stats/versions/3.6.2/topics/aggregate}{\emph{aggregate}} para agregar medidas repetidas utilizando uma função personalizada.

\end{infobox}

\section{Erro de medida}\label{erro-de-medida}

\subsection{O que são erros de medida?}\label{o-que-suxe3o-erros-de-medida}

\begin{itemize}
\item
  A natureza dos erros de medida são em geral atribuídos aos (1) instrumentos utilizados e variações no protocolo, na medida em que o seu tamanho médio pode ser reduzido por modificações e melhorias nesses instrumentos; e (2) variações genuínas medida em de curto prazo.\textsuperscript{\citeproc{ref-healy1978}{142}}
\item
  Estimativa pontual (um número exato) é um evento de probabilidade 0 sob um modelo contínuo.\textsuperscript{\citeproc{ref-REF}{\textbf{REF?}}}
\item
  Precisão como faixa \(±\epsilon\) tem probabilidade \textgreater{} 0, mensurável e dependente de \(\sigma\).\textsuperscript{\citeproc{ref-REF}{\textbf{REF?}}}
\item
  Isso motiva trabalhar com intervalos (faixas) em vez de pontos.\textsuperscript{\citeproc{ref-REF}{\textbf{REF?}}}
\end{itemize}

\begin{figure}

{\centering \includegraphics{Ciencia-com-R_files/figure-latex/erro-medida-1} 

}

\caption{Erro de medida em um modelo simples com erro normal. A linha tracejada indica o valor verdadeiro (desconhecido na prática) A área sombreada representa a probabilidade de cair na faixa $|X - \theta| \leq \varepsilon$, que é $>0$. A probabilidade de 'acertar no ponto' $X = \theta$ é $0$.}\label{fig:erro-medida}
\end{figure}

\subsection{Quais fontes de variabilidade são comumente investigadas?}\label{quais-fontes-de-variabilidade-suxe3o-comumente-investigadas}

\begin{itemize}
\item
  Intra/Entre participantes (isto é, unidades de análise).\textsuperscript{\citeproc{ref-altman1983}{143}}
\item
  Intra/Entre repetições.\textsuperscript{\citeproc{ref-altman1983}{143}}
\item
  Intra/Entre observadores.\textsuperscript{\citeproc{ref-altman1983}{143}}
\end{itemize}

\section{Instrumentos}\label{instrumentos}

\subsection{O que são instrumentos?}\label{o-que-suxe3o-instrumentos}

\begin{itemize}
\tightlist
\item
  .\textsuperscript{\citeproc{ref-REF}{\textbf{REF?}}}
\end{itemize}

\section{Acurácia e precisão}\label{acuruxe1cia-e-precisuxe3o}

\subsection{O que é acurácia?}\label{o-que-uxe9-acuruxe1cia}

\begin{itemize}
\item
  Acurácia expressa a proximidade de concordância entre uma mensuração e o valor real.\textsuperscript{\citeproc{ref-menditto2006}{144}}
\item
  Acurária está para medidas como validade está para instrumentos de medida.\textsuperscript{\citeproc{ref-REF}{\textbf{REF?}}}
\end{itemize}

\subsection{O que é precisão?}\label{o-que-uxe9-precisuxe3o}

\begin{itemize}
\item
  Precisão se refere à proximidade de concordância entre resultados de testes independentes obtidos nas mesmas condições de teste.\textsuperscript{\citeproc{ref-menditto2006}{144}}
\item
  Precisão é um índice de quão próximo os resultados podem ser repetidos entre mensurações repetidas.\textsuperscript{\citeproc{ref-Streiner2006}{145}}
\item
  Precisão está para medidas como confiabilidade está para instrumentos de medida.\textsuperscript{\citeproc{ref-REF}{\textbf{REF?}}}
\end{itemize}

\begin{figure}

{\centering \includegraphics{Ciencia-com-R_files/figure-latex/acuracia-precisao-1} 

}

\caption{Acurácia e precisão como propriedades de uma medida.}\label{fig:acuracia-precisao}
\end{figure}

\section{Viés e variabilidade}\label{viuxe9s-e-variabilidade}

\subsection{Qual é a relação entre viés e variabilidade?}\label{qual-uxe9-a-relauxe7uxe3o-entre-viuxe9s-e-variabilidade}

\begin{itemize}
\tightlist
\item
  .\textsuperscript{\citeproc{ref-REF}{\textbf{REF?}}}
\end{itemize}

\begin{figure}

{\centering \includegraphics{Ciencia-com-R_files/figure-latex/vies-variabilidade-1} 

}

\caption{Viés e variabilidade de uma medida.}\label{fig:vies-variabilidade}
\end{figure}

\chapter{\texorpdfstring{\textbf{Tabulação de dados}}{Tabulação de dados}}\label{tabulacao-dados}

\section{Planilhas eletrônicas}\label{planilhas-eletruxf4nicas}

\subsection{Qual a organização de uma tabela de dados?}\label{qual-a-organizauxe7uxe3o-de-uma-tabela-de-dados}

\begin{itemize}
\item
  As informações podem ser organizadas em formato de dados retangulares (ex.: matrizes, tabelas, quadro de dados) ou não retangulares (ex.: listas).\textsuperscript{\citeproc{ref-REF}{\textbf{REF?}}}
\item
  Cada variável possui sua própria coluna (vertical).\textsuperscript{\citeproc{ref-tierney2023}{146}}
\item
  Cada observação possui sua própria linha (horizontal).\textsuperscript{\citeproc{ref-tierney2023}{146}}
\item
  Cada valor possui sua própria célula especificada em um par (linha, coluna).\textsuperscript{\citeproc{ref-tierney2023}{146}}
\item
  Cada célula possui seu próprio dado.\textsuperscript{\citeproc{ref-tierney2023}{146}}
\end{itemize}

\begin{infobox}{images/Rlogo}
O pacote \emph{DataEditR}\textsuperscript{\citeproc{ref-DataEditR}{147}} fornece a função \href{https://www.rdocumentation.org/packages/DataEditR/versions/0.1.5/topics/dataInput}{\emph{data\_edit}} para interativamente criar, editar e salvar a tabela de dados.

\end{infobox}

\subsection{Qual a estrutura básica de uma tabela para análise estatística?}\label{qual-a-estrutura-buxe1sica-de-uma-tabela-para-anuxe1lise-estatuxedstica}

\begin{itemize}
\item
  Use apenas 1 (uma) planilha eletrônica para conter todas as informações coletadas. Evite múltiplas abas no mesmo arquivo, assim como múltiplos arquivos quando possível.\textsuperscript{\citeproc{ref-broman2018}{148}}
\item
  Use apenas 1 (uma) linha de cabeçalho para nomear os fatores e variáveis do seu estudo.\textsuperscript{\citeproc{ref-broman2018}{148}}
\end{itemize}

\begin{table}
\centering
\caption{\label{tab:tabela-0}Estrutura básica de uma tabela de dados.}
\centering
\begin{tabu} to \linewidth {>{\centering}X>{\centering}X>{\centering}X>{\centering}X}
\toprule
\textbf{V1} & \textbf{V2} & \textbf{V3} & \textbf{V4}\\
\midrule
\cellcolor{gray!10}{$x_{1,1}$} & \cellcolor{gray!10}{$x_{1,2}$} & \cellcolor{gray!10}{$x_{1,3}$} & \cellcolor{gray!10}{$x_{1,4}$}\\
$x_{2,1}$ & $x_{2,2}$ & $x_{2,3}$ & $x_{2,4}$\\
\cellcolor{gray!10}{$x_{3,1}$} & \cellcolor{gray!10}{$x_{3,2}$} & \cellcolor{gray!10}{$x_{3,3}$} & \cellcolor{gray!10}{$x_{3,4}$}\\
$x_{4,1}$ & $x_{4,2}$ & $x_{4,3}$ & $x_{4,4}$\\
\textbf{\cellcolor{gray!10}{$x_{5,1}$}} & \textbf{\cellcolor{gray!10}{$x_{5,2}$}} & \textbf{\cellcolor{gray!10}{$x_{5,3}$}} & \textbf{\cellcolor{gray!10}{$x_{5,4}$}}\\
\bottomrule
\end{tabu}
\end{table}

\begin{itemize}
\tightlist
\item
  Tipicamente, cada linha representa um participante e cada coluna representa uma variável ou fator do estudo. Estudos com medidas repetidas dos participantes podem conter múltiplas linhas para o mesmo participante (repetindo os dados na mesma coluna, conhecido como \emph{formato curto}) ou só uma linha para o participante (repetindo os dados em colunas separadas, conhecido como \emph{formato longo} ).\textsuperscript{\citeproc{ref-Juluru2015}{149}}
\end{itemize}

\begin{table}
\centering
\caption{\label{tab:tabela-0-long}Formato longo para medidas repetidas (múltiplas linhas por sujeito; colunas = variáveis)}
\centering
\begin{tabu} to \linewidth {>{\centering}X>{\centering}X>{\centering}X}
\toprule
\textbf{Linha} & \textbf{Variavel} & \textbf{Valor}\\
\midrule
\cellcolor{gray!10}{1} & \cellcolor{gray!10}{V1} & \cellcolor{gray!10}{$x_{1,1}$}\\
1 & V2 & $x_{1,2}$\\
\cellcolor{gray!10}{1} & \cellcolor{gray!10}{V3} & \cellcolor{gray!10}{$x_{1,3}$}\\
1 & V4 & $x_{1,4}$\\
\cellcolor{gray!10}{2} & \cellcolor{gray!10}{V1} & \cellcolor{gray!10}{$x_{2,1}$}\\
2 & V2 & $x_{2,2}$\\
\cellcolor{gray!10}{2} & \cellcolor{gray!10}{V3} & \cellcolor{gray!10}{$x_{2,3}$}\\
2 & V4 & $x_{2,4}$\\
\cellcolor{gray!10}{3} & \cellcolor{gray!10}{V1} & \cellcolor{gray!10}{$x_{3,1}$}\\
3 & V2 & $x_{3,2}$\\
\cellcolor{gray!10}{3} & \cellcolor{gray!10}{V3} & \cellcolor{gray!10}{$x_{3,3}$}\\
3 & V4 & $x_{3,4}$\\
\cellcolor{gray!10}{4} & \cellcolor{gray!10}{V1} & \cellcolor{gray!10}{$x_{4,1}$}\\
4 & V2 & $x_{4,2}$\\
\cellcolor{gray!10}{4} & \cellcolor{gray!10}{V3} & \cellcolor{gray!10}{$x_{4,3}$}\\
4 & V4 & $x_{4,4}$\\
\cellcolor{gray!10}{5} & \cellcolor{gray!10}{V1} & \cellcolor{gray!10}{$x_{5,1}$}\\
5 & V2 & $x_{5,2}$\\
\cellcolor{gray!10}{5} & \cellcolor{gray!10}{V3} & \cellcolor{gray!10}{$x_{5,3}$}\\
\textbf{5} & \textbf{V4} & \textbf{$x_{5,4}$}\\
\bottomrule
\end{tabu}
\end{table}

\begin{table}
\centering
\caption{\label{tab:tabela-0-wide}Formato largo para medidas repetidas (1 variável; colunas = tempos)}
\centering
\begin{tabu} to \linewidth {>{\centering}X>{\centering}X>{\centering}X>{\centering}X>{\centering}X}
\toprule
\textbf{id} & \textbf{T1} & \textbf{T2} & \textbf{T3} & \textbf{T4}\\
\midrule
1 & $x_{1,T1}$ & $x_{1,T2}$ & $x_{1,T3}$ & $x_{1,T4}$\\
2 & $x_{2,T1}$ & $x_{2,T2}$ & $x_{2,T3}$ & $x_{2,T4}$\\
3 & $x_{3,T1}$ & $x_{3,T2}$ & $x_{3,T3}$ & $x_{3,T4}$\\
4 & $x_{4,T1}$ & $x_{4,T2}$ & $x_{4,T3}$ & $x_{4,T4}$\\
\textbf{5} & \textbf{$x_{5,T1}$} & \textbf{$x_{5,T2}$} & \textbf{$x_{5,T3}$} & \textbf{$x_{5,T4}$}\\
\bottomrule
\end{tabu}
\end{table}

\subsection{O que usar para organizar tabelas para análise computadorizada?}\label{o-que-usar-para-organizar-tabelas-para-anuxe1lise-computadorizada}

\begin{itemize}
\item
  Seja consistente em: códigos para as variáveis categóricas; códigos para dados perdidos; nomes das variáveis; identificadores de participantes; nome dos arquivos; formato de datas; uso de caracteres de espaço.\textsuperscript{\citeproc{ref-broman2018}{148},\citeproc{ref-Juluru2015}{149}}
\item
  Crie um dicionário de dados (metadados) em um arquivo separado contendo: nome da variável, descrição da variável, unidades de medida e valores extremos possíveis.\textsuperscript{\citeproc{ref-broman2018}{148}}
\item
  Use recursos para validação de dados antes e durante a digitação de dados.\textsuperscript{\citeproc{ref-broman2018}{148},\citeproc{ref-Juluru2015}{149}}
\end{itemize}

\begin{infobox}{images/Rlogo}
O pacote \emph{data.table}\textsuperscript{\citeproc{ref-data.table}{150}} fornece a função \href{https://www.rdocumentation.org/packages/data.table/versions/1.14.8/topics/melt.data.table}{\emph{melt.data.table}} para reorganizar a tabela em diferentes formatos.

\end{infobox}

\subsection{O que não usar para organizar tabelas para análise computadorizada?}\label{o-que-nuxe3o-usar-para-organizar-tabelas-para-anuxe1lise-computadorizada}

\begin{itemize}
\item
  Não deixe células em branco: substitua dados perdidos por um código sistemático (ex.: NA {[}\emph{not available}{]}).\textsuperscript{\citeproc{ref-broman2018}{148}}
\item
  Não inclua análises estatísticas ou gráficos nas tabelas de dados brutos.\textsuperscript{\citeproc{ref-broman2018}{148}}
\item
  Não utilize cores como informação. Se necessário, crie colunas adicionais --- variáveis instrumentais ou auxiliares --- para identificar a informação de modo que possa ser analisada.\textsuperscript{\citeproc{ref-broman2018}{148}}
\item
  Não use células mescladas.
\item
  Delete linhas e/ou colunas totalmente em branco (sem unidades de análise e/ou sem variáveis).
\end{itemize}

\subsection{O que é recomendado e o que deve ser evitado na organização das tabelas para análise?}\label{o-que-uxe9-recomendado-e-o-que-deve-ser-evitado-na-organizauxe7uxe3o-das-tabelas-para-anuxe1lise}

\begin{table}
\centering
\caption{\label{tab:tabela-recomendada}Formatação recomendada para tabela de dados.}
\centering
\begin{tabu} to \linewidth {>{\centering}X>{\centering}X>{\centering}X>{\centering}X}
\toprule
\textbf{ID} & \textbf{Data.Coleta} & \textbf{Estado.Civil} & \textbf{Numero.Filhos}\\
\midrule
\cellcolor{gray!10}{1} & \cellcolor{gray!10}{14-01-2026} & \cellcolor{gray!10}{casado} & \cellcolor{gray!10}{NA}\\
2 & 15-01-2026 & casado & 1\\
\cellcolor{gray!10}{3} & \cellcolor{gray!10}{16-01-2026} & \cellcolor{gray!10}{casado} & \cellcolor{gray!10}{NA}\\
4 & 17-01-2026 & solteiro & NA\\
\cellcolor{gray!10}{5} & \cellcolor{gray!10}{18-01-2026} & \cellcolor{gray!10}{casado} & \cellcolor{gray!10}{NA}\\
6 & 19-01-2026 & solteiro & 0\\
\cellcolor{gray!10}{7} & \cellcolor{gray!10}{20-01-2026} & \cellcolor{gray!10}{solteiro} & \cellcolor{gray!10}{NA}\\
8 & 21-01-2026 & solteiro & NA\\
\cellcolor{gray!10}{9} & \cellcolor{gray!10}{22-01-2026} & \cellcolor{gray!10}{casado} & \cellcolor{gray!10}{NA}\\
10 & 23-01-2026 & solteiro & NA\\
\bottomrule
\end{tabu}
\end{table}

\begin{table}
\centering
\caption{\label{tab:tabela-evite}Formatação não recomendada para tabela de dados.}
\centering
\begin{tabu} to \linewidth {>{\centering}X>{\centering}X>{\centering}X>{\centering}X}
\toprule
\textbf{ID} & \textbf{Data de Coleta} & \textbf{Estado Civil} & \textbf{Número de Filhos}\\
\midrule
\cellcolor{gray!10}{1} & \cellcolor{gray!10}{14-01-2026} & \cellcolor{gray!10}{casado} & \cellcolor{gray!10}{NA}\\
2 & 15-01-2026 & Casado & 1\\
\cellcolor{gray!10}{3} & \cellcolor{gray!10}{16-01-2026} & \cellcolor{gray!10}{casado} & \cellcolor{gray!10}{NaN}\\
4 & 17-01-2026 & Solteiro & N/A\\
\cellcolor{gray!10}{5} & \cellcolor{gray!10}{18-01-2026} & \cellcolor{gray!10}{Casado} & \cellcolor{gray!10}{N.A.}\\
6 & 19-01-2026 & solteiro & 0\\
\cellcolor{gray!10}{7} & \cellcolor{gray!10}{20-01-2026} & \cellcolor{gray!10}{solteiro} & \cellcolor{gray!10}{}\\
8 & 21-01-2026 & Solteiro & na\\
\cellcolor{gray!10}{9} & \cellcolor{gray!10}{22-01-2026} & \cellcolor{gray!10}{casado} & \cellcolor{gray!10}{n.a.}\\
10 & 23-01-2026 & Solteiro & 999\\
\bottomrule
\end{tabu}
\end{table}

\chapter{\texorpdfstring{\textbf{Dados perdidos e imputados}}{Dados perdidos e imputados}}\label{dados-perdidos-imputados}

\section{Dados perdidos}\label{dados-perdidos}

\subsection{O que são dados perdidos?}\label{o-que-suxe3o-dados-perdidos}

\begin{itemize}
\tightlist
\item
  Dados perdidos são dados não coletados de um ou mais participantes, para uma ou mais variáveis.\textsuperscript{\citeproc{ref-Altman2007}{151}}
\end{itemize}

\begin{table}
\centering
\caption{\label{tab:dados-perdidos}Simulação de uma amostra (n=10) de um ensaio clínico aleatorizado (dados com perdas aleatórias).}
\centering
\begin{tabu} to \linewidth {>{}c>{\centering}X>{\centering}X>{\centering}X>{\centering}X>{\centering}X}
\toprule
\textbf{id} & \textbf{Grupo} & \textbf{Idade} & \textbf{Sexo} & \textbf{Desfecho (pré)} & \textbf{Desfecho (pós)}\\
\midrule
\textbf{1} & Controle & 53 & F & 57.0 & 41.3\\
\textbf{2} & Controle & 64 & F & 45.3 & 70.0\\
\textbf{3} & Controle & 65 & M & 39.3 & NA\\
\textbf{4} & Intervenção & 66 & F & 47.8 & NA\\
\textbf{5} & Controle & 44 & M & 39.7 & 65.7\\
\textbf{6} & Intervenção & NA & F & 42.7 & NA\\
\textbf{7} & Intervenção & 67 & M & 43.7 & 64.9\\
\textbf{8} & Intervenção & NA & F & 33.1 & 63.3\\
\textbf{9} & Controle & 68 & F & 58.4 & 61.6\\
\textbf{10} & Controle & 74 & M & 51.5 & 54.3\\
\bottomrule
\end{tabu}
\end{table}

\begin{infobox}{images/Rlogo}
O pacote \emph{base}\textsuperscript{\citeproc{ref-base}{55}} fornece a função \href{https://www.rdocumentation.org/packages/base/versions/3.6.2/topics/na}{\emph{is.na}} para identificar que elementos de um objeto são dados perdidos.

\end{infobox}

\subsection{Qual o problema de um estudo ter dados perdidos?}\label{qual-o-problema-de-um-estudo-ter-dados-perdidos}

\begin{itemize}
\item
  Uma grande quantidade de dados perdidos pode comprometer a integridade científica do estudo, considerando-se que o tamanho da amostra foi estimado para observar um determinado tamanho de efeito mínimo.\textsuperscript{\citeproc{ref-Altman2007}{151}}
\item
  Perda de participantes no estudo por dados perdidos pode reduzir o poder estatístico (erro tipo II).\textsuperscript{\citeproc{ref-Altman2007}{151}}
\item
  Não existe solução globalmente satisfatória para o problema de dados perdidos.\textsuperscript{\citeproc{ref-Altman2007}{151}}
\end{itemize}

\section{Mecanismos geradores de dados perdidos}\label{mecanismos-geradores-de-dados-perdidos}

\subsection{Quais os mecanismos geradores de dados perdidos?}\label{quais-os-mecanismos-geradores-de-dados-perdidos}

\begin{itemize}
\tightlist
\item
  Dados perdidos completamente ao acaso (\emph{missing completely at random}, MCAR), em que os dados perdidos estão distribuídos aleatoriamente nos dados da amostra.\textsuperscript{\citeproc{ref-Heymans2022}{152},\citeproc{ref-carpenter2021}{153}}
\end{itemize}

\begin{figure}

{\centering \includegraphics{Ciencia-com-R_files/figure-latex/mcar-1} 

}

\caption{Representação gráfica de dados perdidos completamente ao acaso (MCAR) em um estudo randomizado controlado (RCT).}\label{fig:mcar}
\end{figure}

\begin{itemize}
\tightlist
\item
  Dados perdidos ao acaso (\emph{missing at random}, MAR), em que a probabilidade de ocorrência de dados perdidos é relacionada a outras variáveis medidas.\textsuperscript{\citeproc{ref-Heymans2022}{152},\citeproc{ref-carpenter2021}{153}}
\end{itemize}

\begin{figure}

{\centering \includegraphics{Ciencia-com-R_files/figure-latex/mar-1} 

}

\caption{Representação gráfica de dados perdidos ao acaso (MAR) em um estudo randomizado controlado (RCT).}\label{fig:mar}
\end{figure}

\begin{itemize}
\tightlist
\item
  Dados perdidos não ao acaso (\emph{missing not at random}, MNAR), em que a probabilidade da ocorrência de dados perdidos é relacionada com a própria variável.\textsuperscript{\citeproc{ref-Heymans2022}{152},\citeproc{ref-carpenter2021}{153}}
\end{itemize}

\begin{figure}

{\centering \includegraphics{Ciencia-com-R_files/figure-latex/mnar-1} 

}

\caption{Representação gráfica de dados perdidos não ao acaso (MNAR) em um estudo randomizado controlado (RCT).}\label{fig:mnar}
\end{figure}

\subsection{Como identificar o mecanismo gerador de dados perdidos em um banco de dados?}\label{como-identificar-o-mecanismo-gerador-de-dados-perdidos-em-um-banco-de-dados}

\begin{itemize}
\item
  Por definição, não é possível avaliar se os dados foram perdidos ao acaso (MAR) ou não (MNAR).\textsuperscript{\citeproc{ref-Heymans2022}{152}}
\item
  Testes t e regressões logísticas podem ser aplicados para identificar relações entre variáveis com e sem dados perdidos, criando um fator de análise (`dado perdido' = 1, `dado observado' = 0).\textsuperscript{\citeproc{ref-Heymans2022}{152}}
\end{itemize}

\begin{infobox}{images/Rlogo}
O pacote \emph{misty}\textsuperscript{\citeproc{ref-misty}{154}} fornece a função \href{https://www.rdocumentation.org/packages/misty/versions/0.5.0/topics/na.test}{\emph{na.test}} para executar o Little's Missing Completely at Random (MCAR) test\textsuperscript{\citeproc{ref-little1988}{155}}.

\end{infobox}

\begin{infobox}{images/Rlogo}
O pacote \emph{naniar}\textsuperscript{\citeproc{ref-naniar}{156}} fornece a função \href{https://www.rdocumentation.org/packages/naniar/versions/1.0.0/topics/mcar_test}{\emph{mcar\_test}} para executar o Little's Missing Completely at Random (MCAR) test\textsuperscript{\citeproc{ref-little1988}{155}}.

\end{infobox}

\begin{infobox}{images/Rlogo}
O pacote \emph{naniar}\textsuperscript{\citeproc{ref-naniar}{156}} fornece a função \href{https://www.rdocumentation.org/packages/naniar/versions/1.0.0/topics/gg_miss_upset}{\emph{gg\_miss\_upset}} para gerar o gráfico Upset para visualizar padrões de dados perdidos.

\end{infobox}

\section{Estratégias para lidar com dados perdidos}\label{estratuxe9gias-para-lidar-com-dados-perdidos}

\subsection{Que estratégias podem ser utilizadas na coleta de dados quando há expectativa de perda amostral?}\label{que-estratuxe9gias-podem-ser-utilizadas-na-coleta-de-dados-quando-huxe1-expectativa-de-perda-amostral}

\begin{itemize}
\tightlist
\item
  Na expectativa de ocorrência de perda amostral, com consequente ocorrência de dados perdidos, recomenda-se ampliar o tamanho da amostra com um percentual correspondente a tal estimativa (ex.: 10\%), embora ainda não corrija potenciais vieses pela perda.\textsuperscript{\citeproc{ref-Altman2007}{151}}
\end{itemize}

\subsection{Que estratégias podem ser utilizadas na análise quando há dados perdidos?}\label{que-estratuxe9gias-podem-ser-utilizadas-na-anuxe1lise-quando-huxe1-dados-perdidos}

\begin{itemize}
\item
  Na ocorrência de dados perdidos, a análise mais comum compreende apenas os `casos completos', com exclusão de participantes com algum dado perdido nas variáveis do estudo. Em casos de grande quantidade de dados perdidos, pode-se perder muito poder estatístico (erro tipo II elevado).\textsuperscript{\citeproc{ref-Altman2007}{151}}
\item
  A análise de dados completos é válida quando pode-se argumentar que a probabilidade de o participante ter dados completos depende apenas das covariáveis e não dos desfechos.\textsuperscript{\citeproc{ref-carpenter2021}{153}}
\item
  A análise de dados completos é eficiente quando todos os dados perdidos estão no desfecho, ou quando cada participante com dados perdidos nas covariáveis também possui dados perdidos nos desfechos.\textsuperscript{\citeproc{ref-carpenter2021}{153}}
\end{itemize}

\begin{infobox}{images/Rlogo}
O pacote \emph{base}\textsuperscript{\citeproc{ref-base}{55}} fornece a função \href{https://www.rdocumentation.org/packages/base/versions/3.6.2/topics/na.fail}{\emph{na.omit}} para remover dados perdidos de um objeto em um banco de dados.

\end{infobox}

\begin{infobox}{images/Rlogo}
O pacote \emph{stats}\textsuperscript{\citeproc{ref-stats}{131}} fornece a função \href{https://www.rdocumentation.org/packages/stats/versions/3.6.2/topics/complete.cases}{\emph{complete.cases}} para identificar os casos completos - isto é, sem dados perdidos - em um banco de dados.

\end{infobox}

\subsection{Que estratégias podem ser utilizadas na redação de estudos em que há dados perdidos?}\label{que-estratuxe9gias-podem-ser-utilizadas-na-redauxe7uxe3o-de-estudos-em-que-huxe1-dados-perdidos}

\begin{itemize}
\tightlist
\item
  Informar: o número de participantes com dados perdidos; diferenças nas taxas de dados perdidos entre os braços do estudo; os motivos dos dados perdidos; quaisquer diferenças entre os participantes com e sem dados perdidos; o padrão de perda; os métodos para tratamento de dados perdidos das variáveis em análise; os resultados de análises de sensibilidade; as implicações dos dados perdidos na interpretação do resultados.\textsuperscript{\citeproc{ref-Akl2015}{157}}
\end{itemize}

\section{Dados imputados}\label{dados-imputados}

\subsection{O que são dados imputados?}\label{o-que-suxe3o-dados-imputados}

\begin{itemize}
\tightlist
\item
  .\textsuperscript{\citeproc{ref-REF}{\textbf{REF?}}}
\end{itemize}

\subsection{Quando a imputação de dados é indicada?}\label{quando-a-imputauxe7uxe3o-de-dados-uxe9-indicada}

\begin{itemize}
\item
  A análise com imputação de dados pode ser útil quando pode-se argumentar que os dados foram perdidos ao acaso (MAR); quando o desfecho foi observado e os dados perdidos estão nas covariáveis; e variáveis auxiliares --- preditoras do desfecho e não dos dados perdidos --- estão disponíveis.\textsuperscript{\citeproc{ref-carpenter2021}{153}}
\item
  Na ocorrência de dados perdidos, a imputação de dados (substituição por dados simulados plausíveis preditos pelos dados presentes) pode ser uma alternativa para manter o erro tipo II estipulado no plano de análise.\textsuperscript{\citeproc{ref-Altman2007}{151}}
\end{itemize}

\subsection{Quais são os métodos de imputação de dados?}\label{quais-suxe3o-os-muxe9todos-de-imputauxe7uxe3o-de-dados}

\begin{itemize}
\item
  Modelos lineares e logísticos podem ser utilizados para imputar dados perdidos em variáveis contínuas e dicotômicas, respectivamente.\textsuperscript{\citeproc{ref-austin2023}{158}}
\item
  Os métodos de imputação de dados mais robustos incluem a imputação multivariada por equações encadeadas (\emph{multivariate imputation by chained equations}, MICE)\textsuperscript{\citeproc{ref-mice}{159}} e a correspondência média preditiva (\emph{predictive mean matching}, PMM).\textsuperscript{\citeproc{ref-rubin1986}{160},\citeproc{ref-little1988a}{161}}
\end{itemize}

\begin{figure}

{\centering \includegraphics[width=1\linewidth]{Ciencia-com-R_files/figure-latex/impacto-imputacao-1} 

}

\caption{Impacto de métodos de imputação na distribuição de uma variável contínua com dados perdidos.}\label{fig:impacto-imputacao}
\end{figure}

\begin{infobox}{images/Rlogo}
Os pacotes \emph{mice}\textsuperscript{\citeproc{ref-mice}{159}} e \emph{miceadds}\textsuperscript{\citeproc{ref-miceadds}{162}} fornecem funções \href{https://www.rdocumentation.org/packages/mice/versions/3.16.0/topics/mice}{\emph{mice}} e \href{https://www.rdocumentation.org/packages/miceadds/versions/3.16-18/topics/mi.anova}{\emph{mi.anova}} para imputação multivariada por equações encadeadas, respectivamente, para imputação de dados.

\end{infobox}

\chapter{\texorpdfstring{\textbf{Dados anonimizados e sintéticos}}{Dados anonimizados e sintéticos}}\label{dados-anonimizados-sinteticos}

\section{Dados anonimizados}\label{dados-anonimizados}

\subsection{O que são dados anonimizados?}\label{o-que-suxe3o-dados-anonimizados}

\begin{itemize}
\tightlist
\item
  .\textsuperscript{\citeproc{ref-REF}{\textbf{REF?}}}
\end{itemize}

\subsection{Com anonimizar os dados de um banco?}\label{com-anonimizar-os-dados-de-um-banco}

\begin{itemize}
\tightlist
\item
  .\textsuperscript{\citeproc{ref-REF}{\textbf{REF?}}}
\end{itemize}

\begin{infobox}{images/Rlogo}
O pacote \emph{ids}\textsuperscript{\citeproc{ref-ids}{163}} fornece a função \href{https://www.rdocumentation.org/packages/ids/versions/1.0.1/topics/random_id}{\emph{random\_id}} para criar identificadores aleatórios por criptografia.

\end{infobox}

\begin{infobox}{images/Rlogo}
O pacote \emph{hash}\textsuperscript{\citeproc{ref-hash}{164}} fornece a função \href{https://www.rdocumentation.org/packages/hash/versions/3.0.1/topics/hash}{\emph{hash}} para criar identificadores por objetos \emph{hash}.

\end{infobox}

\begin{infobox}{images/Rlogo}
O pacote \emph{anonimizer}\textsuperscript{\citeproc{ref-anonymizer}{165}} fornece a função \href{https://www.rdocumentation.org/packages/anonymizer/versions/0.2.0/topics/anonymize}{\emph{anonymize}} para criar uma versão anônima de variáveis em um banco de dados.

\end{infobox}

\begin{infobox}{images/Rlogo}
O pacote \emph{digest}\textsuperscript{\citeproc{ref-digest}{166}} fornece a função \href{https://www.rdocumentation.org/packages/digest/versions/0.6.33/topics/digest}{\emph{digest}} para criar identificadores por objetos \emph{hash} criptografados ou não.

\end{infobox}

\section{Dados sintéticos}\label{dados-sintuxe9ticos}

\subsection{O que são dados sintéticos?}\label{o-que-suxe3o-dados-sintuxe9ticos}

\begin{itemize}
\tightlist
\item
  .\textsuperscript{\citeproc{ref-REF}{\textbf{REF?}}}
\end{itemize}

\begin{infobox}{images/Rlogo}
O pacote \emph{synthpop}\textsuperscript{\citeproc{ref-synthpop}{167}} fornece a função \href{https://www.rdocumentation.org/packages/synthpop/versions/1.8-0/topics/syn}{\emph{syn}} para criar bancos de dados sintéticos a partir de um banco de dados real.

\end{infobox}

\cftaddtitleline{toc}{chapter}{\rule{\textwidth}{0.4pt}}{}

\chapter*{\texorpdfstring{\emph{PARTE 4: ANÁLISES DESCRITIVAS E EXPLORATÓRIAS}}{PARTE 4: ANÁLISES DESCRITIVAS E EXPLORATÓRIAS}}\label{parte-4}
\addcontentsline{toc}{chapter}{\emph{PARTE 4: ANÁLISES DESCRITIVAS E EXPLORATÓRIAS}}

\par\noindent\rule{\textwidth}{0.05in}

\section*{Primeiros passos na análise: descrever, visualizar e explorar padrões nos dados}\label{primeiros-passos-na-anuxe1lise-descrever-visualizar-e-explorar-padruxf5es-nos-dados}

\markboth{}{}

\chapter{\texorpdfstring{\textbf{Distribuições e parâmetros}}{Distribuições e parâmetros}}\label{distribuicoes-parametros}

\section{Distribuições de probabilidade}\label{distribuiuxe7uxf5es-de-probabilidade}

\subsection{O que são distribuições de probabilidade?}\label{o-que-suxe3o-distribuiuxe7uxf5es-de-probabilidade}

\begin{itemize}
\tightlist
\item
  Uma distribuição de probabilidade é uma função que descreve os valores possíveis ou o intervalo de valores de uma variável (eixo horizontal) e a frequência com que cada valor é observado (eixo vertical).\textsuperscript{\citeproc{ref-vetter2017}{108}}
\end{itemize}

\subsection{Como representar distribuições de probabilidade?}\label{como-representar-distribuiuxe7uxf5es-de-probabilidade}

\begin{itemize}
\item
  Tabelas de frequência, polígonos de frequência, gráficos de barras, histogramas e \emph{boxplots} são formas de representar distribuições de probabilidade.\textsuperscript{\citeproc{ref-s2011}{168}}
\item
  Tabelas de frequência mostram as categorias de medição e o número de observações em cada uma. É necessário conhecer o intervalo de valores (mínimo e máximo), que é dividido em intervalos arbitrários chamados ``intervalos de classe''.\textsuperscript{\citeproc{ref-s2011}{168}}
\item
  Se houver muitos intervalos, não haverá redução significativa na quantidade de dados, e pequenas variações serão perceptíveis. Se houver poucos intervalos, a forma da distribuição não poderá ser adequadamente determinada.\textsuperscript{\citeproc{ref-s2011}{168}}
\item
  A quantidade de intervalos pode ser determinada pelo método de Sturges, que é dado pela fórmula \(k = 1 + 3.322 \times \log_{10}(n)\), onde \(k\) é o número de intervalos e \(n\) é o número de observações.\textsuperscript{\citeproc{ref-sturges1926}{169}}
\item
  A quantidade de intervalos pode ser determinada pelo método de Scott, que é dado pela fórmula \(h = 3.5 \times \text{sd}(x) \times n^{-1/3}\), onde \(h\) é a largura do intervalo, \(\text{sd}(x)\) é o desvio-padrão e \(n\) é o número de observações.\textsuperscript{\citeproc{ref-scott1979}{170}}
\item
  A quantidade de intervalos pode ser determinada pelo método de Freedman-Diaconis, que é dado pela fórmula \(h = 2 \times \text{IQR}(x) \times n^{-1/3}\), onde \(h\) é a largura do intervalo, \(\text{IQR}(x)\) é o intervalo interquartil e \(n\) é o número de observações.\textsuperscript{\citeproc{ref-freedman1981}{171}}
\end{itemize}

\begin{figure}

{\centering \includegraphics{Ciencia-com-R_files/figure-latex/distribuicao-histograma-1} 

}

\caption{Histogramas com diferentes métodos de binning.: Sturges, Scott e Freedman-Diaconis.}\label{fig:distribuicao-histograma}
\end{figure}

\begin{itemize}
\item
  A largura das classes pode ser determinada dividindo o intervalo total de observações pelo número de classes. Recomenda-se larguras iguais, mas larguras desiguais podem ser usadas quando existirem grandes lacunas nos dados ou em contextos específicos. Os intervalos devem ser mutuamente exclusivos e não sobrepostos, evitando intervalos abertos (ex.: \textless5, \textgreater10).\textsuperscript{\citeproc{ref-s2011}{168}}
\item
  Polígonos de frequência são gráficos de linhas que conectam os pontos médios de cada barra do histograma. Eles são úteis para comparar duas ou mais distribuições de frequência.\textsuperscript{\citeproc{ref-s2011}{168}}
\item
  Gráficos de barra verticais ou horizontais representam a distribuição de frequências de uma variável categórica. A altura de cada barra é proporcional à frequência da classe. A largura da barra é igual à largura da classe. A área de cada barra é proporcional à frequência da classe. A área total do gráfico de barras é igual ao número total de observações.\textsuperscript{\citeproc{ref-s2011}{168}}
\item
  Histogramas representam a distribuição de frequências de uma variável contínua. A altura de cada barra é proporcional à frequência da classe. A largura da barra é igual à largura da classe. A área de cada barra é proporcional à frequência da classe. A área total do histograma é igual ao número total de observações.\textsuperscript{\citeproc{ref-s2011}{168}}
\item
  \emph{Boxplots} representam a distribuição de frequências de uma variável contínua. A linha central divide os dados em duas partes iguais (mediana ou Q2). A caixa inferior representa o primeiro quartil (Q1) e a caixa superior representa o terceiro quartil (Q3). A linha inferior é o mínimo e a linha superior é o máximo. Os valores atípicos são representados por pontos individuais.\textsuperscript{\citeproc{ref-s2011}{168}}
\end{itemize}

\begin{infobox}{images/Rlogo}
O pacote \emph{grDevices}\textsuperscript{\citeproc{ref-grDevices}{172}} fornece a função \href{https://www.rdocumentation.org/packages/grDevices/versions/3.6.2/topics/nclass}{nclass} para determinar a quantidade de classes de um histograma com os métodos de Sturge\textsuperscript{\citeproc{ref-sturges1926}{169}}, Scott\textsuperscript{\citeproc{ref-scott1979}{170}} ou Freedman-Diaconis\textsuperscript{\citeproc{ref-freedman1981}{171}}.

\end{infobox}

\begin{infobox}{images/Rlogo}
O pacote \emph{ggplot2}\textsuperscript{\citeproc{ref-ggplot2}{173}} fornece a função \href{https://www.rdocumentation.org/packages/ggplot2/versions/3.5.2/topics/geom_freqpoly}{geom\_freqpoly} para criar histogramas.

\end{infobox}

\subsection{Quais características definem uma distribuição?}\label{quais-caracteruxedsticas-definem-uma-distribuiuxe7uxe3o}

\begin{itemize}
\tightlist
\item
  Uma distribuição pode ser definida por modelos matemáticos e caracterizada por parâmetros de tendência central, dispersão, simetria e curtose.\textsuperscript{\citeproc{ref-REF}{\textbf{REF?}}}
\end{itemize}

\subsection{Quais são as distribuições mais comuns?}\label{quais-suxe3o-as-distribuiuxe7uxf5es-mais-comuns}

\begin{itemize}
\item
  Distribuções discretas:

  \begin{itemize}
  \item
    Bernoulli: resultado de um único teste com dois possíveis desfechos (sucesso ou fracasso).\textsuperscript{\citeproc{ref-REF}{\textbf{REF?}}}
  \item
    Binomial: número de sucessos em \emph{k} tentativas.\textsuperscript{\citeproc{ref-REF}{\textbf{REF?}}}
  \item
    Geométrica: número de testes até o 1o sucesso.\textsuperscript{\citeproc{ref-REF}{\textbf{REF?}}}
  \item
    Binomial negativa: número de testes até o \emph{k}-ésimo sucesso.\textsuperscript{\citeproc{ref-REF}{\textbf{REF?}}}
  \item
    Hipergeométrica: número de indivíduos na amostra tomados sem reposição.\textsuperscript{\citeproc{ref-REF}{\textbf{REF?}}}
  \item
    Poisson: número de eventos em um intervalo de tempo fixo.\textsuperscript{\citeproc{ref-REF}{\textbf{REF?}}}
  \item
    Uniforme: resultados (finitos) que são igualmente prováveis.\textsuperscript{\citeproc{ref-REF}{\textbf{REF?}}}
  \item
    Multinomial: resultados de múltiplos testes com mais de dois possíveis desfechos.\textsuperscript{\citeproc{ref-REF}{\textbf{REF?}}}
  \end{itemize}
\end{itemize}

\begin{figure}

{\centering \includegraphics{Ciencia-com-R_files/figure-latex/distribuicao-discreta-1} 

}

\caption{Distribuições discretas e suas funções de probabilidade.}\label{fig:distribuicao-discreta}
\end{figure}

\begin{itemize}
\item
  Distribuições contínuas:

  \begin{itemize}
  \item
    Uniforme: .\textsuperscript{\citeproc{ref-REF}{\textbf{REF?}}}
  \item
    Exponencial: .\textsuperscript{\citeproc{ref-REF}{\textbf{REF?}}}
  \item
    Normal: .\textsuperscript{\citeproc{ref-REF}{\textbf{REF?}}}
  \item
    Aproximação binomial: .\textsuperscript{\citeproc{ref-REF}{\textbf{REF?}}}
  \item
    Aproximação Poisson: .\textsuperscript{\citeproc{ref-REF}{\textbf{REF?}}}
  \item
    t-Student: .\textsuperscript{\citeproc{ref-REF}{\textbf{REF?}}}
  \item
    Qui-quadrado: .\textsuperscript{\citeproc{ref-REF}{\textbf{REF?}}}
  \item
    Weibull: .\textsuperscript{\citeproc{ref-REF}{\textbf{REF?}}}
  \item
    Gama: .\textsuperscript{\citeproc{ref-REF}{\textbf{REF?}}}
  \item
    Log-normal: .\textsuperscript{\citeproc{ref-REF}{\textbf{REF?}}}
  \item
    Beta: .\textsuperscript{\citeproc{ref-REF}{\textbf{REF?}}}
  \item
    Logística: .\textsuperscript{\citeproc{ref-REF}{\textbf{REF?}}}
  \item
    Pareto.\textsuperscript{\citeproc{ref-REF}{\textbf{REF?}}}
  \end{itemize}
\end{itemize}

\begin{figure}

{\centering \includegraphics{Ciencia-com-R_files/figure-latex/distribuicao-basica-1} 

}

\caption{Distribuições contínuas básicas e suas funções de densidade.}\label{fig:distribuicao-basica}
\end{figure}

\begin{figure}

{\centering \includegraphics{Ciencia-com-R_files/figure-latex/distribuicao-simetricas-1} 

}

\caption{Distribuições contínuas aproximadas e suas funções de densidade.}\label{fig:distribuicao-simetricas}
\end{figure}

\begin{figure}

{\centering \includegraphics{Ciencia-com-R_files/figure-latex/distribuicao-aproximada-1} 

}

\caption{Distribuições contínuas aproximadas e suas funções de densidade.}\label{fig:distribuicao-aproximada}
\end{figure}

\begin{figure}

{\centering \includegraphics{Ciencia-com-R_files/figure-latex/distribuicao-inferencia-1} 

}

\caption{Distribuições contínuas para inferência e suas funções de densidade.}\label{fig:distribuicao-inferencia}
\end{figure}

\begin{figure}

{\centering \includegraphics{Ciencia-com-R_files/figure-latex/distribuicao-especificos-1} 

}

\caption{Distribuições contínuas para dados específicos e suas funções de densidade.}\label{fig:distribuicao-especificos}
\end{figure}

\begin{figure}

{\centering \includegraphics{Ciencia-com-R_files/figure-latex/distribuicao-proporcoes-1} 

}

\caption{Distribuições contínuas para probabilidades e proporções e suas funções de densidade.}\label{fig:distribuicao-proporcoes}
\end{figure}

\begin{figure}

{\centering \includegraphics{Ciencia-com-R_files/figure-latex/distribuicao-pareto-1} 

}

\caption{Distribuições contínuas com caudas pesadas e suas funções de densidade.}\label{fig:distribuicao-pareto}
\end{figure}

\subsection{Quais são as funções de uma distribuição?}\label{quais-suxe3o-as-funuxe7uxf5es-de-uma-distribuiuxe7uxe3o}

\begin{itemize}
\item
  Função de massa de probabilidade (\emph{probability mass function}, pmf).\textsuperscript{\citeproc{ref-REF}{\textbf{REF?}}}
\item
  Função de distribuição cumulativa (\emph{cumulative distribution function}, cdf).\textsuperscript{\citeproc{ref-REF}{\textbf{REF?}}}
\item
  Função quantílicas (\emph{quantile function}, qf).\textsuperscript{\citeproc{ref-REF}{\textbf{REF?}}}
\item
  Função geradora de números aleatórios (\emph{random function}, rf).\textsuperscript{\citeproc{ref-REF}{\textbf{REF?}}}
\end{itemize}

\begin{infobox}{images/Rlogo}
O pacote \emph{stats}\textsuperscript{\citeproc{ref-stats}{131}} fornece funções de distribuição de probabilidade (p), funções de densidade (d), funções quantílicas (q) e funções geradores de números aleatórios (r) para as distribuições \href{https://www.rdocumentation.org/packages/stats/versions/3.6.2/topics/Normal}{normal}, \href{https://www.rdocumentation.org/packages/stats/versions/3.6.2/topics/TDist}{Student t}, \href{https://www.rdocumentation.org/packages/stats/versions/3.6.2/topics/Binomial}{binomial}, \href{https://www.rdocumentation.org/packages/stats/versions/3.6.2/topics/Chisquare}{qui-quadrado}, \href{https://www.rdocumentation.org/packages/stats/versions/3.6.2/topics/Uniform}{uniforme}, dentre outras.

\end{infobox}

\begin{infobox}{images/Rlogo}
O pacote \emph{ggdist}\textsuperscript{\citeproc{ref-ggdist}{174}} fornece a função \href{https://www.rdocumentation.org/packages/ggdist/versions/3.3.0/topics/geom_slabinterval}{geom\_slabinterval} para criar gráficos de distribuição de probabilidade (p) e funções de densidade (d) as distribuições.

\end{infobox}

\begin{infobox}{images/Rlogo}
O pacote \emph{ggfortify}\textsuperscript{\citeproc{ref-ggfortify}{175}} fornece a função \href{https://www.rdocumentation.org/packages/ggfortify/versions/0.4.16/topics/ggdistribution}{ggdistribution} para criar gráficos de distribuição de probabilidade (p), funções de densidade (d), funções quantílicas (q) e funções geradores de números aleatórios (r) para as distribuições.

\end{infobox}

\subsection{O que é a distribuição normal?}\label{o-que-uxe9-a-distribuiuxe7uxe3o-normal}

\begin{itemize}
\item
  A distribuição normal (ou gaussiana) é uma distribuição com desvios simétricos positivos e negativos em torno de um valor central.\textsuperscript{\citeproc{ref-Ali2016}{109}}
\item
  Em uma distribuição normal, o intervalo de 1 desvio-padrão (±1DP) inclui cerca de 68\% dos dados; de 2 desvios-padrão (±2DP) cerca de 95\% dos dados; e no intervalo de 3 desvios-padrão (±3DP) cerca de 99\% dos dados.\textsuperscript{\citeproc{ref-Ali2016}{109}}
\end{itemize}

\begin{figure}

{\centering \includegraphics{Ciencia-com-R_files/figure-latex/distribuicao-normal-1} 

}

\caption{Distribuições e funções de probabilidade.}\label{fig:distribuicao-normal}
\end{figure}

\subsection{Que métodos podem ser utilizados para identificar a normalidade da distribuição?}\label{que-muxe9todos-podem-ser-utilizados-para-identificar-a-normalidade-da-distribuiuxe7uxe3o}

\begin{itemize}
\item
  Histogramas.\textsuperscript{\citeproc{ref-vetter2017}{108}}
\item
  Gráficos Q-Q.\textsuperscript{\citeproc{ref-vetter2017}{108}}
\item
  Testes de hipótese nula:\textsuperscript{\citeproc{ref-vetter2017}{108}}

  \begin{itemize}
  \item
    Kolmogorov-Smirnov
  \item
    Shapiro-Wilk
  \item
    Anderson-Darling
  \end{itemize}
\end{itemize}

\begin{figure}

{\centering \includegraphics{Ciencia-com-R_files/figure-latex/normalidade-distribuicao-1} 

}

\caption{Distribuição normal e métodos de visualização e testes de normalidade.}\label{fig:normalidade-distribuicao}
\end{figure}

\subsection{O que são distribuições não-normais?}\label{o-que-suxe3o-distribuiuxe7uxf5es-nuxe3o-normais}

\begin{itemize}
\tightlist
\item
  .\textsuperscript{\citeproc{ref-REF}{\textbf{REF?}}}
\end{itemize}

\section{Distribuições multivariadas}\label{distribuiuxe7uxf5es-multivariadas}

\subsection{O que são distribuições multivariadas?}\label{o-que-suxe3o-distribuiuxe7uxf5es-multivariadas}

\begin{itemize}
\item
  Distribuições multivariadas descrevem a probabilidade conjunta de duas ou mais variáveis aleatórias.\textsuperscript{\citeproc{ref-REF}{\textbf{REF?}}}
\item
  Exemplos de distribuições multivariadas incluem a distribuição normal multivariada, a distribuição t multivariada, a distribuição binomial multinomial e a distribuição de Dirichlet.\textsuperscript{\citeproc{ref-REF}{\textbf{REF?}}}
\end{itemize}

\begin{figure}

{\centering \includegraphics{Ciencia-com-R_files/figure-latex/distribuicao-multivariada-1} 

}

\caption{Distribuição normal bivariada e amostra simulada com histogramas marginais.}\label{fig:distribuicao-multivariada}
\end{figure}

\section{Parâmetros}\label{paruxe2metros}

\subsection{O que são parâmetros?}\label{o-que-suxe3o-paruxe2metros}

\begin{itemize}
\item
  Parâmetros são informações que definem um modelo teórico, como propriedades de uma coleção de indivíduos.\textsuperscript{\citeproc{ref-Altman1999}{107}}
\item
  Parâmetros definem características de uma população inteira, tipicamente não observados por ser inviável ter acesso a todos os indivíduos que constituem tal população.\textsuperscript{\citeproc{ref-vetter2017}{108}}
\end{itemize}

\begin{infobox}{images/Rlogo}
O pacote \emph{base}\textsuperscript{\citeproc{ref-base}{55}} fornece a função \href{https://www.rdocumentation.org/packages/base/versions/3.6.2/topics/summary}{\emph{summary}} para calcular diversos parâmetros descritivos.

\end{infobox}

\subsection{O que é uma análise paramétrica?}\label{o-que-uxe9-uma-anuxe1lise-paramuxe9trica}

\begin{itemize}
\item
  Testes paramétricos possuem suposições sobre as características e/ou parâmetros da distribuição dos dados na população.\textsuperscript{\citeproc{ref-vetter2017}{108}}
\item
  Testes paramétricos assumem que: a variável é quantitativa numérica (contínua); os dados foram amostrados de uma população com distribuição normal; a variância da(S) amostra(s) é igual à da população; as amostras foram selecionadas de modo aleatório na população; os valores de cada amostra são independentes entre si.\textsuperscript{\citeproc{ref-vetter2017}{108},\citeproc{ref-Ali2016}{109}}
\item
  Testes paramétricos são baseados na suposição de que os dados amostrais provêm de uma população com parâmetros fixos determinando sua distribuição de probabilidade.\textsuperscript{\citeproc{ref-kwak2017}{8}}
\end{itemize}

\subsection{O que é uma análise não paramétrica?}\label{o-que-uxe9-uma-anuxe1lise-nuxe3o-paramuxe9trica}

\begin{itemize}
\item
  Testes não-paramétricos fazem poucas suposições, ou menos rigorosas, sobre as características e/ou parâmetros da distribuição dos dados na população.\textsuperscript{\citeproc{ref-vetter2017}{108},\citeproc{ref-Ali2016}{109}}
\item
  Testes não-paramétricos são úteis quando as suposições de normalidade não podem ser sustentadas.\textsuperscript{\citeproc{ref-Ali2016}{109}}
\end{itemize}

\subsection{Devemos testar as suposições de normalidade?}\label{devemos-testar-as-suposiuxe7uxf5es-de-normalidade}

\begin{itemize}
\tightlist
\item
  Testes preliminares de normalidade não são necessários para a maioria dos testes paramétricos de comparação, pois eles são robustos contra desvios moderados da normalidade. Normalidade da distribuição deve ser estabelecida para a população.\textsuperscript{\citeproc{ref-rochon2012}{176}}
\end{itemize}

\subsection{Por que as análises paramétricas são preferidas?}\label{por-que-as-anuxe1lises-paramuxe9tricas-suxe3o-preferidas}

\begin{itemize}
\item
  Em geral, testes paramétricos são mais robustos (isto é, possuem menores erros tipo I e II) que seus testes não-paramétricos correspondentes.\textsuperscript{\citeproc{ref-vetter2017}{108},\citeproc{ref-greenhalgh1997}{177},\citeproc{ref-schmider2010}{178}}
\item
  Testes não-paramétricos apresentam menor poder estatístico (maior erro tipo II) comparados aos testes paramétricos correspondentes.\textsuperscript{\citeproc{ref-Ali2016}{109}}
\end{itemize}

\subsection{Que parâmetros podem ser estimados?}\label{que-paruxe2metros-podem-ser-estimados}

\begin{itemize}
\item
  Parâmetros de tendência central.\textsuperscript{\citeproc{ref-Ali2016}{109},\citeproc{ref-kanji2006}{179}}
\item
  Parâmetros de dispersão.\textsuperscript{\citeproc{ref-Ali2016}{109},\citeproc{ref-kanji2006}{179},\citeproc{ref-Curran-Everett2008}{180}}
\item
  Parâmetros de proporção.\textsuperscript{\citeproc{ref-Ali2016}{109},\citeproc{ref-kanji2006}{179},\citeproc{ref-Altman1994}{181},\citeproc{ref-Altman1994}{181}}
\item
  Parâmetros de distribuição.\textsuperscript{\citeproc{ref-kanji2006}{179}}
\item
  Parâmetros de extremos.\textsuperscript{\citeproc{ref-Ali2016}{109}}
\end{itemize}

\begin{infobox}{images/Rlogo}
O pacote \emph{base}\textsuperscript{\citeproc{ref-base}{55}} fornece a função \href{https://www.rdocumentation.org/packages/base/versions/3.6.2/topics/summary}{\emph{summary}} para calcular diversos parâmetros descritivos.

\end{infobox}

\section{Erro}\label{erro}

\subsection{Que parâmetros de erro podem ser estimados?}\label{que-paruxe2metros-de-erro-podem-ser-estimados}

\begin{itemize}
\tightlist
\item
  Margem de erro (ME).\eqref{eq:me}.
\end{itemize}

\begin{equation}
\label{eq:me}
ME = z_{\alpha/2} \cdot \frac{\sigma}{\sqrt{n}}
\end{equation}

\begin{itemize}
\tightlist
\item
  Erro-padrão da média (EPM) \eqref{eq:sem} (\(sigma\) conhecido) e \eqref{eq:sem-unk} (\(sigma\) desconhecido).\textsuperscript{\citeproc{ref-Curran-Everett2008}{180},\citeproc{ref-krzywinski2013}{182}}
\end{itemize}

\begin{equation}
\label{eq:sem}
EPM = \frac{\sigma}{\sqrt{n}}
\end{equation}

\begin{equation}
\label{eq:sem-unk}
\widehat{EPM} = \frac{s}{\sqrt{n}}
\end{equation}

\begin{itemize}
\tightlist
\item
  Intervalo de confiança para variância conhecida \eqref{eq:ic-known} e desconhecida \eqref{eq:ic-unknown}: Captura a média populacional correspondente ao nível de significância \(\alpha\) pré-estabelecido\textsuperscript{\citeproc{ref-Ali2016}{109},\citeproc{ref-kanji2006}{179},\citeproc{ref-krzywinski2013}{182},\citeproc{ref-Cumming2007}{183}}.
\end{itemize}

\begin{equation}
\label{eq:ic-known}
IC = \bar{x} \pm z_{\alpha/2} \cdot \frac{\sigma}{\sqrt{n}}
\end{equation}

\begin{equation}
\label{eq:ic-unknown}
IC = \bar{x} \pm t_{\alpha/2, n-1} \cdot \frac{s}{\sqrt{n}}
\end{equation}

\section{Tendência central}\label{tenduxeancia-central}

\subsection{Que parâmetros de tendência central podem ser estimados?}\label{que-paruxe2metros-de-tenduxeancia-central-podem-ser-estimados}

\begin{itemize}
\tightlist
\item
  Média aritmética \eqref{eq:media-aritm}, ponderada \eqref{eq:media-pond}, geométrica \eqref{eq:media-geom} ou harmônica \eqref{eq:media-harm}.\textsuperscript{\citeproc{ref-Ali2016}{109},\citeproc{ref-kanji2006}{179},\citeproc{ref-s.2011a}{184}}
\end{itemize}

\begin{equation}
\label{eq:media-aritm}
\bar{x} = \frac{\sum_{i=1}^{n} x_i}{n}
\end{equation}

\begin{equation}
\label{eq:media-pond}
\bar{x}_p = \frac{\sum_{i=1}^{n} w_i x_i}{\sum_{i=1}^{n} w_i}
\end{equation}

\begin{equation}
\label{eq:media-geom}
\bar{x}_g = \left( \prod_{i=1}^{n} x_i \right)^{\frac{1}{n}}
\end{equation}

\begin{equation}
\label{eq:media-harm}
\bar{x}_h = \frac{n}{\sum_{i=1}^{n} \frac{1}{x_i}}
\end{equation}

\begin{itemize}
\tightlist
\item
  Mediana \eqref{eq:mediana}.\textsuperscript{\citeproc{ref-Ali2016}{109},\citeproc{ref-kanji2006}{179},\citeproc{ref-s.2011}{185}}
\end{itemize}

\begin{equation}
\label{eq:mediana}
\tilde{x} = \begin{cases}
x_{\left(\frac{n+1}{2}\right)}, & \text{se } n \text{ é ímpar} \\
\frac{x_{\left(\frac{n}{2}\right)} + x_{\left(\frac{n}{2} + 1\right)}}{2}, & \text{se } n \text{ é par}
\end{cases}
\end{equation}

\begin{itemize}
\tightlist
\item
  Moda \eqref{eq:moda}, onde \(f(x)\) é a função de frequência absoluta ou relativa e \(x_1, x_2, \ldots, x_n\) são os valores observados.\textsuperscript{\citeproc{ref-Ali2016}{109},\citeproc{ref-kanji2006}{179},\citeproc{ref-s.2011}{185}}
\end{itemize}

\begin{equation}
\label{eq:moda}
\operatorname{Mo} \in \arg\max_{x \in \{x_1,\ldots,x_n\}} f(x)
\end{equation}

\begin{itemize}
\tightlist
\item
  Moda (dados agrupados) \eqref{eq:moda-agrupada}, onde: \(L\) = limite inferior da classe modal; \(f_1\) = frequência da classe modal; \(f_0\) = frequência da classe anterior à classe modal; \(f_2\) = frequência da classe posterior à classe modal; \(h\) = amplitude da classe modal.
\end{itemize}

\begin{equation}
\label{eq:moda-agrupada}
\operatorname{Mo} = L + \frac{(f_1 - f_0)}{(f_1 - f_0) + (f_1 - f_2)} \cdot h
\end{equation}

\begin{itemize}
\item
  A posição relativa das medidas de tendência central (média, mediana e moda) depende da forma da distribuição.\textsuperscript{\citeproc{ref-s.2011}{185}}
\item
  Em uma distribuição normal, as três medidas são idênticas.\textsuperscript{\citeproc{ref-s.2011}{185}}
\item
  A média é sempre puxada para os valores extremos, por isso é deslocada para a cauda em distribuições assimétricas.\textsuperscript{\citeproc{ref-s.2011}{185}}
\item
  A mediana fica entre a média e a moda em distribuições assimétricas.\textsuperscript{\citeproc{ref-s.2011}{185}}
\item
  A moda é o valor mais frequente e, portanto, se localiza no pico da distribuição assimétrica.\textsuperscript{\citeproc{ref-s.2011}{185}}
\item
  Uma distribuição pode uma moda (unimodal), duas modas (bimodal) ou três ou mais modas (multimodal), indicando a presença de mais de um valor com alta frequência.\textsuperscript{\citeproc{ref-s.2011}{185}}
\end{itemize}

\begin{figure}

{\centering \includegraphics{Ciencia-com-R_files/figure-latex/distribuicoes-unimodal-bimodal-multimodal-1} 

}

\caption{Distribuições unimodal, bimodal e multimodal.}\label{fig:distribuicoes-unimodal-bimodal-multimodal}
\end{figure}

\begin{figure}

{\centering \includegraphics{Ciencia-com-R_files/figure-latex/tendencia-central-1} 

}

\caption{Parâmetros de tendência central em distribuições assimétricas e normais.}\label{fig:tendencia-central}
\end{figure}

\begin{infobox}{images/Rlogo}
O pacote \emph{base}\textsuperscript{\citeproc{ref-base}{55}} fornece a função \href{https://www.rdocumentation.org/packages/base/versions/3.6.2/topics/summary}{\emph{summary}} para calcular diversos parâmetros descritivos.

\end{infobox}

\subsection{Como escolher o parâmetro de tendência central?}\label{como-escolher-o-paruxe2metro-de-tenduxeancia-central}

\begin{itemize}
\item
  A mediana é preferida à média quando existem poucos valores extremos na distribuição, alguns valores são indeterminados, ou há uma distribuição aberta, ou os dados são medidos em uma escala ordinal.\textsuperscript{\citeproc{ref-s.2011}{185}}
\item
  A moda é preferida quando os dados são medidos em uma escala nominal.\textsuperscript{\citeproc{ref-s.2011}{185}}
\item
  A média geométrica é preferida quando os dados são medidos em uma escala logarítmica.\textsuperscript{\citeproc{ref-s.2011}{185}}
\end{itemize}

\section{Dispersão}\label{dispersuxe3o}

\subsection{Que parâmetros de dispersão podem ser estimados?}\label{que-paruxe2metros-de-dispersuxe3o-podem-ser-estimados}

\begin{itemize}
\tightlist
\item
  Variância \eqref{eq:variancia}.\textsuperscript{\citeproc{ref-Ali2016}{109},\citeproc{ref-kanji2006}{179}}
\end{itemize}

\begin{equation}
\label{eq:variancia}
s^2 = \frac{\sum_{i=1}^{n} (x_i - \bar{x})^2}{n-1}
\end{equation}

\begin{itemize}
\tightlist
\item
  Desvio-padrão \eqref{eq:desvpad}.\textsuperscript{\citeproc{ref-Curran-Everett2008}{180},\citeproc{ref-krzywinski2013}{182},\citeproc{ref-manikandan2011}{186}}
\end{itemize}

\begin{equation}
\label{eq:desvpad}
s = \sqrt{\frac{\sum_{i=1}^{n} (x_i - \bar{x})^2}{n-1}}
\end{equation}

\begin{itemize}
\tightlist
\item
  Amplitude \eqref{eq:amplitude}.\textsuperscript{\citeproc{ref-Ali2016}{109},\citeproc{ref-kanji2006}{179},\citeproc{ref-manikandan2011}{186}}
\end{itemize}

\begin{equation}
\label{eq:amplitude}
A = x_{\max} - x_{\min}
\end{equation}

\begin{itemize}
\tightlist
\item
  Intervalo interquartil \eqref{eq:iqr}.\textsuperscript{\citeproc{ref-Ali2016}{109},\citeproc{ref-kanji2006}{179},\citeproc{ref-manikandan2011}{186}}
\end{itemize}

\begin{equation}
\label{eq:iqr}
IQR = Q_3 - Q_1
\end{equation}

\begin{figure}

{\centering \includegraphics{Ciencia-com-R_files/figure-latex/dispersao-1} 

}

\caption{Parâmetros de dispersão em distribuições normais.}\label{fig:dispersao}
\end{figure}

\begin{infobox}{images/Rlogo}
O pacote \emph{base}\textsuperscript{\citeproc{ref-base}{55}} fornece a função \href{https://www.rdocumentation.org/packages/base/versions/3.6.2/topics/summary}{\emph{summary}} para calcular diversos parâmetros descritivos.

\end{infobox}

\begin{infobox}{images/Rlogo}
O pacote \emph{stats}\textsuperscript{\citeproc{ref-stats}{131}} fornece a função \href{https://www.rdocumentation.org/packages/stats/versions/3.6.2/topics/confint}{\emph{confint}} para calcular o intervalo de confiança em um nível de significância \(\alpha\).

\end{infobox}

\subsection{Como escolher o parâmetro de dispersão?}\label{como-escolher-o-paruxe2metro-de-dispersuxe3o}

\begin{itemize}
\item
  Desvio-padrão é apropriado quando a média é utilizada como parâmetro de tendência central em distribuições simétricas.\textsuperscript{\citeproc{ref-manikandan2011}{186}}
\item
  Amplitude ou intervalo interquartil são apropriadas para variáveis ordinais ou distribuições assimétricas.\textsuperscript{\citeproc{ref-manikandan2011}{186}}
\end{itemize}

\subsection{O que é a correção de Bessel para variância?}\label{o-que-uxe9-a-correuxe7uxe3o-de-bessel-para-variuxe2ncia}

\begin{itemize}
\item
  Correção de Bessel é um ajuste feito no denominador da fórmula de variância da amostra --- ou seja, o número de graus de liberdade --- para evitar que a variância amostral seja menor do que a variância populacional.\textsuperscript{\citeproc{ref-sahai1992}{187}}
\item
  A correção de Bessel é feita subtraindo-se 1 do número de observações da amostra, ou seja, \(n - 1\) \eqref{eq:bessel}.\textsuperscript{\citeproc{ref-sahai1992}{187}}
\end{itemize}

\begin{equation}
\label{eq:bessel}
s^2 = \frac{\sum_{i=1}^{n} (x_i - \bar{x})^2}{n-1}
\end{equation}

\subsection{Por que a correção de Bessel para variância é importante?}\label{por-que-a-correuxe7uxe3o-de-bessel-para-variuxe2ncia-uxe9-importante}

\begin{itemize}
\item
  A correção de Bessel é importante porque a variância amostral tende a ser menor do que a variância populacional, especialmente em amostras pequenas.\textsuperscript{\citeproc{ref-sahai1992}{187}}
\item
  A correção de Bessel ajuda a garantir que a variância amostral seja uma estimativa mais precisa da variância populacional, o que é fundamental para a validade dos testes estatísticos e das inferências feitas a partir da amostra.\textsuperscript{\citeproc{ref-sahai1992}{187}}
\end{itemize}

\section{Proporção}\label{proporuxe7uxe3o}

\subsection{Que parâmetros de proporção podem ser estimados?}\label{que-paruxe2metros-de-proporuxe7uxe3o-podem-ser-estimados}

\begin{itemize}
\tightlist
\item
  Frequência absoluta \eqref{eq:freq-absoluta}.\textsuperscript{\citeproc{ref-Ali2016}{109},\citeproc{ref-kanji2006}{179},\citeproc{ref-Altman1994}{181}}
\end{itemize}

\begin{equation}
\label{eq:freq-absoluta}
f_i = n_i
\end{equation}

\begin{itemize}
\tightlist
\item
  Frequência relativa \eqref{eq:freq-relativa}.\textsuperscript{\citeproc{ref-Ali2016}{109},\citeproc{ref-kanji2006}{179},\citeproc{ref-Altman1994}{181}}
\end{itemize}

\begin{equation}
\label{eq:freq-relativa}
fr_i = \frac{n_i}{N}
\end{equation}

\begin{itemize}
\tightlist
\item
  Percentil \eqref{eq:percentil}, onde \(k\) é o percentil desejado (0 a 100) e \(n\) é o número total de observações na amostra.\textsuperscript{\citeproc{ref-Ali2016}{109},\citeproc{ref-kanji2006}{179},\citeproc{ref-Altman1994}{181}}
\end{itemize}

\begin{equation}
\label{eq:percentil}
P_k = x_{\left(\frac{k}{100} \cdot (n+1)\right)}
\end{equation}

\begin{itemize}
\item
  Quantil: é o ponto de corte que define a divisão da amostra em grupos de tamanhos iguais. Portanto, não se referem aos grupos em si, mas aos valores que os dividem:\textsuperscript{\citeproc{ref-Altman1994}{181}}

  \begin{itemize}
  \item
    Tercil: 2 valores que dividem a amostra em 3 grupos de tamanhos iguais.\textsuperscript{\citeproc{ref-Altman1994}{181}}
  \item
    Quartil: 3 valores que dividem a amostra em 4 grupos de tamanhos iguais.\textsuperscript{\citeproc{ref-Altman1994}{181}}
  \item
    Quintil: 4 valores que dividem a amostra em 5 grupos de tamanhos iguais.\textsuperscript{\citeproc{ref-Altman1994}{181}}
  \item
    Decil: 9 valores que dividem a amostra em 10 grupos de tamanhos iguais.\textsuperscript{\citeproc{ref-Altman1994}{181}}
  \end{itemize}
\end{itemize}

\begin{infobox}{images/Rlogo}
O pacote \emph{base}\textsuperscript{\citeproc{ref-base}{55}} fornece a função \href{https://www.rdocumentation.org/packages/base/versions/3.6.2/topics/summary}{\emph{summary}} para calcular diversos parâmetros descritivos.

\end{infobox}

\begin{infobox}{images/Rlogo}
O pacote \emph{base}\textsuperscript{\citeproc{ref-base}{55}} fornece a função \href{https://www.rdocumentation.org/packages/base/versions/3.6.2/topics/table}{\emph{table}} para calcular proporções.

\end{infobox}

\begin{infobox}{images/Rlogo}
O pacote \emph{stats}\textsuperscript{\citeproc{ref-base}{55}} fornece a função \href{https://www.rdocumentation.org/packages/stats/versions/3.6.2/topics/quantile}{\emph{quantile}} para executar análise de percentis.

\end{infobox}

\section{Extremos}\label{extremos}

\subsection{O que são valores extremos?}\label{o-que-suxe3o-valores-extremos}

\begin{itemize}
\tightlist
\item
  Valores extremos podem constituir valores legítimos ou ilegítimos de uma distribuição.\textsuperscript{\citeproc{ref-leys2019}{188}}
\end{itemize}

\subsection{Que parâmetros extremos podem ser estimados?}\label{que-paruxe2metros-extremos-podem-ser-estimados}

\begin{itemize}
\tightlist
\item
  Mínimo \eqref{eq:min}.\textsuperscript{\citeproc{ref-Ali2016}{109}}
\end{itemize}

\begin{equation}
\label{eq:min}
\text{Mínimo} = \min(x_1, x_2, \ldots, x_n)
\end{equation}

\begin{itemize}
\tightlist
\item
  Máximo \eqref{eq:max}.\textsuperscript{\citeproc{ref-Ali2016}{109}}
\end{itemize}

\begin{equation}
\label{eq:max}
\text{Máximo} = \max(x_1, x_2, \ldots, x_n)
\end{equation}

\begin{figure}

{\centering \includegraphics{Ciencia-com-R_files/figure-latex/regressao-extremos-1} 

}

\caption{Regressão linear com valores extremos.}\label{fig:regressao-extremos}
\end{figure}

\section{Distribuição}\label{distribuiuxe7uxe3o}

\subsection{Que parâmetros de distribuição podem ser estimados?}\label{que-paruxe2metros-de-distribuiuxe7uxe3o-podem-ser-estimados}

\begin{itemize}
\tightlist
\item
  Assimetria \eqref{eq:skew}.\textsuperscript{\citeproc{ref-kanji2006}{179}}
\end{itemize}

\begin{equation}
\label{eq:skew}
\gamma_1 = \frac{\frac{1}{n}\sum_{i=1}^{n}(x_i - \bar{x})^3}{\left(\frac{1}{n}\sum_{i=1}^{n}(x_i - \bar{x})^2\right)^{3/2}}
\end{equation}

\begin{itemize}
\tightlist
\item
  Curtose \eqref{eq:kurt}.\textsuperscript{\citeproc{ref-kanji2006}{179}}
\end{itemize}

\begin{equation}
\label{eq:kurt}
\gamma_2 = \frac{\frac{1}{n}\sum_{i=1}^{n}(x_i - \bar{x})^4}{\left(\frac{1}{n}\sum_{i=1}^{n}(x_i - \bar{x})^2\right)^{2}}
\end{equation}

\begin{itemize}
\tightlist
\item
  Excesso de Curtose \eqref{eq:exkurt}.\textsuperscript{\citeproc{ref-kanji2006}{179}}
\end{itemize}

\begin{equation}
\label{eq:exkurt}
\kappa = \gamma_2 - 3
\end{equation}

\begin{figure}

{\centering \includegraphics{Ciencia-com-R_files/figure-latex/distribuicoes-parametros-1} 

}

\caption{Parâmetros de distribuição: Assimetria e Curtose.}\label{fig:distribuicoes-parametros}
\end{figure}

\begin{figure}

{\centering \includegraphics{Ciencia-com-R_files/figure-latex/curtose-1} 

}

\caption{Parâmetros de distribuição: Curtose em distribuições simétricas (normal vs. uniforme).}\label{fig:curtose}
\end{figure}

\section{Robustez em medidas de localização}\label{robustez-em-medidas-de-localizauxe7uxe3o}

\subsection{O que é ponto de quebra?}\label{o-que-uxe9-ponto-de-quebra}

\begin{itemize}
\tightlist
\item
  É a menor proporção de contaminação que pode levar o estimador a resultados arbitrariamente errados; quanto maior, mais robusto.\textsuperscript{\citeproc{ref-rousseeuw2011}{189}}
\end{itemize}

\subsection{Por que a média não é robusta?}\label{por-que-a-muxe9dia-nuxe3o-uxe9-robusta}

\begin{itemize}
\tightlist
\item
  Porque tem ponto de quebra \(~0%
  \) e função influência não limitada; um único \emph{outlier} pode distorcer a média arbitrariamente.\textsuperscript{\citeproc{ref-rousseeuw2011}{189}}
\end{itemize}

\subsection{Qual a alternativa robusta para localização?}\label{qual-a-alternativa-robusta-para-localizauxe7uxe3o}

\begin{itemize}
\tightlist
\item
  Mediana, com \(~50%
  \) de ponto de quebra e função influência limitada.\textsuperscript{\citeproc{ref-rousseeuw2011}{189}}
\end{itemize}

\subsection{Como estimar escala de forma robusta?}\label{como-estimar-escala-de-forma-robusta}

\begin{itemize}
\tightlist
\item
  \emph{Median Absolute Deviation} (MAD) \eqref{eq:mad}, com correção 1,483 para normalidade, com \(~50%
  \) de ponto de quebra.\textsuperscript{\citeproc{ref-rousseeuw2011}{189}}
\end{itemize}

\begin{equation}
\label{eq:max}
MAD = 1.483 \cdot \text{median}(|x_i - \text{median}(x)|)
\end{equation}

\begin{itemize}
\tightlist
\item
  Primeiro quartil das diferenças pareadas (\(Qn\)) \eqref{eq:qn}, com \(~50%
  \) de ponto de quebra.\textsuperscript{\citeproc{ref-rousseeuw2011}{189}}
\end{itemize}

\begin{equation}
\label{eq:qn}
Qn = 2.2219 \cdot \text{first quartile}(|x_i - x_j|; i < j)
\end{equation}

\begin{itemize}
\tightlist
\item
  O intervalo interquartil (\(IQR\)) \eqref{eq:iqr} é robusto, com ponto de quebra \(~25%
  \), sendo simples de interpretar e útil em boxplots.\textsuperscript{\citeproc{ref-rousseeuw2011}{189}}
\end{itemize}

\section{Parâmetros robustos}\label{paruxe2metros-robustos}

\subsection{O que são parâmetros robustos?}\label{o-que-suxe3o-paruxe2metros-robustos}

\begin{itemize}
\tightlist
\item
  Parâmetros robustos são medidas de posição e dispersão que permanecem estáveis mesmo na presença de valores discrepantes.\textsuperscript{\citeproc{ref-daszykowski2007}{190}}
\end{itemize}

\subsection{Que parâmetros robustos podem ser estimados?}\label{que-paruxe2metros-robustos-podem-ser-estimados}

\begin{itemize}
\item
  Mediana em vez da média aritmética, pois é menos sensível a valores extremos.\textsuperscript{\citeproc{ref-daszykowski2007}{190}}
\item
  \emph{Median Absolute Deviation} (MAD) em vez do desvio-padrão \(\sigma\), que pode ser escalonado por 1,483 para comparabilidade.\textsuperscript{\citeproc{ref-daszykowski2007}{190}}
\item
  \(Qn\) e \(Sn\) como estimadores alternativos de dispersão robusta.\textsuperscript{\citeproc{ref-daszykowski2007}{190}}
\item
  Média e variância Winsorizadas como opções intermediárias, reduzindo a influência dos \emph{outliers}.\textsuperscript{\citeproc{ref-daszykowski2007}{190}}
\end{itemize}

\subsection{Por que utilizar parâmetros robustos?}\label{por-que-utilizar-paruxe2metros-robustos}

\begin{itemize}
\item
  Parâmetros robustos garantem maior confiabilidade quando os dados não seguem a normalidade ou apresentam contaminação por \emph{outliers}.\textsuperscript{\citeproc{ref-daszykowski2007}{190}}
\item
  Parâmetros robustos permitem análises mais estáveis em estudos exploratórios, evitando decisões equivocadas sobre variabilidade ou tendência central.\textsuperscript{\citeproc{ref-daszykowski2007}{190}}
\end{itemize}

\chapter{\texorpdfstring{\textbf{Análise inicial de dados}}{Análise inicial de dados}}\label{analise-inicial}

\section{Análise inicial de dados}\label{anuxe1lise-inicial-de-dados}

\subsection{O que é análise inicial de dados?}\label{o-que-uxe9-anuxe1lise-inicial-de-dados}

\begin{itemize}
\item
  Análise inicial de dados\textsuperscript{\citeproc{ref-chatfield1986}{191}} é uma sequência de procedimentos que visam principalmente a transparência e integridade das pré-condições do estudo para conduzir a análise estatística apropriada de modo responsável para responder aos problemas da pesquisa.\textsuperscript{\citeproc{ref-Baillie2022}{134}}
\item
  O objetivo da análise inicial de dados é propiciar dados prontos para análise estatística, incluindo informações confiáveis sobre as propriedades dos dados.\textsuperscript{\citeproc{ref-Baillie2022}{134}}
\item
  A análise inicial de dados pode ser dividida nas seguintes etapas:\textsuperscript{\citeproc{ref-Baillie2022}{134}}

  \begin{itemize}
  \item
    Configuração dos metadados
  \item
    Limpeza dos dados
  \item
    Verificação dos dados
  \item
    Relatório inicial dos dados
  \item
    Refinamento e atualização do plano de análise estatística
  \item
    Documentação e relatório da análise inicial de dados
  \end{itemize}
\item
  A análise inicial de dados não deve ser confundida com análise exploratória,\textsuperscript{\citeproc{ref-Ferketich1986}{192}} nem deve ser utilizada para hipotetizar após os dados serem coletados (conhecido como \emph{Hypothesizing After Results are Known}, HARKing).\textsuperscript{\citeproc{ref-Kerr1998}{92}}
\end{itemize}

\subsection{Como conduzir uma análise inicial de dados?}\label{como-conduzir-uma-anuxe1lise-inicial-de-dados}

\begin{itemize}
\item
  Desenvolva um plano de análise inicial de dados consistente com os objetivos da pesquisa. Por exemplo, verifique a distribuição e escala das variáveis, procure por observações não-usuais ou improváveis, avalie possíveis padrões de dados perdidos.\textsuperscript{\citeproc{ref-Baillie2022}{134}}
\item
  Não altere diretamente os dados de uma tabela obtida de uma fonte. Use scripts para implementar eventuais alterações, de modo a manter o registro de todas as modificações realizadas no banco de dados.\textsuperscript{\citeproc{ref-Baillie2022}{134}}
\item
  Use os metadados do estudo para guiar a análise inicial dos dados e compartilhe com os dados para maior transparência e reprodutibilidade.\textsuperscript{\citeproc{ref-Baillie2022}{134}}
\item
  Representação gráfica dos dados pode ajudar a identificar características e padrões no banco de dados, tais como suposições e tendências.\textsuperscript{\citeproc{ref-Baillie2022}{134}}
\item
  Verifique a frequência e proporção de dados perdidos em cada variável, e depois examine por padrões de dados perdidos simultaneamente por duas ou mais variáveis.\textsuperscript{\citeproc{ref-Baillie2022}{134}}
\item
  Verifique a frequência e proporção de dados perdidos em cada variável, e depois examine por padrões de dados perdidos simultaneamente por duas ou mais variáveis.\textsuperscript{\citeproc{ref-Baillie2022}{134}}
\item
  Exclusão de dados \emph{ad hoc} baseada no desfecho pode influenciar os resultados do estudo, portanto os critérios de exclusão de dados antes da análise estatística (descritiva e/ou inferencial) devem ser reportados.\textsuperscript{\citeproc{ref-Landis2012}{193}}
\end{itemize}

\subsection{Quais problemas podem ser detectados na análise inicial de dados?}\label{quais-problemas-podem-ser-detectados-na-anuxe1lise-inicial-de-dados}

\begin{itemize}
\tightlist
\item
  Ocorrência de dados perdidos, que podem ser excluídos ou imputados para não reduzir o poder do estudo.\textsuperscript{\citeproc{ref-REF}{\textbf{REF?}}}
\end{itemize}

\begin{infobox}{images/Rlogo}
O pacote \emph{stats}\textsuperscript{\citeproc{ref-stats}{131}} fornece a função \href{https://www.rdocumentation.org/packages/stats/versions/3.6.2/topics/na.fail}{\emph{na.omit}} para retornar os dados sem os dados perdidos.

\end{infobox}

\begin{infobox}{images/Rlogo}
O pacote \emph{stats}\textsuperscript{\citeproc{ref-stats}{131}} fornece a função \href{https://www.rdocumentation.org/packages/stats/versions/3.6.2/topics/complete.cases}{\emph{complete.cases}} para identificar os casos completos --- isto é, sem dados perdidos --- em um banco de dados.

\end{infobox}

\begin{itemize}
\tightlist
\item
  Registros duplicados, que devem ser excluídos para não inflar a amostra.\textsuperscript{\citeproc{ref-huebner2016}{194}}
\end{itemize}

\begin{infobox}{images/Rlogo}
O pacote \emph{base}\textsuperscript{\citeproc{ref-base}{55}} fornece a função \href{https://www.rdocumentation.org/packages/base/versions/3.6.2/topics/duplicated}{\emph{duplicated}} para identificar elementos duplcados de um banco de dados.

\end{infobox}

\begin{itemize}
\item
  Codificação 0 ou 1 para variáveis dicotômicas para representar a direção esperada da associação entre elas.\textsuperscript{\citeproc{ref-huebner2016}{194}}
\item
  Ordenação cronológica de variáveis com registros temporais (retrospectivos ou prospectivos).\textsuperscript{\citeproc{ref-huebner2016}{194}}
\item
  A distribuição das variáveis para verificação das suposições das análises planejadas.\textsuperscript{\citeproc{ref-huebner2016}{194}}
\item
  Ocorrência de efeitos teto e piso nas variáveis.\textsuperscript{\citeproc{ref-huebner2016}{194}}
\end{itemize}

\chapter{\texorpdfstring{\textbf{Análise exploratória de dados}}{Análise exploratória de dados}}\label{analise-exploratoria}

\section{Análise exploratória de dados}\label{anuxe1lise-exploratuxf3ria-de-dados}

\subsection{O que é análise exploratória de dados?}\label{o-que-uxe9-anuxe1lise-exploratuxf3ria-de-dados}

\begin{itemize}
\item
  Análise exploratória de dados consiste em um processo iterativo de elaboração e interpretação da síntese de dados, tabelas e gráficos, considerando os aspectos teóricos do estudo.\textsuperscript{\citeproc{ref-Ferketich1986}{192}}
\item
  Análise exploratória deve ser separada da análise inferencial de testes de hipóteses; a decisão sobre os modelos a testar deve ser feita \emph{a priori}.\textsuperscript{\citeproc{ref-zuur2009}{195}}
\end{itemize}

\subsection{Quais são os objetivos centrais da análise exploratória de dados?}\label{quais-suxe3o-os-objetivos-centrais-da-anuxe1lise-exploratuxf3ria-de-dados}

\begin{itemize}
\tightlist
\item
  A análise exploratória de dados (EDA) tem dois objetivos principais: (a) descrição dos dados e (b) formulação de modelos. A descrição envolve resumir os dados e destacar características essenciais, enquanto a formulação de modelos auxilia na geração de hipóteses e na escolha de procedimentos estatísticos adequados.\textsuperscript{\citeproc{ref-chatfield1986}{191}}
\end{itemize}

\subsection{Por que conduzir a análise exploratória de dados?}\label{por-que-conduzir-a-anuxe1lise-exploratuxf3ria-de-dados}

\begin{itemize}
\item
  A condução de análise exploratória de dados pode ajudar a identificar padrões e pode orientar trabalhos futuros, mas os resultados não devem ser interpretados como inferências sobre uma população.\textsuperscript{\citeproc{ref-zuur2009}{195}}
\item
  A análise exploratória não deve ser usada para definir as questões e hipóteses científicas do estudo.\textsuperscript{\citeproc{ref-zuur2009}{195}}
\end{itemize}

\begin{infobox}{images/Rlogo}
O pacote \emph{explore}\textsuperscript{\citeproc{ref-explore}{196}} fornece a função \href{https://www.rdocumentation.org/packages/explore/versions/1.0.2/topics/explore}{\emph{explore}} para análise exploratória de um banco de dados.

\end{infobox}

\begin{infobox}{images/Rlogo}
O pacote \emph{dataMaid}\textsuperscript{\citeproc{ref-dataMaid}{197}} fornece a função \href{https://www.rdocumentation.org/packages/dataMaid/versions/1.4.1/topics/makeDataReport}{\emph{makeDataReport}} para criar um relatório de análise exploratória de um banco de dados.

\end{infobox}

\begin{infobox}{images/Rlogo}
O pacote \emph{DataExplorer}\textsuperscript{\citeproc{ref-DataExplorer}{198}} fornece a função \href{https://www.rdocumentation.org/packages/DataExplorer/versions/0.8.2/topics/create_report}{\emph{create\_report}} para criar um relatório de análise exploratória de um banco de dados.

\end{infobox}

\begin{infobox}{images/Rlogo}
O pacote \emph{SmartEDA}\textsuperscript{\citeproc{ref-SmartEDA}{199}} fornece a função \href{https://www.rdocumentation.org/packages/SmartEDA/versions/0.3.9/topics/ExpReport}{\emph{ExpReport}} para criar um relatório de análise exploratória de um banco de dados.

\end{infobox}

\begin{infobox}{images/Rlogo}
O pacote \emph{gtExtras}\textsuperscript{\citeproc{ref-gtExtras}{200}} fornece a função \href{https://www.rdocumentation.org/packages/gtExtras/versions/0.5.0/topics/gt_plt_summary}{\emph{gt\_plt\_summary}} para criar uma tabela descritiva síntese com histogramas ou gráficos de barra a partir de um banco de dados.

\end{infobox}

\begin{infobox}{images/Rlogo}
O pacote \emph{radiant}\textsuperscript{\citeproc{ref-radiant}{201}} fornece a função \href{https://www.rdocumentation.org/packages/radiant/versions/1.5.0/topics/radiant}{\emph{radiant}} para executar uma interface interativa para análise exploratória de dados.

\end{infobox}

\section{Ingredientes da análise exploratória de dados}\label{ingredientes-da-anuxe1lise-exploratuxf3ria-de-dados}

\subsection{Quais são os principais elementos que compõem a análise exploratória de dados?}\label{quais-suxe3o-os-principais-elementos-que-compuxf5em-a-anuxe1lise-exploratuxf3ria-de-dados}

\begin{itemize}
\item
  Verificação da qualidade dos dados (erros, ausências, \emph{outliers}), o cálculo de estatísticas descritivas (média, desvio-padrão, intervalos, correlações) e o uso de representações gráficas como histogramas, diagramas de dispersão, \emph{boxplots} e gráficos de séries temporais.\textsuperscript{\citeproc{ref-chatfield1986}{191}}
\item
  Técnicas multivariadas exploratórias, como análise de componentes principais e análise de clusters, podem revelar padrões em dados complexos.\textsuperscript{\citeproc{ref-chatfield1986}{191}}
\end{itemize}

\subsection{Quais etapas constituem a análise exploratória de dados?}\label{quais-etapas-constituem-a-anuxe1lise-exploratuxf3ria-de-dados}

\begin{itemize}
\item
  Cada combinação de problema de pesquisa e delineamento de estudo pode demandar um plano de análise exploratório distinto.\textsuperscript{\citeproc{ref-zuur2009}{195}}
\item
  Verifique a existência e/ou influência de valores discrepantes (``fora da curva'' ou \emph{outliers}):\textsuperscript{\citeproc{ref-chatfield1986}{191},\citeproc{ref-Ferketich1986}{192},\citeproc{ref-zuur2009}{195}}

  \begin{itemize}
  \item
    Boxplots
  \item
    Gráficos quantil-quantil (Q-Q)
  \end{itemize}
\item
  A análise exploratória valoriza o uso de gráficos interativos e técnicas de \emph{brushing} e \emph{linking}, que permitem explorar padrões ocultos, relacionar múltiplas variáveis e destacar subconjuntos de observações.\textsuperscript{\citeproc{ref-behrens1997}{202}}
\end{itemize}

\begin{infobox}{images/Rlogo}
O pacote \emph{ggplot2}\textsuperscript{\citeproc{ref-ggplot2}{173}} fornece a função \href{https://www.rdocumentation.org/packages/ggplot2/versions/3.5.2/topics/geom_boxplot}{\emph{geom\_boxplot}} para construção de gráficos \emph{boxplot}.

\end{infobox}

\begin{itemize}
\item
  Verifique a homocedasticidade (homogeneidade da variância):\textsuperscript{\citeproc{ref-zuur2009}{195}}

  \begin{itemize}
  \item
    Boxplots condicionais (por fator de análise)
  \item
    Análise dos resíduos do modelo de regressão
  \item
    Gráfico resíduos vs.~valores ajustados
  \end{itemize}
\end{itemize}

\begin{itemize}
\item
  Verifique a normalidade da distribuição dos dados:\textsuperscript{\citeproc{ref-chatfield1986}{191},\citeproc{ref-zuur2009}{195}}

  \begin{itemize}
  \item
    Histograma das variáveis (por fator de análise)
  \item
    Histograma dos resíduos da regressão
  \end{itemize}
\end{itemize}

\begin{itemize}
\item
  Verifique a existência de grande quantidade de valores nulos (=0):\textsuperscript{\citeproc{ref-zuur2009}{195}}

  \begin{itemize}
  \tightlist
  \item
    Histograma das variáveis (por fator de análise)
  \end{itemize}
\end{itemize}

\begin{itemize}
\item
  Verifique a existência de colinearidade entre variáveis independentes de um modelo de regressão:\textsuperscript{\citeproc{ref-zuur2009}{195}}

  \begin{itemize}
  \item
    Fator de inflação de variância (\emph{variance inflation factor}, VIF)
  \item
    Coeficiente de correlação de Pearson (\(r\))
  \item
    Gráfico de dispersão entre variáveis
  \end{itemize}
\end{itemize}

\begin{itemize}
\item
  Verifique possíveis relações entre as variáveis dependente(s) e independente(s) de um modelo de regressão:\textsuperscript{\citeproc{ref-zuur2009}{195}}

  \begin{itemize}
  \tightlist
  \item
    Gráfico de dispersão entre variáveis independente e dependente
  \end{itemize}
\end{itemize}

\begin{itemize}
\item
  Verifique possíveis interações entre as variáveis dependente(s) de um modelo de regressão:\textsuperscript{\citeproc{ref-zuur2009}{195}}

  \begin{itemize}
  \tightlist
  \item
    Gráfico \emph{coplot} de dispersão entre variáveis dependentes
  \end{itemize}
\end{itemize}

\begin{infobox}{images/Rlogo}
O pacote \emph{ggcleveland}\textsuperscript{\citeproc{ref-ggcleveland}{203}} fornece a função \href{https://www.rdocumentation.org/packages/ggcleveland/versions/0.1.0/topics/gg_coplot}{\emph{gg\_coplot}} para construção de gráficos \emph{boxplot} condicionais.

\end{infobox}

\begin{itemize}
\item
  Verifique por dependência entre variáveis de um modelo de regressão:\textsuperscript{\citeproc{ref-zuur2009}{195}}

  \begin{itemize}
  \item
    Gráfico de série temporal das variáveis
  \item
    Gráfico de autocorrelação entre as variáveis
  \end{itemize}
\end{itemize}

\begin{figure}

{\centering \includegraphics[width=0.8\linewidth]{Ciencia-com-R_files/figure-latex/autocorrelacao-1} 

}

\caption{Séries temporais e autocorrelação de duas séries simuladas com fraca e forte autocorrelação.}\label{fig:autocorrelacao}
\end{figure}

\begin{itemize}
\item
  Medidas como mediana, \emph{trimean}, distância absoluta mediana e procedimentos de \emph{winsorizing} ou \emph{trimming} são preferidos, pois reduzem a influência de valores extremos e oferecem resumos mais fiéis.\textsuperscript{\citeproc{ref-behrens1997}{202}}
\item
  A análise exploratória adota o esquema \texttt{dados\ =\ ajuste\ +\ resíduo}, no qual o analista ajusta modelos provisórios, examina resíduos e refina os modelos em ciclos sucessivos de aproximação.\textsuperscript{\citeproc{ref-behrens1997}{202}}
\item
  Valores discrepantes (\emph{outliers}) não devem ser ignorados; eles podem indicar erros de coleta ou fenômenos relevantes. \emph{Fringeliers}, casos menos extremos mas recorrentes, também merecem atenção.\textsuperscript{\citeproc{ref-behrens1997}{202}}
\item
  Transformar variáveis em novas formas (por exemplo, log ou inverso) pode revelar simetrias ocultas e tornar relações mais claras e lineares.\textsuperscript{\citeproc{ref-behrens1997}{202}}
\end{itemize}

\chapter{\texorpdfstring{\textbf{Análise descritiva}}{Análise descritiva}}\label{analise-descritiva}

\section{Análise descritiva}\label{anuxe1lise-descritiva}

\subsection{O que é análise descritiva?}\label{o-que-uxe9-anuxe1lise-descritiva}

\begin{itemize}
\tightlist
\item
  Análise descritiva é usada para compreendermos algum aspecto de um conjunto de dados, respondendo a perguntas do tipo ``quando?'', ``onde?'', ``quem?'', ``o quê?'', ``como?'' e ``e daí?''.\textsuperscript{\citeproc{ref-vetter2017}{108},\citeproc{ref-gerring2012}{204}}
\end{itemize}

\begin{infobox}{images/Rlogo}
O pacote \emph{base}\textsuperscript{\citeproc{ref-base}{55}} fornece a função \href{https://www.rdocumentation.org/packages/base/versions/3.6.2/topics/summary}{\emph{summary}} para calcular diversos parâmetros descritivos.

\end{infobox}

\subsection{Como apresentar os resultados descritivos?}\label{como-apresentar-os-resultados-descritivos}

\begin{itemize}
\item
  Variáveis categóricas: Reporte valores de frequência absoluta e relativa (n, percentual).\textsuperscript{\citeproc{ref-Cummings2003}{205}}
\item
  Organização das tabelas: as variáveis são exibidas em linhas e os grupos são exibidos em colunas.\textsuperscript{\citeproc{ref-Cummings2003}{205}}
\item
  Calcule percentagens para as colunas (isto é, entre grupos) e não entre linhas.\textsuperscript{\citeproc{ref-Cummings2003}{205}}
\item
  Em caso de dados perdidos, não inclua uma linha com total de dados perdidos, pois distorce as proporções entre colunas e as análises de tabela de contingência. Indique no texto ou em uma coluna separada o total de dados perdidos por variável.\textsuperscript{\citeproc{ref-Cummings2003}{205}}
\end{itemize}

\section{Apresentação de resultados numéricos}\label{apresentauxe7uxe3o-de-resultados-numuxe9ricos}

\subsection{O que são casas decimais?}\label{o-que-suxe3o-casas-decimais}

\begin{itemize}
\item
  O número de casas decimais refere-se à quantidade de dígitos que aparecem após a vírgula decimal.\textsuperscript{\citeproc{ref-Cole2015a}{206},\citeproc{ref-cole2015b}{207}}
\item
  Para tamanhos de efeito: use 2--3 dígitos significativos.\textsuperscript{\citeproc{ref-Weissgerber2019}{208}}
\item
  Para medidas de variabilidade (desvio-padrão/erro-padrão/intervalo de confian';cça): use 1--2 dígitos significativos.\textsuperscript{\citeproc{ref-Weissgerber2019}{208}}
\end{itemize}

\subsection{O que são dígitos significativos?}\label{o-que-suxe3o-duxedgitos-significativos}

\begin{itemize}
\item
  O termo ``dígitos significativos'' é preferido a ``algarismos significativos'' ou ``dígitos efetivos'' e não se relaciona com significância estatística.\textsuperscript{\citeproc{ref-Cole2015a}{206},\citeproc{ref-cole2015b}{207}}
\item
  O número de dígitos significativos é a soma total de dígitos, desconsiderando a vírgula decimal e os zeros à esquerda; os zeros à direita são considerados informativos, salvo exceções.\textsuperscript{\citeproc{ref-Cole2015a}{206},\citeproc{ref-cole2015b}{207}}
\end{itemize}

\subsection{Como arredondar dados numéricos?}\label{como-arredondar-dados-numuxe9ricos}

\begin{itemize}
\item
  Apresentar dados com quantidade excessiva de casas decimais pode dificultar a interpretação e induzir erroneamente uma precisão espúria.\textsuperscript{\citeproc{ref-Cole2015a}{206},\citeproc{ref-cole2015b}{207}}
\item
  A precisão é determinada pelo grau de arredondamento aplicado, medido em casas decimais ou dígitos significativos.\textsuperscript{\citeproc{ref-Cole2015a}{206},\citeproc{ref-cole2015b}{207}}
\end{itemize}

\begin{table}
\centering
\caption{\label{tab:casas-decimais}Quantidade de casas decimais e dígitos significativos.}
\centering
\begin{tabu} to \linewidth {>{}r>{\centering}X>{\centering}X}
\toprule
\textbf{Valor} & \textbf{Casas Decimais} & \textbf{Dígitos Significativos}\\
\midrule
\textbf{0,00789} & 5 & 0\\
\textbf{0,0456} & 4 & 0\\
\textbf{45,6} & 1 & 2\\
\textbf{123,456} & 3 & 3\\
\textbf{7890,0000} & 4 & 4\\
\bottomrule
\end{tabu}
\end{table}

\begin{itemize}
\tightlist
\item
  O arredondamento também introduz erros, uma vez que aumenta a imprecisão (isto é, incerteza) em torno do valor original.\textsuperscript{\citeproc{ref-Cole2015a}{206},\citeproc{ref-cole2015b}{207}}
\end{itemize}

\begin{table}
\centering
\caption{\label{tab:arredondamento}Valores originais, arredondamentos e erros de arredondamento por casas decimais.}
\centering
\begin{tabu} to \linewidth {>{}r>{\centering}X>{\centering}X>{\raggedleft}X>{\raggedleft}X>{\raggedleft}X}
\toprule
\textbf{Valor} & \textbf{Casas Decimais} & \textbf{Dígitos Significativos} & \textbf{2 Casas decimais [Margem de erro]} & \textbf{1 Casa decimal [Margem de erro]} & \textbf{Sem casa decimal [Margem de erro]}\\
\midrule
\textbf{0,00789} & 5 & 0 & 0,01 [0,005, 0,015] & 0,0 [-0,05, 0,05] & 0 [-0,5, 0,5]\\
\textbf{0,0456} & 4 & 0 & 0,05 [0,045, 0,055] & 0,0 [-0,05, 0,05] & 0 [-0,5, 0,5]\\
\textbf{45,6} & 1 & 2 & 45,60 [45,595, 45,605] & 45,6 [45,55, 45,65] & 46 [45,5, 46,5]\\
\textbf{123,456} & 3 & 3 & 123,46 [123,455, 123,465] & 123,5 [123,45, 123,55] & 123 [122,5, 123,5]\\
\textbf{7890,0000} & 4 & 4 & 7890,00 [7889,995, 7890,005] & 7890,0 [7889,95, 7890,05] & 7890 [7889,5, 7890,5]\\
\bottomrule
\end{tabu}
\end{table}

\begin{itemize}
\item
  A regra geral é utilizar 2 ou 3 dígitos significativos para tamanhos de efeito e 1 ou 2 dígitos significativos para medidas de variabilidade.\textsuperscript{\citeproc{ref-cole2015b}{207}}
\item
  Regra dos 3 dígitos significativos para proporção de risco: em média, o erro de arredondamento é menor que os 0,5\% exigidos, de modo que três dígitos significativos são mais precisos do que o necessário.\textsuperscript{\citeproc{ref-Cole2015a}{206}}
\item
  Regra dos 4 dígitos significativos para proporção de risco: divida a proporção de risco por quatro e arredonde para dois dígitos significativos e, em seguida, relate a proporção para esse número de casas decimais.\textsuperscript{\citeproc{ref-Cole2015a}{206}}
\end{itemize}

\section{Tabelas}\label{tabelas}

\subsection{Por que usar tabelas?}\label{por-que-usar-tabelas}

\begin{itemize}
\tightlist
\item
  Tabelas complementam o texto (e vice-versa), e podem apresentar os dados de modo mais acessível e informativo.\textsuperscript{\citeproc{ref-Inskip2017}{209}}
\end{itemize}

\subsection{Que informações incluir nas tabelas?}\label{que-informauxe7uxf5es-incluir-nas-tabelas}

\begin{itemize}
\tightlist
\item
  Título ou legenda, uma síntese descritiva (geralmente por meio de parâmetros descritivos), intervalos de confiança e/ou P-valores conforme necessário para adequada interpretação.\textsuperscript{\citeproc{ref-Inskip2017}{209},\citeproc{ref-Kwak2021}{210}}
\end{itemize}

\subsection{Quais são os tipos de tabelas?}\label{quais-suxe3o-os-tipos-de-tabelas}

\begin{itemize}
\item
  Tabela de frequência: apresenta a quantidade de ocorrências (frequência absoluta e relativa) de cada categoria de uma variável; usada com variáveis qualitativas ou quantitativas discretas.\textsuperscript{\citeproc{ref-REF}{\textbf{REF?}}}
\item
  Tabela de frequência agrupada: organiza dados contínuos em intervalos de classe (ex: faixas etárias) e mostra as frequências correspondentes.\textsuperscript{\citeproc{ref-REF}{\textbf{REF?}}}
\item
  Tabela de contingência (ou tabela cruzada): cruza duas variáveis categóricas, permitindo observar possíveis associações entre elas.\textsuperscript{\citeproc{ref-REF}{\textbf{REF?}}}
\item
  Tabela de medidas descritivas: resume variáveis quantitativas com estatísticas como média, mediana, desvio-padrão, mínimo, máximo e quartis.\textsuperscript{\citeproc{ref-REF}{\textbf{REF?}}}
\item
  Tabela de comparação entre grupos: apresenta médias, desvios-padrão e ocasionalmente resultados de testes de inferência estatística para comparar dois ou mais grupos.\textsuperscript{\citeproc{ref-REF}{\textbf{REF?}}}
\item
  Tabela de resultados de testes estatísticos: exibe valores de estatísticas de teste , P valores e intervalos de confiança; usada para mostrar inferências.\textsuperscript{\citeproc{ref-REF}{\textbf{REF?}}}
\item
  Tabela de regressão (ou de modelos estatísticos): mostra os coeficientes de regressão, erros padrão, intervalos de confiança e P valores para cada variável de um modelo.\textsuperscript{\citeproc{ref-REF}{\textbf{REF?}}}
\item
  Tabela de séries temporais ou longitudinais: organiza dados medidos em diferentes momentos no tempo, permitindo visualizar tendências ou variações longitudinais.\textsuperscript{\citeproc{ref-REF}{\textbf{REF?}}}
\end{itemize}

\begin{infobox}{images/Rlogo}
O pacote \emph{gtsummary}\textsuperscript{\citeproc{ref-gtsummary}{211}} fornece a função \href{https://search.r-project.org/CRAN/refmans/gtsummary/html/tbl_summary.html}{\emph{tbl\_summary}} para construção da `Tabela 1' com dados descritivos.

\end{infobox}

\begin{infobox}{images/Rlogo}
O pacote \emph{table1}\textsuperscript{\citeproc{ref-table1}{212}} fornece a função \href{https://search.r-project.org/CRAN/refmans/table1/html/table1.html}{\emph{table1}} para construção de tabelas.

\end{infobox}

\begin{infobox}{images/Rlogo}
O pacote \emph{flextable}\textsuperscript{\citeproc{ref-flextable}{213}} fornece as funções \href{https://search.r-project.org/CRAN/refmans/flextable/html/flextable.html}{\emph{flextable}}, \href{https://search.r-project.org/CRAN/refmans/flextable/html/as_flextable.html}{\emph{as\_flextable}} e \href{https://search.r-project.org/CRAN/refmans/flextable/html/save_as_docx.html}{\emph{save\_as\_docx}} para criar e salvar tabelas formatadas em DOCX.

\end{infobox}

\begin{infobox}{images/Rlogo}
O pacote \emph{rempsyc}\textsuperscript{\citeproc{ref-rempsyc}{214}} fornece a função \href{https://search.r-project.org/CRAN/refmans/rempsyc/html/nice_table.html}{\emph{nice\_table}} para criar tabelas formatadas.

\end{infobox}

\subsection{Quais são os erros mais comuns de preenchimento de tabelas?}\label{quais-suxe3o-os-erros-mais-comuns-de-preenchimento-de-tabelas}

\begin{itemize}
\item
  Erros tipográficos.\textsuperscript{\citeproc{ref-barnett2023}{215}}
\item
  Ausência de rótulos ou unidades nas variáveis.\textsuperscript{\citeproc{ref-barnett2023}{215}}
\item
  Relatar estatísticas incorretamente, tais como rotular variáveis contínuas como porcentagens.\textsuperscript{\citeproc{ref-barnett2023}{215}}
\item
  Estatísticas descritivas de tendência central (ex.: médias) relatadas sem a estatística de dispersão correspondente (ex.: desvio-padrão).\textsuperscript{\citeproc{ref-barnett2023}{215}}
\item
  Desvio-padrão nulo (\(\sigma=0\)).\textsuperscript{\citeproc{ref-barnett2023}{215}}
\item
  Valores porcentuais que não correspondem ao numerador dividido pelo denominador.\textsuperscript{\citeproc{ref-barnett2023}{215}}
\end{itemize}

\section{Tabela 1}\label{tabela-1}

\subsection{O que é a `Tabela 1'?}\label{o-que-uxe9-a-tabela-1}

\begin{itemize}
\tightlist
\item
  A `Tabela 1' descreve as características demográficas, sociais e clínicas da amostra, completa ou agrupada por algum fator, geralmente por meio de parâmetros de tendência central e dispersão.\textsuperscript{\citeproc{ref-Westreich2013}{216},\citeproc{ref-chen2020}{217}}
\end{itemize}

\subsection{Qual a utilidade da `Tabela 1'?}\label{qual-a-utilidade-da-tabela-1}

\begin{itemize}
\item
  Descrever (conhecer) as características da amostra e dos grupos sendo comparados, quando aplicável.\textsuperscript{\citeproc{ref-chen2020}{217}}
\item
  Verificar aderência ao protocolo do estudo, incluindo critérios de inclusão/exclusão, tamanho da amostra e perdas amostrais.\textsuperscript{\citeproc{ref-chen2020}{217}}
\item
  Permitir a replicação do estudo.\textsuperscript{\citeproc{ref-chen2020}{217}}
\item
  Meta-analisar os dados junto a estudos similares.\textsuperscript{\citeproc{ref-chen2020}{217}}
\item
  Avaliar a generalização (validade externa) das conclusões do estudo.\textsuperscript{\citeproc{ref-chen2020}{217}}
\end{itemize}

\subsection{O que é a falácia da `Tabela 1'?}\label{o-que-uxe9-a-faluxe1cia-da-tabela-1}

\begin{itemize}
\item
  Falácia da Tabela 1 ocorre pela interpretação errônea dos P-valores na comparação entre grupos, na linha de base, de um ensaio clínico aleatorizado.\textsuperscript{\citeproc{ref-pijls2022}{218}}
\item
  Não interprete P da linha de base em ensaios clínicos como ``desequilíbrio'' (falácia da Tabela 1). Mantenha P-valor apenas como descritivo (ou omita), enfatizando desenho e aleatorização.\textsuperscript{\citeproc{ref-Weissgerber2019}{208}}
\end{itemize}

\subsection{Como construir a `Tabela 1'?}\label{como-construir-a-tabela-1}

\begin{itemize}
\tightlist
\item
  A Tabela 1 geralmente é utilizada para descrever as características da amostra estudada, possibilitando a análise de ameaças à validade interna e/ou externa ao estudo.\textsuperscript{\citeproc{ref-greenhalgh1997}{177},\citeproc{ref-Hayes-Larson2019}{219}}
\end{itemize}

\begin{infobox}{images/Rlogo}
O pacote \emph{table1}\textsuperscript{\citeproc{ref-table1}{212}} fornece a função \href{https://search.r-project.org/CRAN/refmans/table1/html/table1.html}{\emph{table1}} para construção de tabelas.

\end{infobox}

\begin{infobox}{images/Rlogo}
O pacote \emph{gtsummary}\textsuperscript{\citeproc{ref-gtsummary}{211}} fornece a função \href{https://search.r-project.org/CRAN/refmans/gtsummary/html/tbl_summary.html}{\emph{tbl\_summary}} para construção da `Tabela 1' com dados descritivos.

\end{infobox}

\begin{table}[t]
\caption{\label{tab:tabela-1}Características da amostra por grupo.} 
\fontsize{12.0pt}{14.0pt}\selectfont
\begin{tabular*}{\linewidth}{@{\extracolsep{\fill}}lcccc}
\toprule
\textbf{Características} & \textbf{N} & \textbf{Controle}  N = 103\textsuperscript{\textit{1}} & \textbf{Intervenção}  N = 97\textsuperscript{\textit{1}} & \textbf{Valor-p}\textsuperscript{\textit{2}} \\ 
\midrule\addlinespace[2.5pt]
Sexo & 200 &  &  & 0.060 \\ 
    F &  & 49 (48\%) & 59 (61\%) &  \\ 
    M &  & 54 (52\%) & 38 (39\%) &  \\ 
Idade & 200 &  &  & 0.8 \\ 
    Média (Desvio Padrão) &  & 61 (12) & 60 (12) &  \\ 
    Mediana [Q1, Q3] &  & 61 [53, 69] & 60 [53, 69] &  \\ 
IMC & 200 &  &  & 0.2 \\ 
    Média (Desvio Padrão) &  & 26.8 (3.7) & 27.5 (4.0) &  \\ 
    Mediana [Q1, Q3] &  & 26.6 [24.5, 29.7] & 27.6 [25.6, 29.9] &  \\ 
\bottomrule
\end{tabular*}
\begin{minipage}{\linewidth}
\vspace{.05em}
\textsuperscript{\textit{1}} n (\%)\\
\textsuperscript{\textit{2}} Teste qui-quadrado de independência; Teste de soma de postos de Wilcoxon\\
\end{minipage}
\end{table}

\section{Tabela 2}\label{tabela-2}

\subsection{Qual a utilidade da `Tabela 2'?}\label{qual-a-utilidade-da-tabela-2}

\begin{itemize}
\tightlist
\item
  A Tabela 2 mostra associações ajustadas multivariadas com o resultado para variáveis resumidas na Tabela 1.\textsuperscript{\citeproc{ref-Westreich2013}{216}}
\end{itemize}

\subsection{O que é a falácia da `Tabela 2'?}\label{o-que-uxe9-a-faluxe1cia-da-tabela-2}

\begin{itemize}
\item
  A Tabela 2 pode induzir ao erro de interpretação pelas estimativas de efeitos para covariáveis do modelo também serem utilizados para controlar a confusão da exposição.\textsuperscript{\citeproc{ref-Westreich2013}{216},\citeproc{ref-bandoli2018}{220}}
\item
  Ao apresentar estimativas de efeito ajustadas para covariáveis juntamente com a estimativa de efeito ajustada para a exposição primária, a Tabela 2 sugere implicitamente que todas estas estimativas podem ser interpretadas de forma semelhante, se não de forma idêntica, como estimativa do efeito total.\textsuperscript{\citeproc{ref-Westreich2013}{216},\citeproc{ref-bandoli2018}{220}}
\item
  A falácia da Tabela 2 pode ser evitada limitando-se a tabela a estimativas das medidas primárias do efeito de exposição nos diferentes modelos, com as covariáveis secundárias de ``ajuste'' relatadas em uma nota de rodapé, juntamente com a forma como foram categorizadas ou modeladas.\textsuperscript{\citeproc{ref-Westreich2013}{216}}
\end{itemize}

\begin{table}[t]
\caption{\label{tab:falacia-tabela-2}Exemplo clássico de apresentação suscetível à Falácia da `Tabela 2'.} 
\fontsize{12.0pt}{14.0pt}\selectfont
\begin{tabular*}{\linewidth}{@{\extracolsep{\fill}}lcccccc}
\toprule
 & \multicolumn{3}{c}{Sem ajuste} & \multicolumn{3}{c}{Ajustado} \\ 
\cmidrule(lr){2-4} \cmidrule(lr){5-7}
\textbf{Características} & \textbf{OR} & \textbf{95\% IC} & \textbf{Valor-p} & \textbf{OR} & \textbf{95\% IC} & \textbf{Valor-p} \\ 
\midrule\addlinespace[2.5pt]
Grupo &  &  &  &  &  &  \\ 
    Controle & — & — &  & — & — &  \\ 
    Intervenção & 1.71 & 0.98, 3.02 & 0.061 & 1.70 & 0.97, 3.03 & 0.067 \\ 
Idade &  &  &  & 1.02 & 1.00, 1.05 & 0.087 \\ 
IMC &  &  &  & 1.05 & 0.97, 1.13 & 0.2 \\ 
\bottomrule
\end{tabular*}
\begin{minipage}{\linewidth}
\vspace{.05em}
Abreviações: IC = Intervalo de Confiança, OR = Razão de chances\\
\end{minipage}
\end{table}

\subsection{Como construir a `Tabela 2'?}\label{como-construir-a-tabela-2}

\begin{itemize}
\tightlist
\item
  A Tabela 2 pode ser utilizada para apresentar estimativas de múltiplos efeitos ajustados de um mesmo modelo estatístico.\textsuperscript{\citeproc{ref-Westreich2013}{216}}
\end{itemize}

\begin{table}[t]
\caption{\label{tab:tabela-2}Exposição (OR; 95\% IC) com e sem ajuste.} 
\fontsize{12.0pt}{14.0pt}\selectfont
\begin{tabular*}{\linewidth}{@{\extracolsep{\fill}}lcccccc}
\toprule
 & \multicolumn{3}{c}{Sem ajuste} & \multicolumn{3}{c}{Ajustado} \\ 
\cmidrule(lr){2-4} \cmidrule(lr){5-7}
\textbf{Características} & \textbf{OR} & \textbf{95\% IC} & \textbf{Valor-p} & \textbf{OR} & \textbf{95\% IC} & \textbf{Valor-p} \\ 
\midrule\addlinespace[2.5pt]
Grupo &  &  &  &  &  &  \\ 
    Controle & — & — &  & — & — &  \\ 
    Intervenção & 1.71 & 0.98, 3.02 & 0.061 & 1.70 & 0.97, 3.03 & 0.067 \\ 
\bottomrule
\end{tabular*}
\begin{minipage}{\linewidth}
\vspace{.05em}
Abreviações: IC = Intervalo de Confiança, OR = Razão de chances\\
\emph{Nota.} Modelo ajustado por Idade (contínua) e IMC (contínuo).
Covariáveis são usadas apenas para controle de confusão e não devem ser interpretadas como efeitos causais (\emph{Falácia da Tabela 2}).\\
\end{minipage}
\end{table}

\begin{infobox}{images/Rlogo}
O pacote \emph{table1}\textsuperscript{\citeproc{ref-table1}{212}} fornece a função \href{https://search.r-project.org/CRAN/refmans/table1/html/table1.html}{\emph{table1}} para construção de tabelas.

\end{infobox}

\begin{infobox}{images/Rlogo}
O pacote \emph{gtsummary}\textsuperscript{\citeproc{ref-gtsummary}{211}} fornece a função \href{https://search.r-project.org/CRAN/refmans/gtsummary/html/tbl_summary.html}{\emph{tbl\_summary}} para construção da `Tabela 1' com dados descritivos.

\end{infobox}

\section{Visualização efetiva de dados}\label{visualizauxe7uxe3o-efetiva-de-dados}

\subsection{Por que começar pela mensagem antes do gráfico?}\label{por-que-comeuxe7ar-pela-mensagem-antes-do-gruxe1fico}

\begin{itemize}
\tightlist
\item
  A figura deve responder a uma pergunta clara (comparação? tendência? composição?) e isso orienta a escolha do tipo de gráfico, dados e anotações. Esboce a mensagem e a pergunta antes de abrir o software.\textsuperscript{\citeproc{ref-midway2020}{221}}
\end{itemize}

\subsection{Como escolher a geometria e ``mostrar os dados''?}\label{como-escolher-a-geometria-e-mostrar-os-dados}

\begin{itemize}
\tightlist
\item
  Prefira geometrias que revelem distribuição/variabilidade (pontos, \emph{boxplots}, violinos) em vez de médias sozinhas. Sempre que possível, exiba os dados brutos (pontos com \emph{jitter}) junto da estatística-resumo.\textsuperscript{\citeproc{ref-midway2020}{221}}
\end{itemize}

\begin{figure}

{\centering \includegraphics{Ciencia-com-R_files/figure-latex/viz-geom-mostrar-dados-1} 

}

\caption{Exemplo de gráfico que mostra os dados brutos junto com um resumo estatístico (média e dispersão).}\label{fig:viz-geom-mostrar-dados}
\end{figure}

\section{Gráficos}\label{gruxe1ficos}

\subsection{O que são gráficos?}\label{o-que-suxe3o-gruxe1ficos}

\begin{itemize}
\tightlist
\item
  Gráficos são utilizados para apresentar dados (geralmente em grande quantidade) de modo mais intuitivo e fácil de compreender.\textsuperscript{\citeproc{ref-Park2022}{222}}
\end{itemize}

\subsection{O que torna um bom gráfico tão poderoso?}\label{o-que-torna-um-bom-gruxe1fico-tuxe3o-poderoso}

\begin{itemize}
\tightlist
\item
  ``Não há ferramenta estatística tão poderosa quanto um gráfico bem escolhido'': gráficos ajudam a explorar dados, comunicar resultados e suportar decisões de forma clara e rápida.\textsuperscript{\citeproc{ref-vandemeulebroecke2018}{223}}
\end{itemize}

\subsection{Que elementos incluir em gráficos?}\label{que-elementos-incluir-em-gruxe1ficos}

\begin{itemize}
\tightlist
\item
  Título, eixos horizontal e vertical com respectivas unidades, escalas em intervalos representativos das variáveis, legenda com símbolos, síntese descritiva dos valores e respectiva margem de erro, conforme necessário para adequada interpretação.\textsuperscript{\citeproc{ref-Park2022}{222}}
\end{itemize}

\subsection{Para que servem as barras de erro em gráficos?}\label{para-que-servem-as-barras-de-erro-em-gruxe1ficos}

\begin{itemize}
\item
  Barras de erro ajudam ao autor a apresentar as informações que descrevem os dados (por exemplo, em uma análise descritiva) ou sobre as inferências ou conclusões tomadas a partir de dados.\textsuperscript{\citeproc{ref-krzywinski2013}{182},\citeproc{ref-Cumming2007}{183}}
\item
  Barras de erro mais longas representam mais imprecisão (maiores erros), enquanto barras mais curtas representam mais precisão na estimativa.\textsuperscript{\citeproc{ref-Cumming2007}{183}}
\item
  Barras de erro descritivas geralmente apresentam a amplitude (mínimo-máximo) ou desvio-padrão.\textsuperscript{\citeproc{ref-Cumming2007}{183}}
\item
  Barras de erro inferenciais geralmente apresentam o erro-padrão ou intervalo de confiança no nível de significância \(\alpha\) pré-estabelecido.\textsuperscript{\citeproc{ref-krzywinski2013}{182},\citeproc{ref-Cumming2007}{183}}
\item
  Barras de erro com desvio-padrão são úteis para descrever a variabilidade dos dados, enquanto as barras de erro com erro padrão da média são úteis para descrever a precisão do parâmetro estimado (média) e sua relação com o tamanho da amostra.\textsuperscript{\citeproc{ref-krzywinski2013}{182}}
\item
  Barras de erro com intervalo de confiança são úteis para fornecer uma estimativa da incerteza da estimativa do parâmetro populacional.\textsuperscript{\citeproc{ref-krzywinski2013}{182}}
\item
  O comprimento das barras de erro sugere graficamente a imprecisão dos dados do estudo, uma vez que o valor verdadeiro da população pode estar em qualquer nível do intervalo da barra.\textsuperscript{\citeproc{ref-Cumming2007}{183}}
\item
  De modo contraintuitivo, um espaço entre as barras não garante significância, nem a sobreposição a descarta---depende do tipo de barra.\textsuperscript{\citeproc{ref-krzywinski2013}{182}}
\item
  Para amostras pequenas é preferível apresentar os dados brutos, uma vez que as barras de erro não serão muito informativas.\textsuperscript{\citeproc{ref-krzywinski2013}{182}}
\end{itemize}

\begin{figure}

{\centering \includegraphics{Ciencia-com-R_files/figure-latex/barras-erro-simples-1} 

}

\caption{Exemplos de gráficos com barras de erro e dados brutos.}\label{fig:barras-erro-simples}
\end{figure}

\begin{figure}

{\centering \includegraphics{Ciencia-com-R_files/figure-latex/barras-erro-1} 

}

\caption{Exemplos de gráficos com barras de erro e dados brutos em diferentes cenários.}\label{fig:barras-erro}
\end{figure}

\begin{infobox}{images/Rlogo}
Os pacotes \emph{ggplot2}\textsuperscript{\citeproc{ref-ggplot2}{173}}, \emph{plotly}\textsuperscript{\citeproc{ref-plotly}{224}} e \emph{corrplot}\textsuperscript{\citeproc{ref-corrplot}{225}} fornecem diversas funções para construção de gráficos tais como \href{https://www.rdocumentation.org/packages/ggplot2/versions/3.4.3/topics/ggplot}{\emph{ggplot}}, \href{https://www.rdocumentation.org/packages/plotly/versions/4.10.2/topics/plot_ly}{\emph{plot\_ly}} e \href{https://www.rdocumentation.org/packages/corrplot/versions/0.92/topics/corrplot}{\emph{corrplot}} respectivamente.

\end{infobox}

\subsection{Quais são os principais obstáculos para bons gráficos?}\label{quais-suxe3o-os-principais-obstuxe1culos-para-bons-gruxe1ficos}

\begin{itemize}
\tightlist
\item
  Dificuldade técnica, negligência no ensino tradicional e o foco em ``beleza'' sem clareza podem levar a gráficos ruins, mesmo quando bem intencionados.\textsuperscript{\citeproc{ref-vandemeulebroecke2018}{223}}
\end{itemize}

\section{Tipos de gráficos}\label{tipos-de-gruxe1ficos}

\subsection{Quais são os tipos de gráficos para variáveis categóricas?}\label{quais-suxe3o-os-tipos-de-gruxe1ficos-para-variuxe1veis-categuxf3ricas}

\begin{itemize}
\tightlist
\item
  Gráfico de barras: Mais usado para comparar frequências absolutas ou relativas entre categorias.\textsuperscript{\citeproc{ref-REF}{\textbf{REF?}}}
\end{itemize}

\begin{figure}

{\centering \includegraphics{Ciencia-com-R_files/figure-latex/graficos-var-categoricas-1} 

}

\caption{Gráfico de barras simples representando frequências por categoria.}\label{fig:graficos-var-categoricas}
\end{figure}

\begin{itemize}
\tightlist
\item
  Gráfico de barras empilhadas: Útil para comparar proporções entre grupos em mais de uma variável categórica.\textsuperscript{\citeproc{ref-REF}{\textbf{REF?}}}
\end{itemize}

\begin{figure}

{\centering \includegraphics{Ciencia-com-R_files/figure-latex/graficos-var-categoricas-empilhados-1} 

}

\caption{Gráfico de barras empilhadas representando frequências por categoria.}\label{fig:graficos-var-categoricas-empilhados}
\end{figure}

\begin{figure}

{\centering \includegraphics{Ciencia-com-R_files/figure-latex/bar-plot-1} 

}

\caption{Gráficos de barras represetando médias, barras de erro e dados individuais.}\label{fig:bar-plot}
\end{figure}

\subsection{Quais são os tipos de gráficos para variáveis numéricas?}\label{quais-suxe3o-os-tipos-de-gruxe1ficos-para-variuxe1veis-numuxe9ricas}

\begin{itemize}
\tightlist
\item
  Histograma: Distribuição de frequência de uma variável contínua. Mostra a forma da distribuição (simétrica, assimétrica, bimodal).\textsuperscript{\citeproc{ref-REF}{\textbf{REF?}}}
\end{itemize}

\begin{figure}

{\centering \includegraphics{Ciencia-com-R_files/figure-latex/graficos-var-numericas-1} 

}

\caption{Histograma da variável 'valor'.}\label{fig:graficos-var-numericas}
\end{figure}

\begin{itemize}
\tightlist
\item
  Gráfico de densidade: Similar ao histograma, mas mais suave. Útil para avaliar a distribuição.\textsuperscript{\citeproc{ref-REF}{\textbf{REF?}}}
\end{itemize}

\begin{figure}

{\centering \includegraphics{Ciencia-com-R_files/figure-latex/grafico-densidade-1} 

}

\caption{Gráfico de densidade da variável 'valor'.}\label{fig:grafico-densidade}
\end{figure}

\begin{itemize}
\tightlist
\item
  Diagrama de caixa (\emph{boxplot}): Resume mediana, quartis e valores extremos. Excelente para comparar grupos.\textsuperscript{\citeproc{ref-REF}{\textbf{REF?}}}
\end{itemize}

\begin{figure}

{\centering \includegraphics{Ciencia-com-R_files/figure-latex/boxplot-grupos-1} 

}

\caption{Boxplot por grupo.}\label{fig:boxplot-grupos}
\end{figure}

\begin{itemize}
\tightlist
\item
  Gráfico de violino: Combina boxplot e densidade, mostrando a distribuição da variável. Útil para comparar grupos.\textsuperscript{\citeproc{ref-REF}{\textbf{REF?}}}
\end{itemize}

\begin{figure}

{\centering \includegraphics{Ciencia-com-R_files/figure-latex/violin-plot-1} 

}

\caption{Violin plot por grupo.}\label{fig:violin-plot}
\end{figure}

\begin{itemize}
\tightlist
\item
  Gráfico de pontos (\emph{dot plot}): Mostra cada valor individualmente, útil para pequenas amostras e para visualizar a distribuição.\textsuperscript{\citeproc{ref-REF}{\textbf{REF?}}}
\end{itemize}

\begin{figure}

{\centering \includegraphics{Ciencia-com-R_files/figure-latex/grafico-pontos-1} 

}

\caption{Gráfico de pontos da variável 'valor'.}\label{fig:grafico-pontos}
\end{figure}

\subsection{Quais são os tipos de gráficos para relações entre variáveis?}\label{quais-suxe3o-os-tipos-de-gruxe1ficos-para-relauxe7uxf5es-entre-variuxe1veis}

\begin{itemize}
\tightlist
\item
  Gráfico de dispersão (\emph{scatter plot}): Mostra a relação entre duas variáveis quantitativas. Ideal para investigar correlações.\textsuperscript{\citeproc{ref-REF}{\textbf{REF?}}}
\end{itemize}

\begin{figure}

{\centering \includegraphics{Ciencia-com-R_files/figure-latex/graficos-relacoes-1} 

}

\caption{Gráfico de dispersão representando a relação entre duas variáveis.}\label{fig:graficos-relacoes}
\end{figure}

\begin{itemize}
\tightlist
\item
  Gráfico de bolhas (\emph{bubble chart}): Expande o gráfico de dispersão adicionando uma terceira variável (tamanho da bolha).\textsuperscript{\citeproc{ref-REF}{\textbf{REF?}}}
\end{itemize}

\begin{figure}

{\centering \includegraphics{Ciencia-com-R_files/figure-latex/graficos-bolhas-1} 

}

\caption{Gráfico de bolhas representando a relação entre três variáveis.}\label{fig:graficos-bolhas}
\end{figure}

\begin{itemize}
\tightlist
\item
  Gr;afico Sankey: Visualiza fluxos entre categorias em diferentes etapas ou grupos. Útil para mostrar proporções e transições.\textsuperscript{\citeproc{ref-REF}{\textbf{REF?}}}
\end{itemize}

\begin{figure}

{\centering \includegraphics{Ciencia-com-R_files/figure-latex/sankey-plot-1} 

}

\caption{Sankey plot representando fluxos entre categorias.}\label{fig:sankey-plot}
\end{figure}

\begin{itemize}
\tightlist
\item
  Grágfico de \emph{parcats}: Mostra relações entre múltiplas variáveis categóricas em paralelo. Útil para visualizar fluxos e proporções.\textsuperscript{\citeproc{ref-REF}{\textbf{REF?}}}
\end{itemize}

\begin{figure}

{\centering \includegraphics{Ciencia-com-R_files/figure-latex/parcats-plot-1} 

}

\caption{Gráfico de categorias paralelas (parcats) representando transições entre categorias ao longo do tempo.}\label{fig:parcats-plot}
\end{figure}

\begin{itemize}
\tightlist
\item
  Gráfico de \emph{parts}: Mostra a composição percentual de uma variável categórica. Útil para visualizar proporções.\textsuperscript{\citeproc{ref-REF}{\textbf{REF?}}}
\end{itemize}

\begin{figure}

{\centering \includegraphics{Ciencia-com-R_files/figure-latex/graficos-pares-1} 

}

\caption{Gráfico de pares representando correlações entre múltiplas variáveis.}\label{fig:graficos-pares}
\end{figure}

\subsection{Quais são os tipos de gráficos para dados longitudinais?}\label{quais-suxe3o-os-tipos-de-gruxe1ficos-para-dados-longitudinais}

\begin{itemize}
\tightlist
\item
  Gráfico de \emph{spaghetti}: Mostra trajetórias individuais ao longo do tempo, útil para dados longitudinais.\textsuperscript{\citeproc{ref-REF}{\textbf{REF?}}}
\end{itemize}

\begin{figure}

{\centering \includegraphics{Ciencia-com-R_files/figure-latex/graficos-longitudinais-1} 

}

\caption{Gráfico spaghetti representando dados longitudinais.}\label{fig:graficos-longitudinais}
\end{figure}

\subsection{Quais são os tipos de gráficos para séries temporais?}\label{quais-suxe3o-os-tipos-de-gruxe1ficos-para-suxe9ries-temporais}

\begin{itemize}
\tightlist
\item
  Gráfico de linhas: Mostra a evolução de uma variável ao longo do tempo, com pontos conectados por linhas.\textsuperscript{\citeproc{ref-REF}{\textbf{REF?}}}
\end{itemize}

\begin{figure}

{\centering \includegraphics{Ciencia-com-R_files/figure-latex/grafico-linha-1} 

}

\caption{Gráfico de linha representando uma série temporal.}\label{fig:grafico-linha}
\end{figure}

\subsection{Quais são os tipos de gráficos para dados multivariados?}\label{quais-suxe3o-os-tipos-de-gruxe1ficos-para-dados-multivariados}

\begin{itemize}
\tightlist
\item
  Gráfico de dispersão: Representa a relação entre duas variáveis, com pontos e uma linha de tendência.\textsuperscript{\citeproc{ref-REF}{\textbf{REF?}}}
\end{itemize}

\begin{figure}

{\centering \includegraphics{Ciencia-com-R_files/figure-latex/graficos-correlacao-1} 

}

\caption{Gráfico de correlação entre duas variáveis com linha de tendência.}\label{fig:graficos-correlacao}
\end{figure}

\begin{itemize}
\tightlist
\item
  Gráfico de matriz de dispersão: Mostra relações entre múltiplas variáveis quantitativas, útil para identificar padrões.\textsuperscript{\citeproc{ref-REF}{\textbf{REF?}}}
\end{itemize}

\begin{figure}

{\centering \includegraphics{Ciencia-com-R_files/figure-latex/graficos-matriz-dispersao-1} 

}

\caption{Matriz de dispersão representando relações entre múltiplas variáveis.}\label{fig:graficos-matriz-dispersao}
\end{figure}

\begin{itemize}
\tightlist
\item
  Gráfico de calor (\emph{heatmap}): Representa dados em uma matriz, com cores indicando intensidade ou frequência.\textsuperscript{\citeproc{ref-REF}{\textbf{REF?}}}
\end{itemize}

\begin{figure}

{\centering \includegraphics{Ciencia-com-R_files/figure-latex/graficos-heatmap-1} 

}

\caption{Mapa de calor da correlação entre variáveis.}\label{fig:graficos-heatmap}
\end{figure}

\begin{itemize}
\tightlist
\item
  Gráfico de radar (ou gráfico de aranha): Representa várias variáveis em um único gráfico, útil para comparar perfis.\textsuperscript{\citeproc{ref-REF}{\textbf{REF?}}}
\end{itemize}

\begin{figure}

{\centering \includegraphics{Ciencia-com-R_files/figure-latex/graficos-multivariados-1} 

}

\caption{Gráfico radar representando múltiplas variáveis.}\label{fig:graficos-multivariados}
\end{figure}

\subsection{Quais são as boas práticas na elaboração de gráficos?}\label{quais-suxe3o-as-boas-pruxe1ticas-na-elaborauxe7uxe3o-de-gruxe1ficos}

\begin{itemize}
\item
  O tamanho da amostra total e subgrupos, se houver, deve estar descrito na figura ou na sua legenda.\textsuperscript{\citeproc{ref-Cumming2007}{183}}
\item
  Para análise inferencial de figuras, as barras de erro representadas por erro-padrão ou intervalo de confiança no nível de significância \(\alpha\) pré-estabelecido são preferíveis à amplitude ou desvio-padrão.\textsuperscript{\citeproc{ref-krzywinski2013}{182},\citeproc{ref-Cumming2007}{183}}
\item
  Evite gráficos de barra e mostre a distribuição dos dados sempre que possível.\textsuperscript{\citeproc{ref-Weissgerber2019}{208}}
\item
  Exiba os pontos de dados em boxplots.\textsuperscript{\citeproc{ref-Weissgerber2019}{208}}
\item
  Use \emph{jitter} simétrico em gráficos de pontos para permitir a visualização de todos os dados.\textsuperscript{\citeproc{ref-Weissgerber2019}{208}}
\item
  Prefira palhetas de cor adaptadas para daltônicos.\textsuperscript{\citeproc{ref-Weissgerber2019}{208}}
\item
  Uma boa legenda torna a figura autossuficiente: descreva amostra (n), geometrias, métricas de incerteza, escalas/unidades e mensagem principal. Se houver modelo, indique fórmula/ajustes em nota.\textsuperscript{\citeproc{ref-midway2020}{221}}
\item
  Evite gráficos de barras com médias para variáveis contínuas; prefira pontos/box/violino e, em amostras pequenas, exiba todos os dados.\textsuperscript{\citeproc{ref-Weissgerber2019}{208}}
\item
  Antes de finalizar um gráfico, faça as seguintes perguntas: (1) Mensagem está explícita? (2) Geometria adequada e dados visíveis? (3) Incerteza correta e rotulada? (4) Cores informativas e acessíveis? (5) Escalas comparáveis (se facetou)? (6) Legenda/caption autossuficiente? (7) Diferença clara entre dados e modelos? (8) Arquivo exportado na resolução/tamanho exigidos?\textsuperscript{\citeproc{ref-midway2020}{221}}
\end{itemize}

\begin{infobox}{images/Rlogo}
O pacote \emph{ggsci}\textsuperscript{\citeproc{ref-ggsci}{226}} fornece palhetas de cores tais como \href{https://www.rdocumentation.org/packages/ggsci/versions/3.0.0/topics/pal_lancet}{\emph{pal\_lancet}}, \href{https://www.rdocumentation.org/packages/ggsci/versions/3.0.0/topics/pal_nejm}{\emph{pal\_nejm}} e \href{https://www.rdocumentation.org/packages/ggsci/versions/3.0.0/topics/pal_npg}{\emph{pal\_npg}} inspiradas em publicações científicas para uso em gráficos.

\end{infobox}

\begin{infobox}{images/Rlogo}
O pacote \emph{grDevices}\textsuperscript{\citeproc{ref-grDevices}{172}} fornece a função \href{https://www.rdocumentation.org/packages/grDevices/versions/3.6.2/topics/dev}{\emph{dev.new}} para controlar diversos aspectos do gráfico, tais como tamanho e resolução.

\end{infobox}

\begin{infobox}{images/Rlogo}
O pacote \emph{tiff}\textsuperscript{\citeproc{ref-tiff}{227}} fornece a função \href{https://www.rdocumentation.org/packages/tiff/versions/0.1-11/topics/writeTIFF}{\emph{writeTIFF}} para exportar gráficos em formato TIFF.

\end{infobox}

\section{Gráficos dinâmicos}\label{gruxe1ficos-dinuxe2micos}

\subsection{O que são visualizações dinâmicas?}\label{o-que-suxe3o-visualizauxe7uxf5es-dinuxe2micas}

\begin{itemize}
\tightlist
\item
  Visualizações dinâmicas combinam interatividade (exploração ativa pelo leitor) e animação (mudanças ao longo do tempo/iterações) para empacotar informação rica em exibições simples, tornando comunicação e exploração mais transparentes.\textsuperscript{\citeproc{ref-wiebels2023}{228}}
\end{itemize}

\subsection{Quando preferir interatividade?}\label{quando-preferir-interatividade}

\begin{itemize}
\tightlist
\item
  Durante exploração de dados em equipe: destacar pontos/linhas por participante, filtrar subconjuntos e inspecionar impactos de escolhas analíticas (p.ex., outliers) sem gerar múltiplas figuras novas.\textsuperscript{\citeproc{ref-wiebels2023}{228}}
\end{itemize}

\begin{figure}

{\centering \includegraphics{Ciencia-com-R_files/figure-latex/grafico-interativo-1} 

}

\caption{Exemplo de gráfico interativo com Plotly.}\label{fig:grafico-interativo}
\end{figure}

\begin{infobox}{images/Rlogo}
O pacote \emph{plotly}\textsuperscript{\citeproc{ref-plotly}{224}} fornece a função \href{https://www.rdocumentation.org/packages/plotly/versions/4.11.0/topics/plot_ly}{\emph{plot\_ly}} para gerar gráficos interativos.

\end{infobox}

\subsection{Quando preferir animação?}\label{quando-preferir-animauxe7uxe3o}

\begin{itemize}
\tightlist
\item
  Em apresentações e para ilustrar variação ao longo de tempo/condição/algoritmo, evitando painéis 3D ou facets excessivos. A animação guia a atenção e revela mudanças de forma passiva e fluida.\textsuperscript{\citeproc{ref-wiebels2023}{228}}
\end{itemize}

\begin{infobox}{images/Rlogo}
O pacote \emph{gganimate}\textsuperscript{\citeproc{ref-gganimate}{229}} fornece a função \href{https://www.rdocumentation.org/packages/gganimate/versions/1.0.7/topics/transition_states}{\emph{transition\_states}} para criar gráficos animados a partir de gráficos estáticos do \emph{ggplot2}\textsuperscript{\citeproc{ref-ggplot2}{173}}.

\end{infobox}

\chapter{\texorpdfstring{\textbf{Análise robusta}}{Análise robusta}}\label{analise-robusta}

\section{Raciocínio inferencial robusto}\label{raciocuxednio-inferencial-robusto}

\subsection{O que é análise robusta?}\label{o-que-uxe9-anuxe1lise-robusta}

\begin{itemize}
\tightlist
\item
  Análise robusta é uma abordagem estatística que busca fornecer resultados confiáveis mesmo quando as suposições clássicas dos modelos estatísticos são violadas.\textsuperscript{\citeproc{ref-WRS2}{230}}
\end{itemize}

\subsection{Por que usar análise robusta?}\label{por-que-usar-anuxe1lise-robusta}

\begin{itemize}
\item
  Métodos clássicos como ANOVA e regressão por mínimos quadrados assumem normalidade e homocedasticidade --- suposições frequentemente violadas na prática. Violações dessas suposições podem comprometer os resultados, reduzindo o poder estatístico, distorcendo os intervalos de confiança e obscurecendo as reais diferenças entre grupos.\textsuperscript{\citeproc{ref-WRS2}{230}}
\item
  Testar previamente as suposições não é suficiente: testes de homocedasticidade têm baixo poder e não garantem segurança analítica.\textsuperscript{\citeproc{ref-WRS2}{230}}
\item
  Métodos estatísticos robustos oferecem uma solução mais segura e eficaz, lidando melhor com dados não ideais.\textsuperscript{\citeproc{ref-WRS2}{230}}
\end{itemize}

\subsection{Quando usar análise robusta?}\label{quando-usar-anuxe1lise-robusta}

\begin{itemize}
\item
  Em alguns casos, os métodos robustos confirmam os resultados clássicos; em outros, revelam interpretações completamente diferentes. A única forma de saber o impacto real dos métodos robustos é usá-los e comparar com os métodos tradicionais.\textsuperscript{\citeproc{ref-WRS2}{230}}
\item
  Mínimos e máximos são parâmetros descritivos, mas em certas condições podem se tornar discrepantes ou influentes, distorcendo análises. Nesses casos, a análise robusta oferece alternativas mais seguras.\textsuperscript{\citeproc{ref-REF}{\textbf{REF?}}}
\end{itemize}

\subsection{Por que métodos robustos são preferíveis?}\label{por-que-muxe9todos-robustos-suxe3o-preferuxedveis}

\begin{itemize}
\item
  Métodos robustos têm a vantagem de resistir à influência de valores extremos, fornecendo medidas de posição e dispersão mais estáveis.\textsuperscript{\citeproc{ref-daszykowski2007}{190}}
\item
  Estimadores robustos oferecem maior segurança na presença de até 50\% de contaminação nos dados, o que representa um ganho significativo em relação aos métodos clássicos.\textsuperscript{\citeproc{ref-daszykowski2007}{190}}
\end{itemize}

\section{Valores discrepantes}\label{valores-discrepantes}

\subsection{\texorpdfstring{O que são valores discrepantes (\emph{outliers})?}{O que são valores discrepantes (outliers)?}}\label{o-que-suxe3o-valores-discrepantes-outliers}

\begin{itemize}
\item
  Em termos gerais, um valor discrepante --- ``fora da curva'' ou \emph{outlier} --- é uma observação que possui um valor relativamente grande ou pequeno em comparação com a maioria das observações.\textsuperscript{\citeproc{ref-zuur2009}{195}}
\item
  Um valor discrepante é uma observação incomum que exerce influência indevida em uma análise.\textsuperscript{\citeproc{ref-zuur2009}{195}}
\item
  Valores discrepantes são dados com valores altos de resíduos.\textsuperscript{\citeproc{ref-leys2019}{188}}
\item
  Nem todo valor extremo é um valor discrepante, e nem todo valor discrepante será influente.\textsuperscript{\citeproc{ref-REF}{\textbf{REF?}}}
\item
  Alguns valores discrepantes são apenas pontos incomuns, outros de fato mudam os resultados e por isso são chamados de influentes.\textsuperscript{\citeproc{ref-REF}{\textbf{REF?}}}
\end{itemize}

\begin{figure}

{\centering \includegraphics{Ciencia-com-R_files/figure-latex/regressao-linear-discrepantes-1} 

}

\caption{Regressão linear com valores discrepantes}\label{fig:regressao-linear-discrepantes}
\end{figure}

\subsection{Quais são os tipos de valores discrepantes?}\label{quais-suxe3o-os-tipos-de-valores-discrepantes}

\begin{itemize}
\item
  Valores discrepantes podem ser categorizados em três subtipos: \emph{outliers} de erro, \emph{outliers} interessantes e \emph{outliers} aleatórios.\textsuperscript{\citeproc{ref-leys2019}{188}}
\item
  Os valores discrepantes de erro são observações claramente não legítimas, distantes de outros dados devido a imprecisões por erro de mensuração e/ou codificação.\textsuperscript{\citeproc{ref-leys2019}{188}}
\item
  Os valores discrepantes interessantes não são claramente erros, mas podem refletir um processo/mecanismo potencialmente interessante para futuras pesquisas.\textsuperscript{\citeproc{ref-leys2019}{188}}
\item
  Os valores discrepantes aleatórios são observações que resultam por acaso, sem qualquer padrão ou tendência conhecida.\textsuperscript{\citeproc{ref-leys2019}{188}}
\item
  Valores discrepantes podem ser univariados ou multivariados.\textsuperscript{\citeproc{ref-leys2019}{188}}
\end{itemize}

\subsection{Por que é importante avaliar valores discrepantes?}\label{por-que-uxe9-importante-avaliar-valores-discrepantes}

\begin{itemize}
\item
  Excluir o valor discrepante implica em reduzir inadequadamente a variância, ao remover um valor que de fato pertence à distribuição considerada.\textsuperscript{\citeproc{ref-leys2019}{188}}
\item
  Manter os dados inalterados (mantendo o valor discrepante) implica em aumentar inadequadamente a variância, pois a observação não pertence à distribuição que fundamenta o experimento.\textsuperscript{\citeproc{ref-leys2019}{188}}
\item
  Em ambos os casos, uma decisão errada pode influenciar o erro do tipo I (\(\alpha\) --- rejeitar uma hipótese verdadeira) ou o erro do tipo II (\(\beta\) --- não rejeitar uma hipótese falsa).\textsuperscript{\citeproc{ref-leys2019}{188}}
\end{itemize}

\subsection{Como detectar valores discrepantes?}\label{como-detectar-valores-discrepantes}

\begin{itemize}
\item
  Na maioria das vezes, não há como saber de qual distribuição uma observação provém. Por isso, não é possível ter certeza se um valor é legítimo ou não dentro do contexto do experimento.\textsuperscript{\citeproc{ref-leys2019}{188}}
\item
  Recomenda-se seguir um procedimento em duas etapas: detectar possíveis candidatos a \emph{outliers} usando ferramentas quantitativas; e gerenciar os outliers, decidindo manter, remover ou recodificar os valores, com base em informações qualitativas.\textsuperscript{\citeproc{ref-leys2019}{188}}
\item
  A detecção de outliers deve ser aplicada apenas uma vez no conjunto de dados; um erro comum é identificar e tratar os outliers (como remover ou recodificar) e, em seguida, reaplicar o procedimento no conjunto de dados já modificado.\textsuperscript{\citeproc{ref-leys2019}{188}}
\item
  A detecção ou o tratamento dos \emph{outliers} não deve ser realizada após a análise dos resultados, pois isso introduz viés nos resultados.\textsuperscript{\citeproc{ref-leys2019}{188}}
\end{itemize}

\subsection{Quais são os métodos para detectar valores discrepantes?}\label{quais-suxe3o-os-muxe9todos-para-detectar-valores-discrepantes}

\begin{itemize}
\item
  Valores univariados são comumente considerados \emph{outliers} quando são mais extremos do que a média ± (desvio padrão × constante), podenso essa constante ser 3 (99,7\% das observações estão dentro de 3 desvios-padrão da média) ou 3,29 (99,9\% estão dentro de 3,29 desvios-padrão).\textsuperscript{\citeproc{ref-leys2019}{188}}
\item
  Para detectar \emph{outliers} univariados, recomenda-se o uso da Mediana da Desviação Absoluta (\emph{Median Absolute Deviation}, MAD), calculado a partir de um intervalo em torno da mediana, multiplicado por uma constante (valor padrão: 1,4826).\textsuperscript{\citeproc{ref-leys2019}{188},\citeproc{ref-leys2013}{231}}
\item
  Para detectar \emph{outliers} multivariados, comumente utiliza-se a distância de Mahalanobis, que identifica valores muito distantes do centróide formado pela maioria dos dados (por exemplo, 99\%).\textsuperscript{\citeproc{ref-leys2019}{188}}
\item
  Para detectar \emph{outliers} multivariados, recomenda-se o Determinante de Mínima Covariância (\emph{Minimum Covariance Determinant}, MCD), pois possui o maior ponto de quebra possível e utiliza a mediana, que é o indicador mais robusto em presença de \emph{outliers}.\textsuperscript{\citeproc{ref-leys2019}{188},\citeproc{ref-leys2018}{232}}
\end{itemize}

\subsection{Quais testes são apropriados para detectar valores discrepantes?}\label{quais-testes-suxe3o-apropriados-para-detectar-valores-discrepantes}

\begin{itemize}
\item
  A escolha do método de detecção depende da natureza do outlier, se univariado ou multivariado.\textsuperscript{\citeproc{ref-daszykowski2007}{190}}
\item
  Para valores univariados, podem ser usados box-plots (com pontos além de 1,5 vezes o intervalo interquartílico), z-scores clássicos (\(|z| > 2.5\) ou \(|z| > 3\)) ou z-scores robustos, que substituem média por mediana e desvio-padrão por estimadores robustos.\textsuperscript{\citeproc{ref-daszykowski2007}{190}}
\item
  Para valores multivariados, recomenda-se a distância de Mahalanobis para medir o afastamento em relação ao centróide, com ajustes robustos de covariância como MCD (\emph{Minimum Covariance Determinant}) ou MVE (\emph{Minimum Volume Ellipsoid}).\textsuperscript{\citeproc{ref-daszykowski2007}{190}}
\item
  Técnicas baseadas em PCA robusta (ROBPCA, PP-PCA, SPCA, EPCA) também podem ser aplicadas para reduzir dimensionalidade e expor \emph{outliers} mascarados.\textsuperscript{\citeproc{ref-daszykowski2007}{190}}
\item
  Métodos de \emph{trimming} multivariado (MVT) podem iterativamente remover observações mais distantes, mas apresentam limitações em alta dimensionalidade.\textsuperscript{\citeproc{ref-daszykowski2007}{190}}
\item
  Estimadores com alto ponto de quebra, como o MCD, permitem detectar até 50\% de \emph{outliers} antes de comprometer a análise.\textsuperscript{\citeproc{ref-daszykowski2007}{190}}
\end{itemize}

\subsection{Como manejar os valores discrepantes?}\label{como-manejar-os-valores-discrepantes}

\begin{itemize}
\item
  Manter \emph{outliers} pode ser uma boa decisão se a maioria desses valores realmente pertence à distribuição de interesse. Manter \emph{outliers} que pertencem a uma distribuição alternativa pode ser problemático, pois um teste pode se tornar significativo apenas por causa de um ou poucos \emph{outliers.}\textsuperscript{\citeproc{ref-leys2019}{188}}
\item
  Remover \emph{outliers} pode ser eficaz quando eles distorcem a estimativa dos parâmetros da distribuição. Remover \emph{outliers} que pertencem legitimamente à distribuição pode reduzir artificialmente a estimativa do erro.\textsuperscript{\citeproc{ref-leys2019}{188}}
\item
  Remover \emph{outliers} leva à perda de observações, especialmente em conjuntos de dados com muitas variáveis, quando outliers univariados são excluídos em cada variável.\textsuperscript{\citeproc{ref-leys2019}{188}}
\item
  Recodificar \emph{outliers} evita a perda de uma grande quantidade de dados, mas deve ser baseada em argumentos razoáveis e convincentes.\textsuperscript{\citeproc{ref-leys2019}{188}}
\item
  Erros de observação e de medição são uma justificativa válida para descartar observações discrepantes.\textsuperscript{\citeproc{ref-zuur2009}{195}}
\end{itemize}

\subsection{Como conduzir análises com valores discrepantes?}\label{como-conduzir-anuxe1lises-com-valores-discrepantes}

\begin{itemize}
\item
  É importante reportar se existem valores discrepantes e como foram tratados.\textsuperscript{\citeproc{ref-zuur2009}{195}}
\item
  Valores discrepantes na variável de desfecho podem exigir uma abordagem mais refinada, especialmente quando representam uma variação real na variável que está sendo medida.\textsuperscript{\citeproc{ref-zuur2009}{195}}
\item
  Valores discrepantes em uma (co)variável podem surgir devido a um projeto experimental inadequado; nesse caso, abandonar a observação ou transformar a covariável são opções adequadas.\textsuperscript{\citeproc{ref-zuur2009}{195}}
\item
  Valores discrepantes podem ser recodificados usando a Winsorização,\textsuperscript{\citeproc{ref-Tukey1963}{233}} que transforma os \emph{outliers} em valores de percentis específicos (como o 5º e o 95º).\textsuperscript{\citeproc{ref-leys2019}{188}}
\end{itemize}

\begin{infobox}{images/Rlogo}
O pacote \emph{outliers}\textsuperscript{\citeproc{ref-outliers}{234}} fornece a função \href{https://www.rdocumentation.org/packages/outliers/versions/0.15/topics/outlier}{\emph{outlier}} para identificar os valores mais distantes da média.

\end{infobox}

\begin{infobox}{images/Rlogo}
O pacote \emph{outliers}\textsuperscript{\citeproc{ref-outliers}{234}} fornece a função \href{https://www.rdocumentation.org/packages/outliers/versions/0.15/topics/rm.outlier}{\emph{rm.outlier}} para remover os valores mais distantes da média detectados por testes de hipótese e/ou substitui-los pela média ou mediana.

\end{infobox}

\subsection{\texorpdfstring{Como lidar com \emph{outliers} na análise exploratória de dados?}{Como lidar com outliers na análise exploratória de dados?}}\label{como-lidar-com-outliers-na-anuxe1lise-exploratuxf3ria-de-dados}

\begin{itemize}
\item
  Após a detecção, três estratégias principais podem ser adotadas: (1) manter os outliers, (2) removê-los ou (3) recodificá-los (por exemplo, com Winsorização). A escolha deve ser justificada com base no contexto teórico e nas características do banco de dados. Idealmente, erros devem ser corrigidos ou removidos, enquanto outliers interessantes podem gerar novas hipóteses de pesquisa.\textsuperscript{\citeproc{ref-leys2019}{188}}
\item
  A decisão sobre como lidar com outliers deve ser definida \emph{a priori} e preferencialmente registrada em plataformas de pré-registro. Essa prática aumenta a transparência, reduz a flexibilidade analítica e evita inflar taxas de erro tipo I.\textsuperscript{\citeproc{ref-leys2019}{188}}
\end{itemize}

\section{Valores influentes}\label{valores-influentes}

\subsection{O que são valores influentes?}\label{o-que-suxe3o-valores-influentes}

\begin{itemize}
\tightlist
\item
  Valores influentes são observações que, se removidas, causariam uma mudança significativa nos resultados da análise estatística.\textsuperscript{\citeproc{ref-REF}{\textbf{REF?}}}
\end{itemize}

\begin{figure}

{\centering \includegraphics{Ciencia-com-R_files/figure-latex/regressao-linear-influentes-simples-1} 

}

\caption{Regressão linear com valores influentes.}\label{fig:regressao-linear-influentes-simples}
\end{figure}

\subsection{O que é função de influência?}\label{o-que-uxe9-funuxe7uxe3o-de-influuxeancia}

\begin{itemize}
\tightlist
\item
  A função de influência mede a sensibilidade de um estimador a pequenas contaminações nos dados. Um estimador é considerado robusto se sua função de influência for limitada, indicando que valores extremos não exercem impacto desproporcional.\textsuperscript{\citeproc{ref-loh2025}{235}}
\end{itemize}

\subsection{O que é ponto de quebra?}\label{o-que-uxe9-ponto-de-quebra-1}

\begin{itemize}
\tightlist
\item
  O ponto de quebra representa a fração mínima de observações contaminadas necessária para distorcer um estimador até o infinito. Por exemplo, a média tem ponto de quebra 0, enquanto a mediana atinge o ponto de quebra máximo (50\%).\textsuperscript{\citeproc{ref-loh2025}{235}}
\end{itemize}

\subsection{Como detectar valores influentes?}\label{como-detectar-valores-influentes}

\begin{itemize}
\tightlist
\item
  A alavancagem (\emph{leverage}) mede o quão distante uma observação está dos valores médios das variáveis independentes. Observações com alta alavancagem têm o potencial de influenciar significativamente a linha de regressão.\textsuperscript{\citeproc{ref-REF}{\textbf{REF?}}}
\end{itemize}

\begin{figure}

{\centering \includegraphics{Ciencia-com-R_files/figure-latex/regressao-linear-influentes-1} 

}

\caption{Alavancagem vs Resíduos Padronizados com distância de Cook para análise da influência de pontos.}\label{fig:regressao-linear-influentes}
\end{figure}

\section{\texorpdfstring{Métodos robustos de tratamento de \emph{outliers}}{Métodos robustos de tratamento de outliers}}\label{muxe9todos-robustos-de-tratamento-de-outliers}

\subsection{O que é Winsorização?}\label{o-que-uxe9-winsorizauxe7uxe3o}

\begin{itemize}
\tightlist
\item
  Winsorização é uma técnica que substitui os valores extremos (\emph{outliers}) por valores menos extremos, preservando a estrutura dos dados. Isso é feito definindo limites superior e inferior e substituindo os valores que ultrapassam esses limites pelos próprios limites.\textsuperscript{\citeproc{ref-WRS2}{230}}
\end{itemize}

\begin{figure}

{\centering \includegraphics{Ciencia-com-R_files/figure-latex/winsorizacao-1} 

}

\caption{Boxplots comparando dados originais e dados Winsorizados.}\label{fig:winsorizacao}
\end{figure}

\subsection{Quais são as alternativas à Winsorização?}\label{quais-suxe3o-as-alternativas-uxe0-winsorizauxe7uxe3o}

\begin{itemize}
\item
  Podar (\emph{trimming}): remove diretamente uma fração fixa das observações mais extremas.\textsuperscript{\citeproc{ref-REF}{\textbf{REF?}}}
\item
  Estimadores robustos: resistem à influência de outliers sem transformar os dados.\textsuperscript{\citeproc{ref-REF}{\textbf{REF?}}}
\item
  Transformações de variáveis: reduzem a assimetria e impacto de valores extremos, mas mudam a escala interpretativa.\textsuperscript{\citeproc{ref-REF}{\textbf{REF?}}}
\end{itemize}

\begin{infobox}{images/Rlogo}
O pacote \emph{WRS2}\textsuperscript{\citeproc{ref-WRS2-2}{236}} fornece as funções \href{https://www.rdocumentation.org/packages/WRS2/versions/1.1-6/topics/trimse}{\emph{winmean}} e \href{https://www.rdocumentation.org/packages/WRS2/versions/1.1-6/topics/trimse}{\emph{winvar}} para calcular a média e variância Winsorizadas.

\end{infobox}

\begin{infobox}{images/Rlogo}
O pacote \emph{WRS2}\textsuperscript{\citeproc{ref-WRS2-2}{236}} fornece a função \href{https://www.rdocumentation.org/packages/WRS2/versions/1.1-6/topics/yuen}{\emph{yuen}} para realizar o teste de comparação de Yuen de médias Winsorizadas para amostras independentes ou dependentes.

\end{infobox}

\begin{infobox}{images/Rlogo}
O pacote \emph{WRS2}\textsuperscript{\citeproc{ref-WRS2-2}{236}} fornece a função \href{https://www.rdocumentation.org/packages/WRS2/versions/1.1-6/topics/pbcor}{\emph{wincor}} para calcular a correlação Winsorizada.

\end{infobox}

\begin{infobox}{images/Rlogo}
O pacote \emph{WRS2}\textsuperscript{\citeproc{ref-WRS2-2}{236}} fornece as funções \href{https://www.rdocumentation.org/packages/WRS2/versions/1.1-6/topics/t1way}{\emph{t1way}}, \href{https://www.rdocumentation.org/packages/WRS2/versions/1.1-6/topics/t2way}{\emph{t2way}} e \href{https://www.rdocumentation.org/packages/WRS2/versions/1.1-6/topics/t3way}{\emph{t3way}} para realizar testes de comparação de médias Winsorizadas para análise de variância para 1, 2 ou 3 fatores, respectivamente.

\end{infobox}

\chapter{\texorpdfstring{\textbf{Análise preditiva}}{Análise preditiva}}\label{analise-preditiva}

\section{Predição}\label{prediuxe7uxe3o}

\subsection{O que são predições?}\label{o-que-suxe3o-prediuxe7uxf5es}

\begin{itemize}
\tightlist
\item
  .\textsuperscript{\citeproc{ref-REF}{\textbf{REF?}}}
\end{itemize}

\begin{infobox}{images/Rlogo}
O pacote \emph{ggeffects}\textsuperscript{\citeproc{ref-ggeffects}{237}} fornece a função \href{https://www.rdocumentation.org/packages/ggeffects/versions/1.6.0/topics/predict_response}{\emph{predict\_response}} para calcular valores preditos marginais ou ajustados das variáveis de desfecho.

\end{infobox}

\begin{infobox}{images/Rlogo}
O pacote \emph{ggeffects}\textsuperscript{\citeproc{ref-ggeffects}{237}} fornece a função \href{https://www.rdocumentation.org/packages/ggeffects/versions/1.6.0/topics/test_response}{\emph{test\_response}} para testar valores preditos marginais ou ajustados das variáveis de desfecho.

\end{infobox}

\subsection{Como árvores de decisão são usadas para predição?}\label{como-uxe1rvores-de-decisuxe3o-suxe3o-usadas-para-prediuxe7uxe3o}

\begin{itemize}
\item
  Utilizam padrões históricos para prever resultados futuros em novos registros.\textsuperscript{\citeproc{ref-Song2015}{238}}
  Identificam combinações de fatores que elevam ou reduzem o risco de um evento clínico.\textsuperscript{\citeproc{ref-Song2015}{238}}
\item
  Podem ser aplicadas em diagnósticos médicos para prever subtipos de condições ou diferentes respostas terapêuticas.\textsuperscript{\citeproc{ref-Song2015}{238}}
\end{itemize}

\section{Interpretação e aplicação}\label{interpretauxe7uxe3o-e-aplicauxe7uxe3o}

\subsection{Quais são as implicações do uso de árvores de decisão em predição?}\label{quais-suxe3o-as-implicauxe7uxf5es-do-uso-de-uxe1rvores-de-decisuxe3o-em-prediuxe7uxe3o}

\begin{itemize}
\item
  Tornam os resultados mais compreensíveis para clínicos e pesquisadores, com regras ``se-então'' claras.\textsuperscript{\citeproc{ref-Song2015}{238}}
\item
  Auxiliam na formulação de hipóteses clínicas e direcionamento de futuras pesquisas.\textsuperscript{\citeproc{ref-Song2015}{238}}
\item
  Podem ser estendidas para predição de sobrevivência, subgrupos de tratamento e análise de custo-benefício.\textsuperscript{\citeproc{ref-Song2015}{238}}
\end{itemize}

\section{Análise de curva de decisão}\label{anuxe1lise-de-curva-de-decisuxe3o}

\subsection{O que é a análise de curva de decisão?}\label{o-que-uxe9-a-anuxe1lise-de-curva-de-decisuxe3o}

\begin{itemize}
\tightlist
\item
  Análise de curva de decisão é um método para avaliar a utilidade clínica de modelos preditivos em comparação às estratégias padrão ``tratar todos'' ou ``tratar ninguém''.\textsuperscript{\citeproc{ref-hozo2023}{239}}
\end{itemize}

\subsection{O que significam os eixos da curva de decisão?}\label{o-que-significam-os-eixos-da-curva-de-decisuxe3o}

\begin{itemize}
\item
  O eixo vertical mostra o benefício líquido, ganho obtido ao usar o modelo em relação às estratégias de referência. O eixo horizontal mostra a preferência, que corresponde ao limiar de decisão: a probabilidade mínima de evento a partir da qual se recomenda intervir.\textsuperscript{\citeproc{ref-vickers2019}{240}}
\item
  O limiar reflete o equilíbrio entre o risco de perder casos da doença e o risco de intervenções desnecessárias.\textsuperscript{\citeproc{ref-vickers2019}{240}}
\end{itemize}

\subsection{Como interpretar o benefício líquido?}\label{como-interpretar-o-benefuxedcio-luxedquido}

\begin{itemize}
\item
  O benefício líquido é o número de verdadeiros positivos obtidos ajustado pelo ``custo'' dos falsos positivos, expresso na mesma unidade dos verdadeiros positivos.\textsuperscript{\citeproc{ref-vickers2019}{240}}
\item
  Quando a estratégia padrão é ``tratar todos'', o benefício líquido também pode ser expresso como intervenções desnecessárias evitadas, facilitando a interpretação clínica.\textsuperscript{\citeproc{ref-vickers2019}{240}}
\end{itemize}

\subsection{Por que é importante comparar sempre com ``tratar todos'' e ``tratar nenhum''?}\label{por-que-uxe9-importante-comparar-sempre-com-tratar-todos-e-tratar-nenhum}

\begin{itemize}
\item
  Porque essas duas estratégias são frequentemente plausíveis em contextos reais: tratar todos os pacientes de risco ou tratar nenhum.\textsuperscript{\citeproc{ref-vickers2019}{240}}
\item
  Um modelo ou teste só é considerado útil se apresentar maior benefício líquido do que ambas as estratégias, garantindo que realmente agrega valor clínico.\textsuperscript{\citeproc{ref-vickers2019}{240}}
\item
  Essa comparação evita que um modelo com bom AUC mas má calibração seja usado, já que pode apresentar net benefit inferior às estratégias padrão.\textsuperscript{\citeproc{ref-vickers2019}{240}}
\end{itemize}

\subsection{Quais são os limites e usos da análise de curva de decisão?}\label{quais-suxe3o-os-limites-e-usos-da-anuxe1lise-de-curva-de-decisuxe3o}

\begin{itemize}
\item
  A análise de curva de decisão não substitui uma análise de decisão completa ou uma avaliação de custo-efetividade, mas é mais simples e prática, exigindo apenas a definição de um intervalo de limiares de decisão razoáveis.\textsuperscript{\citeproc{ref-vickers2019}{240}}
\item
  Em situações claras (modelo com nenhum benefício ou benefício amplo), pode dispensar análises complexas.\textsuperscript{\citeproc{ref-vickers2019}{240}}
\item
  Em casos ambíguos, serve como primeiro passo antes de análises mais detalhadas.\textsuperscript{\citeproc{ref-vickers2019}{240}}
\end{itemize}

\subsection{A análise de curva de decisão pode ser conduzida sem dados individuais de pacientes?}\label{a-anuxe1lise-de-curva-de-decisuxe3o-pode-ser-conduzida-sem-dados-individuais-de-pacientes}

\begin{itemize}
\tightlist
\item
  Em modelos bem calibrados e com tamanho amostral adequado, a análise de curva de decisão pode ser realizada utilizando apenas estimativas sumárias (média e desvio-padrão).\textsuperscript{\citeproc{ref-hozo2023}{239}}
\end{itemize}

\subsection{Como funciona o cálculo do benefício líquido?}\label{como-funciona-o-cuxe1lculo-do-benefuxedcio-luxedquido}

\begin{itemize}
\tightlist
\item
  O benefício líquido é estimado contrastando verdadeiros positivos e falsos positivos em diferentes limiares de decisão.\textsuperscript{\citeproc{ref-hozo2023}{239}}
\end{itemize}

\chapter{\texorpdfstring{\textbf{Análise causal}}{Análise causal}}\label{analise-causal}

\section{Causalidade}\label{causalidade}

\subsection{O que é análise causal?}\label{o-que-uxe9-anuxe1lise-causal}

\begin{itemize}
\item
  Análise causal é usada para explicar a relação entre causa e efeito em um conjunto de dados, respondendo a perguntas do tipo ``por quê?''.\textsuperscript{\citeproc{ref-gerring2012}{204}}
\item
  Análise causal implica em contrafactual, no sentido de que a análise causal é baseada na comparação entre o que realmente aconteceu e o que teria acontecido se uma ou mais variáveis tivessem sido diferentes.\textsuperscript{\citeproc{ref-gerring2012}{204}}
\item
  Análise causal refere-se ao processo conceitual e analítico, enquanto inferência causal diz respeito às conclusões extraídas a partir de dados.\textsuperscript{\citeproc{ref-REF}{\textbf{REF?}}}
\end{itemize}

\subsection{Quais os dois grandes tipos de causalidade?}\label{quais-os-dois-grandes-tipos-de-causalidade}

\begin{itemize}
\item
  Baseada em experiência: conhecimento empírico, muitas vezes sem compreensão dos mecanismos. Estatística tem papel central na causalidade baseada em experiência.\textsuperscript{\citeproc{ref-aalen2007}{241}}
\item
  Mecanicista: busca entender processos internos e mecanismos. Estatística tem seu papel ainda limitado, mas crescente, especialmente em sistemas complexos.\textsuperscript{\citeproc{ref-aalen2007}{241}}
\end{itemize}

\subsection{Como realizar uma análise causal?}\label{como-realizar-uma-anuxe1lise-causal}

\begin{itemize}
\item
  Formular explicitamente uma pergunta causal, definindo qual intervenção hipotética está sendo considerada e qual efeito se deseja estimar.\textsuperscript{\citeproc{ref-REF}{\textbf{REF?}}}
\item
  Especificar a ordem temporal entre exposição, desfecho e demais variáveis, assegurando que a causa preceda o efeito.\textsuperscript{\citeproc{ref-REF}{\textbf{REF?}}}
\item
  Tornar explícitas as suposições causais sobre o sistema em estudo, preferencialmente por meio de um diagrama causal (DAG).\textsuperscript{\citeproc{ref-REF}{\textbf{REF?}}}
\item
  Identificar confundidores, mediadores e colisores com base na estrutura causal assumida, e não apenas em critérios estatísticos.\textsuperscript{\citeproc{ref-REF}{\textbf{REF?}}}
\item
  Definir o conjunto mínimo de ajuste necessário para bloquear caminhos espúrios entre exposição e desfecho.\textsuperscript{\citeproc{ref-REF}{\textbf{REF?}}}
\item
  Escolher métodos estatísticos compatíveis com a pergunta causal (ex.: regressão ajustada, ponderação por escore de propensão, g-methods), reconhecendo que o método não cria causalidade.\textsuperscript{\citeproc{ref-REF}{\textbf{REF?}}}
\item
  Avaliar limitações como confundimento residual, viés de seleção e erro de medida, discutindo seu impacto potencial nas inferências.\textsuperscript{\citeproc{ref-REF}{\textbf{REF?}}}
\item
  Interpretar os resultados em termos causais explícitos, distinguindo claramente associação de efeito causal.\textsuperscript{\citeproc{ref-REF}{\textbf{REF?}}}
\end{itemize}

\section{Abordagens filosóficas e estatísticas da causalidade}\label{abordagens-filosuxf3ficas-e-estatuxedsticas-da-causalidade}

\subsection{O que é realidade causal?}\label{o-que-uxe9-realidade-causal}

\begin{itemize}
\tightlist
\item
  A estatística assume tanto a presença do acaso quanto de causalidade. Entretanto, a natureza de cada um (se essencial ou apenas reflexo de ignorância) é raramente debatida.\textsuperscript{\citeproc{ref-aalen2007}{241}}
\end{itemize}

\subsection{Por que estatísticos historicamente evitaram falar em causalidade?}\label{por-que-estatuxedsticos-historicamente-evitaram-falar-em-causalidade}

\begin{itemize}
\item
  Pearson e Fisher defenderam que estatística trata apenas de associação, não de causalidade, o que gerou cautela excessiva e paralisou avanços em áreas como economia e ciências sociais.\textsuperscript{\citeproc{ref-aalen2007}{241}}
\item
  Autores como Judea Pearl, Robins e Rubin trouxeram definições mais precisas, especialmente via modelos contrafactuais.\textsuperscript{\citeproc{ref-aalen2007}{241}}
\item
  O uso de ensaios clínicos randomizados consolidou o papel da estatística em inferência causal aplicada.\textsuperscript{\citeproc{ref-aalen2007}{241}}
\end{itemize}

\section{Ilusões de causalidade}\label{ilusuxf5es-de-causalidade}

\subsection{O que são ilusões de causalidade?}\label{o-que-suxe3o-ilusuxf5es-de-causalidade}

\begin{itemize}
\tightlist
\item
  Ocorrem quando acreditamos que há uma relação causal entre dois eventos que, na realidade, são independentes. São comuns em superstições, pseudociências e crenças do cotidiano.\textsuperscript{\citeproc{ref-matute2015}{242}}
\end{itemize}

\subsection{Quais fatores favorecem a ilusão?}\label{quais-fatores-favorecem-a-ilusuxe3o}

\begin{itemize}
\item
  Alta frequência do desfecho: quando o resultado ocorre frequentemente por acaso, as pessoas superestimam a eficácia da causa (ex.: melhora espontânea de sintomas atribuída a um tratamento).\textsuperscript{\citeproc{ref-matute2015}{242}}
\item
  Alta frequência da causa: quanto mais vezes um comportamento ou tratamento é aplicado, mais coincidências com o desfecho ocorrem, aumentando a crença no efeito.\textsuperscript{\citeproc{ref-matute2015}{242}}
\item
  Coincidências causa--desfecho: damos peso desproporcional a casos em que causa e efeito ocorrem juntos, mesmo que sejam apenas coincidências.\textsuperscript{\citeproc{ref-matute2015}{242}}
\end{itemize}

\subsection{Como reduzir ilusões de causalidade?}\label{como-reduzir-ilusuxf5es-de-causalidade}

\begin{itemize}
\item
  Ensinar princípios de controle científico, observando casos em que a causa está ausente (comparação necessária para detectar ausência de relação).\textsuperscript{\citeproc{ref-matute2015}{242}}
\item
  Diminuir a frequência da causa (ex.: reduzir uso de um ``remédio ineficaz'' ajuda a perceber que o resultado ocorre independentemente).\textsuperscript{\citeproc{ref-matute2015}{242}}
\item
  Instruções explícitas para testar hipóteses: orientar a aplicar a causa em apenas 50\% das vezes favorece a detecção correta da ausência de efeito.\textsuperscript{\citeproc{ref-matute2015}{242}}
\item
  Promover educação científica prática, mostrando às pessoas como seus próprios julgamentos podem ser enviesados e oferecendo ferramentas para avaliação crítica.\textsuperscript{\citeproc{ref-matute2015}{242}}
\end{itemize}

\section{Inferência causal em estudos observacionais}\label{inferuxeancia-causal-em-estudos-observacionais}

\subsection{Como diferenciar associação de causalidade?}\label{como-diferenciar-associauxe7uxe3o-de-causalidade}

\begin{itemize}
\item
  Associação descreve que duas variáveis variam juntas, mas não garante que uma afete a outra.\textsuperscript{\citeproc{ref-vickers2023}{243}}
\item
  Causalidade exige evidências (diretas ou indiretas) de que modificar a variável de exposição altera o desfecho.\textsuperscript{\citeproc{ref-vickers2023}{243}}
\end{itemize}

\subsection{Quais critérios ajudam a sustentar inferência causal?}\label{quais-crituxe9rios-ajudam-a-sustentar-inferuxeancia-causal}

\begin{itemize}
\item
  Existência de um mecanismo plausível.\textsuperscript{\citeproc{ref-vickers2023}{243}}
\item
  Controle adequado de confundidores (medidos e não medidos).\textsuperscript{\citeproc{ref-vickers2023}{243}}
\item
  Consistência com literatura prévia e plausibilidade do tamanho do efeito.\textsuperscript{\citeproc{ref-vickers2023}{243}}
\item
  Avaliação de alternativas explicativas (ex.: viés de seleção, mediadores não controlados).\textsuperscript{\citeproc{ref-vickers2023}{243}}
\end{itemize}

\subsection{Qual o papel dos caminhos causais (DAGs)?}\label{qual-o-papel-dos-caminhos-causais-dags}

\begin{itemize}
\item
  Ajudam a identificar quais variáveis precisam ser medidas e ajustadas.\textsuperscript{\citeproc{ref-vickers2023}{243}}
\item
  Evitam ajustes indevidos (ex.: em colisores), que podem introduzir viés.\textsuperscript{\citeproc{ref-vickers2023}{243}}
\end{itemize}

\subsection{Como lidar com confundimento residual?}\label{como-lidar-com-confundimento-residual}

\begin{itemize}
\item
  Reconhecer que modelos multivariados e escores de propensão não eliminam completamente o confundimento.\textsuperscript{\citeproc{ref-vickers2023}{243}}
\item
  Comparar características basais entre grupos para identificar diferenças persistentes.\textsuperscript{\citeproc{ref-vickers2023}{243}}
\item
  Considerar análises de sensibilidade, mas com cautela na interpretação.\textsuperscript{\citeproc{ref-vickers2023}{243}}
\end{itemize}

\section{Critérios de Hill para inferência causal}\label{crituxe9rios-de-hill-para-inferuxeancia-causal}

\subsection{Quais são os nove critérios?}\label{quais-suxe3o-os-nove-crituxe9rios}

\begin{itemize}
\item
  Temporalidade: A exposição deve preceder o desfecho. Único critério considerado essencial por Hill.\textsuperscript{\citeproc{ref-hill1965}{244}}
\item
  Força da associação: Associações mais fortes são mais prováveis de refletir causalidade.\textsuperscript{\citeproc{ref-hill1965}{244}}
\item
  Consistência: A associação é observada em diferentes estudos, populações e contextos.\textsuperscript{\citeproc{ref-hill1965}{244}}
\item
  Especificidade: Uma exposição leva a um efeito específico (embora nem sempre aplicável).\textsuperscript{\citeproc{ref-hill1965}{244}}
\item
  Gradiente biológico (dose--resposta): Aumentos na exposição acompanham aumentos no risco.\textsuperscript{\citeproc{ref-hill1965}{244}}
\item
  Plausibilidade biológica: Compatibilidade com o conhecimento científico da época.\textsuperscript{\citeproc{ref-hill1965}{244}}
\item
  Coerência: A associação não deve contradizer a história natural ou biologia da doença.\textsuperscript{\citeproc{ref-hill1965}{244}}
\item
  Evidência experimental: Reduções na exposição devem reduzir o risco observado.\textsuperscript{\citeproc{ref-hill1965}{244}}
\item
  Analogia: Comparação com relações causais já conhecidas.\textsuperscript{\citeproc{ref-hill1965}{244}}
\end{itemize}

\subsection{Hill propôs um checklist rígido?}\label{hill-propuxf4s-um-checklist-ruxedgido}

\begin{itemize}
\tightlist
\item
  Nenhum critério, isoladamente, prova ou refuta causalidade. Devem ser usados como guias para reflexão científica, não como requisitos obrigatórios.\textsuperscript{\citeproc{ref-hill1965}{244}}
\end{itemize}

\section{Críticas contemporâneas aos critérios de Hill}\label{cruxedticas-contemporuxe2neas-aos-crituxe9rios-de-hill}

\subsection{Qual critério é indispensável?}\label{qual-crituxe9rio-uxe9-indispensuxe1vel}

\begin{itemize}
\tightlist
\item
  A temporalidade: a exposição deve preceder o desfecho. Mesmo assim, observar uma ordem temporal inversa apenas invalida a hipótese em casos específicos, não em todos.\textsuperscript{\citeproc{ref-rothman2005}{245}}
\end{itemize}

\subsection{A força da associação garante causalidade?}\label{a-foruxe7a-da-associauxe7uxe3o-garante-causalidade}

\begin{itemize}
\tightlist
\item
  Não. Associações fortes podem ainda ser não-causais e associações fracas podem ser causais.\textsuperscript{\citeproc{ref-rothman2005}{245}}
\end{itemize}

\subsection{A consistência é indispensável?}\label{a-consistuxeancia-uxe9-indispensuxe1vel}

\begin{itemize}
\item
  Não. A ausência de consistência não elimina causalidade, pois alguns efeitos só se manifestam em condições específicas (ex.: transfusão só causa HIV se o vírus estiver presente).\textsuperscript{\citeproc{ref-rothman2005}{245}}
\item
  A consistência ajuda apenas a afastar a hipótese de viés ou erro em um estudo isolado:contentReference.\textsuperscript{\citeproc{ref-rothman2005}{245}}
\end{itemize}

\subsection{O critério da especificidade é válido?}\label{o-crituxe9rio-da-especificidade-uxe9-vuxe1lido}

\begin{itemize}
\tightlist
\item
  Não. É considerado um critério inválido e enganoso. Uma causa pode ter múltiplos efeitos (tabagismo → vários desfechos) e um efeito pode ter múltiplas causas.\textsuperscript{\citeproc{ref-rothman2005}{245}}
\end{itemize}

\subsection{O gradiente biológico (dose--resposta) é confiável?}\label{o-gradiente-bioluxf3gico-doseresposta-uxe9-confiuxe1vel}

\begin{itemize}
\tightlist
\item
  Nem sempre. Pode ser distorcido por confundimento. A ausência de gradiente não invalida a causalidade.\textsuperscript{\citeproc{ref-rothman2005}{245}}
\end{itemize}

\subsection{A plausibilidade e a coerência são objetivas?}\label{a-plausibilidade-e-a-coeruxeancia-suxe3o-objetivas}

\begin{itemize}
\tightlist
\item
  Não. Ambas são fortemente dependentes do conhecimento científico da época. O que parecia implausível no passado depois se confirmou como verdadeiro.\textsuperscript{\citeproc{ref-rothman2005}{245}}
\end{itemize}

\subsection{Evidência experimental é decisiva?}\label{eviduxeancia-experimental-uxe9-decisiva}

\begin{itemize}
\tightlist
\item
  Pode ser útil, mas raramente está disponível em epidemiologia. Mesmo quando disponível, pode ter explicações alternativas.\textsuperscript{\citeproc{ref-rothman2005}{245}}
\end{itemize}

\subsection{Analogia é útil?}\label{analogia-uxe9-uxfatil}

\begin{itemize}
\tightlist
\item
  Tem pouco valor. Analogias podem sempre ser inventadas e, na prática, funcionam mais como fonte de hipóteses do que como prova.\textsuperscript{\citeproc{ref-rothman2005}{245}}
\end{itemize}

\section{Visão atual sobre os critérios de Hill}\label{visuxe3o-atual-sobre-os-crituxe9rios-de-hill}

\subsection{Como os critérios de Hill foram revisitados?}\label{como-os-crituxe9rios-de-hill-foram-revisitados}

\begin{itemize}
\tightlist
\item
  Estudos recentes propõem integrá-los a três abordagens modernas: DAG (destacam estrutura causal e confundimento), modelos de causa suficiente (enfatizam multifatorialidade) e GRADE (orienta sobre a certeza da evidência em corpos de estudos).\textsuperscript{\citeproc{ref-shimonovich2020}{246}}
\end{itemize}

\subsection{Quais mudanças na interpretação?}\label{quais-mudanuxe7as-na-interpretauxe7uxe3o}

\begin{itemize}
\item
  Temporalidade e experimentos: seguem centrais, mas analisados com mais sofisticação.\textsuperscript{\citeproc{ref-shimonovich2020}{246}}
\item
  Força da associação: relevante, mas não garante causalidade (pode haver confundimento).\textsuperscript{\citeproc{ref-shimonovich2020}{246}}
\item
  Consistência: pensada como transportabilidade entre populações.\textsuperscript{\citeproc{ref-shimonovich2020}{246}}
\item
  Especificidade: pouco útil hoje; substituída por falsificação (controles negativos).\textsuperscript{\citeproc{ref-shimonovich2020}{246}}
\item
  Dose--resposta: pode ser espúria, cautela é necessária.\textsuperscript{\citeproc{ref-shimonovich2020}{246}}
\item
  Coerência e analogia: utilidade limitada.\textsuperscript{\citeproc{ref-shimonovich2020}{246}}
\end{itemize}

\section{Linguagem causal em estudos observacionais}\label{linguagem-causal-em-estudos-observacionais}

\subsection{Quais são as principais recomendações para relatar causalidade?}\label{quais-suxe3o-as-principais-recomendauxe7uxf5es-para-relatar-causalidade}

\begin{itemize}
\item
  Usar termos causais de forma explícita e criteriosa (``causa'', ``efeito'', ``reduzir'', ``aumentar''), evitando expressões ambíguas como ``fator de risco''.\textsuperscript{\citeproc{ref-vickers2023}{243}}
\item
  Contextualizar a causalidade em termos práticos, explicando por que identificar a causa é relevante para intervenções.\textsuperscript{\citeproc{ref-vickers2023}{243}}
\item
  Declarar claramente na introdução se existe hipótese causal, justificando quando não houver.\textsuperscript{\citeproc{ref-vickers2023}{243}}
\item
  Descrever caminhos causais (mediadores, confundidores, colisores) em texto claro ou com diagramas.\textsuperscript{\citeproc{ref-vickers2023}{243}}
\item
  Justificar a seleção de covariáveis com base nas relações causais previstas.\textsuperscript{\citeproc{ref-vickers2023}{243}}
\item
  Avaliar o controle de confundimento, reconhecendo limitações e possível confundimento residual.\textsuperscript{\citeproc{ref-vickers2023}{243}}
\item
  Discutir as inferências causais considerando estimativas, vieses e plausibilidade biológica.\textsuperscript{\citeproc{ref-vickers2023}{243}}
\item
  Indicar recomendações específicas para pesquisas futuras ou prática clínica baseadas nas conclusões causais.\textsuperscript{\citeproc{ref-vickers2023}{243}}
\end{itemize}

\section{Efeitos diretos e indiretos}\label{efeitos-diretos-e-indiretos}

\subsection{Como distinguir efeitos diretos de indiretos?}\label{como-distinguir-efeitos-diretos-de-indiretos}

\begin{itemize}
\item
  Um efeito direto ocorre quando uma variável influencia outra sem mediação.\textsuperscript{\citeproc{ref-aalen2007}{241}}
\item
  Um efeito indireto acontece quando a influência é mediada por variáveis intermediárias.\textsuperscript{\citeproc{ref-aalen2007}{241}}
\end{itemize}

\section{O papel do tempo e a causalidade dinâmica}\label{o-papel-do-tempo-e-a-causalidade-dinuxe2mica}

\subsection{O que é causalidade de Granger?}\label{o-que-uxe9-causalidade-de-granger}

\begin{itemize}
\item
  É um conceito estatístico que analisa como processos passados influenciam o futuro, indo além da simples associação.\textsuperscript{\citeproc{ref-aalen2007}{241}}
\item
  Permite identificar relações direcionais entre processos ao longo do tempo (ex.: cérebro controlando contrações musculares).\textsuperscript{\citeproc{ref-aalen2007}{241}}
\item
  A estatística, nesse contexto, busca ``olhar dentro da caixa'', aproximando-se de uma visão mecanicista.\textsuperscript{\citeproc{ref-aalen2007}{241}}
\end{itemize}

\subsection{Por que o tempo é essencial na análise causal?}\label{por-que-o-tempo-uxe9-essencial-na-anuxe1lise-causal}

\begin{itemize}
\item
  Processos causais não ocorrem de forma estática: efeitos diretos e indiretos se acumulam em cadeias temporais.\textsuperscript{\citeproc{ref-aalen2007}{241}}
\item
  Modelos tradicionais (ex.: regressões estáticas ou DAGs sem tempo) podem falhar em capturar a dinâmica.\textsuperscript{\citeproc{ref-aalen2007}{241}}
\item
  A integração de séries temporais e processos estocásticos é fundamental para compreender mecanismos.\textsuperscript{\citeproc{ref-aalen2007}{241}}
\end{itemize}

\section{Diagrama acíclico direcionado (DAG)}\label{diagrama-acuxedclico-direcionado-dag}

\subsection{O que são DAGs?}\label{o-que-suxe3o-dags}

\begin{itemize}
\item
  DAGs são representações gráficas de relações causais entre variáveis, usando nós (variáveis) e arestas direcionadas (relações causais).\textsuperscript{\citeproc{ref-REF}{\textbf{REF?}}}
\item
  DAGs ajudam a identificar confundidores, mediadores e colisores, orientando a seleção de variáveis para ajuste em análises estatísticas.\textsuperscript{\citeproc{ref-REF}{\textbf{REF?}}}
\item
  DAGs são acíclicos, ou seja, não permitem ciclos ou loops, refletindo a natureza unidirecional das relações causais.\textsuperscript{\citeproc{ref-REF}{\textbf{REF?}}}
\end{itemize}

\subsection{Quais são os padrões causais básicos?}\label{quais-suxe3o-os-padruxf5es-causais-buxe1sicos}

\begin{itemize}
\item
  Independência: duas variáveis não têm relação causal direta ou indireta.\textsuperscript{\citeproc{ref-REF}{\textbf{REF?}}}
\item
  Cadeia: uma variável causa outra, que por sua vez causa uma terceira (X → M → Y).\textsuperscript{\citeproc{ref-REF}{\textbf{REF?}}}
\item
  Garfo: uma variável causa duas outras (X ← Z → Y), onde Z é um confundidor.\textsuperscript{\citeproc{ref-REF}{\textbf{REF?}}}
\item
  Colisor: duas variáveis causam uma terceira (X → Z ← Y), onde Z é um colisor.\textsuperscript{\citeproc{ref-REF}{\textbf{REF?}}}
\end{itemize}

\begin{figure}

{\centering \includegraphics[width=0.9\linewidth]{Ciencia-com-R_files/figure-latex/dag-1} 

}

\caption{Padrões causais básicos: independência, cadeia, garfo e colisor.}\label{fig:dag}
\end{figure}

\subsection{Quais são as regras básicas para análise causal?}\label{quais-suxe3o-as-regras-buxe1sicas-para-anuxe1lise-causal}

\begin{itemize}
\item
  Variáveis causalmente independentes tendem a não ser correlacionadas, exceto na presença de confundimento ou viés de seleção.\textsuperscript{\citeproc{ref-REF}{\textbf{REF?}}}
\item
  Influência causal pode criar correlação estatística, mas correlação não é condição necessária nem suficiente para causalidade.\textsuperscript{\citeproc{ref-REF}{\textbf{REF?}}}
\item
  Confundimento cria correlação espúria entre exposição e desfecho, mesmo na ausência de efeito causal direto.\textsuperscript{\citeproc{ref-REF}{\textbf{REF?}}}
\item
  Aleatorização protege a variável de exposição contra influência causal de fatores antecedentes, reduzindo confundimento pré-exposição.\textsuperscript{\citeproc{ref-REF}{\textbf{REF?}}}
\end{itemize}

\subsection{Quais são as regras básicas para ajuste?}\label{quais-suxe3o-as-regras-buxe1sicas-para-ajuste}

\begin{itemize}
\item
  Controlar por um confundidor bloqueia caminhos causais espúrios e reduz viés na estimativa do efeito causal.\textsuperscript{\citeproc{ref-REF}{\textbf{REF?}}}
\item
  Controlar por um mediador bloqueia parte ou todo o efeito causal total, sendo inadequado quando o objetivo é estimar o efeito total da exposição.\textsuperscript{\citeproc{ref-REF}{\textbf{REF?}}}
\item
  Controlar por um colisor cria correlação espúria entre as variáveis que o causam, introduzindo viés mesmo quando não havia associação prévia.\textsuperscript{\citeproc{ref-REF}{\textbf{REF?}}}
\item
  Controlar por uma variável descendente de um colisor pode reabrir parcialmente caminhos espúrios, produzindo viés de seleção.\textsuperscript{\citeproc{ref-REF}{\textbf{REF?}}}
\end{itemize}

\begin{infobox}{images/Rlogo}
O pacote \emph{dagitty}\textsuperscript{\citeproc{ref-dagitty}{247}} fornece a função \href{https://cran.r-project.org/web/packages/dagitty/index.html}{\emph{dagitty}} para criar um objeto grafo a partir de uma descrição textual.

\end{infobox}

\begin{infobox}{images/Rlogo}
O pacote \emph{ggdag}\textsuperscript{\citeproc{ref-ggdag}{248}} fornece a função \href{https://www.rdocumentation.org/packages/ggdag/versions/0.2.10/topics/ggdag}{\emph{ggdag}} para criar figuras de grafos.

\end{infobox}

\begin{infobox}{images/Rlogo}
O pacote \emph{performance}\textsuperscript{\citeproc{ref-performance}{249}} fornece a função \href{https://easystats.github.io/performance/reference/check_dag.html}{\emph{check\_dag}} para criar, verificar e visualizar os modelos em grafos.

\end{infobox}

\chapter{\texorpdfstring{\textbf{Análise qualitativa}}{Análise qualitativa}}\label{analise-qualitativa}

\section{Análise qualitativa}\label{anuxe1lise-qualitativa}

\subsection{O que é análise qualitativa?}\label{o-que-uxe9-anuxe1lise-qualitativa}

\begin{itemize}
\tightlist
\item
  .\textsuperscript{\citeproc{ref-REF}{\textbf{REF?}}}
\end{itemize}

\section{Representação de texto}\label{representauxe7uxe3o-de-texto}

\subsection{O que é tokenização?}\label{o-que-uxe9-tokenizauxe7uxe3o}

\begin{itemize}
\item
  Tokenização é o processo de dividir texto contínuo em unidades menores (tokens), como palavras, pontuação, subpalavras ou caracteres. O objetivo é criar uma representação discreta sobre a qual modelos podem calcular frequências, probabilidades e relações.\textsuperscript{\citeproc{ref-REF}{\textbf{REF?}}}
\item
  É comum combinar tokenização com normalização (\emph{lowercase}), remoção de \emph{stopwords}, lematização/\emph{stemming} e regras para números e pontuação.\textsuperscript{\citeproc{ref-REF}{\textbf{REF?}}}
\end{itemize}

\subsection{Modelagem com N-gramas}\label{modelagem-com-n-gramas}

\subsection{O que são n-gramas?}\label{o-que-suxe3o-n-gramas}

\begin{itemize}
\item
  Um n-grama é uma sequência contígua de n tokens, tais como: 1-gramas (unigramas), 2-gramas (bigramas), 3-gramas (trigramas).\textsuperscript{\citeproc{ref-REF}{\textbf{REF?}}}
\item
  Contagens de n-gramas aproximam dependências locais no texto e servem de base para DTM/TF-IDF, modelos de linguagem clássicos e detecção de coligações.\textsuperscript{\citeproc{ref-REF}{\textbf{REF?}}}
\end{itemize}

\begin{infobox}{images/Rlogo}
O pacote \emph{tidytext}\textsuperscript{\citeproc{ref-tidytext}{250}} fornece a função \href{https://www.rdocumentation.org/packages/tidytext/versions/0.4.3/topics/unnest_tokens}{\emph{unnest\_token}} para transformar um texto em um \emph{data frame} com uma coluna para cada palavra.

\end{infobox}

\begin{infobox}{images/Rlogo}
O pacote \emph{tidytext}\textsuperscript{\citeproc{ref-tidytext}{250}} fornece a função \href{https://www.rdocumentation.org/packages/tidytext/versions/0.4.3/topics/stop_words}{\emph{stop\_words}} para remover palavras comuns que não agregam significado.

\end{infobox}

\begin{infobox}{images/Rlogo}
O pacote \emph{tidytext}\textsuperscript{\citeproc{ref-tidytext}{250}} fornece a função \href{https://www.rdocumentation.org/packages/tidytext/versions/0.4.3/topics/get_sentiments}{\emph{get\_sentiments}} para obter listas de palavras com sentimentos associados.

\end{infobox}

\cftaddtitleline{toc}{chapter}{\rule{\textwidth}{0.4pt}}{}

\chapter*{\texorpdfstring{\emph{PARTE 5: ANÁLISES INFERENCIAIS}}{PARTE 5: ANÁLISES INFERENCIAIS}}\label{parte-5}
\addcontentsline{toc}{chapter}{\emph{PARTE 5: ANÁLISES INFERENCIAIS}}

\par\noindent\rule{\textwidth}{0.05in}

\section*{Testando hipóteses e estimando parâmetros para responder perguntas de pesquisa}\label{testando-hipuxf3teses-e-estimando-paruxe2metros-para-responder-perguntas-de-pesquisa}

\markboth{}{}

\chapter{\texorpdfstring{\textbf{Suposições inferenciais}}{Suposições inferenciais}}\label{suposicoes-inferenciais}

\section{Suposições gerais em análises inferenciais}\label{suposiuxe7uxf5es-gerais-em-anuxe1lises-inferenciais}

\subsection{Quais são as suposições ao nível dos dados (condicionais ao modelo)?}\label{quais-suxe3o-as-suposiuxe7uxf5es-ao-nuxedvel-dos-dados-condicionais-ao-modelo}

\begin{itemize}
\item
  Independência (ou dependência corretamente modelada) das observações: .{[}REF{]}
\item
  Forma da distribuição dos erros ou resíduos (normalidade, assimetria, caudas): .{[}REF{]}
\item
  Homocedasticidade (igualdade de variâncias condicionais): .{[}REF{]}
\end{itemize}

\subsection{Quais são as suposições ao nível do modelo?}\label{quais-suxe3o-as-suposiuxe7uxf5es-ao-nuxedvel-do-modelo}

\begin{itemize}
\item
  Linearidade da relação entre variáveis: .{[}REF{]}
\item
  Multicolinearidade ausente ou controlada: .{[}REF{]}
\item
  Especificação funcional correta do modelo: .{[}REF{]}
\end{itemize}

\subsection{Quais são as suposições ao nível do estudo?}\label{quais-suxe3o-as-suposiuxe7uxf5es-ao-nuxedvel-do-estudo}

\begin{itemize}
\item
  Ausência de confundimento relevante não controlado: .{[}REF{]}
\item
  Estabilidade do processo gerador de dados (invariância temporal, populacional ou contextual): .{[}REF{]}
\end{itemize}

\section{Suposições implícitas e explícitas nos testes}\label{suposiuxe7uxf5es-impluxedcitas-e-expluxedcitas-nos-testes}

\subsection{Quais suposições implícitas são feitas nos testes estatísticos?}\label{quais-suposiuxe7uxf5es-impluxedcitas-suxe3o-feitas-nos-testes-estatuxedsticos}

\begin{itemize}
\item
  Amostragem aleatória ou ignorabilidade condicional: .{[}REF{]}
\item
  Medição sem erro relevante: .{[}REF{]}
\item
  Correspondência entre modelo estatístico e processo gerador de dados: .{[}REF{]}
\item
  Ausência de múltiplas comparações não ajustadas: .{[}REF{]}
\end{itemize}

\subsection{Quais suposições explícitas são feitas nos testes estatísticos?}\label{quais-suposiuxe7uxf5es-expluxedcitas-suxe3o-feitas-nos-testes-estatuxedsticos}

\begin{itemize}
\item
  Normalidade dos erros ou da estatística de teste: .{[}REF{]}
\item
  Homocedasticidade: .{[}REF{]}
\item
  Independência das observações: .{[}REF{]}
\end{itemize}

\section{Suposições causais que conectam dados observados a efeitos causais}\label{suposiuxe7uxf5es-causais-que-conectam-dados-observados-a-efeitos-causais}

\subsection{Quais são as suposições causais que conectam dados observados a efeitos causais?}\label{quais-suxe3o-as-suposiuxe7uxf5es-causais-que-conectam-dados-observados-a-efeitos-causais}

\begin{itemize}
\item
  Ausência de correlação espúria: associações observadas refletem relações sistemáticas e não flutuações aleatórias; quanto maior a amostra, mais plausível essa condição.\textsuperscript{\citeproc{ref-REF}{\textbf{REF?}}}
\item
  Consistência: os valores observados do tratamento correspondem a intervenções bem definidas e coincidem com os valores dos contrafactuais relevantes.\textsuperscript{\citeproc{ref-REF}{\textbf{REF?}}}
\item
  Intercambialidade: condicionalmente às covariáveis medidas, a atribuição do tratamento é independente dos desfechos potenciais.\textsuperscript{\citeproc{ref-REF}{\textbf{REF?}}}
\item
  Positividade: para todos os valores das covariáveis consideradas, a probabilidade de receber cada nível do tratamento é maior que zero.\textsuperscript{\citeproc{ref-REF}{\textbf{REF?}}}
\item
  Fidelidade: efeitos causais não se cancelam sistematicamente no agregado populacional, de modo que efeitos médios nulos correspondem à ausência de efeito causal relevante.\textsuperscript{\citeproc{ref-REF}{\textbf{REF?}}}
\end{itemize}

\subsection{Qual a relação dessas suposições com as demais suposições inferenciais?}\label{qual-a-relauxe7uxe3o-dessas-suposiuxe7uxf5es-com-as-demais-suposiuxe7uxf5es-inferenciais}

\begin{itemize}
\item
  Essas suposições operam antes do modelo estatístico.\textsuperscript{\citeproc{ref-REF}{\textbf{REF?}}}
\item
  Não são verificáveis por diagnóstico residual ou testes de ajuste.\textsuperscript{\citeproc{ref-REF}{\textbf{REF?}}}
\item
  Mesmo com todas as suposições estatísticas satisfeitas, a inferência causal pode falhar se qualquer uma dessas suposições não for atendida.\textsuperscript{\citeproc{ref-REF}{\textbf{REF?}}}
\end{itemize}

\section{Diagnóstico e verificação}\label{diagnuxf3stico-e-verificauxe7uxe3o}

\subsection{O que fazer quando suposições gerais falham?}\label{o-que-fazer-quando-suposiuxe7uxf5es-gerais-falham}

\begin{itemize}
\item
  Transformações: .{[}REF{]}
\item
  Métodos robustos (estimadores e testes): .{[}REF{]}
\item
  Reamostragem: .{[}REF{]}
\item
  Modelos alternativos: .{[}REF{]}
\end{itemize}

\subsection{O que fazer quando as suposições causais falham?}\label{o-que-fazer-quando-as-suposiuxe7uxf5es-causais-falham}

\begin{itemize}
\item
  Clarificar o alvo causal: redefinir a população, o tratamento ou o efeito de interesse.{[}REF{]}
\item
  Análise de sensibilidade: avaliar quanto confundimento não medido seria necessário para invalidar as conclusões.{[}REF{]}
\item
  Restringir o suporte: limitar a análise a regiões com positividade plausível (suporte comum).{[}REF{]}
\item
  Estratificação ou ajuste enriquecido: incluir covariáveis adicionais relevantes, quando disponíveis.{[}REF{]}
\item
  Modelagem causal explícita: usar DAGs para tornar suposições transparentes e discutíveis.{[}REF{]}
\item
  Estimativas parciais ou locais: reportar efeitos condicionais ou locais quando o efeito médio não é identificável.{[}REF{]}
\item
  Conclusões mais fracas: interpretar resultados como associações ajustadas, não como efeitos causais.{[}REF{]}
\item
  Relato explícito das falhas: documentar quais suposições não são plausíveis e por quê.{[}REF{]}
\end{itemize}

\begin{infobox}{images/Rlogo}
O pacote \emph{performance}\textsuperscript{\citeproc{ref-performance}{249}} fornece a função \href{https://www.rdocumentation.org/packages/performance/versions/0.10.4/topics/check_model}{\emph{check\_model}} para analisar a colinearidade entre variáveis, a normalidade da distribuição das variáveis e a heteroscedasticidade.

\end{infobox}

\subsection{Como avaliar as suposições de uma regressão?}\label{como-avaliar-as-suposiuxe7uxf5es-de-uma-regressuxe3o}

\begin{itemize}
\tightlist
\item
  Usando diagnóstico de regressão (ex.: análise de resíduos, gráficos de valores observados vs.~preditos) e comparação com análises estratificadas.\textsuperscript{\citeproc{ref-Greenland1989}{251}}
\end{itemize}

\begin{figure}

{\centering \includegraphics{Ciencia-com-R_files/figure-latex/suposicoes-1} 

}

\caption{Diagnóstico de regressão para avaliar suposições do modelo: linearidade, normalidade dos resíduos, homocedasticidade e alavancagem.}\label{fig:suposicoes}
\end{figure}

\chapter{\texorpdfstring{\textbf{Seleção de testes}}{Seleção de testes}}\label{selecao-testes}

\section{Multiverso de análises estatísticas}\label{multiverso-de-anuxe1lises-estatuxedsticas}

\subsection{Por que escolher o teste é um problema?}\label{por-que-escolher-o-teste-uxe9-um-problema}

\begin{itemize}
\item
  Analisar a mesma hipótese com o mesmo banco de dados pode resultar em diferenças substanciais nas estimativas estatísticas e nas conclusões.\textsuperscript{\citeproc{ref-Breznau2022}{252}}
\item
  As decisões para especificação das análises estatísticas podem ser tão minuciosas que muitas vezes nem sequer são registradas como decisões e, assim, podem impactar negativamente na reprodutibilidade do estudo.\textsuperscript{\citeproc{ref-Breznau2022}{252}}
\end{itemize}

\section{Escolha de testes para análise inferencial}\label{escolha-de-testes-para-anuxe1lise-inferencial}

\subsection{Como selecionar os testes para a análise estatística inferencial?}\label{como-selecionar-os-testes-para-a-anuxe1lise-estatuxedstica-inferencial}

\begin{itemize}
\item
  .\textsuperscript{\citeproc{ref-dwivedi2019}{253}}
\item
  .\textsuperscript{\citeproc{ref-Dwivedi2022}{254}}
\item
  .\textsuperscript{\citeproc{ref-Kim2017}{255}}
\item
  .\textsuperscript{\citeproc{ref-marusteri2010}{256}}
\item
  .\textsuperscript{\citeproc{ref-mishra2019}{257}}
\item
  .\textsuperscript{\citeproc{ref-ray2021}{258}}
\item
  .\textsuperscript{\citeproc{ref-nayak2011}{259}}
\item
  .\textsuperscript{\citeproc{ref-shankar2014}{260}}
\end{itemize}

\blandscape

\begin{longtblr}[         %% tabularray outer open
caption={Tabela de seleção de testes estatísticos: Descrição.},
]                     %% tabularray outer close
{                     %% tabularray inner open
width={1\linewidth},
colspec={X[0.17]X[0.17]X[0.17]X[0.17]X[0.16]X[0.16]},
hline{2}={1-6}{solid, black, 0.05em},
hline{1}={1-6}{solid, black, 0.1em},
hline{4}={1-6}{solid, black, 0.1em},
row{1}={}{font=\bfseries},
rowhead=1,
}                     %% tabularray inner close
Delineamento do estudo & Qtd. de variáveis / fatores & Níveis do fator & Relação entre amostras & Tipo de variáveis & Teste estatístico \\
Transversal & 1 variável & – & – & Contínua & Média (DP) IC95\% / Mediana (IIQ) IC95\% \\
Transversal & 1 variável & – & – & Categórica & Frequências e proporções \\
\end{longtblr}

\begin{longtblr}[         %% tabularray outer open
caption={Tabela de seleção de testes estatísticos: Comparação.},
]                     %% tabularray outer close
{                     %% tabularray inner open
width={1\linewidth},
colspec={X[0.17]X[0.17]X[0.17]X[0.17]X[0.16]X[0.16]},
hline{2}={1-6}{solid, black, 0.05em},
hline{1}={1-6}{solid, black, 0.1em},
hline{16}={1-6}{solid, black, 0.1em},
row{1}={}{font=\bfseries},
rowhead=1,
}                     %% tabularray inner close
Delineamento do estudo & Qtd. de variáveis / fatores & Níveis do fator & Relação entre amostras & Tipo de variáveis & Teste estatístico \\
Transversal & 1 variável & – & – & Contínua & t de Student (1 amostra) / Wilcoxon one-sample \\
Transversal & 1 variável & ≥ 2 & – & Categórica & Qui-quadrado de aderência \\
Transversal & 1 variável & – & – & Categórica dicotômica & Teste binomial (1 amostra) \\
Experimental / observacional & 1 fator + 1 resposta & 2 & Independentes & Contínua ~ categórica & t de Student independente / t de Welch \\
Pareado / longitudinal & 1 fator + 1 resposta & 2 & Dependentes & Contínua ~ categórica & t pareado \\
Experimental / observacional & 1 fator + 1 resposta & ≥ 3 & Independentes & Contínua ~ categórica & ANOVA one-way \\
Experimental / observacional & 1 fator + ≥ 1 covariável + 1 resposta & ≥ 2 & Independentes & Contínua ~ categórica + covariável & ANCOVA \\
Longitudinal & 1 fator + 1 resposta & ≥ 3 & Dependentes & Contínua ~ categórica & ANOVA de medidas repetidas \\
Experimental / observacional & 1 fator + 1 resposta & 2 & Independentes & Ordinal / não normal & Mann–Whitney U / Kruskal–Wallis \\
Longitudinal & 1 fator + 1 resposta & ≥ 3 & Dependentes & Ordinal / não normal & Wilcoxon pareado / Friedman \\
Experimental / observacional & 1 fator + ≥ 2 respostas & ≥ 2 & Independentes & Contínuas múltiplas & MANOVA \\
Experimental / observacional & 1 fator + 1 resposta & 2 & Independentes & Contínua / ordinal & Brunner–Munzel \\
Transversal & 2 correlações & – & Independentes & Contínua × contínua & Fisher r-to-z (correlações independentes) \\
Transversal & ≥ 3 variáveis contínuas & – & Dependentes & Contínua × contínua & Steiger / Meng–Rosenthal–Rubin (correlações dependentes) \\
\end{longtblr}

\begin{longtblr}[         %% tabularray outer open
caption={Tabela de seleção de testes estatísticos: Associação.},
]                     %% tabularray outer close
{                     %% tabularray inner open
width={1\linewidth},
colspec={X[0.17]X[0.17]X[0.17]X[0.17]X[0.16]X[0.16]},
hline{2}={1-6}{solid, black, 0.05em},
hline{1}={1-6}{solid, black, 0.1em},
hline{10}={1-6}{solid, black, 0.1em},
row{1}={}{font=\bfseries},
rowhead=1,
}                     %% tabularray inner close
Delineamento do estudo & Qtd. de variáveis / fatores & Níveis do fator & Relação entre amostras & Tipo de variáveis & Teste estatístico \\
Transversal & 2 variáveis & – & – & Contínua × contínua & Correlação de Pearson \\
Transversal & 2 variáveis & – & – & Ordinal / não normal & Correlação de Spearman / Kendall \\
Transversal & 2 variáveis & – & – & Ordinal × ordinal & Gamma de Goodman–Kruskal \\
Transversal & 2 variáveis & – & – & Ordinal × ordinal & Tau-b de Kendall \\
Transversal & 2 variáveis & – & – & Ordinal × ordinal & Somers’ D \\
Transversal & 2 variáveis & – & – & Categórica × categórica (2×2) & Qui-quadrado / Fisher + Phi (φ) \\
Transversal & 2 variáveis & – & – & Categórica × categórica (freq. pequenas) & Exato de Fisher \\
Transversal & 2 variáveis & – & – & Categórica × categórica (R×C) & Qui-quadrado + V de Cramér \\
\end{longtblr}

\begin{longtblr}[         %% tabularray outer open
caption={Tabela de seleção de testes estatísticos: Predição.},
]                     %% tabularray outer close
{                     %% tabularray inner open
width={1\linewidth},
colspec={X[0.17]X[0.17]X[0.17]X[0.17]X[0.16]X[0.16]},
hline{2}={1-6}{solid, black, 0.05em},
hline{1}={1-6}{solid, black, 0.1em},
hline{4}={1-6}{solid, black, 0.1em},
row{1}={}{font=\bfseries},
rowhead=1,
}                     %% tabularray inner close
Delineamento do estudo & Qtd. de variáveis / fatores & Níveis do fator & Relação entre amostras & Tipo de variáveis & Teste estatístico \\
Observacional / experimental & ≥ 1 preditor + 1 resposta & – & – & Contínua & Regressão linear \\
Observacional / experimental & ≥ 1 preditor + 1 resposta & – & – & Binária & Regressão logística \\
\end{longtblr}

\begin{longtblr}[         %% tabularray outer open
caption={Tabela de seleção de testes estatísticos: Contagem.},
]                     %% tabularray outer close
{                     %% tabularray inner open
width={1\linewidth},
colspec={X[0.17]X[0.17]X[0.17]X[0.17]X[0.16]X[0.16]},
hline{2}={1-6}{solid, black, 0.05em},
hline{1}={1-6}{solid, black, 0.1em},
hline{3}={1-6}{solid, black, 0.1em},
row{1}={}{font=\bfseries},
rowhead=1,
}                     %% tabularray inner close
Delineamento do estudo & Qtd. de variáveis / fatores & Níveis do fator & Relação entre amostras & Tipo de variáveis & Teste estatístico \\
Observacional & ≥ 1 preditor + 1 resposta & – & – & Contagem & Poisson / Binomial negativa \\
\end{longtblr}

\begin{longtblr}[         %% tabularray outer open
caption={Tabela de seleção de testes estatísticos: Sobrevida.},
]                     %% tabularray outer close
{                     %% tabularray inner open
width={1\linewidth},
colspec={X[0.17]X[0.17]X[0.17]X[0.17]X[0.16]X[0.16]},
hline{2}={1-6}{solid, black, 0.05em},
hline{1}={1-6}{solid, black, 0.1em},
hline{4}={1-6}{solid, black, 0.1em},
row{1}={}{font=\bfseries},
rowhead=1,
}                     %% tabularray inner close
Delineamento do estudo & Qtd. de variáveis / fatores & Níveis do fator & Relação entre amostras & Tipo de variáveis & Teste estatístico \\
Longitudinal & ≥ 1 preditor + tempo & ≥ 2 & Independentes & Tempo & Log-rank / Modelo de Cox \\
Longitudinal & ≥ 1 preditor + tempo & ≥ 2 & Independentes & Tempo (riscos competitivos) & Modelo de riscos competitivos de Fine–Gray \\
\end{longtblr}

\begin{longtblr}[         %% tabularray outer open
caption={Tabela de seleção de testes estatísticos: Concordância.},
]                     %% tabularray outer close
{                     %% tabularray inner open
width={1\linewidth},
colspec={X[0.17]X[0.17]X[0.17]X[0.17]X[0.16]X[0.16]},
hline{2}={1-6}{solid, black, 0.05em},
hline{1}={1-6}{solid, black, 0.1em},
hline{6}={1-6}{solid, black, 0.1em},
row{1}={}{font=\bfseries},
rowhead=1,
}                     %% tabularray inner close
Delineamento do estudo & Qtd. de variáveis / fatores & Níveis do fator & Relação entre amostras & Tipo de variáveis & Teste estatístico \\
Metodológico & 2 avaliadores & – & Dependentes & Categórica nominal & Kappa de Cohen \\
Metodológico & ≥ 2 avaliadores & – & Dependentes & Categórica nominal & Kappa de Fleiss \\
Metodológico & 2 avaliadores & ≥ 3 & Dependentes & Categórica ordinal & Kappa de Light \\
Metodológico & 2 medidas & – & Dependentes & Contínua & Coeficiente de Correlação Intraclasse (ICC) \\
\end{longtblr}

\begin{longtblr}[         %% tabularray outer open
caption={Tabela de seleção de testes estatísticos: Ajuste multivariado.},
]                     %% tabularray outer close
{                     %% tabularray inner open
width={1\linewidth},
colspec={X[0.17]X[0.17]X[0.17]X[0.17]X[0.16]X[0.16]},
hline{2}={1-6}{solid, black, 0.05em},
hline{1}={1-6}{solid, black, 0.1em},
hline{3}={1-6}{solid, black, 0.1em},
row{1}={}{font=\bfseries},
rowhead=1,
}                     %% tabularray inner close
Delineamento do estudo & Qtd. de variáveis / fatores & Níveis do fator & Relação entre amostras & Tipo de variáveis & Teste estatístico \\
Longitudinal / clusterizado & ≥ 1 fator + resposta & – & Dependentes / clusters & Contínua / categórica & Modelos lineares mistos (LMM / GLMM) \\
\end{longtblr}

\begin{longtblr}[         %% tabularray outer open
caption={Tabela de seleção de testes estatísticos: Desempenho diagnóstico.},
]                     %% tabularray outer close
{                     %% tabularray inner open
width={1\linewidth},
colspec={X[0.17]X[0.17]X[0.17]X[0.17]X[0.16]X[0.16]},
hline{2}={1-6}{solid, black, 0.05em},
hline{1}={1-6}{solid, black, 0.1em},
hline{6}={1-6}{solid, black, 0.1em},
row{1}={}{font=\bfseries},
rowhead=1,
}                     %% tabularray inner close
Delineamento do estudo & Qtd. de variáveis / fatores & Níveis do fator & Relação entre amostras & Tipo de variáveis & Teste estatístico \\
Transversal / longitudinal & 1 teste + 1 padrão-ouro & 2 & Pareadas & Binária × binária & Sensibilidade, Especificidade, VPP, VPN, Acurácia \\
Transversal / longitudinal & 1 teste + 1 padrão-ouro & 2 & Pareadas & Binária × binária & Razões de verossimilhança (LR+ / LR−) \\
Transversal / longitudinal & 1 escore contínuo + 1 desfecho & – & Pareadas & Contínua × binária & Curva ROC + AUC (IC95\%) \\
Transversal / longitudinal & 1 modelo + 1 desfecho & – & Pareadas & Probabilidade × binária & ROC / AUC + calibração (Brier, Hosmer–Lemeshow) \\
\end{longtblr}

\begin{longtblr}[         %% tabularray outer open
caption={Tabela de seleção de testes estatísticos: Diagnóstico.},
]                     %% tabularray outer close
{                     %% tabularray inner open
width={1\linewidth},
colspec={X[0.17]X[0.17]X[0.17]X[0.17]X[0.16]X[0.16]},
hline{2}={1-6}{solid, black, 0.05em},
hline{1}={1-6}{solid, black, 0.1em},
hline{12}={1-6}{solid, black, 0.1em},
row{1}={}{font=\bfseries},
rowhead=1,
}                     %% tabularray inner close
Delineamento do estudo & Qtd. de variáveis / fatores & Níveis do fator & Relação entre amostras & Tipo de variáveis & Teste estatístico \\
Transversal & 1 variável & – & – & Contínua & Shapiro–Wilk \\
Transversal & 1 variável & – & – & Contínua & Kolmogorov–Smirnov / Lilliefors \\
Transversal & 1 variável & – & – & Contínua & Anderson–Darling \\
Transversal & 1 variável & – & – & Contínua & Jarque–Bera \\
Transversal & ≥ 2 variáveis & – & – & Contínuas múltiplas & Teste de Mardia (assimetria e curtose) \\
Transversal & ≥ 2 variáveis & – & – & Contínuas múltiplas & Henze–Zirkler \\
Transversal & ≥ 2 variáveis & – & – & Contínuas múltiplas & Royston (Shapiro–Wilk multivariado) \\
Transversal & 1 fator + 1 resposta & ≥ 2 & Independentes & Contínua ~ categórica & Levene \\
Transversal & 1 fator + 1 resposta & ≥ 2 & Independentes & Contínua ~ categórica & Brown–Forsythe \\
Transversal & 1 fator + 1 resposta & ≥ 2 & Independentes & Contínua ~ categórica & Bartlett (quando normalidade é plausível) \\
\end{longtblr}

\begin{longtblr}[         %% tabularray outer open
caption={Tabela de seleção de testes estatísticos: Transição de estados.},
]                     %% tabularray outer close
{                     %% tabularray inner open
width={1\linewidth},
colspec={X[0.17]X[0.17]X[0.17]X[0.17]X[0.16]X[0.16]},
hline{2}={1-6}{solid, black, 0.05em},
hline{1}={1-6}{solid, black, 0.1em},
hline{5}={1-6}{solid, black, 0.1em},
row{1}={}{font=\bfseries},
rowhead=1,
}                     %% tabularray inner close
Delineamento do estudo & Qtd. de variáveis / fatores & Níveis do fator & Relação entre amostras & Tipo de variáveis & Teste estatístico \\
Longitudinal & ≥ 2 estados ao longo do tempo & ≥ 2 & Dependentes (medidas repetidas) & Categórica (estados) & Modelo de Markov (tempo discreto) \\
Longitudinal & Estados + covariáveis & ≥ 2 & Dependentes (medidas repetidas) & Estados categóricos + preditores & Modelo de Markov com covariáveis / HMM \\
Longitudinal & Estados + tempo contínuo & ≥ 2 & Dependentes (medidas repetidas) & Estados categóricos ao longo do tempo & Modelo de Markov em tempo contínuo (CTMC) \\
\end{longtblr}

\elandscape

\chapter{\texorpdfstring{\textbf{Análise inferencial}}{Análise inferencial}}\label{analise-inferencial}

\section{Raciocínio inferencial}\label{raciocuxednio-inferencial}

\subsection{O que é análise inferencial?}\label{o-que-uxe9-anuxe1lise-inferencial}

\begin{itemize}
\item
  Na análise inferencial são utilizados dados da(s) amostra(s) para fazer uma inferência válida (isto é, estimativa) sobre os parâmetros populacionais desconhecidos.\textsuperscript{\citeproc{ref-vetter2017}{108}}
\item
  No paradigma de Jerzy Neyman e Egon Pearson, um teste de hipótese científica envolve a tomada de decisão sobre hipóteses nulas (\(H_{0}\)) e alternativa (\(H_{1}\)) concorrentes e mutuamente exclusivas.\textsuperscript{\citeproc{ref-Curran-Everett2009}{261}}
\end{itemize}

\subsection{Quais são os tipos de raciocínio inferencial?}\label{quais-suxe3o-os-tipos-de-raciocuxednio-inferencial}

\begin{itemize}
\item
  Inferência dedutiva: Uma dada hipótese inicial é utilizada para prever o que seria observado caso tal hipótese fosse verdadeira.\textsuperscript{\citeproc{ref-goodman1999}{262}}
\item
  Inferência indutiva: Com base nos dados observados, avalia-se qual hipótese é mais defensável (isto é, mais provável).\textsuperscript{\citeproc{ref-goodman1999}{262}}
\end{itemize}

\subsection{Quais são as questões fundamentais da análise inferencial?}\label{quais-suxe3o-as-questuxf5es-fundamentais-da-anuxe1lise-inferencial}

\begin{itemize}
\item
  A direção do efeito\textsuperscript{\citeproc{ref-mccaskey2015}{263}}
\item
  A magnitude do efeito\textsuperscript{\citeproc{ref-mccaskey2015}{263}}
\item
  A importância do efeito\textsuperscript{\citeproc{ref-mccaskey2015}{263}}
\end{itemize}

\section{Hipóteses científicas}\label{hipuxf3teses-cientuxedficas}

\subsection{O que é hipótese científica?}\label{o-que-uxe9-hipuxf3tese-cientuxedfica}

\begin{itemize}
\item
  Hipótese científica é uma ideia que pode ser testada.\textsuperscript{\citeproc{ref-Curran-Everett2009}{261}}
\item
  Definir claramente os problemas e os objetivos da pesquisa são o ponto de partida de todos os estudos científicos.\textsuperscript{\citeproc{ref-van2022a}{133}}
\item
  Além do papel técnico, os testes de hipótese carregam uma dimensão interpretativa que molda como os pesquisadores comunicam descobertas. Estudos recentes destacam o caráter pragmático e dicotômico dessas decisões.\textsuperscript{\citeproc{ref-uyguntunuxe72023}{264}}
\end{itemize}

\subsection{Quais são as fontes de ideias para gerar hipóteses científicas?}\label{quais-suxe3o-as-fontes-de-ideias-para-gerar-hipuxf3teses-cientuxedficas}

\begin{itemize}
\item
  Revisão das práticas atuais.\textsuperscript{\citeproc{ref-Vandenbroucke2018}{265}}
\item
  Desafio a ideias aceitas.\textsuperscript{\citeproc{ref-Vandenbroucke2018}{265}}
\item
  Conflito entre ideias divergentes.\textsuperscript{\citeproc{ref-Vandenbroucke2018}{265}}
\item
  Variações regionais, temporais e populacionais.\textsuperscript{\citeproc{ref-Vandenbroucke2018}{265}}
\item
  Experiências dos próprios pesquisadores.\textsuperscript{\citeproc{ref-Vandenbroucke2018}{265}}
\item
  Imaginação sem fronteiras ou limites convencionais.\textsuperscript{\citeproc{ref-Vandenbroucke2018}{265}}
\end{itemize}

\section{Hipóteses estatísticas}\label{hipuxf3teses-estatuxedsticas}

\subsection{O que é hipótese nula?}\label{o-que-uxe9-hipuxf3tese-nula}

\begin{itemize}
\tightlist
\item
  A hipótese nula (\(H_{0}\)) é uma expressão que representa o estado atual do conhecimento (\emph{status quo}), em geral a não existência de um determinado efeito.\textsuperscript{\citeproc{ref-kanji2006}{179}}
\end{itemize}

\subsection{O que é hipótese alternativa?}\label{o-que-uxe9-hipuxf3tese-alternativa}

\begin{itemize}
\tightlist
\item
  A hipótese alternativa (\(H_{1}\)) é uma expressão que contém as situações que serão testadas, de modo que um resultado positivo indique alguma ação a ser conduzida.\textsuperscript{\citeproc{ref-kanji2006}{179}}
\end{itemize}

\subsection{Qual hipótese está sendo testada?}\label{qual-hipuxf3tese-estuxe1-sendo-testada}

\begin{itemize}
\item
  A hipótese nula (\(H_{0}\)) é a hipótese sob teste em análises inferenciais.\textsuperscript{\citeproc{ref-Ali2016}{109}}
\item
  Pode-se concluir sobre rejeitar ou não rejeitar a hipótese nula (\(H_{0}\)).\textsuperscript{\citeproc{ref-Ali2016}{109}}
\item
  Não se conclui sobre a hipótese alternativa (\(H_{1}\)).\textsuperscript{\citeproc{ref-kanji2006}{179}}
\item
  Para testar a hipótese nula, deve-se selecionar o nível de significância crítica (P-valor de corte); a probabilidade de rejeitarmos uma hipótese nula verdadeira (\(\alpha\)); e a probabilidade de não rejeitarmos uma hipótese nula falsa (\(\beta\)).\textsuperscript{\citeproc{ref-Curran-Everett2009}{261}}
\end{itemize}

\section{Testes de hipóteses}\label{testes-de-hipuxf3teses}

\subsection{Quais são os tipos de teste de hipóteses?}\label{quais-suxe3o-os-tipos-de-teste-de-hipuxf3teses}

\begin{itemize}
\item
  Teste (clássico) de significância da hipótese nula: verifica evidência contra \(H_{0}\) usando P-valor.\textsuperscript{\citeproc{ref-lakens2018}{266}}
\item
  Teste de mínimos efeitos (MOTE/MOST/SESOI): testa se o efeito é pelo menos tão grande quanto um limiar de relevância (SESOI). Rejeitar \(H_{0}\) sugere efeito grande o suficiente.\textsuperscript{\citeproc{ref-lakens2018}{266}}
\item
  Teste de equivalência (TOST): testa se o efeito está dentro de uma margem de equivalência clinicamente irrelevante (entre \(\Delta\) e \(-\Delta\)). Rejeitar \(H_{0}\) sugere equivalência prática.\textsuperscript{\citeproc{ref-lakens2018}{266}}
\item
  Teste de superioridade: avalia se um tratamento/intervenção supera o controle por uma margem \(>0\) ou \(>\Delta\).\textsuperscript{\citeproc{ref-lakens2018}{266}}
\item
  Teste de não-inferioridade: avalia se o tratamento não é pior que o controle por mais do que \(-\Delta\).\textsuperscript{\citeproc{ref-REF}{\textbf{REF?}}}
\item
  Teste de inferioridade: avalia se o tratamento é pior que o controle (por exemplo, para checar segurança).\textsuperscript{\citeproc{ref-REF}{\textbf{REF?}}}
\end{itemize}

\subsection{O que reportar após um teste de hipótese?}\label{o-que-reportar-apuxf3s-um-teste-de-hipuxf3tese}

\begin{itemize}
\item
  P-valores, como estimativa da significância estatística.\textsuperscript{\citeproc{ref-Sullivan2012}{267}}
\item
  Tamanho do efeito, como estimativa de significância substantiva.\textsuperscript{\citeproc{ref-Sullivan2012}{267}}
\end{itemize}

\section{Intervalos de confiança e raciocínio de longo prazo}\label{intervalos-de-confianuxe7a-e-raciocuxednio-de-longo-prazo}

\subsection{O que é um intervalo de confiança?}\label{o-que-uxe9-um-intervalo-de-confianuxe7a}

\begin{itemize}
\item
  Um intervalo de confiança é um procedimento inferencial utilizado para estimar um parâmetro populacional desconhecido a partir de dados amostrais, levando em conta a variabilidade inerente ao processo de amostragem.\textsuperscript{\citeproc{ref-neyman1937}{268}}
\item
  Diferentemente de uma estatística descritiva, o intervalo de confiança não é uma propriedade fixa do parâmetro, mas uma propriedade do procedimento estatístico utilizado para estimá-lo.\textsuperscript{\citeproc{ref-goodman2016}{269}}
\item
  O parâmetro populacional é considerado fixo (embora desconhecido), enquanto o intervalo de confiança é uma quantidade aleatória, pois depende da amostra observada.\textsuperscript{\citeproc{ref-REF}{\textbf{REF?}}}
\end{itemize}

\subsection{Intervalos de confiança e a lógica frequentista}\label{intervalos-de-confianuxe7a-e-a-luxf3gica-frequentista}

\begin{itemize}
\item
  No paradigma frequentista, a probabilidade é interpretada como uma frequência relativa observável no longo prazo, associada a um processo repetível.\textsuperscript{\citeproc{ref-neyman1937}{268}}
\item
  Um intervalo de confiança de nível \((1-\alpha)\) é construído de modo que, se o mesmo procedimento de amostragem e análise fosse repetido indefinidamente sob as mesmas condições, aproximadamente \((1-\alpha)\times100\%\) dos intervalos assim gerados conteriam o verdadeiro valor do parâmetro populacional.\textsuperscript{\citeproc{ref-goodman2016}{269}}
\item
  Portanto, a probabilidade associada ao intervalo de confiança refere-se ao desempenho do procedimento no longo prazo, e não à probabilidade de o parâmetro estar contido em um intervalo específico observado.
\end{itemize}

\begin{figure}

{\centering \includegraphics{Ciencia-com-R_files/figure-latex/intervalo-confianca-long-run-1} 

}

\caption{Simulação ilustrativa de intervalos de confiança (IC) em 100 experimentos independentes, cada um com 1.000 observações amostradas de uma população normal padrão (média = 0, desvio-padrão = 1). Os ICs são construídos no nível de 95\%. O gráfico superior mostra os ICs individuais para cada experimento, indicando se o IC cobre ou não a média verdadeira ($\mu = 0$). O gráfico inferior apresenta a distribuição das médias amostrais obtidas nos experimentos, juntamente com o IC teórico para a média populacional. Observe que aproximadamente 95\% dos ICs individuais cobrem a média verdadeira, ilustrando o conceito de cobertura no longo prazo associado aos intervalos de confiança.}\label{fig:intervalo-confianca-long-run}
\end{figure}

\subsection{O que um intervalo de confiança não representa}\label{o-que-um-intervalo-de-confianuxe7a-nuxe3o-representa}

\begin{itemize}
\item
  Um intervalo de confiança não deve ser interpretado como a probabilidade de que o parâmetro esteja contido naquele intervalo específico.\textsuperscript{\citeproc{ref-greenland2016}{270}}
\item
  Após os dados terem sido observados e o intervalo calculado, o parâmetro populacional ou está dentro do intervalo ou não está; não há incerteza probabilística sobre isso no sentido frequentista.\textsuperscript{\citeproc{ref-greenland2016}{270}}
\item
  A incerteza expressa pelo intervalo de confiança refere-se à variabilidade do processo inferencial, e não a uma distribuição de probabilidade do parâmetro.\textsuperscript{\citeproc{ref-greenland2016}{270}}
\end{itemize}

\subsection{Relação entre intervalos de confiança e testes de hipóteses}\label{relauxe7uxe3o-entre-intervalos-de-confianuxe7a-e-testes-de-hipuxf3teses}

\begin{itemize}
\item
  Intervalos de confiança e testes de hipótese frequentistas são derivados do mesmo modelo probabilístico subjacente e utilizam as mesmas suposições estatísticas.\textsuperscript{\citeproc{ref-cumming2005}{271}}
\item
  Em testes bicaudais, qualquer valor hipotético do parâmetro que esteja fora do intervalo de confiança de nível \((1-\alpha)\) corresponde a uma hipótese nula que seria rejeitada ao nível de significância \(\alpha\).\textsuperscript{\citeproc{ref-cumming2005}{271}}
\item
  De forma análoga, valores do parâmetro que estejam dentro do intervalo de confiança correspondem a hipóteses nulas para as quais não haveria evidência suficiente para rejeição ao nível \(\alpha\).\textsuperscript{\citeproc{ref-cumming2005}{271}}
\item
  Apesar dessa equivalência formal, intervalos de confiança e testes de hipótese respondem a perguntas distintas: testes enfatizam decisões dicotômicas, enquanto intervalos de confiança enfatizam estimação e incerteza.\textsuperscript{\citeproc{ref-cumming2005}{271}}
\end{itemize}

\subsection{Por que intervalos de confiança são centrais na inferência científica}\label{por-que-intervalos-de-confianuxe7a-suxe3o-centrais-na-inferuxeancia-cientuxedfica}

\begin{itemize}
\item
  Intervalos de confiança permitem avaliar simultaneamente a magnitude do efeito e a incerteza associada à sua estimativa, na mesma unidade de medida da variável de interesse.\textsuperscript{\citeproc{ref-cumming2005}{271}}
\item
  Essa abordagem favorece interpretações substantivas e científicas dos resultados, em oposição a decisões puramente dicotômicas baseadas em pontos de corte arbitrários.\textsuperscript{\citeproc{ref-cumming2005}{271}}
\item
  Quando corretamente interpretados, intervalos de confiança promovem uma comunicação mais informativa da evidência científica e reduzem interpretações equivocadas associadas ao uso exclusivo de P-valores.\textsuperscript{\citeproc{ref-greenland2016}{270}}
\end{itemize}

\section{Comparações múltiplas}\label{comparauxe7uxf5es-muxfaltiplas}

\subsection{O que é uma família de hipóteses?}\label{o-que-uxe9-uma-famuxedlia-de-hipuxf3teses}

\begin{itemize}
\item
  Família de hipóteses é um conjunto de comparações/inferências que, por desenho ou análise, devem ser consideradas juntas para controle do erro tipo I global (ex.: todas as comparações de um desfecho primário, todos os subgrupos pré-especificados, todos os desfechos coprimários).\textsuperscript{\citeproc{ref-REF}{\textbf{REF?}}}
\item
  O controle do \emph{family-wise error rate} (FWER) ou do \emph{false discovery rate} (FDR) deve considerar a família pertinente, não comparações isoladas.\textsuperscript{\citeproc{ref-REF}{\textbf{REF?}}}
\end{itemize}

\subsection{\texorpdfstring{O que são testes \emph{ad hoc} e \emph{post hoc}?}{O que são testes ad hoc e post hoc?}}\label{o-que-suxe3o-testes-ad-hoc-e-post-hoc}

\begin{itemize}
\item
  \emph{Ad hoc}: análises/decisões não planejadas a priori, motivadas por inspeção dos dados. Úteis para gerar hipóteses, não para confirmá-las.\textsuperscript{\citeproc{ref-REF}{\textbf{REF?}}}
\item
  \emph{Post hoc}: procedimentos de comparações múltiplas aplicados após um teste global ter indicado efeito significativo. Visam controlar o erro tipo I em múltiplas comparações.\textsuperscript{\citeproc{ref-REF}{\textbf{REF?}}}
\end{itemize}

\subsection{Como ajustar a análise inferencial para hipóteses múltiplas?}\label{como-ajustar-a-anuxe1lise-inferencial-para-hipuxf3teses-muxfaltiplas}

\begin{itemize}
\item
  Defina a família (o que entra no ajuste) e priorize desfechos (primário, coprimários, secundários).\textsuperscript{\citeproc{ref-REF}{\textbf{REF?}}}
\item
  Aplique métodos de controle FWER (Bonferroni, Holm, Hochberg, Dunnett para múltiplos vs.~controle) ou controle FDR (Benjamini--Hochberg) conforme o objetivo (confirmação vs.~exploração).\textsuperscript{\citeproc{ref-REF}{\textbf{REF?}}}
\item
  Em planos confirmatórios, use hierarquização/gatekeeping: testa-se em sequência; a alocação de \(\alpha\) passa adiante apenas se houver significância no nível anterior.\textsuperscript{\citeproc{ref-REF}{\textbf{REF?}}}
\end{itemize}

\begin{infobox}{images/Rlogo}
O pacote \emph{stats}\textsuperscript{\citeproc{ref-stats}{131}} fornece a função \href{https://www.rdocumentation.org/packages/stats/versions/3.6.2/topics/p.adjust}{\emph{p.adjust}} para ajustar o P-valor utilizando diversos métodos.

\end{infobox}

\subsection{O que são testes unicaudais e bicaudais?}\label{o-que-suxe3o-testes-unicaudais-e-bicaudais}

\begin{itemize}
\tightlist
\item
  Teste unicaudal procura evidência em uma direção específica (ex.: ``maior que 0''). Toda a região crítica está numa só cauda; tem maior poder para aquela direção, mas não detecta sinal oposto.\textsuperscript{\citeproc{ref-REF}{\textbf{REF?}}}
\end{itemize}

\begin{figure}

{\centering \includegraphics{Ciencia-com-R_files/figure-latex/unicaudal-direita-1} 

}

\caption{Representação gráfica de um teste de hipótese unicaudal à direita, aplicado quando se busca evidência de efeitos positivos (valores significativamente maiores que o esperado sob $H_0$).}\label{fig:unicaudal-direita}
\end{figure}

\begin{figure}

{\centering \includegraphics{Ciencia-com-R_files/figure-latex/unicaudal-esquerda-1} 

}

\caption{Representação gráfica de um teste de hipótese unicaudal à esquerda, aplicado quando se busca evidência de efeitos negativos (valores significativamente menores que o esperado sob $H_0$).}\label{fig:unicaudal-esquerda}
\end{figure}

\begin{itemize}
\tightlist
\item
  Teste bicaudal procura evidência em qualquer direção (ex.: ``diferente de 0''). Divide \(\alpha\) em duas caudas (direita e esquerda). É a escolha padrão quando ambas as direções são plausíveis.\textsuperscript{\citeproc{ref-REF}{\textbf{REF?}}}
\end{itemize}

\begin{figure}

{\centering \includegraphics{Ciencia-com-R_files/figure-latex/bicaudal-1} 

}

\caption{Representação gráfica de um teste de hipótese bicaudal, aplicado quando se busca evidência de efeitos positivos ou negativos (valores significativamente diferentes do esperado sob $H_0$).}\label{fig:bicaudal}
\end{figure}

\section{Inferência visual}\label{inferuxeancia-visual}

\subsection{O que é inferência visual?}\label{o-que-uxe9-inferuxeancia-visual}

\begin{itemize}
\item
  Inferência visual consiste na interpretação de dados apresentados em gráficos.\textsuperscript{\citeproc{ref-cumming2005}{271}}
\item
  Para inferência visual, recomenda-se a apresentação dos dados em gráficos com estimativas de tendência central e seu intervalo (preferencialmete intervalo de confiança no nível de significância \(\alpha\) pré-estabelecido).\textsuperscript{\citeproc{ref-cumming2005}{271}}
\end{itemize}

\subsection{Por que usar intervalos de confiança para inferência visual?}\label{por-que-usar-intervalos-de-confianuxe7a-para-inferuxeancia-visual}

\begin{itemize}
\item
  Intervalos de confiança fornecem estimativas pontuais e intervalares na mesma unidade de medida da variável.\textsuperscript{\citeproc{ref-cumming2005}{271}}
\item
  Existe uma relação entre o intervalo de confiança e o P-valor obtido pelo teste de significância de hipótese nula \(H_{0}\), em que ambos consideram o mesmo nível de significância \(\alpha\) pré-estabelecido.\textsuperscript{\citeproc{ref-cumming2005}{271}}
\end{itemize}

\subsection{Como interpretar intervalos de confiança em uma figura?}\label{como-interpretar-intervalos-de-confianuxe7a-em-uma-figura}

\begin{itemize}
\item
  Identifique o que as tendências centrais e as barras de erro representam. Qual é a variável dependente? É expressa em unidades originais ou é padronizada ? A figura mostra intervalos de confiança, erro-padrão ou desvio-padrão? Qual é o desenho experimental?\textsuperscript{\citeproc{ref-cumming2005}{271}}
\item
  Faça uma interpretação substantiva dos valores de tendência central e dos intervalos de confiança.\textsuperscript{\citeproc{ref-cumming2005}{271}}
\item
  O intervalo de confiança é uma faixa de valores plausíveis para a tendência central. Valores fora do intervalo são relativamente implausíveis, no nível de significância \(\alpha\) pré-estabelecido.\textsuperscript{\citeproc{ref-cumming2005}{271}}
\item
  Qualquer valor fora do intervalo de confiança, quando considerado como hipótese nula (\(H_{0}\)), equivale a \(P < \alpha\) pré-estabelecido (bicaudal).\textsuperscript{\citeproc{ref-cumming2005}{271}}
\item
  Qualquer valor dentro do intervalo, quando considerado como hipótese nula (\(H_{0}\)), equivale a \(P > \alpha\) pré-estabelecido (bicaudal).\textsuperscript{\citeproc{ref-cumming2005}{271}}
\end{itemize}

\section{Interpretação de análise inferencial}\label{interpretauxe7uxe3o-de-anuxe1lise-inferencial}

\subsection{Como interpretar uma análise inferencial?}\label{como-interpretar-uma-anuxe1lise-inferencial}

\begin{itemize}
\item
  Testes de hipótese nula (\(H_{0}\)) vs.~alternativa (\(H_{1}\)) a partir de um nível de significância (\(\alpha\)) pré-especificado.\textsuperscript{\citeproc{ref-goodman2016}{269}}
\item
  P-valor como evidência estatística sobre (\(H_{0}\)).\textsuperscript{\citeproc{ref-goodman2016}{269}}
\item
  Estimação de intervalos de confiança de um nível de significância (\(\alpha\)) pré-especificado bicaudal (\(IC_{1-\alpha/2}\)) ou unicaudal (\(IC_{1-\alpha}\)).\textsuperscript{\citeproc{ref-goodman2016}{269}}
\item
  Análise Bayesiana.\textsuperscript{\citeproc{ref-goodman2016}{269}}
\end{itemize}

\subsection{O que são resultados positivos'' e ``negativos'' (inconclusivos) em teste de hipótese?}\label{o-que-suxe3o-resultados-positivos-e-negativos-inconclusivos-em-teste-de-hipuxf3tese}

\begin{itemize}
\item
  Resultados ``positivos'' compreendem um P-valor dentro da zona crítica estatisticamente significativa (ex.: \(P<0,05\) ou outro ponto de corte) e sugerem que os autores rejeitem a hipótese nula (\(H_{0}\)), confirmando assim sua hipótese científica.\textsuperscript{\citeproc{ref-greenhalgh1997a}{272}}
\item
  Resultados ``negativos'' ou inconclusivos compreendem um P-valor fora da zona crítica estatisticamente significativa (ex.: \(P \geq 0,05\) ou outro ponto de corte) e sugerem que os autores não rejeitem a hipótese nula (\(H_{0}\)) porque o efeito observado é nulo (logo, ``negativo''), ou porque o estudo não possui poder suficiente para detectá-lo, não permitindo portanto afirmar a hipótese científica (logo, inconclusivo).\textsuperscript{\citeproc{ref-greenhalgh1997a}{272}}
\end{itemize}

\subsection{Qual a importância de resultados ``negativos''?}\label{qual-a-importuxe2ncia-de-resultados-negativos}

\begin{itemize}
\item
  Conhecer resultados negativos contribui com uma visão mais ampla do campo de estudo junto aos resultados positivos.\textsuperscript{\citeproc{ref-weintraub2016}{273}}
\item
  Resultados negativos permitem um melhor planejamento das pesquisas futuras e pode aumentar suas chances de sucesso.\textsuperscript{\citeproc{ref-weintraub2016}{273}}
\end{itemize}

\subsection{Resultados inconclusivos: Ausência de evidência ou evidência de ausência?}\label{resultados-inconclusivos-ausuxeancia-de-eviduxeancia-ou-eviduxeancia-de-ausuxeancia}

\begin{itemize}
\item
  Em estudos (geralmente com amostras grandes), resultados estatisticamente significativos (com P-valores menores do limiar pré-estabelecido, \(P<\alpha\)) podem não ser clinicamente relevantes.\textsuperscript{\citeproc{ref-altman1995}{274}}
\item
  Em estudos (geralmente com amostras pequenas), resultados estatisticamente não significativos (com P-valores iguais ou maiores do limiar pré-estabelecido, \(P \geq \alpha\)) não devem ser interpretados como evidência de inexistência do efeito.\textsuperscript{\citeproc{ref-altman1995}{274}}
\item
  Geralmente é razoável aceitar uma nova conclusão apenas quando há dados a seu favor (`resultados positivos'). Também é razoável questionar se apenas a ausência de dados a seu favor (`resultados negativos') justifica suficientemente a rejeição de tal conclusão.\textsuperscript{\citeproc{ref-altman1995}{274}}
\item
  A prática estatística convencional tende a reduzir a incerteza científica a decisões docotômicas. Essa simplificação possui implicações epistemológicas importantes: os testes de hipótese produzem não apenas juízos empíricos, mas também atos pragmáticos que comunicam graus de confiança e orientam ações.\textsuperscript{\citeproc{ref-uyguntunuxe72023}{264}}
\end{itemize}

\section{Erros de inferência I, II, S e M}\label{erros-de-inferuxeancia-i-ii-s-e-m}

\subsection{O que são erros de inferência estatística?}\label{o-que-suxe3o-erros-de-inferuxeancia-estatuxedstica}

\begin{itemize}
\tightlist
\item
  Um erro de inferência é a tomada de decisão incorreta, seja a favor ou contra a hipótese nula (\(H_{0}\)).\textsuperscript{\citeproc{ref-Curran-Everett2009}{261}}
\end{itemize}

\subsection{O que são erros Tipo I e Tipo II?}\label{o-que-suxe3o-erros-tipo-i-e-tipo-ii}

\begin{itemize}
\item
  Erro Tipo I significa a rejeição de uma hipótese nula (\(H_{0}\)) quando esta é verdadeira.\textsuperscript{\citeproc{ref-Curran-Everett2009}{261}}
\item
  Erro Tipo II significa a não rejeição de uma hipótese nula (\(H_{0}\)) quando esta é falsa.\textsuperscript{\citeproc{ref-Curran-Everett2009}{261}}
\end{itemize}

\begin{table}
\centering
\caption{\label{tab:tabela-erros-I-II}Tabela de erros tipos I e II de inferência estatística.}
\centering
\begin{tabu} to \linewidth {>{}l>{\centering}X>{\centering}X}
\toprule
\textbf{ } & \textbf{Hipótese nula $H_{0}$ é falsa} & \textbf{Hipótese nula $H_{0}$ é verdadeira}\\
\midrule
\textbf{Hipótese nula $H_{0}$ foi rejeitada} & Decisão correta & Decisão incorreta (erro tipo I)\\
\textbf{Hipótese nula $H_{0}$ não foi rejeitada} & Decisão incorreta (erro tipo II) & Decisão correta\\
\bottomrule
\end{tabu}
\end{table}

\begin{figure}

{\centering \includegraphics{Ciencia-com-R_files/figure-latex/grafico-erro-tipo-I-II-1} 

}

\caption{Representação gráfica dos erros tipo I e tipo II em um teste de hipótese (bicaudal).}\label{fig:grafico-erro-tipo-I-II}
\end{figure}

\begin{figure}

{\centering \includegraphics{Ciencia-com-R_files/figure-latex/erro-tipo-I-1} 

}

\caption{Erro tipo I: Distribuição dos p-valores em 100 testes de hipótese de amostras aleatórias de tamanho 30. A linha vermelha pontilhada indica o nível de significância estatística de 0,05.}\label{fig:erro-tipo-I}
\end{figure}

\begin{figure}

{\centering \includegraphics{Ciencia-com-R_files/figure-latex/erro-tipo-II-1} 

}

\caption{Erro tipo II: Distribuição dos p-valores em 100 testes de hipótese de amostras aleatórias de tamanho 10. A linha vermelha pontilhada indica o nível de significância estatística de 0,05.}\label{fig:erro-tipo-II}
\end{figure}

\subsection{O que são erros Tipo S e Tipo M?}\label{o-que-suxe3o-erros-tipo-s-e-tipo-m}

\begin{itemize}
\tightlist
\item
  Erro Tipo S (do inglês \emph{sign}) significa a identificação errônea da direção --- positiva ou negativa --- do efeito observado.\textsuperscript{\citeproc{ref-gelman2014}{275},\citeproc{ref-lu2018}{276}}
\end{itemize}

\begin{table}
\centering
\caption{\label{tab:grafico-erro-tipo-S}Tabela de erro tipo S de inferência estatística.}
\centering
\begin{tabu} to \linewidth {>{}l>{\centering}X>{\centering}X}
\toprule
\textbf{ } & \textbf{Sinal positivo} & \textbf{Sinal negativo}\\
\midrule
\textbf{Sinal positivo} & Decisão correta & Decisão incorreta (erro tipo S)\\
\textbf{Sinal negativo} & Decisão incorreta (erro tipo S) & Decisão correta\\
\bottomrule
\end{tabu}
\end{table}

\begin{figure}

{\centering \includegraphics{Ciencia-com-R_files/figure-latex/grafico-erro-tipo-S-bicaudal-1} 

}

\caption{Representação gráfica do erro tipo S (sinal) em um teste de hipótese (bicaudal).}\label{fig:grafico-erro-tipo-S-bicaudal}
\end{figure}

\begin{itemize}
\tightlist
\item
  Erro Tipo M (do inglês \emph{magnitude}) significa a identificação errônea --- em geral, exagerada --- da magnitude do efeito observado.\textsuperscript{\citeproc{ref-gelman2014}{275},\citeproc{ref-lu2018}{276}}
\end{itemize}

\begin{table}
\centering
\caption{\label{tab:tabela-erro-tipo-M}Tabela de erro tipo M de inferência estatística.}
\centering
\begin{tabu} to \linewidth {>{}l>{\centering}X>{\centering}X}
\toprule
\textbf{ } & \textbf{Magnitude alta} & \textbf{Magnitude baixa}\\
\midrule
\textbf{Magnitude alta} & Decisão correta & Decisão incorreta (erro tipo M)\\
\textbf{Magnitude baixa} & Decisão incorreta (erro tipo M) & Decisão correta\\
\bottomrule
\end{tabu}
\end{table}

\begin{figure}

{\centering \includegraphics{Ciencia-com-R_files/figure-latex/grafico-erro-tipo-M-1} 

}

\caption{Representação gráfica do erro tipo M (magnitude) em um teste de hipótese (bicaudal).}\label{fig:grafico-erro-tipo-M}
\end{figure}

\chapter{\texorpdfstring{\textbf{Tamanho do efeito e P-valor}}{Tamanho do efeito e P-valor}}\label{tamanhoefeito-pvalor}

\section{Tamanho do efeito}\label{tamanho-do-efeito}

\subsection{O que é o tamanho do efeito?}\label{o-que-uxe9-o-tamanho-do-efeito}

\begin{itemize}
\tightlist
\item
  Tamanho do efeito quantifica a magnitude de um efeito real da análise, expressando uma importância descritiva dos resultados.\textsuperscript{\citeproc{ref-Kim2015}{277}}
\end{itemize}

\section{Tipos de tamanho do efeito}\label{tipos-de-tamanho-do-efeito}

\begin{itemize}
\item
  Diferenças padronizadas entre grupos:\textsuperscript{\citeproc{ref-Sullivan2012}{267},\citeproc{ref-Kim2015}{277}}

  \begin{itemize}
  \item
    Cohen's \(d\)
  \item
    Glass's \(\Delta\)
  \item
    Razão de chances (\(RC\) ou \(OR\))
  \item
    Risco relativo ou razão de risco (\(RR\))
  \end{itemize}
\end{itemize}

\begin{infobox}{images/Rlogo}
O pacote \emph{epitools}\textsuperscript{\citeproc{ref-epitools}{278}} fornece a função \href{https://www.rdocumentation.org/packages/epitools/versions/0.09/topics/odds.ratio}{\emph{oddsratio.wald}} para calcular a razão de chances.

\end{infobox}

\begin{infobox}{images/Rlogo}
O pacote \emph{epitools}\textsuperscript{\citeproc{ref-epitools}{278}} fornece a função \href{https://www.rdocumentation.org/packages/epitools/versions/0.09/topics/riskratio.wald}{\emph{riskratio.wald}} para calcular a razão de risco.

\end{infobox}

\begin{itemize}
\item
  Medidas de associação:\textsuperscript{\citeproc{ref-Sullivan2012}{267},\citeproc{ref-Kim2015}{277}}

  \begin{itemize}
  \item
    Coeficiente de correlação de Pearson (\(r\)), ponto-bisserial (\(r_{s}\)), Spearman (\(\rho\)), Kendall (\(\tau\)), Cramér (\(V\)) e \(\phi\).
  \item
    Coeficiente de determinação (\(R^2\))
  \end{itemize}
\end{itemize}

\subsection{Como interpretar um tamanho do efeito?}\label{como-interpretar-um-tamanho-do-efeito}

\begin{itemize}
\tightlist
\item
  Tamanhos de efeito podem ser comparadores entre diferentes estudos.\textsuperscript{\citeproc{ref-Sullivan2012}{267}}
\end{itemize}

\begin{infobox}{images/Rlogo}
O pacote \emph{effectsize}\textsuperscript{\citeproc{ref-effectsize}{279}} fornece a função \href{https://www.rdocumentation.org/packages/effectsize/versions/0.8.3/topics/rules}{\emph{rules}} para criar regras de interpretação de tamanhos de efeito.

\end{infobox}

\begin{infobox}{images/Rlogo}
O pacote \emph{effectsize}\textsuperscript{\citeproc{ref-effectsize}{279}} fornece a função \href{https://www.rdocumentation.org/packages/effectsize/versions/0.8.3/topics/interpret}{\emph{interpret}} para interpretar os tamanhos de efeito com base em uma lista de regras pré-definidas.

\end{infobox}

\begin{infobox}{images/Rlogo}
O pacote \emph{pwr}\textsuperscript{\citeproc{ref-pwr}{280}} fornece a função \href{https://www.rdocumentation.org/packages/pwr/versions/1.3-0/topics/cohen.ES}{\emph{cohen.ES}} para obter os tamanhos de efeito ``pequeno'', ``médio'' e ``grande'' para diversos testes de hipóteses.

\end{infobox}

\subsection{O que é a diferença de média bruta?}\label{o-que-uxe9-a-diferenuxe7a-de-muxe9dia-bruta}

\begin{itemize}
\item
  A diferença de média bruta representa a diferença absoluta entre as médias de dois grupos, expressa na unidade original da variável.\textsuperscript{\citeproc{ref-REF}{\textbf{REF?}}}
\item
  Trata-se de uma medida não padronizada, sendo particularmente útil quando a escala possui interpretação clínica ou substantiva direta (por exemplo, mmHg, pontos de escore).\textsuperscript{\citeproc{ref-REF}{\textbf{REF?}}}
\item
  Por depender da unidade de medida, não permite comparações diretas entre estudos com métricas diferentes.\textsuperscript{\citeproc{ref-REF}{\textbf{REF?}}}
\end{itemize}

\subsection{Correlações podem ser consideradas tamanhos de efeito?}\label{correlauxe7uxf5es-podem-ser-consideradas-tamanhos-de-efeito}

\begin{itemize}
\item
  Sim. Coeficientes de correlação podem ser interpretados diretamente como tamanhos de efeito, pois expressam a força e a direção da associação entre variáveis.\textsuperscript{\citeproc{ref-REF}{\textbf{REF?}}}
\item
  O coeficiente de Spearman (\(\rho\)) mede associações monotônicas e é robusto a violações de normalidade.\textsuperscript{\citeproc{ref-REF}{\textbf{REF?}}}
\item
  Kendall (\(\tau\)) é especialmente indicado para amostras pequenas ou dados com empates.\textsuperscript{\citeproc{ref-REF}{\textbf{REF?}}}
\item
  Esses coeficientes são frequentemente utilizados como tamanhos de efeito em testes não paramétricos.\textsuperscript{\citeproc{ref-REF}{\textbf{REF?}}}
\end{itemize}

\subsection{\texorpdfstring{O que é o \(q\) de Cohen?}{O que é o q de Cohen?}}\label{o-que-uxe9-o-q-de-cohen}

\begin{itemize}
\item
  O tamanho de efeito \(q\) quantifica a diferença entre dois coeficientes de correlação, após transformação de Fisher (\(z\)).\textsuperscript{\citeproc{ref-REF}{\textbf{REF?}}}
\item
  É utilizado principalmente para comparar associações observadas em grupos independentes.\textsuperscript{\citeproc{ref-REF}{\textbf{REF?}}}
\item
  Cohen propôs valores de referência para interpretação (pequeno, médio e grande), reforçando seu caráter descritivo.\textsuperscript{\citeproc{ref-REF}{\textbf{REF?}}}
\end{itemize}

\subsection{\texorpdfstring{O que o \(g\) no teste do sinal?}{O que o g no teste do sinal?}}\label{o-que-o-g-no-teste-do-sinal}

\begin{itemize}
\item
  O coeficiente \(g\) é utilizado como tamanho de efeito no teste do sinal.\textsuperscript{\citeproc{ref-REF}{\textbf{REF?}}}
\item
  Representa a diferença padronizada entre a proporção de observações positivas e negativas.\textsuperscript{\citeproc{ref-REF}{\textbf{REF?}}}
\item
  Aplica-se a delineamentos pareados em que apenas a direção do efeito é considerada.\textsuperscript{\citeproc{ref-REF}{\textbf{REF?}}}
\item
  É uma medida robusta, porém menos informativa do que medidas baseadas em magnitude contínua.\textsuperscript{\citeproc{ref-REF}{\textbf{REF?}}}
\end{itemize}

\subsection{\texorpdfstring{O que é o \(h\) de Cohen?}{O que é o h de Cohen?}}\label{o-que-uxe9-o-h-de-cohen}

\begin{itemize}
\item
  O tamanho de efeito \(h\) mede a diferença entre duas proporções, após transformação angular para estabilizar a variância.\textsuperscript{\citeproc{ref-REF}{\textbf{REF?}}}
\item
  É indicado para comparações entre desfechos binários.\textsuperscript{\citeproc{ref-REF}{\textbf{REF?}}}
\item
  Por ser padronizado, permite comparações entre estudos com diferentes proporções absolutas.\textsuperscript{\citeproc{ref-REF}{\textbf{REF?}}}
\end{itemize}

\subsection{\texorpdfstring{O que representa o tamanho de efeito \(w\)?}{O que representa o tamanho de efeito w?}}\label{o-que-representa-o-tamanho-de-efeito-w}

\begin{itemize}
\item
  O coeficiente \(w\) é utilizado como tamanho de efeito em testes do qui-quadrado, tanto de aderência quanto de independência.\textsuperscript{\citeproc{ref-REF}{\textbf{REF?}}}
\item
  Quantifica o grau global de discrepância entre frequências observadas e esperadas.\textsuperscript{\citeproc{ref-REF}{\textbf{REF?}}}
\item
  Sua interpretação depende do número de categorias e do tamanho da amostra, devendo ser feita com cautela.\textsuperscript{\citeproc{ref-REF}{\textbf{REF?}}}
\end{itemize}

\subsection{\texorpdfstring{O que é o tamanho de efeito \(f\) em ANOVA?}{O que é o tamanho de efeito f em ANOVA?}}\label{o-que-uxe9-o-tamanho-de-efeito-f-em-anova}

\begin{itemize}
\item
  O coeficiente \(f\) é utilizado como tamanho de efeito em análises de variância (ANOVA).\textsuperscript{\citeproc{ref-REF}{\textbf{REF?}}}
\item
  Está relacionado à proporção da variância explicada pelo fator em relação à variância residual.\textsuperscript{\citeproc{ref-REF}{\textbf{REF?}}}
\item
  É amplamente empregado em cálculos de poder estatístico e no planejamento amostral.\textsuperscript{\citeproc{ref-REF}{\textbf{REF?}}}
\end{itemize}

\subsection{\texorpdfstring{O que é o tamanho de efeito \(f^2\) em regressão?}{O que é o tamanho de efeito f\^{}2 em regressão?}}\label{o-que-uxe9-o-tamanho-de-efeito-f2-em-regressuxe3o}

\begin{itemize}
\item
  O coeficiente \(f^2\) mede o impacto incremental de um conjunto de preditores em modelos de regressão.\textsuperscript{\citeproc{ref-REF}{\textbf{REF?}}}
\item
  É definido como a razão entre a variância explicada adicional e a variância não explicada.\textsuperscript{\citeproc{ref-REF}{\textbf{REF?}}}
\item
  É particularmente útil para avaliar contribuições locais em modelos hierárquicos ou multivariados.\textsuperscript{\citeproc{ref-REF}{\textbf{REF?}}}
\end{itemize}

\subsection{\texorpdfstring{O que é a estatística \(\Lambda\) de Wilks na MANOVA?}{O que é a estatística \textbackslash Lambda de Wilks na MANOVA?}}\label{o-que-uxe9-a-estatuxedstica-lambda-de-wilks-na-manova}

\begin{itemize}
\item
  O Lambda de Wilks (\(\Lambda\)) é utilizado como estatística global em análises multivariadas de variância (MANOVA).\textsuperscript{\citeproc{ref-REF}{\textbf{REF?}}}
\item
  Representa a proporção da variância multivariada não explicada pelo modelo.\textsuperscript{\citeproc{ref-REF}{\textbf{REF?}}}
\item
  Embora menos intuitivo como medida de magnitude, é amplamente utilizado e pode ser convertido em outras estatísticas.\textsuperscript{\citeproc{ref-REF}{\textbf{REF?}}}
\end{itemize}

\subsection{Como escolher o tamanho de efeito adequado?}\label{como-escolher-o-tamanho-de-efeito-adequado}

\begin{itemize}
\item
  Não existe um tamanho de efeito universalmente superior; a escolha depende da pergunta científica, do delineamento, do tipo de variável e do modelo estatístico.\textsuperscript{\citeproc{ref-REF}{\textbf{REF?}}}
\item
  A boa prática estatística recomenda reportar estimativa pontual, intervalo de confiança e tamanho de efeito, evitando decisões baseadas exclusivamente em significância estatística.\textsuperscript{\citeproc{ref-REF}{\textbf{REF?}}}
\item
  Sempre que possível, a interpretação deve considerar relevância prática, contexto científico e incerteza associada.\textsuperscript{\citeproc{ref-REF}{\textbf{REF?}}}
\end{itemize}

\section{Conversão entre tamanhos do efeito}\label{conversuxe3o-entre-tamanhos-do-efeito}

\subsection{Como converter um tamanho de efeito em outro?}\label{como-converter-um-tamanho-de-efeito-em-outro}

\begin{itemize}
\tightlist
\item
  .\textsuperscript{\citeproc{ref-Kim2015}{277}}
\end{itemize}

\begin{infobox}{images/Rlogo}
O pacote \emph{effectsize}\textsuperscript{\citeproc{ref-effectsize}{279}} fornece diversas funções para conversão de diferentes estimativas de tamanhos de efeito.

\end{infobox}

\section{Efeitos bruto e padronizado}\label{efeitos-bruto-e-padronizado}

\subsection{O que é efeito bruto?}\label{o-que-uxe9-efeito-bruto}

\begin{itemize}
\item
  .\textsuperscript{\citeproc{ref-greenland1986}{281}}
\item
  .\textsuperscript{\citeproc{ref-greenland1991}{282}}
\end{itemize}

\subsection{O que é efeito padronizado?}\label{o-que-uxe9-efeito-padronizado}

\begin{itemize}
\item
  .\textsuperscript{\citeproc{ref-greenland1986}{281}}
\item
  .\textsuperscript{\citeproc{ref-greenland1991}{282}}
\end{itemize}

\section{P-valor}\label{p-valor}

\subsection{O que é significância estatística?}\label{o-que-uxe9-significuxe2ncia-estatuxedstica}

\begin{itemize}
\tightlist
\item
  A expressão ``significância estatística''\textsuperscript{\citeproc{ref-latter1902}{283}} ou ``evidência estatística de significância'' sugere apenas que um experimento merece ser repetido, uma vez que um baixo P-valor (calculado a partir dos dados, modelos e demais suposições do estudo) sugere ser improvável que os dados coletados sejam coletados no contexto de que a hipótese nula (\(H_{0}\)) assumida é verdadeira.\textsuperscript{\citeproc{ref-aylmerfisher1926}{284}}
\end{itemize}

\subsection{Como justificar o nível de significância estatística de um teste?}\label{como-justificar-o-nuxedvel-de-significuxe2ncia-estatuxedstica-de-um-teste}

\begin{itemize}
\tightlist
\item
  .\textsuperscript{\citeproc{ref-REF}{\textbf{REF?}}}
\end{itemize}

\begin{infobox}{images/Rlogo}
O pacote \emph{Superpower}\textsuperscript{\citeproc{ref-Superpower}{285}} fornece a função \href{https://www.rdocumentation.org/packages/Superpower/versions/0.2.0/topics/optimal_alpha}{\emph{optimal\_alpha}} para calcular e justificar o nível de significância \(\alpha\) por balanço dos erros tipo I e II.

\end{infobox}

\begin{infobox}{images/Rlogo}
O pacote \emph{Superpower}\textsuperscript{\citeproc{ref-Superpower}{285}} fornece a função \href{https://www.rdocumentation.org/packages/Superpower/versions/0.2.0/topics/ANOVA_compromise}{\emph{ANOVA\_compromise}} para calcular e justificar o nível de significância \(\alpha\) por balanço dos erros tipo I e II em análise de variância (ANOVA).

\end{infobox}

\subsection{O que é o P-valor?}\label{o-que-uxe9-o-p-valor}

\begin{itemize}
\item
  P-valor é a probabilidade, assumindo-se um dado modelo estatístico, de que um efeito calculado a partir dos dados seria igual ou mais extremo do que o seu valor observado.\textsuperscript{\citeproc{ref-wasserstein2016}{286}}
\item
  P-valor é uma variável aleatória que possui distribuição uniforme quando a hipótese nula (\(H_{0}\)) é verdadeira.\textsuperscript{\citeproc{ref-altman2017}{287}}
\end{itemize}

\subsection{Como interpretar o P-valor?}\label{como-interpretar-o-p-valor}

\begin{itemize}
\item
  P-valores abaixo de um nível de significância estatística pré-especificado representam que um experimento merece ser repetido, com a rejeição da hipótese nula (\(H_{0}\)) justificada apenas quando experimentos adicionais frequentemente reportem igualmente resultados positivos (rejeição da hipótese nula (\(H_{0}\)).\textsuperscript{\citeproc{ref-goodman2016}{269}}
\item
  P-valor resulta da coleta e análise de dados, e assim quantifica a plausibilidade dos dados observados sob a hipótese nula (\(H_{0}\)).\textsuperscript{\citeproc{ref-heinze2016}{288}}
\item
  P-valores podem indicar quantitativamente a incompatibilidade entre os dados obtidos e o modelo estatístico especificado a priori (geralmente constituído pela hipótese nula (\(H_{0}\)).\textsuperscript{\citeproc{ref-wasserstein2016}{286}}
\item
  P-valores menores/maiores do que o nível de significância estatístico pré-estabelecido não devem ser utilizados como única fonte de informação para tomada de decisão em ciência.\textsuperscript{\citeproc{ref-wasserstein2016}{286}}
\end{itemize}

\subsection{O que o P-valor não é?}\label{o-que-o-p-valor-nuxe3o-uxe9}

\begin{itemize}
\item
  P-valor não representa a probabilidade de que a hipótese nula (\(H_{0}\)) seja verdadeira, nem a probabilidade de que os dados tenham sido produzidos pelo acaso.\textsuperscript{\citeproc{ref-wasserstein2016}{286}}
\item
  P-valor não mede o tamanho do efeito ou a relevância da sua observação.\textsuperscript{\citeproc{ref-wasserstein2016}{286}}
\item
  P-valor sozinho não provê informação suficiente sobre a evidência sobre um modelo teórico. A sua interpretação correta requer uma descrição ampla sobre o delineamento, métodos e análises estatísticas aplicados no estudo.\textsuperscript{\citeproc{ref-wasserstein2016}{286}}
\item
  Evidência estatística de significância não provê informação sobre a magnitude do efeito observado e não necessariamente implica que o efeito é robusto.\textsuperscript{\citeproc{ref-Landis2012}{193},\citeproc{ref-altman2017}{287}}
\end{itemize}

\subsection{Qual a origem do `P\textless0,05'?}\label{qual-a-origem-do-p005}

\begin{itemize}
\item
  A origem do P\textless0,05 remonta aos trabalhos de R. A. Fisher nas décadas de 1920 e 1930. Fisher introduziu o conceito de P-valor dentro de uma abordagem frequentista de inferência estatística.\textsuperscript{\citeproc{ref-goodman2016}{269}}
\item
  O P\textless0,05 foi sugerido por Ronald A. Fisher como um limiar prático para indicar que um resultado era ``estatisticamente significativo''.\textsuperscript{\citeproc{ref-goodman2016}{269}}
\item
  Para Ronald A. Fisher, a significância estatística não era prova definitiva, mas um sinal de que o resultado merecia investigação adicional. A rejeição da hipótese nula só deveria ocorrer após repetidas observações significativas, e não com base em um único teste.\textsuperscript{\citeproc{ref-goodman2016}{269}}
\end{itemize}

\begin{figure}

{\centering \includegraphics{Ciencia-com-R_files/figure-latex/p-valores-1} 

}

\caption{Visualização espacial de p < 0,05 (5 quadrados aleatórios em 100).}\label{fig:p-valores}
\end{figure}

\subsection{Quais são os complementos ou alternativas ao P-valor?}\label{quais-suxe3o-os-complementos-ou-alternativas-ao-p-valor}

\begin{itemize}
\item
  Intervalos de confiança, credibilidade ou predição.\textsuperscript{\citeproc{ref-wasserstein2016}{286}}
\item
  Razão de verossimilhança.\textsuperscript{\citeproc{ref-wasserstein2016}{286}}
\item
  Métodos Bayesianos, fator Bayes.\textsuperscript{\citeproc{ref-wasserstein2016}{286}}
\end{itemize}

\section{P-valor de 2ª geração}\label{p-valor-de-2uxaa-gerauxe7uxe3o}

\subsection{O que é o P-valor de 2ª geração?}\label{o-que-uxe9-o-p-valor-de-2uxaa-gerauxe7uxe3o}

\begin{itemize}
\item
  O P-valor de 2ª geração (SGPV) resume a fração das hipóteses apoiadas pelos dados que também pertencem à hipótese nula intervalar (intervalo de equivalência previamente especificado). Quantifica quanto do intervalo de estimativa (p.ex., IC95\%) recai dentro da zona de indiferença científica/clinicamente irrelevante.\textsuperscript{\citeproc{ref-blume2018}{289}}
\item
  Essa abordagem exige declarar a hipótese nula como intervalo (e não um ponto), incorporando o que é considerado ``efeito sem relevância prática'' segundo o contexto científico (precisão de medida, relevância clínica etc.).\textsuperscript{\citeproc{ref-blume2018}{289}}
\end{itemize}

\subsection{\texorpdfstring{Como definir a hipótese nula intervalar e \(\delta\)?}{Como definir a hipótese nula intervalar e \textbackslash delta?}}\label{como-definir-a-hipuxf3tese-nula-intervalar-e-delta}

\begin{itemize}
\item
  Especifique \(H_0\) como um intervalo de equivalência \([H_0^{-}, H_0^{+}]\) que contém efeitos considerados praticamente nulos. Defina \(\delta\) como a meia-largura do intervalo de equivalência (\(\delta = (H_0^{+} - H_0^{-})/2\)).\textsuperscript{\citeproc{ref-blume2018}{289}}
\item
  A escolha deve ser a priori e justificada por critérios científicos (p.ex., MCID, precisão de medida).\textsuperscript{\citeproc{ref-blume2018}{289}}
\end{itemize}

\subsection{Como calcular o SGPV?}\label{como-calcular-o-sgpv}

\begin{itemize}
\tightlist
\item
  Seja \(I=[a,b]\) o intervalo apoiado pelos dados (p.ex., IC 95\%) e \(H_0\) o intervalo nulo. O SGPV é \eqref{eq:sgpv}, onde \(|I|\) é a largura do intervalo de estimativa, \(|H_0|\) é a largura do intervalo nulo e \(|I \cap H_0|\) é a largura da sobreposição entre os dois intervalos. O SGPV é restrito ao intervalo \([0,1]\).\textsuperscript{\citeproc{ref-blume2018}{289}}
\end{itemize}

\begin{equation}
\label{eq:sgpv}
p_{\delta} = \frac{|\,I \cap H_0\,|}{|\,I\,|} \times \max\!\left\{ \frac{|\,I\,|}{2|\,H_0\,|}, \, 1 \right\}
\end{equation}

\begin{itemize}
\item
  Quando \(|I|<2|H_0|\), \(p_{\delta}\) é apenas a fração de sobreposição \(|I\cap H_0|/|I|\).\textsuperscript{\citeproc{ref-blume2018}{289}}
\item
  Quando \(|I|>2|H_0|\), o SGPV reduz-se a \(\tfrac{1}{2}\times \dfrac{|,I\cap H_0,|}{|,H_0,|}\le \tfrac{1}{2}\), sinalizando inconclusão por imprecisão.\textsuperscript{\citeproc{ref-blume2018}{289}}
\end{itemize}

\begin{table}
\centering
\caption{\label{tab:sgpv}Comparação entre p-valor (bicaudal, inferido do IC95\%) e SGPV ($p_{\delta}$) nos cenários simulados.}
\centering
\begin{tabu} to \linewidth {>{}c>{\centering}X>{\centering}X>{\centering}X>{\centering}X>{\centering}X>{\centering}X>{\centering}X>{\centering}X>{\centering}X}
\toprule
\textbf{Cenário} & \textbf{$a$} & \textbf{$b$} & \textbf{$H_0^{-}$} & \textbf{$H_0^{+}$} & \textbf{$\hat\theta$} & \textbf{$SE$} & \textbf{p-valor (bicaudal)} & \textbf{$p_{\delta}$} & \textbf{Conclusão (SGPV)}\\
\midrule
\textbf{1} & 0.350 & 0.550 & -0.100 & 0.100 & 0.450 & 0.0510 & <0,001 & 0.000 & Apoia alternativas (SGPV=0)\\
\textbf{2} & -0.050 & 0.080 & -0.100 & 0.100 & 0.015 & 0.0332 & 0.651 & 1.000 & Equivalência (SGPV=1)\\
\textbf{3} & -0.500 & 0.700 & -0.100 & 0.100 & 0.100 & 0.3061 & 0.744 & 0.500 & Inconclusivo (0<pδ<1)\\
\textbf{4} & 0.050 & 0.250 & -0.100 & 0.100 & 0.150 & 0.0510 & 0.003 & 0.250 & Inconclusivo (0<pδ<1)\\
\textbf{5} & -0.250 & -0.050 & -0.100 & 0.100 & -0.150 & 0.0510 & 0.003 & 0.250 & Inconclusivo (0<pδ<1)\\
\textbf{6} & 0.150 & 0.550 & -0.100 & 0.100 & 0.350 & 0.1020 & 0.001 & 0.000 & Apoia alternativas (SGPV=0)\\
\textbf{7} & -0.550 & -0.150 & -0.100 & 0.100 & -0.350 & 0.1020 & 0.001 & 0.000 & Apoia alternativas (SGPV=0)\\
\bottomrule
\end{tabu}
\end{table}

\subsection{Como interpretar o SGPV?}\label{como-interpretar-o-sgpv}

\begin{itemize}
\item
  \(p_{\delta}=0\): dados apoiam apenas hipóteses alternativas relevantes (IC totalmente fora da equivalência).\textsuperscript{\citeproc{ref-blume2018}{289}}
\item
  \(p_{\delta}=1\): dados apoiam apenas hipóteses nulas (equivalentes) (IC totalmente dentro da equivalência).\textsuperscript{\citeproc{ref-blume2018}{289}} \(0<p_{\delta}<1\): inconclusivo; o valor expressa o grau de inconclusão. Em particular, \(p_{\delta}=\tfrac{1}{2}\) indica inconclusão estrita.\textsuperscript{\citeproc{ref-blume2018}{289}}
\item
  O SGPV é descritivo (não é probabilidade posterior de \(H_0\)).\textsuperscript{\citeproc{ref-blume2018}{289}}
\end{itemize}

\subsection{Relação com testes de equivalência (TOST)}\label{relauxe7uxe3o-com-testes-de-equivaluxeancia-tost}

\begin{itemize}
\item
  Tanto SGPV quanto TOST comparam o IC com os limites de equivalência. Se o IC \((1-2\alpha)\) (p.ex., 90\% quando \(\alpha=0{,}05\)) cai inteiro dentro dos limites, TOST conclui equivalência --- situação análoga a \(p_{\delta}=1\).\textsuperscript{\citeproc{ref-lakens2020}{290}}
\item
  Com ICs simétricos, há pontos de ancoragem em que as estatísticas coincidem: quando \(p_{\text{TOST}}=0{,}5\), então \(\mathrm{SGPV}=0{,}5\); quando o IC toca o limite mas fica inteiramente dentro (fronteira), \(p_{\text{TOST}}=0{,}025\) e \(\mathrm{SGPV}=1\); quando o IC fica inteiramente fora tocando o limite, \(p_{\text{TOST}}=0{,}975\) e \(\mathrm{SGPV}=0\).\textsuperscript{\citeproc{ref-lakens2020}{290}}
\item
  Em ICs assimétricos ou quando \(|I|>2|H_0|\), o SGPV fica difícil de interpretar quando \(0<p_{\delta}<1\); nesses cenários, o TOST costuma diferenciar melhor os resultados.\textsuperscript{\citeproc{ref-lakens2020}{290}}
\end{itemize}

\subsection{Propriedades frequenciais e múltiplas comparações}\label{propriedades-frequenciais-e-muxfaltiplas-comparauxe7uxf5es}

\begin{itemize}
\item
  Usando ICs \(100(1-\alpha)%
  \), sob qualquer hipótese em \(H_0\), \(\Pr(p_{\delta}=0)\le \alpha\) e \(\to 0\) com o aumento de \(n\); fora de \(H_0\), \(\Pr(p_{\delta}=0)\to 1\) quando \(n\) cresce.\textsuperscript{\citeproc{ref-blume2018}{289}}
\item
  O SGPV mitiga naturalmente inflação de erro Tipo I em muitas comparações e prioriza relevância científica (não requer ajustes ad hoc).\textsuperscript{\citeproc{ref-blume2018}{289}}
\end{itemize}

\section{Distribuição de confiança}\label{distribuiuxe7uxe3o-de-confianuxe7a}

\subsection{O que é distribuição de confiança?}\label{o-que-uxe9-distribuiuxe7uxe3o-de-confianuxe7a}

\begin{itemize}
\tightlist
\item
  Distribuição de confiança é uma representação contínua da evidência inferencial sobre um parâmetro de interesse. Ela mostra, para cada valor possível do tamanho do efeito, o nível de confiança associado, sendo uma generalização visual do intervalo de confiança e do P-valor.\textsuperscript{\citeproc{ref-REF}{\textbf{REF?}}}
\end{itemize}

\begin{figure}

{\centering \includegraphics{Ciencia-com-R_files/figure-latex/confidence-distribution-1} 

}

\caption{Distribuição de confiança para o tamanho do efeito estimado.}\label{fig:confidence-distribution}
\end{figure}

\section{Boas práticas}\label{boas-pruxe1ticas}

\begin{itemize}
\item
  Defina \(H_0\) intervalar e \(\delta\) a priori com justificativa científica.\textsuperscript{\citeproc{ref-blume2018}{289},\citeproc{ref-lakens2020}{290}}
\item
  Reporte: estimativa pontual, IC, limites de equivalência e \(p_{\delta}\); interprete \(p_{\delta}\in{0,1}\) de forma dicotômica e \(0<p_{\delta}<1\) como inconclusivo; quando necessário, complemente com TOST.\textsuperscript{\citeproc{ref-blume2018}{289},\citeproc{ref-lakens2020}{290}}
\end{itemize}

\chapter{\texorpdfstring{\textbf{Testes estatísticos}}{Testes estatísticos}}\label{testes-estatisticos}

\section{Variáveis categóricas}\label{variuxe1veis-categuxf3ricas}

\subsection{\texorpdfstring{Testes de Qui-quadrado (\(\chi^2\))}{Testes de Qui-quadrado (\textbackslash chi\^{}2)}}\label{testes-de-qui-quadrado-chi2}

\begin{table}[t]
\caption{\label{tab:qui-quadrado}Teste Qui-quadrado (com correção de Yates)} 
\fontsize{12.0pt}{14.0pt}\selectfont
\begin{tabular*}{\linewidth}{@{\extracolsep{\fill}}lcccc}
\toprule
 & \multicolumn{2}{c}{\textbf{Resposta}} &  &  \\ 
\cmidrule(lr){2-3}
 & Respondeu & Não respondeu & \textbf{Total} & \textbf{P-valor}\textsuperscript{\textit{1}} \\ 
\midrule\addlinespace[2.5pt]
{\bfseries Tratamento} &  &  &  & <0.001 \\ 
    A & 60 (30\%) & 37 (19\%) & 97 (49\%) &  \\ 
    B & 35 (18\%) & 68 (34\%) & 103 (52\%) &  \\ 
{\bfseries Total} & 95 (48\%) & 105 (53\%) & 200 (100\%) &  \\ 
\bottomrule
\end{tabular*}
\begin{minipage}{\linewidth}
\vspace{.05em}
\textsuperscript{\textit{1}} Teste qui-quadrado de independência\\
Tamanho do efeito (Cramér's V): 0.269\\
\end{minipage}
\end{table}

\begin{table}[t]
\caption{\label{tab:qui-quadrado-yates}Teste Qui-quadrado (sem correção de Yates)} 
\fontsize{12.0pt}{14.0pt}\selectfont
\begin{tabular*}{\linewidth}{@{\extracolsep{\fill}}lcccc}
\toprule
 & \multicolumn{2}{c}{\textbf{Resposta}} &  &  \\ 
\cmidrule(lr){2-3}
 & Respondeu & Não respondeu & \textbf{Total} & \textbf{P-valor}\textsuperscript{\textit{1}} \\ 
\midrule\addlinespace[2.5pt]
{\bfseries Tratamento} &  &  &  & <0.001 \\ 
    A & 60 (30\%) & 37 (19\%) & 97 (49\%) &  \\ 
    B & 35 (18\%) & 68 (34\%) & 103 (52\%) &  \\ 
{\bfseries Total} & 95 (48\%) & 105 (53\%) & 200 (100\%) &  \\ 
\bottomrule
\end{tabular*}
\begin{minipage}{\linewidth}
\vspace{.05em}
\textsuperscript{\textit{1}} Teste qui-quadrado de independência\\
Tamanho do efeito (Cramér's V): 0.269\\
\end{minipage}
\end{table}

\subsection{Teste exato de Fisher}\label{teste-exato-de-fisher}

\begin{table}[t]
\caption{\label{tab:fisher}Teste exato de Fisher} 
\fontsize{12.0pt}{14.0pt}\selectfont
\begin{tabular*}{\linewidth}{@{\extracolsep{\fill}}lcccc}
\toprule
 & \multicolumn{2}{c}{\textbf{Resposta}} &  &  \\ 
\cmidrule(lr){2-3}
 & Respondeu & Não respondeu & \textbf{Total} & \textbf{P-valor}\textsuperscript{\textit{1}} \\ 
\midrule\addlinespace[2.5pt]
{\bfseries Tratamento} &  &  &  & <0.001 \\ 
    A & 60 (30\%) & 37 (19\%) & 97 (49\%) &  \\ 
    B & 35 (18\%) & 68 (34\%) & 103 (52\%) &  \\ 
{\bfseries Total} & 95 (48\%) & 105 (53\%) & 200 (100\%) &  \\ 
\bottomrule
\end{tabular*}
\begin{minipage}{\linewidth}
\vspace{.05em}
\textsuperscript{\textit{1}} Teste exato de Fisher\\
Tamanho do efeito (Cramér's V): 0.269\\
\end{minipage}
\end{table}

\subsection{Teste de McNemar}\label{teste-de-mcnemar}

\begin{table}[t]
\caption{\label{tab:mcnemar}Teste de McNemar} 
\fontsize{12.0pt}{14.0pt}\selectfont
\begin{tabular*}{\linewidth}{@{\extracolsep{\fill}}lccc}
\toprule
 & \multicolumn{2}{c}{\textbf{Pos}} &  \\ 
\cmidrule(lr){2-3}
 & Sim & Não & \textbf{Total} \\ 
\midrule\addlinespace[2.5pt]
{\bfseries Pre} &  &  &  \\ 
    Sim & 41 (41\%) & 25 (25\%) & 66 (66\%) \\ 
    Não & 11 (11\%) & 23 (23\%) & 34 (34\%) \\ 
{\bfseries Total} & 52 (52\%) & 48 (48\%) & 100 (100\%) \\ 
\bottomrule
\end{tabular*}
\begin{minipage}{\linewidth}
\vspace{.05em}
McNemar χ² = 4.694; gl = 1; p = 0.030\\
\end{minipage}
\end{table}

\subsection{\texorpdfstring{Teste \emph{Q} de Cochran}{Teste Q de Cochran}}\label{teste-q-de-cochran}

\begin{table}[t]
\caption{\label{tab:cochran-q}Teste Q de Cochran} 
\fontsize{12.0pt}{14.0pt}\selectfont
\begin{tabular*}{\linewidth}{@{\extracolsep{\fill}}lcc}
\toprule
 & \multicolumn{2}{c}{Resposta} \\ 
\cmidrule(lr){2-3}
 & Positivo & Negativo \\ 
\midrule\addlinespace[2.5pt]
Tempo &  &  \\ 
    T1 & 38 (16\%) & 42 (18\%) \\ 
    T2 & 52 (22\%) & 28 (12\%) \\ 
    T3 & 56 (23\%) & 24 (10\%) \\ 
Total & 146 (61\%) & 94 (39\%) \\ 
\bottomrule
\end{tabular*}
\begin{minipage}{\linewidth}
\vspace{.05em}
Cochran Q χ²(2) = 8.645, p = 0.013\\
\end{minipage}
\end{table}

\subsection{Teste de Cochran--Armitage}\label{teste-de-cochranarmitage}

\begin{table}[t]
\caption{\label{tab:cochran-armitage}Teste de Cochran--Armitage} 
\fontsize{12.0pt}{14.0pt}\selectfont
\begin{tabular*}{\linewidth}{@{\extracolsep{\fill}}lcc}
\toprule
 & \multicolumn{2}{c}{\textbf{Status}} \\ 
\cmidrule(lr){2-3}
 & Sucesso & Fracasso \\ 
\midrule\addlinespace[2.5pt]
{\bfseries Grupo} &  &  \\ 
    Alta & 1 (50\%) & 1 (50\%) \\ 
    Baixa & 1 (50\%) & 1 (50\%) \\ 
    Média & 1 (50\%) & 1 (50\%) \\ 
{\bfseries Total} & 3 (50\%) & 3 (50\%) \\ 
\bottomrule
\end{tabular*}
\begin{minipage}{\linewidth}
\vspace{.05em}
χ²(1) = 12.047; p = <0.001\\
\end{minipage}
\end{table}

\subsection{\texorpdfstring{Odds ratio (\(OR\)) e risco relativo (\(RR\))}{Odds ratio (OR) e risco relativo (RR)}}\label{odds-ratio-or-e-risco-relativo-rr}

\begin{table}[t]
\caption{\label{tab:or-rr}Medidas de associação} 
\fontsize{12.0pt}{14.0pt}\selectfont
\begin{tabular*}{\linewidth}{@{\extracolsep{\fill}}lcccc}
\toprule
 & \multicolumn{2}{c}{\textbf{Resposta}} &  &  \\ 
\cmidrule(lr){2-3}
 & Não & Sim & \textbf{Total} & \textbf{Valor-p}\textsuperscript{\textit{1}} \\ 
\midrule\addlinespace[2.5pt]
{\bfseries Tratamento} &  &  &  & >0.9 \\ 
    A & 1 (50\%) & 1 (50\%) & 2 (100\%) &  \\ 
    B & 1 (50\%) & 1 (50\%) & 2 (100\%) &  \\ 
{\bfseries Total} & 2 (50\%) & 2 (50\%) & 4 (100\%) &  \\ 
\bottomrule
\end{tabular*}
\begin{minipage}{\linewidth}
\vspace{.05em}
\textsuperscript{\textit{1}} Teste qui-quadrado de independência\\
OR = 2.1 [1.17, 3.79]; RR = 1.61 [1.1, 2.35]. \\
\end{minipage}
\end{table}

\section{Variáveis contínuas}\label{variuxe1veis-contuxednuas}

\subsection{\texorpdfstring{Teste \emph{t} de Student}{Teste t de Student}}\label{teste-t-de-student}

\begin{table}[t]
\caption{\label{tab:t-student}Teste t de Student} 
\fontsize{12.0pt}{14.0pt}\selectfont
\begin{tabular*}{\linewidth}{@{\extracolsep{\fill}}lccc}
\toprule
\textbf{Características} & \textbf{A}  N = 45\textsuperscript{\textit{1}} & \textbf{B}  N = 55\textsuperscript{\textit{1}} & \textbf{P-valor}\textsuperscript{\textit{2}} \\ 
\midrule\addlinespace[2.5pt]
{\bfseries Desfecho} & 51.30 (8.88) & 55.38 (10.05) & 0.034 \\ 
\bottomrule
\end{tabular*}
\begin{minipage}{\linewidth}
\vspace{.05em}
\textsuperscript{\textit{1}} Média (Desvio Padrão)\\
\textsuperscript{\textit{2}} Teste t com correção de Welch\\
Tamanho do efeito (d de Cohen): -0.432\\
\end{minipage}
\end{table}

\subsection{\texorpdfstring{Teste \emph{t} de Welch}{Teste t de Welch}}\label{teste-t-de-welch}

\begin{table}[t]
\caption{\label{tab:welch}Teste t de Welch} 
\fontsize{12.0pt}{14.0pt}\selectfont
\begin{tabular*}{\linewidth}{@{\extracolsep{\fill}}lccc}
\toprule
\textbf{Características} & \textbf{A}  N = 45\textsuperscript{\textit{1}} & \textbf{B}  N = 55\textsuperscript{\textit{1}} & \textbf{P-valor}\textsuperscript{\textit{2}} \\ 
\midrule\addlinespace[2.5pt]
{\bfseries Desfecho} & 51.30 (8.88) & 55.38 (10.05) & 0.034 \\ 
\bottomrule
\end{tabular*}
\begin{minipage}{\linewidth}
\vspace{.05em}
\textsuperscript{\textit{1}} Média (Desvio Padrão)\\
\textsuperscript{\textit{2}} Teste t com correção de Welch\\
Tamanho do efeito (d de Cohen): -0.432\\
\end{minipage}
\end{table}

\subsection{Teste de Mann-Whitney}\label{teste-de-mann-whitney}

\begin{table}[t]
\caption{\label{tab:mann-whitney}Teste de Mann--Whitney (Wilcoxon rank-sum)} 
\fontsize{12.0pt}{14.0pt}\selectfont
\begin{tabular*}{\linewidth}{@{\extracolsep{\fill}}lccc}
\toprule
\textbf{Características} & \textbf{A}  N = 51\textsuperscript{\textit{1}} & \textbf{B}  N = 69\textsuperscript{\textit{1}} & \textbf{P-valor}\textsuperscript{\textit{2}} \\ 
\midrule\addlinespace[2.5pt]
{\bfseries Desfecho} & 50.07 [46.34, 56.48] & 54.47 [48.51, 61.73] & 0.023 \\ 
\bottomrule
\end{tabular*}
\begin{minipage}{\linewidth}
\vspace{.05em}
\textsuperscript{\textit{1}} Mediana {[}Q1, Q3{]}\\
\textsuperscript{\textit{2}} Teste de soma de postos de Wilcoxon\\
Tamanho do efeito (r): 0.208\\
\end{minipage}
\end{table}

\subsection{Teste de Wilcoxon}\label{teste-de-wilcoxon}

\begin{table}[t]
\caption{\label{tab:wilcoxon}Teste de Wilcoxon (signed-rank)} 
\fontsize{12.0pt}{14.0pt}\selectfont
\begin{tabular*}{\linewidth}{@{\extracolsep{\fill}}lccc}
\toprule
\textbf{Características} & \textbf{Pré}  N = 60\textsuperscript{\textit{1}} & \textbf{Pós}  N = 60\textsuperscript{\textit{1}} & \textbf{P-valor}\textsuperscript{\textit{2}} \\ 
\midrule\addlinespace[2.5pt]
{\bfseries Desfecho} & 44.44 [39.93, 50.25] & 53.00 [48.35, 59.65] & <0.001 \\ 
\bottomrule
\end{tabular*}
\begin{minipage}{\linewidth}
\vspace{.05em}
\textsuperscript{\textit{1}} Mediana {[}Q1, Q3{]}\\
\textsuperscript{\textit{2}} Teste de soma de postos de Wilcoxon\\
Tamanho do efeito (r): 0.616\\
\end{minipage}
\end{table}

\subsection{Análise de variância}\label{anuxe1lise-de-variuxe2ncia}

\begin{table}[t]
\caption{\label{tab:anova}Análise de variância de um fator} 
\fontsize{12.0pt}{14.0pt}\selectfont
\begin{tabular*}{\linewidth}{@{\extracolsep{\fill}}lcccc}
\toprule
\textbf{Características} & \textbf{A}  N = 25\textsuperscript{\textit{1}} & \textbf{B}  N = 34\textsuperscript{\textit{1}} & \textbf{C}  N = 31\textsuperscript{\textit{1}} & \textbf{P-valor}\textsuperscript{\textit{2}} \\ 
\midrule\addlinespace[2.5pt]
{\bfseries Desfecho} & 50.58 (7.80) & 55.64 (9.52) & 61.65 (9.55) & <0.001 \\ 
\bottomrule
\end{tabular*}
\begin{minipage}{\linewidth}
\vspace{.05em}
\textsuperscript{\textit{1}} Média (Desvio Padrão)\\
\textsuperscript{\textit{2}} One-way analysis of means\\
Tamanho do efeito (eta²): 0.193\\
\end{minipage}
\end{table}
\begin{table}[t]
\caption*{
{\fontsize{20}{25}\selectfont  Post hoc de Tukey\fontsize{12}{15}\selectfont }
} 
\fontsize{12.0pt}{14.0pt}\selectfont
\begin{tabular*}{\linewidth}{@{\extracolsep{\fill}}lrrrr}
\toprule
Comparação & Diferença de médias & IC95\% inferior & IC95\% superior & p (ajustado) \\ 
\midrule\addlinespace[2.5pt]
B-A & 5.06 & -0.65 & 10.77 & 0.093 \\ 
C-A & 11.07 & 5.25 & 16.89 & <0.001 \\ 
C-B & 6.01 & 0.63 & 11.39 & 0.025 \\ 
\bottomrule
\end{tabular*}
\end{table}

\subsection{Análise de variância (Welch)}\label{anuxe1lise-de-variuxe2ncia-welch}

\begin{table}[t]
\caption{\label{tab:welch-anova}Análise de variância de Welch} 
\fontsize{12.0pt}{14.0pt}\selectfont
\begin{tabular*}{\linewidth}{@{\extracolsep{\fill}}lcccc}
\toprule
\textbf{Características} & \textbf{A}  N = 43\textsuperscript{\textit{1}} & \textbf{B}  N = 57\textsuperscript{\textit{1}} & \textbf{C}  N = 50\textsuperscript{\textit{1}} & \textbf{P-valor (Welch)} \\ 
\midrule\addlinespace[2.5pt]
{\bfseries Desfecho} & 53.42 (8.43) & 55.90 (10.17) & 58.10 (14.91) & 0.137 \\ 
\bottomrule
\end{tabular*}
\begin{minipage}{\linewidth}
\vspace{.05em}
\textsuperscript{\textit{1}} Média (Desvio Padrão)\\
Tamanho do efeito (eta², via SS): 0.025\\
\end{minipage}
\end{table}
\begin{table}[t]
\caption*{
{\fontsize{20}{25}\selectfont  Post hoc de Games-Howell\fontsize{12}{15}\selectfont }
} 
\fontsize{12.0pt}{14.0pt}\selectfont
\begin{tabular*}{\linewidth}{@{\extracolsep{\fill}}llrrrr}
\toprule
Grupo 1 & Grupo 2 & Diferença de médias & IC95\% inferior & IC95\% superior & p (ajustado) \\ 
\midrule\addlinespace[2.5pt]
A & B & 2.48 & -1.95 & 6.91 & 0.381 \\ 
A & C & 4.68 & -1.21 & 10.58 & 0.146 \\ 
B & C & 2.20 & -3.76 & 8.17 & 0.654 \\ 
\bottomrule
\end{tabular*}
\end{table}

\subsection{Teste de Kruskal-Wallis}\label{teste-de-kruskal-wallis}

\begin{table}[t]
\caption{\label{tab:kruskal-wallis}Teste de Kruskal--Wallis} 
\fontsize{12.0pt}{14.0pt}\selectfont
\begin{tabular*}{\linewidth}{@{\extracolsep{\fill}}lcccc}
\toprule
\textbf{Características} & \textbf{A}  N = 54\textsuperscript{\textit{1}} & \textbf{B}  N = 67\textsuperscript{\textit{1}} & \textbf{C}  N = 59\textsuperscript{\textit{1}} & \textbf{P-valor}\textsuperscript{\textit{2}} \\ 
\midrule\addlinespace[2.5pt]
{\bfseries Desfecho} & 40.98 [31.12, 51.85] & 40.48 [31.75, 62.18] & 64.96 [45.37, 77.91] & <0.001 \\ 
\bottomrule
\end{tabular*}
\begin{minipage}{\linewidth}
\vspace{.05em}
\textsuperscript{\textit{1}} Mediana {[}Q1, Q3{]}\\
\textsuperscript{\textit{2}} Teste de Kruskal-Wallis\\
Tamanho do efeito: epsilon² = 0.14\\
\end{minipage}
\end{table}
\begin{table}[t]
\caption*{
{\fontsize{20}{25}\selectfont  Post hoc de Dunn (Bonferroni)\fontsize{12}{15}\selectfont }
} 
\fontsize{12.0pt}{14.0pt}\selectfont
\begin{tabular*}{\linewidth}{@{\extracolsep{\fill}}llrll}
\toprule
Grupo 1 & Grupo 2 & Z & p & p (ajustado) \\ 
\midrule\addlinespace[2.5pt]
A & B & 0.742 & 0.458 & 1.000 \\ 
A & C & 4.714 & <0.001 & <0.001 \\ 
B & C & 4.213 & <0.001 & <0.001 \\ 
\bottomrule
\end{tabular*}
\end{table}

\chapter{\texorpdfstring{\textbf{Descrição}}{Descrição}}\label{descricao}

\section{Análise de descrição}\label{anuxe1lise-de-descriuxe7uxe3o}

\subsection{O que é análise de descrição de dados?}\label{o-que-uxe9-anuxe1lise-de-descriuxe7uxe3o-de-dados}

\begin{itemize}
\item
  A análise descritiva utiliza métodos para calcular, descrever e resumir os dados coletados da(s) amostra(s) de modo que sejam interpretadas adequadamente.\textsuperscript{\citeproc{ref-vetter2017}{108}}
\item
  As análises descritivas geralmente compreendem a apresentação quantitativa (numérica) em tabelas e/ou gráficos.\textsuperscript{\citeproc{ref-vetter2017}{108}}
\end{itemize}

\begin{infobox}{images/Rlogo}
O pacote \emph{explore}\textsuperscript{\citeproc{ref-explore}{196}} fornece a função \href{https://www.rdocumentation.org/packages/explore/versions/1.0.2/topics/explore}{\emph{explore}} para análise exploratória de um banco de dados.

\end{infobox}

\begin{infobox}{images/Rlogo}
O pacote \emph{dataMaid}\textsuperscript{\citeproc{ref-dataMaid}{197}} fornece a função \href{https://www.rdocumentation.org/packages/dataMaid/versions/1.4.1/topics/makeDataReport}{\emph{makeDataReport}} para criar um relatório de análise exploratória de um banco de dados.

\end{infobox}

\begin{infobox}{images/Rlogo}
O pacote \emph{DataExplorer}\textsuperscript{\citeproc{ref-DataExplorer}{198}} fornece a função \href{https://www.rdocumentation.org/packages/DataExplorer/versions/0.8.2/topics/create_report}{\emph{create\_report}} para criar um relatório de análise exploratória de um banco de dados.

\end{infobox}

\begin{infobox}{images/Rlogo}
O pacote \emph{SmartEDA}\textsuperscript{\citeproc{ref-SmartEDA}{199}} fornece a função \href{https://www.rdocumentation.org/packages/SmartEDA/versions/0.3.9/topics/ExpReport}{\emph{ExpReport}} para criar um relatório de análise exploratória de um banco de dados.

\end{infobox}

\begin{infobox}{images/Rlogo}
O pacote \emph{esquisse}\textsuperscript{\citeproc{ref-esquisse}{291}} fornece a função \href{https://www.rdocumentation.org/packages/esquisse/versions/1.1.2/topics/esquisser}{\emph{esquisser}} para executar uma interface interativa para visualização de dados.

\end{infobox}

\section{Estimação}\label{estimauxe7uxe3o}

\subsection{O que é estimativa?}\label{o-que-uxe9-estimativa}

\begin{itemize}
\tightlist
\item
  Estimativa é o valor de uma variável de interesse calculado a partir de uma amostra.\textsuperscript{\citeproc{ref-REF}{\textbf{REF?}}}
\end{itemize}

\subsection{O que é estimativa pontual?}\label{o-que-uxe9-estimativa-pontual}

\begin{itemize}
\tightlist
\item
  Estimativa pontual é o valor único de uma variável de interesse calculado a partir de uma amostra.\textsuperscript{\citeproc{ref-REF}{\textbf{REF?}}}
\end{itemize}

\subsection{O que é estimativa intervalar?}\label{o-que-uxe9-estimativa-intervalar}

\begin{itemize}
\tightlist
\item
  Estimativa intervalar é um intervalo de valores de uma variável de interesse calculado a partir de uma amostra.\textsuperscript{\citeproc{ref-REF}{\textbf{REF?}}}
\end{itemize}

\subsection{O que é estimativa de parâmetro?}\label{o-que-uxe9-estimativa-de-paruxe2metro}

\begin{itemize}
\tightlist
\item
  Estimativa de parâmetro é o valor de uma variável de interesse calculado a partir de uma amostra que representa o valor da população.\textsuperscript{\citeproc{ref-REF}{\textbf{REF?}}}
\end{itemize}

\chapter{\texorpdfstring{\textbf{Comparação}}{Comparação}}\label{comparacao}

\section{Análise inferencial de comparação}\label{anuxe1lise-inferencial-de-comparauxe7uxe3o}

\subsection{O que é análise de comparação de dados?}\label{o-que-uxe9-anuxe1lise-de-comparauxe7uxe3o-de-dados}

\begin{itemize}
\tightlist
\item
  .\textsuperscript{\citeproc{ref-REF}{\textbf{REF?}}}
\end{itemize}

\begin{infobox}{images/Rlogo}
O pacote \emph{cocor}\textsuperscript{\citeproc{ref-cocor}{292}} fornece as funções \href{https://www.rdocumentation.org/packages/stats/versions/3.6.2/topics/cor.test}{cocor.indep.groups}, \href{https://www.rdocumentation.org/packages/stats/versions/3.6.2/topics/cor.test}{cocor.dep.groups.overlap} e \href{https://www.rdocumentation.org/packages/stats/versions/3.6.2/topics/cor.test}{cocor.dep.groups.nonoverlap} para comparar 2 coeficientes de correlação entre grupos independentes, grupos sobrepostos ou independentes, respectivamente.\textsuperscript{\citeproc{ref-cocor}{292}}

\end{infobox}

\section{F-teste}\label{f-teste}

\subsection{O que é o F-teste?}\label{o-que-uxe9-o-f-teste}

\begin{itemize}
\item
  O F-teste é uma estatística que compara a variabilidade entre grupos com a variabilidade dentro dos grupos.\textsuperscript{\citeproc{ref-REF}{\textbf{REF?}}}
\item
  A estatística é calculada como \eqref{eq:f-teste}, onde \(\text{QM}\) são ``quadrados médios'', com \(gl_{1}\) e \(gl_{2}\) definidos pelo desenho (ex.: fatores e resíduos).\textsuperscript{\citeproc{ref-REF}{\textbf{REF?}}}
\end{itemize}

\begin{equation}
\label{eq:f-teste}
F=\dfrac{\text{QM}_{\text{entre}}}{\text{QM}_{\text{dentro}}}
\end{equation}

\subsection{Quando usar o F-teste?}\label{quando-usar-o-f-teste}

\begin{itemize}
\item
  Análise de variância de um fator (≥3 grupos) e ANOVA multifatorial (efeitos principais e interações).\textsuperscript{\citeproc{ref-REF}{\textbf{REF?}}}
\item
  Modelo linear generalizado / regressão linear: teste global \(H_{0}:\beta_{1}=\cdots=\beta_{p}=0\).\textsuperscript{\citeproc{ref-REF}{\textbf{REF?}}}
\item
  Análise de covariância (comparação de grupos ajustando covariáveis).\textsuperscript{\citeproc{ref-REF}{\textbf{REF?}}}
\item
  Contrastes planejados ou pós-hoc (usando a razão F correspondente).\textsuperscript{\citeproc{ref-REF}{\textbf{REF?}}}
\end{itemize}

\subsection{Quais são os pressupostos?}\label{quais-suxe3o-os-pressupostos}

\begin{itemize}
\item
  Observações independentes.\textsuperscript{\citeproc{ref-REF}{\textbf{REF?}}}
\item
  Normalidade (aproximada) dos resíduos.\textsuperscript{\citeproc{ref-REF}{\textbf{REF?}}}
\item
  Homogeneidade de variâncias entre grupos (homoscedasticidade).\textsuperscript{\citeproc{ref-REF}{\textbf{REF?}}}
\item
  Se houver violações importantes: considerar ANOVA de Welch, transformações apropriadas ou alternativas não paramétricas (ex.: Kruskal--Wallis para um fator).\textsuperscript{\citeproc{ref-REF}{\textbf{REF?}}}
\end{itemize}

\subsection{Como interpretar o resultado?}\label{como-interpretar-o-resultado}

\begin{itemize}
\item
  Valor de \(F\) elevado com \(P<\alpha\) indica evidência contra \(H_{0}\) (diferenças entre grupos/modelo com ajuste significativo).\textsuperscript{\citeproc{ref-REF}{\textbf{REF?}}}
\item
  Relate sempre \(gl_{1}\), \(gl_{2}\), \(F\) e \(P\), além de um tamanho de efeito (ex.: \(\eta^{2}\), \(\eta^{2}_{p}\) ou \(\omega^{2}\)) e intervalo de confianca quando possível.\textsuperscript{\citeproc{ref-REF}{\textbf{REF?}}}
\item
  Após rejeitar \(H_{0}\), use contrastes ou pós-hoc com ajuste para múltiplas comparações para localizar as diferenças.\textsuperscript{\citeproc{ref-REF}{\textbf{REF?}}}
\end{itemize}

\subsection{O que reportar em publicações?}\label{o-que-reportar-em-publicauxe7uxf5es}

\begin{itemize}
\item
  Estrutura do desenho (fatores, níveis, balanceamento).\textsuperscript{\citeproc{ref-REF}{\textbf{REF?}}}
\item
  Verificação/diagnóstico dos pressupostos.\textsuperscript{\citeproc{ref-REF}{\textbf{REF?}}}
\item
  Estatística \(F\) com \(gl\) e \(P\).\textsuperscript{\citeproc{ref-REF}{\textbf{REF?}}}
\item
  Tamanho de efeito e intervalo de confiança.\textsuperscript{\citeproc{ref-REF}{\textbf{REF?}}}
\item
  Método de ajuste para múltiplas comparações quando aplicável.\textsuperscript{\citeproc{ref-REF}{\textbf{REF?}}}
\end{itemize}

\chapter{\texorpdfstring{\textbf{Associação}}{Associação}}\label{associacao}

\section{Análise inferencial de associação}\label{anuxe1lise-inferencial-de-associauxe7uxe3o}

\subsection{O que é análise de associação?}\label{o-que-uxe9-anuxe1lise-de-associauxe7uxe3o}

\begin{itemize}
\tightlist
\item
  .\textsuperscript{\citeproc{ref-REF}{\textbf{REF?}}}
\end{itemize}

\section{Associação bivariada}\label{associauxe7uxe3o-bivariada}

\subsection{O que são análises de associação bivariada?}\label{o-que-suxe3o-anuxe1lises-de-associauxe7uxe3o-bivariada}

\begin{itemize}
\tightlist
\item
  .\textsuperscript{\citeproc{ref-REF}{\textbf{REF?}}}
\end{itemize}

\subsection{Quais testes podem ser usados para análises de associação bivariada?}\label{quais-testes-podem-ser-usados-para-anuxe1lises-de-associauxe7uxe3o-bivariada}

\begin{itemize}
\item
  Teste Qui-quadrado (\(\chi^2\)).\textsuperscript{\citeproc{ref-McHugh2013}{293},\citeproc{ref-Kim2017a}{294}}

  \begin{itemize}
  \item
    O teste qui-quadrado (\(\chi^2\)) avalia uma hipótese global se a relação entre duas variáveis e/ou fatores é independente ou associada.\textsuperscript{\citeproc{ref-Kim2017a}{294}}
  \item
    O teste qui-quadrado é utilizado para comparar a distribuição de uma variável categórica em uma amostra ou grupo com a distribuição em outro. Se a distribuição da variável categórica não for muito diferente nos diferentes grupos, pode-se concluir que a distribuição da variável categórica não está relacionada com a variável dos grupos. Pode-se também concluir que a variável categórica e os grupos são independentes.\textsuperscript{\citeproc{ref-Kim2017a}{294}}
  \item
    Tipo: não paramétrico.\textsuperscript{\citeproc{ref-McHugh2013}{293},\citeproc{ref-Kim2017a}{294}}
  \item
    Suposições:\textsuperscript{\citeproc{ref-McHugh2013}{293},\citeproc{ref-Kim2017a}{294}}

    \begin{itemize}
    \item
      As variáveis são ordinais ou categóricas nominais, de modo que as células representem frequência.
    \item
      Os níveis dos fatores (variáveis categóricas) são mutuamente exclusivos.
    \item
      Tamanho de amostra grande e adequado porque é baseado em uma abordagem de aproximação.
    \item
      Menos de 20\% das células com frequências esperadas \textless{} 5
    \item
      Nenhuma célula com frequência esperada \textless{} 1.
    \end{itemize}
  \item
    Hipóteses:\textsuperscript{\citeproc{ref-Kim2017a}{294}}

    \begin{itemize}
    \item
      Nula (\(H_{0}\)): independente (sem associação)
    \item
      Alternativa (\(H_{1}\)): não independente (associação)
    \end{itemize}
  \item
    Tamanho do efeito:\textsuperscript{\citeproc{ref-Kim2017a}{294}}

    \begin{itemize}
    \item
      Phi (\(\phi\)), para tabelas de contingência 2x2
    \item
      Razão de chances (\(RC\) ou \(OR\)), para tabelas de contingência 2x2
    \item
      Cramer V (\(V\)), para tabelas de contingência NxM
    \end{itemize}
  \end{itemize}
\end{itemize}

\begin{infobox}{images/Rlogo}
O pacote \emph{gtsummary}\textsuperscript{\citeproc{ref-gtsummary}{211}} fornece a função \href{https://www.rdocumentation.org/packages/gtsummary/versions/1.6.3/topics/tbl_cross}{\emph{tbl\_cross}} para criar uma tabela NxM.

\end{infobox}

\begin{itemize}
\item
  Teste Exato de Fisher.\textsuperscript{\citeproc{ref-McHugh2013}{293},\citeproc{ref-Kim2017a}{294}}

  \begin{itemize}
  \item
    O teste exato de Fisher avalia a hipótese nula de independência aplicando a distribuição hipergeométrica dos números nas células da tabela.\textsuperscript{\citeproc{ref-Kim2017a}{294}}
  \item
    Hipóteses:\textsuperscript{\citeproc{ref-McHugh2013}{293},\citeproc{ref-Kim2017a}{294}}

    \begin{itemize}
    \item
      Nula (\(H_{0}\)): independente (sem associação)
    \item
      Alternativa (\(H_{1}\)): não independente (associação)
    \end{itemize}
  \item
    Tamanho do efeito:\textsuperscript{\citeproc{ref-McHugh2013}{293},\citeproc{ref-Kim2017a}{294}}

    \begin{itemize}
    \item
      Phi (\(\phi\)), para tabelas de contingência 2x2
    \item
      Razão de chances (\(RC\) ou \(OR\)), para tabelas de contingência 2x2
    \item
      Cramer V (\(V\)), para tabelas de contingência NxM
    \end{itemize}
  \end{itemize}
\end{itemize}

\begin{infobox}{images/Rlogo}
O pacote \emph{gtsummary}\textsuperscript{\citeproc{ref-gtsummary}{211}} fornece a função \href{https://www.rdocumentation.org/packages/gtsummary/versions/1.6.3/topics/tbl_cross}{\emph{tbl\_cross}} para criar uma tabela NxM.

\end{infobox}

\section{Associação multivariada}\label{associauxe7uxe3o-multivariada}

\subsection{O que são análises de associação multivariada?}\label{o-que-suxe3o-anuxe1lises-de-associauxe7uxe3o-multivariada}

\begin{itemize}
\tightlist
\item
  .\textsuperscript{\citeproc{ref-REF}{\textbf{REF?}}}
\end{itemize}

\subsection{Quais testes podem ser usados para análises de associação multivariada?}\label{quais-testes-podem-ser-usados-para-anuxe1lises-de-associauxe7uxe3o-multivariada}

\begin{itemize}
\tightlist
\item
  .\textsuperscript{\citeproc{ref-REF}{\textbf{REF?}}}
\end{itemize}

\chapter{\texorpdfstring{\textbf{Correlação}}{Correlação}}\label{correlacao}

\section{Análise inferencial de correlação}\label{anuxe1lise-inferencial-de-correlauxe7uxe3o}

\subsection{O que é covariância?}\label{o-que-uxe9-covariuxe2ncia}

\begin{itemize}
\tightlist
\item
  .\textsuperscript{\citeproc{ref-REF}{\textbf{REF?}}}
\end{itemize}

\subsection{O que é correlação?}\label{o-que-uxe9-correlauxe7uxe3o}

\begin{itemize}
\tightlist
\item
  .\textsuperscript{\citeproc{ref-REF}{\textbf{REF?}}}
\end{itemize}

\subsection{Qual é a interpretação das medidas de correlação?}\label{qual-uxe9-a-interpretauxe7uxe3o-das-medidas-de-correlauxe7uxe3o}

\begin{itemize}
\item
  Os valores de correlação estão no intervalo \([-1; 1]\).\textsuperscript{\citeproc{ref-barkan2015}{112},\citeproc{ref-khamis2008}{295},\citeproc{ref-allison2022}{296}}
\item
  Valores de correlação positivos representam uma relação direta entre as variáveis, tal que valores maiores de uma variável estão associados a valores maiores de outra variável.\textsuperscript{\citeproc{ref-khamis2008}{295},\citeproc{ref-allison2022}{296}}
\item
  Valores de correlação negativos representam uma relação indireta (ou inversa) entre as variáveis, tal que valores maiores (menores) de uma variável estão associados a valores maiores (menores) de outra variável.\textsuperscript{\citeproc{ref-khamis2008}{295},\citeproc{ref-allison2022}{296}}
\item
  Valores de correlação próximos de \(0\) representam a inexistência de relação entre as variáveis.\textsuperscript{\citeproc{ref-khamis2008}{295},\citeproc{ref-allison2022}{296}}
\end{itemize}

\begin{figure}

{\centering \includegraphics{Ciencia-com-R_files/figure-latex/correlacao-1} 

}

\caption{Exemplo de diferentes forças e direção de correlação entre duas variáveis X e Y.}\label{fig:correlacao}
\end{figure}

\subsection{Quais precauções devem ser tomadas na interpretação de medidas de correlação?}\label{quais-precauuxe7uxf5es-devem-ser-tomadas-na-interpretauxe7uxe3o-de-medidas-de-correlauxe7uxe3o}

\begin{itemize}
\item
  Tamanhos de efeito grande (ou qualquer outro) não representam necessariamente uma relação causa-efeito entre as variáveis.\textsuperscript{\citeproc{ref-khamis2008}{295}}
\item
  Tamanhos de efeito grande (ou qualquer outro) não representam necessariamente uma relação de concordância ou confiabilidade entre as variáveis.\textsuperscript{\citeproc{ref-khamis2008}{295}}
\item
  Uma escala de medição com representação agregada do constructo na coleta de dados pode subestimar o tamanho do efeito da correlação \(r\) em de cerca de 13\% e do coeficiente de determinação \(R^2\) de cerca de 30\%.\textsuperscript{\citeproc{ref-aguinis2008}{125}} Neste caso, a correlação desatenuada \(r_{x'y'}\) pode ser calculada por \eqref{eq:r-corrected}, utilizando a correlação observada \(r_{xy}\) e os fatores de correção \(r_{xx'}\) e \(r_{yy'}\) para o número de intervalos nas variáveis X e Y, respectivamennte:\textsuperscript{\citeproc{ref-aguinis2008}{125}}
\end{itemize}

\begin{equation}
\label{eq:r-corrected}
r_{x'y'} = \dfrac{r_{xy}}{r_{xx'}r_{yy'}}
\end{equation}

\begin{infobox}{images/Rlogo}
O pacote \emph{psychmeta}\textsuperscript{\citeproc{ref-psychmeta}{297}} fornece a função \href{https://www.rdocumentation.org/packages/psychmeta/versions/2.7.0/topics/correct_r_coarseness}{\emph{correct\_r\_coarseness}} para calcular o coeficiente de correlação desatenuado (\(r_{x'y'}\)).

\end{infobox}

\begin{infobox}{images/Rlogo}
O pacote \emph{psychmeta}\textsuperscript{\citeproc{ref-psychmeta}{297}} fornece a função \href{https://www.rdocumentation.org/packages/psychmeta/versions/2.7.0/topics/correct_r}{\emph{correct\_r}} para calcular o coeficiente de correlação em escala restrita e/ou com erro de mensuração (\(r_{x'y'}\)).

\end{infobox}

\begin{itemize}
\item
  Os coeficientes de correlação possuem suposições que, se violadas, podem levar a interpretações equivocadas. Nestes cenários, visualizar os dados e as relações entre as variáveis pode contribuir com a interpretação e utilidade dos coeficientes de correlação.\textsuperscript{\citeproc{ref-anscombe1973}{298}}
\item
  O quarteto de Anscombe é um conjunto de quatro bancos de dados bivariados que possuem a mesma média, variância, correlação e regressão linear (até a 2a casa decimal), mas que são visualmente diferentes e, assim, demonstram a importância da análise gráfica da correlação.\textsuperscript{\citeproc{ref-anscombe1973}{298}}
\end{itemize}

\begin{table}
\centering
\caption{\label{tab:anscombe-data}Quarteto de Anscombe.}
\centering
\begin{tabu} to \linewidth {>{}c>{\centering}X>{\centering}X>{\centering}X>{\centering}X>{\centering}X>{\centering}X>{\centering}X>{\centering}X}
\toprule
\textbf{ID} & \textbf{x1} & \textbf{x2} & \textbf{x3} & \textbf{x4} & \textbf{y1} & \textbf{y2} & \textbf{y3} & \textbf{y4}\\
\midrule
\textbf{1} & 10 & 10 & 10 & 8 & 8.04 & 9.14 & 7.46 & 6.58\\
\textbf{2} & 8 & 8 & 8 & 8 & 6.95 & 8.14 & 6.77 & 5.76\\
\textbf{3} & 13 & 13 & 13 & 8 & 7.58 & 8.74 & 12.74 & 7.71\\
\textbf{4} & 9 & 9 & 9 & 8 & 8.81 & 8.77 & 7.11 & 8.84\\
\textbf{5} & 11 & 11 & 11 & 8 & 8.33 & 9.26 & 7.81 & 8.47\\
\textbf{6} & 14 & 14 & 14 & 8 & 9.96 & 8.10 & 8.84 & 7.04\\
\textbf{7} & 6 & 6 & 6 & 8 & 7.24 & 6.13 & 6.08 & 5.25\\
\textbf{8} & 4 & 4 & 4 & 19 & 4.26 & 3.10 & 5.39 & 12.50\\
\textbf{9} & 12 & 12 & 12 & 8 & 10.84 & 9.13 & 8.15 & 5.56\\
\textbf{10} & 7 & 7 & 7 & 8 & 4.82 & 7.26 & 6.42 & 7.91\\
\textbf{11} & 5 & 5 & 5 & 8 & 5.68 & 4.74 & 5.73 & 6.89\\
\bottomrule
\end{tabu}
\end{table}

\begin{table}
\centering
\caption{\label{tab:anscombe-summary}Análise descritiva do Quarteto de Anscombe demostrando os conjuntos de dados bivariados com parâmetros quase idênticos.}
\centering
\begin{tabu} to \linewidth {>{}l>{\centering}X>{\centering}X>{\centering}X>{\centering}X}
\toprule
\textbf{ } & \textbf{X1Y1} & \textbf{X2Y2} & \textbf{X3Y3} & \textbf{X4Y4}\\
\midrule
\textbf{Observações} & 11.00 & 11.00 & 11.00 & 11.00\\
\textbf{Média x} & 9.00 & 9.00 & 9.00 & 9.00\\
\textbf{Média y} & 7.50 & 7.50 & 7.50 & 7.50\\
\textbf{Variância x} & 11.00 & 11.00 & 11.00 & 11.00\\
\textbf{Variância y} & 4.13 & 4.13 & 4.12 & 4.12\\
\textbf{Correlação} & 0.82 & 0.82 & 0.82 & 0.82\\
\textbf{Coeficiente angular} & 0.50 & 0.50 & 0.50 & 0.50\\
\textbf{Coeficiente linear} & 3.00 & 3.00 & 3.00 & 3.00\\
\textbf{Coeficiente de determinação} & 0.67 & 0.67 & 0.67 & 0.67\\
\bottomrule
\end{tabu}
\end{table}

\begin{figure}

{\centering \includegraphics{Ciencia-com-R_files/figure-latex/anscombe-plot-1} 

}

\caption{Gráfico de dispersão do Quarteto de Anscombe para representação gráfica de conjuntos de dados bivariados com parâmetros quase idênticos e relações muito distintas.}\label{fig:anscombe-plot}
\end{figure}

\begin{infobox}{images/Rlogo}
O pacote \emph{anscombiser}\textsuperscript{\citeproc{ref-anscombiser}{299}} fornece a função \href{https://www.rdocumentation.org/packages/anscombiser/versions/1.1.0/topics/anscombise}{\emph{anscombise}} para gerar bancos de dados que compartilham os mesmos valores de parâmetros do Quarteto de Anscombe.

\end{infobox}

\section{Coeficientes de correlação}\label{coeficientes-de-correlauxe7uxe3o}

\subsection{Quais coeficientes podem ser usados em análises de correlação?}\label{quais-coeficientes-podem-ser-usados-em-anuxe1lises-de-correlauxe7uxe3o}

\begin{itemize}
\tightlist
\item
  Coeficiente de correlação de Pearson (\(r\)) \eqref{eq:pearson}.\textsuperscript{\citeproc{ref-khamis2008}{295},\citeproc{ref-allison2022}{296}}
\end{itemize}

\begin{equation}
\label{eq:pearson}
r = \dfrac{n \sum{x_i y_i} - \sum{x_i} \sum{y_i}}{\sqrt{\left[n \sum{x_i^2} - (\sum{x_i})^2\right]\left[n \sum{y_i^2} - (\sum{y_i})^2\right]}}
\end{equation}

\begin{itemize}
\item
  O coeficiente de correlação de Pearson (\(r\)) avalia a força e direção da relação linear entre duas variáveis quantitativas.\textsuperscript{\citeproc{ref-khamis2008}{295},\citeproc{ref-allison2022}{296}}
\item
  Tipo: paramétrico.\textsuperscript{\citeproc{ref-khamis2008}{295},\citeproc{ref-allison2022}{296}}
\item
  Hipóteses:\textsuperscript{\citeproc{ref-allison2022}{296}}
\item
  Nula (\(H_{0}\)): \(r=0\)
\item
  Alternativa (\(H_{1}\)): \(r≠0\)
\item
  Tamanho do efeito:\textsuperscript{\citeproc{ref-khamis2008}{295},\citeproc{ref-allison2022}{296}}
\item
  Coeficiente de correlação de Pearson (\(r\))
\end{itemize}

\begin{infobox}{images/Rlogo}
O pacote \emph{stats}\textsuperscript{\citeproc{ref-stats}{131}} fornece a função \href{https://www.rdocumentation.org/packages/stats/versions/3.6.2/topics/cor.test}{\emph{cor.test}} para calcular o coeficiente de correlação de Pearson (\(r\)).

\end{infobox}

\begin{infobox}{images/Rlogo}
O pacote \emph{correlation}\textsuperscript{\citeproc{ref-correlation}{300}} do projeto \emph{easystats}\textsuperscript{\citeproc{ref-easystats}{301}} fornece a função \href{https://cloud.r-project.org/web/packages/correlation/index.html}{\emph{correlation}} para calcular o coeficiente de correlação de Pearson (\(r\)).

\end{infobox}

\begin{itemize}
\tightlist
\item
  Coeficiente de correlação ponto-bisserial (\(r_{s}\)) \eqref{eq:biserial}.\textsuperscript{\citeproc{ref-khamis2008}{295}}
\end{itemize}

\begin{equation}
\label{eq:biserial}
r_{s} = \dfrac{M_{1} - M_{0}}{s_{y}} \sqrt{\dfrac{n_{1}n_{0}}{n^2}}
\end{equation}

\begin{itemize}
\item
  O coeficiente de correlação ponto-bisserial (\(r_{s}\)) avalia a força e direção da relação linear entre uma variável quantitativa e outra dicotômica.\textsuperscript{\citeproc{ref-khamis2008}{295}}
\item
  Tipo: paramétrico.\textsuperscript{\citeproc{ref-khamis2008}{295}}
\item
  Hipóteses:\textsuperscript{\citeproc{ref-khamis2008}{295}}
\item
  Nula (\(H_{0}\)): \(r_{s}=0\)
\item
  Alternativa (\(H_{1}\)): \(r_{s}≠0\)
\item
  Tamanho do efeito:\textsuperscript{\citeproc{ref-khamis2008}{295}}
\item
  Coeficiente de correlação ponto-bisserial (\(r_{s}\))
\end{itemize}

\begin{infobox}{images/Rlogo}
O pacote \emph{stats}\textsuperscript{\citeproc{ref-stats}{131}} fornece a função \href{https://www.rdocumentation.org/packages/stats/versions/3.6.2/topics/cor.test}{\emph{cor.test}} para calcular o coeficiente de correlação ponto-bisserial (\(r_{s}\)).

\end{infobox}

\begin{infobox}{images/Rlogo}
O pacote \emph{correlation}\textsuperscript{\citeproc{ref-correlation}{300}} do projeto \emph{easystats}\textsuperscript{\citeproc{ref-easystats}{301}} fornece a função \href{https://cloud.r-project.org/web/packages/correlation/index.html}{\emph{correlation}} para calcular o coeficiente de correlação ponto-bisserial (\(r_{s}\)).

\end{infobox}

\begin{itemize}
\tightlist
\item
  Coeficiente de correlação de Spearman (\(\rho\)) \eqref{eq:spearman}.\textsuperscript{\citeproc{ref-khamis2008}{295},\citeproc{ref-allison2022}{296}}
\end{itemize}

\begin{equation}
\label{eq:spearman}
\rho = 1 - \dfrac{6 \Sigma d_{i}^2}{n(n^2 - 1)}
\end{equation}

\begin{itemize}
\item
  O coeficiente de correlação de Spearman (\(\rho\)) avalia a força e direção da relação monotônica entre duas variáveis quantitativas.\textsuperscript{\citeproc{ref-khamis2008}{295},\citeproc{ref-allison2022}{296}}
\item
  O coeficiente de correlação de Spearman (\(\rho\)) pode ser também definida como a correlação de Pearson (\(r\)) entre as classificações (\emph{ranks}) das duas variáveis quantitativas.\textsuperscript{\citeproc{ref-khamis2008}{295},\citeproc{ref-allison2022}{296}}
\item
  Tipo: não-paramétrico.\textsuperscript{\citeproc{ref-khamis2008}{295},\citeproc{ref-allison2022}{296}}
\item
  Hipóteses:\textsuperscript{\citeproc{ref-khamis2008}{295},\citeproc{ref-allison2022}{296}}
\item
  Nula (\(H_{0}\)): \(\rho=0\)
\item
  Alternativa (\(H_{1}\)): \(\rho≠0\)
\item
  Tamanho do efeito:\textsuperscript{\citeproc{ref-khamis2008}{295},\citeproc{ref-allison2022}{296}}
\item
  Coeficiente de correlação de Spearman (\(\rho\))
\end{itemize}

\begin{infobox}{images/Rlogo}
O pacote \emph{stats}\textsuperscript{\citeproc{ref-stats}{131}} fornece a função \href{https://www.rdocumentation.org/packages/stats/versions/3.6.2/topics/cor.test}{\emph{cor.test}} para calcular o coeficiente de correlação de Spearman (\(\rho\)).

\end{infobox}

\begin{infobox}{images/Rlogo}
O pacote \emph{correlation}\textsuperscript{\citeproc{ref-correlation}{300}} do projeto \emph{easystats}\textsuperscript{\citeproc{ref-easystats}{301}} fornece a função \href{https://cloud.r-project.org/web/packages/correlation/index.html}{\emph{correlation}} para calcular o coeficiente de correlação de Spearman (\(\rho\)).

\end{infobox}

\begin{itemize}
\tightlist
\item
  Coeficiente de Kendall (\(\tau\)) \eqref{eq:kendall}.\textsuperscript{\citeproc{ref-khamis2008}{295},\citeproc{ref-allison2022}{296}}
\end{itemize}

\begin{equation}
\label{eq:kendall}
\tau = \dfrac{(n_{c} - n_{d})}{\dfrac{1}{2}n(n-1)}
\end{equation}

\begin{itemize}
\item
  O coeficiente Kendall \(\tau\) avalia a força e direção da relação monotônica entre duas variáveis quantitativas ou qualitativas.\textsuperscript{\citeproc{ref-khamis2008}{295},\citeproc{ref-allison2022}{296}}
\item
  O coeficiente Kendall \(\tau\) é definido como a proporção de todos os pares concordantes menos a proporção de todos os pares discordantes.\textsuperscript{\citeproc{ref-khamis2008}{295},\citeproc{ref-allison2022}{296}}
\item
  Tipo: não-paramétrico.\textsuperscript{\citeproc{ref-khamis2008}{295},\citeproc{ref-allison2022}{296}}
\item
  Hipóteses:\textsuperscript{\citeproc{ref-khamis2008}{295},\citeproc{ref-allison2022}{296}}
\item
  Nula (\(H_{0}\)): \(\tau=0\)
\item
  Alternativa (\(H_{1}\)): \(\tau≠0\)
\item
  Tamanho do efeito:\textsuperscript{\citeproc{ref-khamis2008}{295},\citeproc{ref-allison2022}{296}}
\item
  Kendall \(\tau\)
\end{itemize}

\begin{infobox}{images/Rlogo}
O pacote \emph{stats}\textsuperscript{\citeproc{ref-stats}{131}} fornece a função \href{https://www.rdocumentation.org/packages/stats/versions/3.6.2/topics/cor.test}{\emph{cor.test}} para calcular o coeficiente Kendall \(\tau\).

\end{infobox}

\begin{infobox}{images/Rlogo}
O pacote \emph{correlation}\textsuperscript{\citeproc{ref-correlation}{300}} do projeto \emph{easystats}\textsuperscript{\citeproc{ref-easystats}{301}} fornece a função \href{https://cloud.r-project.org/web/packages/correlation/index.html}{\emph{correlation}} para calcular o coeficiente coeficiente Kendall \(\tau\).

\end{infobox}

\begin{itemize}
\tightlist
\item
  Coeficiente de Cramér (\(V\)) \eqref{eq:cramer}.\textsuperscript{\citeproc{ref-REF}{\textbf{REF?}}}
\end{itemize}

\begin{equation}
\label{eq:cramer}
V = \sqrt{\dfrac{\chi^2/n}{\min(k-1, r-1)}}
\end{equation}

\begin{itemize}
\item
  O coeficiente Cramér (\(V\)) avalia a força e direção da relação entre duas variáveis qualitativas.\textsuperscript{\citeproc{ref-REF}{\textbf{REF?}}}
\item
  Tipo: não-paramétrico.\textsuperscript{\citeproc{ref-REF}{\textbf{REF?}}}
\item
  Hipóteses:\textsuperscript{\citeproc{ref-REF}{\textbf{REF?}}}
\item
  Nula (\(H_{0}\)): \(V=0\)
\item
  Alternativa (\(H_{1}\)): \(V≠0\)
\item
  Tamanho do efeito:\textsuperscript{\citeproc{ref-REF}{\textbf{REF?}}}
\item
  Coeficiente Cramer (\(V\))
\end{itemize}

\begin{itemize}
\tightlist
\item
  Coeficiente de Sheperd (\(\phi\)) \eqref{eq:sheperd}.\textsuperscript{\citeproc{ref-REF}{\textbf{REF?}}}
\end{itemize}

\begin{equation}
\label{eq:sheperd}
\phi = \sqrt{\dfrac{\chi^2}{n}}
\end{equation}

\begin{itemize}
\item
  O coeficiente Phi (\(\phi\)) avalia a força e direção da relação entre duas variáveis dicotômicas.\textsuperscript{\citeproc{ref-REF}{\textbf{REF?}}}
\item
  Tipo: não-paramétrico.\textsuperscript{\citeproc{ref-REF}{\textbf{REF?}}}
\item
  Hipóteses:\textsuperscript{\citeproc{ref-REF}{\textbf{REF?}}}
\item
  Nula (\(H_{0}\)): \(\phi=0\)
\item
  Alternativa (\(H_{1}\)): \(\phi≠0\)
\item
  Tamanho do efeito:\textsuperscript{\citeproc{ref-REF}{\textbf{REF?}}}
\item
  Coeficiente Phi (\(\phi\))
\end{itemize}

\begin{infobox}{images/Rlogo}
O pacote \emph{correlation}\textsuperscript{\citeproc{ref-correlation}{300}} do projeto \emph{easystats}\textsuperscript{\citeproc{ref-easystats}{301}} fornece a função \href{https://cloud.r-project.org/web/packages/correlation/index.html}{\emph{correlation}} para calcular o coeficiente coeficiente Sheperd \(\phi\).

\end{infobox}

\begin{infobox}{images/Rlogo}
O pacote \emph{corrplot}\textsuperscript{\citeproc{ref-corrplot}{225}} fornece a função \href{https://www.rdocumentation.org/packages/corrplot/versions/0.92/topics/cor.mtest}{\emph{cor.mtest}} para calcular os P-valores e intervalos de confiança da matriz de correlação.

\end{infobox}

\begin{infobox}{images/Rlogo}
O pacote \emph{corrplot}\textsuperscript{\citeproc{ref-corrplot}{225}} fornece a função \href{https://www.rdocumentation.org/packages/corrplot/versions/0.92/topics/corrplot}{\emph{corrplot}} para visualização da matriz de correlação.

\end{infobox}

\section{Colinearidade}\label{colinearidade}

\subsection{O que é colinearidade?}\label{o-que-uxe9-colinearidade}

\begin{itemize}
\item
  Colinearidade representa a correlação entre duas variáveis.\textsuperscript{\citeproc{ref-Kim2019}{302}}
\item
  Colinearidade exata indica uma relação linear perfeita entre duas variáveis.\textsuperscript{\citeproc{ref-Kim2019}{302}}
\end{itemize}

\subsection{Como identificar colinearidade na matriz de correlação?}\label{como-identificar-colinearidade-na-matriz-de-correlauxe7uxe3o}

\begin{itemize}
\item
  A colinearidade pode ser identificada na matriz de correlação por meio da análise dos coeficientes de correlação entre as variáveis.\textsuperscript{\citeproc{ref-Kim2019}{302}}
\item
  Valores de correlação próximos de \(1\) ou \(-1\) indicam colinearidade entre as variáveis.\textsuperscript{\citeproc{ref-Kim2019}{302}}
\end{itemize}

\begin{infobox}{images/Rlogo}
O pacote \emph{GGally}\textsuperscript{\citeproc{ref-GGally}{303}} fornece a função \href{https://www.rdocumentation.org/packages/GGally/versions/2.2.1/topics/ggally_cor}{\emph{ggally\_cor}} para estimar a correlação bivariada e exibir o coeficiente de correlação e o P-valor na matriz de correlação.\textsuperscript{\citeproc{ref-GGally}{303}}

\end{infobox}

\section{Correlação entre conjuntos de variáveis}\label{correlauxe7uxe3o-entre-conjuntos-de-variuxe1veis}

\subsection{O que é correlação entre conjuntos de variáveis?}\label{o-que-uxe9-correlauxe7uxe3o-entre-conjuntos-de-variuxe1veis}

\begin{itemize}
\item
  A Correlação Canônica (CCA) analisa a relação entre dois conjuntos de variáveis simultaneamente.\textsuperscript{\citeproc{ref-REF}{\textbf{REF?}}}
\item
  Busca combinações lineares que maximizam a correlação entre os dois blocos.\textsuperscript{\citeproc{ref-REF}{\textbf{REF?}}}
\end{itemize}

\subsection{Quando usar CCA?}\label{quando-usar-cca}

\begin{itemize}
\item
  Quando existem dois blocos distintos de variáveis.\textsuperscript{\citeproc{ref-REF}{\textbf{REF?}}}
\item
  Quando a correlação bivariada é insuficiente para captar padrões multivariados.\textsuperscript{\citeproc{ref-REF}{\textbf{REF?}}}
\end{itemize}

\subsection{Quais são os principais resultados?}\label{quais-suxe3o-os-principais-resultados}

\begin{itemize}
\item
  Correlação canônica (\(\rho_1\), \(\rho_2\), \(...\)): força da associação entre os escores dos blocos.\textsuperscript{\citeproc{ref-REF}{\textbf{REF?}}}
\item
  Escores canônicos (\(U\) e \(V\)): novas variáveis representando os blocos.\textsuperscript{\citeproc{ref-REF}{\textbf{REF?}}}
\item
  \emph{Loadings} e \emph{cross-loadings}: indicam quais variáveis mais contribuem para cada eixo.\textsuperscript{\citeproc{ref-REF}{\textbf{REF?}}}
\end{itemize}

\subsection{Como interpretar a CCA?}\label{como-interpretar-a-cca}

\begin{itemize}
\item
  \(\rho_1\) indica a força do primeiro eixo canônico (\(U_1 \leftrightarrow V_1\)).\textsuperscript{\citeproc{ref-REF}{\textbf{REF?}}}
\item
  Gráficos de \(U_1\) vs \(V_1\) podem revelar padrões por grupo ou gradiente.\textsuperscript{\citeproc{ref-REF}{\textbf{REF?}}}
\item
  \emph{Loadings/cross-loadings} mostram quais variáveis explicam a correlação.\textsuperscript{\citeproc{ref-REF}{\textbf{REF?}}}
\end{itemize}

\subsection{Quais suposições e cuidados?}\label{quais-suposiuxe7uxf5es-e-cuidados}

\begin{itemize}
\item
  As variáveis devem estar padronizadas (escalas comparáveis).\textsuperscript{\citeproc{ref-REF}{\textbf{REF?}}}
\item
  Preferível \(n>\) número de variáveis em cada bloco.\textsuperscript{\citeproc{ref-REF}{\textbf{REF?}}}
\item
  Atenção a multicolinearidade alta (pode exigir CCA regularizada).\textsuperscript{\citeproc{ref-REF}{\textbf{REF?}}}
\end{itemize}

\subsection{O que reportar nos resultados?}\label{o-que-reportar-nos-resultados}

\begin{itemize}
\item
  Valores de \(\rho_1\), \(\rho_2\), \(...\) comteste de Wilks e p-valores.\textsuperscript{\citeproc{ref-REF}{\textbf{REF?}}}
\item
  Figura \(U_1\) vs.~\(V_1\) com interpretação.\textsuperscript{\citeproc{ref-REF}{\textbf{REF?}}}
\item
  Tabela de \emph{loadings} ou \emph{cross-loadings} destacando contribuições relevantes.\textsuperscript{\citeproc{ref-REF}{\textbf{REF?}}}
\item
  Uma interpretação substantiva da relação entre os blocos.\textsuperscript{\citeproc{ref-REF}{\textbf{REF?}}}
\end{itemize}

\begin{Shaded}
\begin{Highlighting}[]
\CommentTok{\# Reproducibilidade}
\FunctionTok{set.seed}\NormalTok{(}\DecValTok{123}\NormalTok{)}

\CommentTok{\# Suponha X: variáveis ambientais; Y: traços de plantas}
\NormalTok{n  }\OtherTok{\textless{}{-}} \DecValTok{120}
\NormalTok{X  }\OtherTok{\textless{}{-}} \FunctionTok{scale}\NormalTok{(}\FunctionTok{cbind}\NormalTok{(}\AttributeTok{pH =} \FunctionTok{rnorm}\NormalTok{(n, }\FloatTok{6.5}\NormalTok{, .}\DecValTok{4}\NormalTok{),}
\AttributeTok{temp =} \FunctionTok{rnorm}\NormalTok{(n, }\DecValTok{20}\NormalTok{, }\DecValTok{3}\NormalTok{),}
\AttributeTok{rain =} \FunctionTok{rnorm}\NormalTok{(n, }\DecValTok{1000}\NormalTok{, }\DecValTok{120}\NormalTok{)))}
\NormalTok{Y  }\OtherTok{\textless{}{-}} \FunctionTok{scale}\NormalTok{(}\FunctionTok{cbind}\NormalTok{(}\AttributeTok{height =} \FloatTok{0.4}\SpecialCharTok{*}\NormalTok{X[, }\StringTok{"temp"}\NormalTok{] }\SpecialCharTok{{-}} \FloatTok{0.3}\SpecialCharTok{*}\NormalTok{X[, }\StringTok{"pH"}\NormalTok{] }\SpecialCharTok{+} \FunctionTok{rnorm}\NormalTok{(n,}\DecValTok{0}\NormalTok{,.}\DecValTok{6}\NormalTok{),}
\AttributeTok{leaf   =} \FloatTok{0.3}\SpecialCharTok{*}\NormalTok{X[, }\StringTok{"rain"}\NormalTok{] }\SpecialCharTok{+} \FloatTok{0.25}\SpecialCharTok{*}\NormalTok{X[, }\StringTok{"temp"}\NormalTok{] }\SpecialCharTok{+} \FunctionTok{rnorm}\NormalTok{(n,}\DecValTok{0}\NormalTok{,.}\DecValTok{6}\NormalTok{),}
\AttributeTok{chl    =} \SpecialCharTok{{-}}\FloatTok{0.35}\SpecialCharTok{*}\NormalTok{X[, }\StringTok{"pH"}\NormalTok{] }\SpecialCharTok{+} \FloatTok{0.3}\SpecialCharTok{*}\NormalTok{X[, }\StringTok{"rain"}\NormalTok{] }\SpecialCharTok{+} \FunctionTok{rnorm}\NormalTok{(n,}\DecValTok{0}\NormalTok{,.}\DecValTok{6}\NormalTok{)))}

\CommentTok{\# (Opcional) fator de cor para o scatter}
\NormalTok{classe }\OtherTok{\textless{}{-}} \FunctionTok{factor}\NormalTok{(}\FunctionTok{sample}\NormalTok{(}\FunctionTok{c}\NormalTok{(}\StringTok{"Forest"}\NormalTok{,}\StringTok{"Herb"}\NormalTok{,}\StringTok{"Planted"}\NormalTok{,}\StringTok{"Shrub"}\NormalTok{), n, }\ConstantTok{TRUE}\NormalTok{))}

\CommentTok{\# {-}{-}{-}{-} CCA (base R) {-}{-}{-}{-}}
\NormalTok{fit }\OtherTok{\textless{}{-}} \FunctionTok{cancor}\NormalTok{(X, Y)   }\CommentTok{\# stats::cancor}
\NormalTok{rho }\OtherTok{\textless{}{-}}\NormalTok{ fit}\SpecialCharTok{$}\NormalTok{cor        }\CommentTok{\# correlações canônicas}

\CommentTok{\# Escores canônicos (U e V)}

\NormalTok{U }\OtherTok{\textless{}{-}} \FunctionTok{scale}\NormalTok{(X) }\SpecialCharTok{\%*\%}\NormalTok{ fit}\SpecialCharTok{$}\NormalTok{xcoef}
\NormalTok{V }\OtherTok{\textless{}{-}} \FunctionTok{scale}\NormalTok{(Y) }\SpecialCharTok{\%*\%}\NormalTok{ fit}\SpecialCharTok{$}\NormalTok{ycoef}
\NormalTok{U1 }\OtherTok{\textless{}{-}}\NormalTok{ U[,}\DecValTok{1}\NormalTok{]; V1 }\OtherTok{\textless{}{-}}\NormalTok{ V[,}\DecValTok{1}\NormalTok{]}

\CommentTok{\# {-}{-}{-}{-} Teste sequencial (Wilks) {-}{-}{-}{-}}

\CommentTok{\# wilks \textless{}{-} CCP::p.asym(rho, N=n, p=ncol(X), q=ncol(Y), tstat="Wilks")}
\CommentTok{\# wilks\_tab \textless{}{-} transform(wilks, rho=round(rho,3),}
\CommentTok{\# p.value=signif(p.value,3))}

\CommentTok{\# {-}{-}{-}{-} Loadings e cross{-}loadings (correlações com escores) {-}{-}{-}{-}}

\CommentTok{\# Loadings: var{-}X com U1.., var{-}Y com V1..}

\NormalTok{loadX }\OtherTok{\textless{}{-}} \FunctionTok{cor}\NormalTok{(X, U[,}\DecValTok{1}\SpecialCharTok{:}\FunctionTok{min}\NormalTok{(}\FunctionTok{ncol}\NormalTok{(X), }\FunctionTok{ncol}\NormalTok{(Y))])}
\NormalTok{loadY }\OtherTok{\textless{}{-}} \FunctionTok{cor}\NormalTok{(Y, V[,}\DecValTok{1}\SpecialCharTok{:}\FunctionTok{min}\NormalTok{(}\FunctionTok{ncol}\NormalTok{(X), }\FunctionTok{ncol}\NormalTok{(Y))])}

\CommentTok{\# Cross{-}loadings: var{-}X com V1.., var{-}Y com U1..}

\NormalTok{crossX }\OtherTok{\textless{}{-}} \FunctionTok{cor}\NormalTok{(X, V[,}\DecValTok{1}\SpecialCharTok{:}\FunctionTok{min}\NormalTok{(}\FunctionTok{ncol}\NormalTok{(X), }\FunctionTok{ncol}\NormalTok{(Y))])}
\NormalTok{crossY }\OtherTok{\textless{}{-}} \FunctionTok{cor}\NormalTok{(Y, U[,}\DecValTok{1}\SpecialCharTok{:}\FunctionTok{min}\NormalTok{(}\FunctionTok{ncol}\NormalTok{(X), }\FunctionTok{ncol}\NormalTok{(Y))])}

\CommentTok{\# {-}{-}{-}{-} Visualizações {-}{-}{-}{-}}

\NormalTok{p1 }\OtherTok{\textless{}{-}}\NormalTok{ ggplot2}\SpecialCharTok{::}\FunctionTok{ggplot}\NormalTok{(}\FunctionTok{data.frame}\NormalTok{(}\AttributeTok{U1=}\NormalTok{U1, }\AttributeTok{V1=}\NormalTok{V1, }\AttributeTok{classe=}\NormalTok{classe),}
\NormalTok{ggplot2}\SpecialCharTok{::}\FunctionTok{aes}\NormalTok{(U1, V1, }\AttributeTok{shape=}\NormalTok{classe)) }\SpecialCharTok{+}
\NormalTok{ggplot2}\SpecialCharTok{::}\FunctionTok{geom\_point}\NormalTok{(}\AttributeTok{alpha=}\NormalTok{.}\DecValTok{75}\NormalTok{) }\SpecialCharTok{+}
\NormalTok{ggplot2}\SpecialCharTok{::}\FunctionTok{geom\_smooth}\NormalTok{(}\AttributeTok{method=}\StringTok{"lm"}\NormalTok{, }\AttributeTok{se=}\ConstantTok{FALSE}\NormalTok{) }\SpecialCharTok{+}
\NormalTok{ggplot2}\SpecialCharTok{::}\FunctionTok{labs}\NormalTok{(}\AttributeTok{x=}\StringTok{"U1 (X → a1)"}\NormalTok{, }\AttributeTok{y=}\StringTok{"V1 (Y → b1)"}\NormalTok{,}
\AttributeTok{subtitle=}\FunctionTok{paste0}\NormalTok{(}\StringTok{"ρ1 = "}\NormalTok{, }\FunctionTok{round}\NormalTok{(rho[}\DecValTok{1}\NormalTok{], }\DecValTok{3}\NormalTok{))) }\SpecialCharTok{+}
\NormalTok{ggplot2}\SpecialCharTok{::}\FunctionTok{theme\_minimal}\NormalTok{()}

\CommentTok{\# Heatmap de cross{-}loadings (quais variáveis de X e Y conectam{-}se ao outro bloco)}

\NormalTok{cx }\OtherTok{\textless{}{-}}\NormalTok{ reshape2}\SpecialCharTok{::}\FunctionTok{melt}\NormalTok{(}\FunctionTok{round}\NormalTok{(crossX, }\DecValTok{2}\NormalTok{), }\AttributeTok{varnames=}\FunctionTok{c}\NormalTok{(}\StringTok{"VarX"}\NormalTok{,}\StringTok{"CompV"}\NormalTok{),}
\AttributeTok{value.name=}\StringTok{"cross"}\NormalTok{)}
\NormalTok{cy }\OtherTok{\textless{}{-}}\NormalTok{ reshape2}\SpecialCharTok{::}\FunctionTok{melt}\NormalTok{(}\FunctionTok{round}\NormalTok{(crossY, }\DecValTok{2}\NormalTok{), }\AttributeTok{varnames=}\FunctionTok{c}\NormalTok{(}\StringTok{"VarY"}\NormalTok{,}\StringTok{"CompU"}\NormalTok{),}
\AttributeTok{value.name=}\StringTok{"cross"}\NormalTok{)}

\FunctionTok{print}\NormalTok{(p1)}
\end{Highlighting}
\end{Shaded}

\begin{figure}

{\centering \includegraphics{Ciencia-com-R_files/figure-latex/cca-example-1} 

}

\caption{Exemplo de análise de correlação canônica (CCA) entre dois conjuntos de variáveis.}\label{fig:cca-example}
\end{figure}

\chapter{\texorpdfstring{\textbf{Regressão}}{Regressão}}\label{regressao}

\section{Análise de regressão}\label{anuxe1lise-de-regressuxe3o}

\subsection{O que é regressão?}\label{o-que-uxe9-regressuxe3o}

\begin{itemize}
\item
  Regressão refere-se a uma equação matemática que permite que uma ou mais variável(is) de desfecho (dependentes) seja(m) prevista(s) a partir de uma ou mais variável(is) independente(s). A regressão implica em uma direção de efeito, mas não garante causalidade.\textsuperscript{\citeproc{ref-greenhalgh1997a}{272}}
\item
  Para estimar os efeitos imparciais de um fator de exposição primária sobre uma variável de desfecho, frequentemente constroem-se modelos estatísticos de regressão.\textsuperscript{\citeproc{ref-bandoli2018}{220}}
\end{itemize}

\begin{infobox}{images/Rlogo}
O pacote \emph{modelsummary}\textsuperscript{\citeproc{ref-modelsummary}{304}} fornece as funções \href{https://www.rdocumentation.org/packages/modelsummary/versions/1.4.1/topics/modelsummary}{\emph{modelsummary}} e \href{https://www.rdocumentation.org/packages/modelsummary/versions/1.4.1/topics/modelplot}{\emph{modelplot}} para gerar tabelas e gráficos de coeficientes de regressão.

\end{infobox}

\begin{infobox}{images/Rlogo}
O pacote \emph{gtsummary}\textsuperscript{\citeproc{ref-gtsummary}{211}} fornece a função \href{https://www.rdocumentation.org/packages/gtsummary/versions/1.6.3/topics/tbl_regression}{\emph{tbl\_regression}} para construção da `Tabela 2' com dados do modelo de regressão.

\end{infobox}

\subsection{Quais são os algoritmos de regressão?}\label{quais-suxe3o-os-algoritmos-de-regressuxe3o}

\begin{itemize}
\item
  Linear: Simples, Múltipla, Polinomial.\textsuperscript{\citeproc{ref-REF}{\textbf{REF?}}}
\item
  Linear generalizado: Binomial (logística), Multinomial, Ordinal, Poisson, Binomial negativa, Gama.\textsuperscript{\citeproc{ref-REF}{\textbf{REF?}}}
\item
  Não-linear (nos parâmetros).\textsuperscript{\citeproc{ref-REF}{\textbf{REF?}}}
\item
  Aditivo generalizado.\textsuperscript{\citeproc{ref-REF}{\textbf{REF?}}}
\item
  Efeitos mistos: Linear, Generalizado.\textsuperscript{\citeproc{ref-REF}{\textbf{REF?}}}
\item
  Sobrevida: Cox, Weibull, Exponencial, Log-normal, Log-logístico.\textsuperscript{\citeproc{ref-REF}{\textbf{REF?}}}
\item
  Regularização: Ridge, LASSO\textsuperscript{\citeproc{ref-REF}{\textbf{REF?}}}
\end{itemize}

\section{Estruturas de análise de regressão}\label{estruturas-de-anuxe1lise-de-regressuxe3o}

\subsection{O que são análises de regressão simples?}\label{o-que-suxe3o-anuxe1lises-de-regressuxe3o-simples}

\begin{itemize}
\item
  A análise de regressão simples consiste em modelos estatísticos com uma variável dependente (desfecho) e uma variável independente (preditor).\textsuperscript{\citeproc{ref-Hidalgo2013}{305}}
\item
  A equação de regressão simples é expressa como \eqref{eq:regressao-simples}, onde \(Y\) é a variável dependente, \(X\) é a variável independente, \(\beta_0\) é o intercepto (constante), \(\beta_1\) é o coeficiente de regressão da variável independente e \(\epsilon\) representa o erro aleatório do modelo.\textsuperscript{\citeproc{ref-Hidalgo2013}{305}}
\end{itemize}

\begin{equation}
\label{eq:regressao-simples}
Y = \beta_0 + \beta_1 X + \epsilon
\end{equation}

\subsection{O que são análises de regressão multivariável?}\label{o-que-suxe3o-anuxe1lises-de-regressuxe3o-multivariuxe1vel}

\begin{itemize}
\item
  A análise multivariável (ou múltiplo) consiste em modelos estatísticos com uma variável dependente (desfecho) e duas ou mais variáveis independentes.\textsuperscript{\citeproc{ref-Hidalgo2013}{305}}
\item
  A equação de regressão multivariável é expressa como \eqref{eq:regressao-multivariavel}, onde \(Y\) é a variável dependente, \(X_1, X_2, ..., X_n\) são as variáveis independentes, \(\beta_0\) é o intercepto (constante), \(\beta_1, \beta_2, ..., \beta_n\) são os coeficientes de regressão das variáveis independentes e \(\epsilon\) representa o erro aleatório do modelo.\textsuperscript{\citeproc{ref-Hidalgo2013}{305}}
\end{itemize}

\begin{equation}
\label{eq:regressao-multivariavel}
Y = \beta_0 + \beta_1 X_1 + \beta_2 X_2 + ... + \beta_n X_n + \epsilon
\end{equation}

\subsection{O que são análises de regressão multivariada?}\label{o-que-suxe3o-anuxe1lises-de-regressuxe3o-multivariada}

\begin{itemize}
\item
  A análise multivariada consiste em modelos estatísticos com duas ou mais variáveis dependente (desfechos) e duas ou mais variáveis independentes.\textsuperscript{\citeproc{ref-Hidalgo2013}{305}}
\item
  A equação de regressão multivariada é expressa como \eqref{eq:regressao-multivariada}, onde \(Y_1, Y_2, ..., Y_m\) são as variáveis dependentes, \(X_1, X_2, ..., X_n\) são as variáveis independentes, \(\beta_{0j}\) é o intercepto (constante) da variável dependente \(Y_j\), \(\beta_{ij}\) são os coeficientes de regressão das variáveis independentes para a variável dependente \(Y_j\) e \(\epsilon_j\) representa o erro aleatório do modelo para a variável dependente \(Y_j\).\textsuperscript{\citeproc{ref-Hidalgo2013}{305}}
\end{itemize}

\begin{align}
\label{eq:regressao-multivariada}
Y_1 &= \beta_{01} + \beta_{11} X_1 + \beta_{12} X_2 + \dots + \beta_{1n} X_n + \epsilon_1 \\
Y_2 &= \beta_{02} + \beta_{21} X_1 + \beta_{22} X_2 + \dots + \beta_{2n} X_n + \epsilon_2 \\
&\vdots \\
Y_m &= \beta_{0m} + \beta_{m1} X_1 + \beta_{m2} X_2 + \dots + \beta_{mn} X_n + \epsilon_m
\end{align}

\section{Tipos e famílias de regressão}\label{tipos-e-famuxedlias-de-regressuxe3o}

\subsection{O que são modelos de regressão linear?}\label{o-que-suxe3o-modelos-de-regressuxe3o-linear}

\begin{itemize}
\tightlist
\item
  Modelos lineares \eqref{eq:regressao-linear} descrevem uma relação linear nos parâmetros entre um desfecho contínuo \(Y\) e um ou mais preditores \(X\).\textsuperscript{\citeproc{ref-REF}{\textbf{REF?}}}
\end{itemize}

\begin{equation}
\label{eq:regressao-linear}
Y = \beta_0 + \sum_{i=1}^{n} \beta_i X_i + \epsilon
\end{equation}

\begin{itemize}
\item
  Assumem erros independentes, de média zero e variância constante (homocedasticidade).\textsuperscript{\citeproc{ref-REF}{\textbf{REF?}}}
\item
  A normalidade dos resíduos é uma hipótese comum para inferência estatística, mas não obrigatória para estimação dos coeficientes.\textsuperscript{\citeproc{ref-REF}{\textbf{REF?}}}
\end{itemize}

\begin{figure}

{\centering \includegraphics{Ciencia-com-R_files/figure-latex/regressao-linear-1} 

}

\caption{Regressão linear.}\label{fig:regressao-linear}
\end{figure}

\subsection{O que são modelos de regressão polinomial?}\label{o-que-suxe3o-modelos-de-regressuxe3o-polinomial}

\begin{itemize}
\item
  São extensões da regressão linear em que se incluem termos elevados a potências das variáveis independentes (ex.: \(X^2\), \(X^3\)), permitindo capturar relações curvas.\textsuperscript{\citeproc{ref-REF}{\textbf{REF?}}}
\item
  Modelos de regressão polinomial continuam sendo lineares nos parâmetros, por isso ainda se enquadram como um caso particular da regressão linear.\textsuperscript{\citeproc{ref-REF}{\textbf{REF?}}}
\end{itemize}

\begin{figure}

{\centering \includegraphics{Ciencia-com-R_files/figure-latex/regressao-polinomial-1} 

}

\caption{Regressão polinomial.}\label{fig:regressao-polinomial}
\end{figure}

\subsection{O que são modelos de regressão não-linear?}\label{o-que-suxe3o-modelos-de-regressuxe3o-nuxe3o-linear}

\begin{itemize}
\item
  São modelos em que a relação entre os parâmetros e a variável resposta não é linear.
\item
  Podem assumir formas funcionais mais complexas (ex.: exponencial, logarítmica, logística).\textsuperscript{\citeproc{ref-REF}{\textbf{REF?}}}
\item
  Importante diferenciar ``não-linear na variável'' (ex.: polinomial) de ``não-linear no parâmetro'' (ex.: modelos logísticos de crescimento).\textsuperscript{\citeproc{ref-REF}{\textbf{REF?}}}
\end{itemize}

\begin{figure}

{\centering \includegraphics{Ciencia-com-R_files/figure-latex/regressao-nao-linear-1} 

}

\caption{Regressão não-linear.}\label{fig:regressao-nao-linear}
\end{figure}

\subsection{O que são modelos de regressão logística?}\label{o-que-suxe3o-modelos-de-regressuxe3o-loguxedstica}

\begin{itemize}
\item
  Modelos logísticos são casos de regressão linear generalizada em que a resposta \(Y\) é binária.\textsuperscript{\citeproc{ref-REF}{\textbf{REF?}}}
\item
  A equação \eqref{eq:regressao-logistica} modela a razão de chances (\emph{odds}) em função dos preditores.\textsuperscript{\citeproc{ref-REF}{\textbf{REF?}}}
\end{itemize}

\begin{equation}
\label{eq:regressao-logistica}
\log\left(\frac{p}{1-p}\right) = \beta_0 + \beta_1 X + ... + \beta_n X_n
\end{equation}

\begin{itemize}
\tightlist
\item
  A ligação (\emph{link}) usada é o logit \eqref{eq:link-logit}.\textsuperscript{\citeproc{ref-REF}{\textbf{REF?}}}
\end{itemize}

\begin{equation}
\label{eq:link-logit}
g(p) = \log\left(\frac{p}{1-p}\right)
\end{equation}

\begin{figure}

{\centering \includegraphics{Ciencia-com-R_files/figure-latex/regressao-logistica-1} 

}

\caption{Regressão logística.}\label{fig:regressao-logistica}
\end{figure}

\subsection{O que são modelos de regressão multinomial?}\label{o-que-suxe3o-modelos-de-regressuxe3o-multinomial}

\begin{itemize}
\item
  Modelos de regressão multinomial são usados quando a variável resposta é categórica com mais de dois níveis não ordenados.\textsuperscript{\citeproc{ref-REF}{\textbf{REF?}}}
\item
  Estendem a regressão logística binária, modelando as razões de chances (\emph{odds ratios}) de cada categoria em relação a uma categoria de referência.\textsuperscript{\citeproc{ref-REF}{\textbf{REF?}}}
\end{itemize}

\begin{figure}

{\centering \includegraphics{Ciencia-com-R_files/figure-latex/regressao-multinomial-1} 

}

\caption{Regressão multinomial}\label{fig:regressao-multinomial}
\end{figure}

\subsection{O que são modelos de regressão ordinal?}\label{o-que-suxe3o-modelos-de-regressuxe3o-ordinal}

\begin{itemize}
\item
  Modelos de regressão ordinal são usados quando a variável resposta é categórica com mais de dois níveis ordenados.\textsuperscript{\citeproc{ref-REF}{\textbf{REF?}}}
\item
  Modelam a probabilidade acumulada de estar em ou abaixo de cada categoria, usando uma função de ligação logit, probit ou log-log.\textsuperscript{\citeproc{ref-REF}{\textbf{REF?}}}
\item
  Assumem a proporcionalidade dos coeficientes entre as categorias (\emph{proportional odds}).\textsuperscript{\citeproc{ref-REF}{\textbf{REF?}}}
\end{itemize}

\subsection{O que são modelos de regressão de Poisson?}\label{o-que-suxe3o-modelos-de-regressuxe3o-de-poisson}

\begin{itemize}
\item
  Modelos de regressão de Poisson são usados quando a variável resposta é uma contagem de eventos não negativos.\textsuperscript{\citeproc{ref-REF}{\textbf{REF?}}}
\item
  Assumem que \(Y \sim Poisson(\mu)\), com \(\mu = E[Y|X]\) relacionado aos preditores via função de ligação log.\textsuperscript{\citeproc{ref-REF}{\textbf{REF?}}}
\item
  A sobre-dispersão (variância maior que a média) pode exigir modelos alternativos como a regressão binomial negativa.\textsuperscript{\citeproc{ref-REF}{\textbf{REF?}}}
\end{itemize}

\begin{figure}

{\centering \includegraphics{Ciencia-com-R_files/figure-latex/regressao-poisson-1} 

}

\caption{Regressão de Poisson.}\label{fig:regressao-poisson}
\end{figure}

\subsection{O que são modelos de regressão binomial negativa?}\label{o-que-suxe3o-modelos-de-regressuxe3o-binomial-negativa}

\begin{itemize}
\item
  Modelos de regressão binomial negativa são usados para contagens superdispersas, onde a variância excede a média.\textsuperscript{\citeproc{ref-REF}{\textbf{REF?}}}
\item
  Introduzem um parâmetro de dispersão adicional para modelar a variabilidade extra.\textsuperscript{\citeproc{ref-REF}{\textbf{REF?}}}
\item
  A função de ligação log é comumente usada, semelhante à regressão de Poisson.\textsuperscript{\citeproc{ref-REF}{\textbf{REF?}}}
\end{itemize}

\subsection{O que são modelos de regressão Gama?}\label{o-que-suxe3o-modelos-de-regressuxe3o-gama}

\begin{itemize}
\item
  Modelos de regressão Gama são usados para variáveis resposta contínuas e positivas, frequentemente com distribuição assimétrica.\textsuperscript{\citeproc{ref-REF}{\textbf{REF?}}}
\item
  A função de ligação log é comumente usada para garantir predições positivas.\textsuperscript{\citeproc{ref-REF}{\textbf{REF?}}}
\end{itemize}

\subsection{O que são modelos de regressão com efeitos mistos?}\label{o-que-suxe3o-modelos-de-regressuxe3o-com-efeitos-mistos}

\begin{itemize}
\item
  Modelos de efeitos mistos incorporam efeitos fixos (coeficientes comuns a todos os indivíduos) e efeitos aleatórios (variações específicas de grupos ou indivíduos).\textsuperscript{\citeproc{ref-REF}{\textbf{REF?}}}
\item
  Usados para dados hierárquicos ou longitudinais, onde observações estão agrupadas.\textsuperscript{\citeproc{ref-REF}{\textbf{REF?}}}
\item
  Permitem modelar correlações intra-grupo e variabilidade entre grupos.\textsuperscript{\citeproc{ref-REF}{\textbf{REF?}}}
\end{itemize}

\subsection{O que são modelos de regressão com efeitos mistos generalizados?}\label{o-que-suxe3o-modelos-de-regressuxe3o-com-efeitos-mistos-generalizados}

\begin{itemize}
\item
  Modelos de efeitos mistos generalizados (GLMM) estendem os modelos de efeitos mistos para variáveis resposta que seguem distribuições da família exponencial (ex.: binomial, Poisson).\textsuperscript{\citeproc{ref-REF}{\textbf{REF?}}}
\item
  Combinam a flexibilidade dos modelos lineares generalizados com a capacidade de modelar correlações e variabilidade entre grupos.\textsuperscript{\citeproc{ref-REF}{\textbf{REF?}}}
\item
  Usados para dados hierárquicos ou longitudinais com desfechos não normais.\textsuperscript{\citeproc{ref-REF}{\textbf{REF?}}}
\end{itemize}

\subsection{O que são modelos de regressão ridge?}\label{o-que-suxe3o-modelos-de-regressuxe3o-ridge}

\begin{itemize}
\item
  Regressão ridge é um modelo linear regularizado que adiciona uma penalização L2 à soma dos quadrados dos coeficientes.\textsuperscript{\citeproc{ref-REF}{\textbf{REF?}}}
\item
  Ajuda a reduzir multicolinearidade e overfitting, encolhendo os coeficientes em direção a zero, mas nunca os tornando exatamente nulos.\textsuperscript{\citeproc{ref-REF}{\textbf{REF?}}}
\item
  O hiperparâmetro de regularização é \(\lambda\), controlando a intensidade da penalização. Valores maiores de \(\lambda\) resultam em maior encolhimento dos coeficientes.\textsuperscript{\citeproc{ref-REF}{\textbf{REF?}}}
\end{itemize}

\begin{figure}

{\centering \includegraphics{Ciencia-com-R_files/figure-latex/regressao-ridge-1} 

}

\caption{Regressão ridge.}\label{fig:regressao-ridge}
\end{figure}

\subsection{O que são modelos de regressão LASSO?}\label{o-que-suxe3o-modelos-de-regressuxe3o-lasso}

\begin{itemize}
\item
  Regressão LASSO (\emph{Least Absolute Shrinkage and Selection Operator}) utiliza penalização L1, que pode zerar coeficientes.\textsuperscript{\citeproc{ref-REF}{\textbf{REF?}}}
\item
  Além de reduzir \emph{overfitting}, também realiza seleção automática de variáveis.\textsuperscript{\citeproc{ref-REF}{\textbf{REF?}}}
\item
  Enquanto a regressão ridge mantém todos os preditores, a LASSO pode excluir variáveis irrelevantes.\textsuperscript{\citeproc{ref-REF}{\textbf{REF?}}}
\end{itemize}

\section{Preparação de variáveis}\label{preparauxe7uxe3o-de-variuxe1veis}

\subsection{Como preparar as variáveis categóricas para análise de regressão?}\label{como-preparar-as-variuxe1veis-categuxf3ricas-para-anuxe1lise-de-regressuxe3o}

\begin{itemize}
\item
  Variáveis fictícias (\emph{dummy}) compreendem variáveis criadas para introduzir, nos modelos de regressão, informações contidas em outras variáveis que não podem ser medidas em escala numérica.\textsuperscript{\citeproc{ref-suits1957}{306}}
\item
  Variáveis categóricas nominais, com 2 ou mais níveis, devem ser subdivididas em variáveis fictícias dicotômicas para ser usada em modelos de regressão.\textsuperscript{\citeproc{ref-Healy1995}{307}}
\item
  Cada nível da variável categórica nominal será convertido em uma nova variável fictícias dicotômica, tal que a nova variável dicotômica assume valor 1 para a presença do nível correspondente e 0 em qualquer outro caso.\textsuperscript{\citeproc{ref-Healy1995}{307}}
\end{itemize}

\begin{infobox}{images/Rlogo}
O pacote \emph{fastDummies}\textsuperscript{\citeproc{ref-fastDummies}{308}} fornece a função \href{https://www.rdocumentation.org/packages/fastDummies/versions/1.7.3/topics/dummy_columns}{\emph{dummy\_cols}} para preparar as variáveis categóricas fictícias para análise de regressão.

\end{infobox}

\subsection{Por que é comum escolher a categoria mais frequente como referência em modelos epidemiológicos?}\label{por-que-uxe9-comum-escolher-a-categoria-mais-frequente-como-referuxeancia-em-modelos-epidemioluxf3gicos}

\begin{itemize}
\item
  Maior estabilidade estatística: a categoria mais frequente costuma gerar estimativas mais estáveis, com menor erro padrão nos coeficientes das demais categorias.\textsuperscript{\citeproc{ref-REF}{\textbf{REF?}}}
\item
  A escolha da referência não altera o ajuste nem o valor predito pelo modelo --- apenas muda o ponto de comparação.\textsuperscript{\citeproc{ref-REF}{\textbf{REF?}}}
\end{itemize}

\section{Multicolinearidade}\label{multicolinearidade}

\subsection{O que é multicolinearidade?}\label{o-que-uxe9-multicolinearidade}

\begin{itemize}
\tightlist
\item
  Multicolinearidade representa a intercorrelação entre as variáveis independentes (explanatórias) de um modelo.\textsuperscript{\citeproc{ref-Kim2019}{302}}
\end{itemize}

\subsection{Como diagnosticar multicolinearidade de forma quantitativa?}\label{como-diagnosticar-multicolinearidade-de-forma-quantitativa}

\begin{itemize}
\item
  Verifique a existência de multicolinearidade entre as variáveis candidatas.\textsuperscript{\citeproc{ref-Sun1996}{309}}
\item
  O Coeficiente de determinação (\(R^2\)) é uma medida de quão bem as variáveis independentes explicam a variabilidade da variável dependente. Valores próximos a 1 indicam que as variáveis independentes estão fortemente correlacionadas entre si, o que pode indicar multicolinearidade.\textsuperscript{\citeproc{ref-Kim2019}{302}}
\item
  O Fator de Inflação da Variância (\emph{variance inflation factor}, VIF) é uma medida que quantifica o quanto a variância de um coeficiente de regressão é inflacionada devido à multicolinearidade. Valores de VIF maiores que 10 são frequentemente considerados indicativos de multicolinearidade significativa.\textsuperscript{\citeproc{ref-Kim2019}{302}}
\item
  O recíproco da VIF é chamado de Tolerância, que mede a proporção da variância de uma variável independente que não é explicada pelas outras variáveis independentes. Valores baixos de Tolerância (geralmente abaixo de 0.1) indicam multicolinearidade.\textsuperscript{\citeproc{ref-Kim2019}{302}}
\item
  O número de condições (\emph{Condition Number}) é uma medida que avalia a estabilidade numérica de um modelo de regressão. Valores altos (entre 10 de 30) indicam multicolinearidade, e valores maiores que 30 indicam forte multicolinearidade.\textsuperscript{\citeproc{ref-Kim2019}{302}}
\end{itemize}

\begin{figure}

{\centering \includegraphics{Ciencia-com-R_files/figure-latex/multicolinearidade-1} 

}

\caption{Multicolinearidade entre variáveis candidatas em modelos de regressão multivariável.}\label{fig:multicolinearidade}
\end{figure}

\begin{infobox}{images/Rlogo}
O pacote \emph{GGally}\textsuperscript{\citeproc{ref-GGally}{303}} fornece a função \href{https://www.rdocumentation.org/packages/GGally/versions/2.2.1/topics/ggpairs}{\emph{ggpairs}} para criar uma matriz gráfica de correlações bivariadas.

\end{infobox}

\begin{infobox}{images/Rlogo}
O pacote \emph{car}\textsuperscript{\citeproc{ref-car}{310}} fornece a função \href{https://www.rdocumentation.org/packages/car/versions/3.1-3/topics/vif}{\emph{vif}} para calcular o fator de inflação da variância (VIF).

\end{infobox}

\subsection{O que fazer em caso de multicolinearidade elevada?}\label{o-que-fazer-em-caso-de-multicolinearidade-elevada}

\begin{itemize}
\item
  Verifique a transformação (codificação) de variáveis numéricas em categóricas.\textsuperscript{\citeproc{ref-Kim2019}{302}}
\item
  Aumente o tamanho da amostra, se possível, para reduzir a multicolinearidade.\textsuperscript{\citeproc{ref-Kim2019}{302}}
\item
  Combine níveis de variáveis categóricas com baixa frequência de ocorrência.\textsuperscript{\citeproc{ref-Kim2019}{302}}
\item
  Combine variáveis numéricas altamente correlacionadas em uma única variável composta, como a média ou soma das variáveis.\textsuperscript{\citeproc{ref-Kim2019}{302}}
\item
  Considere a exclusão de variáveis altamente correlacionadas do modelo, especialmente se elas não forem essenciais para a análise.\textsuperscript{\citeproc{ref-Kim2019}{302}}
\item
  Use técnicas de seleção de variáveis, como seleção passo a passo, para identificar e remover variáveis redundantes.\textsuperscript{\citeproc{ref-Kim2019}{302}}
\item
  Use técnicas de regularização, como regressão ridge ou LASSO, que podem lidar com multicolinearidade ao penalizar coeficientes de regressão.\textsuperscript{\citeproc{ref-Kim2019}{302}}
\end{itemize}

\section{Redução de dimensionalidade}\label{reduuxe7uxe3o-de-dimensionalidade}

\subsection{Correlação bivariada pode ser usada para seleção de variáveis em modelos de regressão multivariável?}\label{correlauxe7uxe3o-bivariada-pode-ser-usada-para-seleuxe7uxe3o-de-variuxe1veis-em-modelos-de-regressuxe3o-multivariuxe1vel}

\begin{itemize}
\item
  Seleção bivariada de variáveis --- isto é, aplicação de testes de correlação em pares de variáveis candidatas e variável de desfecho afim de selecionar quais serão incluídas no modelo multivariável --- é um dos erros mais comuns na literatura.\textsuperscript{\citeproc{ref-heinze2016}{288},\citeproc{ref-Sun1996}{309},\citeproc{ref-Dales1978}{311}}
\item
  A seleção bivariada de variáveis torna o modelo mais suscetível a otimismo no ajuste se as variáveis de confundimento não são adequadamente controladas.\textsuperscript{\citeproc{ref-Sun1996}{309},\citeproc{ref-Dales1978}{311}}
\end{itemize}

\subsection{Variáveis sem significância estatística devem ser excluídas do modelo final?}\label{variuxe1veis-sem-significuxe2ncia-estatuxedstica-devem-ser-excluuxeddas-do-modelo-final}

\begin{itemize}
\item
  Eliminar uma variável de um modelo significa anular o seu coeficiente de regressão (\(\beta = 0\)), mesmo que o valor estimado pelos dados seja outro. Desta forma, os resultados se afastam de uma solução de máxima verossimilhança (que tem fundamento teórico) e o modelo resultante é intencionalmente subótimo.\textsuperscript{\citeproc{ref-heinze2016}{288}}
\item
  Os coeficientes de regressão geralmente dependem do conjunto de variáveis do modelo e, portanto, podem mudam de valor (``mudança na estimativa'' positiva ou negativa) se uma (ou mais) variável(is) for(em) eliminada(s) do modelo.\textsuperscript{\citeproc{ref-heinze2016}{288}}
\end{itemize}

\subsection{Por que métodos de regressão gradual não são recomendados para seleção de variáveis em modelos de regressão multivariável?}\label{por-que-muxe9todos-de-regressuxe3o-gradual-nuxe3o-suxe3o-recomendados-para-seleuxe7uxe3o-de-variuxe1veis-em-modelos-de-regressuxe3o-multivariuxe1vel}

\begin{itemize}
\item
  Métodos diferentes de regressão gradual podem produzir diferentes seleções de variáveis de um mesmo banco de dados.\textsuperscript{\citeproc{ref-Healy1995}{307}}
\item
  Nenhum método de regressão gradual garante a seleção ótima de variáveis de um banco de dados.\textsuperscript{\citeproc{ref-Healy1995}{307}}
\item
  As regras de término da regressão baseadas em P-valor tendem a ser arbitrárias.\textsuperscript{\citeproc{ref-Healy1995}{307}}
\end{itemize}

\subsection{O que pode ser feito para reduzir o número de variáveis candidatas em modelos de regressão multivariável?}\label{o-que-pode-ser-feito-para-reduzir-o-nuxfamero-de-variuxe1veis-candidatas-em-modelos-de-regressuxe3o-multivariuxe1vel}

\begin{itemize}
\item
  Em caso de uma proporção baixa entre o número de participantes e de variáveis, use o conhecimento prévio da literatura para selecionar um pequeno conjunto de variáveis candidatas.\textsuperscript{\citeproc{ref-Sun1996}{309}}
\item
  Colapse categorias com contagem nula (células com valor igual a 0) de variáveis candidatas.\textsuperscript{\citeproc{ref-Sun1996}{309}}
\item
  Use simulações de dados para identificar qual(is) variável(is) está(ão) causando problemas de convergência do ajuste do modelo.\textsuperscript{\citeproc{ref-Sun1996}{309}}
\item
  A eliminação retroativa tem sido recomendada como a abordagem de regressão gradual mais confiável entre aquelas que podem ser facilmente alcançadas com programas de computador.\textsuperscript{\citeproc{ref-heinze2016}{288}}
\end{itemize}

\subsection{Quando devemos forçar uma variável no modelo?}\label{quando-devemos-foruxe7ar-uma-variuxe1vel-no-modelo}

\begin{itemize}
\tightlist
\item
  Sempre que houver base teórica ou evidência prévia forte, ou se for a variável de exposição principal.\textsuperscript{\citeproc{ref-Greenland1989}{251}}
\end{itemize}

\section{Seleção de variáveis em regressão}\label{seleuxe7uxe3o-de-variuxe1veis-em-regressuxe3o}

\subsection{O que é seleção de variáveis em regressão?}\label{o-que-uxe9-seleuxe7uxe3o-de-variuxe1veis-em-regressuxe3o}

\begin{itemize}
\tightlist
\item
  Seleção de variáveis em regressão consiste em identificar, dentre um conjunto de preditores disponíveis, quais devem ser incluídos no modelo para otimizar o equilíbrio entre ajuste e parcimônia.\textsuperscript{\citeproc{ref-lindsey2011}{312}}
\end{itemize}

\subsection{Quais são os principais critérios de informação usados na seleção de variáveis?}\label{quais-suxe3o-os-principais-crituxe9rios-de-informauxe7uxe3o-usados-na-seleuxe7uxe3o-de-variuxe1veis}

\begin{itemize}
\item
  Critérios de informação avaliam o ajuste do modelo penalizando a complexidade (número de preditores), ajudando a evitar \emph{overfitting}.\textsuperscript{\citeproc{ref-lindsey2011}{312}}
\item
  \(R^2_{adj}\) \eqref{eq:R2-ajustado} penaliza o \(R^2\) pelo número de preditores, reduzindo o viés em modelos com muitas variáveis, onde \(n\) é o tamanho amostral, \(k\) o número de preditores, \(RSS\) a soma dos quadrados dos resíduos e \(SST\) a soma total dos quadrados.
\end{itemize}

\begin{equation}
\label{eq:R2-ajustado}
R^2_{adj} = 1 - \frac{(n-1)}{(n - k - 1)} \cdot \frac{RSS}{SST}
\end{equation}

\begin{itemize}
\tightlist
\item
  \(AIC\) (Akaike Information Criterion) \eqref{eq:AIC} mede o equilíbrio entre ajuste e complexidade:
\end{itemize}

\begin{equation}
\label{eq:AIC}
AIC = n \cdot \log\left(\frac{RSS}{n}\right) + 2k + n + n \cdot \log(2\pi)
\end{equation}

\begin{itemize}
\tightlist
\item
  \(AICc\) \eqref{eq:AICc} é uma versão corrigida do AIC, preferida para amostras pequenas:
\end{itemize}

\begin{equation}
\label{eq:AICc}
AIC_c = AIC + \frac{2(k+2)(k+3)}{n - (k + 2) - 1}
\end{equation}

\begin{itemize}
\tightlist
\item
  \(C_p\) de Mallows compara o erro do modelo reduzido com o modelo completo, idealmente satisfazendo \(C_p \approx p\), onde \(m\) é o número total de preditores disponíveis, \(p\) o número de parâmetros (incluindo o intercepto), e \(RSS_{FULL}\) o erro quadrático residual do modelo completo:
\end{itemize}

\begin{equation}
\label{eq:Cp}
C_p = (n - m - 1)\frac{RSS}{RSS_{FULL}} - (n - 2p)
\end{equation}

\begin{itemize}
\tightlist
\item
  \(BIC\) (Bayesian Information Criterion) \eqref{eq:BIC} penaliza fortemente modelos complexos:
\end{itemize}

\begin{equation}
\label{eq:BIC}
BIC = n \cdot \log\left(\frac{RSS}{n}\right) + k \cdot \log(n) + n + n \cdot \log(2\pi)
\end{equation}

\subsection{Quais algoritmos podem ser usados para seleção automática?}\label{quais-algoritmos-podem-ser-usados-para-seleuxe7uxe3o-automuxe1tica}

\begin{itemize}
\item
  Seleção progressiva (\emph{forward selection}): começa com o modelo nulo e adiciona, a cada iteração, a variável que mais melhora o critério escolhido. O processo para quando nenhuma nova variável melhora o modelo.\textsuperscript{\citeproc{ref-lindsey2011}{312}}
\item
  Eliminação retrógrada (\emph{backward elimination}): parte do modelo completo e remove, a cada iteração, a variável cuja exclusão mais melhora o critério. O processo para quando nenhuma remoção melhora o ajuste.\textsuperscript{\citeproc{ref-lindsey2011}{312}}
\item
  \emph{Leaps-and-bounds}: método exato que examina apenas uma fração dos \(2^m\) modelos possíveis, determinando os melhores subconjuntos para cada tamanho de preditor (usando os critérios AIC, BIC, AICc, R² ajustado e Cp).\textsuperscript{\citeproc{ref-lindsey2011}{312}}
\item
  Esses métodos podem divergir em presença de alta multicolinearidade ou amostras pequenas, e devem ser acompanhados de diagnóstico de resíduos e validação cruzada.\textsuperscript{\citeproc{ref-lindsey2011}{312}}
\end{itemize}

\begin{infobox}{images/Rlogo}
O pacote \emph{leaps}\textsuperscript{\citeproc{ref-leaps}{313}} fornece a função \href{https://www.rdocumentation.org/packages/leaps/versions/3.1/topics/regsubsets}{\emph{regsubsets}} para realizar os métodos de seleção de variáveis.

\end{infobox}

\begin{infobox}{images/Rlogo}
O pacote \emph{olsrr}\textsuperscript{\citeproc{ref-olsrr}{314}} fornece a função \href{https://www.rdocumentation.org/packages/olsrr/versions/0.6.0/topics/ols_step_all_possible}{\emph{ols\_step\_all\_possible}} para testar todos os subconjuntos de potenciais preditores de uma regressão.

\end{infobox}

\begin{infobox}{images/Rlogo}
O pacote \emph{olsrr}\textsuperscript{\citeproc{ref-olsrr}{314}} fornece a função \href{https://www.rdocumentation.org/packages/olsrr/versions/0.6.0/topics/ols_step_best_subset}{\emph{ols\_step\_best\_subset}} para selecionar o melhor de todos os subconjuntos de potenciais preditores de uma regressão, de acordo com critérios objetivos.

\end{infobox}

\chapter{\texorpdfstring{\textbf{Redes}}{Redes}}\label{redes}

\section{Análise de redes}\label{anuxe1lise-de-redes}

\subsection{O que é análise de rede?}\label{o-que-uxe9-anuxe1lise-de-rede}

\begin{itemize}
\tightlist
\item
  .\textsuperscript{\citeproc{ref-REF}{\textbf{REF?}}}
\end{itemize}

\cftaddtitleline{toc}{chapter}{\rule{\textwidth}{0.4pt}}{}

\chapter*{\texorpdfstring{\emph{PARTE 6: MODELAGEM}}{PARTE 6: MODELAGEM}}\label{parte-6}
\addcontentsline{toc}{chapter}{\emph{PARTE 6: MODELAGEM}}

\par\noindent\rule{\textwidth}{0.05in}

\section*{Estratégias para entender relações complexas, prever resultados e explorar padrões ocultos}\label{estratuxe9gias-para-entender-relauxe7uxf5es-complexas-prever-resultados-e-explorar-padruxf5es-ocultos}

\markboth{}{}

\chapter{\texorpdfstring{\textbf{Modelos}}{Modelos}}\label{modelos}

\section{Modelos}\label{modelos-1}

\subsection{O que são modelos?}\label{o-que-suxe3o-modelos}

\begin{itemize}
\tightlist
\item
  Modelos são representações simplificadas de um sistema real, usados para entender, prever ou controlar fenômenos complexos.\textsuperscript{\citeproc{ref-REF}{\textbf{REF?}}}
\end{itemize}

\subsection{O que é modelagem?}\label{o-que-uxe9-modelagem}

\begin{itemize}
\tightlist
\item
  Modelagem é o processo de usar dados para selecionar um modelo matemático explícito que represente o processo gerador dos dados.\textsuperscript{\citeproc{ref-Greenland1989}{251}}
\end{itemize}

\subsection{Por que a escolha do modelo é complexa?}\label{por-que-a-escolha-do-modelo-uxe9-complexa}

\begin{itemize}
\item
  Há inúmeras combinações possíveis de variáveis, formas funcionais (lineares, quadráticas, transformações), interações e formas do desfecho, o que torna o espaço de possibilidades muito amplo.\textsuperscript{\citeproc{ref-Greenland1989}{251}}
\item
  Todos os modelos são errados, mas alguns são úteis.\textsuperscript{\citeproc{ref-box1976}{315}}
\end{itemize}

\begin{infobox}{images/Rlogo}
O pacote \emph{equatiomatic}\textsuperscript{\citeproc{ref-equatiomatic}{316}} fornece a função \href{https://www.rdocumentation.org/packages/equatiomatic/versions/0.3.1/topics/extract_eq}{\emph{extract\_eq}} para extrair a equação dos modelos em formato LaTeX para visualização.

\end{infobox}

\subsection{O que diferencia modelos clássicos e modernos em predição?}\label{o-que-diferencia-modelos-cluxe1ssicos-e-modernos-em-prediuxe7uxe3o}

\begin{itemize}
\tightlist
\item
  Modelos clássicos, como a regressão logística e as árvores de decisão, contrastam com os modelos modernos, como máquinas de vetor de suporte, redes neurais e \emph{random forests} , principalmente pela maior flexibilidade e capacidade destes últimos de capturar não linearidades e interações.\textsuperscript{\citeproc{ref-vanderploeg2014}{317}}
\end{itemize}

\section{Modelos estocásticos}\label{modelos-estocuxe1sticos}

\subsection{O que são modelos estocásticos?}\label{o-que-suxe3o-modelos-estocuxe1sticos}

\begin{itemize}
\tightlist
\item
  .\textsuperscript{\citeproc{ref-REF}{\textbf{REF?}}}
\end{itemize}

\subsection{O que são cadeias de Markov?}\label{o-que-suxe3o-cadeias-de-markov}

\begin{itemize}
\tightlist
\item
  As cadeias de Markov descrevem processos em que o estado futuro depende apenas do estado presente, e não da trajetória passada.\textsuperscript{\citeproc{ref-huxe4ggstruxf6m2007}{318}}
\end{itemize}

\begin{figure}

{\centering \includegraphics{Ciencia-com-R_files/figure-latex/markov-1} 

}

\caption{Cadeia de Markov com 3 estados (a, b, c) e suas probabilidades de transição.}\label{fig:markov}
\end{figure}

\begin{figure}

{\centering \includegraphics{Ciencia-com-R_files/figure-latex/markov-sim-1} 

}

\caption{Trajetória de estados e proporção acumulada por estado em uma cadeia de Markov com 3 estados (a, b, c).}\label{fig:markov-sim}
\end{figure}

\begin{infobox}{images/Rlogo}
O pacote \emph{markovchain}\textsuperscript{\citeproc{ref-markovchain}{319}} fornece a função \href{https://www.rdocumentation.org/packages/markovchain/versions/0.9.5/topics/createSequenceMatrix}{\emph{markovchainFit}} ajusta uma cadeia com base em dados observados.

\end{infobox}

\section{Efeito fixo}\label{efeito-fixo}

\subsection{O que é efeito fixo?}\label{o-que-uxe9-efeito-fixo}

\begin{itemize}
\item
  Efeito fixo é a relação média entre variáveis assumida como igual para todos os grupos ou indivíduos, representando o comportamento populacional esperado.\textsuperscript{\citeproc{ref-REF}{\textbf{REF?}}}
\item
  Ele descreve tendências sistemáticas e reprodutíveis que não dependem de pertencer a um grupo específico.\textsuperscript{\citeproc{ref-REF}{\textbf{REF?}}}
\item
  Em modelos estatísticos, corresponde aos parâmetros estimados globalmente a partir de todos os dados.\textsuperscript{\citeproc{ref-REF}{\textbf{REF?}}}
\end{itemize}

\section{Efeito aleatório}\label{efeito-aleatuxf3rio}

\subsection{O que é efeito aleatório?}\label{o-que-uxe9-efeito-aleatuxf3rio}

\begin{itemize}
\item
  Efeito aleatório representa desvios específicos de grupos ou unidades em relação ao efeito fixo.\textsuperscript{\citeproc{ref-REF}{\textbf{REF?}}}
\item
  Ele modela a variabilidade entre grupos, assumindo que esses desvios são amostras de uma distribuição comum.\textsuperscript{\citeproc{ref-REF}{\textbf{REF?}}}
\item
  Não busca estimar cada grupo isoladamente, mas sim quantificar a variabilidade entre eles.\textsuperscript{\citeproc{ref-REF}{\textbf{REF?}}}
\end{itemize}

\section{Efeito misto}\label{efeito-misto}

\subsection{O que é efeito misto?}\label{o-que-uxe9-efeito-misto}

\begin{itemize}
\item
  Um modelo de efeitos mistos combina efeitos fixos e aleatórios em uma única estrutura estatística.\textsuperscript{\citeproc{ref-REF}{\textbf{REF?}}}
\item
  Ele permite estimar tendências globais ao mesmo tempo em que ajusta variações específicas por grupo.\textsuperscript{\citeproc{ref-REF}{\textbf{REF?}}}
\item
  Essa combinação possibilita inferência correta mesmo na presença de heterogeneidade, evitando armadilhas como o Paradoxo de Simpson.\textsuperscript{\citeproc{ref-REF}{\textbf{REF?}}}
\end{itemize}

\begin{figure}

{\centering \includegraphics{Ciencia-com-R_files/figure-latex/modelo-efeitos-1} 

}

\caption{Efeitos fixos, aleatórios e mistos em dados simulados com paradoxo de Simpson. As linhas vermelhas representam os efeitos dentro dos grupos, enquanto as linhas cinza e preta representam os efeitos globais (naive e fixo, respectivamente). O modelo misto (linhas coloridas) captura os efeitos dentro dos grupos sem extrapolar além dos dados observados.}\label{fig:modelo-efeitos}
\end{figure}

\section{Efeito principal}\label{efeito-principal}

\subsection{O que é efeito principal?}\label{o-que-uxe9-efeito-principal}

\begin{itemize}
\tightlist
\item
  .\textsuperscript{\citeproc{ref-Bours2023}{320}}
\end{itemize}

\section{Efeito de interação}\label{efeito-de-interauxe7uxe3o}

\subsection{O que é efeito de interação?}\label{o-que-uxe9-efeito-de-interauxe7uxe3o}

\begin{itemize}
\item
  A interação - representada pelo símbolo * - é o termo estatístico empregado para representar a heterogeneidade de um determinado efeito.\textsuperscript{\citeproc{ref-Altman1996}{321}}
\item
  .\textsuperscript{\citeproc{ref-Bours2023}{320}}
\end{itemize}

\begin{figure}

{\centering \includegraphics{Ciencia-com-R_files/figure-latex/efeito-interacao-direta-1} 

}

\caption{Análise de efeito de interação (direta) entre grupos e tempo. Retas paralelas sugerem ausência de efeito de interação.}\label{fig:efeito-interacao-direta}
\end{figure}

\begin{figure}

{\centering \includegraphics{Ciencia-com-R_files/figure-latex/efeito-interacao-inversa-1} 

}

\caption{Análise de efeito de interação (inversa) entre grupos e tempo. Retas paralelas sugerem ausência de efeito de interação.}\label{fig:efeito-interacao-inversa}
\end{figure}

\begin{infobox}{images/Rlogo}
O pacote \emph{nlme}\textsuperscript{\citeproc{ref-nlme}{322}} fornece a função \href{https://www.rdocumentation.org/packages/nlme/versions/3.1-163/topics/nlme}{\emph{nlme}} para ajustar um modelo de regressão misto não linear.

\end{infobox}

\begin{infobox}{images/Rlogo}
O pacote \emph{mmrm}\textsuperscript{\citeproc{ref-mmrm}{323}} fornece a função \href{https://rdrr.io/cran/mmrm/man/mmrm.html}{\emph{mmrm}} para ajuste de um modelo de regressão misto linear.

\end{infobox}

\begin{infobox}{images/Rlogo}
O pacote \emph{emmeans}\textsuperscript{\citeproc{ref-emmeans}{324}} fornece a função \href{https://www.rdocumentation.org/packages/emmeans/versions/1.8.7/topics/emmeans}{\emph{emmeans}} para calcular as médias marginais dos fatores e suas combinações de um modelo de regressão misto linear.

\end{infobox}

\section{Efeito de mediação}\label{efeito-de-mediauxe7uxe3o}

\subsection{O que é um mediador de efeito?}\label{o-que-uxe9-um-mediador-de-efeito}

\begin{itemize}
\item
  .\textsuperscript{\citeproc{ref-Baron1986}{325}}
\item
  .\textsuperscript{\citeproc{ref-Bours2023}{320}}
\end{itemize}

\subsection{O que é efeito de mediação?}\label{o-que-uxe9-efeito-de-mediauxe7uxe3o}

\begin{itemize}
\item
  .\textsuperscript{\citeproc{ref-Baron1986}{325}}
\item
  .\textsuperscript{\citeproc{ref-Bours2023}{320}}
\end{itemize}

\subsection{O que é efeito direto?}\label{o-que-uxe9-efeito-direto}

\begin{itemize}
\item
  .\textsuperscript{\citeproc{ref-Baron1986}{325}}
\item
  .\textsuperscript{\citeproc{ref-Bours2023}{320}}
\end{itemize}

\subsection{O que é efeito indireto?}\label{o-que-uxe9-efeito-indireto}

\begin{itemize}
\item
  .\textsuperscript{\citeproc{ref-Baron1986}{325}}
\item
  .\textsuperscript{\citeproc{ref-Bours2023}{320}}
\end{itemize}

\subsection{O que é efeito total?}\label{o-que-uxe9-efeito-total}

\begin{itemize}
\item
  .\textsuperscript{\citeproc{ref-Baron1986}{325}}
\item
  .\textsuperscript{\citeproc{ref-Bours2023}{320}}
\end{itemize}

\section{Efeito de modificação}\label{efeito-de-modificauxe7uxe3o}

\subsection{O que é um modificador de efeito?}\label{o-que-uxe9-um-modificador-de-efeito}

\begin{itemize}
\tightlist
\item
  .\textsuperscript{\citeproc{ref-Bours2023}{320}}
\end{itemize}

\subsection{O que é efeito de modificação?}\label{o-que-uxe9-efeito-de-modificauxe7uxe3o}

\begin{itemize}
\tightlist
\item
  .\textsuperscript{\citeproc{ref-Bours2023}{320}}
\end{itemize}

\section{Preditores e desfechos}\label{preditores-e-desfechos}

\subsection{O que são desfechos de um modelo?}\label{o-que-suxe3o-desfechos-de-um-modelo}

\begin{itemize}
\tightlist
\item
  .\textsuperscript{\citeproc{ref-REF}{\textbf{REF?}}}
\end{itemize}

\subsection{O que são preditores de um modelo?}\label{o-que-suxe3o-preditores-de-um-modelo}

\begin{itemize}
\tightlist
\item
  .\textsuperscript{\citeproc{ref-REF}{\textbf{REF?}}}
\end{itemize}

\subsection{Como selecionar preditores para um modelo?}\label{como-selecionar-preditores-para-um-modelo}

\begin{itemize}
\tightlist
\item
  .\textsuperscript{\citeproc{ref-REF}{\textbf{REF?}}}
\end{itemize}

\section{Desempenho e estabilidade de modelos}\label{desempenho-e-estabilidade-de-modelos}

\subsection{Como avaliar o desempenho dos modelos?}\label{como-avaliar-o-desempenho-dos-modelos}

\begin{itemize}
\item
  Pela área sob a curva ROC em conjunto com o otimismo (diferença entre AUC aparente e validada).\textsuperscript{\citeproc{ref-vanderploeg2014}{317}}
\item
  O desempenho melhora com maior tamanho amostral, mas de forma desigual entre técnicas.\textsuperscript{\citeproc{ref-vanderploeg2014}{317}}
\end{itemize}

\subsection{Qual modelo alcança estabilidade mais rapidamente?}\label{qual-modelo-alcanuxe7a-estabilidade-mais-rapidamente}

\begin{itemize}
\item
  Regressão logística é o mais estável e menos \emph{data hungry}.\textsuperscript{\citeproc{ref-vanderploeg2014}{317}}
\item
  Árvore de decisão para classificação e regressão estabiliza rápido, mas em nível de desempenho baixo.\textsuperscript{\citeproc{ref-vanderploeg2014}{317}}
\item
  Máquina de vetores de suporte, redes neurais e \emph{random forests} apresentam instabilidade mesmo em amostras muito grandes.\textsuperscript{\citeproc{ref-vanderploeg2014}{317}}
\end{itemize}

\section{Comparação de modelos}\label{comparauxe7uxe3o-de-modelos}

\subsection{Como comparar modelos estatísticos?}\label{como-comparar-modelos-estatuxedsticos}

\begin{itemize}
\tightlist
\item
  .\textsuperscript{\citeproc{ref-REF}{\textbf{REF?}}}
\end{itemize}

\subsection{Como comparar modelos de aprendizagem de máquina?}\label{como-comparar-modelos-de-aprendizagem-de-muxe1quina}

\begin{itemize}
\tightlist
\item
  .\textsuperscript{\citeproc{ref-REF}{\textbf{REF?}}}
\end{itemize}

\begin{infobox}{images/Rlogo}
O pacote \emph{correctR}\textsuperscript{\citeproc{ref-correctR}{326}} fornece funções para comparar o desempenho e a qualidade do ajuste de diversos modelos de aprendizagem de máquina em amostras correlacionadas.

\end{infobox}

\section{Avaliação de modelos}\label{avaliauxe7uxe3o-de-modelos}

\subsection{Como avaliar a qualidade de ajuste de um modelo?}\label{como-avaliar-a-qualidade-de-ajuste-de-um-modelo}

\begin{itemize}
\tightlist
\item
  Coeficiente de determinação (\(R^2\)) \eqref{eq:r2} e \(R^2\) ajustado \eqref{eq:r2adj}: Medem a proporção da variabilidade dos dados explicada pelo modelo. O \(R^2\) ajustado penaliza a inclusão de variáveis irrelevantes.\textsuperscript{\citeproc{ref-REF}{\textbf{REF?}}}
\end{itemize}

\begin{equation}
\label{eq:r2}
R^2 = 1 - \frac{SS_{res}}{SS_{tot}}
\end{equation}

\begin{equation}
\label{eq:r2adj}
R^2_{ajustado} = 1 - (1 - R^2)\frac{n - 1}{n - p - 1}
\end{equation}

\begin{figure}

{\centering \includegraphics[width=1\linewidth]{Ciencia-com-R_files/figure-latex/r2-ajuste-1} 

}

\caption{Exemplos de ajuste de modelos de regressão linear simples (y ~ x) com diferentes níveis de ruído (R²). Cada painel mostra a reta ajustada (cinza) e os valores observados (pontos). Os valores anotados indicam o coeficiente angular simulado (β), o coeficiente angular estimado (β̂) e o R² observado.}\label{fig:r2-ajuste}
\end{figure}

\begin{itemize}
\tightlist
\item
  Erro quadrático médio (\(RMSE\)) \eqref{eq:rmse}: Mede a média dos erros ao quadrado entre os valores observados e os valores previstos pelo modelo, onde \(y_i\) são os valores observados, \(\hat{y}_i\) são os valores previstos pelo modelo, e \(n\) é o número de observações. Valores menores indicam melhor ajuste.\textsuperscript{\citeproc{ref-REF}{\textbf{REF?}}}
\end{itemize}

\begin{equation}
\label{eq:rmse}
RMSE = \sqrt{\frac{1}{n} \sum_{i=1}^{n} (y_i - \hat{y}_i)^2}
\end{equation}

\begin{itemize}
\tightlist
\item
  Critério de Informação Akaike (\(AIC\)) \eqref{eq:aic} e Critério de Informação Bayesiano (\(BIC\)) \eqref{eq:bic}: Avaliam o ajuste do modelo penalizando a complexidade (número de parâmetros), onde \(k\) é o número de parâmetros do modelo, \(L\) é a verossimilhança máxima do modelo, e \(n\) é o tamanho da amostra. Modelos com menor AIC ou BIC são preferíveis.\textsuperscript{\citeproc{ref-REF}{\textbf{REF?}}}
\end{itemize}

\begin{equation}
\label{eq:aic}
AIC = 2k - 2\ln(L)
\end{equation}

\begin{equation}
\label{eq:bic}
BIC = \ln(n)k - 2\ln(L)
\end{equation}

\begin{itemize}
\tightlist
\item
  Desvio residual (\(\sigma\)): Mede a variabilidade dos resíduos do modelo. Valores menores indicam melhor ajuste.\textsuperscript{\citeproc{ref-REF}{\textbf{REF?}}}
\end{itemize}

\begin{table}
\centering
\caption{\label{tab:metricas-ajuste}Métricas de desempenho do modelo de regressão linear.}
\centering
\begin{tabu} to \linewidth {>{}c>{\centering}X}
\toprule
\textbf{Métrica} & \textbf{Valor}\\
\midrule
\textbf{AIC} & 513.017\\
\textbf{AIC corrigido} & 513.267\\
\textbf{BIC} & 520.833\\
\textbf{$R^2$} & 0.007\\
\textbf{$R^2$ ajustado} & -0.003\\
\textbf{Erro quadrático médio (RMSE)} & 3.053\\
\textbf{Desvio residual (sigma)} & 3.084\\
\bottomrule
\end{tabu}
\end{table}

\begin{infobox}{images/Rlogo}
O pacote \emph{performance}\textsuperscript{\citeproc{ref-performance}{249}} fornece a função \href{https://www.rdocumentation.org/packages/performance/versions/0.10.4/topics/model_performance}{\emph{model\_performance}} para calcular as métricas de ajuste da regressão adequadas ao modelo pré-especificado.

\end{infobox}

\begin{infobox}{images/Rlogo}
O pacote \emph{performance}\textsuperscript{\citeproc{ref-performance}{249}} fornece a função \href{https://www.rdocumentation.org/packages/performance/versions/0.10.4/topics/compare_performance}{\emph{compare\_performance}} para comparar o desempenho e a qualidade do ajuste de diversos modelos de regressão pré-especificados.

\end{infobox}

\section{Validação de modelos}\label{validauxe7uxe3o-de-modelos}

\subsection{Como validar modelos estatísticos?}\label{como-validar-modelos-estatuxedsticos}

\begin{itemize}
\tightlist
\item
  .\textsuperscript{\citeproc{ref-REF}{\textbf{REF?}}}
\end{itemize}

\section{Calibração de modelos}\label{calibrauxe7uxe3o-de-modelos}

\subsection{Como calibrar modelos estatísticos?}\label{como-calibrar-modelos-estatuxedsticos}

\begin{itemize}
\tightlist
\item
  .\textsuperscript{\citeproc{ref-REF}{\textbf{REF?}}}
\end{itemize}

\chapter{\texorpdfstring{\textbf{Modelagem temporal}}{Modelagem temporal}}\label{modelagem-temporal}

\section{Modelos temporais}\label{modelos-temporais}

\subsection{O que são modelos temporais?}\label{o-que-suxe3o-modelos-temporais}

\begin{itemize}
\tightlist
\item
  .\textsuperscript{\citeproc{ref-REF}{\textbf{REF?}}}
\end{itemize}

\begin{infobox}{images/Rlogo}
O pacote \emph{UComp}\textsuperscript{\citeproc{ref-UComp}{327}} fornece a função \href{https://www.rdocumentation.org/packages/UComp/versions/5.1.5/topics/UCsetup}{\emph{UCsetup}} para preparar modelos temporais gerais univariados.

\end{infobox}

\chapter{\texorpdfstring{\textbf{Modelagem espacial}}{Modelagem espacial}}\label{modelagem-espacial}

\section{Modelos espaciais}\label{modelos-espaciais}

\subsection{O que são modelos espaciais?}\label{o-que-suxe3o-modelos-espaciais}

\begin{itemize}
\tightlist
\item
  .\textsuperscript{\citeproc{ref-REF}{\textbf{REF?}}}
\end{itemize}

\begin{infobox}{images/Rlogo}
Os pacotes \emph{sf}\textsuperscript{\citeproc{ref-sf}{328}} e \emph{ggplot}\textsuperscript{\citeproc{ref-ggplot2}{173}} fornecem a função \href{https://www.rdocumentation.org/packages/ggplot2/versions/3.4.3/topics/CoordSf}{\emph{geom\_sf}} para visualização de dados espaciais em R.

\end{infobox}

\begin{infobox}{images/Rlogo}
O pacote \emph{leaflet}{[}\textsuperscript{\citeproc{ref-leaflet}{329}}{]} fornece a função \href{https://www.rdocumentation.org/packages/leaflet/versions/2.2.2/topics/leaflet}{\emph{leaflet}} para criar um mapa interativo.

\end{infobox}

\chapter{\texorpdfstring{\textbf{Modelagem de sobrevida}}{Modelagem de sobrevida}}\label{modelagem-sobrevida}

\section{Sobrevida}\label{sobrevida}

\subsection{O que é sobrevida?}\label{o-que-uxe9-sobrevida}

\begin{itemize}
\tightlist
\item
  A sobrevida é um termo utilizado em estatística e análise de dados para descrever o tempo que decorre até a ocorrência de um evento específico, como a morte, a falha de um equipamento ou a recidiva de uma doença. Em estudos clínicos, por exemplo, a sobrevida pode referir-se ao tempo que um paciente vive após o diagnóstico de uma doença ou após o início de um tratamento.\textsuperscript{\citeproc{ref-REF}{\textbf{REF?}}}
\end{itemize}

\section{Análise de sobrevida}\label{anuxe1lise-de-sobrevida}

\subsection{O que é análise de sobrevida?}\label{o-que-uxe9-anuxe1lise-de-sobrevida}

\begin{itemize}
\item
  A análise de sobrevida é uma área da estatística que se concentra no estudo desses tempos até o evento, levando em consideração que nem todos os indivíduos podem ter experimentado o evento durante o período de estudo (censura).\textsuperscript{\citeproc{ref-REF}{\textbf{REF?}}}
\item
  Métodos comuns de análise de sobrevida incluem a estimativa da função de sobrevivência, a análise de Kaplan-Meier e modelos de regressão como o modelo de riscos proporcionais de Cox.\textsuperscript{\citeproc{ref-REF}{\textbf{REF?}}}
\end{itemize}

\begin{figure}

{\centering \includegraphics{Ciencia-com-R_files/figure-latex/kaplan-meier-1} 

}

\caption{Curvas de Kaplan–Meier simuladas para dois grupos (controle e tratamento).}\label{fig:kaplan-meier}
\end{figure}

\begin{infobox}{images/Rlogo}
O pacote \emph{survival}\textsuperscript{\citeproc{ref-survival}{330}} fornece a função {[}\emph{survfit}{]}\url{https://www.rdocumentation.org/packages/survival/versions/3.8-3/topics/survfit}) para criar curvas de sobrevida.

\end{infobox}

\cftaddtitleline{toc}{chapter}{\rule{\textwidth}{0.4pt}}{}

\chapter*{\texorpdfstring{\emph{PARTE 7: REPRESENTAÇÃO, APRENDIZADO DE MÁQUINA E INTELIGÊNCIA ARTIFICIAL}}{PARTE 7: REPRESENTAÇÃO, APRENDIZADO DE MÁQUINA E INTELIGÊNCIA ARTIFICIAL}}\label{parte-7}
\addcontentsline{toc}{chapter}{\emph{PARTE 7: REPRESENTAÇÃO, APRENDIZADO DE MÁQUINA E INTELIGÊNCIA ARTIFICIAL}}

\par\noindent\rule{\textwidth}{0.05in}

\section*{Do avanço estatístico ao poder computacional: Métodos modernos para problemas complexos}\label{do-avanuxe7o-estatuxedstico-ao-poder-computacional-muxe9todos-modernos-para-problemas-complexos}

\markboth{}{}

\chapter{\texorpdfstring{\textbf{Representações}}{Representações}}\label{representacoes}

\section{Representações de dados e extração de atributos}\label{representauxe7uxf5es-de-dados-e-extrauxe7uxe3o-de-atributos}

\subsection{Por que modelos não podem acessar o fenômeno diretamente?}\label{por-que-modelos-nuxe3o-podem-acessar-o-fenuxf4meno-diretamente}

\begin{itemize}
\tightlist
\item
  .\textsuperscript{\citeproc{ref-REF}{\textbf{REF?}}}
\end{itemize}

\subsection{Como um fenômeno do mundo real é traduzido em uma estrutura matemática?}\label{como-um-fenuxf4meno-do-mundo-real-uxe9-traduzido-em-uma-estrutura-matemuxe1tica}

\begin{itemize}
\tightlist
\item
  .\textsuperscript{\citeproc{ref-REF}{\textbf{REF?}}}
\end{itemize}

\subsection{De que forma a representação limita ou expande o que um modelo pode aprender?}\label{de-que-forma-a-representauxe7uxe3o-limita-ou-expande-o-que-um-modelo-pode-aprender}

\begin{itemize}
\tightlist
\item
  .\textsuperscript{\citeproc{ref-REF}{\textbf{REF?}}}
\end{itemize}

\subsection{O que é um atributo e por que ela não é um algoritmo?}\label{o-que-uxe9-um-atributo-e-por-que-ela-nuxe3o-uxe9-um-algoritmo}

\begin{itemize}
\tightlist
\item
  .\textsuperscript{\citeproc{ref-REF}{\textbf{REF?}}}
\end{itemize}

\subsection{Quais são as principais formas de representar diferentes tipos de dados?}\label{quais-suxe3o-as-principais-formas-de-representar-diferentes-tipos-de-dados}

\begin{itemize}
\tightlist
\item
  .\textsuperscript{\citeproc{ref-REF}{\textbf{REF?}}}
\end{itemize}

\subsection{O que se perde --- e o que se ganha --- ao escolher uma representação?}\label{o-que-se-perde-e-o-que-se-ganha-ao-escolher-uma-representauxe7uxe3o}

\begin{itemize}
\tightlist
\item
  .\textsuperscript{\citeproc{ref-REF}{\textbf{REF?}}}
\end{itemize}

\subsection{Como essas escolhas antecipam os desafios da inteligência artificial e do aprendizado de máquina?}\label{como-essas-escolhas-antecipam-os-desafios-da-inteliguxeancia-artificial-e-do-aprendizado-de-muxe1quina}

\begin{itemize}
\tightlist
\item
  .\textsuperscript{\citeproc{ref-REF}{\textbf{REF?}}}
\end{itemize}

\chapter{\texorpdfstring{\textbf{Inteligência artificial}}{Inteligência artificial}}\label{inteligencia-artificial}

\section{Inteligência artificial}\label{inteliguxeancia-artificial}

\subsection{O que é inteligência artificial (IA)?}\label{o-que-uxe9-inteliguxeancia-artificial-ia}

\begin{itemize}
\tightlist
\item
  .\textsuperscript{\citeproc{ref-REF}{\textbf{REF?}}}
\end{itemize}

\subsection{Como ela se relaciona com estatística, ciência de dados e aprendizado de máquina?}\label{como-ela-se-relaciona-com-estatuxedstica-ciuxeancia-de-dados-e-aprendizado-de-muxe1quina}

\begin{itemize}
\tightlist
\item
  .\textsuperscript{\citeproc{ref-REF}{\textbf{REF?}}}
\end{itemize}

\section{IA generativa e grandes modelos de linguagem}\label{ia-generativa-e-grandes-modelos-de-linguagem}

\subsection{\texorpdfstring{O que são grandes modelos de linguagem (\emph{large language models}, LLM)?}{O que são grandes modelos de linguagem (large language models, LLM)?}}\label{o-que-suxe3o-grandes-modelos-de-linguagem-large-language-models-llm}

\begin{itemize}
\tightlist
\item
  .\textsuperscript{\citeproc{ref-REF}{\textbf{REF?}}}
\end{itemize}

\subsection{Como funcionam modelos como GPT, BERT e similares?}\label{como-funcionam-modelos-como-gpt-bert-e-similares}

\begin{itemize}
\tightlist
\item
  .\textsuperscript{\citeproc{ref-REF}{\textbf{REF?}}}
\end{itemize}

\begin{figure}

{\centering \includegraphics{Ciencia-com-R_files/figure-latex/llm-1} 

}

\caption{Representação esquemática de um modelo de linguagem grande (LLM)}\label{fig:llm}
\end{figure}

\begin{infobox}{images/Rlogo}
O pacote \emph{keras}\href{https://cran.r-project.org/web/packages/keras/index.html}{@keras} possuu funções para criar, treinar e avaliar modelos de redes neurais.

\end{infobox}

\begin{infobox}{images/Rlogo}
O pacote \emph{tensorflow}\href{https://cran.r-project.org/web/packages/tensorflow/index.html}{\textsuperscript{\citeproc{ref-tensorflow}{331}}} fornece uma interface para o TensorFlow, uma biblioteca de código aberto amplamente utilizada para aprendizado de máquina e redes neurais.

\end{infobox}

\begin{infobox}{images/Rlogo}
O pacote \emph{torch}\href{https://cran.r-project.org/web/packages/torch/index.html}{\textsuperscript{\citeproc{ref-torch}{332}}} permite criar e treinar redes neurais com alto desempenho.

\end{infobox}

\begin{infobox}{images/Rlogo}
O pacote \emph{reticulate}\href{https://cran.r-project.org/web/packages/reticulate/index.html}{@reticulate} integra R e Python em um mesmo ambiente de trabalho, permitindo chamar funções Python a partir de R e facilitar o uso de bibliotecas de IA disponíveis nesse ecossistema.

\end{infobox}

\chapter{\texorpdfstring{\textbf{Aprendizado de máquina}}{Aprendizado de máquina}}\label{aprendizado-maquina}

\section{Aprendizado de máquina}\label{aprendizado-de-muxe1quina}

\subsection{O que é aprendizado de máquina?}\label{o-que-uxe9-aprendizado-de-muxe1quina}

\begin{itemize}
\tightlist
\item
  .\textsuperscript{\citeproc{ref-REF}{\textbf{REF?}}}
\end{itemize}

\begin{figure}

{\centering \includegraphics[width=1\linewidth]{Ciencia-com-R_files/figure-latex/aprendizado-maquina} 

}

\caption{Mapa mental de algoritmos de aprendizado de máquina.}\label{fig:aprendizado-maquina}
\end{figure}

\begin{infobox}{images/Rlogo}
O pacote \emph{fastml}\textsuperscript{\citeproc{ref-fastml}{333}} fornece a função \href{https://cran.r-project.org/web/packages/fastml/fastml.pdf}{\emph{train\_models}} para treinar algoritmos de aprendizado de máquina em dados de treinamento pré-processados.

\end{infobox}

\section{Tipos de aprendizado}\label{tipos-de-aprendizado}

\subsection{O que é aprendizado supervisionado?}\label{o-que-uxe9-aprendizado-supervisionado}

\begin{itemize}
\tightlist
\item
  .\textsuperscript{\citeproc{ref-REF}{\textbf{REF?}}}
\end{itemize}

\subsection{O que é aprendizado não supervisionado?}\label{o-que-uxe9-aprendizado-nuxe3o-supervisionado}

\begin{itemize}
\tightlist
\item
  .\textsuperscript{\citeproc{ref-REF}{\textbf{REF?}}}
\end{itemize}

\subsection{O que é aprendizado semi-supervisionado?}\label{o-que-uxe9-aprendizado-semi-supervisionado}

\begin{itemize}
\tightlist
\item
  .\textsuperscript{\citeproc{ref-REF}{\textbf{REF?}}}
\end{itemize}

\subsection{O que é aprendizado por reforço?}\label{o-que-uxe9-aprendizado-por-reforuxe7o}

\begin{itemize}
\tightlist
\item
  .\textsuperscript{\citeproc{ref-REF}{\textbf{REF?}}}
\end{itemize}

\subsection{O que é aprendizado profundo?}\label{o-que-uxe9-aprendizado-profundo}

\begin{itemize}
\tightlist
\item
  .\textsuperscript{\citeproc{ref-REF}{\textbf{REF?}}}
\end{itemize}

\subsection{Quais são os limites do progresso em classificadores supervisionados?}\label{quais-suxe3o-os-limites-do-progresso-em-classificadores-supervisionados}

\begin{itemize}
\item
  Os maiores ganhos de acurácia vêm de modelos simples, como análise discriminante linear; métodos mais sofisticados oferecem apenas ganhos marginais.\textsuperscript{\citeproc{ref-hand2006}{334}}
\item
  O aumento da complexidade do modelo traz retornos decrescentes em termos de redução da taxa de erro.\textsuperscript{\citeproc{ref-hand2006}{334}}
\end{itemize}

\subsection{Quais problemas práticos limitam a generalização de classificadores?}\label{quais-problemas-pruxe1ticos-limitam-a-generalizauxe7uxe3o-de-classificadores}

\begin{itemize}
\item
  \emph{Population drift}: mudanças na distribuição dos dados ao longo do tempo degradam a performance de modelos.\textsuperscript{\citeproc{ref-hand2006}{334}}
\item
  \emph{Sample selectivity bias}: amostras de treino podem não representar a população futura, levando a superestimação de desempenho.\textsuperscript{\citeproc{ref-hand2006}{334}}
\item
  Erros de rótulo e definições arbitrárias de classes comprometem a validade dos modelos.\textsuperscript{\citeproc{ref-hand2006}{334}}
\end{itemize}

\subsection{Por que estudos comparativos entre classificadores podem ser enganosos?}\label{por-que-estudos-comparativos-entre-classificadores-podem-ser-enganosos}

\begin{itemize}
\item
  Resultados dependem da experiência do pesquisador com cada método, da escolha dos conjuntos de dados e do critério de avaliação usado.\textsuperscript{\citeproc{ref-hand2006}{334}}
\item
  Diferenças pequenas em acurácia frequentemente desaparecem quando se consideram incertezas reais de aplicação.\textsuperscript{\citeproc{ref-hand2006}{334}}
\end{itemize}

\section{Principais algoritmos}\label{principais-algoritmos}

\subsection{Quais são os principais algoritmos de aprendizado de máquina?}\label{quais-suxe3o-os-principais-algoritmos-de-aprendizado-de-muxe1quina}

\begin{itemize}
\item
  Modelos de regressão não penalizados, modelos de regressão penalizados, modelos baseados em árvores, modelos baseados em vizinhos, redes neurais, máquinas de vetores de suporte, Naive Bayes e ensembles do tipo Superlearner.\textsuperscript{\citeproc{ref-andaurnavarro2023}{335}}
\item
  Do ponto de vista matemático, redes neurais não contradizem a estatística clássica; elas a estendem, substituindo modelos explícitos por representações aprendidas.\textsuperscript{\citeproc{ref-REF}{\textbf{REF?}}}
\end{itemize}

\begin{tabu} to \linewidth {>{}l>{\raggedright}X>{\raggedright}X}
\toprule
\textbf{Modelos de regressão} & \textbf{Redes neurais artificiais} & \textbf{Papel conceitual}\\
\midrule
\textbf{Variável preditora (x)} & Neurônio de entrada & Informação observada fornecida ao modelo\\
\textbf{Coeficiente (β)} & Peso (w) & Intensidade e direção da influência da variável\\
\textbf{Intercepto (β₀)} & Viés (b) & Deslocamento da fronteira de decisão\\
\textbf{Combinação linear (β₀ + Σ βᵢxᵢ)} & Soma ponderada (Σ wᵢxᵢ + b) & Agregação das entradas antes da não linearidade\\
\textbf{Função de ligação (link)} & Função de ativação & Introdução de não linearidade\\
\textbf{Regressão linear} & Neurônio linear & Modelo puramente linear\\
\textbf{Regressão logística} & Perceptron com ativação sigmoide & Classificação binária probabilística\\
\textbf{Log-odds} & Entrada da função sigmoide & Escala interna antes da probabilidade\\
\textbf{Predição (ŷ)} & Saída do neurônio & Resposta estimada do modelo\\
\textbf{Função de perda} & Função de perda (loss) & Quantificação do erro de predição\\
\textbf{Máxima verossimilhança} & Otimização da função de perda & Ajuste dos parâmetros do modelo\\
\textbf{Gradiente da verossimilhança} & Retropropagação (backpropagation) & Direção de atualização dos parâmetros\\
\textbf{Regularização (L1, L2)} & Penalização de pesos (weight decay) & Controle de complexidade e overfitting\\
\textbf{Interações explícitas} & Interações aprendidas implicitamente & Modelagem de efeitos combinados\\
\textbf{Modelo interpretável} & Modelo geralmente opaco & Trade-off entre interpretação e flexibilidade\\
\bottomrule
\end{tabu}

\section{Regressão logística}\label{regressuxe3o-loguxedstica}

\subsection{O que são é regressão logística?}\label{o-que-suxe3o-uxe9-regressuxe3o-loguxedstica}

\begin{itemize}
\tightlist
\item
  .\textsuperscript{\citeproc{ref-REF}{\textbf{REF?}}}
\end{itemize}

\section{Máquina de vetores de suporte}\label{muxe1quina-de-vetores-de-suporte}

\subsection{O que são máquinas de vetores de suporte?}\label{o-que-suxe3o-muxe1quinas-de-vetores-de-suporte}

\begin{itemize}
\tightlist
\item
  .\textsuperscript{\citeproc{ref-REF}{\textbf{REF?}}}
\end{itemize}

\section{\texorpdfstring{\emph{K-nearest neighbours}}{K-nearest neighbours}}\label{k-nearest-neighbours}

\subsection{\texorpdfstring{O que é \emph{K-nearest neighbours}?}{O que é K-nearest neighbours?}}\label{o-que-uxe9-k-nearest-neighbours}

\begin{itemize}
\tightlist
\item
  .\textsuperscript{\citeproc{ref-REF}{\textbf{REF?}}}
\end{itemize}

\section{\texorpdfstring{\emph{K-means Clustering}}{K-means Clustering}}\label{k-means-clustering}

\subsection{\texorpdfstring{O que é \emph{K-means clustering}?}{O que é K-means clustering?}}\label{o-que-uxe9-k-means-clustering}

\begin{itemize}
\tightlist
\item
  .\textsuperscript{\citeproc{ref-REF}{\textbf{REF?}}}
\end{itemize}

\section{Árvores de decisão}\label{uxe1rvores-de-decisuxe3o}

\subsection{O que são árvores de decisão?}\label{o-que-suxe3o-uxe1rvores-de-decisuxe3o}

\begin{itemize}
\item
  São modelos de aprendizado supervisionado que dividem os dados em ramos e folhas, representando regras de decisão de forma hierárquica.\textsuperscript{\citeproc{ref-hozo2023}{239}}
\item
  Podem lidar eficientemente com grandes conjuntos de dados sem pressupor estrutura paramétrica complexa.\textsuperscript{\citeproc{ref-Song2015}{238}}
\item
  São aplicáveis a variáveis contínuas e discretas, tanto como preditoras quanto como desfechos.\textsuperscript{\citeproc{ref-Song2015}{238}}
\end{itemize}

\begin{figure}

{\centering \includegraphics{Ciencia-com-R_files/figure-latex/arvore-decisao-1} 

}

\caption{Exemplo de árvore de decisão para predizer depressão a partir de idade, tabagismo e sintomas.}\label{fig:arvore-decisao}
\end{figure}

\subsection{Quais são os principais usos de árvores de decisão?}\label{quais-suxe3o-os-principais-usos-de-uxe1rvores-de-decisuxe3o}

\begin{itemize}
\item
  Seleção de variáveis relevantes em cenários com muitos preditores, como registros clínicos eletrônicos.\textsuperscript{\citeproc{ref-Song2015}{238}}
\item
  Avaliação da importância relativa das variáveis, com base na redução da pureza dos nós ou da acurácia ao remover variáveis.\textsuperscript{\citeproc{ref-Song2015}{238}}
\item
  Tratamento de valores ausentes, seja classificando-os como categoria própria ou imputando-os por previsão dentro da árvore.\textsuperscript{\citeproc{ref-Song2015}{238}}
\item
  Predição de novos casos a partir de dados históricos.\textsuperscript{\citeproc{ref-Song2015}{238}}
\item
  Manipulação de dados, colapsando categorias muito numerosas ou subdividindo variáveis contínuas assimétricas.\textsuperscript{\citeproc{ref-Song2015}{238}}
\end{itemize}

\subsection{Quais são os componentes básicos de uma árvore de decisão?}\label{quais-suxe3o-os-componentes-buxe1sicos-de-uma-uxe1rvore-de-decisuxe3o}

\begin{itemize}
\item
  Nós raiz (ou de decisão): subdividem todos os registros iniciais.\textsuperscript{\citeproc{ref-Song2015}{238}}
\item
  Nós internos (ou de chance): representam subdivisões intermediárias.\textsuperscript{\citeproc{ref-Song2015}{238}}
\item
  Nós folha (ou finais): resultados finais após sucessivas divisões.\textsuperscript{\citeproc{ref-Song2015}{238}}
\item
  Ramos: representam condições ``se-então'', ligando nós em sequência até a classificação final.\textsuperscript{\citeproc{ref-Song2015}{238}}
\end{itemize}

\subsection{Como funcionam splitting, stopping e pruning?}\label{como-funcionam-splitting-stopping-e-pruning}

\begin{itemize}
\item
  \emph{Splitting}: divide registros em subconjuntos mais homogêneos com base em métricas como entropia, índice de Gini e ganho de informação.\textsuperscript{\citeproc{ref-Song2015}{238}}
\item
  \emph{Stopping}: evita árvores excessivamente complexas ao definir parâmetros como número mínimo de registros por nó ou profundidade máxima.\textsuperscript{\citeproc{ref-Song2015}{238}}
\item
  \emph{Pruning}: reduz árvores grandes eliminando ramos pouco informativos, usando validação ou métodos como qui-quadrado.\textsuperscript{\citeproc{ref-Song2015}{238}}
\end{itemize}

\subsection{Quais são as vantagens e limitações de árvores de decisão?}\label{quais-suxe3o-as-vantagens-e-limitauxe7uxf5es-de-uxe1rvores-de-decisuxe3o}

\begin{itemize}
\item
  Vantagens: simplificam relações complexas; são intuitivas e fáceis de interpretar; não exigem pressupostos de distribuição; lidam bem com valores ausentes e dados enviesados; são robustas a \emph{outliers}.\textsuperscript{\citeproc{ref-Song2015}{238}}
\item
  Limitações: podem sofrer \emph{overfitting} ou \emph{underfitting} em amostras pequenas; podem selecionar variáveis correlacionadas sem relação causal real.\textsuperscript{\citeproc{ref-Song2015}{238}}
\end{itemize}

\subsection{Espaço de decisão em árvores de decisão vs.~regressão logística}\label{espauxe7o-de-decisuxe3o-em-uxe1rvores-de-decisuxe3o-vs.-regressuxe3o-loguxedstica}

\begin{itemize}
\item
  A regressão logística assume relações lineares entre variáveis e log-odds.\textsuperscript{\citeproc{ref-hozo2023}{239}}
\item
  Árvores de decisão permitem capturar relações não lineares e interações de forma automática.\textsuperscript{\citeproc{ref-hozo2023}{239}}
\end{itemize}

\begin{figure}

{\centering \includegraphics{Ciencia-com-R_files/figure-latex/logistica-vs-arvore-1} 

}

\caption{Comparação entre modelos de regressão logística e árvore de decisão.}\label{fig:logistica-vs-arvore}
\end{figure}

\begin{infobox}{images/Rlogo}
O pacote \emph{h2o}\href{https://cran.r-project.org/web/packages/h2o/index.html}{@correctR} fornece funções construir modelos de aprendizado de máquina.

\end{infobox}

\begin{infobox}{images/Rlogo}
O pacote \emph{correctR}\textsuperscript{\citeproc{ref-correctR}{326}} fornece as funções \href{https://cloud.r-project.org/web/packages/correctR/correctR.pdf}{\emph{kfold\_ttest}}, \href{https://cloud.r-project.org/web/packages/correctR/correctR.pdf}{\emph{repkfold\_ttest}} e \href{https://cloud.r-project.org/web/packages/correctR/correctR.pdf}{\emph{resampled\_ttest}} para calcular estatística para comparação de modelos de aprendizado de máquina em amostras dependentes.

\end{infobox}

\begin{infobox}{images/Rlogo}
O pacote \emph{caret}\href{https://cran.r-project.org/web/packages/caret/index.html}{@caret} fornece um conjunto de funções para pré-processamento, ajuste, avaliação e comparação de modelos de aprendizado de máquina.

\end{infobox}

\begin{infobox}{images/Rlogo}
O pacote \emph{mlr3}\href{https://cran.r-project.org/web/packages/mlr3/index.html}{@mlr3} fornece funções para fluxos de trabalho complexos, incluindo pré-processamento, ajuste de hiperparâmetros e integração com diversos algoritmos.

\end{infobox}

\section{Análise de componentes principais}\label{anuxe1lise-de-componentes-principais}

\subsection{O que é análise de componentes principais?}\label{o-que-uxe9-anuxe1lise-de-componentes-principais}

\begin{itemize}
\tightlist
\item
  .\textsuperscript{\citeproc{ref-REF}{\textbf{REF?}}}
\end{itemize}

\section{\texorpdfstring{\emph{Random forests}}{Random forests}}\label{random-forests}

\subsection{\texorpdfstring{O que são \emph{random forests}?}{O que são random forests?}}\label{o-que-suxe3o-random-forests}

\begin{itemize}
\tightlist
\item
  .\textsuperscript{\citeproc{ref-REF}{\textbf{REF?}}}
\end{itemize}

\section{\texorpdfstring{\emph{Ensemble}}{Ensemble}}\label{ensemble}

\subsection{\texorpdfstring{O que são \emph{ensemble}?}{O que são ensemble?}}\label{o-que-suxe3o-ensemble}

\begin{itemize}
\tightlist
\item
  .\textsuperscript{\citeproc{ref-REF}{\textbf{REF?}}}
\end{itemize}

\section{Desbalanceamento de classes}\label{desbalanceamento-de-classes}

\subsection{\texorpdfstring{O que é desbalanceamento de classes (\emph{class imbalance})?}{O que é desbalanceamento de classes (class imbalance)?}}\label{o-que-uxe9-desbalanceamento-de-classes-class-imbalance}

\begin{itemize}
\tightlist
\item
  Ocorre quando as classes do desfecho (por exemplo, presença vs.~ausência de um evento) não estão igualmente representadas nos dados de treinamento.\textsuperscript{\citeproc{ref-REF}{\textbf{REF?}}}
\end{itemize}

\subsection{Por que o desbalanceamento é um problema?}\label{por-que-o-desbalanceamento-uxe9-um-problema}

\begin{itemize}
\item
  Modelos podem aprender a priorizar a classe mais frequente, obtendo alta acurácia global, mas baixo desempenho para a classe minoritária.\textsuperscript{\citeproc{ref-REF}{\textbf{REF?}}}
\item
  Isso pode comprometer métricas como sensibilidade, especificidade e, em alguns casos, a calibração.\textsuperscript{\citeproc{ref-REF}{\textbf{REF?}}}
\end{itemize}

\subsection{Quais são as abordagens mais comuns para lidar com desbalanceamento de classes?}\label{quais-suxe3o-as-abordagens-mais-comuns-para-lidar-com-desbalanceamento-de-classes}

\begin{itemize}
\item
  Reamostragem aleatória: superamostragem da classe minoritária; subamostragem da classe majoritária).\textsuperscript{\citeproc{ref-REF}{\textbf{REF?}}}
\item
  Ajuste de pesos: penaliza mais os erros na classe menos frequente.\textsuperscript{\citeproc{ref-REF}{\textbf{REF?}}}
\item
  Alteração do limiar de decisão: muda o ponto de corte de probabilidade para otimizar métricas específicas.\textsuperscript{\citeproc{ref-REF}{\textbf{REF?}}}
\end{itemize}

\subsection{Qual é o impacto do desbalanceamento de classes na calibração de modelos?}\label{qual-uxe9-o-impacto-do-desbalanceamento-de-classes-na-calibrauxe7uxe3o-de-modelos}

\begin{itemize}
\item
  Corrigir o desbalanceamento de classes nem sempre melhora a calibração e, em alguns casos, pode piorá-la.\textsuperscript{\citeproc{ref-carriero2025}{336}}
\item
  Em simulações computacionais, modelos sem correção tiveram calibração igual ou superior aos corrigidos.\textsuperscript{\citeproc{ref-carriero2025}{336}}
\item
  A piora observada foi caracterizada por superestimação do risco, nem sempre reversível com re-calibração.\textsuperscript{\citeproc{ref-carriero2025}{336}}
\end{itemize}

\chapter{\texorpdfstring{\textbf{Redes neurais}}{Redes neurais}}\label{redes-meurais}

\section{Neurônios artificiais}\label{neuruxf4nios-artificiais}

\subsection{O que são neurônios artificiais?}\label{o-que-suxe3o-neuruxf4nios-artificiais}

\begin{itemize}
\item
  Neuronios artificiais (ou perceptrons) são modelos matemáticos que imitam o funcionamento dos neurônios biológicos, recebendo entradas, aplicando pesos e uma função de ativação para produzir uma saída.\textsuperscript{\citeproc{ref-mcculloch1943}{337}--\citeproc{ref-rosenblatt1960}{339}}
\item
  A equação geral de um neurônio artificial é dada por \eqref{eq:neuronio}, onde \(x_i\) são as entradas, \(w_i\) os pesos, \(b\) o viés e \(\phi\) a função de ativação:
\end{itemize}

\begin{equation}
\label{eq:neuronio}
y = \phi\left(\sum_{i=1}^{d} w_i\,x_i + b\right)
\end{equation}

\begin{figure}

{\centering \includegraphics{Ciencia-com-R_files/figure-latex/neuronio-artificial-1} 

}

\caption{Representação esquemática de um neurônio computacional.}\label{fig:neuronio-artificial}
\end{figure}

\section{Rede neural artificial}\label{rede-neural-artificial}

\subsection{O que é uma rede neural artificial?}\label{o-que-uxe9-uma-rede-neural-artificial}

\begin{itemize}
\tightlist
\item
  Redes neurais artificiais são modelos computacionais compostos por camadas de neurônios artificiais interconectados, nos quais cada camada aplica transformações lineares seguidas de funções não lineares, permitindo a aproximação de relações complexas entre variáveis de entrada e saída.\textsuperscript{\citeproc{ref-REF}{\textbf{REF?}}}
\end{itemize}

\begin{figure}

{\centering \includegraphics{Ciencia-com-R_files/figure-latex/rede-neural-1} 

}

\caption{Representação esquemática de uma rede neural simples com camada de entrada e saída.}\label{fig:rede-neural}
\end{figure}

\begin{infobox}{images/Rlogo}
O pacote \emph{neuralnet}\textsuperscript{\citeproc{ref-neuralnet}{340}} fornece a função \href{https://www.rdocumentation.org/packages/neuralnet/versions/1.44.2/topics/neuralnet}{\emph{neuralnet}} para treinar redes neurais artificiais.

\end{infobox}

\section{Funções de ativação}\label{funuxe7uxf5es-de-ativauxe7uxe3o}

\subsection{Quais são as funções de ativação mais comuns?}\label{quais-suxe3o-as-funuxe7uxf5es-de-ativauxe7uxe3o-mais-comuns}

\begin{itemize}
\tightlist
\item
  As funções de ativação introduzem não-linearidades nas redes neurais, permitindo que aprendam padrões complexos, como sigmoide \eqref{eq:sigmoide}, tangente hiperbólica \eqref{eq:tanh} e unidade linear retificada (ReLU) \eqref{eq:relu}.\textsuperscript{\citeproc{ref-REF}{\textbf{REF?}}}
\end{itemize}

\begin{equation}
\label{eq:sigmoide}
\sigma(z) = \frac{1}{1 + e^{-z}}
\end{equation}

\begin{equation}
\label{eq:tanh}
\tanh(z) = \frac{e^{z} - e^{-z}}{e^{z} + e^{-z}}
\end{equation}

\begin{equation}
\label{eq:relu}
\operatorname{ReLU}(z) = \max(0, z)
\end{equation}

\begin{itemize}
\item
  Ao manter gradientes constantes na região positiva, a ReLU favorece estabilidade numérica e eficiência computacional em redes multicamadas.{[}\textsuperscript{\citeproc{ref-REF}{\textbf{REF?}}}
\item
  Diferentemente das funções sigmoide e tangente hiperbólica, a ReLU preserva gradientes úteis em regiões amplas do espaço de entrada.\textsuperscript{\citeproc{ref-REF}{\textbf{REF?}}}
\item
  Sem funções de ativação não lineares, uma rede neural profunda se reduz a um modelo linear equivalente.\textsuperscript{\citeproc{ref-REF}{\textbf{REF?}}}
\end{itemize}

\begin{figure}

{\centering \includegraphics{Ciencia-com-R_files/figure-latex/funcoes-ativacao-1} 

}

\caption{Gráficos das funções de ativação mais comuns.}\label{fig:funcoes-ativacao}
\end{figure}

\section{Funções de perda}\label{funuxe7uxf5es-de-perda}

\subsection{O que são funções de perda?}\label{o-que-suxe3o-funuxe7uxf5es-de-perda}

\begin{itemize}
\item
  Funções de perda (\emph{loss functions}) quantificam o erro cometido por um modelo ao comparar suas predições com os valores reais observados.\textsuperscript{\citeproc{ref-REF}{\textbf{REF?}}}
\item
  Funções de perda definem formalmente o objetivo do aprendizado, indicando o que significa ``errar pouco'' ou ``errar muito'' em um problema específico.\textsuperscript{\citeproc{ref-REF}{\textbf{REF?}}}
\item
  Durante o treinamento de modelos supervisionados, a função de perda orienta o ajuste dos parâmetros ao medir a discrepância entre saída prevista e desfecho verdadeiro.\textsuperscript{\citeproc{ref-REF}{\textbf{REF?}}}
\item
  Em redes neurais, a minimização da função de perda é realizada por métodos iterativos baseados em gradientes, como a retropropagação do erro.\textsuperscript{\citeproc{ref-REF}{\textbf{REF?}}}
\item
  A escolha da função de perda está intimamente ligada à natureza do problema (regressão, classificação, probabilidade, ranking) e influencia diretamente o espaço de decisão aprendido pelo modelo.\textsuperscript{\citeproc{ref-REF}{\textbf{REF?}}}
\end{itemize}

\subsection{Quais são as funções de perda mais comuns?}\label{quais-suxe3o-as-funuxe7uxf5es-de-perda-mais-comuns}

\begin{itemize}
\tightlist
\item
  Erro quadrático médio (Mean Squared Error, MSE \eqref{eq:loss-mse}): Essa função penaliza erros grandes de forma mais severa, sendo adequada quando desvios elevados são indesejáveis e a média do erro quadrático é uma medida relevante de desempenho.\textsuperscript{\citeproc{ref-REF}{\textbf{REF?}}}
\end{itemize}

\begin{equation}
\label{eq:loss-mse}
\mathcal{L}_{\mathrm{MSE}}(y, \hat{y})
= \frac{1}{n}\sum_{i=1}^{n} (y_i - \hat{y}_i)^2
\end{equation}

\begin{itemize}
\tightlist
\item
  Erro absoluto médio (Mean Absolute Error, MAE\eqref{eq:loss-mae}): Essa função atribui peso linear aos erros, tornando-se mais robusta a valores extremos quando comparada ao erro quadrático médio.\textsuperscript{\citeproc{ref-REF}{\textbf{REF?}}}
\end{itemize}

\begin{equation}
\label{eq:loss-mae}
\mathcal{L}_{\mathrm{MAE}}(y, \hat{y})
= \frac{1}{n}\sum_{i=1}^{n} \lvert y_i - \hat{y}_i \rvert
\end{equation}

\begin{itemize}
\tightlist
\item
  Erro quadrático médio logarítmico (Mean Squared Logarithmic Error, MSLE \eqref{eq:loss-msle}): Essa função enfatiza erros relativos, sendo particularmente útil quando diferenças proporcionais são mais relevantes do que diferenças absolutas.\textsuperscript{\citeproc{ref-REF}{\textbf{REF?}}}
\end{itemize}

\begin{equation}
\label{eq:loss-msle}
\mathcal{L}_{\mathrm{MSLE}}(y, \hat{y})
= \frac{1}{n}\sum_{i=1}^{n}
\left(\log(1+y_i) - \log(1+\hat{y}_i)\right)^2
\end{equation}

\begin{itemize}
\tightlist
\item
  Entropia cruzada binária (Binary Cross-Entropy, BCE \eqref{eq:loss-bce}): Essa função mede a discrepância entre probabilidades previstas e observadas, sendo o critério padrão em problemas de classificação binária probabilística.
\end{itemize}

\begin{equation}
\label{eq:loss-bce}
\mathcal{L}_{\mathrm{BCE}}(y, \hat{y})
= -\frac{1}{n}\sum_{i=1}^{n}
\left[
y_i \log(\hat{y}_i) + (1-y_i)\log(1-\hat{y}_i)
\right]
\end{equation}

\begin{itemize}
\tightlist
\item
  Entropia cruzada categórica (Categorical Cross-Entropy \eqref{eq:loss-cce}): Essa função generaliza a entropia cruzada para múltiplas classes, penalizando previsões probabilísticas inconsistentes com a classe verdadeira.\textsuperscript{\citeproc{ref-REF}{\textbf{REF?}}}
\end{itemize}

\begin{equation}
\label{eq:loss-cce}
\mathcal{L}_{\mathrm{CCE}}(y, \hat{y})
= -\frac{1}{n}\sum_{i=1}^{n}
\sum_{k=1}^{K} y_{ik}\log(\hat{y}_{ik})
\end{equation}

\begin{itemize}
\tightlist
\item
  Função logística (log-verossimilhança negativa \eqref{eq:loss-logistic}): Essa função expressa o critério de máxima verossimilhança da regressão logística, conectando inferência estatística e aprendizado supervisionado.\textsuperscript{\citeproc{ref-REF}{\textbf{REF?}}}
\end{itemize}

\begin{equation}
\label{eq:loss-logistic}
\mathcal{L}_{\mathrm{log}}(y, \hat{p})
= -\sum_{i=1}^{n}
\left[
y_i \log(\hat{p}_i) + (1-y_i)\log(1-\hat{p}_i)
\right]
\end{equation}

\begin{itemize}
\tightlist
\item
  Função hinge (\emph{Support Vector Machines} \eqref{eq:loss-hinge}): Essa função busca maximizar a margem entre classes, penalizando classificações incorretas ou pouco confiantes em relação à fronteira de decisão.
\end{itemize}

\begin{equation}
\label{eq:loss-hinge}
\mathcal{L}_{\mathrm{hinge}}(y, f(x))
= \frac{1}{n}\sum_{i=1}^{n}
\max\left(0,\, 1 - y_i f(x_i)\right)
\end{equation}

\section{Espaço de decisão}\label{espauxe7o-de-decisuxe3o}

\subsection{O que é espaço de decisão?}\label{o-que-uxe9-espauxe7o-de-decisuxe3o}

\begin{itemize}
\tightlist
\item
  O espaço de decisão é a região do espaço de entrada onde o modelo classifica as entradas em diferentes categorias. Ele é definido pelas fronteiras de decisão aprendidas pelo modelo durante o treinamento.\textsuperscript{\citeproc{ref-REF}{\textbf{REF?}}}
\end{itemize}

\subsection{Como ele é visualizado?}\label{como-ele-uxe9-visualizado}

\begin{itemize}
\tightlist
\item
  O espaço de decisão pode ser visualizado graficamente, especialmente em problemas de classificação binária ou multiclasse, onde as regiões correspondem às classes previstas pelo modelo.\textsuperscript{\citeproc{ref-REF}{\textbf{REF?}}}
\end{itemize}

\begin{figure}

{\centering \includegraphics{Ciencia-com-R_files/figure-latex/perceptron-exemplo-1} 

}

\caption{Espaço de decisão de um perceptron (regressão logística).}\label{fig:perceptron-exemplo}
\end{figure}

\begin{figure}

{\centering \includegraphics{Ciencia-com-R_files/figure-latex/espaco-de-decisao-1} 

}

\caption{Comparação do espaço de decisão entre um modelo linear (regressão logística) e um modelo não linear (MLP).}\label{fig:espaco-de-decisao}
\end{figure}

\section{Redes neurais multicamadas}\label{redes-neurais-multicamadas}

\subsection{O que são redes neurais multicamadas?}\label{o-que-suxe3o-redes-neurais-multicamadas}

\begin{itemize}
\item
  Redes neurais multicamadas são redes neurais artificiais que contêm uma ou mais camadas escondidas entre a camada de entrada e a camada de saída, permitindo a composição sucessiva de transformações não lineares e, consequentemente, maior capacidade de representação de funções complexas.\textsuperscript{\citeproc{ref-REF}{\textbf{REF?}}}
\item
  Essa estrutura amplia o espaço de decisão do modelo sem exigir especificação explícita de interações ou transformações entre variáveis.\textsuperscript{\citeproc{ref-REF}{\textbf{REF?}}}
\item
  À medida que a profundidade da rede aumenta, passa-se do ajuste de parâmetros para o aprendizado de representações internas dos dados, caracterizando um regime distinto de modelagem conhecido como aprendizado profundo.\textsuperscript{\citeproc{ref-REF}{\textbf{REF?}}}
\end{itemize}

\begin{figure}

{\centering \includegraphics{Ciencia-com-R_files/figure-latex/rede-neural-multicamadas-1} 

}

\caption{Representação esquemática de uma rede neural multicamadas com 2 camadas escondidas além das camadas de entrada e saída.}\label{fig:rede-neural-multicamadas}
\end{figure}

\cftaddtitleline{toc}{chapter}{\rule{\textwidth}{0.4pt}}{}

\chapter*{\texorpdfstring{\emph{PARTE 8: PLANEJAMENTO DE ESTUDOS}}{PARTE 8: PLANEJAMENTO DE ESTUDOS}}\label{parte-8}
\addcontentsline{toc}{chapter}{\emph{PARTE 8: PLANEJAMENTO DE ESTUDOS}}

\par\noindent\rule{\textwidth}{0.05in}

\section*{Definindo poder, tamanho amostral e plano de análise}\label{definindo-poder-tamanho-amostral-e-plano-de-anuxe1lise}

\markboth{}{}

\chapter{\texorpdfstring{\textbf{Poder estatístico}}{Poder estatístico}}\label{poder-estatistico}

\section{Poder do teste}\label{poder-do-teste}

\subsection{O que é poder do teste?}\label{o-que-uxe9-poder-do-teste}

\begin{itemize}
\item
  Poder do teste é a probabilidade de rejeitar corretamente a hipótese nula (\(H_{0}\)) quando esta é falsa.\textsuperscript{\citeproc{ref-Curran-Everett2009}{261}}
\item
  Poder do teste pode ser calculado como \(1 - \beta\).\textsuperscript{\citeproc{ref-Curran-Everett2009}{261}}
\end{itemize}

\subsection{O que é análise de poder do teste?}\label{o-que-uxe9-anuxe1lise-de-poder-do-teste}

\begin{itemize}
\item
  Poder é a probabilidade de que um dado tamanho de efeito será observado em um experimento futuro sob um conjunto de hipóteses --- tamanho de efeito real e erro tipo I --- para um dado tamanho de amostra.\textsuperscript{\citeproc{ref-heckman2022}{341}}
\item
  O objetivo geral da análise de poder ao projetar um estudo é escolher um tamanho de amostra que controle os 2 tipos de erros de inferência estatística: tipo I (\(\alpha\), resultado falso-positivo) e tipo II (\(\beta\), resultado falso-negativo).\textsuperscript{\citeproc{ref-heckman2022}{341}}
\item
  Numericamente, o poder de um estudo é calculado como \(1-\beta\) e reportado em valor percentual.\textsuperscript{\citeproc{ref-heckman2022}{341}}
\end{itemize}

\subsection{Quando realizar a análise de poder do teste?}\label{quando-realizar-a-anuxe1lise-de-poder-do-teste}

\begin{itemize}
\item
  Na fase de projeto de pesquisa: a análise de poder para determinar o tamanho da amostra objetiva que o tamanho da amostra permita uma probabilidade razoável de detectar um efeito significativo pré-especificado.\textsuperscript{\citeproc{ref-heckman2022}{341}}
\item
  Após a coleta de dados: a análise de poder objetiva informar estudos futuros a respeito do tamanho da amostra necessário para a detecção de um efeito significativo pré-especificado.\textsuperscript{\citeproc{ref-heckman2022}{341}}
\end{itemize}

\begin{infobox}{images/Rlogo}
O pacote \emph{pwr}\textsuperscript{\citeproc{ref-pwr}{280}} fornece a função \href{https://www.rdocumentation.org/packages/pwr/versions/1.3-0/topics/pwr.2p.test}{\emph{pwr.2p.test}} para cálculo do poder do teste de proporção balanceado (2 amostras com mesmo número de participantes).

\end{infobox}

\begin{infobox}{images/Rlogo}
O pacote \emph{pwr}\textsuperscript{\citeproc{ref-pwr}{280}} fornece a função \href{https://www.rdocumentation.org/packages/pwr/versions/1.3-0/topics/pwr.2p.test}{\emph{pwr.2p2n.test}} para cálculo do poder do teste de proporção não balanceado (2 amostras com diferente número de participantes).

\end{infobox}

\begin{infobox}{images/Rlogo}
O pacote \emph{pwr}\textsuperscript{\citeproc{ref-pwr}{280}} fornece a função \href{https://www.rdocumentation.org/packages/pwr/versions/1.3-0/topics/pwr.anova.test}{\emph{pwr.anova.test}} para cálculo do poder do teste de análise de variância balanceado (3 ou mais amostras com mesmo número de participantes).

\end{infobox}

\begin{infobox}{images/Rlogo}
O pacote \emph{pwr}\textsuperscript{\citeproc{ref-pwr}{280}} fornece a função \href{https://www.rdocumentation.org/packages/pwr/versions/1.3-0/topics/pwr.chisq.test}{\emph{pwr.chisq.test}} para cálculo do poder do teste de qui-quadrado \(\chi^2\).

\end{infobox}

\begin{infobox}{images/Rlogo}
O pacote \emph{pwr}\textsuperscript{\citeproc{ref-pwr}{280}} fornece a função \href{https://www.rdocumentation.org/packages/pwr/versions/1.3-0/topics/pwr.f2.test}{\emph{pwr.f2.test}} para cálculo do poder do teste com modelo linear geral.

\end{infobox}

\begin{infobox}{images/Rlogo}
O pacote \emph{pwr}\textsuperscript{\citeproc{ref-pwr}{280}} fornece a função \href{https://www.rdocumentation.org/packages/pwr/versions/1.3-0/topics/pwr.norm.test}{\emph{pwr.norm.test}} para cálculo do poder do teste de média de uma distribuição normal com variância conhecida.

\end{infobox}

\begin{infobox}{images/Rlogo}
O pacote \emph{pwr}\textsuperscript{\citeproc{ref-pwr}{280}} fornece a função \href{https://www.rdocumentation.org/packages/pwr/versions/1.3-0/topics/pwr.p.test}{\emph{pwr.p.test}} para cálculo do poder do teste de proporção (1 amostra).

\end{infobox}

\begin{infobox}{images/Rlogo}
O pacote \emph{pwr}\textsuperscript{\citeproc{ref-pwr}{280}} fornece a função \href{https://www.rdocumentation.org/packages/pwr/versions/1.3-0/topics/pwr.r.test}{\emph{pwr.r.test}} para cálculo do poder to teste de correlação (1 amostra).

\end{infobox}

\begin{infobox}{images/Rlogo}
O pacote \emph{pwr}\textsuperscript{\citeproc{ref-pwr}{280}} fornece a função \href{https://www.rdocumentation.org/packages/pwr/versions/1.3-0/topics/pwr.t.test}{\emph{pwr.t.test}} para cálculo do poder do teste \emph{t} de diferença de 1 amostra, 2 amostras dependentes ou 2 amostras independentes (grupos balanceados).

\end{infobox}

\begin{infobox}{images/Rlogo}
O pacote \emph{pwr}\textsuperscript{\citeproc{ref-pwr}{280}} fornece a função \href{https://www.rdocumentation.org/packages/pwr/versions/1.3-0/topics/pwr.t2n.test}{\emph{pwr.t2n.test}} para cálculo do poder do teste \emph{t} de diferença de 2 amostras independentes (grupos não balanceados).

\end{infobox}

\begin{infobox}{images/Rlogo}
O pacote \emph{longpower}\textsuperscript{\citeproc{ref-longpower}{342}} fornece a função \href{https://www.rdocumentation.org/packages/longpower/versions/1.0.24/topics/power.mmrm}{\emph{power.mmrm}} para calcular o poder de testes com análises por modelo de regressão linear misto.

\end{infobox}

\begin{infobox}{images/Rlogo}
O pacote \emph{Superpower}\textsuperscript{\citeproc{ref-Superpower}{285}} fornece a função \href{https://www.rdocumentation.org/packages/Superpower/versions/0.2.0/topics/power.ftest}{\emph{power.ftest}} para calcular o poder do teste por análise de testes F.

\end{infobox}

\begin{infobox}{images/Rlogo}
O pacote \emph{Superpower}\textsuperscript{\citeproc{ref-Superpower}{285}} fornece a função \href{https://www.rdocumentation.org/packages/Superpower/versions/0.2.0/topics/power_oneway_between}{\emph{power\_oneway\_between}} para calcular o poder do teste por análise de variância (ANOVA) de 1 fator entre-sujeitos.

\end{infobox}

\begin{infobox}{images/Rlogo}
O pacote \emph{Superpower}\textsuperscript{\citeproc{ref-Superpower}{285}} fornece a função \href{https://www.rdocumentation.org/packages/Superpower/versions/0.2.0/topics/power_oneway_within}{\emph{power\_oneway\_within}} para calcular o poder do teste por análise de variância (ANOVA) de 1 fator intra-sujeitos.

\end{infobox}

\begin{infobox}{images/Rlogo}
O pacote \emph{Superpower}\textsuperscript{\citeproc{ref-Superpower}{285}} fornece a função \href{https://www.rdocumentation.org/packages/Superpower/versions/0.2.0/topics/power_oneway_ancova}{\emph{power\_oneway\_ancova}} para calcular o poder do teste por análise de covariância (ANCOVA).

\end{infobox}

\begin{infobox}{images/Rlogo}
O pacote \emph{Superpower}\textsuperscript{\citeproc{ref-Superpower}{285}} fornece a função \href{https://www.rdocumentation.org/packages/Superpower/versions/0.2.0/topics/power_twoway_between}{\emph{power\_twoway\_between}} para calcular o poder do teste por análise de covariância (ANOVA) de 2 fatores entre-sujeitos.

\end{infobox}

\begin{infobox}{images/Rlogo}
O pacote \emph{Superpower}\textsuperscript{\citeproc{ref-Superpower}{285}} fornece a função \href{https://www.rdocumentation.org/packages/Superpower/versions/0.2.0/topics/power_threeway_between}{\emph{power\_threeway\_between}} para calcular o poder do teste por análise de covariância (ANOVA) de 3 fatores entre-sujeitos.

\end{infobox}

\begin{infobox}{images/Rlogo}
O pacote \emph{InteractionPoweR}\textsuperscript{\citeproc{ref-InteractionPoweR}{343}} fornece a função \href{https://www.rdocumentation.org/packages/InteractionPoweR/versions/0.2.1/topics/power_interaction}{\emph{power\_interaction}} para calcular o poder do teste por análise de efeito de interações.

\end{infobox}

\subsection{\texorpdfstring{Por que a análise de poder do teste \emph{post hoc} é inadequada?}{Por que a análise de poder do teste post hoc é inadequada?}}\label{por-que-a-anuxe1lise-de-poder-do-teste-post-hoc-uxe9-inadequada}

\begin{itemize}
\tightlist
\item
  A análise do poder é teoricamente incorreta, uma vez que a probabilidade calculada \(1-\beta\) expressa a probabilidade de um evento futuro, o que não é mais relevante quando o evento de interesse já ocorreu.\textsuperscript{\citeproc{ref-Cummings2003}{205},\citeproc{ref-heckman2022}{341}}
\end{itemize}

\subsection{O que pode ser realizado ao invés da análise de poder?}\label{o-que-pode-ser-realizado-ao-invuxe9s-da-anuxe1lise-de-poder}

\begin{itemize}
\tightlist
\item
  Após a coleta e análise de dados, recomenda-se realizar a análise e interpretação dos resultados a partir do tamanho do efeito e do seu intervalo de confiança no nível de significância \(\alpha\) pré-estabelecido.\textsuperscript{\citeproc{ref-heckman2022}{341}}
\end{itemize}

\chapter{\texorpdfstring{\textbf{Tamanho da amostra}}{Tamanho da amostra}}\label{tamanho-amostral}

\section{Tamanho da amostra}\label{tamanho-da-amostra}

\subsection{O que é tamanho da amostra?}\label{o-que-uxe9-tamanho-da-amostra}

\begin{itemize}
\item
  Tamanho da amostra \(n\) é a quantidade de participantes (ou unidades de análise) necessárias para conduzir um estudo a fim de testar uma hipótese.\textsuperscript{\citeproc{ref-rodruxedguezdeluxe1guila2014}{344}}
\item
  O cálculo do tamanho da amostra depende de quatro pilares interligados --- tamanho de efeito esperado, variabilidade, nível de significância (\(\alpha\)) e poder (\(1-\beta\)) --- cuja combinação determina o \(n\) necessário para detectar efeitos de interesse com precisão adequada.\textsuperscript{\citeproc{ref-Banerjee2010}{15}}
\end{itemize}

\subsection{Por que determinar o tamanho da amostra é importante?}\label{por-que-determinar-o-tamanho-da-amostra-uxe9-importante}

\begin{itemize}
\item
  É virtualmente impossível, devido a limitações de recursos --- tempo, acesso, custo, dentre outros --- coletar dados da população completa.\textsuperscript{\citeproc{ref-kwak2017}{8}}
\item
  Uma amostra muito pequena para o estudo pode resultar em ajuste exagerado, imprecisão e baixo poder do teste.\textsuperscript{\citeproc{ref-van2022a}{133}}
\end{itemize}

\subsection{Quais fatores devem ser considerados para determinar o tamanho da amostra?}\label{quais-fatores-devem-ser-considerados-para-determinar-o-tamanho-da-amostra}

\begin{itemize}
\item
  Tamanho da população (\(N\)): O tamanho da amostra depende parcialmente do tamanho da população de origem. Geralmente assume-se que a população tem tamanho desconhecido ou infinito. Em alguns estudos serão amostradas populações de tamanho finito (inferior a 100.000 indivíduos), geralmente em pesquisas descritivas, em que esse tamanho deve ser incorporado nos cálculos.\textsuperscript{\citeproc{ref-rodruxedguezdeluxe1guila2014}{344}}
\item
  Delineamento do estudo.\textsuperscript{\citeproc{ref-rodruxedguezdeluxe1guila2014}{344}}
\item
  Quantidade e características (dependente vs.~independente) dos grupos de participantes do estudo.\textsuperscript{\citeproc{ref-rodruxedguezdeluxe1guila2014}{344}}
\item
  Erros tipo I (\(\alpha\)) e tipo II (\(\beta\)).\textsuperscript{\citeproc{ref-rodruxedguezdeluxe1guila2014}{344}}
\item
  Tipo de variável a ser observada (contínua, intervalo, ordinal, nominal, dicotômica).\textsuperscript{\citeproc{ref-rodruxedguezdeluxe1guila2014}{344}}
\item
  Tamanho de efeito mínimo a ser observado.\textsuperscript{\citeproc{ref-rodruxedguezdeluxe1guila2014}{344}}
\item
  Variabilidade da(s) variável(eis) coletada(s).\textsuperscript{\citeproc{ref-rodruxedguezdeluxe1guila2014}{344}}
\item
  Lateralidade do teste de hipótese (uni- ou bicaudais).\textsuperscript{\citeproc{ref-rodruxedguezdeluxe1guila2014}{344}}
\item
  Perdas de dados durante a coleta e/ou acompanhamento dos participantes do estudo.\textsuperscript{\citeproc{ref-rodruxedguezdeluxe1guila2014}{344}}
\end{itemize}

\begin{infobox}{images/Rlogo}
O pacote \emph{pwr}\textsuperscript{\citeproc{ref-pwr}{280}} fornece a função \href{https://www.rdocumentation.org/packages/pwr/versions/1.3-0/topics/plot.power.htest}{\emph{plot.power.htest}} para apresentar graficamente a relação entre o tamanho da amostra e o poder de testes de hipóteses.

\end{infobox}

\subsection{Quais aspectos éticos estão envolvidos no tamanho da amostra?}\label{quais-aspectos-uxe9ticos-estuxe3o-envolvidos-no-tamanho-da-amostra}

\begin{itemize}
\item
  Determinar a priori o tamanho da amostra pode diminuir o risco de realizar testes ou intervenções desnecessários, de desperdício de recursos (tempo e dinheiro) associados e, por outro lado, de coletar dados insuficientes para testar as hipóteses do estudo.\textsuperscript{\citeproc{ref-rodruxedguezdeluxe1guila2014}{344}}
\item
  O tratamento ético dos participantes do estudo, portanto, não exige que se considere se o poder do estudo é inferior à meta convencional de 80\% ou 90\%.\textsuperscript{\citeproc{ref-Bacchetti2005}{345}}
\item
  Estudos com poder \textless80\% não são necessariamente antiéticos.\textsuperscript{\citeproc{ref-Bacchetti2005}{345}}
\item
  Metas convencionais de poder (80--90\%) são guias pragmáticos e não regras morais rígidas; estudos com poder \textless80\% não são automaticamente antiéticos quando bem justificados.\textsuperscript{\citeproc{ref-Bacchetti2005}{345}}
\item
  Grandes estudos podem ser desejáveis por outras razões que não as éticas.\textsuperscript{\citeproc{ref-Bacchetti2005}{345}}
\end{itemize}

\section{Saturação em pesquisas qualitativas}\label{saturauxe7uxe3o-em-pesquisas-qualitativas}

\subsection{O que é saturação de dados em pesquisas qualitativas?}\label{o-que-uxe9-saturauxe7uxe3o-de-dados-em-pesquisas-qualitativas}

\begin{itemize}
\item
  Saturação é o ponto em que a coleta de dados não produz novas informações, categorias ou temas, indicando que o fenômeno investigado já foi suficientemente explorado.\textsuperscript{\citeproc{ref-ahmed2025}{346}}
\item
  Essa noção surgiu na teoria fundamentada com o termo ``saturação teórica'', mas hoje é amplamente usada em diferentes tradições qualitativas, incluindo fenomenologia, etnografia e análise temática.\textsuperscript{\citeproc{ref-hennink2022}{347}}
\end{itemize}

\subsection{Quais tipos de saturação existem?}\label{quais-tipos-de-saturauxe7uxe3o-existem}

\begin{itemize}
\item
  Saturação de códigos: ocorre quando não emergem novos códigos relevantes nos dados\textsuperscript{\citeproc{ref-hennink2022}{347}}
\item
  Saturação de significados: atinge-se quando a profundidade e a variação dos significados de um tema foram plenamente exploradas.\textsuperscript{\citeproc{ref-hennink2022}{347}}
\item
  Saturação teórica: quando categorias estão suficientemente desenvolvidas e suas relações esclarecidas.\textsuperscript{\citeproc{ref-ahmed2025}{346}}
\item
  Saturação de metatemas: em pesquisas multicêntricas, quando os grandes temas transversais já foram identificados.\textsuperscript{\citeproc{ref-wutich2024}{348}}
\end{itemize}

\begin{figure}

{\centering \includegraphics[width=1\linewidth]{Ciencia-com-R_files/figure-latex/saturacao-1} 

}

\caption{Curvas de poder para testes t (quantitativo). Linhas sólidas: α=0,05 | tracejadas: α=0,01 | linhas horizontais em 80\% e 90\% de poder.}\label{fig:saturacao}
\end{figure}

\begin{figure}

{\centering \includegraphics[width=1\linewidth]{Ciencia-com-R_files/figure-latex/saturacao2-1} 

}

\caption{Curvas de saturação para estudos qualitativos de descoberta de temas.}\label{fig:saturacao2}
\end{figure}

\subsection{Quantas entrevistas ou grupos focais são necessários para alcançar saturação?}\label{quantas-entrevistas-ou-grupos-focais-suxe3o-necessuxe1rios-para-alcanuxe7ar-saturauxe7uxe3o}

\begin{itemize}
\item
  Estudos empíricos mostram que a saturação de códigos pode ser atingida com 9 a 17 entrevistas em populações homogêneas e objetivos específicos.\textsuperscript{\citeproc{ref-hennink2022}{347}}
\item
  Para saturação de significados, podem ser necessárias entre 16 e 24 entrevistas.\textsuperscript{\citeproc{ref-hennink2022}{347}}
\item
  Em grupos focais, a saturação temática pode ocorrer com 4 a 8 grupos homogêneos.\textsuperscript{\citeproc{ref-hennink2022}{347}}
\item
  Revisões recentes sugerem que a saturação teórica exige 20 a 30 entrevistas ou mais, dependendo da complexidade do estudo.\textsuperscript{\citeproc{ref-wutich2024}{348}}
\end{itemize}

\subsection{Quais debates existem sobre o conceito de saturação?}\label{quais-debates-existem-sobre-o-conceito-de-saturauxe7uxe3o}

\begin{itemize}
\item
  Defensores argumentam que a saturação é central para garantir rigor e confiança nos resultados qualitativos.\textsuperscript{\citeproc{ref-ahmed2025}{346}}
\item
  Críticos sugerem que o conceito pode ser usado de forma rígida, levando a coletas excessivas ou pouco sensíveis a perspectivas únicas.\textsuperscript{\citeproc{ref-ahmed2025}{346}}
\item
  Pesquisadores contemporâneos recomendam usar a saturação de forma flexível, adaptada ao contexto, método e população estudada.\textsuperscript{\citeproc{ref-wutich2024}{348}}
\end{itemize}

\subsection{Quais recomendações práticas para tamanho de amostras de estudos qualitativos?}\label{quais-recomendauxe7uxf5es-pruxe1ticas-para-tamanho-de-amostras-de-estudos-qualitativos}

\begin{itemize}
\item
  Para entrevistas individuais: 9--12 entrevistas podem ser suficientes para saturação temática em contextos homogêneos, mas estudos heterogêneos ou multicêntricos exigem mais casos.\textsuperscript{\citeproc{ref-hennink2022}{347},\citeproc{ref-wutich2024}{348}}
\item
  Para grupos focais: 4--8 grupos são geralmente adequados.\textsuperscript{\citeproc{ref-hennink2022}{347}}
\item
  Para estudos multicêntricos: recomenda-se 20--40 entrevistas por local para alcançar saturação de metatemas.\textsuperscript{\citeproc{ref-wutich2024}{348}}
\item
  É importante relatar não apenas o número de entrevistas, mas também como e quando a saturação foi avaliada.\textsuperscript{\citeproc{ref-vasileiou2018}{349}}
\end{itemize}

\section{``Fome de dados''}\label{fome-de-dados}

\subsection{O que significa ``fome de dados''?}\label{o-que-significa-fome-de-dados}

\begin{itemize}
\item
  \emph{Data hungry} descreve a necessidade de um modelo contar com muitos eventos por variável (EPV) para alcançar estabilidade estatística.
\item
  Enquanto a regressão logística (LR) atinge desempenho estável com cerca de 20--50 EPV, modelos como random forest (RF), redes neurais (NN) e máquinas de vetor de suporte (SVM) podem demandar \textgreater200 EPV para reduzir o otimismo e estabilizar a AUC.
\end{itemize}

\subsection{Por que a ``fome de dados'' é relevante?}\label{por-que-a-fome-de-dados-uxe9-relevante}

\begin{itemize}
\item
  Em bases de dados pequenas, modelos clássicos tendem a ser mais robustos e menos suscetíveis a superajuste.\textsuperscript{\citeproc{ref-vanderploeg2014}{317}}
\item
  O uso de modelos modernos só se justifica quando há grandes bases de dados, caso contrário o ganho em acurácia é marginal.\textsuperscript{\citeproc{ref-vanderploeg2014}{317}}
\item
  Esse conceito conecta diretamente a escolha do modelo ao planejamento amostral.\textsuperscript{\citeproc{ref-vanderploeg2014}{317}}
\end{itemize}

\section{Eventos por variável (EPV) em modelos preditivos}\label{eventos-por-variuxe1vel-epv-em-modelos-preditivos}

\subsection{Quantos eventos por variável (EPV) são necessários?}\label{quantos-eventos-por-variuxe1vel-epv-suxe3o-necessuxe1rios}

\begin{itemize}
\item
  Regressão logística: entre 20 e 50 EPV.\textsuperscript{\citeproc{ref-vanderploeg2014}{317}}
\item
  Árvore de decisão para classificação e regressão: cerca de 60 EPV.\textsuperscript{\citeproc{ref-vanderploeg2014}{317}}
\item
  Máquina de vetores de suporte, redes neurais e \emph{random forests}: muitas vezes \textgreater200 EPV e ainda instáveis.\textsuperscript{\citeproc{ref-vanderploeg2014}{317}}
\end{itemize}

\subsection{O que acontece se não houver eventos suficientes?}\label{o-que-acontece-se-nuxe3o-houver-eventos-suficientes}

\begin{itemize}
\item
  Modelos modernos podem apresentar alto otimismo (desempenho inflado no treino, mas ruim na validação).\textsuperscript{\citeproc{ref-vanderploeg2014}{317}}
\item
  Pequenos bancos de dados favorecem o uso de modelos clássicos.\textsuperscript{\citeproc{ref-vanderploeg2014}{317}}
\end{itemize}

\section{Cálculo do tamanho da amostra}\label{cuxe1lculo-do-tamanho-da-amostra}

\subsection{Como calcular o tamanho da amostra?}\label{como-calcular-o-tamanho-da-amostra}

\begin{itemize}
\item
  O tamanho amostral pode ser calculado por meio de fórmulas matemáticas que tendem a assegurar margens de erros tipos I (\(\alpha\)) e II (\(\beta\)) para a estimação dos parâmetros populacionais (tamanho de efeito) a partir dos dados amostrais.\textsuperscript{\citeproc{ref-rodruxedguezdeluxe1guila2014}{344}}
\item
  O tamanho da amostra deve ser calculado para cada um dos objetivos primários e/ou secundários, sendo escolhido o maior tamanho de amostra calculado para o estudo.\textsuperscript{\citeproc{ref-rodruxedguezdeluxe1guila2014}{344}}
\item
  Geralmente é recomendado ser cético em relação às regras práticas para o tamanho da amostra, tais como a proporção entre o número de variáveis (ou eventos) e de participantes.\textsuperscript{\citeproc{ref-van2022a}{133}}
\end{itemize}

\subsection{Como especificar o tamanho do efeito esperado?}\label{como-especificar-o-tamanho-do-efeito-esperado}

\begin{itemize}
\item
  Estudo-piloto --- realizados nas mesmas condições do estudo, mas envolvendo um tamanho de amostra limitado --- pode ser útil na estimativa do tamanho da amostra a partir do tamanho do efeito estimado.\textsuperscript{\citeproc{ref-rodruxedguezdeluxe1guila2014}{344}}
\item
  Utilizar os limites dos intervalos de confiança de estudos-piloto de ensaios clínicos como estimativa do tamanho do efeito pode aumentar o poder estatístico da análise se comparado ao uso das estimativas pontuais obtidas no mesmo piloto.\textsuperscript{\citeproc{ref-ying2023}{350}}
\item
  Embora os testes de hipótese considerem efeito nulo para a hipótese nula --- ex.: diferença de média (\(H_{0}: \mu_{1} - \mu_{2}=0\)), correlação (\(H_{0}: r=0\)), associação (\(H_{0}: \beta=0\) ou \(H_{0}: OR=1\)) ---, em geral é improvável que os efeitos populacionais sejam de fato nulos (isto é, exatamente 0).\textsuperscript{\citeproc{ref-Andrade2020}{351}}
\end{itemize}

\begin{infobox}{images/Rlogo}
O pacote \emph{pwr}\textsuperscript{\citeproc{ref-pwr}{280}} fornece a função \href{https://www.rdocumentation.org/packages/pwr/versions/1.3-0/topics/pwr.2p.test}{\emph{pwr.2p.test}} para cálculo do tamanho da amostra para testes de proporção balanceados (2 amostras com mesmo número de participantes).

\end{infobox}

\begin{infobox}{images/Rlogo}
O pacote \emph{pwr}\textsuperscript{\citeproc{ref-pwr}{280}} fornece a função \href{https://www.rdocumentation.org/packages/pwr/versions/1.3-0/topics/pwr.2p.test}{\emph{pwr.2p2n.test}} para cálculo do tamanho da amostra para testes de proporção não balanceados (2 amostras com diferente número de participantes).

\end{infobox}

\begin{infobox}{images/Rlogo}
O pacote \emph{pwr}\textsuperscript{\citeproc{ref-pwr}{280}} fornece a função \href{https://www.rdocumentation.org/packages/pwr/versions/1.3-0/topics/pwr.anova.test}{\emph{pwr.anova.test}} para cálculo do tamanho da amostra para testes de análise de variância balanceados (3 ou mais amostras com mesmo número de participantes).

\end{infobox}

\begin{infobox}{images/Rlogo}
O pacote \emph{pwr}\textsuperscript{\citeproc{ref-pwr}{280}} fornece a função \href{https://www.rdocumentation.org/packages/pwr/versions/1.3-0/topics/pwr.chisq.test}{\emph{pwr.chisq.test}} para cálculo do tamanho da amostra para testes de qui-quadrado \(\chi^2\).

\end{infobox}

\begin{infobox}{images/Rlogo}
O pacote \emph{pwr}\textsuperscript{\citeproc{ref-pwr}{280}} fornece a função \href{https://www.rdocumentation.org/packages/pwr/versions/1.3-0/topics/pwr.f2.test}{\emph{pwr.f2.test}} para cálculo do tamanho da amostra para testes com modelo linear geral.

\end{infobox}

\begin{infobox}{images/Rlogo}
O pacote \emph{pwr}\textsuperscript{\citeproc{ref-pwr}{280}} fornece a função \href{https://www.rdocumentation.org/packages/pwr/versions/1.3-0/topics/pwr.norm.test}{\emph{pwr.norm.test}} para cálculo do tamanho da amostra para a média de uma distribuição normal com variância conhecida.

\end{infobox}

\begin{infobox}{images/Rlogo}
O pacote \emph{pwr}\textsuperscript{\citeproc{ref-pwr}{280}} fornece a função \href{https://www.rdocumentation.org/packages/pwr/versions/1.3-0/topics/pwr.p.test}{\emph{pwr.p.test}} para cálculo do tamanho da amostra para testes de proporção (1 amostra).

\end{infobox}

\begin{infobox}{images/Rlogo}
O pacote \emph{pwr}\textsuperscript{\citeproc{ref-pwr}{280}} fornece a função \href{https://www.rdocumentation.org/packages/pwr/versions/1.3-0/topics/pwr.r.test}{\emph{pwr.r.test}} para cálculo do tamanho da amostra para testes de correlação (1 amostra).

\end{infobox}

\begin{infobox}{images/Rlogo}
O pacote \emph{pwr}\textsuperscript{\citeproc{ref-pwr}{280}} fornece a função \href{https://www.rdocumentation.org/packages/pwr/versions/1.3-0/topics/pwr.t.test}{\emph{pwr.t.test}} para cálculo do tamanho da amostra para testes \emph{t} de diferença de 1 amostra, 2 amostras dependentes ou 2 amostras independentes (grupos balanceados).

\end{infobox}

\begin{infobox}{images/Rlogo}
O pacote \emph{pwr}\textsuperscript{\citeproc{ref-pwr}{280}} fornece a função \href{https://www.rdocumentation.org/packages/pwr/versions/1.3-0/topics/pwr.t2n.test}{\emph{pwr.t2n.test}} para cálculo do tamanho da amostra para testes \emph{t} de diferença de 2 amostras independentes (grupos não balanceados).

\end{infobox}

\begin{infobox}{images/Rlogo}
O pacote \emph{longpower}\textsuperscript{\citeproc{ref-longpower}{342}} fornece a função \href{https://www.rdocumentation.org/packages/longpower/versions/1.0.24/topics/power.mmrm}{\emph{power.mmrm}} para calcular o tamanho da amostra para estudos com análises por modelo de regressão linear misto.

\end{infobox}

\section{Perdas de amostra}\label{perdas-de-amostra}

\subsection{O que é perda de amostra?}\label{o-que-uxe9-perda-de-amostra}

\begin{itemize}
\item
  Perda de amostra(s) --- isto é, participante(s) ou unidade(s) de análise --- pode ocorrer durante a coleta e/ou acompanhamento dos participantes do estudo.\textsuperscript{\citeproc{ref-rodruxedguezdeluxe1guila2014}{344}}
\item
  Perda amostral pode ocorrer por: abandono ou desistência do participante, perda de contato com o participante, perda de informação, ocorrência de eventos adversos, morte do participante, entre outros.\textsuperscript{\citeproc{ref-rodruxedguezdeluxe1guila2014}{344}}
\end{itemize}

\subsection{Por que a perda de amostra é um problema?}\label{por-que-a-perda-de-amostra-uxe9-um-problema}

\begin{itemize}
\item
  A perda de amostra reduz o tamanho efetivo de \(n\) e, portanto, o poder estatístico do estudo, elevando a probabilidade de erro tipo II (\(\beta\)).\textsuperscript{\citeproc{ref-van2022a}{133},\citeproc{ref-rodruxedguezdeluxe1guila2014}{344}}
\item
  A atrição diferencial também pode introduzir viés de seleção (ou de atrito), quando as características dos participantes que permanecem diferem sistematicamente das daqueles que se perdem ao seguimento.\textsuperscript{\citeproc{ref-rodruxedguezdeluxe1guila2014}{344}}
\end{itemize}

\subsection{Como evitar perda de amostra?}\label{como-evitar-perda-de-amostra}

\begin{itemize}
\item
  A perda de amostra pode ser evitada por meio de um planejamento cuidadoso do estudo, incluindo a definição de critérios de inclusão e exclusão claros e apropriados, bem como a definição de estratégias para minimizar a perda de amostra.\textsuperscript{\citeproc{ref-REF}{\textbf{REF?}}}
\item
  A perda de amostra pode ser compensada pelo aumento do tamanho da amostra, desde que o aumento seja suficiente para manter o poder do estudo.\textsuperscript{\citeproc{ref-rodruxedguezdeluxe1guila2014}{344}}
\end{itemize}

\section{Ajustes no tamanho da amostra}\label{ajustes-no-tamanho-da-amostra}

\subsection{Por que ajustar o tamanho da amostra?}\label{por-que-ajustar-o-tamanho-da-amostra}

\begin{itemize}
\tightlist
\item
  O tamanho da amostra pode ser ajustado durante o estudo para compensar a perda de amostra, desde que o aumento seja suficiente para manter o poder do estudo.\textsuperscript{\citeproc{ref-rodruxedguezdeluxe1guila2014}{344}}
\end{itemize}

\subsection{Como ajustar para perda amostral?}\label{como-ajustar-para-perda-amostral}

\begin{itemize}
\tightlist
\item
  Aumentar o tamanho da amostra estimada \(n\) pela porcentagem \(d\) de perdas esperada ou prevista, para obter o tamanho da amostra efetiva \(n'\) \eqref{eq:samplesizeadj1}:\textsuperscript{\citeproc{ref-rodruxedguezdeluxe1guila2014}{344}}
\end{itemize}

\begin{equation}
\label{eq:samplesizeadj1}
n' = \dfrac{n}{1-d}
\end{equation}

\section{Justificativa do tamanho da amostra}\label{justificativa-do-tamanho-da-amostra}

\subsection{Como justificar o tamanho da amostra de um estudo?}\label{como-justificar-o-tamanho-da-amostra-de-um-estudo}

\begin{itemize}
\item
  Em estudos que envolvem condições raras, pode ser difícil recrutar o número necessário de participantes devido à limitada disponibilidade de casos da população. Mesmo assim, é aconselhável determinar o tamanho da amostra.\textsuperscript{\citeproc{ref-rodruxedguezdeluxe1guila2014}{344}}
\item
  Quando um estudo deste tipo não é possível, as considerações referentes ao tamanho da amostra são justificadas de acordo com o número máximo de pacientes que podem ser recrutados no decorrer do estudo.\textsuperscript{\citeproc{ref-rodruxedguezdeluxe1guila2014}{344}}
\end{itemize}

\subsection{Como justificar o tamanho da amostra em estudos qualitativos?}\label{como-justificar-o-tamanho-da-amostra-em-estudos-qualitativos}

\begin{itemize}
\item
  Pesquisas qualitativas devem apresentar uma justificativa explícita da amostra, relacionando-a à estratégia de coleta, aos objetivos e ao critério de saturação adotado.\textsuperscript{\citeproc{ref-vasileiou2018}{349}}
\item
  A noção de ``poder da informação'' (\emph{information power}) indica que quanto mais relevante e focada é a amostra em relação à pergunta de pesquisa, menor pode ser o número de participantes.\textsuperscript{\citeproc{ref-vasileiou2018}{349}}
\item
  Relatar claramente o processo de decisão aumenta a transparência e a credibilidade da pesquisa.\textsuperscript{\citeproc{ref-vasileiou2018}{349}}
\end{itemize}

\chapter{\texorpdfstring{\textbf{Plano de análise}}{Plano de análise}}\label{plano-analise}

\section{Plano de análise estatística}\label{plano-de-anuxe1lise-estatuxedstica}

\subsection{O que é plano de análise estatística?}\label{o-que-uxe9-plano-de-anuxe1lise-estatuxedstica}

\begin{itemize}
\tightlist
\item
  .\textsuperscript{\citeproc{ref-REF}{\textbf{REF?}}}
\end{itemize}

\section{Diretrizes para redação}\label{diretrizes-para-redauxe7uxe3o-1}

\subsection{Quais são as diretrizes para redação de planos de análise estatística?}\label{quais-suxe3o-as-diretrizes-para-redauxe7uxe3o-de-planos-de-anuxe1lise-estatuxedstica}

\begin{itemize}
\item
  Visite a rede \emph{Enhancing the QUAlity and Transparency Of health Research} (\href{https://www.equator-network.org/}{EQUATOR Network}) para encontrar diretrizes específicas.
\item
  \emph{Guidelines for the Content of Statistical Analysis Plans in Clinical Trials}:\textsuperscript{\citeproc{ref-Gamble2017}{352}} \url{https://www.equator-network.org/reporting-guidelines/guidelines-for-the-content-of-statistical-analysis-plans-in-clinical-trials/}
\end{itemize}

\cftaddtitleline{toc}{chapter}{\rule{\textwidth}{0.4pt}}{}

\chapter*{\texorpdfstring{\emph{PARTE 9: DELINEAMENTOS E SÍNTESE DE EVIDÊNCIAS}}{PARTE 9: DELINEAMENTOS E SÍNTESE DE EVIDÊNCIAS}}\label{parte-9}
\addcontentsline{toc}{chapter}{\emph{PARTE 9: DELINEAMENTOS E SÍNTESE DE EVIDÊNCIAS}}

\par\noindent\rule{\textwidth}{0.05in}

\section*{Tipos de estudo e integração de resultados: observacionais, experimentais e revisões}\label{tipos-de-estudo-e-integrauxe7uxe3o-de-resultados-observacionais-experimentais-e-revisuxf5es}

\markboth{}{}

\chapter{\texorpdfstring{\textbf{Delineamento de estudos}}{Delineamento de estudos}}\label{delineamento-estudos}

\section{Critérios de delineamento}\label{crituxe9rios-de-delineamento}

\subsection{Quais critérios são utilizados para classificar os delineamentos de estudos?}\label{quais-crituxe9rios-suxe3o-utilizados-para-classificar-os-delineamentos-de-estudos}

\begin{itemize}
\tightlist
\item
  .\textsuperscript{\citeproc{ref-REF}{\textbf{REF?}}}
\end{itemize}

\section{Alocação}\label{alocauxe7uxe3o}

\subsection{O que é alocação?}\label{o-que-uxe9-alocauxe7uxe3o}

\begin{itemize}
\tightlist
\item
  .\textsuperscript{\citeproc{ref-REF}{\textbf{REF?}}}
\end{itemize}

\begin{figure}

{\centering \includegraphics{Ciencia-com-R_files/figure-latex/alocacao-1-1-1} 

}

\caption{Alocação 1:1 entre dois grupos de participantes}\label{fig:alocacao-1-1}
\end{figure}

\section{Cegamento}\label{cegamento}

\begin{itemize}
\tightlist
\item
  .\textsuperscript{\citeproc{ref-REF}{\textbf{REF?}}}
\end{itemize}

\subsection{O que é cegamento?}\label{o-que-uxe9-cegamento}

\section{Pareamento}\label{pareamento}

\subsection{O que é pareamento?}\label{o-que-uxe9-pareamento}

\begin{itemize}
\item
  Pareamento significa que para cada participante de um grupo (por exemplo, com alguma condição clínica), existe um (ou mais) participantes (por exemplo, grupo controle) que possui características iguais ou similares relativas a algumas variáveis de interesse.\textsuperscript{\citeproc{ref-Bland1994}{353}}
\item
  As variáveis escolhidas para pareamento devem ter relação com as variáveis de desfecho, mas não são de interesse elas mesmas.\textsuperscript{\citeproc{ref-Bland1994}{353}}
\item
  O ajuste por pareamento deve ser incluído nas análises estatísticas mesmo que as variáveis de pareamento não sejam consideradas prognósticas ou confundidores na amostra estudada.\textsuperscript{\citeproc{ref-Bland1994}{353}}
\item
  A ausência de evidência estatística de diferença entre grupos não é considerada pareamento.\textsuperscript{\citeproc{ref-Bland1994}{353}}
\end{itemize}

\section{Aleatorização}\label{aleatorizauxe7uxe3o}

\subsection{O que é aleatorização?}\label{o-que-uxe9-aleatorizauxe7uxe3o}

\begin{itemize}
\tightlist
\item
  .\textsuperscript{\citeproc{ref-REF}{\textbf{REF?}}}
\end{itemize}

\section{Taxonomia de estudos}\label{taxonomia-de-estudos}

\subsection{Como podem ser classificados os estudos científicos?}\label{como-podem-ser-classificados-os-estudos-cientuxedficos}

\begin{itemize}
\item
  Estudos científicos podem ser classificados em \emph{básicos}, \emph{observacionais}, \emph{experimentais}, \emph{acurácia diagnóstica}, \emph{propriedades psicométricas}, \emph{avaliação econômica} e \emph{revisões de literatura}:\textsuperscript{\citeproc{ref-Grant2009}{354}--\citeproc{ref-chipman2022}{363}}
\item
  \emph{Estudos básicos}\textsuperscript{\citeproc{ref-Suxfct2014}{355},\citeproc{ref-Chidambaram2019}{360}}

  \begin{itemize}
  \item
    Genética
  \item
    Celular
  \item
    Experimentos com animais
  \item
    Desenvolvimento de métodos
  \end{itemize}
\item
  \emph{Estudos de simulação computacional}\textsuperscript{\citeproc{ref-Erdemir2020}{361},\citeproc{ref-chipman2022}{363}}
\item
  \emph{Estudos de propriedades psicométricas}\textsuperscript{\citeproc{ref-Souza2017}{356},\citeproc{ref-echevarruxeda-guanilo2019}{358}}

  \begin{itemize}
  \item
    Validade
  \item
    Concordância
  \item
    Confiabilidade
  \end{itemize}
\item
  \emph{Estudos de desempenho diagnóstico}\textsuperscript{\citeproc{ref-Chassuxe92019}{359},\citeproc{ref-Yang2021}{362}}

  \begin{itemize}
  \item
    Transversal
  \item
    Caso-Controle
  \item
    Comparativo
  \item
    Totalmente pareado
  \item
    Parcialmente pareado com subgrupo aleatório
  \item
    Parcialmente pareado com subgrupo não aleatório
  \item
    Não pareado aleatório
  \item
    Não pareado não aleatório
  \end{itemize}
\item
  \emph{Estudos observacionais}\textsuperscript{\citeproc{ref-Suxfct2014}{355},\citeproc{ref-Chidambaram2019}{360}}

  \begin{itemize}
  \item
    Descritivo

    \begin{itemize}
    \item
      Estudo de caso
    \item
      Série de casos
    \item
      Transversal
    \end{itemize}
  \item
    Analítico

    \begin{itemize}
    \item
      Transversal
    \item
      Caso-Controle

      \begin{itemize}
      \item
        Caso-Controle aninhado
      \item
        Caso-Coorte
      \end{itemize}
    \end{itemize}
  \item
    Coorte prospectiva ou retrospectiva
  \end{itemize}
\item
  \emph{Estudos quase-experimentais}\textsuperscript{\citeproc{ref-reeves2017}{357}}

  \begin{itemize}
  \item
    Quase-aleatorizado controlado
  \item
    Estimação de variável instrumental
  \item
    Descontinuidade de regressão
  \item
    Série temporal interrompida controlada
  \item
    Série temporal interrompida
  \item
    Diferença
  \end{itemize}
\item
  \emph{Estudos experimentais}\textsuperscript{\citeproc{ref-Suxfct2014}{355},\citeproc{ref-Chidambaram2019}{360}}

  \begin{itemize}
  \item
    Fases I a IV

    \begin{itemize}
    \item
      Aleatorizado controlado
    \item
      Não-aleatorizado controlado
    \item
      Autocontrolado
    \item
      Cruzado
    \item
      Fatorial
    \end{itemize}
  \item
    Campo
  \item
    Comunitário
  \end{itemize}
\item
  \emph{Estudos de avaliação econômica}\textsuperscript{\citeproc{ref-Suxfct2014}{355}}

  \begin{itemize}
  \item
    Análise de custo
  \item
    Análise de minimização de custo
  \item
    Análise de custo-utilidade
  \item
    Análise de custo-efetividade
  \item
    Análise de custo-benefício
  \end{itemize}
\item
  \emph{Estudos de revisão}\textsuperscript{\citeproc{ref-Grant2009}{354}}

  \begin{itemize}
  \item
    Estado-da-arte
  \item
    Narrativa
  \item
    Crítica
  \item
    Mapeamento
  \item
    Escopo
  \item
    Busca e revisão sistemática
  \item
    Sistematizada
  \item
    Sistemática

    \begin{itemize}
    \item
      Meta-análise
    \item
      Bibliométrica.\textsuperscript{\citeproc{ref-donthu2021}{364},\citeproc{ref-lim2023}{365}}
    \end{itemize}
  \item
    Sistemática qualitativa
  \item
    Mista
  \item
    Visão geral
  \item
    Rápida
  \item
    Guarda-chuva
  \end{itemize}
\end{itemize}

\chapter{\texorpdfstring{\textbf{Simulação computacional}}{Simulação computacional}}\label{simulacao-computacional}

\section{Simulações computacionais}\label{simulauxe7uxf5es-computacionais}

\subsection{O que são simulações computacionais?}\label{o-que-suxe3o-simulauxe7uxf5es-computacionais}

\begin{itemize}
\item
  Simulações computacionais consistem na geração de dados artificiais baseados em regras matemáticas e estatísticas, permitindo testar hipóteses, validar métodos e explorar cenários complexos sem necessidade de dados reais.\textsuperscript{\citeproc{ref-hinsen2011}{29}}
\item
  A simulação é frequentemente usada em estatística para avaliar o desempenho de testes, estimadores e modelos sob diferentes condições.\textsuperscript{\citeproc{ref-REF}{\textbf{REF?}}}
\end{itemize}

\subsection{Por que usar simulações?}\label{por-que-usar-simulauxe7uxf5es}

\begin{itemize}
\item
  Testar o comportamento de métodos estatísticos sob diferentes premissas (ex: normalidade, homocedasticidade, tamanho amostral).\textsuperscript{\citeproc{ref-REF}{\textbf{REF?}}}
\item
  Avaliar a robustez de algoritmos computacionais.\textsuperscript{\citeproc{ref-REF}{\textbf{REF?}}}
\item
  Reproduzir processos naturais ou sociais para compreensão teórica.\textsuperscript{\citeproc{ref-REF}{\textbf{REF?}}}
\end{itemize}

\subsection{Quais são as boas práticas em simulações computacionais?}\label{quais-suxe3o-as-boas-pruxe1ticas-em-simulauxe7uxf5es-computacionais}

\begin{itemize}
\item
  Defina claramente o objetivo da simulação e as hipóteses a serem testadas, incluindo quais aspectos do fenômeno ou do método você pretende avaliar.\textsuperscript{\citeproc{ref-Eglen2017}{42}}
\item
  Use uma semente para o gerador de números aleatórios com set.seed() para garantir a reprodutibilidade dos resultados.\textsuperscript{\citeproc{ref-REF}{\textbf{REF?}}}
\item
  Documente detalhadamente o processo de simulação, incluindo os parâmetros utilizados, a lógica do algoritmo e as suposições feitas.\textsuperscript{\citeproc{ref-trisovic2022a}{366}}
\item
  Realize múltiplas simulações (ex.: 1000 ou mais) para obter estimativas estáveis e resultados mais robustos e confiáveis.\textsuperscript{\citeproc{ref-REF}{\textbf{REF?}}}
\item
  Analise os resultados de forma crítica, considerando a variabilidade, as limitações do modelo e possíveis vieses do processo de simulação.\textsuperscript{\citeproc{ref-REF}{\textbf{REF?}}}
\item
  Use funções vetorizadas para otimizar o desempenho e reduzir o tempo de execução da simulação.\textsuperscript{\citeproc{ref-REF}{\textbf{REF?}}}
\end{itemize}

\begin{infobox}{images/Rlogo}
O pacote \emph{base}\textsuperscript{\citeproc{ref-base}{55}} fornece a função \href{https://www.rdocumentation.org/packages/base/versions/3.6.2/topics/Random}{\emph{set.seed}} para especificar uma semente e garantir a reprodutibilidade de computações que envolvem números aleatórios.

\end{infobox}

\section{Características}\label{caracteruxedsticas}

\subsection{Quais são as características de estudos de simulação computacional?}\label{quais-suxe3o-as-caracteruxedsticas-de-estudos-de-simulauxe7uxe3o-computacional}

\begin{itemize}
\tightlist
\item
  .\textsuperscript{\citeproc{ref-REF}{\textbf{REF?}}}
\end{itemize}

\section{Métodos de simulação}\label{muxe9todos-de-simulauxe7uxe3o}

\subsection{Simulações computacionais dependem da distribuição Normal?}\label{simulauxe7uxf5es-computacionais-dependem-da-distribuiuxe7uxe3o-normal}

\begin{itemize}
\item
  Não. Simulações computacionais não dependem da distribuição Normal.
  Qualquer distribuição de probabilidade pode ser usada para gerar dados artificiais, desde que represente adequadamente o mecanismo gerador do fenômeno em estudo.\textsuperscript{\citeproc{ref-REF}{\textbf{REF?}}}
\item
  A escolha da distribuição depende da natureza da variável (contínua ou discreta), do domínio dos valores possíveis e da estrutura probabilística do processo observado.\textsuperscript{\citeproc{ref-REF}{\textbf{REF?}}}
\item
  Distribuições como Binomial, Poisson e Exponencial são frequentemente utilizadas em simulações para representar proporções, contagens de eventos e tempos até a ocorrência de eventos, respectivamente.\textsuperscript{\citeproc{ref-REF}{\textbf{REF?}}}
\item
  Simulações baseadas em diferentes distribuições permitem avaliar o desempenho de métodos estatísticos e algoritmos inferenciais sob condições realistas e variadas, incluindo assimetria, discretação e presença de valores extremos.\textsuperscript{\citeproc{ref-REF}{\textbf{REF?}}}
\end{itemize}

\subsection{Como escolher a distribuição adequada em um estudo de simulação?}\label{como-escolher-a-distribuiuxe7uxe3o-adequada-em-um-estudo-de-simulauxe7uxe3o}

\begin{itemize}
\item
  A escolha da distribuição deve ser guiada pelo mecanismo gerador dos dados do fenômeno de interesse, e não por conveniência matemática.\textsuperscript{\citeproc{ref-REF}{\textbf{REF?}}}
\item
  Aspectos como o tipo da variável, assimetria, limites naturais e frequência de eventos devem ser considerados na especificação da distribuição.\textsuperscript{\citeproc{ref-REF}{\textbf{REF?}}}
\item
  Em estudos de simulação, é comum avaliar múltiplas distribuições para investigar a sensibilidade dos resultados às suposições do modelo.\textsuperscript{\citeproc{ref-Eglen2017}{42}}
\end{itemize}

\subsection{Como simular dados de diferentes distribuições?}\label{como-simular-dados-de-diferentes-distribuiuxe7uxf5es}

\begin{itemize}
\tightlist
\item
  Use funções específicas para cada distribuição, disponíveis nas bibliotecas estatísticas da linguagem utilizada.\textsuperscript{\citeproc{ref-REF}{\textbf{REF?}}}
\end{itemize}

\begin{infobox}{images/Rlogo}
O pacote \emph{stats}\textsuperscript{\citeproc{ref-stats}{131}} fornece a função \href{https://www.rdocumentation.org/packages/stats/versions/3.5.2/topics/Normal}{\emph{rnorm}} para simular dados de uma distribuição normal.

\end{infobox}

\begin{infobox}{images/Rlogo}
O pacote \emph{stats}\textsuperscript{\citeproc{ref-stats}{131}} fornece a função \href{https://www.rdocumentation.org/packages/stats/versions/3.5.2/topics/rbinom}{\emph{rbinom}} para simular dados de uma distribuição Binomial.

\end{infobox}

\begin{infobox}{images/Rlogo}
O pacote \emph{stats}\textsuperscript{\citeproc{ref-stats}{131}} fornece a função \href{https://www.rdocumentation.org/packages/stats/versions/3.5.2/topics/rpois}{\emph{rpois}} para simular dados de uma distribuição Poisson.

\end{infobox}

\begin{infobox}{images/Rlogo}
O pacote \emph{stats}\textsuperscript{\citeproc{ref-stats}{131}} fornece a função \href{https://www.rdocumentation.org/packages/stats/versions/3.5.2/topics/rexp}{\emph{rexp}} para simular dados de uma distribuição Exponencial.

\end{infobox}

\begin{figure}

{\centering \includegraphics{Ciencia-com-R_files/figure-latex/simulacao-distribuicoes-1} 

}

\caption{Dados simulados a partir de diferentes distribuições: Normal(0,1), Binomial(1,0.4), Poisson(2) e Exponencial(1).}\label{fig:simulacao-distribuicoes}
\end{figure}

\begin{itemize}
\tightlist
\item
  Independentemente da distribuição utilizada, o procedimento de simulação segue os mesmos princípios computacionais, diferindo apenas na função geradora dos números aleatórios.\textsuperscript{\citeproc{ref-REF}{\textbf{REF?}}}
\end{itemize}

\subsection{O que é o método de Monte Carlo?}\label{o-que-uxe9-o-muxe9todo-de-monte-carlo}

\begin{itemize}
\item
  .\textsuperscript{\citeproc{ref-metropolis1949b}{367}}
\item
  No método Markov Chain Monte Carlo (MCMC), o modelo de Markov é usado para gerar amostras de distribuições complexas a partir da simulação de cadeias com distribuição estacionária prescrita.\textsuperscript{\citeproc{ref-huxe4ggstruxf6m2007}{318}}
\end{itemize}

\begin{figure}

{\centering \includegraphics{Ciencia-com-R_files/figure-latex/monte-carlo-distribuicao-1} 

}

\caption{Convergência do histograma para a PDF teórica da Normal(0,1) com o aumento do tamanho amostral (n = 10, 100, 1000, 10000).}\label{fig:monte-carlo-distribuicao}
\end{figure}

\begin{figure}

{\centering \includegraphics{Ciencia-com-R_files/figure-latex/monte-carlo-convergencia-1} 

}

\caption{Convergência da média e do desvio-padrão amostral para os valores teóricos (0 e 1, respectivamente) com o aumento do tamanho amostral (n = 10, 20, 50, 100, 200, 500, 1000, 2000, 5000, 10000).}\label{fig:monte-carlo-convergencia}
\end{figure}

\begin{infobox}{images/Rlogo}
O pacote \emph{base}\textsuperscript{\citeproc{ref-base}{55}} fornece a função \href{https://www.rdocumentation.org/packages/base/versions/3.6.2/topics/Random}{\emph{set.seed}} para especificar uma semente para reprodutibilidade de computações que envolvem números aleatórios.

\end{infobox}

\begin{infobox}{images/Rlogo}
O pacote \emph{simstudy}\textsuperscript{\citeproc{ref-simstudy}{368}} fornece as funções \href{https://www.rdocumentation.org/packages/simstudy/versions/0.7.0/topics/defData}{\emph{defData}} e \href{https://www.rdocumentation.org/packages/simstudy/versions/0.7.0/topics/genData}{\emph{genData}} para criar variáveis e simular um banco de dados de acordo com o delineamento pré-especificado, respectivamente.

\end{infobox}

\begin{infobox}{images/Rlogo}
O pacote \emph{faux}\textsuperscript{\citeproc{ref-faux}{369}} fornece a função \href{https://www.rdocumentation.org/packages/faux/versions/1.2.1/topics/sim_design}{\emph{sim\_design}} para simular um banco de dados de acordo com o delineamento pré-especificado.

\end{infobox}

\begin{infobox}{images/Rlogo}
O pacote \emph{InteractionPoweR}\textsuperscript{\citeproc{ref-InteractionPoweR}{343}} fornece a função \href{https://www.rdocumentation.org/packages/InteractionPoweR/versions/0.2.1/topics/generate_interaction}{\emph{generate\_interaction}} para simular bancos de dads com efeitos de interação.

\end{infobox}

\section{Diretrizes para redação}\label{diretrizes-para-redauxe7uxe3o-2}

\subsection{Quais são as diretrizes para redação de estudos de simulação computacional?}\label{quais-suxe3o-as-diretrizes-para-redauxe7uxe3o-de-estudos-de-simulauxe7uxe3o-computacional}

\begin{itemize}
\item
  Visite a rede \emph{Enhancing the QUAlity and Transparency Of health Research} (\href{https://www.equator-network.org/}{EQUATOR Network}) para encontrar diretrizes específicas.
\item
  \emph{Strengthening the reporting of empirical simulation studies: Introducing the STRESS guidelines}:\textsuperscript{\citeproc{ref-monks2018}{370}} \url{https://www.equator-network.org/reporting-guidelines/strengthening-the-reporting-of-empirical-simulation-studies-introducing-the-stress-guidelines/}
\end{itemize}

\chapter{\texorpdfstring{\textbf{Estudos observacionais}}{Estudos observacionais}}\label{observacional}

\section{Características}\label{caracteruxedsticas-1}

\subsection{Quais são as características de estudos observacionais?}\label{quais-suxe3o-as-caracteruxedsticas-de-estudos-observacionais}

\begin{itemize}
\tightlist
\item
  .\textsuperscript{\citeproc{ref-REF}{\textbf{REF?}}}
\end{itemize}

\section{Diretrizes para redação}\label{diretrizes-para-redauxe7uxe3o-3}

\subsection{Quais são as diretrizes para redação de estudos observacionais?}\label{quais-suxe3o-as-diretrizes-para-redauxe7uxe3o-de-estudos-observacionais}

\begin{itemize}
\item
  Visite a rede \emph{Enhancing the QUAlity and Transparency Of health Research} (\href{https://www.equator-network.org/}{EQUATOR Network}) para encontrar diretrizes específicas.
\item
  \emph{The Strengthening the Reporting of Observational Studies in Epidemiology (STROBE) Statement: guidelines for reporting observational studies}:\textsuperscript{\citeproc{ref-vonelm2007}{371}} \url{https://www.equator-network.org/reporting-guidelines/strobe/}
\end{itemize}

\chapter{\texorpdfstring{\textbf{Propriedades psicométricas}}{Propriedades psicométricas}}\label{propriedades-psicometricas}

\section{Características}\label{caracteruxedsticas-2}

\subsection{O que são propriedades psicométricas?}\label{o-que-suxe3o-propriedades-psicomuxe9tricas}

\begin{itemize}
\tightlist
\item
  .\textsuperscript{\citeproc{ref-REF}{\textbf{REF?}}}
\end{itemize}

\begin{table}
\centering
\caption{\label{tab:crosstable-psicometria}Tabela de confusão sobre propriedades psicométricas de instrumentos.}
\centering
\begin{tabu} to \linewidth {>{}l>{\centering}X>{\centering}X}
\toprule
\textbf{ } & \textbf{Concordância alta} & \textbf{Concordância baixa}\\
\midrule
\textbf{Validade alta} & Adequado & Inadequado\\
\textbf{Validade baixa} & Inadequado & Inadequado\\
\bottomrule
\end{tabu}
\end{table}

\begin{infobox}{images/Rlogo}
O pacote \emph{lavaan}\textsuperscript{\citeproc{ref-lavaan}{372}} fornece a função \href{https://www.rdocumentation.org/packages/lavaan/versions/0.6-16/topics/modificationIndices}{\emph{modificationIndices}} para calcular os índices de modificação.

\end{infobox}

\section{Análise fatorial exploratória}\label{anuxe1lise-fatorial-exploratuxf3ria}

\subsection{O que é análise fatorial exploratória?}\label{o-que-uxe9-anuxe1lise-fatorial-exploratuxf3ria}

\begin{itemize}
\tightlist
\item
  .\textsuperscript{\citeproc{ref-REF}{\textbf{REF?}}}
\end{itemize}

\section{Análise fatorial confirmatória}\label{anuxe1lise-fatorial-confirmatuxf3ria}

\subsection{O que é análise fatorial confirmatória?}\label{o-que-uxe9-anuxe1lise-fatorial-confirmatuxf3ria}

\begin{itemize}
\tightlist
\item
  .\textsuperscript{\citeproc{ref-REF}{\textbf{REF?}}}
\end{itemize}

\begin{infobox}{images/Rlogo}
O pacote \emph{lavaan}\textsuperscript{\citeproc{ref-lavaan}{372}} fornece a função \href{https://www.rdocumentation.org/packages/lavaan/versions/0.6-16/topics/cfa}{\emph{cfa}} para implementar modelos de análise fatorial confirmatória.

\end{infobox}

\section{Validade de conteúdo}\label{validade-de-conteuxfado}

\subsection{O que é validade interna?}\label{o-que-uxe9-validade-interna}

\begin{itemize}
\tightlist
\item
  .\textsuperscript{\citeproc{ref-findley2021}{373}}
\end{itemize}

\subsection{O que é validade externa?}\label{o-que-uxe9-validade-externa}

\begin{itemize}
\tightlist
\item
  .\textsuperscript{\citeproc{ref-findley2021}{373}}
\end{itemize}

\subsection{Que fatores afetam a validade?}\label{que-fatores-afetam-a-validade}

\begin{itemize}
\item
  A amostragem não probabilística pode dificultar a generalização dos achados da amostra para a população, diminuindo assim a validade externa do estudo.\textsuperscript{\citeproc{ref-Banerjee2010}{15}}
\item
  Quando as características da amostra obtida por seleção não probabilística forem similares às da população, a validade externa pode ser maior.\textsuperscript{\citeproc{ref-Banerjee2010}{15}}
\end{itemize}

\subsection{Como avaliar a validade de um estudo?}\label{como-avaliar-a-validade-de-um-estudo}

\begin{itemize}
\tightlist
\item
  As características da amostra apresentadas na Tabela 1 são úteis para interpretação da validade interna e externa dos achados do estudo.\textsuperscript{\citeproc{ref-Westreich2013}{216}}
\end{itemize}

\section{Validade de face}\label{validade-de-face}

\subsection{O que é validade de face?}\label{o-que-uxe9-validade-de-face}

\begin{itemize}
\tightlist
\item
  .{[}RF{]}
\end{itemize}

\section{Validade do construto}\label{validade-do-construto}

\subsection{O que é construto?}\label{o-que-uxe9-construto}

\begin{itemize}
\tightlist
\item
  .{[}RF{]}
\end{itemize}

\section{Validade fatorial}\label{validade-fatorial}

\subsection{O que é validade fatorial?}\label{o-que-uxe9-validade-fatorial}

\begin{itemize}
\tightlist
\item
  .{[}RF{]}
\end{itemize}

\section{Validade convergente}\label{validade-convergente}

\subsection{O que é validade convergente?}\label{o-que-uxe9-validade-convergente}

\begin{itemize}
\tightlist
\item
  .{[}RF{]}
\end{itemize}

\section{Validade discriminante}\label{validade-discriminante}

\subsection{O que é validade discriminante?}\label{o-que-uxe9-validade-discriminante}

\begin{itemize}
\tightlist
\item
  .{[}RF{]}
\end{itemize}

\section{Validade de critério}\label{validade-de-crituxe9rio}

\subsection{O que é validade de critério?}\label{o-que-uxe9-validade-de-crituxe9rio}

\begin{itemize}
\tightlist
\item
  .{[}RF{]}
\end{itemize}

\section{Validade concorrente}\label{validade-concorrente}

\subsection{O que é concorrente?}\label{o-que-uxe9-concorrente}

\begin{itemize}
\tightlist
\item
  .{[}RF{]}
\end{itemize}

\subsection{O que é validade concorrente?}\label{o-que-uxe9-validade-concorrente}

\begin{itemize}
\tightlist
\item
  .{[}RF{]}
\end{itemize}

\subsection{O que é validade preditiva?}\label{o-que-uxe9-validade-preditiva}

\begin{itemize}
\tightlist
\item
  .{[}RF{]}
\end{itemize}

\section{Responsividade}\label{responsividade}

\subsection{O que é responsividade?}\label{o-que-uxe9-responsividade}

\begin{itemize}
\tightlist
\item
  .\textsuperscript{\citeproc{ref-REF}{\textbf{REF?}}}
\end{itemize}

\section{Concordância}\label{concorduxe2ncia}

\subsection{O que é concordância?}\label{o-que-uxe9-concorduxe2ncia}

\begin{itemize}
\tightlist
\item
  .\textsuperscript{\citeproc{ref-REF}{\textbf{REF?}}}
\end{itemize}

\subsection{Quais métodos são adequados para análise de concordância de variáveis dicotômicas?}\label{quais-muxe9todos-suxe3o-adequados-para-anuxe1lise-de-concorduxe2ncia-de-variuxe1veis-dicotuxf4micas}

\begin{itemize}
\tightlist
\item
  Coeficiente de Cohen \(\kappa\): mede a concordância corrigida pelo acaso.\textsuperscript{\citeproc{ref-scott1955}{374},\citeproc{ref-cohen1960}{375}}
\end{itemize}

\begin{table}
\centering
\caption{\label{tab:crosstable-kappa-2x2}Tabela de confusão 2x2 para análise de concordância de testes e variáveis dicotômicas.}
\centering
\begin{tabu} to \linewidth {>{}l>{\centering}X>{\centering}X>{\centering}X}
\toprule
\textbf{ } & \textbf{Teste positivo} & \textbf{Teste negativo} & \textbf{Total}\\
\midrule
\textbf{Teste positivo} & $a$ & $b$ & $g=a+b$\\
\textbf{Teste negativo} & $c$ & $d$ & $h=c+d$\\
\textbf{Total} & $e=a+c$ & $f=b+d$ & $N=a+b+c+d$\\
\bottomrule
\end{tabu}
\end{table}

\begin{infobox}{images/Rlogo}
O pacote \emph{irr}\textsuperscript{\citeproc{ref-irr}{376}} fornece a função \href{https://www.rdocumentation.org/packages/irr/versions/0.84.1/topics/kappa2}{\emph{kappa2}} para estimar a confiabilidade utilizando coeficiente \(\kappa\) de Cohen para 2 examinadores.

\end{infobox}

\begin{infobox}{images/Rlogo}
O pacote \emph{irr}\textsuperscript{\citeproc{ref-irr}{376}} fornece a função \href{https://www.rdocumentation.org/packages/irr/versions/0.84.1/topics/kappam.fleiss}{\emph{kappam.fleiss}} para estimar a confiabilidade utilizando coeficiente \(\kappa\) de Fleiss para mais de 2 examinadores.

\end{infobox}

\begin{infobox}{images/Rlogo}
O pacote \emph{irr}\textsuperscript{\citeproc{ref-irr}{376}} fornece a função \href{https://www.rdocumentation.org/packages/irr/versions/0.84.1/topics/kappam.light}{\emph{kappam.light}} para estimar a confiabilidade utilizando coeficiente \(\kappa\) de Light para mais de 2 examinadores em dados categóricos.

\end{infobox}

\begin{itemize}
\tightlist
\item
  Coeficiente de correlação tetracórica \(r_{tet}\).\textsuperscript{\citeproc{ref-i.mathe1901}{377},\citeproc{ref-banerjee1999}{378}}
\end{itemize}

\begin{infobox}{images/Rlogo}
O pacote \emph{psych}\textsuperscript{\citeproc{ref-psych}{379}} fornece a função \href{https://www.rdocumentation.org/packages/psych/versions/2.3.6/topics/tetrachoric}{\emph{tetrachoric}} para calcular o coeficiente de correlação tetracórica (\(r_{tet}\)).

\end{infobox}

\subsection{Quais métodos não são adequados para análise de concordância de variáveis dicotômicas?}\label{quais-muxe9todos-nuxe3o-suxe3o-adequados-para-anuxe1lise-de-concorduxe2ncia-de-variuxe1veis-dicotuxf4micas}

\begin{itemize}
\item
  Concordância absoluta \(C_{A}\) --- quantidade de casos em que examinadores concordam --- não é recomendada porque não corrige a estimativa para possíveis concordâncias ao acaso.\textsuperscript{\citeproc{ref-banerjee1999}{378}}
\item
  Concordância percentual \(C_{\%}\) --- proporção de casos em que examinadores concordam pela quantidade total de casos --- não é recomendada porque não corrige a estimativa para possíveis concordâncias ao acaso.\textsuperscript{\citeproc{ref-banerjee1999}{378}}
\item
  Qui-quadrado \(\chi^2\) a partir da tabela de contigência não é recomendado porque tal teste analisa associação.\textsuperscript{\citeproc{ref-banerjee1999}{378}}
\item
  A família de coeficientes de Cohen \(\kappa\) não é adequada para analisar concordância quando as variáveis são aparentemente (e não originalmente) dicotômicas.\textsuperscript{\citeproc{ref-banerjee1999}{378}}
\end{itemize}

\begin{infobox}{images/Rlogo}
O pacote \emph{irr}\textsuperscript{\citeproc{ref-irr}{376}} fornece a função \href{https://www.rdocumentation.org/packages/irr/versions/0.84.1/topics/agree}{\emph{agree}} para estimar a concordância percentual entre examinadores.

\end{infobox}

\subsection{Quais métodos são adequados para análise de concordância de variáveis categóricas?}\label{quais-muxe9todos-suxe3o-adequados-para-anuxe1lise-de-concorduxe2ncia-de-variuxe1veis-categuxf3ricas}

\begin{itemize}
\item
  Coeficiente de Cohen \(\kappa\): mede a concordância corrigida pelo acaso.\textsuperscript{\citeproc{ref-scott1955}{374},\citeproc{ref-cohen1960}{375}}
\item
  Coeficiente de Cohen ponderado \(\kappa_{w}\): mede a concordância corrigida pelo acaso.\textsuperscript{\citeproc{ref-scott1955}{374},\citeproc{ref-cohen1960}{375}}
\end{itemize}

\begin{table}
\centering
\caption{\label{tab:crosstable-kappa-3x3}Tabela de confusão 3x3 para análise de concordância de testes e variáveis dicotômicas.}
\centering
\begin{tabu} to \linewidth {>{}l>{\centering}X>{\centering}X>{\centering}X>{\centering}X}
\toprule
\textbf{ } & \textbf{Grave} & \textbf{Moderado} & \textbf{Leve} & \textbf{Total}\\
\midrule
\textbf{Grave} & $a$ & $b$ & $c$ & $j=a+b+c$\\
\textbf{Moderado} & $d$ & $e$ & $f$ & $k=d+e+f$\\
\textbf{Leve} & $g$ & $h$ & $i$ & $l=g+h+i$\\
\textbf{Total} & $j=a+d+g$ & $k=b+e+h$ & $l=c+f+i$ & $N=a+b+c+d+e+f+g+h+i$\\
\bottomrule
\end{tabu}
\end{table}

\begin{itemize}
\tightlist
\item
  Coeficiente de correlação policórica \(r_{pol}\).\textsuperscript{\citeproc{ref-banerjee1999}{378}}
\end{itemize}

\begin{infobox}{images/Rlogo}
O pacote \emph{psych}\textsuperscript{\citeproc{ref-psych}{379}} fornece a função \href{https://www.rdocumentation.org/packages/psych/versions/2.3.6/topics/tetrachoric}{\emph{tetrachoric}} para calcular o coeficiente de correlação policórica (\(r_{pol}\)).

\end{infobox}

\subsection{Quais métodos são adequados para análise de concordância de variáveis categóricas e contínuas?}\label{quais-muxe9todos-suxe3o-adequados-para-anuxe1lise-de-concorduxe2ncia-de-variuxe1veis-categuxf3ricas-e-contuxednuas}

\begin{itemize}
\tightlist
\item
  Coeficiente de correlação bisserial \(r_{s}\).\textsuperscript{\citeproc{ref-banerjee1999}{378}}
\end{itemize}

\begin{infobox}{images/Rlogo}
O pacote \emph{psych}\textsuperscript{\citeproc{ref-psych}{379}} fornece a função \href{https://www.rdocumentation.org/packages/psych/versions/2.3.6/topics/tetrachoric}{\emph{tetrachoric}} para calcular o coeficiente de correlação bisserial (\(r_{s}\)).

\end{infobox}

\subsection{Quais métodos são adequados para análise de concordância de variáveis ordinais?}\label{quais-muxe9todos-suxe3o-adequados-para-anuxe1lise-de-concorduxe2ncia-de-variuxe1veis-ordinais}

\begin{itemize}
\tightlist
\item
  Coeficiente de Cohen ponderado \(\kappa_{w}\): mede a concordância corrigida pelo acaso.\textsuperscript{\citeproc{ref-scott1955}{374},\citeproc{ref-cohen1960}{375}}
\end{itemize}

\subsection{Quais métodos são adequados para análise de concordância de variáveis contínuas?}\label{quais-muxe9todos-suxe3o-adequados-para-anuxe1lise-de-concorduxe2ncia-de-variuxe1veis-contuxednuas}

\begin{itemize}
\item
  Gráfico de dispersão com a reta de regressão.\textsuperscript{\citeproc{ref-altman1983}{143}}
\item
  Gráfico de limites de concordância (média dos testes vs.~diferença entre testes) com a reta de regressão do viés e respectivo no nível de significância \(\alpha\) pré-estabelecido.\textsuperscript{\citeproc{ref-altman1983}{143}}
\end{itemize}

\begin{infobox}{images/Rlogo}
O pacote \emph{BlandAltmanLeh}\textsuperscript{\citeproc{ref-BlandAltmanLeh}{380}} fornece as funções \href{https://www.rdocumentation.org/packages/BlandAltmanLeh/versions/0.3.1/topics/bland.altman.stats}{\emph{bland.altman.stats}} e \href{https://www.rdocumentation.org/packages/BlandAltmanLeh/versions/0.3.1/topics/bland.altman.plot}{\emph{bland.altman.plot}} para calcular e apresentar, respectivamente, o gráfico com os limites de concordância entre dois métodos.

\end{infobox}

\subsection{Quais métodos não são adequados para análise de concordância de variáveis contínuas?}\label{quais-muxe9todos-nuxe3o-suxe3o-adequados-para-anuxe1lise-de-concorduxe2ncia-de-variuxe1veis-contuxednuas}

\begin{itemize}
\item
  Comparação de médias: dois métodos apresentarem médias similares --- isto é, `sem diferença estatística' após um teste inferencial de hipótese nula \(H_{0}:\mu_{1} = \mu_{2}\) --- não informa sobre a concordância deles. Métodos com maior erro de medida tendem a ter menos chance de rejeição da hipótese nula.\textsuperscript{\citeproc{ref-altman1983}{143}}
\item
  Correlação bivariada: o coeficiente de correlação dependente tanto da variação entre indivíduos (isto é, entre os valores verdadeiros) quanto da variação intraindividual (isto é, erro de medida). Se a variância dos erros de medida de ambos os métodos não for pequena comparadas à variância dos valores verdadeiros, o tamanho do efeito da correlação será pequeno mesmo que os métodos possuam boa concordância.\textsuperscript{\citeproc{ref-altman1983}{143}}
\item
  Regressão linear: o teste da hipótese nula da inclinação da reta de regressão (\(H_{0}:\beta = 0\)) é equivalente a testar a correlação bivariada (\(H_{0}:\rho = 0\)).\textsuperscript{\citeproc{ref-altman1983}{143}}
\end{itemize}

\subsection{Quais métodos são adequados para modelagem de concordância?}\label{quais-muxe9todos-suxe3o-adequados-para-modelagem-de-concorduxe2ncia}

\begin{itemize}
\tightlist
\item
  Modelo log-linear.\textsuperscript{\citeproc{ref-banerjee1999}{378}}
\end{itemize}

\section{Confiabilidade}\label{confiabilidade}

\subsection{O que é confiabilidade?}\label{o-que-uxe9-confiabilidade}

\begin{itemize}
\tightlist
\item
  .\textsuperscript{\citeproc{ref-REF}{\textbf{REF?}}}
\end{itemize}

\subsection{Quais métodos são adequados para análise de confiabilidade?}\label{quais-muxe9todos-suxe3o-adequados-para-anuxe1lise-de-confiabilidade}

\begin{itemize}
\tightlist
\item
  .\textsuperscript{\citeproc{ref-REF}{\textbf{REF?}}}
\end{itemize}

\begin{infobox}{images/Rlogo}
O pacote \emph{semTools}\textsuperscript{\citeproc{ref-semTools}{381}} fornece a função \href{https://www.rdocumentation.org/packages/semTools/versions/0.5-6/topics/reliability-deprecated}{\emph{reliability}} para analisar a confiabilidade de um instrumento.

\end{infobox}

\begin{infobox}{images/Rlogo}
O pacote \emph{psych}\textsuperscript{\citeproc{ref-psych}{379}} fornece a função \href{https://www.rdocumentation.org/packages/psych/versions/2.3.6/topics/ICC}{\emph{icc}} para estimar a confiabilidade utilizando coeficientes de correlação intraclasse.

\end{infobox}

\begin{infobox}{images/Rlogo}
O pacote \emph{multilevel}\textsuperscript{\citeproc{ref-jomo}{382}} fornece a função \href{https://www.rdocumentation.org/packages/multilevel/versions/2.5/topics/mult.icc}{\emph{mult.icc}} para estimar a confiabilidade utlizando diversos coeficientes de correlação intraclasse.

\end{infobox}

\begin{infobox}{images/Rlogo}
O pacote \emph{multilevel}\textsuperscript{\citeproc{ref-jomo}{382}} fornece a função \href{https://www.rdocumentation.org/packages/multilevel/versions/2.5/topics/cronbach}{\emph{cronbach}} para estimar a confiabilidade utlizando o \(\alpha\) de Cronbach.

\end{infobox}

\begin{infobox}{images/Rlogo}
O pacote \emph{irr}\textsuperscript{\citeproc{ref-irr}{376}} fornece a função \href{https://www.rdocumentation.org/packages/irr/versions/0.84.1/topics/kripp.alpha}{\emph{kripp.alpha}} para estimar a confiabilidade utilizando coeficiente \(\alpha\) de Krippendorff.

\end{infobox}

\begin{infobox}{images/Rlogo}
O pacote \emph{irr}\textsuperscript{\citeproc{ref-irr}{376}} fornece a função \href{https://www.rdocumentation.org/packages/irr/versions/0.84.1/topics/iota}{\emph{iota}} para estimar a confiabilidade utilizando coeficiente \(\iota\).

\end{infobox}

\section{Diretrizes para redação}\label{diretrizes-para-redauxe7uxe3o-4}

\subsection{Quais são as diretrizes para redação de estudos de propriedades psicométricas?}\label{quais-suxe3o-as-diretrizes-para-redauxe7uxe3o-de-estudos-de-propriedades-psicomuxe9tricas}

\begin{itemize}
\item
  Visite a rede \emph{Enhancing the QUAlity and Transparency Of health Research} (\href{https://www.equator-network.org/}{EQUATOR Network}) para encontrar diretrizes específicas.
\item
  \emph{COSMIN reporting guideline for studies on measurement properties of patient-reported outcome measures}:\textsuperscript{\citeproc{ref-gagnier2021}{383}} \url{https://www.equator-network.org/reporting-guidelines/cosmin-reporting-guideline-for-studies-on-measurement-properties-of-patient-reported-outcome-measures/}
\item
  \emph{Recommendations for reporting the results of studies of instrument and scale development and testing}:\textsuperscript{\citeproc{ref-streiner2014}{384}} \url{https://www.equator-network.org/reporting-guidelines/recommendations-for-reporting-the-results-of-studies-of-instrument-and-scale-development-and-testing/}
\item
  \emph{Guidelines for reporting reliability and agreement studies (GRRAS) were proposed}:\textsuperscript{\citeproc{ref-kottner2011}{385}} \url{https://www.equator-network.org/reporting-guidelines/guidelines-for-reporting-reliability-and-agreement-studies-grras-were-proposed/}\textgreater{}
\end{itemize}

\chapter{\texorpdfstring{\textbf{Desempenho diagnóstico}}{Desempenho diagnóstico}}\label{desempenho-diagnostico}

\section{Características}\label{caracteruxedsticas-3}

\subsection{Quais são as características de estudos de desempenho diagnóstico?}\label{quais-suxe3o-as-caracteruxedsticas-de-estudos-de-desempenho-diagnuxf3stico}

\begin{itemize}
\tightlist
\item
  .\textsuperscript{\citeproc{ref-REF}{\textbf{REF?}}}
\end{itemize}

\section{Tabelas 2x2}\label{tabelas-2x2}

\subsection{O que é uma tabela de confusão 2x2?}\label{o-que-uxe9-uma-tabela-de-confusuxe3o-2x2}

\begin{itemize}
\tightlist
\item
  Tabela de confusão é uma matriz de 2 linhas por 2 colunas que permite analisar o desempenho de classificação de uma variável dicotômica (padrão-ouro ou referência) versus outra variável dicotômica (novo teste).\textsuperscript{\citeproc{ref-steckelberg2004}{386}}
\end{itemize}

\subsection{Como analisar o desempenho diagnóstico em tabelas 2x2?}\label{como-analisar-o-desempenho-diagnuxf3stico-em-tabelas-2x2}

\begin{itemize}
\item
  Verdadeiro-positivo (\(VP\)): caso com a condição presente e corretamente identificado como tal.\textsuperscript{\citeproc{ref-greenhalgh1997b}{387}}
\item
  Falso-negativo (\(FN\)): caso com a condição presente e erroneamente identificado como ausente.\textsuperscript{\citeproc{ref-greenhalgh1997b}{387}}
\item
  Verdadeiro-negativo (\(VN\)): controle sem a condição presente e corretamente identificados como tal.\textsuperscript{\citeproc{ref-greenhalgh1997b}{387}}
\item
  Falso-positivo (\(FP\)): controle sem a condição presente e erroneamente identificado como presente.\textsuperscript{\citeproc{ref-greenhalgh1997b}{387}}
\end{itemize}

\begin{table}
\centering
\caption{\label{tab:crosstable-2x2}Tabela de confusão 2x2 para análise de desempenho diagnóstico de testes e variáveis dicotômicas.}
\centering
\begin{tabu} to \linewidth {>{}l>{\centering}X>{\centering}X>{\centering}X}
\toprule
\textbf{ } & \textbf{Condição presente} & \textbf{Condição ausente} & \textbf{Total}\\
\midrule
\textbf{Teste positivo} & $VP$ & $FP$ & $VP+FP$\\
\textbf{\textbf{Teste negativo}} & \textbf{$FN$} & \textbf{$VN$} & \textbf{$FN+VN$}\\
\textbf{Total} & $VP+FN$ & $FP+VN$ & $N=VP+VN+FP+FN$\\
\bottomrule
\end{tabu}
\end{table}

\begin{itemize}
\tightlist
\item
  Tabelas de confusão também podem ser visualizadas em formato de árvores de frequência.\textsuperscript{\citeproc{ref-steckelberg2004}{386}}
\end{itemize}

\begin{figure}

{\centering \includegraphics{Ciencia-com-R_files/figure-latex/frequency-tree-1} 

}

\caption{Árvore de frequência do desempenho diagnóstico de uma tabela de confusão 2x2 representando um método novo (dicotômico) comparado ao método padrão-ouro ou referência (dicotômico).}\label{fig:frequency-tree}
\end{figure}

\begin{infobox}{images/Rlogo}
O pacote \emph{riskyr}\textsuperscript{\citeproc{ref-riskyr}{388}} fornece a função \href{https://www.rdocumentation.org/packages/riskyr/versions/0.4.0/topics/plot_prism}{\emph{plot\_prism}} para construir árvores de frequência a partir de diferentes cenários.

\end{infobox}

\subsection{Quais probabilidades caracterizam o desempenho diagnóstico de um teste em tabelas 2x2?}\label{quais-probabilidades-caracterizam-o-desempenho-diagnuxf3stico-de-um-teste-em-tabelas-2x2}

\begin{itemize}
\tightlist
\item
  Sensibilidade (\(SEN\)) \eqref{eq:sen}: Proporção de verdadeiro-positivos dentre aqueles com a condição.\textsuperscript{\citeproc{ref-greenhalgh1997b}{387}}
\end{itemize}

\begin{equation}
\label{eq:sen}
SEN = \dfrac{VP}{VP+FN}
\end{equation}

\begin{itemize}
\tightlist
\item
  Especificidade (\(ESP\)) \eqref{eq:esp}: Proporção de verdadeiro-negativos dentre aqueles sem a condição.\textsuperscript{\citeproc{ref-greenhalgh1997b}{387}}
\end{itemize}

\begin{equation}
\label{eq:esp}
ESP = \dfrac{VN}{VN+FP}
\end{equation}

\begin{figure}

{\centering \includegraphics{Ciencia-com-R_files/figure-latex/sensibilidade-especificidade-1} 

}

\caption{Trade-off entre sensibilidade e especificidade em função do limiar de probabilidade (t) para um modelo de classificação.}\label{fig:sensibilidade-especificidade}
\end{figure}

\begin{itemize}
\tightlist
\item
  Valor preditivo positivo (\(VPP\)) \eqref{eq:vpp}: Proporção de casos corretamente identificados como verdadeiro-positivos.\textsuperscript{\citeproc{ref-greenhalgh1997b}{387}}
\end{itemize}

\begin{equation}
\label{eq:vpp}
VPP = \dfrac{VP}{VP+FP}
\end{equation}

\begin{itemize}
\tightlist
\item
  Valor preditivo negativo (\(VPN\)) \eqref{eq:vpn}: Proporção de controles corretamente identificados como verdadeiro-negativos.\textsuperscript{\citeproc{ref-greenhalgh1997b}{387}}
\end{itemize}

\begin{equation}
\label{eq:vpn}
VPN = \dfrac{VN}{VN+FN}
\end{equation}

\begin{itemize}
\tightlist
\item
  Razão de verossimilhança positiva (\(LR+\)) \eqref{eq:lrplus}: Quantifica o quanto a probabilidade de a condição estar presente aumenta quando o teste é positivo.\textsuperscript{\citeproc{ref-REF}{\textbf{REF?}}}
\end{itemize}

\begin{equation}
\label{eq:lrplus}
LR+ = \dfrac{SEN}{1 - ESP} = \dfrac{VP/(VP+FN)}{FP/(FP+VN)}
\end{equation}

\begin{itemize}
\tightlist
\item
  Razão de verossimilhança negativa (\(LR-\)) \eqref{eq:lrminus}: Quantifica o quanto a probabilidade de a condição estar presente diminui quando o teste é negativo.\textsuperscript{\citeproc{ref-REF}{\textbf{REF?}}}
\end{itemize}

\begin{equation}
\label{eq:lrminus}
LR- = \dfrac{1 - SEN}{ESP} = \dfrac{FN/(VP+FN)}{VN/(FP+VN)}
\end{equation}

\begin{itemize}
\tightlist
\item
  Acurácia (\(ACU\)), \eqref{eq:acu}: Proporção de casos e controles corretamente identificados.\textsuperscript{\citeproc{ref-greenhalgh1997b}{387}}
\end{itemize}

\begin{equation}
\label{eq:acu}
ACU = \dfrac{VP+VN}{VP+VN+FP+FN}
\end{equation}

\begin{itemize}
\tightlist
\item
  Razão de chances diagnóstica (\(DOR\)) \eqref{eq:dor1}, \eqref{eq:dor2} e \eqref{eq:dor3}: Razão entre a chance de um teste ser positivo quando a condição está presente e a chance de um teste ser positivo quando a condição está ausente.\textsuperscript{\citeproc{ref-Glas2003}{389}}
\end{itemize}

\begin{equation}
\label{eq:dor1}
DOR = \dfrac{VP}{FN} \div \dfrac{FP}{VN} = \dfrac{VP \cdot VN}{FP \cdot FN}
\end{equation}

\begin{equation}
\label{eq:dor2}
DOR = \dfrac{SEN/(1-SEN)}{(1-ESP)/ESP} = \dfrac{SEN \cdot ESP}{(1-SEN) \cdot (1-ESP)}
\end{equation}

\begin{equation}
\label{eq:dor3}
DOR = \dfrac{LR+}{LR-}
\end{equation}

\begin{table}
\centering
\caption{\label{tab:crosstable-prob}Probabilidades calculados a partir da tabela de confusão 2x2 para análise de desempenho diagnóstico de testes e variáveis dicotômicas.}
\centering
\begin{tabu} to \linewidth {>{}l>{\centering}X>{\centering}X>{\centering}X>{\centering}X}
\toprule
\textbf{ } & \textbf{Condição presente} & \textbf{Condição ausente} & \textbf{Total} & \textbf{Probabilidades}\\
\midrule
\textbf{Teste positivo} & $VP$ & $FP$ & $VP+FP$ & $VPP = \frac{VP}{VP+FP}$\\
\textbf{Teste negativo} & $FN$ & $VN$ & $FN+VN$ & $VPN = \frac{VN}{VN+FN}$\\
\textbf{\textbf{Total}} & \textbf{$VP+FN$} & \textbf{$FP+VN$} & \textbf{$N=VP+VN+FP+FN$} & \textbf{}\\
\textbf{Probabilidades} & $SEN = \frac{VP}{VP+FN}$ & $ESP = \frac{VN}{VN+FP}$ &  & $ACU = \frac{VP+VN}{VP+VN+FP+FN}$ \ $DOR = \frac{VP \cdot VN}{FP \cdot FN}$\\
\bottomrule
\end{tabu}
\end{table}

\begin{infobox}{images/Rlogo}
O pacote \emph{riskyr}\textsuperscript{\citeproc{ref-riskyr}{388}} fornece a função \href{https://www.rdocumentation.org/packages/riskyr/versions/0.4.0/topics/comp_prob}{\emph{comp\_prob}} para estimar 13 probabilidades relacionadas ao desempenho diagnóstico em tabelas 2x2.

\end{infobox}

\begin{infobox}{images/Rlogo}
O pacote \emph{caret}\textsuperscript{\citeproc{ref-caret}{390}} fornece a função \href{https://www.rdocumentation.org/packages/caret/versions/3.45/topics/confusionMatrix}{\emph{confusionMatrix}} para estimar 11 probabilidades relacionadas ao desempenho diagnóstico em tabelas 2x2.

\end{infobox}

\section{Tabelas 2x3}\label{tabelas-2x3}

\subsection{O que é uma tabela de confusão 2x3?}\label{o-que-uxe9-uma-tabela-de-confusuxe3o-2x3}

\begin{itemize}
\item
  É a extensão da tabela 2×2 que inclui uma terceira decisão (deferimento/\emph{boundary}) além de aceitar (positivo) e rejeitar (negativo).\textsuperscript{\citeproc{ref-xu2020}{391}}
\item
  As colunas** representam as decisões** (\(POS\), \(BND\), \(NEG\)) e as linhas representam a verdade de referência (condição presente vs ausente).\textsuperscript{\citeproc{ref-xu2020}{391}}
\item
  Essa formulação vem do arcabouço de \emph{Three-Way Decisions (3WD)}, que particiona o universo em três regiões por dois limiares \(\alpha\) e \(\beta\).\textsuperscript{\citeproc{ref-xu2020}{391}}
\end{itemize}

\subsection{Como as regiões POS, BND e NEG são definidas?}\label{como-as-regiuxf5es-pos-bnd-e-neg-suxe3o-definidas}

\begin{itemize}
\tightlist
\item
  Dado um escore ou probabilidade condicional \(Pr(C\mid[x])\) para a classe \(C\), classifica-se como \(POS\) (aceitar) quando \(Pr(C\mid[x]) \ge \alpha\), como \(BND\) (deferir) quando \(\beta < Pr(C\mid[x]) < \alpha\) e como \(NEG\) (rejeitar) quando \(Pr(C\mid[x]) \le \beta\), sendo que os limiares \((\alpha,\beta)\) determinam simultaneamente as três regiões e os \emph{trade-offs} entre acurácia e comprometimento.\textsuperscript{\citeproc{ref-xu2020}{391}}
\end{itemize}

\subsection{Qual é o formato de uma tabela 2×3?}\label{qual-uxe9-o-formato-de-uma-tabela-23}

\begin{itemize}
\tightlist
\item
  Estrutura geral (linhas = condição real; colunas = decisão):
\end{itemize}

\begin{table}
\centering
\caption{\label{tab:crosstable-2x3}Tabela de confusão 3-vias (2×3) com totais: referência vs decisão (3WD).}
\centering
\begin{tabu} to \linewidth {>{}l>{\centering}X>{\centering}X>{\centering}X>{\centering}X}
\toprule
\textbf{ } & \textbf{POS (aceitar)} & \textbf{BND (deferir)} & \textbf{NEG (rejeitar)} & \textbf{Total}\\
\midrule
\textbf{Condição presente (C)} & $|POS\cap C|$ & $|BND\cap C|$ & $|NEG\cap C|$ & $|POS\cap C|+|BND\cap C|+|NEG\cap C|$\\
\textbf{\textbf{Condição ausente ($C^c$)}} & \textbf{$|POS\cap C^c|$} & \textbf{$|BND\cap C^c|$} & \textbf{$|NEG\cap C^c|$} & \textbf{$|POS\cap C^c|+|BND\cap C^c|+|NEG\cap C^c|$}\\
\textbf{Total} & $|POS\cap C|+|POS\cap C^c|$ & $|BND\cap C|+|BND\cap C^c|$ & $|NEG\cap C|+|NEG\cap C^c|$ & $N$\\
\bottomrule
\end{tabu}
\end{table}

\subsection{Quais são as medidas básicas na 2×3?}\label{quais-suxe3o-as-medidas-buxe1sicas-na-23}

\begin{itemize}
\tightlist
\item
  Acurácia em POS (\(M_{PT}\)), equação \eqref{eq:mpt}: Proporção de positivos corretamente identificados na região POS.\textsuperscript{\citeproc{ref-xu2020}{391}}
\end{itemize}

\begin{equation}
\label{eq:mpt}
M_{PT} = \dfrac{|POS \cap C|}{|POS|}
\end{equation}

\begin{itemize}
\tightlist
\item
  Erro em POS (\(M_{PF}\)), equação \eqref{eq:mpf}: Proporção de negativos incorretamente classificados na região POS.\textsuperscript{\citeproc{ref-xu2020}{391}}
\end{itemize}

\begin{equation}
\label{eq:mpf}
M_{PF} = \dfrac{|POS \cap C^{c}|}{|POS|}
\end{equation}

\begin{itemize}
\tightlist
\item
  Acurácia em NEG (\(M_{NF}\)), equação \eqref{eq:mnf}: Proporção de negativos corretamente identificados na região NEG.\textsuperscript{\citeproc{ref-xu2020}{391}}
\end{itemize}

\begin{equation}
\label{eq:mnf}
M_{NF} = \dfrac{|NEG \cap C^{c}|}{|NEG|}
\end{equation}

\begin{itemize}
\tightlist
\item
  Erro em NEG (\(M_{NT}\)), equação \eqref{eq:mnt}: Proporção de positivos incorretamente classificados na região NEG.\textsuperscript{\citeproc{ref-xu2020}{391}}
\end{itemize}

\begin{equation}
\label{eq:mnt}
M_{NT} = \dfrac{|NEG \cap C|}{|NEG|}
\end{equation}

\begin{itemize}
\tightlist
\item
  Frações em BND (\(M_{BT}\) e \(M_{BF}\)), equações \eqref{eq:mbt} e \eqref{eq:mbf}: Proporção de deferimentos verdadeiros e falsos.\textsuperscript{\citeproc{ref-xu2020}{391}}
\end{itemize}

\begin{equation}
\label{eq:mbt}
M_{BT} = \dfrac{|BND \cap C|}{|BND|}
\end{equation}
\begin{equation}
\label{eq:mbf}
M_{BF} = \dfrac{|BND \cap C^{c}|}{|BND|}
\end{equation}

\subsection{\texorpdfstring{Como escolher os limiares \(\alpha\) e \(\beta\)?}{Como escolher os limiares \textbackslash alpha e \textbackslash beta?}}\label{como-escolher-os-limiares-alpha-e-beta}

\begin{itemize}
\tightlist
\item
  Os limiares \((\alpha,\beta)\) controlam o tamanho das regiões \(POS\), \(NEG\) e \(BND\) e, portanto, os \emph{trade-offs} entre ``acertar mais'' (acurácia nas regiões) e ``decidir mais'' (comprometimento; menos deferimentos).\textsuperscript{\citeproc{ref-xu2020}{391}}
\end{itemize}

\subsection{Quando preferir 3-vias em vez de 2×2?}\label{quando-preferir-3-vias-em-vez-de-22}

\begin{itemize}
\item
  Quando o custo de erro é assimétrico e/ou há incerteza relevante.\textsuperscript{\citeproc{ref-xu2020}{391}}
\item
  O deferimento (\(BND\)) evita decisões precipitadas e permite avaliação adicional, equilibrando acurácia e cobertura.\textsuperscript{\citeproc{ref-xu2020}{391}}
\item
  É particularmente útil em triagens diagnósticas com etapas confirmatórias.\textsuperscript{\citeproc{ref-xu2020}{391}}
\end{itemize}

\section{Curvas ROC}\label{curvas-roc}

\subsection{O que representa a curva ROC?}\label{o-que-representa-a-curva-roc}

\begin{itemize}
\item
  A relação entre sensibilidade (\(SEN\)) no eixo vertical e \(1 - ESP\) no eixo horizontal.\textsuperscript{\citeproc{ref-he2024}{392}}
\item
  Cada ponto na curva corresponde a um ponto de corte possível do teste.\textsuperscript{\citeproc{ref-he2024}{392}}
\end{itemize}

\subsection{Quais são os tipos de curva ROC?}\label{quais-suxe3o-os-tipos-de-curva-roc}

\begin{itemize}
\item
  Curva empírica: conecta diretamente os pontos obtidos a partir dos diferentes pontos de corte observados.\textsuperscript{\citeproc{ref-park2004}{393}}
\item
  Curva suavizada (paramétrica): assume uma distribuição binormal e gera uma curva ajustada por máxima verossimilhança.\textsuperscript{\citeproc{ref-park2004}{393}}
\end{itemize}

\subsection{Como definir o melhor ponto de corte?}\label{como-definir-o-melhor-ponto-de-corte}

\begin{itemize}
\item
  O ponto de corte em uma curva ROC representa um balanço entre sensibilidade e especificidade, ou seja, a taxa de verdadeiros positivos e a taxa de falsos positivos.\textsuperscript{\citeproc{ref-he2024}{392},\citeproc{ref-park2004}{393}}
\item
  O método de Youden (equação \eqref{eq:youden} maximiza a diferença entre a taxa de verdadeiros positivos e a taxa de falsos positivos. O ponto de corte ideal será aquele com maior valor de \(Y\).\textsuperscript{\citeproc{ref-YOUDEN1950}{126}}
\end{itemize}

\begin{equation}
\label{eq:youden}
Y = SEN + ESP - 1
\end{equation}

\begin{itemize}
\tightlist
\item
  O método da distância Euclidiana (\eqref{eq:euclidean} minimiza a distância entre um ponto da curva ROC e o ponto (0,1), que representa sensibilidade perfeita (\(SEN = 100%
  \)) e especificidade perfeita (\(ESP = 100%
  \)). O ponto de corte ideal será aquele com menor valor de \(D\).\textsuperscript{\citeproc{ref-yarnold2014}{394}}
\end{itemize}

\begin{equation}
\label{eq:euclidean}
D = \sqrt{(1 - SEN)^2 + (1 - ESP)^2}
\end{equation}

\subsection{O que é a área sob a curva (AUROC)?}\label{o-que-uxe9-a-uxe1rea-sob-a-curva-auroc}

\begin{itemize}
\item
  A área sob a curva ROC (AUC ou AUROC) quantifica o poder de discriminação ou desempenho diagnóstico na classificação de uma variável dicotômica.\textsuperscript{\citeproc{ref-de2022}{395}}
\item
  A área sob a curva (\(AUC\)) resume o desempenho global e representa a probabilidade de o teste classificar corretamente um caso positivo selecionado aleatoriamente em relação a um caso negativo selecionado aleatoriamente.\textsuperscript{\citeproc{ref-he2024}{392}}
\end{itemize}

\subsection{Como calcular a AUC?}\label{como-calcular-a-auc}

\begin{itemize}
\tightlist
\item
  Método não paramétrico: soma das áreas trapezoidais sob a curva empírica \eqref{eq:auc}. Pode subestimar AUC quando os dados são discretos.\textsuperscript{\citeproc{ref-park2004}{393}}
\end{itemize}

\begin{equation}
\label{eq:auc}
AUC = \sum_{i=1}^{n-1} (x_{i+1} - x_i) \cdot \dfrac{y_i + y_{i+1}}{2}
\end{equation}

\begin{itemize}
\tightlist
\item
  Método paramétrico (binormal): mais robusto para dados em escala ordinal, com viés reduzido \eqref{eq:auc-binormal}, onde\(\Phi\) é a função de distribuição acumulada da Normal padrão, \(\mu_1\) e \(\mu_0\) são as médias dos escores para os grupos positivo e negativo, respectivamente, e \(\sigma_1^2\) e \(\sigma_0^2\) são as variâncias dos escores para os grupos positivo e negativo, respectivamente.\textsuperscript{\citeproc{ref-park2004}{393}}
\end{itemize}

\begin{equation}
\label{eq:auc-binormal}
AUC = \Phi\left(\dfrac{\mu_1 - \mu_0}{\sqrt{\sigma_1^2 + \sigma_0^2}}\right)
\end{equation}

\begin{itemize}
\tightlist
\item
  AUC deve sempre vir acompanhada de intervalo de confiança (IC95\%).\textsuperscript{\citeproc{ref-park2004}{393}}
\end{itemize}

\begin{infobox}{images/Rlogo}
O pacote \emph{proc}\textsuperscript{\citeproc{ref-pROC}{396}} fornece a função \href{https://www.rdocumentation.org/packages/pROC/versions/1.18.4/topics/plot.roc}{\emph{plot.roc}} para criar uma curva ROC.

\end{infobox}

\subsection{Como interpretar a área sob a curva (ROC)?}\label{como-interpretar-a-uxe1rea-sob-a-curva-roc}

\begin{itemize}
\item
  A área sob a curva AUC varia no intervalo \([0.5; 1]\), com valores mais elevados indicando melhor discriminação ou desempenho do modelo de classificação.\textsuperscript{\citeproc{ref-de2022}{395}}
\item
  As interpretações qualitativas (isto é: pobre/fraca/baixa, moderada/razoável/aceitável, boa ou muito boa/alta/excelente) dos valores de área sob a curva são arbitrários e não devem ser considerados isoladamente.\textsuperscript{\citeproc{ref-de2022}{395}}
\item
  Modelos de classificação com valores altos de área sob a curva podem ser enganosos se os valores preditos por esses modelos não estiverem adequadamente calibrados.\textsuperscript{\citeproc{ref-de2022}{395}}
\item
  Diferenças pequenas entre AUCs podem não ser estatisticamente significativas.\textsuperscript{\citeproc{ref-he2024}{392}}
\item
  A interpretação clínica pode ser equivocada se não houver teste estatístico adequado.\textsuperscript{\citeproc{ref-he2024}{392}}
\item
  Se as curvas vêm do mesmo conjunto de pacientes, aplique o teste de DeLong.\textsuperscript{\citeproc{ref-he2024}{392}}
\item
  Se as curvas vêm de amostras independentes, use métodos como Dorfman-Alf.\textsuperscript{\citeproc{ref-he2024}{392}}
\end{itemize}

\subsection{Por que uma AUC menor que 0.5 está errada?}\label{por-que-uma-auc-menor-que-0.5-estuxe1-errada}

\begin{itemize}
\item
  Porque indica desempenho pior que o acaso.\textsuperscript{\citeproc{ref-he2024}{392}}
\item
  Geralmente decorre de seleção incorreta da direção do teste ou da variável de estado.\textsuperscript{\citeproc{ref-he2024}{392}}
\item
  Verifique se o software está configurado para maiores valores indicam presença do evento ou o inverso.\textsuperscript{\citeproc{ref-he2024}{392}}
\item
  Ajuste a direção do teste para que \(AUC \geq 0.5\).\textsuperscript{\citeproc{ref-he2024}{392}}
\end{itemize}

\begin{figure}

{\centering \includegraphics{Ciencia-com-R_files/figure-latex/roc-1} 

}

\caption{Curva ROC (Receiver Operating Characteristic) para um modelos de classificação com diferentes desempenhos diagnósticos.}\label{fig:roc}
\end{figure}

\subsection{Como analisar o desempenho diagnóstico em desfechos com distribuição trimodal na população?}\label{como-analisar-o-desempenho-diagnuxf3stico-em-desfechos-com-distribuiuxe7uxe3o-trimodal-na-populauxe7uxe3o}

\begin{itemize}
\tightlist
\item
  Limiares duplos podem ser utilizados para análise de desempenho diagnóstico de testes com distribuição trimodal.\textsuperscript{\citeproc{ref-ferreira2021}{397}}
\end{itemize}

\section{\texorpdfstring{Gráficos \emph{crosshair}}{Gráficos crosshair}}\label{gruxe1ficos-crosshair}

\subsection{\texorpdfstring{O que um gráfico \emph{crosshair}?}{O que um gráfico crosshair?}}\label{o-que-um-gruxe1fico-crosshair}

\begin{itemize}
\tightlist
\item
  .\textsuperscript{\citeproc{ref-phillips2010}{398}}
\end{itemize}

\begin{figure}

{\centering \includegraphics{Ciencia-com-R_files/figure-latex/crosshair-1} 

}

\caption{Gráfico *crosshair* em espaço ROC (Receiver Operating Characteristic) para 15 estudos simulados de desempenho diagnóstico.}\label{fig:crosshair}
\end{figure}

\begin{infobox}{images/Rlogo}
O pacote \emph{mada}\textsuperscript{\citeproc{ref-mada}{399}} fornece a função \href{https://www.rdocumentation.org/packages/mada/versions/0.5.11/topics/crosshair}{\emph{crosshair}} para criar um gráfico \emph{crosshair}\textsuperscript{\citeproc{ref-phillips2010}{398}} a partir de dados de verdadeiro-positivo, falso-positivo, verdadeiro-negativo e verdadeiro-positivo de tabelas de confusão 2x2.

\end{infobox}

\section{Interpretação da validade de um teste}\label{interpretauxe7uxe3o-da-validade-de-um-teste}

\subsection{Que itens devem ser verificados na interpretação de um estudo de validade?}\label{que-itens-devem-ser-verificados-na-interpretauxe7uxe3o-de-um-estudo-de-validade}

\begin{itemize}
\item
  O novo teste foi comparado junto ao método padrão-ouro.\textsuperscript{\citeproc{ref-greenhalgh1997b}{387}}
\item
  As probabilidades pontuais estimadas que caracterizam o desempenho diagnóstico do novo teste são altas e adequadas para sua aplicação clínica.\textsuperscript{\citeproc{ref-greenhalgh1997b}{387}}
\item
  Os intervalos de confiança estimados para as probabilidades do novo teste são estreitos e adequadas para sua aplicação clínica.\textsuperscript{\citeproc{ref-greenhalgh1997b}{387}}
\item
  O novo teste possui adequada confiabilidade intra/inter examinadores.\textsuperscript{\citeproc{ref-greenhalgh1997b}{387}}
\item
  O estudo de validação incluiu um espectro adequado da amostra.\textsuperscript{\citeproc{ref-greenhalgh1997b}{387}}
\item
  Todos os participantes realizaram ambos o novo teste e o padrão-ouro no estudo de validação.\textsuperscript{\citeproc{ref-greenhalgh1997b}{387}}
\item
  Os examinadores do novo teste estavam cegados para o resultado do teste padrão-ouro.\textsuperscript{\citeproc{ref-greenhalgh1997b}{387}}
\end{itemize}

\section{Diretrizes para redação}\label{diretrizes-para-redauxe7uxe3o-5}

\subsection{Quais são as diretrizes para redação de estudos diagnósticos?}\label{quais-suxe3o-as-diretrizes-para-redauxe7uxe3o-de-estudos-diagnuxf3sticos}

\begin{itemize}
\item
  Visite a rede \emph{Enhancing the QUAlity and Transparency Of health Research} (\href{https://www.equator-network.org/}{EQUATOR Network}) para encontrar diretrizes específicas.
\item
  \emph{STARD 2015: An Updated List of Essential Items for Reporting Diagnostic Accuracy Studies}:\textsuperscript{\citeproc{ref-bossuyt2015}{400}} \url{https://www.equator-network.org/reporting-guidelines/stard/}
\end{itemize}

\chapter{\texorpdfstring{\textbf{Ensaios quase-experimentais}}{Ensaios quase-experimentais}}\label{ensaio-quase-experimental}

\section{Características}\label{caracteruxedsticas-4}

\subsection{Quais são as características de ensaios quase-experimentais?}\label{quais-suxe3o-as-caracteruxedsticas-de-ensaios-quase-experimentais}

\begin{itemize}
\tightlist
\item
  .\textsuperscript{\citeproc{ref-REF}{\textbf{REF?}}}
\end{itemize}

\section{Diretrizes para redação}\label{diretrizes-para-redauxe7uxe3o-6}

\subsection{Quais são as diretrizes para redação de ensaios quase-experimentais?}\label{quais-suxe3o-as-diretrizes-para-redauxe7uxe3o-de-ensaios-quase-experimentais}

\begin{itemize}
\item
  Visite a rede \emph{Enhancing the QUAlity and Transparency Of health Research} (\href{https://www.equator-network.org/}{EQUATOR Network}) para encontrar diretrizes específicas.
\item
  \emph{Guidelines for reporting non-randomised studies}:\textsuperscript{\citeproc{ref-reeves2004}{401}} \url{https://www.equator-network.org/reporting-guidelines/guidelines-for-reporting-non-randomised-studies/}
\end{itemize}

\chapter{\texorpdfstring{\textbf{Ensaios experimentais}}{Ensaios experimentais}}\label{ensaio-experimental}

\section{Ensaio clínico aleatorizado}\label{ensaio-cluxednico-aleatorizado}

\subsection{Quais são as características de ensaios clínicos aleatorizados?}\label{quais-suxe3o-as-caracteruxedsticas-de-ensaios-cluxednicos-aleatorizados}

\begin{itemize}
\item
  A característica essencial de um ensaio clínico aleatorizado é a comparação entre grupos.\textsuperscript{\citeproc{ref-bland2011}{402}}
\item
  Quanto à unidade de alocação:\textsuperscript{\citeproc{ref-Bruce2022}{403}}

  \begin{itemize}
  \item
    Individual
  \item
    Agrupado
  \end{itemize}
\item
  Quanto ao número de braços:\textsuperscript{\citeproc{ref-Bruce2022}{403}}

  \begin{itemize}
  \tightlist
  \item
    Múltiplos
  \end{itemize}
\item
  Quanto ao número de centros:\textsuperscript{\citeproc{ref-Bruce2022}{403}}

  \begin{itemize}
  \item
    Único
  \item
    Múltiplos
  \end{itemize}
\item
  Quanto ao cegamento:\textsuperscript{\citeproc{ref-Bruce2022}{403}}

  \begin{itemize}
  \item
    Aberto
  \item
    Simples-cego
  \item
    Duplo-cego
  \item
    Triplo-cego
  \item
    Quádruplo-cego
  \end{itemize}
\item
  Quanto à alocação:\textsuperscript{\citeproc{ref-Bruce2022}{403}}

  \begin{itemize}
  \item
    Sem sorteio
  \item
    Estratificada (centro apenas)
  \item
    Estratificada
  \item
    Minimizada
  \item
    Estratificada e minimizada
  \end{itemize}
\end{itemize}

\subsection{Quais são as estratégias metodológicas para reduzir vieses?}\label{quais-suxe3o-as-estratuxe9gias-metodoluxf3gicas-para-reduzir-vieses}

\begin{itemize}
\item
  Grupo controle: comparar a intervenção a um cuidado usual ou controle ativo ajuda a isolar o efeito específico do tratamento, reduzindo vieses de confusão e maturação.\textsuperscript{\citeproc{ref-REF}{\textbf{REF?}}}
\item
  Grupo placebo: prepara uma intervenção indistinguível da ativa para mitigar expectativas de participantes e profissionais, reduzindo viés de desempenho e detecção.\textsuperscript{\citeproc{ref-REF}{\textbf{REF?}}}
\item
  Controle \emph{sham}: em intervenções de procedimento (p.ex., cirúrgicas/fisioterapêuticas), um comparador que reproduz etapas não-específicas do procedimento controla efeitos placebo e da atenção.\textsuperscript{\citeproc{ref-REF}{\textbf{REF?}}}
\item
  Cegamento: mascarar participantes, profissionais, avaliadores e/ou analistas diminui vieses de desempenho e detecção; deve-se explicitar quem foi cegado e como a manutenção do cegamento foi assegurada.\textsuperscript{\citeproc{ref-REF}{\textbf{REF?}}}
\end{itemize}

\section{Modelos de análise de comparação}\label{modelos-de-anuxe1lise-de-comparauxe7uxe3o}

\subsection{Que modelos podem ser utilizados para comparações?}\label{que-modelos-podem-ser-utilizados-para-comparauxe7uxf5es}

\begin{itemize}
\item
  As abordagens compreendem a comparação da variável de desfecho medida entre os momentos antes e depois ou da sua mudança (pré - pós) entre os momentos.\textsuperscript{\citeproc{ref-Vickers2001a}{404}}
\item
  Se a média da variável é igual entre grupos no início do acompanhamento, ambas abordagens estimam o mesmo efeito. Caso contrário, o efeito será influenciado pela correlação entre as medidas antes e depois. A análise da mudança não controla para desbalanços no início do estudo.\textsuperscript{\citeproc{ref-Vickers2001a}{404}}
\item
  A abordagem mais recomendada é a análise de covariância (ANCOVA) \eqref{eq:ancova1}, pois ajusta os valores pós-intervenção (\(Y_{ij}\)) aos valores pré-intervenção (\(X_{ij}\)) para cada participante (\(i\)) de cada grupo \{\(Z_{ij}\)\}, e portanto não é afetada pelas diferenças entre grupos no início do estudo.\textsuperscript{\citeproc{ref-barnett2004}{10},\citeproc{ref-Vickers2001a}{404}}
\end{itemize}

\begin{equation}
\label{eq:ancova1}
Y_{ij} = \beta_0 + \beta_1 X_{ij} + \beta_2 Z_j + \epsilon_{ij}
\end{equation}

\begin{itemize}
\item
  A ANCOVA modelando seja a mudança (pré - pós: \(\Delta = X_{ij} - Y_{ij}\)) quando o desfecho no pós-tratamento parece ser o método mais efetivo considerando-se o viés de estimação dos parâmetros, a precisão das estimativas, a cobertura nominal (isto é, intervalo de confiança) e o poder do teste.\textsuperscript{\citeproc{ref-OConnell2017}{405}}
\item
  Quando a ANCOVA \eqref{eq:ancova2} é utilizada com a mudança (pré - pós) com variável de desfecho (\(Y_{ij}\)), o coeficiente de regressão \(\beta_1\) é diminuído em 1 unidade.\textsuperscript{\citeproc{ref-barnett2004}{10},\citeproc{ref-laird1983}{406}}
\end{itemize}

\begin{equation}
\label{eq:ancova2}
(X_{ij} - Y_{ij}) = \beta_0 + \beta_1 Z_j + \epsilon_{ij}
\end{equation}

\begin{itemize}
\item
  Análise de variância (ANOVA) e modelos lineares mistos (MLM) são outras opções de métodos, embora apresentem maior variância, menor poder, e cobertura nominal comparados à ANCOVA.\textsuperscript{\citeproc{ref-OConnell2017}{405}}
\item
  Em desenhos com múltiplas medições por participante, modelos lineares mistos (efeitos aleatórios para indivíduo e, se pertinente, para centro) permitem lidar com correlação intra-sujeito e dados ausentes sob MAR, oferecendo estimativas válidas do efeito de tratamento no tempo.\textsuperscript{\citeproc{ref-Cnaan1997}{407}}
\item
  Para dados longitudinais com desfechos contínuos, estratégias de modelo de efeitos mistos com medidas repetidas evitam a imputação explícita e, sob suposições de MAR, tendem a melhor cobertura e controle de erro tipo I do que abordagens tipo ``última observação transportada''.\textsuperscript{\citeproc{ref-mallinckrodt2008}{408}}
\end{itemize}

\section{Comparação na linha de base}\label{comparauxe7uxe3o-na-linha-de-base}

\subsection{O que são dados na linha de base?}\label{o-que-suxe3o-dados-na-linha-de-base}

\begin{itemize}
\item
  Dados sociodemográficos, clínicos e funcionais são coletados na linha de base sobre cada participante no momento da aleatorização.\textsuperscript{\citeproc{ref-Assmann2000}{409}}
\item
  Os dados de linha de base são usados para caracterizar os pacientes incluídos no estudo e para mostrar que os grupos de tratamento estão bem equilibrados.\textsuperscript{\citeproc{ref-Assmann2000}{409}}
\item
  Dados da linha de base podem ser utilizados para a aleatorização de modo a equilíbrar ou estratificar os grupos considerando alguns fatores-chave.\textsuperscript{\citeproc{ref-Assmann2000}{409}}
\item
  Dados da linha de base podem ser utilizados como ajuste de covariável para análise do resultado por grupo de tratamento.\textsuperscript{\citeproc{ref-Assmann2000}{409}}
\end{itemize}

\subsection{O que é comparação entre grupos na linha de base em ensaios clínicos aleatorizados?}\label{o-que-uxe9-comparauxe7uxe3o-entre-grupos-na-linha-de-base-em-ensaios-cluxednicos-aleatorizados}

\begin{itemize}
\item
  A comparação se refere ao teste de hipótese nula de não haver diferença (`balanço' ou `equilíbrio') entre grupos de tratamento nas (co)variáveis na linha de base, geralmente apresentadas apenas de modo descritivo na `Tabela 1'.\textsuperscript{\citeproc{ref-Stang2018}{410}}
\item
  A interpretação isolada do P-valor da comparação entre grupos na linha de base não permite identificar as razões para eventuais diferenças.\textsuperscript{\citeproc{ref-Stang2018}{410}}
\end{itemize}

\subsection{Para quê comparar grupos na linha de base em ensaios clínicos aleatorizados?}\label{para-quuxea-comparar-grupos-na-linha-de-base-em-ensaios-cluxednicos-aleatorizados}

\begin{itemize}
\item
  Os P-valores estão relacionados à aleatorização dos participantes em grupos.\textsuperscript{\citeproc{ref-Bolzern2019}{411}}
\item
  Em ensaios clínicos aleatorizados, a comparação de (co)variáveis na linha de base é usada para avaliar se aleatorização foi `bem sucedida'.\textsuperscript{\citeproc{ref-Bolzern2019}{411}}
\end{itemize}

\subsection{Quais são as razões para diferenças entre grupos de tratamento nas (co)variáveis na linha de base?}\label{quais-suxe3o-as-razuxf5es-para-diferenuxe7as-entre-grupos-de-tratamento-nas-covariuxe1veis-na-linha-de-base}

\begin{itemize}
\item
  Acaso.\textsuperscript{\citeproc{ref-chen2020}{217},\citeproc{ref-Stang2018}{410}}
\item
  Viés.\textsuperscript{\citeproc{ref-chen2020}{217},\citeproc{ref-Stang2018}{410}}
\item
  Tamanho da amostra.\textsuperscript{\citeproc{ref-chen2020}{217},\citeproc{ref-Stang2018}{410}}
\item
  Má conduta científica.\textsuperscript{\citeproc{ref-chen2020}{217}}
\end{itemize}

\subsection{Quais cenários permitem a comparação entre grupos na linha de base em ensaios clínicos aleatorizados?}\label{quais-cenuxe1rios-permitem-a-comparauxe7uxe3o-entre-grupos-na-linha-de-base-em-ensaios-cluxednicos-aleatorizados}

\begin{itemize}
\item
  Em ensaios clínicos aleatorizados agregados, os P-valores possuem interpretação diferente de estudos aleatorizados individualmente.\textsuperscript{\citeproc{ref-Bolzern2019}{411}}
\item
  Em ensaios clínicos com agrupamento, nos quais o recrutamento ocorreu após a aleatorização, os P-valores já não estão inteiramente relacionados ao processo de aleatorização, mas sim ao método de recrutamento, o que pode resultar na comparação de amostras não aleatórias.\textsuperscript{\citeproc{ref-Bolzern2019}{411}}
\end{itemize}

\subsection{Por que não se deve comparar grupos na linha de base em ensaios clínicos aleatorizados?}\label{por-que-nuxe3o-se-deve-comparar-grupos-na-linha-de-base-em-ensaios-cluxednicos-aleatorizados}

\begin{itemize}
\item
  A interpretação errônea dos P-valores na comparação entre grupos, na linha de base, de um ensaio clínico aleatorizado constitui a `falácia da Tabela 1'.\textsuperscript{\citeproc{ref-pijls2022}{218}}
\item
  Quando a aleatorização é bem-sucedida, a hipótese nula de diferença entre grupos na linha de base é verdadeira.\textsuperscript{\citeproc{ref-roberts1999}{412}}
\item
  Testes de significância estatística na linha de base avaliam a probabilidade de que as diferenças observadas possam ter ocorrido por acaso; no entanto, já sabemos --- pelo delineamento do experimento --- que quaisquer diferenças são causadas pelo acaso.\textsuperscript{\citeproc{ref-gruijters2020}{413}}
\end{itemize}

\subsection{Quais estratégias podem ser adotadas para substituir a comparação entre grupos na linha de base em ensaios clínicos aleatorizados?}\label{quais-estratuxe9gias-podem-ser-adotadas-para-substituir-a-comparauxe7uxe3o-entre-grupos-na-linha-de-base-em-ensaios-cluxednicos-aleatorizados}

\begin{itemize}
\item
  Na fase de projeto: identifique as variáveis prognósticas do desfecho de acordo com a literatura.\textsuperscript{\citeproc{ref-roberts1999}{412}}
\item
  Na fase de análise: inclua as variáveis prognósticas nos modelos para ajuste.\textsuperscript{\citeproc{ref-roberts1999}{412}}
\end{itemize}

\section{Comparação intragrupos}\label{comparauxe7uxe3o-intragrupos}

\subsection{Por que não se deve comparar intragrupos (pré - pós) em ensaios clínicos aleatorizados?}\label{por-que-nuxe3o-se-deve-comparar-intragrupos-pruxe9---puxf3s-em-ensaios-cluxednicos-aleatorizados}

\begin{itemize}
\tightlist
\item
  Testar por mudanças a partir da linha de base separadamente em cada grupos aleatorizados não permite concluir sobre diferenças entre grupos; não se pode fazer inferências a partir da comparação de P-valores.\textsuperscript{\citeproc{ref-bland2011}{402}}
\end{itemize}

\section{Comparação entre grupos}\label{comparauxe7uxe3o-entre-grupos}

\subsection{O que é comparação entre grupos em ensaios clínicos aleatorizados?}\label{o-que-uxe9-comparauxe7uxe3o-entre-grupos-em-ensaios-cluxednicos-aleatorizados}

\begin{itemize}
\tightlist
\item
  A comparação se refere ao teste de hipótese nula de não haver diferença (`alteração' ou `mudança') pós-tratamento entre grupos de tratamento.\textsuperscript{\citeproc{ref-bland2011}{402}}
\end{itemize}

\subsection{O que pode ser comparado entre grupos?}\label{o-que-pode-ser-comparado-entre-grupos}

\begin{itemize}
\tightlist
\item
  Valores pós-tratamento; mudança entre linha de base e pós-tratamento; mudança percentual da linha de base.\textsuperscript{\citeproc{ref-vickers2001b}{414}}
\end{itemize}

\subsection{Qual é a comparação entre grupos mais adequada em ensaios clínicos aleatorizados?}\label{qual-uxe9-a-comparauxe7uxe3o-entre-grupos-mais-adequada-em-ensaios-cluxednicos-aleatorizados}

\begin{itemize}
\item
  Análise de covariância (ANCOVA) que analisa o pós-tratamento com a linha de base como covariável é a opção que possui maior poder estatístico.\textsuperscript{\citeproc{ref-vickers2001b}{414}}
\item
  Mudança entre linha de base e pós-tratamento tem poder adequado apenas quando a correlação entre linha de base e pós-tratamento é alta.\textsuperscript{\citeproc{ref-vickers2001b}{414}}
\item
  Mudança percentual da linha de base é a opção menos eficiente em termos de poder estatístico.\textsuperscript{\citeproc{ref-vickers2001b}{414}}
\end{itemize}

\section{Comparação de subgrupos}\label{comparauxe7uxe3o-de-subgrupos}

\subsection{O que é comparação de subgrupos em ensaios clínicos aleatorizados?}\label{o-que-uxe9-comparauxe7uxe3o-de-subgrupos-em-ensaios-cluxednicos-aleatorizados}

\begin{itemize}
\tightlist
\item
  Análises de subgrupos podem ser realizadas para avaliar se as diferenças no resultado do tratamento (ou a falta delas) dependem de algumas características na linha de base dos pacientes.\textsuperscript{\citeproc{ref-Assmann2000}{409}}
\end{itemize}

\subsection{Como realizar a comparação de subgrupos em ensaios clínicos aleatorizados?}\label{como-realizar-a-comparauxe7uxe3o-de-subgrupos-em-ensaios-cluxednicos-aleatorizados}

\begin{itemize}
\tightlist
\item
  Testes estatísticos de interação (que avaliam se um efeito de tratamento difere entre subgrupos) devem ser usados, e não apenas a inspeção dos P-valores do subgrupo. Somente se o teste de interação estatística apoiar um efeito de subgrupo as conclusões poderão ser influenciadas.\textsuperscript{\citeproc{ref-Assmann2000}{409},\citeproc{ref-Brookes2004}{415}}
\end{itemize}

\subsection{Como interpretar a comparação de subgrupos em ensaios clínicos aleatorizados?}\label{como-interpretar-a-comparauxe7uxe3o-de-subgrupos-em-ensaios-cluxednicos-aleatorizados}

\begin{itemize}
\item
  Análises de subgrupos devem ser consideradas de natureza exploratória e raramente elas afetam as conclusões obtidas a partir do estudo.\textsuperscript{\citeproc{ref-Assmann2000}{409},\citeproc{ref-Brookes2004}{415}}
\item
  A credibilidade das análises de subgrupos é melhor se restrita ao desfecho primário e a alguns subgrupos predefinidos e baseadas em hipóteses biologicamente plausíveis.\textsuperscript{\citeproc{ref-Assmann2000}{409}}
\item
  Deve-se verificar se o estudo possui poder estatístico suficiente para detectar tamanhos de efeitos realistas em subgrupos e interpretar com cautela uma diferença de tratamento em um subgrupo quando a comparação global do tratamento não é significativa.\textsuperscript{\citeproc{ref-Assmann2000}{409}}
\end{itemize}

\section{Efeito de interação}\label{efeito-de-interauxe7uxe3o-1}

\subsection{Por que analisar o efeito de interação?}\label{por-que-analisar-o-efeito-de-interauxe7uxe3o}

\begin{itemize}
\item
  Em ensaios clínicos aleatorizados, o principal problema de pesquisa é se há uma diferença pré - pós maior em um grupo do que em outro(s).\textsuperscript{\citeproc{ref-bland2011}{402}}
\item
  A comparação de subgrupos por meio de testes de significância de hipótese nula separados é enganosa por não testar (comparar) diretamente os tamanhos dos efeitos dos tratamentos.\textsuperscript{\citeproc{ref-Matthews1996}{416}}
\item
  Revisões recentes destacam que a interpretação de interações requer parcimônia (predefinição, plausibilidade biológica e controle do error-rate), e recomendam relatar estimativas e intervalos de confiança por subgrupo junto com o teste formal de interação.\textsuperscript{\citeproc{ref-Bours2023}{320}}
\end{itemize}

\subsection{Quando usar o termo de interação?}\label{quando-usar-o-termo-de-interauxe7uxe3o}

\begin{itemize}
\item
  Análise de efeito de interação pode ser usada para testar se o efeito de um tratamento varia entre dois ou mais subgrupos de indivíduos, ou seja, se um efeito é modificado pelo(s) outros(s) efeito(s).\textsuperscript{\citeproc{ref-Altman1996}{321}}
\item
  A interação entre duas (ou mais) variáveis pode ser utilizada para comparar efeitos do tratamento em subgrupos de ensaios clínicos.\textsuperscript{\citeproc{ref-Altman2003}{417}}
\item
  O poder estatístico para detectar efeitos de interação é limitado.\textsuperscript{\citeproc{ref-Altman2003}{417}}
\end{itemize}

\section{Ajuste de covariáveis}\label{ajuste-de-covariuxe1veis}

\subsection{Quais variáveis devem ser utilizadas no ajuste de covariáveis?}\label{quais-variuxe1veis-devem-ser-utilizadas-no-ajuste-de-covariuxe1veis}

\begin{itemize}
\tightlist
\item
  A escolha das características de linha de base pelas quais uma análise é ajustada deve ser determinada pelo conhecimento prévio de uma possível influência no resultado, em vez da evidência de desequilíbrio entre os grupos de tratamento no estudo.\textsuperscript{\citeproc{ref-roberts1999}{412}}
\end{itemize}

\subsection{Quais os benefícios do ajuste de covariáveis?}\label{quais-os-benefuxedcios-do-ajuste-de-covariuxe1veis}

\begin{itemize}
\item
  Ajustar por covariáveis ajuda a estimar os efeitos do tratamento para o indivíduo, assim como aumenta a eficiência dos testes para hipótese nula e a validade externa do estudo.\textsuperscript{\citeproc{ref-Hauck1998}{418}}
\item
  Incluir a variável de desfecho medida na linha de base como covariável --- independentemente de a análise ser realizada com a medida pós-tratamento da mesma variável ou a diferença para a linha de base --- pode aumentar o poder estatístico do estudo.\textsuperscript{\citeproc{ref-Kahan2014}{419}}
\item
  Incluir outras variáveis medidas na linha de base, com potencial para serem desbalanceadas entre grupos após a aleatorização, diminui a chance de afetar as estimativas de efeito dos tratamentos.\textsuperscript{\citeproc{ref-Kahan2014}{419}}
\end{itemize}

\subsection{Quais os riscos do ajuste de covariáveis?}\label{quais-os-riscos-do-ajuste-de-covariuxe1veis}

\begin{itemize}
\item
  Incluir covariáveis que não são prognósticas do desfecho pode reduzir o poder estatístico do estudo.\textsuperscript{\citeproc{ref-Kahan2014}{419}}
\item
  Incluir covariáveis com dados perdidos pode reduzir o tamanho amostral e consequentemente o poder estatístico do estudo (análise de casos completos) ou levar a desvios do plano de análise por exclusão de covariáveis prognósticas.\textsuperscript{\citeproc{ref-Kahan2014}{419}}
\end{itemize}

\section{Imputação de dados perdidos}\label{imputauxe7uxe3o-de-dados-perdidos}

\subsection{Como lidar com os dados perdidos em desfechos?}\label{como-lidar-com-os-dados-perdidos-em-desfechos}

\begin{itemize}
\item
  Em dados longitudinais com um pequeno número de `ondas' (medidas repetidas) e poucas variáveis, para análise com modelos de regressão univariados, a imputação paramétrica via especificação condicional completa - também conhecido como imputação multivariada por equações encadeadas (\emph{multivariate imputation by chained equations}, MICE) --- é eficiente do ponto de vista computacional e possui acurácia e precisão para estimação de parâmetros.\textsuperscript{\citeproc{ref-Heymans2022}{152},\citeproc{ref-Cao2022}{420}}
\item
  Para dados perdidos em desfechos dicotômicos, o desempenho dos métodos de imputação multivariada por equações encadeadas (\emph{multivariate imputation by chained equations}, MICE)\textsuperscript{\citeproc{ref-mice}{159}} e por correspondência média preditiva (\emph{predictive mean matching}, PMM)\textsuperscript{\citeproc{ref-rubin1986}{160},\citeproc{ref-little1988a}{161}} é similar.\textsuperscript{\citeproc{ref-austin2023}{158}}
\end{itemize}

\subsection{Como lidar com os dados perdidos em covariáveis?}\label{como-lidar-com-os-dados-perdidos-em-covariuxe1veis}

\begin{itemize}
\item
  Imputação de dados perdidos de uma covariável pela média dos dados do respectivo grupo permite estimativas não enviesadas dos efeitos do tratamento, preserva o erro tipo I e aumenta o poder estatístico comparado à análise de dados completos.\textsuperscript{\citeproc{ref-Kahan2014}{419}}
\item
  Para desfechos ausentes, recomenda-se evitar transportar a última observação e, quando aplicável, preferir modelos lineares mistos ou imputação múltipla consistentes com o estimando de interesse.\textsuperscript{\citeproc{ref-mallinckrodt2008}{408}}
\end{itemize}

\begin{infobox}{images/Rlogo}
Os pacotes \emph{mice}\textsuperscript{\citeproc{ref-mice}{159}} e \emph{miceadds}\textsuperscript{\citeproc{ref-miceadds}{162}} fornecem funções \href{https://www.rdocumentation.org/packages/mice/versions/3.16.0/topics/mice}{\emph{mice}} e \href{https://www.rdocumentation.org/packages/miceadds/versions/3.16-18/topics/mi.anova}{\emph{mi.anova}} para imputação multivariada por equações encadeadas, respectivamente, para imputação de dados.

\end{infobox}

\section{Diretrizes para redação}\label{diretrizes-para-redauxe7uxe3o-7}

\subsection{Quais são as diretrizes para redação de ensaios experimentais?}\label{quais-suxe3o-as-diretrizes-para-redauxe7uxe3o-de-ensaios-experimentais}

\begin{itemize}
\item
  Visite a rede \emph{Enhancing the QUAlity and Transparency Of health Research} (\href{https://www.equator-network.org/}{EQUATOR Network}) para encontrar diretrizes específicas.
\item
  \emph{CONSORT 2010 Statement: updated guidelines for reporting parallel group randomised trials}:\textsuperscript{\citeproc{ref-schulz2010}{421}} \url{https://www.equator-network.org/reporting-guidelines/consort/}
\end{itemize}

\begin{infobox}{images/Rlogo}
O pacote \emph{consort}\textsuperscript{\citeproc{ref-consort}{422}} fornece a função \href{\%60r\%20paste0(\%22https://search.r-project.org/CRAN/refmans/\%22,\%20\%22consort\%22,\%20\%22/html/\%22,\%20\%22consort_plot\%22,\%20\%22.html\%22)\%60}{\emph{consort\_plot}} para elaboração do fluxograma de ensaios experimentais no formato padrão.

\end{infobox}

\chapter{\texorpdfstring{\textbf{Ensaios cruzados}}{Ensaios cruzados}}\label{ensaio-cruzado}

\section{Características}\label{caracteruxedsticas-5}

\subsection{Quais são as características de ensaios cruzados?}\label{quais-suxe3o-as-caracteruxedsticas-de-ensaios-cruzados}

\begin{itemize}
\tightlist
\item
  .\textsuperscript{\citeproc{ref-REF}{\textbf{REF?}}}
\end{itemize}

\section{Diretrizes para redação}\label{diretrizes-para-redauxe7uxe3o-8}

\subsection{Quais são as diretrizes para redação de ensaios cruzados?}\label{quais-suxe3o-as-diretrizes-para-redauxe7uxe3o-de-ensaios-cruzados}

\begin{itemize}
\item
  Visite a rede \emph{Enhancing the QUAlity and Transparency Of health Research} (\href{https://www.equator-network.org/}{EQUATOR Network}) para encontrar diretrizes específicas.
\item
  \emph{CONSORT 2010 statement: extension to randomised crossover trials}:\textsuperscript{\citeproc{ref-dwan2019}{423}} \url{https://www.equator-network.org/reporting-guidelines/consort-2010-statement-extension-to-randomised-crossover-trials/}
\end{itemize}

\chapter{\texorpdfstring{\textbf{N de 1}}{N de 1}}\label{n-de-1}

\section{Ensaio N-de-1}\label{ensaio-n-de-1}

\subsection{O que são ensaios N-de-1?}\label{o-que-suxe3o-ensaios-n-de-1}

\begin{itemize}
\item
  Ensaios N-de-1 são delineamentos experimentais em que um único paciente recebe, em períodos alternados, duas ou mais intervenções (ex.: tratamento A e tratamento B).\textsuperscript{\citeproc{ref-senn2024}{424}}
\item
  Cada ciclo é formado por dois períodos (AB ou BA), cuja ordem é randomizada, garantindo controle temporal e redução de vieses.\textsuperscript{\citeproc{ref-senn2024}{424}}
\item
  O foco está na comparação intraindivíduo, permitindo avaliar diretamente se o paciente em questão responde melhor a uma intervenção.\textsuperscript{\citeproc{ref-senn2024}{424}}
\end{itemize}

\subsection{Quando usar ensaios N-de-1?}\label{quando-usar-ensaios-n-de-1}

\begin{itemize}
\item
  Doenças crônicas estáveis, em que o desfecho pode ser observado repetidamente.\textsuperscript{\citeproc{ref-senn2024}{424}}
\item
  Condições raras ou com grande heterogeneidade de resposta entre pacientes.\textsuperscript{\citeproc{ref-senn2024}{424}}
\item
  Situações clínicas de incerteza, quando se deseja personalizar o tratamento.\textsuperscript{\citeproc{ref-senn2024}{424}}
\end{itemize}

\subsection{Qual a relevância dos ensaios N-de-1?}\label{qual-a-relevuxe2ncia-dos-ensaios-n-de-1}

\begin{itemize}
\item
  Os ensaios N-de-1 permitem decisões clínicas personalizadas e baseadas em evidência direta.\textsuperscript{\citeproc{ref-senn2024}{424}}
\item
  Quando combinados, podem gerar estimativas comparáveis às de ensaios clínicos convencionais, mantendo o foco centrado no paciente.\textsuperscript{\citeproc{ref-senn2024}{424}}
\item
  Representam uma alternativa metodológica robusta para cenários de incerteza terapêutica.\textsuperscript{\citeproc{ref-senn2024}{424}}
\end{itemize}

\section{Aspectos metodológicos}\label{aspectos-metodoluxf3gicos}

\subsection{Como é feita a randomização?}\label{como-uxe9-feita-a-randomizauxe7uxe3o}

\begin{itemize}
\item
  A ordem dos tratamentos em cada ciclo é definida aleatoriamente (ex.: AB, BA, AB\ldots).\textsuperscript{\citeproc{ref-senn2024}{424}}
\item
  Essa randomização reduz viés de período e efeito de expectativa.\textsuperscript{\citeproc{ref-senn2024}{424}}
\end{itemize}

\subsection{Como são feitas as análises?}\label{como-suxe3o-feitas-as-anuxe1lises}

\begin{itemize}
\item
  Comparações intraindivíduo (testes pareados ou estimativas de efeito médio por paciente).\textsuperscript{\citeproc{ref-senn2024}{424}}
\item
  Combinação de múltiplos N-de-1 por meio de meta-análises ou modelos mistos para inferências em nível populacional.\textsuperscript{\citeproc{ref-senn2024}{424}}
\end{itemize}

\subsection{Quais perguntas de inferência podem ser respondidas?}\label{quais-perguntas-de-inferuxeancia-podem-ser-respondidas}

\begin{itemize}
\item
  Q1: Há efeito do tratamento dentro dos ciclos de um paciente?\textsuperscript{\citeproc{ref-senn2024}{424}}
\item
  Q2: Qual é o efeito médio observado nos pacientes estudados?\textsuperscript{\citeproc{ref-senn2024}{424}}
\item
  Q3: O efeito é homogêneo ou heterogêneo entre pacientes?\textsuperscript{\citeproc{ref-senn2024}{424}}
\item
  Q4: Qual é o efeito específico em cada paciente individual?\textsuperscript{\citeproc{ref-senn2024}{424}}
\item
  Q5: Qual é o efeito esperado em populações semelhantes?\textsuperscript{\citeproc{ref-senn2024}{424}}
\end{itemize}

\section{Limitações e cuidados}\label{limitauxe7uxf5es-e-cuidados}

\subsection{Quais são os principais desafios dos ensaios N-de-1?}\label{quais-suxe3o-os-principais-desafios-dos-ensaios-n-de-1}

\begin{itemize}
\item
  Baixo poder estatístico quando poucos ciclos são realizados.\textsuperscript{\citeproc{ref-senn2024}{424}}
\item
  Necessidade de períodos de \emph{washout} para evitar efeito de \emph{carry-over}.\textsuperscript{\citeproc{ref-senn2024}{424}}
\item
  Interpretação dependente de pressupostos sobre homogeneidade ou heterogeneidade dos efeitos.\textsuperscript{\citeproc{ref-senn2024}{424}}
\item
  Em amostras muito pequenas, pode ser necessário usar variâncias externas ou modelos mistos.\textsuperscript{\citeproc{ref-senn2024}{424}}
\end{itemize}

\chapter{\texorpdfstring{\textbf{Revisão sistemática}}{Revisão sistemática}}\label{revisao-sistematica}

\section{Tipologia de revisões}\label{tipologia-de-revisuxf5es}

\subsection{O que é a tipologia de revisões?}\label{o-que-uxe9-a-tipologia-de-revisuxf5es}

\begin{itemize}
\tightlist
\item
  Foram mapeados 14 tipos de revisão e suas metodologias, organizando-as pelo modelo SALSA (\emph{Search}, \emph{Appraisal}, \emph{Synthesis}, \emph{Analysis}). Essa tipologia demonstra que diferentes revisões têm diferentes níveis de rigor, sistematicidade e propósito, ajudando pesquisadores a escolher o delineamento mais adequado.\textsuperscript{\citeproc{ref-Grant2009}{354}}
\end{itemize}

\section{Revisão sistemática de literatura}\label{revisuxe3o-sistemuxe1tica-de-literatura}

\subsection{O que é revisão sistemática?}\label{o-que-uxe9-revisuxe3o-sistemuxe1tica}

\begin{itemize}
\tightlist
\item
  Uma revisão sistemática é um tipo de estudo secundário que utiliza métodos explícitos, transparentes e reprodutíveis para identificar, selecionar, avaliar criticamente e sintetizar todas as evidências relevantes sobre uma pergunta claramente definida. Trata-se de uma estratégia formal para reduzir vieses e aumentar a confiabilidade das conclusões, distinguindo-se de revisões narrativas tradicionais por seu rigor metodológico.\textsuperscript{\citeproc{ref-Grant2009}{354},\citeproc{ref-baker2014}{425}}
\end{itemize}

\begin{infobox}{images/Rlogo}
O pacote \emph{easyPubMed}\textsuperscript{\citeproc{ref-easyPubMed}{426}} fornece a função \href{https://cran.r-project.org/web/packages/easyPubMed/index.html}{\emph{get\_pubmed\_ids}} e \href{https://cran.r-project.org/web/packages/easyPubMed/index.html}{\emph{fetch\_pubmed\_data}} para buscar artigos no \href{https://pubmed.ncbi.nlm.nih.gov}{PubMed}.

\end{infobox}

\begin{infobox}{images/Rlogo}
O pacote \emph{rcrossref}\textsuperscript{\citeproc{ref-rcrossref}{427}} fornece a função \href{https://cloud.r-project.org/web/packages/rcrossref/index.html}{\emph{cr\_works}} para buscar artigos.

\end{infobox}

\begin{infobox}{images/Rlogo}
O pacote \emph{roadoi}\textsuperscript{\citeproc{ref-roadoi}{428}} fornece a função \href{https://cloud.r-project.org/web/packages/roadoi/index.html}{\emph{oadoi\_fetch}} para recuperar dados de acesso aberto do \href{https://unpaywall.org}{Unpaywall}.

\end{infobox}

\section{Tipos de revisão sistemática}\label{tipos-de-revisuxe3o-sistemuxe1tica}

\subsection{Quais são os principais tipos de revisão sistemática?}\label{quais-suxe3o-os-principais-tipos-de-revisuxe3o-sistemuxe1tica}

\begin{itemize}
\item
  Revisão sistemática tradicional (\emph{systematic review}): Síntese estruturada de estudos primários (ex.: ensaios clínicos, estudos observacionais) para responder a uma pergunta específica. Caracteriza-se por protocolo explícito, critérios de inclusão definidos e busca exaustiva.\textsuperscript{\citeproc{ref-Grant2009}{354},\citeproc{ref-baker2014}{425}}
\item
  \emph{Overview of Reviews} (também chamada de \emph{review of reviews}): Tipo de síntese que reúne resultados de revisões sistemáticas previamente publicadas. O foco não está em estudos primários, mas nos achados das próprias revisões. É útil quando já existem muitas revisões sistemáticas sobre um mesmo tema e há necessidade de sumarizar o conjunto da evidência disponível de forma organizada.\textsuperscript{\citeproc{ref-Grant2009}{354},\citeproc{ref-baker2014}{425},\citeproc{ref-silva2012}{429},\citeproc{ref-silva2014}{430}}
\item
  \emph{Umbrella Review}: Um subtipo de \emph{overview} em que a unidade de análise é a própria revisão sistemática. Tem escopo mais amplo e busca integrar achados de múltiplas revisões sobre um tópico, permitindo identificar consistências e discrepâncias entre revisões. São cada vez mais comuns devido ao aumento do número de revisões sistemáticas publicadas anualmente.\textsuperscript{\citeproc{ref-Grant2009}{354},\citeproc{ref-stern2025}{431}}
\item
  Revisões sistemáticas qualitativas (\emph{Qualitative evidence synthesis}): Focam em estudos qualitativos para responder questões sobre experiências, percepções ou processos. Podem empregar métodos como meta-aggregation, meta-ethnography ou thematic synthesis. Têm crescido em importância porque muitas questões da prática clínica não podem ser respondidas apenas por estudos experimentais.\textsuperscript{\citeproc{ref-Grant2009}{354},\citeproc{ref-baker2014}{425}}
\item
  Revisões mistas (\emph{Mixed-methods review}): Integram evidências quantitativas e qualitativas. São úteis quando a compreensão de um fenômeno exige tanto medidas objetivas quanto interpretações subjetivas, mas apresentam desafios metodológicos adicionais pela necessidade de combinar técnicas analíticas distintas.\textsuperscript{\citeproc{ref-Grant2009}{354},\citeproc{ref-baker2014}{425}}
\item
  Revisão rápida (\emph{Rapid review}): Uma forma condensada de revisão sistemática, com métodos acelerados (redução de bases de dados, um único revisor em algumas etapas). É útil em contextos que exigem respostas em curto prazo, embora apresente maior risco de viés.\textsuperscript{\citeproc{ref-Grant2009}{354},\citeproc{ref-baker2014}{425}}
\end{itemize}

\subsection{Quais delineamentos de revisão parecem mas não são revisões sistemáticas?}\label{quais-delineamentos-de-revisuxe3o-parecem-mas-nuxe3o-suxe3o-revisuxf5es-sistemuxe1ticas}

\begin{itemize}
\item
  Revisão narrativa estruturada (\emph{Structured narrative review}): Narrativas estruturadas não seguem métodos reprodutíveis como buscas explícitas e critérios formais de inclusão, portanto não devem ser confundidas com revisões sistemáticas, como alerta Baker et al.~(2014), que destaca a importância de diferenciar revisões críticas, integrativas e narrativas de revisões sistemáticas verdadeiras.\textsuperscript{\citeproc{ref-Grant2009}{354},\citeproc{ref-baker2014}{425}}
\item
  Revisão integrativa (\emph{Integrative review}): Embora possa incluir estudos primários, não segue os rigorosos métodos de uma revisão sistemática, como busca exaustiva e critérios de inclusão pré-definidos. Portanto, não deve ser confundida com uma revisão sistemática.\textsuperscript{\citeproc{ref-Grant2009}{354},\citeproc{ref-baker2014}{425}}
\item
  Revisão crítica (\emph{Criticalreview}): Foca na avaliação crítica da literatura, mas não necessariamente segue os métodos sistemáticos de uma revisão sistemática, como busca estruturada e critérios de inclusão claros.\textsuperscript{\citeproc{ref-Grant2009}{354},\citeproc{ref-baker2014}{425}}
\end{itemize}

\section{Diretrizes para redação}\label{diretrizes-para-redauxe7uxe3o-9}

\subsection{Quais são as diretrizes para revisão sistemática?}\label{quais-suxe3o-as-diretrizes-para-revisuxe3o-sistemuxe1tica}

\begin{itemize}
\item
  Visite a rede \emph{Enhancing the QUAlity and Transparency Of health Research} (\href{https://www.equator-network.org/}{EQUATOR Network}) para encontrar diretrizes específicas.
\item
  \emph{Transparent reporting of multivariable prediction models for individual prognosis or diagnosis: checklist for systematic reviews and meta-analyses (TRIPOD-SRMA)}.\textsuperscript{\citeproc{ref-snell2023}{432}} \url{https://www.equator-network.org/reporting-guidelines/tripod-srma/}
\item
  \emph{Critical Appraisal and Data Extraction for Systematic Reviews of Prediction Modelling Studies: The CHARMS Checklist}.\textsuperscript{\citeproc{ref-moons2014}{433}} \url{https://doi.org/10.1371/journal.pmed.1001744}
\end{itemize}

\chapter{\texorpdfstring{\textbf{Meta-análise}}{Meta-análise}}\label{meta-analise}

\section{Características}\label{caracteruxedsticas-6}

\subsection{O que é meta-análise?}\label{o-que-uxe9-meta-anuxe1lise}

\begin{itemize}
\item
  Meta-análise é um método estatístico que combina quantitativamente os resultados de múltiplos estudos independentes sobre uma mesma questão de pesquisa, aumentando o poder estatístico e a precisão das estimativas de efeito.\textsuperscript{\citeproc{ref-borenstein2010}{434}}
\item
  Meta-análise sintetiza evidências considerando o peso de cada estudo (geralmente inversamente proporcional à variância) e permite avaliar a consistência dos resultados, identificar fontes de heterogeneidade e estimar um efeito global.\textsuperscript{\citeproc{ref-borenstein2010}{434}}
\end{itemize}

\section{Modelos de meta-análise}\label{modelos-de-meta-anuxe1lise}

\subsection{Quais são os principais modelos de meta-análise?}\label{quais-suxe3o-os-principais-modelos-de-meta-anuxe1lise}

\begin{itemize}
\item
  Modelo de efeitos fixos: assume que todos os estudos avaliam o mesmo efeito verdadeiro, e a variação observada é apenas devido ao erro de amostragem. É adequado quando os estudos são homogêneos e as diferenças entre eles são pequenas.\textsuperscript{\citeproc{ref-borenstein2010}{434}}
\item
  Modelo de efeitos aleatórios: assume que os estudos avaliam efeitos verdadeiros diferentes, com uma distribuição normal. É mais apropriado quando há heterogeneidade entre os estudos, pois considera a variação entre eles além do erro de amostragem.\textsuperscript{\citeproc{ref-borenstein2010}{434}}
\item
  Modelo de efeitos de rede: estende a meta-análise para comparar múltiplas intervenções simultaneamente, mesmo que não tenham sido comparadas diretamente em estudos. É útil para avaliar a eficácia relativa de várias intervenções.\textsuperscript{\citeproc{ref-REF}{\textbf{REF?}}}
\end{itemize}

\begin{infobox}{images/Rlogo}
O pacote \emph{metafor}\textsuperscript{\citeproc{ref-metafor}{435}} fornece a função \href{https://cran.r-project.org/web/packages/metafor/index.html}{\emph{forest}} para criar figuras tipo \emph{forest plot}.

\end{infobox}

\begin{infobox}{images/Rlogo}
O pacote \emph{netmeta}\textsuperscript{\citeproc{ref-netmeta}{436}} fornece a função \href{https://CRAN.R-project.org/package=netmeta}{\emph{netmeta}} para realizar meta-análise de rede usando método de grafo.

\end{infobox}

\begin{infobox}{images/Rlogo}
O pacote \emph{gemtc}\textsuperscript{\citeproc{ref-gemtc}{437}} fornece a função \href{https://CRAN.R-project.org/package=gemtc}{\emph{mtc.model}} para criar modelos de meta-análise de rede.

\end{infobox}

\begin{figure}

{\centering \includegraphics{Ciencia-com-R_files/figure-latex/ma-comparacao-efeitos-1} 

}

\caption{Comparação entre modelos de efeito fixo e aleatório com 10 ensaios clínicos simulados.}\label{fig:ma-comparacao-efeitos}
\end{figure}

\section{Conversão de Medidas em Meta-análises}\label{conversuxe3o-de-medidas-em-meta-anuxe1lises}

\subsection{O que fazer quando os estudos apresentam resultados com diferentes parâmetros?}\label{o-que-fazer-quando-os-estudos-apresentam-resultados-com-diferentes-paruxe2metros}

\begin{itemize}
\item
  Quando os estudos reportam médias e desvios-padrão, os dados podem ser usados diretamente na metanálise.\textsuperscript{\citeproc{ref-REF}{\textbf{REF?}}}
\item
  Quando apresentam mediana e intervalo interquartil (ou mínimo--máximo), existem métodos estatísticos para converter em média e DP.\textsuperscript{\citeproc{ref-hozo2005}{438}}
\item
  Hozo et al.~(2005) propuseram fórmulas para estimar a média e o desvio-padrão a partir da mediana, amplitude e tamanho da amostra.\textsuperscript{\citeproc{ref-hozo2005}{438}}
\item
  Wan et al.~(2014) aperfeiçoaram essas estimativas, oferecendo métodos mais precisos para converter mediana e IQR em média e DP.\textsuperscript{\citeproc{ref-wan2014}{439}}
\end{itemize}

\begin{infobox}{images/Rlogo}
O pacote \emph{metafor}\textsuperscript{\citeproc{ref-metafor}{435}} fornece a função \href{https://cran.r-project.org/web/packages/metafor/refman/metafor.html\#conv.fivenum}{\emph{conv.fivenum}} para converter mínimo/mediana/máximo ou Q1/mediana/Q2 em média e desvio-padrão.

\end{infobox}

\section{Interpretação de efeitos em meta-análise}\label{interpretauxe7uxe3o-de-efeitos-em-meta-anuxe1lise}

\subsection{Como avaliar a variação do tamanho do efeito?}\label{como-avaliar-a-variauxe7uxe3o-do-tamanho-do-efeito}

\begin{itemize}
\item
  O intervalo de predição contém informação sobre a variação do tamanho do efeito.\textsuperscript{\citeproc{ref-Borenstein2022}{440}}
\item
  Se o intervalo de predição não contém a hipótese nula (\(H_{0}\)), podemos concluir que (a) o tratamento funciona igualmente bem em todas as populações, ou que ele funciona melhor em algumas populações do que em outras.\textsuperscript{\citeproc{ref-Borenstein2022}{440}}
\item
  Se o intervalo de predição contém a hipótese nula (\(H_{0}\)), podemos concluir que o tratamento pode ser benéfico em algumas populações, mas prejudicial em outras, de modo que a estimativa pontual (geralmente a média) torna-se amplamente irrelevante. Nesse caso, é recomendado investigar em que populações o tratamento seria benéfico e em quais causaria danos.\textsuperscript{\citeproc{ref-Borenstein2022}{440}}
\end{itemize}

\subsection{Como avaliar a heterogeneidade entre os estudos?}\label{como-avaliar-a-heterogeneidade-entre-os-estudos}

\begin{itemize}
\tightlist
\item
  A heterogeneidade --- variação não-aleatória --- no efeito do tratamento entre os estudos incluídos em uma meta-análise pode ser avaliada pelo \(I^{2}\) \eqref{eq:i-sqr}.\textsuperscript{\citeproc{ref-Borenstein2022}{440},\citeproc{ref-Ruxfccker2008}{441}}
\end{itemize}

\begin{equation}
\label{eq:i-sqr}
I^{2} = \max \left( 0, \frac{Q - df}{Q} \right) \times 100\%
\end{equation}

\begin{itemize}
\item
  \(I^{2}\) representa qual proporção da variância observada reflete a variância nos efeitos verdadeiros em vez do erro de amostragem.\textsuperscript{\citeproc{ref-Borenstein2022}{440}}
\item
  \(I^{2}\) mede a proporção da variância total que pode ser atribuída à heterogeneidade entre os estudos incluídos.\textsuperscript{\citeproc{ref-Ruxfccker2008}{441}}
\item
  \(I^{2}\) não depende da quantidade de estudos incluídos na meta-análise. Entretanto, \(I^{2}\) aumenta com a quantidade de participantes incluídos nos estudos meta-analisados.\textsuperscript{\citeproc{ref-Ruxfccker2008}{441}}
\item
  A heterogeneidade entre estudos é explicada de modo mais confiável utilizando dados de pacientes individuais, uma vez que a direção verdadeira da modificação de efeito não pode ser observada a partir de dados agregados no estudo.\textsuperscript{\citeproc{ref-degrooth2023}{442}}
\end{itemize}

\begin{infobox}{images/Rlogo}
O pacote \emph{psychmeta}\textsuperscript{\citeproc{ref-psychmeta}{297}} fornece a função \href{https://www.rdocumentation.org/packages/psychmeta/versions/2.7.0/topics/ma_d}{\emph{ma\_d}} para meta-analisar valores \emph{d}.

\end{infobox}

\begin{infobox}{images/Rlogo}
O pacote \emph{psychmeta}\textsuperscript{\citeproc{ref-psychmeta}{297}} fornece a função \href{https://www.rdocumentation.org/packages/psychmeta/versions/2.7.0/topics/ma_r}{\emph{ma\_r}} para meta-analisar correlações.

\end{infobox}

\section{\texorpdfstring{Interpretação do \emph{forest plot}}{Interpretação do forest plot}}\label{interpretauxe7uxe3o-do-forest-plot}

\subsection{\texorpdfstring{O que é um \emph{forest plot}?}{O que é um forest plot?}}\label{o-que-uxe9-um-forest-plot}

\begin{itemize}
\tightlist
\item
  Um \emph{forest plot} é uma representação gráfica dos achados de uma meta-análise. Ele resume os resultados de estudos individuais e apresenta uma estimativa combinada do efeito, permitindo interpretação visual da magnitude, direção e significância estatística dos resultados.\textsuperscript{\citeproc{ref-dettori2021}{443}}
\end{itemize}

\begin{figure}

{\centering \includegraphics{Ciencia-com-R_files/figure-latex/ma-efeito-fixo-1} 

}

\caption{Forest plot de uma meta-análise de efeito fixo com 10 ensaios clínicos simulados.}\label{fig:ma-efeito-fixo}
\end{figure}

\begin{figure}

{\centering \includegraphics{Ciencia-com-R_files/figure-latex/ma-efeito-aleatorio-1} 

}

\caption{Forest plot de uma meta-análise de efeito aleatório com 10 ensaios clínicos simulados.}\label{fig:ma-efeito-aleatorio}
\end{figure}

\subsection{\texorpdfstring{Quais são as seis colunas básicas que um \emph{forest plot} geralmente apresenta?}{Quais são as seis colunas básicas que um forest plot geralmente apresenta?}}\label{quais-suxe3o-as-seis-colunas-buxe1sicas-que-um-forest-plot-geralmente-apresenta}

\begin{itemize}
\tightlist
\item
  As seis colunas básicas incluem: estudos incluídos (e subgrupos, se analisados); dados do grupo de intervenção, dados do grupo controle; peso de cada estudo; medida numérica do efeito; representação gráfica do efeito.\textsuperscript{\citeproc{ref-dettori2021}{443}}
\end{itemize}

\subsection{\texorpdfstring{Como diferenciar um desfecho binário de um contínuo em um \emph{forest plot}?}{Como diferenciar um desfecho binário de um contínuo em um forest plot?}}\label{como-diferenciar-um-desfecho-binuxe1rio-de-um-contuxednuo-em-um-forest-plot}

\begin{itemize}
\item
  Em desfechos binários, são mostrados número de eventos e total da amostra, sendo o efeito medido por \emph{risk ratio} (\(RR\)) ou \emph{odds ratio} (\(OR\)).\textsuperscript{\citeproc{ref-dettori2021}{443}}
\item
  Em desfechos contínuos, apresentam-se médias, desvios-padrão e tamanhos amostrais, com o efeito medido pela diferença de médias.\textsuperscript{\citeproc{ref-dettori2021}{443}}
\end{itemize}

\subsection{O que representa o ponto central da caixa e o tamanho desta no gráfico?}\label{o-que-representa-o-ponto-central-da-caixa-e-o-tamanho-desta-no-gruxe1fico}

\begin{itemize}
\item
  O ponto central indica a estimativa pontual do efeito (melhor estmativa para o efeito real).\textsuperscript{\citeproc{ref-dettori2021}{443}}
\item
  O tamanho da caixa é proporcional ao peso do estudo na meta-análise, geralmente maior para estudos com amostras maiores.\textsuperscript{\citeproc{ref-dettori2021}{443}}
\end{itemize}

\subsection{Qual é o significado da linha vertical do ``nenhum efeito''?}\label{qual-uxe9-o-significado-da-linha-vertical-do-nenhum-efeito}

\begin{itemize}
\item
  É a linha de referência que indica efeito nulo.\textsuperscript{\citeproc{ref-dettori2021}{443}}
\item
  Para desfechos binários, corresponde ao valor 1 (\(RR\) ou \(OR\) = 1).\textsuperscript{\citeproc{ref-dettori2021}{443}}
\item
  Para desfechos contínuos, corresponde ao valor 0 (diferença de médias = 0).\textsuperscript{\citeproc{ref-dettori2021}{443}}
\item
  Se o intervalo de confiança de um estudo ou do resultado combinado cruza essa linha, o resultado não é estatisticamente significativo.\textsuperscript{\citeproc{ref-dettori2021}{443}}
\end{itemize}

\subsection{\texorpdfstring{Como interpretar o diamante na parte inferior do \emph{forest plot}?}{Como interpretar o diamante na parte inferior do forest plot?}}\label{como-interpretar-o-diamante-na-parte-inferior-do-forest-plot}

\begin{itemize}
\item
  O diamante representa o efeito combinado dos estudos incluídos.\textsuperscript{\citeproc{ref-dettori2021}{443}}
\item
  O ponto central do diamante é a estimativa global.\textsuperscript{\citeproc{ref-dettori2021}{443}}
\item
  A largura do diamante representa o intervalo de confiança de 95\% para o efeito combinado.\textsuperscript{\citeproc{ref-dettori2021}{443}}
\end{itemize}

\subsection{\texorpdfstring{Como a heterogeneidade pode ser avaliada no \emph{forest plot}?}{Como a heterogeneidade pode ser avaliada no forest plot?}}\label{como-a-heterogeneidade-pode-ser-avaliada-no-forest-plot}

\begin{itemize}
\tightlist
\item
  A variabilidade nos resultados dos estudos incluídos é avaliada pela sobreposição dos intervalos de confiança dos estudos; pelo teste do qui-quadrado (\(\chi^2\)) e pelo valor de \(I^{2}\).\textsuperscript{\citeproc{ref-dettori2021}{443}}
\end{itemize}

\subsection{Quais são as interpretações usuais para os valores de heterogeneidade?}\label{quais-suxe3o-as-interpretauxe7uxf5es-usuais-para-os-valores-de-heterogeneidade}

\begin{itemize}
\tightlist
\item
  \(I^{2}\) de 0\% a 40\%: pode não ser importante; 30\% a 60\%: heterogeneidade moderada; 50\% a 90\%: heterogeneidade substancial; 75\% a 100\%: heterogeneidade considerável.\textsuperscript{\citeproc{ref-dettori2021}{443}}
\end{itemize}

\begin{figure}

{\centering \includegraphics{Ciencia-com-R_files/figure-latex/ma-forest-plot-i2-1} 

}

\caption{Forest plots ilustrativos para faixas usuais de $I^2$.}\label{fig:ma-forest-plot-i2}
\end{figure}

\section{Viés de publicação em meta-análises}\label{viuxe9s-de-publicauxe7uxe3o-em-meta-anuxe1lises}

\subsection{O que é viés de publicação?}\label{o-que-uxe9-viuxe9s-de-publicauxe7uxe3o}

\begin{itemize}
\tightlist
\item
  O viés de publicação ocorre quando estudos com resultados não significativos ou contrários à hipótese tendem a não ser publicados, afetando a estimativa final da meta-análise e podendo levar a conclusões incorretas.\textsuperscript{\citeproc{ref-song2000}{444}}
\end{itemize}

\subsection{Quais métodos podem ser usados para identificar viés de publicação?}\label{quais-muxe9todos-podem-ser-usados-para-identificar-viuxe9s-de-publicauxe7uxe3o}

\begin{itemize}
\item
  O método mais simples é o \emph{funnel plot}, que representa a estimativa de efeito de cada estudo em função de sua precisão (\(1/SE\)).\textsuperscript{\citeproc{ref-egger1997}{445}}
\item
  Na ausência de viés, espera-se uma distribuição simétrica (``forma de funil''). Assimetria pode indicar viés de publicação, heterogeneidade entre estudos ou efeitos de tamanho de estudo (\emph{small-study effects}).\textsuperscript{\citeproc{ref-egger1997}{445}}
\item
  Para \emph{odds ratios} (\(OR\)), a correlação entre \(ln(OR)\) e seu erro padrão pode gerar assimetria mesmo sem viés, por isso recomenda-se, nesses casos, plotar em função do tamanho amostral.\textsuperscript{\citeproc{ref-peters2006}{446}}
\end{itemize}

\subsection{\texorpdfstring{O que é um gráfico de funil (\emph{funnel plot})?}{O que é um gráfico de funil (funnel plot)?}}\label{o-que-uxe9-um-gruxe1fico-de-funil-funnel-plot}

\begin{itemize}
\tightlist
\item
  É um gráfico de dispersão que relaciona a estimativa de efeito de cada estudo com uma medida de seu tamanho ou precisão (por exemplo, erro-padrão no eixo vertical, em escala invertida). Em condições ideais (ausência de viés e heterogeneidade), os estudos se distribuem de forma simétrica, formando um ``funil invertido''.\textsuperscript{\citeproc{ref-sterne2011}{447}}
\end{itemize}

\begin{figure}

{\centering \includegraphics{Ciencia-com-R_files/figure-latex/ma-funnel-plot-1} 

}

\caption{Gráficos de funil simulados com baixa e alta heterogeneidade.}\label{fig:ma-funnel-plot}
\end{figure}

\subsection{\texorpdfstring{A assimetria do \emph{funnel plot} indica sempre viés de publicação?}{A assimetria do funnel plot indica sempre viés de publicação?}}\label{a-assimetria-do-funnel-plot-indica-sempre-viuxe9s-de-publicauxe7uxe3o}

\begin{itemize}
\item
  Viéses de relato (\emph{reporting biases}), como viés de publicação, viés de linguagem ou de citação.\textsuperscript{\citeproc{ref-sterne2011}{447}}
\item
  Diferenças metodológicas entre estudos pequenos e grandes.\textsuperscript{\citeproc{ref-sterne2011}{447}}
\item
  Heterogeneidade verdadeira (diferença real no efeito conforme o tamanho ou o contexto do estudo).\textsuperscript{\citeproc{ref-sterne2011}{447}}
\item
  Artefatos estatísticos ou mero acaso.\textsuperscript{\citeproc{ref-sterne2011}{447}}
\end{itemize}

\subsection{\texorpdfstring{O que é \emph{trim and fill}?}{O que é trim and fill?}}\label{o-que-uxe9-trim-and-fill}

\begin{itemize}
\item
  O método \emph{trim and fill} ``apara'' (trim) os estudos que causam assimetria no funnel plot, estima o número de estudos possivelmente ausentes (não publicados) e ``preenche'' (fill) o gráfico com esses estudos simulados, recalculando o efeito combinado.\textsuperscript{\citeproc{ref-duval2000}{448}}
\item
  O método assume que a assimetria é causada unicamente por viés de publicação, podendo levar a conclusões equivocadas quando há outras causas, como heterogeneidade.\textsuperscript{\citeproc{ref-duval2000}{448}}
\end{itemize}

\subsection{O que é o teste de Egger?}\label{o-que-uxe9-o-teste-de-egger}

\begin{itemize}
\item
  É um teste estatístico amplamente utilizado que avalia a relação entre o efeito padronizado (\(efeito/SE\)) e a precisão (\(1/SE\)).\textsuperscript{\citeproc{ref-egger1997}{445}}
\item
  No entanto, para meta-análises com \(OR\), apresenta taxas de erro tipo I excessivas, especialmente quando o efeito é grande ou há alta heterogeneidade.\textsuperscript{\citeproc{ref-peters2006}{446}}
\end{itemize}

\subsection{O que é o teste de Peters?}\label{o-que-uxe9-o-teste-de-peters}

\begin{itemize}
\item
  Uma regressão linear ponderada com \(ln(OR)\) como variável dependente e o inverso do tamanho total da amostra como variável independente (modificação do teste de Macaskill).\textsuperscript{\citeproc{ref-peters2006}{446}}
\item
  Essa abordagem reduz a correlação entre \(ln(OR)\) e seu \(SE\), resultando em taxas de erro tipo I mais adequadas (\textasciitilde10\%) independentemente do tamanho do efeito, número de estudos ou heterogeneidade.\textsuperscript{\citeproc{ref-peters2006}{446}}
\item
  O teste de Peters é preferível ao teste de Egger quando o desfecho é expresso como OR, pois mantém taxas de erro tipo I adequadas e ainda apresenta poder comparável para detectar viés em condições de baixa heterogeneidade.\textsuperscript{\citeproc{ref-peters2006}{446}}
\item
  Em casos de alta heterogeneidade, o teste de Egger pode apresentar maior poder, mas sua alta taxa de falsos positivos compromete a interpretação.\textsuperscript{\citeproc{ref-peters2006}{446}}
\end{itemize}

\subsection{Quais são as recomendações para testar a assimetria?}\label{quais-suxe3o-as-recomendauxe7uxf5es-para-testar-a-assimetria}

\begin{itemize}
\item
  Evitar testes quando há menos de 10 estudos, devido ao baixo poder estatístico.\textsuperscript{\citeproc{ref-sterne2011}{447}}
\item
  Inspecionar visualmente o \emph{funnel plot} junto com os resultados dos testes.\textsuperscript{\citeproc{ref-sterne2011}{447}}
\item
  Para desfechos contínuos (diferença de médias), o teste de Egger pode ser usado.\textsuperscript{\citeproc{ref-sterne2011}{447}}
\item
  Para desfechos dicotômicos expressos como \emph{odds ratio} (\(OR\)) com baixa heterogeneidade (\(\tau^2 < 0,1\)), considerar os testes de Harbord, Peters ou Rücker.\textsuperscript{\citeproc{ref-sterne2011}{447}}
\item
  Para desfechos dicotômicos expressos como \emph{odds ratio}(\(OR\)) com alta heterogeneidade (\(\tau^2 > 0,1\)), o teste de Rücker com transformação \(arcsine\) é mais indicado.\textsuperscript{\citeproc{ref-sterne2011}{447}}
\end{itemize}

\subsection{Como interpretar os resultados de testes de viés de publicação?}\label{como-interpretar-os-resultados-de-testes-de-viuxe9s-de-publicauxe7uxe3o}

\begin{itemize}
\item
  Um resultado não significativo não garante ausência de viés.\textsuperscript{\citeproc{ref-peters2006}{446}}
\item
  Recomenda-se complementar com inspeção visual do \emph{funnel plot} e considerar métodos adicionais como \emph{trim and fill}.\textsuperscript{\citeproc{ref-peters2006}{446},\citeproc{ref-duval2000}{448}}
\item
  Quando há suspeita de viés, discutir as implicações e interpretar o efeito combinado com cautela.\textsuperscript{\citeproc{ref-peters2006}{446}}
\end{itemize}

\begin{infobox}{images/Rlogo}
O pacote \emph{psychmeta}\textsuperscript{\citeproc{ref-psychmeta}{297}} fornece a função \href{https://www.rdocumentation.org/packages/psychmeta/versions/2.7.0/topics/plot_funnel}{\emph{plot\_funnel}} para criar figuras tipo \emph{funnel plot}.

\end{infobox}

\section{Diretrizes para redação}\label{diretrizes-para-redauxe7uxe3o-10}

\subsection{Quais são as diretrizes para redação de meta-análises?}\label{quais-suxe3o-as-diretrizes-para-redauxe7uxe3o-de-meta-anuxe1lises}

\begin{itemize}
\item
  Visite a rede \emph{Enhancing the QUAlity and Transparency Of health Research} (\href{https://www.equator-network.org/}{EQUATOR Network}) para encontrar diretrizes específicas.
\item
  \emph{The PRISMA 2020 statement: An updated guideline for reporting systematic reviews}:\textsuperscript{\citeproc{ref-page2021}{449}} \url{https://www.equator-network.org/reporting-guidelines/prisma/}
\item
  \emph{Transparent reporting of multivariable prediction models for individual prognosis or diagnosis: checklist for systematic reviews and meta-analyses (TRIPOD-SRMA)}.\textsuperscript{\citeproc{ref-snell2023}{432}} \url{https://www.equator-network.org/reporting-guidelines/tripod-srma/}
\end{itemize}

\begin{infobox}{images/Rlogo}
O pacote \emph{metagear}\textsuperscript{\citeproc{ref-metagear}{450}} fornece a função \href{https://www.rdocumentation.org/packages/metagear/versions/0.7/topics/plot_PRISMA}{\emph{plot\_PRISMA}} para gerar o fluxograma de uma revisão sistemática de acordo com o \emph{Preferred Reporting Items for Systematic Reviews and Meta-Analyses}\textsuperscript{\citeproc{ref-Moher2015}{451}}.

\end{infobox}

\begin{infobox}{images/Rlogo}
O pacote \emph{PRISMA2020}\textsuperscript{\citeproc{ref-PRISMA2020}{452}} fornece a função \href{https://www.rdocumentation.org/packages/PRISMA2020/versions/1.1.1/topics/PRISMA_flowdiagram}{\emph{PRISMA\_flowdiagram}} para elaboração do fluxograma de revisões sistemáticas no formato padrão.

\end{infobox}

\chapter{\texorpdfstring{\textbf{Revisão guarda-chuva}}{Revisão guarda-chuva}}\label{revisao-guarda-chuva}

\section{Revisão guarda-chuva}\label{revisuxe3o-guarda-chuva}

\subsection{O que é revisão guarda-chuva?}\label{o-que-uxe9-revisuxe3o-guarda-chuva}

\begin{itemize}
\tightlist
\item
  .\textsuperscript{\citeproc{ref-REF}{\textbf{REF?}}}
\end{itemize}

\section{Diretrizes para redação}\label{diretrizes-para-redauxe7uxe3o-11}

\subsection{Quais são as diretrizes para revisão sistemática?}\label{quais-suxe3o-as-diretrizes-para-revisuxe3o-sistemuxe1tica-1}

\begin{itemize}
\item
  Visite a rede \emph{Enhancing the QUAlity and Transparency Of health Research} (\href{https://www.equator-network.org/}{EQUATOR Network}) para encontrar diretrizes específicas.
\item
  \emph{Reporting guideline for overviews of reviews of healthcare interventions: development of the PRIOR statement}.\textsuperscript{\citeproc{ref-gates2022}{453}} \url{https://www.equator-network.org/reporting-guidelines/reporting-guideline-for-overviews-of-reviews-of-healthcare-interventions-development-of-the-prior-statement/}
\end{itemize}

\chapter{\texorpdfstring{\textbf{Pesquisa qualitativa}}{Pesquisa qualitativa}}\label{pesquisa-qualitativa}

\section{Pesquisa qualitativa}\label{pesquisa-qualitativa-1}

\subsection{O que é pesquisa qualitativa?}\label{o-que-uxe9-pesquisa-qualitativa}

\begin{itemize}
\item
  No contexto de ensaios clínicos randomizados, pesquisa qualitativa é usada para compreender a complexidade das intervenções e dos contextos sociais em que são testadas, contribuindo para gerar evidências de efetividade.\textsuperscript{\citeproc{ref-ocathain2013}{454}}
\item
  O valor potencial da pesquisa qualitativa inclui: otimizar a intervenção e procedimentos dos ensaios clínicos randomizados; facilitar a interpretação dos achados; fortalecer a condução ética; ampliar a validade externa ; e economizar recursos ao direcionar futuros ensaios clínicos randomizados para intervenções mais promissoras.\textsuperscript{\citeproc{ref-ocathain2013}{454}}
\end{itemize}

\section{Diretrizes para redação}\label{diretrizes-para-redauxe7uxe3o-12}

\subsection{Quais são as diretrizes para redação de estudos de propriedades psicométricas?}\label{quais-suxe3o-as-diretrizes-para-redauxe7uxe3o-de-estudos-de-propriedades-psicomuxe9tricas-1}

\begin{itemize}
\item
  Visite a rede \emph{Enhancing the QUAlity and Transparency Of health Research} (\href{https://www.equator-network.org/}{EQUATOR Network}) para encontrar diretrizes específicas.
\item
  \emph{Standards for reporting qualitative research: a synthesis of recommendations}:\textsuperscript{\citeproc{ref-obrien2014}{455}} \url{https://www.equator-network.org/reporting-guidelines/srqr/}
\item
  \emph{Enhancing transparency in reporting the synthesis of qualitative research: ENTREQ}:\textsuperscript{\citeproc{ref-tong2012}{456}} \url{https://www.equator-network.org/reporting-guidelines/entreq/}
\item
  \emph{Consolidated criteria for reporting qualitative research (COREQ) : a 32-item checklist for interviews and focus groups}:\textsuperscript{\citeproc{ref-tong2007}{457}} \url{https://www.equator-network.org/reporting-guidelines/coreq/}
\end{itemize}

\cftaddtitleline{toc}{chapter}{\rule{\textwidth}{0.4pt}}{}

\chapter*{\texorpdfstring{\emph{PARTE 10: COMUNICAÇÃO E RELATO CIENTÍFICO}}{PARTE 10: COMUNICAÇÃO E RELATO CIENTÍFICO}}\label{parte-10}
\addcontentsline{toc}{chapter}{\emph{PARTE 10: COMUNICAÇÃO E RELATO CIENTÍFICO}}

\par\noindent\rule{\textwidth}{0.05in}

\section*{Transformando resultados em narrativas claras, completas e alinhadas às boas práticas}\label{transformando-resultados-em-narrativas-claras-completas-e-alinhadas-uxe0s-boas-pruxe1ticas}

\markboth{}{}

\chapter{\texorpdfstring{\textbf{Redação de resultados}}{Redação de resultados}}\label{redacao-resultados}

\section{Resultados da análise estatística}\label{resultados-da-anuxe1lise-estatuxedstica}

\subsection{Como redigir os resultados da análise estatística?}\label{como-redigir-os-resultados-da-anuxe1lise-estatuxedstica}

\begin{itemize}
\tightlist
\item
  .\textsuperscript{\citeproc{ref-REF}{\textbf{REF?}}}
\end{itemize}

\begin{infobox}{images/Rlogo}
O pacote \emph{report}\textsuperscript{\citeproc{ref-report}{458}} fornece a função \href{https://www.rdocumentation.org/packages/report/versions/0.5.8/topics/report}{\emph{report}} para redigir a descrição de diversas análises estatísticas.

\end{infobox}

\begin{infobox}{images/Rlogo}
O pacote \emph{statcheck}\textsuperscript{\citeproc{ref-statcheck}{459}} fornece a função \href{https://www.rdocumentation.org/packages/statcheck/versions/1.4.0/topics/statcheck}{\emph{statcheck}} para extrair resultados de testes de significância de hipótese nula.

\end{infobox}

\section{Diretrizes e Listas}\label{diretrizes-e-listas}

\subsection{Quais diretrizes estão disponíveis para redação estatística?}\label{quais-diretrizes-estuxe3o-disponuxedveis-para-redauxe7uxe3o-estatuxedstica}

\begin{itemize}
\item
  \emph{Review of guidance papers on regression modeling in statistical series of medical journals}.\textsuperscript{\citeproc{ref-Wallisch2022}{460}}
\item
  \emph{Principles and recommendations for incorporating estimands into clinical study protocol templates}.\textsuperscript{\citeproc{ref-Lynggaard2022}{461}}
\item
  \emph{How to write statistical analysis section in medical research}.\textsuperscript{\citeproc{ref-Dwivedi2022}{254}}
\item
  \emph{Recommendations for Statistical Reporting in Cardiovascular Medicine: A Special Report From the American Heart Association}.\textsuperscript{\citeproc{ref-Althouse2021}{462}}
\item
  \emph{Framework for the treatment and reporting of missing data in observational studies: The Treatment And Reporting of Missing data in Observational Studies framework}.\textsuperscript{\citeproc{ref-Lee2021a}{463}}
\item
  \emph{Guidelines for reporting of figures and tables for clinical research in urology}.\textsuperscript{\citeproc{ref-Vickers2020}{464}}
\item
  \emph{Who is in this study, anyway? Guidelines for a useful Table 1}.\textsuperscript{\citeproc{ref-Hayes-Larson2019}{219}}
\item
  \emph{Guidelines for Reporting of Statistics for Clinical Research in Urology}.\textsuperscript{\citeproc{ref-assel2019}{465}}
\item
  \emph{Reveal, Don't Conceal: Transforming Data Visualization to Improve Transparency}.\textsuperscript{\citeproc{ref-Weissgerber2019}{208}}
\item
  \emph{Guidelines for the Content of Statistical Analysis Plans in Clinical Trials}.\textsuperscript{\citeproc{ref-Gamble2017}{352}}
\item
  \emph{Basic statistical reporting for articles published in Biomedical Journals: The `'Statistical Analyses and Methods in the Published Literature'\,' or the SAMPL Guidelines}.\textsuperscript{\citeproc{ref-Lang2015}{466}}
\item
  \emph{Beyond Bar and Line Graphs: Time for a New Data Presentation Paradigm}.\textsuperscript{\citeproc{ref-Weissgerber2015}{467}}
\item
  \emph{STRengthening analytical thinking for observational studies: the STRATOS initiative}.\textsuperscript{\citeproc{ref-Sauerbrei2014}{468}}
\item
  \emph{Research methods and reporting}.\textsuperscript{\citeproc{ref-groves2008}{469}}
\item
  \emph{How to ensure your paper is rejected by the statistical reviewer}.\textsuperscript{\citeproc{ref-stratton2005}{470}}
\end{itemize}

\subsection{Quais listas de verificação estão disponíveis para redação estatística?}\label{quais-listas-de-verificauxe7uxe3o-estuxe3o-disponuxedveis-para-redauxe7uxe3o-estatuxedstica}

\begin{itemize}
\item
  \emph{A CHecklist for statistical Assessment of Medical Papers (the CHAMP statement): explanation and elaboration}.\textsuperscript{\citeproc{ref-Mansournia2021}{471}}
\item
  \emph{Checklist for clinical applicability of subgroup analysis}.\textsuperscript{\citeproc{ref-Gil-Sierra2020}{472}}
\item
  \emph{Evidence-based statistical analysis and methods in biomedical research (SAMBR) checklists according to design features}.\textsuperscript{\citeproc{ref-dwivedi2019}{253}}
\end{itemize}

\chapter{\texorpdfstring{\textbf{Diretrizes e Listas}}{Diretrizes e Listas}}\label{diretrizes-listas}

\section{Diretrizes}\label{diretrizes}

\subsection{Quais são as diretrizes para relatórios estatísticos em pesquisas?}\label{quais-suxe3o-as-diretrizes-para-relatuxf3rios-estatuxedsticos-em-pesquisas}

\begin{itemize}
\item
  Visite a rede \emph{Enhancing the QUAlity and Transparency Of health Research} (\href{https://www.equator-network.org/}{EQUATOR Network}) para encontrar diretrizes específicas.
\item
  \emph{Guidelines for Reporting Observational Research in Urology: The Importance of Clear Reference to Causality}.\textsuperscript{\citeproc{ref-vickers2023}{243}}
\item
  \emph{Review of guidance papers on regression modeling in statistical series of medical journals}.\textsuperscript{\citeproc{ref-Wallisch2022}{460}}
\item
  \emph{Principles and recommendations for incorporating estimands into clinical study protocol templates}.\textsuperscript{\citeproc{ref-Lynggaard2022}{461}}
\item
  \emph{How to write statistical analysis section in medical research}.\textsuperscript{\citeproc{ref-Dwivedi2022}{254}}
\item
  \emph{A Guideline for Reporting Mediation Analyses of Randomized Trials and Observational Studies: The AGReMA Statement}.\textsuperscript{\citeproc{ref-lee2021b}{473}}
\item
  \emph{Recommendations for Statistical Reporting in Cardiovascular Medicine: A Special Report From the American Heart Association}.\textsuperscript{\citeproc{ref-Althouse2021}{462}}
\item
  \emph{Framework for the treatment and reporting of missing data in observational studies: The Treatment And Reporting of Missing data in Observational Studies framework}.\textsuperscript{\citeproc{ref-Lee2021a}{463}}
\item
  \emph{Guidelines for reporting of figures and tables for clinical research in urology}.\textsuperscript{\citeproc{ref-Vickers2020}{464}}
\item
  \emph{Who is in this study, anyway? Guidelines for a useful Table 1}.\textsuperscript{\citeproc{ref-Hayes-Larson2019}{219}}
\item
  \emph{Guidelines for Reporting of Statistics for Clinical Research in Urology}.\textsuperscript{\citeproc{ref-assel2019}{465}}
\item
  \emph{Reveal, Don't Conceal: Transforming Data Visualization to Improve Transparency}.\textsuperscript{\citeproc{ref-Weissgerber2019}{208}}
\item
  \emph{Guidelines for the Content of Statistical Analysis Plans in Clinical Trials}.\textsuperscript{\citeproc{ref-Gamble2017}{352}}
\item
  \emph{Basic statistical reporting for articles published in Biomedical Journals: The `'Statistical Analyses and Methods in the Published Literature'\,' or the SAMPL Guidelines}.\textsuperscript{\citeproc{ref-Lang2015}{466}}
\item
  \emph{Beyond Bar and Line Graphs: Time for a New Data Presentation Paradigm}.\textsuperscript{\citeproc{ref-Weissgerber2015}{467}}
\item
  \emph{STRengthening analytical thinking for observational studies: the STRATOS initiative}.\textsuperscript{\citeproc{ref-Sauerbrei2014}{468}}
\item
  \emph{Research methods and reporting}.\textsuperscript{\citeproc{ref-groves2008}{469}}
\item
  \emph{How to ensure your paper is rejected by the statistical reviewer}.\textsuperscript{\citeproc{ref-stratton2005}{470}}
\end{itemize}

\section{Listas de verificação}\label{listas-de-verificauxe7uxe3o}

\subsection{Quais são as listas de verificação para relatórios estatísticos em pesquisas?}\label{quais-suxe3o-as-listas-de-verificauxe7uxe3o-para-relatuxf3rios-estatuxedsticos-em-pesquisas}

\begin{itemize}
\item
  \emph{A CHecklist for statistical Assessment of Medical Papers (the CHAMP statement): explanation and elaboration}.\textsuperscript{\citeproc{ref-Mansournia2021}{471}}
\item
  \emph{Checklist for clinical applicability of subgroup analysis}.\textsuperscript{\citeproc{ref-Gil-Sierra2020}{472}}
\item
  \emph{Evidence-based statistical analysis and methods in biomedical research (SAMBR) checklists according to design features}.\textsuperscript{\citeproc{ref-dwivedi2019}{253}}
\end{itemize}

\cftaddtitleline{toc}{chapter}{\rule{\textwidth}{0.4pt}}{}

\chapter*{\texorpdfstring{\emph{REFERÊNCIAS}}{REFERÊNCIAS}}\label{parte-11}
\addcontentsline{toc}{chapter}{\emph{REFERÊNCIAS}}

\par\noindent\rule{\textwidth}{0.05in}

\markboth{}{}

\chapter*{\texorpdfstring{\textbf{Produção científica do autor}}{Produção científica do autor}}\label{producao-cientifica}
\addcontentsline{toc}{chapter}{\textbf{Produção científica do autor}}

\section*{Artigos em periódicos científicos}\label{artigos-em-periuxf3dicos-cientuxedficos}
\addcontentsline{toc}{section}{Artigos em periódicos científicos}

\begin{enumerate}
\def\labelenumi{\arabic{enumi}.}
\item
  Barros I de A, Ferreira A de S, Horsth T de S, Holmes T de J, dos Santos AA, Lunkes LC. Type of health locus of control predicting pain, function, and global perceived effect in patients with chronic low back pain receiving active versus passive interventions: an observational study. Brazilian Journal of Physical Therapy. 2026;30(2):101560. \url{doi:10.1016/j.bjpt.2025.101560}
\item
  Amaravadi SK, Ferreira A de S, Vigário P dos S. Comparative effects of combined aerobic and resistance training versus high-intensity interval training on insulin resistance, glycaemic control, body composition and quality of life in type 2 diabetes: A 12-week randomised controlled trial. Cobucci RNO, ed.~PLOS One. 2025;20(12):e0336898. \url{doi:10.1371/journal.pone.0336898}
\item
  Parisotto G, Sant'Anna Junior M, Papathanasiou J, Reis LF da F, Ferreira A de S. Functional determinants and perceived barriers to cardiac rehabilitation as predictors of short-term hospital readmission in acute coronary syndrome: an observational longitudinal cohort study. Acta Cardiologica. 2025;80(10):1102-1111. \url{doi:10.1080/00015385.2025.2576437}
\item
  Correia IMT, Ferreira A de S, Gomes JFM, Reis FJJ, Nogueira LAC, Meziat-Filho N. Cervical flexion posture during smartphone use was not a risk factor for neck pain, but low sleep quality and insufficient levels of physical activity were. A longitudinal investigation. Brazilian Journal of Physical Therapy. 2025;29(6):101258. \url{doi:10.1016/j.bjpt.2025.101258}
\item
  de Souza Horsth T, de Araujo Barros I, de Souza RC, de Sá Ferreira A, de Almeida RS. Heart rate variability responses to instrument-assisted atlas (C1) chiropractic manipulation: A randomized placebo-controlled trial. Journal of Bodywork and Movement Therapies. 2025;44:784-788. \url{doi:10.1016/j.jbmt.2025.06.027}
\item
  Ferreira ALNV, Ferreira A de S. POP do ``quarto terapêutico'' proposto pela equipe profissional aos pacientes com neoplasia de tireoide com Iodoradioativo I-131. Revista JRG de Estudos Acadêmicos. 2025;8(18):e082222. \url{doi:10.55892/jrg.v8i18.2222}
\item
  Monteiro ER, Aguilera LM, Ruá-Alonso M, et al.~Effect of Manual Massage, Foam Rolling, and Strength Training on Hemodynamic and Autonomic Responses in Adults: A Scoping Review. Healthcare. 2025;13(12):1371. \url{doi:10.3390/healthcare13121371}
\item
  Monteiro ER, de Oliveira Muniz Cunha JC, de Souza Horsth T, et al.~Effect of cervical manipulation on blood pressure and heart rate variability responses in adults: A scoping review. Journal of Bodywork and Movement Therapies. 2025;42:1120-1127. \url{doi:10.1016/j.jbmt.2025.03.023}
\item
  Resende PA, Tavares Correia IM, de Sá Ferreira A, Meziat-Filho N, Lunkes LC. Neck pain and text neck using Hill's criteria of causation: A scoping review. Journal of Bodywork and Movement Therapies. 2025;42:132-138. \url{doi:10.1016/j.jbmt.2024.12.016}
\item
  Ferreira AS, Parisotto G. Rethinking methodologies in nocturnal blood pressure dipping research: insights from Lopez et~al.~Acta Cardiologica. 2025;80(5):529-530. \url{doi:10.1080/00015385.2025.2453799}
\item
  Da Costa CCM, Olímpio Júnior H, Da Silva Pinto PVL, et al.~Contribution of small airway disease to dynamic hyperinflation in patients with chronic obstructive pulmonary disease. Monaldi Archives for Chest Disease. May 2025. \url{doi:10.4081/monaldi.2025.3402}
\item
  Reis FJJ, Neves G de A, Carvalho MBL de, et al.~Mapping global research on artificial intelligence in physical therapy: a bibliometric analysis from 1990 to 2023. European Journal of Physiotherapy. May 2025:1-11. \url{doi:10.1080/21679169.2025.2497780}
\item
  Ferreira ALNV, De Sá Ferreira A. Contribuições do POP à assistência de enfermagem ao tratamento da neoplasia de tireoide no quarto terapêutico. Revista JRG de Estudos Acadêmicos. 2025;8(18):e082001. \url{doi:10.55892/jrg.v8i18.2001}
\item
  Shanmugam S, Anjos FV dos, Ferreira A de S, Muthukrishnan R, Kandakurti PK, Durairaj S. Effectiveness of intramuscular electrical stimulation using conventional and inverse electrode placement methods on pressure pain threshold and electromyographic activity of the upper trapezius muscle with myofascial trigger points: a randomized clinical trial. The Korean Journal of Pain. 2025;38(2):187-197. \url{doi:10.3344/kjp.24332}
\item
  Alaparthi GK, Moustafa IM, Lopes AJ, Ferreira A de S. Pulmonary function, body posture and balance in young adults with asthma: A cross-sectional study. Kweh B, ed.~PLOS ONE. 2025;20(3):e0316663. \url{doi:10.1371/journal.pone.0316663}
\item
  Silva M de C, Ferreira A de S, Baldon R de M, et al.~Immediate Effects of Manual Therapy Techniques on Ankle Dorsiflexion: A Randomized Clinical Trial. Journal of Manipulative and Physiological Therapeutics. 2025;48(1-5):166-176. \url{doi:10.1016/j.jmpt.2025.09.002}
\item
  Bastos JRM, Ferreira AS, Lopes AJ, Pinto TP, Rodrigues E, dos Anjos FV. The Tinetti Balance Test Is an Effective Predictor of Functional Decline in Non-Hospitalized Post-COVID-19 Individuals: A Cross-Sectional Study. Journal of Clinical Medicine. 2024;13(21):6626. \url{doi:10.3390/jcm13216626}
\item
  Vieira L, De Sá Ferreira A, Meziat Filho NA, Frare JC, Zaidan de Barros P, Santos de Almeida R. Body composition does not interfere with urinary incontinence in adult women with grade III obesity. Revista de Ciências Médicas e Biológicas. 2024;23(2):389-393. \url{doi:10.9771/cmbio.v23i2.57359}
\item
  Kumar P, Ferreira A de S, Nogueira LAC, Arulsingh W, Patil MrS. Influence of Kinesiophobia on muscle endurance in patients with chronic low back pain- A case-control study. F1000Research. 2024;13:1016. \url{doi:10.12688/f1000research.152751.1}
\item
  Banks HC, Lemos T, Oliveira LAS, Ferreira AS. Short-term effects of Pilates-based exercise on upper limb strength and function in people with Parkinson's disease. Journal of Bodywork and Movement Therapies. 2024;39:237-242. \url{doi:10.1016/j.jbmt.2024.02.032}
\item
  Ali NM, Alaparthi GK, de Sá Ferreira A, Arumugam A, Bairapareddy KC. A national survey of physiotherapists' assessment and management practices for patients with COVID-19 in acute and rehabilitation care in the United Arab Emirates. Fizjoterapia Polska. 2024;24(2):309-317. \url{doi:10.56984/8zg5608sr5}
\item
  Mocarzel R, Kornin A, Tesser C, Ferreira A de S. Quem pode atuar com acupuntura no Brasil? Saúde e Sociedade. 2024;33(2). \url{doi:10.1590/s0104-12902024230197pt}
\item
  Farias JP, Ferreira A de S. Evidence map on burnout syndrome in higher education teachers and its relationship with ergonomic and biopsychosocial factors: a scoping review. International Journal of Occupational Safety and Ergonomics. 2024;30(2):579-586. \url{doi:10.1080/10803548.2024.2325819}
\item
  Al Yammahi RJ, Alaparthi GK, de Sá Ferreira A, Bairapareddy KC, Hegazy FA. Cardiopulmonary Response in Post-COVID-19 Individuals: A Cross-Sectional Study Comparing the Londrina Activities of Daily Living Protocol, 6-Minute Walk Test, and Glittre Activities of Daily Living Test. Healthcare. 2024;12(7):712. \url{doi:10.3390/healthcare12070712}
\item
  Leivas EG, Corrêa LA, Ferreira A de S, De Almeida RS, Nogueira LAC. Falta de conhecimento sobre os fatores de risco não ocupacionais da dor lombar entre profissionais de saúde ocupacional brasileiros: um estudo observacional transversal. Revista Pesquisa em Fisioterapia. 2024;14:e5427. \url{doi:10.17267/2238-2704rpf.2024.e5427}
\item
  Santos LE, de Sá Ferreira A, Vilella RC, Lunkes LC. The Importance of Physical Therapy in the Evaluation of Fall Prevention Programs in Older Adults. Topics in Geriatric Rehabilitation. 2024;40(1):83-92. \url{doi:10.1097/tgr.0000000000000426}
\item
  Avila L, da Silva MD, Neves ML, et al.~Effectiveness of Cognitive Functional Therapy Versus Core Exercises and Manual Therapy in Patients With Chronic Low Back Pain After Spinal Surgery: Randomized Controlled Trial. Physical Therapy. 2023;104(1). \url{doi:10.1093/ptj/pzad105}
\item
  Ferreira APA, Maddaluno MLM, Curi ACC, Ferreira A de S. Interrater agreement and reliability of a palpation method for locating C1 transverse process in the cervical spine. International Journal of Osteopathic Medicine. 2024;51:100699. \url{doi:10.1016/j.ijosm.2023.100699}
\item
  Vieira JE de A, Ferreira A de S, Monnerat LB, et al.~Prediction models for physical function in COVID-19 survivors. Journal of Bodywork and Movement Therapies. 2024;37:70-75. \url{doi:10.1016/j.jbmt.2023.11.002}
\item
  Reis LF da F, Oliveira JPA de, Ferreira A de S, Lopes AJ. Reply to: Factors associated with mortality in mechanically ventilated patients with severe acute respiratory syndrome due to COVID-19 evolution. Critical Care Science. 2024;36. \url{doi:10.62675/2965-2774.20240213-en}
\item
  Reis LF da F, Oliveira JPA de, Ferreira A de S, Lopes AJ. Resposta para: Fatores associados à mortalidade em pacientes ventilados mecanicamente com síndrome respiratória aguda grave por evolução da COVID-19. Critical Care Science. 2024;36. \url{doi:10.62675/2965-2774.20240213-pt}
\item
  Ribeiro Moço VJ, Gulart AA, Lopes AJ, de Sá Ferreira A, da Fonseca Reis LF. Minimal-Resource Home Exercise Program Improves Activities of Daily Living, Perceived Health Status, and Shortness of Breath in Individuals with COPD Stages GOLD II to IV. COPD: Journal of Chronic Obstructive Pulmonary Disease. 2023;20(1):298-306. \url{doi:10.1080/15412555.2023.2253907}
\item
  Silva AL dos S, Collange LA, Ferreira A de S. Hybrid maneuver for benign paroxysmal positional vertigo in individuals with limited neck mobility: Case series. Journal of Bodywork and Movement Therapies. 2024;37:386-391. \url{doi:10.1016/j.jbmt.2023.11.056}
\item
  Rafagnin CZ, Ferreira A de S, Telles GF, Lemos de Carvalho T, Alexandre DJ de A, Nogueira LAC. Anterior component of Y‐Balance test is correlated to ankle dorsiflexion range of motion in futsal players: A cross‐sectional study. Physiotherapy Research International. 2023;28(4). \url{doi:10.1002/pri.2028}
\item
  Casagrande CMZ, Farias JP, Meziat-Filho N, Nogueira LAC, Ferreira AS. Better Work Ability Is Associated With Lower Levels of Both Occupational Stress and Occupational Physical Activity in Professional Drivers. Journal of Occupational \& Environmental Medicine. 2023;65(10):846-852. \url{doi:10.1097/jom.0000000000002918}
\item
  Ferreira A de S, Miranda MG de, Vianna MA, Nascimento WR do. Tecnologias da Informação e Comunicação: visão de moradores e gestores para implementação da política pública de inclusão digital. EaD \& Tecnologias Digitais na Educação. 2023;11(13):128-140. \url{doi:10.30612/eadtde.v11i13.17329}
\item
  Bittencourt JV, Leivas EG, de Sá Ferreira A, Nogueira LAC. Does the painDETECT questionnaire identify impaired conditioned pain modulation in people with musculoskeletal pain? -- a diagnostic accuracy study. Archives of Physiotherapy. 2023;13(1). \url{doi:10.1186/s40945-023-00171-8}
\item
  Custódio LA, Marques YA, de Toledo AM, et al.~The care pathway of individuals with spinal disorders in a Health Care Network in the Federal District, Brazil: a retrospective study. Brazilian Journal of Physical Therapy. 2023;27(5):100553. \url{doi:10.1016/j.bjpt.2023.100553}
\item
  Deucher RA de O, Reis LF da F, Papathanasiou JV, et al.~Estimating cardiopulmonary fitness with a new sampling technology in patients with rheumatoid arthritis‐associated interstitial lung disease. Physiotherapy Research International. 2023;28(3). \url{doi:10.1002/pri.2005}
\item
  Balata MR, Ferreira AS, da Silva Sousa A, et al.~Assessment of Functional Capacity in Patients with Nondialysis-Dependent Chronic Kidney Disease with the Glittre Activities of Daily Living Test. Healthcare. 2023;11(12):1809. \url{doi:10.3390/healthcare11121809}
\item
  Omar A, Ferreira A de S, Hegazy FA, Alaparthi GK. Cardiorespiratory Response to Six-Minute Step Test in Post COVID-19 Patients---A Cross Sectional Study. Healthcare. 2023;11(10):1386. \url{doi:10.3390/healthcare11101386}
\item
  dos Santos Bento AP, Filho NM, Ferreira A de S, Cassetta AP, de Almeida RS. Sleep quality and polysomnographic changes in patients with chronic pain with and without central sensitization signs. Brazilian Journal of Physical Therapy. 2023;27(3):100504. \url{doi:10.1016/j.bjpt.2023.100504}
\item
  Costa M, Saldanha PEC, Ferreira AS, Felicio LR, Lemos T. Posturography measures in specific ballet stance position discriminate ballet dancers with different occurrences of musculoskeletal injuries. Journal of Bodywork and Movement Therapies. 2023;34:41-45. \url{doi:10.1016/j.jbmt.2023.04.020}
\item
  Parisotto G, Reis LFF, Junior MS, Papathanasiou J, Lopes AJ, Ferreira AS. Association of Multiple Cardiovascular Risk Factors with Musculoskeletal Function in Acute Coronary Syndrome Ward Inpatients. Healthcare. 2023;11(7):954. \url{doi:10.3390/healthcare11070954}
\item
  Pinto TP, Inácio JC, de Aguiar E, et al.~Prefrontal tDCS modulates autonomic responses in COVID-19 inpatients. Brain Stimulation. 2023;16(2):657-666. \url{doi:10.1016/j.brs.2023.03.001}
\item
  Cunha J de A, Silva MM, Casagrande CMZ, Ferreira A de S. AMBIENTE DE TRABALHO SEGURO E SUSTENTÁVEL: COMO A ERGONOMIA DE CONSCIENTIZAÇÃO E PARTICIPATIVA SE APLICA AOS SERVIDORES PÚBLICOS? Arquivos de Ciências da Saúde da UNIPAR. 2023;27(1). \url{doi:10.25110/arqsaude.v27i1.2023.9145}
\item
  Fonseca GF, Michalski AC, Ferreira AS, et al.~Is postexercise hypotension a method‐dependent phenomenon in chronic stroke? A crossover randomized controlled trial. Clinical Physiology and Functional Imaging. 2023;43(4):242-252. \url{doi:10.1111/cpf.12812}
\item
  Lunkes LC, Dias Neto MA, Barra LF, de Castro LR, Ferreira AS, Meziat-Filho N. Education to keep the abdomen relaxed versus contracted during pilates in patients with chronic low back pain: study protocol for a randomised controlled trial. BMC Musculoskeletal Disorders. 2023;24(1). \url{doi:10.1186/s12891-023-06160-z}
\item
  Silva S de O, Barbosa JB, Lemos T, Oliveira LAS, Ferreira A de S. Agreement and predictive performance of fall risk assessment methods and factors associated with falls in hospitalized older adults: A longitudinal study. Geriatric Nursing. 2023;49:109-114. \url{doi:10.1016/j.gerinurse.2022.11.016}
\item
  Lunkes LC, Neto MAD, Barra LF, de Castro LR, Ferreira AS, Meziat-Filho N. Correction: Education to keep the abdomen relaxed versus contracted during pilates in patients with chronic low back pain: study protocol for a randomised controlled trial. BMC Musculoskeletal Disorders. 2023;24(1). \url{doi:10.1186/s12891-023-06224-0}
\item
  Reis FJJ, Bittencourt JV, Calestini L, de Sá Ferreira A, Meziat-Filho N, Nogueira LC. Exploratory analysis of 5 supervised machine learning models for predicting the efficacy of the endogenous pain inhibitory pathway in patients with musculoskeletal pain. Musculoskeletal Science and Practice. 2023;66:102788. \url{doi:10.1016/j.msksp.2023.102788}
\item
  de Oliveira JPA, Costa ACT, Lopes AJ, Ferreira A de S, Reis LF da F. Factors associated with mortality in mechanically ventilated patients with severe acute respiratory syndrome due to COVID-19 evolution. Critical Care Science. 2023;35(1). \url{doi:10.5935/2965-2774.20230203-en}
\item
  Neto RBD, Reis LFF, Ferreira A de S, Alexandre DJ de A, Almeida RS de. Hospital admission is associated with disability and late musculoskeletal pain in individuals with long COVID. Frontiers in Rehabilitation Sciences. 2023;4. \url{doi:10.3389/fresc.2023.1186499}
\item
  Bittencourt JV, Bezerra MC, Pina MR, Reis FJJ, de Sá Ferreira A, Nogueira LAC. Use of the painDETECT to discriminate musculoskeletal pain phenotypes. Archives of Physiotherapy. 2022;12(1). \url{doi:10.1186/s40945-022-00129-2}
\item
  Michalski AC, Ferreira AS, Midgley AW, et al.~Mixed circuit training acutely reduces arterial stiffness in patients with chronic stroke: a crossover randomized controlled trial. European Journal of Applied Physiology. 2022;123(1):121-134. \url{doi:10.1007/s00421-022-05061-8}
\item
  Oliveira FAF, Martins CP, de Oliveira LAS, Rodrigues EC, Ferreira AS, Lemos T. Poststroke consequences upon optimization properties of postural sway during upright stance: a cross-sectional study. Topics in Stroke Rehabilitation. 2022;30(7):663-671. \url{doi:10.1080/10749357.2022.2130620}
\item
  Ferreira APA, Zanier JFC, Santos EBG, Ferreira AS. Accuracy of Palpation Procedures for Locating the C1 Transverse Process and Masseter Muscle as Confirmed by Computed Tomography Images. Journal of Manipulative and Physiological Therapeutics. 2022;45(5):337-345. \url{doi:10.1016/j.jmpt.2022.07.005}
\item
  Xavier DD, Graf RM, Ferreira AS. Short-Term Changes in Posture and Pain of the Neck and Lower Back of Women Undergoing Lipoabdominoplasty: A Case Series Report. Journal of Chiropractic Medicine. 2023;22(2):138-147. \url{doi:10.1016/j.jcm.2022.07.003}
\item
  Telles GF, Ferreira A de S, Junior PMP, Lemos T, Bittencourt JV, Nogueira LAC. Concurrent validity of the inertial sensors for assessment of balance control during quiet standing in patients with chronic low back pain and asymptomatic individuals. Journal of Medical Engineering \& Technology. 2022;46(5):354-362. \url{doi:10.1080/03091902.2022.2043947}
\item
  Paz T da SR, Rodrigues PTV, Silva BM, de Sá Ferreira A, Nogueira LAC. Mediation Analysis in Manual Therapy Research. Journal of Chiropractic Medicine. 2023;22(1):35-44. \url{doi:10.1016/j.jcm.2022.04.007}
\item
  Silva CA, Lopes AJ, Papathanasiou J, Reis LFF, Ferreira AS. Association of Functional Characteristics and Physiotherapy with COVID-19 Mortality in Intensive Care Unit in Inpatients with Cardiovascular Diseases. Medicina. 2022;58(6):823. \url{doi:10.3390/medicina58060823}
\item
  Leivas EG, Bittencourt JV, Ferreira AS, Nogueira LAC. Is it possible to discriminate workers with a higher prevalence of low back pain considering daily exposure time in a work-related lumbar posture? A diagnostic accuracy study. Ergonomics. 2021;65(6):877-885. \url{doi:10.1080/00140139.2021.2001577}
\item
  Galvão AF, Lemos T, Martins CP, Horsczaruk CHR, Oliveira LAS, Ferreira A de S. Body sway and movement strategies for control of postural stability in people with spinocerebellar ataxia type 3: A cross-sectional study. Clinical Biomechanics. 2022;97:105711. \url{doi:10.1016/j.clinbiomech.2022.105711}
\item
  Certain Curi AC, Antunes Ferreira AP, Calazans Nogueira LA, Meziat Filho NAM, Sá Ferreira A. Osteopathy and physiotherapy compared to physiotherapy alone on fatigue in long COVID: Study protocol for a pragmatic randomized controlled superiority trial. International Journal of Osteopathic Medicine. 2022;44:22-28. \url{doi:10.1016/j.ijosm.2022.04.004}
\item
  Tedla JS, Rodrigues E, Ferreira AS, et al.~Transcranial direct current stimulation combined with trunk-targeted, proprioceptive neuromuscular facilitation in subacute stroke: a randomized controlled trial. PeerJ. 2022;10:e13329. \url{doi:10.7717/peerj.13329}
\item
  Castro J, Correia L, Donato B de S, et al.~Cognitive functional therapy compared with core exercise and manual therapy in patients with chronic low back pain: randomised controlled trial. Pain. 2022;163(12):2430-2437. \url{doi:10.1097/j.pain.0000000000002644}
\item
  Tedla JS, Gular K, Reddy RS, et al.~Effectiveness of Constraint-Induced Movement Therapy (CIMT) on Balance and Functional Mobility in the Stroke Population: A Systematic Review and Meta-Analysis. Healthcare. 2022;10(3):495. \url{doi:10.3390/healthcare10030495}
\item
  CASAGRANDE CMZ, FERREIRA A de S. Challenges and Perspectives for Research on Work Ability in Professional Drivers: A Scoping Review. Journal of UOEH. 2022;44(1):25-34. \url{doi:10.7888/juoeh.44.25}
\item
  Sá R de A, Ferreira A de S, Lemos T, de Oliveira LAS. Correlation Analysis of Lower-Limb Muscle Function With Clinical Status, Balance Tests, and Quality of Life in People With Parkinson Disease. Topics in Geriatric Rehabilitation. 2022;38(1):56-64. \url{doi:10.1097/tgr.0000000000000343}
\item
  Volpato MP, Menezes M, Prado TS, Piccini A, Ferreira AS, Botelho S. Electromyographic analysis of maximal voluntary contraction of female pelvic floor muscles: Intrarater and interrater reliability study. Neurourology and Urodynamics. 2021;41(1):383-390. \url{doi:10.1002/nau.24834}
\item
  Nascimento MM, Silva PRO, Felício LR, et al.~Postural control in football players with vision impairment: Effect of sports adaptation or visual input restriction? Motriz: Revista de Educação Física. 2022;28. \url{doi:10.1590/s1980-657420220010821}
\item
  Corrêa LA, Mathieson S, Meziat-Filho NA de M, Reis FJ, Ferreira A de S, Nogueira LAC. Which psychosocial factors are related to severe pain and functional limitation in patients with low back pain? Brazilian Journal of Physical Therapy. 2022;26(3):100413. \url{doi:10.1016/j.bjpt.2022.100413}
\item
  Vilela AC, Nogueira LAC, Ferreira A de S, Kochem FB, de Almeida RS. Musculoskeletal Pain and Musical Performance in First and Second Violinists of Professional Youth Chamber Orchestras: A Comparative Study. Medical Problems of Performing Artists. 2021;36(4):263-268. \url{doi:10.21091/mppa.2021.4029}
\item
  Willuweit MGA, Lopes AJ, Ferreira AS. Development of a multivariable prediction model of functional exercise capacity in liver transplant recipients. Journal of Liver Transplantation. 2022;6:100067. \url{doi:10.1016/j.liver.2021.100067}
\item
  Reddy RS, Gautam AP, Tedla JS, et al.~The Aftermath of the COVID-19 Crisis in Saudi Arabia: Respiratory Rehabilitation Recommendations by Physical Therapists. Healthcare. 2021;9(11):1560. \url{doi:10.3390/healthcare9111560}
\item
  Pena Junior PM, de Sá Ferreira A, Telles G, Lemos T, Calazans Nogueira LA. Concurrent validation of the centre of pressure displacement analyzed by baropodometry in patients with chronic non-specific low back pain during functional tasks. Journal of Bodywork and Movement Therapies. 2021;28:489-495. \url{doi:10.1016/j.jbmt.2021.06.020}
\item
  Varella NC, Almeida RS, Nogueira LAC, Ferreira AS. Cross-cultural adaptation of the Richards-Campbell Sleep Questionnaire for intensive care unit inpatients in Brazil: internal consistency, test-retest reliability, and measurement error. Sleep Medicine. 2021;85:38-44. \url{doi:10.1016/j.sleep.2021.06.039}
\item
  Ferreira ADS, Meziat-Filho N, Ferreira APA. Double threshold receiver operating characteristic plot for three-modal continuous predictors. Computational Statistics. 2021;36(3):2231-2245. \url{doi:10.1007/s00180-021-01080-9}
\item
  Deucher RA de O, Ferreira A de S, Nascimento LPA da S, Cal MS da, Papathanasiou JV, Lopes AJ. Heart Rate Variability in Adults with Sickle Cell Anemia During a Multitasking Field Test. Asian Journal of Sports Medicine. 2021;12(3). \url{doi:10.5812/asjsm.108537}
\item
  PAPATHANASIOU JV, PETROV I, TSEKOURA D, et al.~Does group-based high-intensity aerobic interval training improve the inflammatory status in patients with chronic heart failure? European Journal of Physical and Rehabilitation Medicine. 2022;58(2). \url{doi:10.23736/s1973-9087.21.06894-5}
\item
  Saraiva NAO, Ferreira AS, Papathanasiou JV, Guimarães FS, Lopes AJ. Kinematic evaluation of patients with chronic obstructive pulmonary disease during the 6-min walk test. Journal of Bodywork and Movement Therapies. 2021;27:134-140. \url{doi:10.1016/j.jbmt.2021.01.005}
\item
  Souza MF, Ferreira AS. Education of traditional medicine for people with visual impairments in Brazil: Challenges and strategies. Integrative Medicine Research. 2021;10(2):100687. \url{doi:10.1016/j.imr.2020.100687}
\item
  Reddy RS, Meziat-Filho N, Ferreira AS, Tedla JS, Kandakurti PK, Kakaraparthi VN. Comparison of neck extensor muscle endurance and cervical proprioception between asymptomatic individuals and patients with chronic neck pain. Journal of Bodywork and Movement Therapies. 2021;26:180-186. \url{doi:10.1016/j.jbmt.2020.12.040}
\item
  Maddaluno MLM, Ferreira APA, Tavares ACLC, Meziat-Filho N, Ferreira AS. Craniocervical Posture Assessed With Photogrammetry and the Accuracy of Palpation Methods for Locating the Seventh Cervical Spinous Process: A Cross-sectional Study. Journal of Manipulative and Physiological Therapeutics. 2021;44(3):196-204. \url{doi:10.1016/j.jmpt.2020.07.012}
\item
  Queiroz dos Santos AN, Lemos T, Duarte Carvalho PH, Ferreira AS, Silva JG. Immediate effects of myofascial release maneuver applied in different lower limb muscle chains on postural sway. Journal of Bodywork and Movement Therapies. 2021;25:151-156. \url{doi:10.1016/j.jbmt.2020.10.024}
\item
  Correia IMT, Ferreira A de S, Fernandez J, Reis FJJ, Nogueira LAC, Meziat-Filho N. Association Between Text Neck and Neck Pain in Adults. Spine. 2020;46(9):571-578. \url{doi:10.1097/brs.0000000000003854}
\item
  Ferreira N de A, Ferreira A de S, Guimarães FS. Cough peak flow to predict extubation outcome: a systematic review and meta-analysis. Revista Brasileira de Terapia Intensiva. 2021;33(3). \url{doi:10.5935/0103-507x.20210060}
\item
  Palugan MJA, Assis ACB, Bessa EJC, Ferreira AS, Lopes AJ. Predictors of functional capacity as measured by the Glittre activities of daily living test in women with rheumatoid arthritis. Brazilian Journal of Medical and Biological Research. 2021;54(5). \url{doi:10.1590/1414-431x202010040}
\item
  Castro P, Ferreira A de S, Lopes AJ, et al.~Validity of the Polar V800 heart rate monitor for assessing cardiac autonomic control in individuals with spinal cord injury. Motriz: Revista de Educação Física. 2021;27. \url{doi:10.1590/s1980-65742021003221}
\item
  Menezes M, Meziat-Filho NAM, Lemos T, Ferreira AS. `Believe the positive' aggregation of fall risk assessment methods reduces the detection of risk of falling in older adults. Archives of Gerontology and Geriatrics. 2020;91:104228. \url{doi:10.1016/j.archger.2020.104228}
\item
  Gomes AS, de Sá Ferreira A, Reis FJJ, de Jesus-Moraleida FR, Nogueira LAC, Meziat-Filho N. Association Between Low Back Pain and Biomedical Beliefs in Academics of Physiotherapy. Spine. 2020;45(19):1354-1359. \url{doi:10.1097/brs.0000000000003487}
\item
  PAPATHANASIOU JV, PETROV I, TOKMAKOVA MP, et al.~Group-based cardiac rehabilitation interventions. A challenge for physical and rehabilitation medicine physicians: a randomized controlled trial. European Journal of Physical and Rehabilitation Medicine. 2020;56(4). \url{doi:10.23736/s1973-9087.20.06013-x}
\item
  Galvão TS, Magalhães Júnior ES, Orsini Neves MA, de Sá Ferreira A. Lower-limb muscle strength, static and dynamic postural stabilities, risk of falling and fear of falling in polio survivors and healthy subjects. Physiotherapy Theory and Practice. 2018;36(8):899-906. \url{doi:10.1080/09593985.2018.1512178}
\item
  Jeronymo BF, Silva PR de O, Mainenti M, et al.~The Relationship Between Postural Stability, Anthropometry Measurements, Body Composition, and Sport Experience in Judokas with Visual Impairment. Asian Journal of Sports Medicine. 2020;11(3). \url{doi:10.5812/asjsm.103030}
\item
  Corrêa LA, Bittencourt JV, Ferreira A de S, Reis FJJ dos, de Almeida RS, Nogueira LAC. The Reliability and Concurrent Validity of PainMAP Software for Automated Quantification of Pain Drawings on Body Charts of Patients With Low Back Pain. Pain Practice. 2020;20(5):462-470. \url{doi:10.1111/papr.12872}
\item
  Ferreira AS, Maior AS. Two decades of research in soccer and acupuncture: to what point should we stick? Longhua Chinese Medicine. 2020;3:2-2. \url{doi:10.21037/lcm.2020.02.01}
\item
  de Andrade Junior AB, Ferreira A de S, Assis ACB, et al.~Cardiac Autonomic Control in Women with Rheumatoid Arthritis During the Glittre Activities of Daily Living Test. Asian Journal of Sports Medicine. 2020;11(2). \url{doi:10.5812/asjsm.101400}
\item
  Sant'Anna do Carmo Aprigio P, Ramathur Telles de Jesus I, Porto C, Lemos T, de Sá Ferreira A. Lower limb muscle fatigability is not associated with changes in movement strategies for balance control in the upright stance. Human Movement Science. 2020;70:102588. \url{doi:10.1016/j.humov.2020.102588}
\item
  Porto C, Lemos T, Sá Ferreira A. Reliability and robustness of optimization properties for stabilization of the upright stance as determined using posturography. Journal of Biomechanics. 2020;103:109686. \url{doi:10.1016/j.jbiomech.2020.109686}
\item
  Menezes M, de Mello Meziat-Filho NA, Araújo CS, Lemos T, Ferreira AS. Agreement and predictive power of six fall risk assessment methods in community-dwelling older adults. Archives of Gerontology and Geriatrics. 2020;87:103975. \url{doi:10.1016/j.archger.2019.103975}
\item
  Papathanasiou J, Dimitrova D, Dzhafer N, et al.~Are group-based high-intensity aerobic interval training modalities the future of the cardiac rehabilitation? Hellenic Journal of Cardiology. 2020;61(2):141-144. \url{doi:10.1016/j.hjc.2019.10.015}
\item
  Sam-Kit Tin T, Daniel Weng CH, Vigário P dos S, Ferreira A de S. Effects of A Short-term Cardio Tai Chi Program on Cardiorespiratory Fitness and Hemodynamic Parameters in Sedentary Adults: A Pilot Study. Journal of Acupuncture and Meridian Studies. 2020;13(1):12-18. \url{doi:10.1016/j.jams.2019.12.002}
\item
  Maior AS, Tannure M, Eiras F, de Sá Ferreira A. Effects of intermittent negative pressure and active recovery therapies in the post-match period in elite soccer players: A randomized, parallel arm, comparative study. Biomedical Human Kinetics. 2020;12(1):59-68. \url{doi:10.2478/bhk-2020-0008}
\item
  Almeida VP, Ferreira AS, Guimarães FS, Papathanasiou J, Lopes AJ. Predictive models for the six-minute walk test considering the walking course and physical activity level. European Journal of Physical and Rehabilitation Medicine. 2020;55(6). \url{doi:10.23736/s1973-9087.19.05687-9}
\item
  Fernandez J, Ferreira A de S, Castro J, Correia LCL, Meziat‐Filho N. Comment on the paper ``Cognitive functional therapy in patients with non specific chronic low back pain'', by Vibe Fersum et al.~European Journal of Pain. 2019;23(8):1574-1575. \url{doi:10.1002/ejp.1441}
\item
  Gomes BSQ, Coelho VK, Terra BS, et al.~Patients with Subacromial Pain Syndrome Present no Reduction of Shoulder Proprioception: A Matched Case‐Control Study. PM\&R. 2019;11(9):972-978. \url{doi:10.1002/pmrj.12055}
\item
  Porto C, Lemos T, Ferreira AS. Analysis of the postural stabilization in the upright stance using optimization properties. Biomedical Signal Processing and Control. 2019;52:171-178. \url{doi:10.1016/j.bspc.2019.04.009}
\item
  Almeida VP, Ferreira AS, Guimarães FS, Papathanasiou J, Lopes AJ. The impact of physical activity level, degree of dyspnoea and pulmonary function on the performance of healthy young adults during exercise. Journal of Bodywork and Movement Therapies. 2019;23(3):494-501. \url{doi:10.1016/j.jbmt.2018.05.005}
\item
  Vieira ÉCN, Meziat-Filho NAM, Ferreira AS. Photogrammetric Variables Used by Physical Therapists to Detect Neck Pain and to Refer for Physiotherapeutic Intervention: A Cross-Sectional Study. Journal of Manipulative and Physiological Therapeutics. 2019;42(4):254-266. \url{doi:10.1016/j.jmpt.2018.11.014}
\item
  Michalski A da C, Ferreira A de S, Kasuki L, Gadelha MR, Lopes AJ, Guimarães FS. Clinical and functional variables can predict general fatigue in patients with acromegaly: an explanatory model approach. Archives of Endocrinology and Metabolism. April 2019. \url{doi:10.20945/2359-3997000000127}
\item
  Ramos R de A, Guimarães FS, Dionyssiotis Y, Tsekoura D, Papathanasiou J, Ferreira A de S. Development of a multivariate model of the six-minute walked distance to predict functional exercise capacity in hypertension. Journal of Bodywork and Movement Therapies. 2019;23(1):32-38. \url{doi:10.1016/j.jbmt.2018.01.010}
\item
  Ferreira AS, Cunha FA. The circadian blood pressure variability: There is a signal in the noise. The Journal of Clinical Hypertension. 2018;21(1):46-47. \url{doi:10.1111/jch.13430}
\item
  Lima TRL, Almeida VP, Ferreira AS, Guimarães FS, Lopes AJ. Handgrip Strength and Pulmonary Disease in the Elderly: What is the Link? Aging and disease. 2019;10(5):1109. \url{doi:10.14336/ad.2018.1226}
\item
  LOPES AJ, VIGÁRIO PS, HORA AL, et al.~Ventilation Distribution, Pulmonary Diffusion and Peripheral Muscle Endurance as Determinants of Exercise Intolerance in Elderly Patients With Chronic Obstructive Pulmonary Disease. Physiological Research. December 2018:863-874. \url{doi:10.33549/physiolres.933867}
\item
  Alvim DT, Ferreira AS. Pragmatic Combinations of Acupuncture Points for Lateral Epicondylalgia are Unreliable in the Physiotherapy Setting. Journal of Acupuncture and Meridian Studies. 2018;11(6):367-374. \url{doi:10.1016/j.jams.2018.07.006}
\item
  da Silva DCL, Lemos T, de Sá Ferreira A, et al.~Effects of Acute Transcranial Direct Current Stimulation on Gait Kinematics of Individuals With Parkinson Disease. Topics in Geriatric Rehabilitation. 2018;34(4):262-268. \url{doi:10.1097/tgr.0000000000000203}
\item
  Damasceno GM, Ferreira AS, Nogueira LAC, Reis FJJ, Lara RW, Meziat-Filho N. Reliability of two pragmatic tools for assessing text neck. Journal of Bodywork and Movement Therapies. 2018;22(4):963-967. \url{doi:10.1016/j.jbmt.2018.01.007}
\item
  Damasceno GM, Ferreira AS, Nogueira LAC, Reis FJJ, Andrade ICS, Meziat-Filho N. Text neck and neck pain in 18--21-year-old young adults. European Spine Journal. 2018;27(6):1249-1254. \url{doi:10.1007/s00586-017-5444-5}
\item
  Silva PO, Ferreira AS, Lima CM de A, Guimarães FS, Lopes AJ. Balance control is impaired in adults with sickle cell anaemia. Somatosensory \& Motor Research. 2018;35(2):109-118. \url{doi:10.1080/08990220.2018.1481829}
\item
  Saraiva NAO, Guimarães FS, Lopes AJ, Papathanasiou J, Ferreira AS. Feasibility of whole-body gait kinematics to assess the validity of the six-minute walk test over a 10-m walkway in the elderly. Biomedical Signal Processing and Control. 2018;42:202-209. \url{doi:10.1016/j.bspc.2018.02.002}
\item
  Meziat-Filho N, Ferreira AS, Nogueira LAC, Reis FJJ. ``Text-neck'': an epidemic of the modern era of cell phones? The Spine Journal. 2018;18(4):714-715. \url{doi:10.1016/j.spinee.2017.11.022}
\item
  Lima M, Ferreira AS, Reis FJJ, Paes V, Meziat-Filho N. Chronic low back pain and back muscle activity during functional tasks. Gait \& Posture. 2018;61:250-256. \url{doi:10.1016/j.gaitpost.2018.01.021}
\item
  Fonseca GF, Farinatti PTV, Midgley AW, et al.~Continuous and Accumulated Bouts of Cycling Matched by Intensity and Energy Expenditure Elicit Similar Acute Blood Pressure Reductions in Prehypertensive Men. Journal of Strength and Conditioning Research. 2018;32(3):857-866. \url{doi:10.1519/jsc.0000000000002317}
\item
  Alvim DT, Ferreira AS. Inter-expert agreement and similarity analysis of traditional diagnoses and acupuncture prescriptions in textbook- and pragmatic-based practices. Complementary Therapies in Clinical Practice. 2018;30:38-43. \url{doi:10.1016/j.ctcp.2017.12.002}
\item
  de Mello MC, de Sá Ferreira A, Felicio LR. Postural Control during Different Unipodal Positions in Professional Ballet Dancers. Journal of Dance Medicine \& Science. 2017;21(4):151-155. \url{doi:10.12678/1089-313x.21.4.151}
\item
  Santos PBR, Vigário PS, Mainenti MRM, Ferreira AS, Lemos T. Seated limits‐of‐stability of athletes with disabilities with regard to competitive levels and sport classification. Scandinavian Journal of Medicine \& Science in Sports. 2017;27(12):2019-2026. \url{doi:10.1111/sms.12847}
\item
  Lopes AJ, Justo AC, Ferreira AS, Guimaraes FS. Systemic sclerosis: Association between physical function, handgrip strength and pulmonary function. Journal of Bodywork and Movement Therapies. 2017;21(4):972-977. \url{doi:10.1016/j.jbmt.2017.03.018}
\item
  Ferreira APA, Póvoa LC, Zanier JFC, Machado DC, Ferreira AS. Sensitivity for palpating lumbopelvic soft- tissues and bony landmarks and its associated factors: A single-blinded diagnostic accuracy study. Journal of Back and Musculoskeletal Rehabilitation. 2017;30(4):735-744. \url{doi:10.3233/bmr-150356}
\item
  Justo AC, Guimarães FS, Ferreira AS, Soares MS, Bunn PS, Lopes AJ. Muscle function in women with systemic sclerosis: Association with fatigue and general physical function. Clinical Biomechanics. 2017;47:33-39. \url{doi:10.1016/j.clinbiomech.2017.05.011}
\item
  Facchinetti LD, Araújo AQ, Silva MT, et al.~Home-based exercise program in TSP/HAM individuals: a feasibility and effectiveness study. Arquivos de Neuro-Psiquiatria. 2017;75(4):221-227. \url{doi:10.1590/0004-282x20170022}
\item
  Ferreira APA, Póvoa LC, Zanier JFC, Ferreira AS. Locating the Seventh Cervical Spinous Process: Accuracy of the Thorax-Rib Static Method and the Effects of Clinical Data on Its Performance. Journal of Manipulative and Physiological Therapeutics. 2017;40(2):98-105. \url{doi:10.1016/j.jmpt.2016.10.011}
\item
  Ferreira APA, Póvoa LC, Zanier JFC, Ferreira AS. Locating the Seventh Cervical Spinous Process: Development and Validation of a Multivariate Model Using Palpation and Personal Information. Journal of Manipulative and Physiological Therapeutics. 2017;40(2):89-97. \url{doi:10.1016/j.jmpt.2016.10.012}
\item
  Marques NLXR, de Sá Ferreira A, da Silva DPG, de Menezes SLS, Guimarães FS, Dias CM. Performance of National and Foreign Models for Predicting the 6-Minute Walk Distance for Assessment of Functional Exercise Capacity of Brazilian Elderly Women. Topics in Geriatric Rehabilitation. 2017;33(1):68-75. \url{doi:10.1097/tgr.0000000000000134}
\item
  Santos Neves R, Lopes AJ. Hand grip strength in healthy young and older Brazilian adults: Kinesiology. 2017;49(2):208-216. \url{doi:10.26582/k.49.2.5}
\item
  Lopes AJ, Ferreira AS, Walchan EM, Soares MS, Bunn PS, Guimarães FS. Explanatory models of muscle performance in acromegaly patients evaluated by knee isokinetic dynamometry: Implications for rehabilitation. Human Movement Science. 2016;49:160-169. \url{doi:10.1016/j.humov.2016.07.005}
\item
  Marinho C de L, Maioli MCP, Soares AR, et al.~Predictive models of six-minute walking distance in adults with sickle cell anemia: Implications for rehabilitation. Journal of Bodywork and Movement Therapies. 2016;20(4):824-831. \url{doi:10.1016/j.jbmt.2016.02.005}
\item
  Costa MSS, Ferreira AS, Orsini M, Silva EB, Felicio LR. Characteristics and prevalence of musculoskeletal injury in professional and non-professional ballet dancers. Brazilian Journal of Physical Therapy. 2016;20(2):166-175. \url{doi:10.1590/bjpt-rbf.2014.0142}
\item
  Gomes Ribeiro Moura N, Sá Ferreira A. Pulse Waveform Analysis of Chinese Pulse Images and Its Association with Disability in Hypertension. Journal of Acupuncture and Meridian Studies. 2016;9(2):93-98. \url{doi:10.1016/j.jams.2015.06.012}
\item
  Ribeiro de Moura NG, Cordovil I, de Sá Ferreira A. Traditional Chinese medicine wrist pulse-taking is associated with pulse waveform analysis and hemodynamics in hypertension. Journal of Integrative Medicine. 2016;14(2):100-113. \url{doi:10.1016/s2095-4964(16)60233-9}
\item
  Papathanasiou J, Troev T, Ferreira AS, et al.~Advanced Role and Field of Competence of the Physical and Rehabilitation Medicine Specialist in Contemporary Cardiac Rehabilitation. Hellenic Journal of Cardiology. 2016;57(1):16-22. \url{doi:10.1016/s1109-9666(16)30013-6}
\item
  Lopes AJ, Ferreira A de S, Lima TRL, Menezes SLS, Guimarães FS. An explanatory model of functional exercise capacity in patients with systemic sclerosis: considerations for rehabilitation programs. Journal of Physical Therapy Science. 2016;28(2):569-575. \url{doi:10.1589/jpts.28.569}
\item
  Ferreira NA, Lopes AJ, Ferreira AS, Ntoumenopoulos G, Dias J, Guimaraes FS. Determination of functional prognosis in hospitalized patients following an intensive care admission. World Journal of Critical Care Medicine. 2016;5(4):219. \url{doi:10.5492/wjccm.v5.i4.219}
\item
  Lopes AJ, Mafort TT, De Sá Ferreira A, Santos de Castro MC, De Cássia Firmida M, De Andrade Marques E. Is the type of chronic pulmonary infection a determinant of lung function outcomes in adult patients with cystic fibrosis? Monaldi Archives for Chest Disease. 2015;77(3-4). \url{doi:10.4081/monaldi.2012.145}
\item
  Monteiro-Junior R, Ferreira A, Puell V, et al.~Wii Balance Board: Reliability and Clinical Use in Assessment of Balance in Healthy Elderly Women. CNS \& Neurological Disorders - Drug Targets. 2015;14(9):1165-1170. \url{doi:10.2174/1871527315666151111120403}
\item
  Sá MRC, Ribeiro CT, Fracho FG, et al.~Age of independent sitting posture acquisition for children with myelomeningocele. Journal of the Neurological Sciences. 2015;357:e198. \url{doi:10.1016/j.jns.2015.08.685}
\item
  Ferreira AS. Immunity, Inflammation, and Prehypertension: In What Order? The Journal of Clinical Hypertension. 2015;17(10):775-776. \url{doi:10.1111/jch.12611}
\item
  Orsini M, Reis CHM, Ferreira AS, et al.~Postural balance in Machado-Joseph disease. Journal of the Neurological Sciences. 2015;357:e266. \url{doi:10.1016/j.jns.2015.08.935}
\item
  de Sá Ferreira A. Plasma Homocysteine and Arterial Stiffness: Risk Factors or Risk Markers for Cardiovascular Diseases? The Journal of Clinical Hypertension. 2015;17(8):601-602. \url{doi:10.1111/jch.12549}
\item
  Orsini M, De Souza JA, Leite MAA, et al.~Previous acute polio and post-polio syndrome: recognizing the pathophysiology for the establishment of rehabilitation programs. Neurology International. 2015;7(1). \url{doi:10.4081/ni.2015.5452}
\item
  Gonçalves BL, Guimarães FS, Souza MLL de, Ferreira A de S, Mainenti MRM. Association among body composition, muscle performance and functional autonomy in older adults. Fisioterapia em Movimento. 2015;28(1):49-59. \url{doi:10.1590/0103-5150.028.001.ao05}
\item
  de Sá Ferreira A, Pacheco AG. SimTCM: A human patient simulator with application to diagnostic accuracy studies of Chinese medicine. Journal of Integrative Medicine. 2015;13(1):9-19. \url{doi:10.1016/s2095-4964(15)60151-0}
\item
  Oliveira IJ de AS, de Sá Ferreira A. Effects of Diagnostic Errors in Pattern Differentiation and Acupuncture Prescription: A Single-Blinded, Interrater Agreement Study. Evidence-Based Complementary and Alternative Medicine. 2015;2015:1-11. \url{doi:10.1155/2015/469675}
\item
  Lopes AJ, Guedes da Silva DP, Ferreira A de S, Kasuki L, Gadelha MR, Guimarães FS. What is the effect of peripheral muscle fatigue, pulmonary function, and body composition on functional exercise capacity in acromegalic patients? Journal of Physical Therapy Science. 2015;27(3):719-724. \url{doi:10.1589/jpts.27.719}
\item
  Concordar ou discordar: (eis) a questão da diversidade. Fisioterapia em Movimento. 2014;27(4):491-492. \url{doi:10.1590/0103-5150.027.004.ed01}
\item
  Portela FM, Ferreira AS. Kinematic Mapping Reveals Different Spatial Distributions of Center of Pressure High-Speed Regions Under Somatosensory Loss. Journal of Motor Behavior. 2014;46(5):369-379. \url{doi:10.1080/00222895.2014.916651}
\item
  de Sá Ferreira A, Junqueira Ferraz Baracat P. Test--retest reliability for assessment of postural stability using center of pressure spatial patterns of three-dimensional statokinesigrams in young health participants. Journal of Biomechanics. 2014;47(12):2919-2924. \url{doi:10.1016/j.jbiomech.2014.07.010}
\item
  Ramos RA, Guimarães FS, Cordovil I, de Sa Ferreira A. The six-minute walk distance is a marker of hemodynamic-related functional capacity in hypertension: a case--control study. Hypertension Research. 2014;37(8):746-752. \url{doi:10.1038/hr.2014.59}
\item
  Ng SS, Fong SS, Lam SS, Lai CW, Chow LP, de Sá Ferreira A. Acupressure and task-related training after stroke: A case study. International Journal of Therapy and Rehabilitation. 2014;21(4):183-189. \url{doi:10.12968/ijtr.2014.21.4.183}
\item
  Mainenti MRM, Rodrigues EDC, Ferreira ADS, Sousa RCM de, Silva DTR da. Alinhamento articular de membros inferiores e controle postural em idosas. Revista Brasileira de Cineantropometria e Desempenho Humano. 2014;16(3):287. \url{doi:10.5007/1980-0037.2014v16n3p287}
\item
  Pereira RB, Felício LR, Ferreira A de S, Menezes SL de, Freitas MRG de, Orsini M. Immediate effects of using ankle-foot orthoses in the kinematics of gait and in the balance reactions in Charcot-Marie-Tooth disease. Fisioterapia e Pesquisa. 2014;21(1):87-93. \url{doi:10.1590/1809-2950/515210114}
\item
  Lima TRL, Guimarães FS, Sá Ferreira A, Penafortes JTS, Almeida VP, Lopes AJ. Correlation between posture, balance control, and peripheral muscle function in adults with cystic fibrosis. Physiotherapy Theory and Practice. 2013;30(2):79-84. \url{doi:10.3109/09593985.2013.820246}
\item
  Portela FM, Rodrigues EC, Ferreira A de S. A critical review of position- and velocity-based concepts of postural control during upright stance. Human Movement. 2018;15(4):227-233. \url{doi:10.1515/humo-2015-0016}
\item
  Ferreira A de S, Moura NGR de. Asserted and neglected issues linking evidence-based and Chinese medicines for cardiac rehabilitation. World Journal of Cardiology. 2014;6(5):295. \url{doi:10.4330/wjc.v6.i5.295}
\item
  Xiong X, Borrelli F, de Sá Ferreira A, Ashfaq T, Feng B. Herbal Medicines for Cardiovascular Diseases. Evidence-Based Complementary and Alternative Medicine. 2014;2014(1). \url{doi:10.1155/2014/809741}
\item
  Baracat PJF, de Sá Ferreira A. Postural tasks are associated with center of pressure spatial patterns of three-dimensional statokinesigrams in young and elderly healthy subjects. Human Movement Science. 2013;32(6):1325-1338. \url{doi:10.1016/j.humov.2013.06.005}
\item
  de Almeida VP, Guimarães FS, Moço VJR, de Sá Ferreira A, de Menezes SLS, Lopes AJ. Is there an association between postural balance and pulmonary function in adults with asthma? Clinics. 2013;68(11):1421-1427. \url{doi:10.6061/clinics/2013(11)07}
\item
  de Sá Ferreira A. Evidence-based practice of Chinese medicine in physical rehabilitation science. Chinese Journal of Integrative Medicine. 2013;19(10):723-729. \url{doi:10.1007/s11655-013-1451-5}
\item
  Ferreira A de S, Guimarães FS, Magalhães MAR, Silva RCS e. Accuracy and learning curves of inexperienced observers for manual segmentation of electromyograms. Fisioterapia em Movimento. 2013;26(3):559-567. \url{doi:10.1590/s0103-51502013000300009}
\item
  Costa MS da S, Ferreira A de S, Felicio LR. Equilíbrio estático e dinâmico em bailarinos: revisão da literatura. Fisioterapia e Pesquisa. 2013;20(3):299-305. \url{doi:10.1590/s1809-29502013000300016}
\item
  Mainenti MRM, Sousa RCM de, Dias CM, et al.~Body Composition and Chest Expansion of Type II and III Spinal Muscular Atrophy Patients. Journal of Human Growth and Development. 2013;23(2):164. \url{doi:10.7322/jhgd.61291}
\item
  Ferreira A de S. Integrative medicine for hypertension: the earlier the better for treating who and what are not yet ill.~Hypertension Research. 2013;36(7):583-585. \url{doi:10.1038/hr.2013.15}
\item
  Ferreira A de S. Resonance phenomenon during wrist pulse-taking: A stochastic simulation, model-based study of the `pressing with one finger' technique. Biomedical Signal Processing and Control. 2013;8(3):229-236. \url{doi:10.1016/j.bspc.2012.10.004}
\item
  de Sá Ferreira A, Lopes AJ. Pulse waveform analysis as a bridge between pulse examination in Chinese medicine and cardiology. Chinese Journal of Integrative Medicine. 2013;19(4):307-314. \url{doi:10.1007/s11655-013-1412-z}
\item
  de Sá Ferreira A. Promoting integrative medicine by computerization of traditional Chinese medicine for scientific research and clinical practice: The SuiteTCM Project. Journal of Integrative Medicine. 2013;11(2):135-139. \url{doi:10.3736/jintegrmed2013013}
\item
  Ferreira AS, Luiz AB. Role of dermatomes in the determination of therapeutic characteristics of channel acupoints: a similarity-based analysis of data compiled from literature. Chinese Medicine. 2013;8(1):24. \url{doi:10.1186/1749-8546-8-24}
\item
  Oliveira PC, Silva MCS, Silva MCS, Ferreira ADS. Tratamento da fibromialgia por acupuntura baseado na diferenciação de padrões: Revisão sistemática. Cadernos de Naturologia e Terapias Complementares. 2013;2(3):39. \url{doi:10.19177/cntc.v2e3201339-47}
\item
  Fragoso AP de S, Ferreira A de S. Statistical distribution of acupoint prescriptions for sensory-motor impairments in post-stroke subjects. Chinese Journal of Integrative Medicine. December 2012. \url{doi:10.1007/s11655-012-1245-1}
\item
  Pereira RB, Orsini M, Ferreira A de S, et al.~Efeitos do uso de órteses na Doença de Charcot-Marie-Tooth: atualização da literatura. Fisioterapia e Pesquisa. 2012;19(4):388-393. \url{doi:10.1590/s1809-29502012000400016}
\item
  Ferreira A de S, Filho JB, Cordovil I, Souza MN de. Noninvasive pressure pulse waveform analysis of flow-mediated vasodilation evoked by post-occlusive reactive hyperemia maneuver. Biomedical Signal Processing and Control. 2012;7(6):616-621. \url{doi:10.1016/j.bspc.2012.03.001}
\item
  Lopes AJ, Costa W, Thomaz Mafort T, de Sá Ferreira A, Silveira de Menezes SL, Silva Guimarães F. Silicose em jateadores de areia de estaleiro versus silicose em escultores de pedra no Brasil: uma comparação dos achados de imagem, função pulmonar e teste de exercício cardiopulmonar. Revista Portuguesa de Pneumologia. 2012;18(6):260-266. \url{doi:10.1016/j.rppneu.2012.04.006}
\item
  Lopes AJ, Costa W, Thomaz Mafort T, de Sá Ferreira A, Silveira de Menezes SL, Silva Guimarães F. Silicosis in sandblasters of shipyard versus silicosis in stone carvers in Brazil: A comparison of imaging findings, lung function variables and cardiopulmonary exercise testing parameters. Revista Portuguesa de Pneumologia (English Edition). 2012;18(6):260-266. \url{doi:10.1016/j.rppnen.2012.06.002}
\item
  Lima JGM, Oliveira Filho GR, Lima MTBRM, Ferreira AS, Silva JG. Influence of low intensity laser therapy (AsGa) on the cicatrization process of mechanic tendon injury in wistar rats. Laser Physics. 2012;22(9):1445-1448. \url{doi:10.1134/s1054660x12090083}
\item
  Fragoso A. Evaluation of the immediate effects of manual acupuncture on brachial bicep muscle function in healthy individuals and poststroke patients: a study protocol of a parallel-group randomized clinical trial. Journal of Chinese Integrative Medicine. 2012;10(3):303-309. \url{doi:10.3736/jcim20120309}
\item
  Fragoso APS, Ferreira AS. Immediate effects of acupuncture on biceps brachii muscle function in healthy and post-stroke subjects. Chinese Medicine. 2012;7(1):7. \url{doi:10.1186/1749-8546-7-7}
\item
  Ferreira A de S, Oliveira JF de, Cordovil I, Barbosa Filho J. Quadriceps short-term resistance exercise in subjects with resistant hypertension. Fisioterapia em Movimento. 2011;24(4):629-636. \url{doi:10.1590/s0103-51502011000400006}
\item
  Luiz AB, Cordovil I, Filho JB, Ferreira AS. Zangfu zheng (patterns) are associated with clinical manifestations of zang shang (target-organ damage) in arterial hypertension. Chinese Medicine. 2011;6(1). \url{doi:10.1186/1749-8546-6-23}
\item
  Ferreira AS, Lopes AJ. Chinese medicine pattern differentiation and its implications for clinical practice. Chinese Journal of Integrative Medicine. 2011;17(11):818-823. \url{doi:10.1007/s11655-011-0892-y}
\item
  Ferreira A de S, Filho JB, Souza MN de. Model for post-occlusive reactive hyperemia as measured noninvasively with pressure pulse waveform. Biomedical Signal Processing and Control. 2011;6(4):410-413. \url{doi:10.1016/j.bspc.2010.11.003}
\item
  Povoa LC, Vanuzzi FK, Ferreira APA, Ferreira A de S. Intervenção osteopática em idosos e o impacto na qualidade de vida. Fisioterapia em Movimento. 2011;24(3):429-436. \url{doi:10.1590/s0103-51502011000300007}
\item
  Silva AL dos S, Marinho MRC, Gouveia FM de V, Silva JG, Ferreira A de S, Cal R. Benign Paroxysmal Positional Vertigo: comparison of two recent international guidelines. Brazilian Journal of Otorhinolaryngology. 2011;77(2):191-200. \url{doi:10.1590/s1808-86942011000200009}
\item
  Mainenti MRM, de Carvalho Rodrigues É, de Oliveira JF, de Sá Ferreira A, Dias CM, dos Santos Silva AL. Adiposity and postural balance control: Correlations between bioelectrical impedance and stabilometric signals in elderly Brazilian women. Clinics. 2011;66(9):1513-1518. \url{doi:10.1590/s1807-59322011000900001}
\item
  Sá Ferreira A. Misdiagnosis and undiagnosis due to pattern similarity in Chinese medicine: a stochastic simulation study using pattern differentiation algorithm. Chinese Medicine. 2011;6(1):1. \url{doi:10.1186/1749-8546-6-1}
\item
  Ferreira A de S, Gave N de S, Abrahão F, Silva JG. Influência da morfologia de pés e joelhos no equilíbrio durante apoio bipodal. Fisioterapia em Movimento. 2010;23(2):193-200. \url{doi:10.1590/s0103-51502010000200003}
\item
  Silva JG, Antonioli R de S, Orsini M, Santos Júnior MAJ dos, Ferreira A de S. Mobilização do osso pisiforme no tratamento da neuropraxia do nervo ulnar no canal de Guyon: relato de caso. Fisioterapia e Pesquisa. 2009;16(4):363-367. \url{doi:10.1590/s1809-29502009000400014}
\item
  Ferre D. Statistical validation of strategies for Zang-Fu single patterndifferentiation. Journal of Chinese Integrative Medicine. 2008;6(11):1109-1116. \url{doi:10.3736/jcim20081103}
\item
  Peng Q. Effects of Huoxue Tongmai Lishui method on fundus fluorescein angiography of non-ischemic retinal vein occlusion: a randomized controlled trial. Journal of Chinese Integrative Medicine. 2009;7(11):1035-1041. \url{doi:10.3736/jcim20091103}
\item
  Ferreira A. Prophylactic effects of short-term acupuncture on Zusanli (ST36) in Wistar rats with lipopolysaccharide-induced acute lung injury. Journal of Chinese Integrative Medicine. 2009;7(10):969-975. \url{doi:10.3736/jcim20091011}
\item
  de Sá Ferreira A, Filho JB, Cordovil I, de Souza MN. Three-section transmission-line arterial model for noninvasive assessment of vascular remodeling in primary hypertension. Biomedical Signal Processing and Control. 2009;4(1):2-6. \url{doi:10.1016/j.bspc.2008.07.001}
\item
  Ferreira A. Diagnostic accuracy of pattern differentiation algorithm based on Chinese medicine theory: a stochastic simulation study. Chinese Medicine. 2009;4(1):24. \url{doi:10.1186/1749-8546-4-24}
\item
  Ferreira A de S, Souza MN de, Filho JB. Avaliação de um modelo de parâmetros distribuídos aplicado à estimação da geometria arterial na hipertensão. Revista Brasileira de Engenharia Biomédica. 2008;24(3):193-200. \url{doi:10.4322/rbeb.2012.057}
\item
  Ferreira AS, Santos MAR, Filho JB, Cordovil I, Souza MN. Determination of radial artery compliance can increase the diagnostic power of pulse wave velocity measurement. Physiological Measurement. 2003;25(1):37-50. \url{doi:10.1088/0967-3334/25/1/004}
\item
  Oliveira CL, Ferreira AS. Reabilitação de pessoas com amputação de membros inferiores: uma revisão sistematizada. Ciência em Movimento. 2022;23(48):99-109. \url{doi:10.15602/1983-9480/cm.v23n48p99-109}
\end{enumerate}

\section*{Preprints}\label{preprints}
\addcontentsline{toc}{section}{Preprints}

\begin{enumerate}
\def\labelenumi{\arabic{enumi}.}
\item
  Menezes K, B. Lima MA, Xerez DR, et al.~RETURN OF VOLUNTARY MOTOR CONTRACTION AFTER COMPLETE SPINAL CORD INJURY: A PILOT HUMAN STUDY ON POLYLAMININ. February 2024. \url{doi:10.1101/2024.02.19.24301010}
\item
  Alaparthi GK, Gatty AP, Isa H, et al.~Reference Values for the Grocery Shelving Test among United Arab Emirates Population: A Cross Sectional Study. June 2023. \url{doi:10.20944/preprints202305.1684.v3}
\item
  Alaparthi GK, Gatty AP, Isa H, et al.~Reference Values for the Grocery Shelving Test among United Arab Emirates Population: A Cross Sectional Study. May 2023. \url{doi:10.20944/preprints202305.1684.v2}
\item
  Alaparthi GK, Gatty AP, Isa H, et al.~Reference Values for the Grocery Shelving Test among United Arab Emirates Population: A Cross Sectional Study. May 2023. \url{doi:10.20944/preprints202305.1684.v1}
\end{enumerate}

\section*{Resumos publicados em eventos científicos}\label{resumos-publicados-em-eventos-cientuxedficos}
\addcontentsline{toc}{section}{Resumos publicados em eventos científicos}

\begin{enumerate}
\def\labelenumi{\arabic{enumi}.}
\item
  Ferreira AET, Almeida APGS de, Vidal F de B. Autonomous Vehicle Steering Wheel Estimation from a Video using Multichannel Convolutional Neural Networks. Proceedings of the 15th International Conference on Informatics in Control, Automation and Robotics. 2018:517-524. \url{doi:10.5220/0006920605170524}
\item
  de Araujo Ferreira N, Lopes AJ, de Sá Ferreira A, Oliveira Dias J, Silva Guimarães F. Fatores determinantes do prognóstico funcional do paciente crítico sob intervenção fisioterapeutica. Anais do Congresso Internacional de Qualidade em Serviços e Sistemas de Saúde. May 2017. \url{doi:10.17648/qualihosp-2017-69645}
\item
  Ferreira AS, Filho JB, Souza MN. Simplified Distributed-Parametermodel of Brachial-Radial Arteries for Noninvasive Determination of Mechanical Characteristics of Vessel. 2006 International Conference of the IEEE Engineering in Medicine and Biology Society. August 2006:1814-1817. \url{doi:10.1109/iembs.2006.260102}
\item
  Ferreira AS, Filho JB, Souza MN. Comparison of segmental arterial compliance determined with three and four element Windkessel models. Proceedings of the 25th Annual International Conference of the IEEE Engineering in Medicine and Biology Society (IEEE Cat No03CH37439).:3161-3164. \url{doi:10.1109/iembs.2003.1280813}
\item
  Ferreira AS, Filho JB, Souza MN. Identification of vascular parameters based on the same pressure pulses waves used to measure pulse wave velocity. 2001 Conference Proceedings of the 23rd Annual International Conference of the IEEE Engineering in Medicine and Biology Society. 4:3418-3421. \url{doi:10.1109/iembs.2001.1019564}
\end{enumerate}

\section*{Livros editorados}\label{livros-editorados}
\addcontentsline{toc}{section}{Livros editorados}

\begin{enumerate}
\def\labelenumi{\arabic{enumi}.}
\tightlist
\item
  Vigário P dos S, Ferreira A de S. DESENVOLVIMENTO LOCAL E OS DESAFIOS PARA O ALCANCE DO DESENVOLVIMENTO SUSTENTÁVEL SOB O PRISMA DA AGENDA 2030. 2024. \url{doi:10.47879/ed.ep.2024745}
\end{enumerate}

\section*{Capítulos de livro}\label{capuxedtulos-de-livro}
\addcontentsline{toc}{section}{Capítulos de livro}

\begin{enumerate}
\def\labelenumi{\arabic{enumi}.}
\item
  Luna L de O, Tavares EM, Cezário BS, Guedes ALA, Ferreira A de S. A IMPORTÂNCIA DE MEDIDAS PROTETIVAS NA FLORESTA AMAZÔNICA PARA A PRESERVAÇÃO DA BIODIVERSIDADE - ODS 15. DESENVOLVIMENTO LOCAL E OS DESAFIOS PARA O ALCANCE DO DESENVOLVIMENTO SUSTENTÁVEL SOB O PRISMA DA AGENDA 2030. 2024:135-148. \url{doi:10.47879/ed.ep.2024745p135}
\item
  Tavares EM, Luna L de O, Cezário BS, Guedes ALA, Ferreira A de S. CIDADES INTELIGENTES: UMA REVISÃO DA RELAÇÃO ENTRE A TECNOLOGIA E A MOBILIDADE URBANA SUSTENTÁVEL. DESENVOLVIMENTO LOCAL E OS DESAFIOS PARA O ALCANCE DO DESENVOLVIMENTO SUSTENTÁVEL SOB O PRISMA DA AGENDA 2030. 2024:106-114. \url{doi:10.47879/ed.ep.2024745p106}
\item
  Ferreira AS, Oliveira IJAS. Methods for Assessment of Interrater Reliability for Diagnosis and Intervention in Traditional Chinese Medicine Studies. Evidence-based Research Methods for Chinese Medicine. 2016:89-111. \url{doi:10.1007/978-981-10-2290-6_7}
\item
  Sa Ferreira A de. Advances in Chinese Medicine Diagnosis: From Traditional Methods to Computational Models. Recent Advances in Theories and Practice of Chinese Medicine. January 2012. \url{doi:10.5772/27703}
\end{enumerate}

\section*{Aulas}\label{aulas}
\addcontentsline{toc}{section}{Aulas}

\begin{enumerate}
\def\labelenumi{\arabic{enumi}.}
\item
  Laett CT, de Sá Ferreira A. Instrumentação Virtual e Tecnologia em Reabilitação. OSF. 2025. \url{doi:10.17605/OSF.IO/XR784}
\item
  De Sá Ferreira A, Laett CT. Bioestatística I. 2024. \url{doi:10.17605/OSF.IO/YZ8WG}
\item
  Laett CT, de Sá Ferreira A. Bioestatística II. 2024. \url{doi:10.17605/OSF.IO/C7VKP}
\item
  Thiago Lemos, Da Hora AL, Felix BP, et al.~Didática e Prática do Ensino Superior. 2024. \url{doi:10.17605/OSF.IO/GHXUD}
\end{enumerate}

\section*{Programas de computador}\label{programas-de-computador}
\addcontentsline{toc}{section}{Programas de computador}

\begin{enumerate}
\def\labelenumi{\arabic{enumi}.}
\item
  Ferreira Ade S, Meziat Filho N. RCTapp. Zenodo; 2025. \url{doi:10.5281/ZENODO.13848815}
\item
  Ferreira A. UsIA \textbar{} Ultrasound Image Analysis. Zenodo; 2023. \url{doi:10.5281/ZENODO.10439719}
\item
  Arthur de Sá Ferreira. Observatório. Zenodo; 2023. \url{doi:10.5281/ZENODO.8322622}
\item
  De Sá Ferreira A. Ciência com R. setembro 2023. \url{doi:10.5281/ZENODO.8320233}
\item
  Ferreira AS. SuiteEBG. Zenodo; 2023. \url{doi:10.5281/ZENODO.8210025}
\item
  Ferreira Ade S. SuiteMYO. Zenodo; 2023. \url{doi:10.5281/ZENODO.8211266}
\item
  Ferreira Ade S. SuiteTCM. Zenodo; 2023. \url{doi:10.5281/ZENODO.8211409}
\item
  Ferreira Ade S. wiiVIEW. Zenodo; 2019. \url{doi:10.5281/ZENODO.8233112}
\end{enumerate}

\section*{Bancos de dados}\label{bancos-de-dados}
\addcontentsline{toc}{section}{Bancos de dados}

\begin{enumerate}
\def\labelenumi{\arabic{enumi}.}
\item
  Alaparthi G, Moustafa I, Lopes A, Ferreira A. Data and code for ``Pulmonary function, body posture and balance in young adults with asthma: A cross-sectional study''. dezembro 2024. \url{doi:10.17632/PT5SCMP33Z.1}
\item
  Abrahão P, Meziat-Filho N, Rodrigues E, Pinto T, Ferreira A. Dataset and code for ``Physiotherapists' attitudes and beliefs about pain treatment influence their visual assessment of posture in individuals with neck pain''. outubro 2024. \url{doi:10.17632/7CBWM57PCD.1}
\item
  Arthur Ferreira. Data for ``Locating the Seventh Cervical Spinous Process: Accuracy of the Thorax-Rib Static Method and the Effects of Clinical Data on Its Performance''. setembro 2022. \url{doi:10.17632/DXJ78TYYHJ.2}
\item
  Ferreira A. Dataset for ``Accuracy and learning curves of inexperienced observers for manual segmentation of electromyograms''. janeiro 2021. \url{doi:10.17632/5HBFNNJFFF.1}
\item
  Ferreira A. Dataset for ``Feasibility of whole-body gait kinematics to assess the validity of the six-minute walk test over a 10-m walkway in the elderly''. janeiro 2021. \url{doi:10.17632/47S5JDW7VN.1}
\item
  Ferreira A. Dataset for ``Kinematic evaluation of patients with chronic obstructive pulmonary disease during the 6-min walk test''. janeiro 2021. \url{doi:10.17632/5C97XJMCW2.1}
\item
  Ferreira A. Data for ``Agreement and predictive power of six fall risk assessment methods in community-dwelling older adults''. janeiro 2021. \url{doi:10.17632/3D4VR4DWJS.3}
\item
  Ferreira A. Dataset for ``Lower limb muscle fatigability is not associated with changes in movement strategies for balance control in the upright stance''. janeiro 2021. \url{doi:10.17632/XBGX8GBYMS.4}
\item
  Ferreira A. Dataset for ``Optimization of postural stability after repeated motor tasks with visual feedback in the upright stance''. janeiro 2021. \url{doi:10.17632/TB5FMH8ZNJ.1}
\item
  Ferreira A. Data for: Pragmatic combinations of acupuncture points for lateral epicondylalgia are unreliable in the physiotherapy setting. novembro 2018. \url{doi:10.17632/6WYN9KTX7F.1}
\item
  Ferreira A. Inter-expert agreement and similarity analysis of traditional diagnoses and acupuncture prescriptions in textbook- and pragmatic-based practices. dezembro 2017. \url{doi:10.17632/KBGSNJTBVY.1}
\end{enumerate}

\section*{Outros}\label{outros}
\addcontentsline{toc}{section}{Outros}

\begin{enumerate}
\def\labelenumi{\arabic{enumi}.}
\item
  Williane Nascimento, Rosa F, de Sá Ferreira A. Tecnologias digitais na periferia: Uma perspectiva para o desenvolvimento multidimensional de populações em situação de vulnerabilidade socioeconômica. 2025. \url{doi:10.17605/OSF.IO/8CWJ5}
\item
  de Sá Ferreira A. Advanced Computational Methods for Studying Postural Balance in Health and Neuromuscular Diseases. OSF. 2024. \url{doi:10.17605/OSF.IO/UNPZW}
\item
  de Sá Ferreira A. Cardiac Rehabilitation: Current Evidence and Future Perspectives. OSF. 2024. \url{doi:10.17605/OSF.IO/SQ3WU}
\item
  de Sá Ferreira A. Inteligência Artificial na Academia. OSF. 2024. \url{doi:10.17605/OSF.IO/EYSPK}
\item
  de Sá Ferreira A. Inteligência Artificial na Academia. 2024. \url{doi:10.17605/OSF.IO/GFEQ5}
\item
  de Sá Ferreira A. Inteligência Artificial na Academia: Da Teoria às Aplicações. OSF. 2024. \url{doi:10.17605/OSF.IO/H8JBD}
\item
  de Sá Ferreira A. Inteligência Artificial na Saúde. OSF. 2024. \url{doi:10.17605/OSF.IO/HVY9W}
\item
  de Sá Ferreira A, Bittencourt JV. Universidade Corporativa. OSF. 2024. \url{doi:10.17605/OSF.IO/2FCVX}
\item
  de Sá Ferreira A. Uso de Inteligência Artificial em Pesquisas. OSF. 2024. \url{doi:10.17605/OSF.IO/XTYGW}
\end{enumerate}

\chapter*{\texorpdfstring{\textbf{Fontes externas}}{Fontes externas}}\label{fontes-externas}
\addcontentsline{toc}{chapter}{\textbf{Fontes externas}}

\section*{Fontes de informação externas}\label{fontes-de-informauxe7uxe3o-externas}
\addcontentsline{toc}{section}{Fontes de informação externas}

\subsection*{American Heart Association}\label{american-heart-association}
\addcontentsline{toc}{subsection}{American Heart Association}

\begin{itemize}
\tightlist
\item
  \href{https://www.ahajournals.org/statistical-recommendations}{\emph{Statistical Reporting Recommendations - AHA/ASA journals}}
\end{itemize}

\subsection*{American Physiological Society}\label{american-physiological-society}
\addcontentsline{toc}{subsection}{American Physiological Society}

\begin{itemize}
\item
  \href{https://journals.physiology.org/topic/advances-collections/statistics?seriesKey=&tagCode=&}{\emph{Statistics}}
\item
  \href{https://journals.physiology.org/topic/advances-collections/explorations-in-statistics?seriesKey=&tagCode=&}{\emph{Exploration in Statistics}}
\item
  \href{https://journals.physiology.org/topic/advances-collections/general-statistics?seriesKey=&tagCode=&}{\emph{General Statistics}}
\item
  \href{https://journals.physiology.org/topic/advances-collections/reporting-statistics?seriesKey=&tagCode=&}{\emph{Reporting Statistics}}
\end{itemize}

\subsection*{American Statistical Association}\label{american-statistical-association}
\addcontentsline{toc}{subsection}{American Statistical Association}

\begin{itemize}
\tightlist
\item
  \href{https://www.tandfonline.com/toc/utas20/73/sup1?nav=tocList}{\emph{Statistical Inference in the 21st Century: A World Beyond p \textless{} 0.05 - The American Statistical Association}}
\end{itemize}

\subsection*{British Medicine Journal}\label{british-medicine-journal}
\addcontentsline{toc}{subsection}{British Medicine Journal}

\begin{itemize}
\item
  \href{https://www.bmj.com/specialties/statistics}{\emph{Statistics - Latest from The BMJ}}
\item
  \href{https://www.bmj.com/specialties/statistics-notes}{\emph{Statistics notes - Latest from The BMJ}}
\item
  \href{https://www.bmj.com/specialties/statistics-and-research-methods}{\emph{Statistics and research methods - Latest from The BMJ}}
\item
  \href{https://www.bmj.com/about-bmj/resources-readers/publications/statistics-square-one}{\emph{Statistics at Square One}}
\item
  \href{https://www.bmj.com/research/research-methods-and-reporting}{\emph{Research methods \& reporting}}
\end{itemize}

\subsection*{Enhancing the QUality And Transparency Of health Research Network}\label{enhancing-the-quality-and-transparency-of-health-research-network}
\addcontentsline{toc}{subsection}{Enhancing the QUality And Transparency Of health Research Network}

\begin{itemize}
\tightlist
\item
  \emph{Enhancing the Quality and Transparency of health research} \href{https://www.equator-network.org}{EQUATOR Network}
\end{itemize}

\subsection*{Journal of the Amercan Medical Association}\label{journal-of-the-amercan-medical-association}
\addcontentsline{toc}{subsection}{Journal of the Amercan Medical Association}

\begin{itemize}
\tightlist
\item
  \href{https://jamanetwork.com/collections/44042/jama-guide-to-statistics-and-methods}{\emph{JAMA Guide to Statistics and Methods - JAMA}}
\end{itemize}

\subsection*{Nature Publishing Group}\label{nature-publishing-group}
\addcontentsline{toc}{subsection}{Nature Publishing Group}

\begin{itemize}
\tightlist
\item
  \href{https://www.nature.com/collections/qghhqm}{\emph{Statistics for Biologists - Nature Publising Group}}
\end{itemize}

\subsection*{Oxford Reference}\label{oxford-reference}
\addcontentsline{toc}{subsection}{Oxford Reference}

\begin{itemize}
\tightlist
\item
  \href{https://www.oxfordreference.com/display/10.1093/acref/9780199679188.001.0001/acref-9780199679188}{\emph{A Dictionary of Statistics}}
\end{itemize}

\subsection*{Royal Statistical Society}\label{royal-statistical-society}
\addcontentsline{toc}{subsection}{Royal Statistical Society}

\begin{itemize}
\tightlist
\item
  \href{https://royal-statistical-society.github.io/datavisguide}{\emph{Best Practices for Data Visualisation - Royal Statistical Society}}
\end{itemize}

\subsection*{Statistics in Medicine}\label{statistics-in-medicine}
\addcontentsline{toc}{subsection}{Statistics in Medicine}

\begin{itemize}
\tightlist
\item
  \href{https://onlinelibrary.wiley.com/page/journal/10970258/homepage/tutorials.htm}{\emph{Tutorials in Biostatistics Papers}}
\end{itemize}

\subsection*{BMC Trials}\label{bmc-trials}
\addcontentsline{toc}{subsection}{BMC Trials}

\begin{itemize}
\tightlist
\item
  \href{https://www.biomedcentral.com/collections/DANT}{\emph{Design and analysis of n-of-1 trials}}
\end{itemize}

\subsection*{The Journal of Applied Statistics in the Pharmaceutical Industry}\label{the-journal-of-applied-statistics-in-the-pharmaceutical-industry}
\addcontentsline{toc}{subsection}{The Journal of Applied Statistics in the Pharmaceutical Industry}

\begin{itemize}
\tightlist
\item
  \href{https://onlinelibrary.wiley.com/page/journal/15391612/homepage/tutorial_papers.htm}{\emph{Tutorial Papers}}
\end{itemize}

\chapter*{\texorpdfstring{\textbf{Referências}}{Referências}}\label{referencias}
\addcontentsline{toc}{chapter}{\textbf{Referências}}

\DisableFootNotes

\phantomsection\label{refs}
\begin{CSLReferences}{0}{1}
\bibitem[\citeproctext]{ref-grami2023}
\CSLLeftMargin{1. }%
\CSLRightInline{Grami A. Discrete Probability. Em: \emph{Discrete Mathematics: Essentials and Applications}. Elsevier; 2023:285--305. doi:\href{https://doi.org/10.1016/b978-0-12-820656-0.00016-2}{10.1016/b978-0-12-820656-0.00016-2}}

\bibitem[\citeproctext]{ref-Viti2015}
\CSLLeftMargin{2. }%
\CSLRightInline{Viti A, Terzi A, Bertolaccini L. A practical overview on probability distributions. \emph{Journal of Thoracic Disease}. 2015;7(3). \href{https://jtd.amegroups.org/article/view/4086}{https://jtd.amegroups.org/article/view/4086.}}

\bibitem[\citeproctext]{ref-Benford1938}
\CSLLeftMargin{3. }%
\CSLRightInline{Benford F. The Law of Anomalous Numbers. \emph{Proceedings of the American Philosophical Society}. 1938;78(4):551--572. \url{http://www.jstor.org/stable/984802}. Acessado novembro 24, 2024.}

\bibitem[\citeproctext]{ref-tversky1971}
\CSLLeftMargin{4. }%
\CSLRightInline{Tversky A, Kahneman D. Belief in the law of small numbers. \emph{Psychological Bulletin}. 1971;76(2):105--110. doi:\href{https://doi.org/10.1037/h0031322}{10.1037/h0031322}}

\bibitem[\citeproctext]{ref-bishop2022}
\CSLLeftMargin{5. }%
\CSLRightInline{Bishop DVM, Thompson J, Parker AJ. Can we shift belief in the {`}Law of Small Numbers{'}? \emph{Royal Society Open Science}. 2022;9(3). doi:\href{https://doi.org/10.1098/rsos.211028}{10.1098/rsos.211028}}

\bibitem[\citeproctext]{ref-guy1988}
\CSLLeftMargin{6. }%
\CSLRightInline{Guy RK. The Strong Law of Small Numbers. \emph{The American Mathematical Monthly}. 1988;95(8):697. doi:\href{https://doi.org/10.2307/2322249}{10.2307/2322249}}

\bibitem[\citeproctext]{ref-guy1990}
\CSLLeftMargin{7. }%
\CSLRightInline{Guy RK. The Second Strong Law of Small Numbers. \emph{Mathematics Magazine}. 1990;63(1):3--20. doi:\href{https://doi.org/10.1080/0025570x.1990.11977475}{10.1080/0025570x.1990.11977475}}

\bibitem[\citeproctext]{ref-kwak2017}
\CSLLeftMargin{8. }%
\CSLRightInline{Kwak SG, Kim JH. Central limit theorem: the cornerstone of modern statistics. \emph{Korean Journal of Anesthesiology}. 2017;70(2):144. doi:\href{https://doi.org/10.4097/kjae.2017.70.2.144}{10.4097/kjae.2017.70.2.144}}

\bibitem[\citeproctext]{ref-galton1886}
\CSLLeftMargin{9. }%
\CSLRightInline{Galton F. Regression Towards Mediocrity in Hereditary Stature. \emph{The Journal of the Anthropological Institute of Great Britain and Ireland}. 1886;15:246. doi:\href{https://doi.org/10.2307/2841583}{10.2307/2841583}}

\bibitem[\citeproctext]{ref-barnett2004}
\CSLLeftMargin{10. }%
\CSLRightInline{Barnett AG. Regression to the mean: what it is and how to deal with it. \emph{International Journal of Epidemiology}. 2004;34(1):215--220. doi:\href{https://doi.org/10.1093/ije/dyh299}{10.1093/ije/dyh299}}

\bibitem[\citeproctext]{ref-senn2011}
\CSLLeftMargin{11. }%
\CSLRightInline{Senn S. Francis Galton and Regression to the Mean. \emph{Significance}. 2011;8(3):124--126. doi:\href{https://doi.org/10.1111/j.1740-9713.2011.00509.x}{10.1111/j.1740-9713.2011.00509.x}}

\bibitem[\citeproctext]{ref-regtomean}
\CSLLeftMargin{12. }%
\CSLRightInline{Recchia D. \emph{regtomean: Regression Toward the Mean}.; 2022. \href{https://CRAN.R-project.org/package=regtomean}{https://CRAN.R-project.org/package=regtomean.}}

\bibitem[\citeproctext]{ref-Altman1997}
\CSLLeftMargin{13. }%
\CSLRightInline{Altman DG, Bland JM. Statistics Notes: Units of analysis. \emph{BMJ}. 1997;314(7098):1874--1874. doi:\href{https://doi.org/10.1136/bmj.314.7098.1874}{10.1136/bmj.314.7098.1874}}

\bibitem[\citeproctext]{ref-Matthews1990}
\CSLLeftMargin{14. }%
\CSLRightInline{Matthews JN, Altman DG, Campbell MJ, Royston P. Analysis of serial measurements in medical research. \emph{BMJ}. 1990;300(6719):230--235. doi:\href{https://doi.org/10.1136/bmj.300.6719.230}{10.1136/bmj.300.6719.230}}

\bibitem[\citeproctext]{ref-Banerjee2010}
\CSLLeftMargin{15. }%
\CSLRightInline{Banerjee A, Chaudhury S. Statistics without tears: Populations and samples. \emph{Industrial Psychiatry Journal}. 2010;19(1):60. doi:\href{https://doi.org/10.4103/0972-6748.77642}{10.4103/0972-6748.77642}}

\bibitem[\citeproctext]{ref-martinez-mesa2016}
\CSLLeftMargin{16. }%
\CSLRightInline{Martínez-Mesa J, González-Chica DA, Duquia RP, Bonamigo RR, Bastos JL. Sampling: how to select participants in my research study? \emph{Anais Brasileiros de Dermatologia}. 2016;91(3):326--330. doi:\href{https://doi.org/10.1590/abd1806-4841.20165254}{10.1590/abd1806-4841.20165254}}

\bibitem[\citeproctext]{ref-Bland2015}
\CSLLeftMargin{17. }%
\CSLRightInline{Bland JM, Altman DG. Statistics Notes: Bootstrap resampling methods. \emph{BMJ}. 2015;350(jun02 13):h2622--h2622. doi:\href{https://doi.org/10.1136/bmj.h2622}{10.1136/bmj.h2622}}

\bibitem[\citeproctext]{ref-amatuzzi2006}
\CSLLeftMargin{18. }%
\CSLRightInline{Amatuzzi MLL, Barreto M do CC, Litvoc J, Leme LEG. Linguagem metodológica: parte 1. \emph{Acta Ortopédica Brasileira}. 2006;14(1):53--56. doi:\href{https://doi.org/10.1590/s1413-78522006000100012}{10.1590/s1413-78522006000100012}}

\bibitem[\citeproctext]{ref-amatuzzi2006a}
\CSLLeftMargin{19. }%
\CSLRightInline{Amatuzzi MLL, Barreto M do CC, Litvoc J, Leme LEG. Linguagem metodológica: parte 2. \emph{Acta Ortopédica Brasileira}. 2006;14(2):108--112. doi:\href{https://doi.org/10.1590/s1413-78522006000200012}{10.1590/s1413-78522006000200012}}

\bibitem[\citeproctext]{ref-munafuxf22017}
\CSLLeftMargin{20. }%
\CSLRightInline{Munafò MR, Nosek BA, Bishop DVM, et al. A manifesto for reproducible science. \emph{Nature Human Behaviour}. 2017;1(1). doi:\href{https://doi.org/10.1038/s41562-016-0021}{10.1038/s41562-016-0021}}

\bibitem[\citeproctext]{ref-wood2010}
\CSLLeftMargin{21. }%
\CSLRightInline{Wood M, Welch C. Are {`}Qualitative{'} and {`}Quantitative{'} Useful Terms for Describing Research? \emph{Methodological Innovations Online}. 2010;5(1):56--71. doi:\href{https://doi.org/10.4256/mio.2010.0010}{10.4256/mio.2010.0010}}

\bibitem[\citeproctext]{ref-lall2021}
\CSLLeftMargin{22. }%
\CSLRightInline{Lall D. Mixed-Methods Research. \emph{Indian Journal of Continuing Nursing Education}. 2021;22(2):143--147. doi:\href{https://doi.org/10.4103/ijcn.ijcn_107_21}{10.4103/ijcn.ijcn\_107\_21}}

\bibitem[\citeproctext]{ref-schoonenboom2017}
\CSLLeftMargin{23. }%
\CSLRightInline{Schoonenboom J, Johnson RB. How to Construct a Mixed Methods Research Design. \emph{KZfSS Kölner Zeitschrift für Soziologie und Sozialpsychologie}. 2017;69(S2):107--131. doi:\href{https://doi.org/10.1007/s11577-017-0454-1}{10.1007/s11577-017-0454-1}}

\bibitem[\citeproctext]{ref-rubin2022}
\CSLLeftMargin{24. }%
\CSLRightInline{Rubin M, Donkin C. Exploratory hypothesis tests can be more compelling than confirmatory hypothesis tests. \emph{Philosophical Psychology}. 2022;37(8):2019--2047. doi:\href{https://doi.org/10.1080/09515089.2022.2113771}{10.1080/09515089.2022.2113771}}

\bibitem[\citeproctext]{ref-spuxe4th2025}
\CSLLeftMargin{25. }%
\CSLRightInline{Späth C. From best practices to severe testing: A methodological response to Büsch and Loffing (2024). \emph{German Journal of Exercise and Sport Research}. outubro 2025. doi:\href{https://doi.org/10.1007/s12662-025-01072-7}{10.1007/s12662-025-01072-7}}

\bibitem[\citeproctext]{ref-resnik2016}
\CSLLeftMargin{26. }%
\CSLRightInline{Resnik DB, Shamoo AE. Reproducibility and Research Integrity. \emph{Accountability in Research}. 2016;24(2):116--123. doi:\href{https://doi.org/10.1080/08989621.2016.1257387}{10.1080/08989621.2016.1257387}}

\bibitem[\citeproctext]{ref-hofner2015}
\CSLLeftMargin{27. }%
\CSLRightInline{Hofner B, Schmid M, Edler L. Reproducible research in statistics: A review and guidelines for the {\emph{Biometrical Journal}}. \emph{Biometrical Journal}. 2015;58(2):416--427. doi:\href{https://doi.org/10.1002/bimj.201500156}{10.1002/bimj.201500156}}

\bibitem[\citeproctext]{ref-mair2016}
\CSLLeftMargin{28. }%
\CSLRightInline{Mair P. Thou Shalt Be Reproducible! A Technology Perspective. \emph{Frontiers in Psychology}. 2016;7. doi:\href{https://doi.org/10.3389/fpsyg.2016.01079}{10.3389/fpsyg.2016.01079}}

\bibitem[\citeproctext]{ref-hinsen2011}
\CSLLeftMargin{29. }%
\CSLRightInline{Hinsen K. A data and code model for reproducible research and executable papers. \emph{Procedia Computer Science}. 2011;4:579--588. doi:\href{https://doi.org/10.1016/j.procs.2011.04.061}{10.1016/j.procs.2011.04.061}}

\bibitem[\citeproctext]{ref-ihaka1996}
\CSLLeftMargin{30. }%
\CSLRightInline{Ihaka R, Gentleman R. R: A Language for Data Analysis and Graphics. \emph{Journal of Computational and Graphical Statistics}. 1996;5(3):299. doi:\href{https://doi.org/10.2307/1390807}{10.2307/1390807}}

\bibitem[\citeproctext]{ref-introduc2020}
\CSLLeftMargin{31. }%
\CSLRightInline{Nwanganga F, Chapple M. Introduction to R and RStudio. Em: Nwanganga F, Chapple M, orgs. \emph{Practical Machine Learning in R}. John Wiley \& Sons, Ltd; 2020:25--52. doi:\href{https://doi.org/10.1002/9781119591542.ch2}{10.1002/9781119591542.ch2}}

\bibitem[\citeproctext]{ref-CRAN}
\CSLLeftMargin{32. }%
\CSLRightInline{R Core Team. The Comprehensive R Archive Network. 2021. \href{https://cran.r-project.org}{https://cran.r-project.org.}}

\bibitem[\citeproctext]{ref-R-rmarkdown}
\CSLLeftMargin{33. }%
\CSLRightInline{Allaire J, Xie Y, Dervieux C, et al. \emph{rmarkdown: Dynamic Documents for R}.; 2023. \href{https://CRAN.R-project.org/package=rmarkdown}{https://CRAN.R-project.org/package=rmarkdown.}}

\bibitem[\citeproctext]{ref-holmes2021}
\CSLLeftMargin{34. }%
\CSLRightInline{Holmes DT, Mobini M, McCudden CR. Reproducible manuscript preparation with RMarkdown application to JMSACL and other Elsevier Journals. \emph{Journal of Mass Spectrometry and Advances in the Clinical Lab}. 2021;22:8--16. doi:\href{https://doi.org/10.1016/j.jmsacl.2021.09.002}{10.1016/j.jmsacl.2021.09.002}}

\bibitem[\citeproctext]{ref-love2019}
\CSLLeftMargin{35. }%
\CSLRightInline{Love J, Selker R, Marsman M, et al. {\textbf{JASP}}: Graphical Statistical Software for Common Statistical Designs. \emph{Journal of Statistical Software}. 2019;88(2). doi:\href{https://doi.org/10.18637/jss.v088.i02}{10.18637/jss.v088.i02}}

\bibitem[\citeproctext]{ref-sahin2020}
\CSLLeftMargin{36. }%
\CSLRightInline{ŞAHİN M, AYBEK E. Jamovi: An Easy to Use Statistical Software for the Social Scientists. \emph{International Journal of Assessment Tools in Education}. 2020;6(4):670--692. doi:\href{https://doi.org/10.21449/ijate.661803}{10.21449/ijate.661803}}

\bibitem[\citeproctext]{ref-jmv}
\CSLLeftMargin{37. }%
\CSLRightInline{Selker R, Love J, Dropmann D. \emph{jmv: The {\textbraceleft}jamovi{\textbraceright} Analyses}.; 2023. \href{https://CRAN.R-project.org/package=jmv}{https://CRAN.R-project.org/package=jmv.}}

\bibitem[\citeproctext]{ref-jmvconnect}
\CSLLeftMargin{38. }%
\CSLRightInline{Love J. \emph{jmvconnect: Connect to the {\textbraceleft}jamovi{\textbraceright} Statistical Spreadsheet}.; 2022. \href{https://CRAN.R-project.org/package=jmvconnect}{https://CRAN.R-project.org/package=jmvconnect.}}

\bibitem[\citeproctext]{ref-racine2011}
\CSLLeftMargin{39. }%
\CSLRightInline{Racine JS. RStudio: A Platform{-}Independent IDE for R and Sweave. \emph{Journal of Applied Econometrics}. 2011;27(1):167--172. doi:\href{https://doi.org/10.1002/jae.1278}{10.1002/jae.1278}}

\bibitem[\citeproctext]{ref-learnr}
\CSLLeftMargin{40. }%
\CSLRightInline{Aden-Buie G, Schloerke B, Allaire J, Rossell Hayes A. \emph{learnr: Interactive Tutorials for R}.; 2023. \href{https://CRAN.R-project.org/package=learnr}{https://CRAN.R-project.org/package=learnr.}}

\bibitem[\citeproctext]{ref-SchwabSimon2021}
\CSLLeftMargin{41. }%
\CSLRightInline{Schwab, Simon, Held, Leonhard. Statistical programming: Small mistakes, big impacts. \emph{Wiley-Blackwell Publishing, Inc}. 2021. doi:\href{https://doi.org/10.5167/UZH-205154}{10.5167/UZH-205154}}

\bibitem[\citeproctext]{ref-Eglen2017}
\CSLLeftMargin{42. }%
\CSLRightInline{Eglen SJ, Marwick B, Halchenko YO, et al. Toward standard practices for sharing computer code and programs in neuroscience. \emph{Nature Neuroscience}. 2017;20(6):770--773. doi:\href{https://doi.org/10.1038/nn.4550}{10.1038/nn.4550}}

\bibitem[\citeproctext]{ref-formatR}
\CSLLeftMargin{43. }%
\CSLRightInline{Xie Y. \emph{formatR: Format R Code Automatically}.; 2022. \href{https://CRAN.R-project.org/package=formatR}{https://CRAN.R-project.org/package=formatR.}}

\bibitem[\citeproctext]{ref-styler}
\CSLLeftMargin{44. }%
\CSLRightInline{Müller K, Walthert L. \emph{styler: Non-Invasive Pretty Printing of R Code}.; 2023. \href{https://CRAN.R-project.org/package=styler}{https://CRAN.R-project.org/package=styler.}}

\bibitem[\citeproctext]{ref-lintr}
\CSLLeftMargin{45. }%
\CSLRightInline{Hester J, Angly F, Hyde R, et al. \emph{lintr: A {\textbraceleft}Linter{\textbraceright} for R Code}.; 2023. \href{https://CRAN.R-project.org/package=lintr}{https://CRAN.R-project.org/package=lintr.}}

\bibitem[\citeproctext]{ref-R_Packages}
\CSLLeftMargin{46. }%
\CSLRightInline{All R CRAN packages {[}Full List{]}. 2025. \url{https://r-packages.io/packages}. Acessado fevereiro 11, 2025.}

\bibitem[\citeproctext]{ref-utils}
\CSLLeftMargin{47. }%
\CSLRightInline{R Core Team. \emph{R: A Language and Environment for Statistical Computing}. Vienna, Austria: R Foundation for Statistical Computing; 2023. \href{https://www.R-project.org/}{https://www.R-project.org/.}}

\bibitem[\citeproctext]{ref-roxygen2}
\CSLLeftMargin{48. }%
\CSLRightInline{Wickham H, Danenberg P, Csárdi G, Eugster M. \emph{roxygen2: In-Line Documentation for R}.; 2024. doi:\href{https://doi.org/10.32614/CRAN.package.roxygen2}{10.32614/CRAN.package.roxygen2}}

\bibitem[\citeproctext]{ref-trisovic2022}
\CSLLeftMargin{49. }%
\CSLRightInline{Trisovic A, Lau MK, Pasquier T, Crosas M. A large-scale study on research code quality and execution. \emph{Scientific Data}. 2022;9(1). doi:\href{https://doi.org/10.1038/s41597-022-01143-6}{10.1038/s41597-022-01143-6}}

\bibitem[\citeproctext]{ref-officedown}
\CSLLeftMargin{50. }%
\CSLRightInline{Gohel D, Ross N. \emph{officedown: Enhanced {\textbraceleft}R Markdown{\textbraceright} Format for {\textbraceleft}Word{\textbraceright} and {\textbraceleft}PowerPoint{\textbraceright}}.; 2023. \href{https://CRAN.R-project.org/package=officedown}{https://CRAN.R-project.org/package=officedown.}}

\bibitem[\citeproctext]{ref-bookdown}
\CSLLeftMargin{51. }%
\CSLRightInline{Xie Y. \emph{bookdown: Authoring Books and Technical Documents with R Markdown}. Chapman; Hall/CRC; 2023. \href{https://bookdown.org/yihui/bookdown/}{https://bookdown.org/yihui/bookdown/.}}

\bibitem[\citeproctext]{ref-ioannidis2014}
\CSLLeftMargin{52. }%
\CSLRightInline{Ioannidis JPA. How to Make More Published Research True. \emph{PLoS Medicine}. 2014;11(10):e1001747. doi:\href{https://doi.org/10.1371/journal.pmed.1001747}{10.1371/journal.pmed.1001747}}

\bibitem[\citeproctext]{ref-projects}
\CSLLeftMargin{53. }%
\CSLRightInline{Krieger N, Perzynski A, Dalton J. \emph{projects: A Project Infrastructure for Researchers}.; 2021. \href{https://CRAN.R-project.org/package=projects}{https://CRAN.R-project.org/package=projects.}}

\bibitem[\citeproctext]{ref-schultze2023}
\CSLLeftMargin{54. }%
\CSLRightInline{Schultze A, Tazare J. The role of programming code sharing in improving the transparency of medical research. \emph{BMJ}. outubro 2023:p2402. doi:\href{https://doi.org/10.1136/bmj.p2402}{10.1136/bmj.p2402}}

\bibitem[\citeproctext]{ref-base}
\CSLLeftMargin{55. }%
\CSLRightInline{R Core Team. \emph{{R}: A Language and Environment for Statistical Computing}. Vienna, Austria: R Foundation for Statistical Computing; 2023. \href{https://www.R-project.org/}{https://www.R-project.org/.}}

\bibitem[\citeproctext]{ref-Zhao2023}
\CSLLeftMargin{56. }%
\CSLRightInline{Zhao Y, Xiao N, Anderson K, Zhang Y. Electronic common technical document submission with analysis using R. \emph{Clinical Trials}. 2022;20(1):89--92. doi:\href{https://doi.org/10.1177/17407745221123244}{10.1177/17407745221123244}}

\bibitem[\citeproctext]{ref-grateful}
\CSLLeftMargin{57. }%
\CSLRightInline{Francisco Rodríguez-Sánchez, Connor P. Jackson, Shaurita D. Hutchins. \emph{grateful: Facilitate citation of R packages}.; 2023. \href{https://github.com/Pakillo/grateful}{https://github.com/Pakillo/grateful.}}

\bibitem[\citeproctext]{ref-shields2005}
\CSLLeftMargin{58. }%
\CSLRightInline{Shields M. Information Literacy, Statistical Literacy, Data Literacy. \emph{IASSIST Quarterly}. 2005;28(2):6. doi:\href{https://doi.org/10.29173/iq790}{10.29173/iq790}}

\bibitem[\citeproctext]{ref-gal2002a}
\CSLLeftMargin{59. }%
\CSLRightInline{Gal I. Adults' Statistical Literacy: Meanings, Components, Responsibilities. \emph{International Statistical Review}. 2002;70(1):1--25. doi:\href{https://doi.org/10.1111/j.1751-5823.2002.tb00336.x}{10.1111/j.1751-5823.2002.tb00336.x}}

\bibitem[\citeproctext]{ref-sharma2017}
\CSLLeftMargin{60. }%
\CSLRightInline{Sharma S. Definitions and models of statistical literacy: a literature review. \emph{Open Review of Educational Research}. 2017;4(1):118--133. doi:\href{https://doi.org/10.1080/23265507.2017.1354313}{10.1080/23265507.2017.1354313}}

\bibitem[\citeproctext]{ref-hidayati2020}
\CSLLeftMargin{61. }%
\CSLRightInline{Hidayati NA, Waluya SB, Rochmad, Wardono. Statistics literacy: what, why and how? \emph{Journal of Physics: Conference Series}. 2020;1613(1):012080. doi:\href{https://doi.org/10.1088/1742-6596/1613/1/012080}{10.1088/1742-6596/1613/1/012080}}

\bibitem[\citeproctext]{ref-gould2017}
\CSLLeftMargin{62. }%
\CSLRightInline{GOULD R. DATA LITERACY IS STATISTICAL LITERACY. \emph{STATISTICS EDUCATION RESEARCH JOURNAL}. 2017;16(1):22--25. doi:\href{https://doi.org/10.52041/serj.v16i1.209}{10.52041/serj.v16i1.209}}

\bibitem[\citeproctext]{ref-callingham2017}
\CSLLeftMargin{63. }%
\CSLRightInline{CALLINGHAM R, WATSON JM. THE DEVELOPMENT OF STATISTICAL LITERACY AT SCHOOL. \emph{STATISTICS EDUCATION RESEARCH JOURNAL}. 2017;16(1):181--201. doi:\href{https://doi.org/10.52041/serj.v16i1.223}{10.52041/serj.v16i1.223}}

\bibitem[\citeproctext]{ref-koga2022}
\CSLLeftMargin{64. }%
\CSLRightInline{Koga S. Characteristics of statistical literacy skills from the perspective of critical thinking. \emph{Teaching Statistics}. 2022;44(2):59--67. doi:\href{https://doi.org/10.1111/test.12302}{10.1111/test.12302}}

\bibitem[\citeproctext]{ref-wolff2019}
\CSLLeftMargin{65. }%
\CSLRightInline{Wolff RF, Moons KGM, Riley RD, et al. PROBAST: A Tool to Assess the Risk of Bias and Applicability of Prediction Model Studies. \emph{Annals of Internal Medicine}. 2019;170(1):51--58. doi:\href{https://doi.org/10.7326/m18-1376}{10.7326/m18-1376}}

\bibitem[\citeproctext]{ref-sterne2019}
\CSLLeftMargin{66. }%
\CSLRightInline{Sterne JAC, Savović J, Page MJ, et al. RoB 2: a revised tool for assessing risk of bias in randomised trials. \emph{BMJ}. agosto 2019:l4898. doi:\href{https://doi.org/10.1136/bmj.l4898}{10.1136/bmj.l4898}}

\bibitem[\citeproctext]{ref-shea2017}
\CSLLeftMargin{67. }%
\CSLRightInline{Shea BJ, Reeves BC, Wells G, et al. AMSTAR 2: a critical appraisal tool for systematic reviews that include randomised or non-randomised studies of healthcare interventions, or both. \emph{BMJ}. setembro 2017:j4008. doi:\href{https://doi.org/10.1136/bmj.j4008}{10.1136/bmj.j4008}}

\bibitem[\citeproctext]{ref-sterne2016}
\CSLLeftMargin{68. }%
\CSLRightInline{Sterne JA, Hernán MA, Reeves BC, et al. ROBINS-I: a tool for assessing risk of bias in non-randomised studies of interventions. \emph{BMJ}. outubro 2016:i4919. doi:\href{https://doi.org/10.1136/bmj.i4919}{10.1136/bmj.i4919}}

\bibitem[\citeproctext]{ref-whiting2016}
\CSLLeftMargin{69. }%
\CSLRightInline{Whiting P, Savović J, Higgins JPT, et al. ROBIS: A new tool to assess risk of bias in systematic reviews was developed. \emph{Journal of Clinical Epidemiology}. 2016;69:225--234. doi:\href{https://doi.org/10.1016/j.jclinepi.2015.06.005}{10.1016/j.jclinepi.2015.06.005}}

\bibitem[\citeproctext]{ref-whiting2011}
\CSLLeftMargin{70. }%
\CSLRightInline{Whiting PF, Rutjes AWS, Westwood ME, et al. QUADAS-2: A Revised Tool for the Quality Assessment of Diagnostic Accuracy Studies. \emph{Annals of Internal Medicine}. 2011;155(8):529--536. doi:\href{https://doi.org/10.7326/0003-4819-155-8-201110180-00009}{10.7326/0003-4819-155-8-201110180-00009}}

\bibitem[\citeproctext]{ref-polin2023}
\CSLLeftMargin{71. }%
\CSLRightInline{Polin BA, Benisaac E. A longitudinal analysis of the hot hand and gambler{'}s fallacy biases. \emph{Judgment and Decision Making}. 2023;18. doi:\href{https://doi.org/10.1017/jdm.2023.23}{10.1017/jdm.2023.23}}

\bibitem[\citeproctext]{ref-meng2018}
\CSLLeftMargin{72. }%
\CSLRightInline{Meng XL. Statistical paradises and paradoxes in big data (I): Law of large populations, big data paradox, and the 2016 US presidential election. \emph{The Annals of Applied Statistics}. 2018;12(2). doi:\href{https://doi.org/10.1214/18-aoas1161sf}{10.1214/18-aoas1161sf}}

\bibitem[\citeproctext]{ref-abelson1985}
\CSLLeftMargin{73. }%
\CSLRightInline{Abelson RP. A variance explanation paradox: When a little is a lot. \emph{Psychological Bulletin}. 1985;97(1):129--133. doi:\href{https://doi.org/10.1037/0033-2909.97.1.129}{10.1037/0033-2909.97.1.129}}

\bibitem[\citeproctext]{ref-berkson1946}
\CSLLeftMargin{74. }%
\CSLRightInline{Berkson J. Limitations of the Application of Fourfold Table Analysis to Hospital Data. \emph{Biometrics Bulletin}. 1946;2(3):47. doi:\href{https://doi.org/10.2307/3002000}{10.2307/3002000}}

\bibitem[\citeproctext]{ref-ellsberg1961}
\CSLLeftMargin{75. }%
\CSLRightInline{Ellsberg D. Risk, Ambiguity, and the Savage Axioms. \emph{The Quarterly Journal of Economics}. 1961;75(4):643. doi:\href{https://doi.org/10.2307/1884324}{10.2307/1884324}}

\bibitem[\citeproctext]{ref-freedman1983}
\CSLLeftMargin{76. }%
\CSLRightInline{Freedman DA, Freedman DA. A Note on Screening Regression Equations. \emph{The American Statistician}. 1983;37(2):152--155. doi:\href{https://doi.org/10.1080/00031305.1983.10482729}{10.1080/00031305.1983.10482729}}

\bibitem[\citeproctext]{ref-freedman1989}
\CSLLeftMargin{77. }%
\CSLRightInline{Freedman LS, Pee D. Return to a Note on Screening Regression Equations. \emph{The American Statistician}. 1989;43(4):279. doi:\href{https://doi.org/10.2307/2685389}{10.2307/2685389}}

\bibitem[\citeproctext]{ref-hand1992}
\CSLLeftMargin{78. }%
\CSLRightInline{Hand DJ. On Comparing Two Treatments. \emph{The American Statistician}. 1992;46(3):190--192. doi:\href{https://doi.org/10.1080/00031305.1992.10475881}{10.1080/00031305.1992.10475881}}

\bibitem[\citeproctext]{ref-lindley1957}
\CSLLeftMargin{79. }%
\CSLRightInline{LINDLEY DV. A STATISTICAL PARADOX. \emph{Biometrika}. 1957;44(1-2):187--192. doi:\href{https://doi.org/10.1093/biomet/44.1-2.187}{10.1093/biomet/44.1-2.187}}

\bibitem[\citeproctext]{ref-lord1967}
\CSLLeftMargin{80. }%
\CSLRightInline{Lord FM. A paradox in the interpretation of group comparisons. \emph{Psychological Bulletin}. 1967;68(5):304--305. doi:\href{https://doi.org/10.1037/h0025105}{10.1037/h0025105}}

\bibitem[\citeproctext]{ref-lord1969}
\CSLLeftMargin{81. }%
\CSLRightInline{Lord FM. Statistical adjustments when comparing preexisting groups. \emph{Psychological Bulletin}. 1969;72(5):336--337. doi:\href{https://doi.org/10.1037/h0028108}{10.1037/h0028108}}

\bibitem[\citeproctext]{ref-simpson1951}
\CSLLeftMargin{82. }%
\CSLRightInline{Simpson EH. The Interpretation of Interaction in Contingency Tables. \emph{Journal of the Royal Statistical Society: Series B (Methodological)}. 1951;13(2):238--241. doi:\href{https://doi.org/10.1111/j.2517-6161.1951.tb00088.x}{10.1111/j.2517-6161.1951.tb00088.x}}

\bibitem[\citeproctext]{ref-blyth1972}
\CSLLeftMargin{83. }%
\CSLRightInline{Blyth CR. On Simpson's Paradox and the Sure-Thing Principle. \emph{Journal of the American Statistical Association}. 1972;67(338):364--366. doi:\href{https://doi.org/10.1080/01621459.1972.10482387}{10.1080/01621459.1972.10482387}}

\bibitem[\citeproctext]{ref-pearl2014}
\CSLLeftMargin{84. }%
\CSLRightInline{Pearl J. Comment: Understanding Simpson{'}s Paradox. \emph{The American Statistician}. 2014;68(1):8--13. doi:\href{https://doi.org/10.1080/00031305.2014.876829}{10.1080/00031305.2014.876829}}

\bibitem[\citeproctext]{ref-stein1956}
\CSLLeftMargin{85. }%
\CSLRightInline{Stein C. INADMISSIBILITY OF THE USUAL ESTIMATOR FOR THE MEAN OF A MULTIVARIATE NORMAL DISTRIBUTION. Em: \emph{Proceedings of the Third Berkeley Symposium on Mathematical Statistics and Probability, Volume I}. University of California Press; 1956:197--206. doi:\href{https://doi.org/10.1525/9780520313880-018}{10.1525/9780520313880-018}}

\bibitem[\citeproctext]{ref-de1996}
\CSLLeftMargin{86. }%
\CSLRightInline{De S, Sen A. The generalised Gamow-Stern problem. \emph{The Mathematical Gazette}. 1996;80(488):345--348. doi:\href{https://doi.org/10.2307/3619568}{10.2307/3619568}}

\bibitem[\citeproctext]{ref-feld1991}
\CSLLeftMargin{87. }%
\CSLRightInline{Feld SL. Why Your Friends Have More Friends Than You Do. \emph{American Journal of Sociology}. 1991;96(6):1464--1477. doi:\href{https://doi.org/10.1086/229693}{10.1086/229693}}

\bibitem[\citeproctext]{ref-john2012}
\CSLLeftMargin{88. }%
\CSLRightInline{John LK, Loewenstein G, Prelec D. Measuring the Prevalence of Questionable Research Practices With Incentives for Truth Telling. \emph{Psychological Science}. 2012;23(5):524--532. doi:\href{https://doi.org/10.1177/0956797611430953}{10.1177/0956797611430953}}

\bibitem[\citeproctext]{ref-bausell2021}
\CSLLeftMargin{89. }%
\CSLRightInline{Bausell RB. Too Much Medicine: Not Enough Health. Em: \emph{The Problem with Science: The Reproducibility Crisis and What to do About It}. New York: Oxford University Press; 2021:56--C3.P203. doi:\href{https://doi.org/10.1093/oso/9780197536537.003.0004}{10.1093/oso/9780197536537.003.0004}}

\bibitem[\citeproctext]{ref-neoh2023}
\CSLLeftMargin{90. }%
\CSLRightInline{Neoh MJY, Carollo A, Lee A, Esposito G. Fifty years of research on questionable research practises in science: quantitative analysis of co-citation patterns. \emph{Royal Society Open Science}. 2023;10(10). doi:\href{https://doi.org/10.1098/rsos.230677}{10.1098/rsos.230677}}

\bibitem[\citeproctext]{ref-Kleinert2009}
\CSLLeftMargin{91. }%
\CSLRightInline{Kleinert S. COPE's retraction guidelines. \emph{The Lancet}. 2009;374(9705):1876--1877. doi:\href{https://doi.org/10.1016/s0140-6736(09)62074-2}{10.1016/s0140-6736(09)62074-2}}

\bibitem[\citeproctext]{ref-Kerr1998}
\CSLLeftMargin{92. }%
\CSLRightInline{Kerr NL. HARKing: Hypothesizing After the Results are Known. \emph{Personality and Social Psychology Review}. 1998;2(3):196--217. doi:\href{https://doi.org/10.1207/s15327957pspr0203_4}{10.1207/s15327957pspr0203\_4}}

\bibitem[\citeproctext]{ref-degroot2014}
\CSLLeftMargin{93. }%
\CSLRightInline{Groot AD de. The meaning of {``}significance{''} for different types of research {[}translated and annotated by Eric-Jan Wagenmakers, Denny Borsboom, Josine Verhagen, Rogier Kievit, Marjan Bakker, Angelique Cramer, Dora Matzke, Don Mellenbergh, and Han L. J. van der Maas{]}. \emph{Acta Psychologica}. 2014;148:188--194. doi:\href{https://doi.org/10.1016/j.actpsy.2014.02.001}{10.1016/j.actpsy.2014.02.001}}

\bibitem[\citeproctext]{ref-andrade2021}
\CSLLeftMargin{94. }%
\CSLRightInline{Andrade C. HARKing, Cherry-Picking, P-Hacking, Fishing Expeditions, and Data Dredging and Mining as Questionable Research Practices. \emph{The Journal of Clinical Psychiatry}. 2021;82(1). doi:\href{https://doi.org/10.4088/jcp.20f13804}{10.4088/jcp.20f13804}}

\bibitem[\citeproctext]{ref-stefan2023}
\CSLLeftMargin{95. }%
\CSLRightInline{Stefan AM, Schönbrodt FD. Big little lies: a compendium and simulation of{\emph{p}}-hacking strategies. \emph{Royal Society Open Science}. 2023;10(2). doi:\href{https://doi.org/10.1098/rsos.220346}{10.1098/rsos.220346}}

\bibitem[\citeproctext]{ref-chuard2019}
\CSLLeftMargin{96. }%
\CSLRightInline{Chuard PJC, Vrtílek M, Head ML, Jennions MD. Evidence that nonsignificant results are sometimes preferred: Reverse P-hacking or selective reporting? \emph{PLOS Biology}. 2019;17(1):e3000127. doi:\href{https://doi.org/10.1371/journal.pbio.3000127}{10.1371/journal.pbio.3000127}}

\bibitem[\citeproctext]{ref-Sasaki2023}
\CSLLeftMargin{97. }%
\CSLRightInline{Sasaki K, Yamada Y. SPARKing: Sample-size planning after the results are known. \emph{Frontiers in Human Neuroscience}. 2023;17. doi:\href{https://doi.org/10.3389/fnhum.2023.912338}{10.3389/fnhum.2023.912338}}

\bibitem[\citeproctext]{ref-armitage1969}
\CSLLeftMargin{98. }%
\CSLRightInline{Armitage P, McPherson CK, Rowe BC. Repeated Significance Tests on Accumulating Data. \emph{Journal of the Royal Statistical Society Series A (General)}. 1969;132(2):235. doi:\href{https://doi.org/10.2307/2343787}{10.2307/2343787}}

\bibitem[\citeproctext]{ref-hutton2000}
\CSLLeftMargin{99. }%
\CSLRightInline{Hutton JL, Williamson PR. Bias in Meta-Analysis Due to Outcome Variable Selection Within Studies. \emph{Journal of the Royal Statistical Society Series C: Applied Statistics}. 2000;49(3):359--370. doi:\href{https://doi.org/10.1111/1467-9876.00197}{10.1111/1467-9876.00197}}

\bibitem[\citeproctext]{ref-horton1995}
\CSLLeftMargin{100. }%
\CSLRightInline{Horton R. The rhetoric of research. \emph{BMJ}. 1995;310(6985):985--987. doi:\href{https://doi.org/10.1136/bmj.310.6985.985}{10.1136/bmj.310.6985.985}}

\bibitem[\citeproctext]{ref-chiu2017}
\CSLLeftMargin{101. }%
\CSLRightInline{Chiu K, Grundy Q, Bero L. {`}Spin{'} in published biomedical literature: A methodological systematic review. Boutron I, org. \emph{PLOS Biology}. 2017;15(9):e2002173. doi:\href{https://doi.org/10.1371/journal.pbio.2002173}{10.1371/journal.pbio.2002173}}

\bibitem[\citeproctext]{ref-picano2024}
\CSLLeftMargin{102. }%
\CSLRightInline{Picano E. Who is the author: genuine, honorary, ghost, gold, and fake authors? \emph{Exploration of Cardiology}. 2024;2(3):88--96. doi:\href{https://doi.org/10.37349/ec.2024.00024}{10.37349/ec.2024.00024}}

\bibitem[\citeproctext]{ref-montori2000}
\CSLLeftMargin{103. }%
\CSLRightInline{Montori VM, Smieja M, Guyatt GH. Publication Bias: A Brief Review for Clinicians. \emph{Mayo Clinic Proceedings}. 2000;75(12):1284--1288. doi:\href{https://doi.org/10.4065/75.12.1284}{10.4065/75.12.1284}}

\bibitem[\citeproctext]{ref-nosek2018}
\CSLLeftMargin{104. }%
\CSLRightInline{Nosek BA, Ebersole CR, DeHaven AC, Mellor DT. The preregistration revolution. \emph{Proceedings of the National Academy of Sciences}. 2018;115(11):2600--2606. doi:\href{https://doi.org/10.1073/pnas.1708274114}{10.1073/pnas.1708274114}}

\bibitem[\citeproctext]{ref-p.simmons2021}
\CSLLeftMargin{105. }%
\CSLRightInline{P. Simmons J, D. Nelson L, Simonsohn U. Pre{-}registration: Why and How. \emph{Journal of Consumer Psychology}. 2021;31(1):151--162. doi:\href{https://doi.org/10.1002/jcpy.1208}{10.1002/jcpy.1208}}

\bibitem[\citeproctext]{ref-retractcheck}
\CSLLeftMargin{106. }%
\CSLRightInline{Hartgerink C, Aust F. \emph{retractcheck: Retraction Scanner}.; 2025. \href{https://github.com/chartgerink/retractcheck}{https://github.com/chartgerink/retractcheck.}}

\bibitem[\citeproctext]{ref-Altman1999}
\CSLLeftMargin{107. }%
\CSLRightInline{Altman DG, Bland JM. Statistics notes Variables and parameters. \emph{BMJ}. 1999;318(7199):1667--1667. doi:\href{https://doi.org/10.1136/bmj.318.7199.1667}{10.1136/bmj.318.7199.1667}}

\bibitem[\citeproctext]{ref-vetter2017}
\CSLLeftMargin{108. }%
\CSLRightInline{Vetter TR. Fundamentals of Research Data and Variables. \emph{Anesthesia \& Analgesia}. 2017;125(4):1375--1380. doi:\href{https://doi.org/10.1213/ane.0000000000002370}{10.1213/ane.0000000000002370}}

\bibitem[\citeproctext]{ref-Ali2016}
\CSLLeftMargin{109. }%
\CSLRightInline{Ali Z, Bhaskar Sb. Basic statistical tools in research and data analysis. \emph{Indian Journal of Anaesthesia}. 2016;60(9):662. doi:\href{https://doi.org/10.4103/0019-5049.190623}{10.4103/0019-5049.190623}}

\bibitem[\citeproctext]{ref-Dettori2018}
\CSLLeftMargin{110. }%
\CSLRightInline{Dettori JR, Norvell DC. The Anatomy of Data. \emph{Global Spine Journal}. 2018;8(3):311--313. doi:\href{https://doi.org/10.1177/2192568217746998}{10.1177/2192568217746998}}

\bibitem[\citeproctext]{ref-kaliyadan2019}
\CSLLeftMargin{111. }%
\CSLRightInline{Kaliyadan F, Kulkarni V. Types of variables, descriptive statistics, and sample size. \emph{Indian Dermatology Online Journal}. 2019;10(1):82. doi:\href{https://doi.org/10.4103/idoj.idoj_468_18}{10.4103/idoj.idoj\_468\_18}}

\bibitem[\citeproctext]{ref-barkan2015}
\CSLLeftMargin{112. }%
\CSLRightInline{Barkan H. Statistics in clinical research: Important considerations. \emph{Annals of Cardiac Anaesthesia}. 2015;18(1):74. doi:\href{https://doi.org/10.4103/0971-9784.148325}{10.4103/0971-9784.148325}}

\bibitem[\citeproctext]{ref-Bland1996}
\CSLLeftMargin{113. }%
\CSLRightInline{Bland JM, Altman DG. Statistics Notes: Transforming data. \emph{BMJ}. 1996;312(7033):770--770. doi:\href{https://doi.org/10.1136/bmj.312.7033.770}{10.1136/bmj.312.7033.770}}

\bibitem[\citeproctext]{ref-Fedorov2009}
\CSLLeftMargin{114. }%
\CSLRightInline{Fedorov V, Mannino F, Zhang R. Consequences of dichotomization. \emph{Pharmaceutical Statistics}. 2009;8(1):50--61. doi:\href{https://doi.org/10.1002/pst.331}{10.1002/pst.331}}

\bibitem[\citeproctext]{ref-osborne2010}
\CSLLeftMargin{115. }%
\CSLRightInline{Osborne J. Improving your data transformations: Applying the Box-Cox transformation. \emph{University of Massachusetts Amherst}. 2010. doi:\href{https://doi.org/10.7275/QBPC-GK17}{10.7275/QBPC-GK17}}

\bibitem[\citeproctext]{ref-box1964}
\CSLLeftMargin{116. }%
\CSLRightInline{Box GEP, Cox DR. An Analysis of Transformations. \emph{Journal of the Royal Statistical Society: Series B (Methodological)}. 1964;26(2):211--243. doi:\href{https://doi.org/10.1111/j.2517-6161.1964.tb00553.x}{10.1111/j.2517-6161.1964.tb00553.x}}

\bibitem[\citeproctext]{ref-MASS}
\CSLLeftMargin{117. }%
\CSLRightInline{Venables WN, Ripley BD. \emph{Modern Applied Statistics with S}. Springer; 2002. \href{https://www.stats.ox.ac.uk/pub/MASS4/}{https://www.stats.ox.ac.uk/pub/MASS4/.}}

\bibitem[\citeproctext]{ref-MacCallum2002}
\CSLLeftMargin{118. }%
\CSLRightInline{MacCallum RC, Zhang S, Preacher KJ, Rucker DD. On the practice of dichotomization of quantitative variables. \emph{Psychological Methods}. 2002;7(1):19--40. doi:\href{https://doi.org/10.1037/1082-989x.7.1.19}{10.1037/1082-989x.7.1.19}}

\bibitem[\citeproctext]{ref-Altman2006}
\CSLLeftMargin{119. }%
\CSLRightInline{Altman DG, Royston P. The cost of dichotomising continuous variables. \emph{BMJ}. 2006;332(7549):1080.1. doi:\href{https://doi.org/10.1136/bmj.332.7549.1080}{10.1136/bmj.332.7549.1080}}

\bibitem[\citeproctext]{ref-Royston2006}
\CSLLeftMargin{120. }%
\CSLRightInline{Royston P, Altman DG, Sauerbrei W. Dichotomizing continuous predictors in multiple regression: a bad idea. \emph{Statistics in Medicine}. 2005;25(1):127--141. doi:\href{https://doi.org/10.1002/sim.2331}{10.1002/sim.2331}}

\bibitem[\citeproctext]{ref-Collins2016}
\CSLLeftMargin{121. }%
\CSLRightInline{Collins GS, Ogundimu EO, Cook JA, Manach YL, Altman DG. Quantifying the impact of different approaches for handling continuous predictors on the performance of a prognostic model. \emph{Statistics in Medicine}. 2016;35(23):4124--4135. doi:\href{https://doi.org/10.1002/sim.6986}{10.1002/sim.6986}}

\bibitem[\citeproctext]{ref-Prince2017}
\CSLLeftMargin{122. }%
\CSLRightInline{Nelson SLP, Ramakrishnan V, Nietert PJ, Kamen DL, Ramos PS, Wolf BJ. An evaluation of common methods for dichotomization of continuous variables to discriminate disease status. \emph{Communications in Statistics -- Theory and Methods}. 2017;46(21):10823--10834. doi:\href{https://doi.org/10.1080/03610926.2016.1248783}{10.1080/03610926.2016.1248783}}

\bibitem[\citeproctext]{ref-Bennette2012}
\CSLLeftMargin{123. }%
\CSLRightInline{Bennette C, Vickers A. Against quantiles: categorization of continuous variables in epidemiologic research, and its discontents. \emph{BMC Medical Research Methodology}. 2012;12(1). doi:\href{https://doi.org/10.1186/1471-2288-12-21}{10.1186/1471-2288-12-21}}

\bibitem[\citeproctext]{ref-questionr}
\CSLLeftMargin{124. }%
\CSLRightInline{Barnier J, Briatte F, Larmarange J. \emph{questionr: Functions to Make Surveys Processing Easier}.; 2023. \href{https://CRAN.R-project.org/package=questionr}{https://CRAN.R-project.org/package=questionr.}}

\bibitem[\citeproctext]{ref-aguinis2008}
\CSLLeftMargin{125. }%
\CSLRightInline{Aguinis H, Pierce CA, Culpepper SA. Scale Coarseness as a Methodological Artifact. \emph{Organizational Research Methods}. 2008;12(4):623--652. doi:\href{https://doi.org/10.1177/1094428108318065}{10.1177/1094428108318065}}

\bibitem[\citeproctext]{ref-YOUDEN1950}
\CSLLeftMargin{126. }%
\CSLRightInline{Youden WJ. Index for rating diagnostic tests. \emph{Cancer}. 1950;3(1):32--35. doi:\href{https://doi.org/10.1002/1097-0142(1950)3:1\%3C32::aid-cncr2820030106\%3E3.0.co;2-3}{10.1002/1097-0142(1950)3:1\textless32::aid-cncr2820030106\textgreater3.0.co;2-3}}

\bibitem[\citeproctext]{ref-strobl2007}
\CSLLeftMargin{127. }%
\CSLRightInline{Strobl C, Boulesteix AL, Augustin T. Unbiased split selection for classification trees based on the Gini Index. \emph{Computational Statistics \& Data Analysis}. 2007;52(1):483--501. doi:\href{https://doi.org/10.1016/j.csda.2006.12.030}{10.1016/j.csda.2006.12.030}}

\bibitem[\citeproctext]{ref-pearson1900}
\CSLLeftMargin{128. }%
\CSLRightInline{Pearson K. X. {\emph{On the criterion that a given system of deviations from the probable in the case of a correlated system of variables is such that it can be reasonably supposed to have arisen from random sampling}}. \emph{The London, Edinburgh, and Dublin Philosophical Magazine and Journal of Science}. 1900;50(302):157--175. doi:\href{https://doi.org/10.1080/14786440009463897}{10.1080/14786440009463897}}

\bibitem[\citeproctext]{ref-Greiner2000}
\CSLLeftMargin{129. }%
\CSLRightInline{Greiner M, Pfeiffer D, Smith RD. Principles and practical application of the receiver-operating characteristic analysis for diagnostic tests. \emph{Preventive Veterinary Medicine}. 2000;45(1-2):23--41. doi:\href{https://doi.org/10.1016/s0167-5877(00)00115-x}{10.1016/s0167-5877(00)00115-x}}

\bibitem[\citeproctext]{ref-fleiss1971}
\CSLLeftMargin{130. }%
\CSLRightInline{Fleiss JL. Measuring nominal scale agreement among many raters. \emph{Psychological Bulletin}. 1971;76(5):378--382. doi:\href{https://doi.org/10.1037/h0031619}{10.1037/h0031619}}

\bibitem[\citeproctext]{ref-stats}
\CSLLeftMargin{131. }%
\CSLRightInline{R Core Team. \emph{R: A Language and Environment for Statistical Computing}.; 2025. \href{https://www.R-project.org/}{https://www.R-project.org/.}}

\bibitem[\citeproctext]{ref-Olson2021}
\CSLLeftMargin{132. }%
\CSLRightInline{Olson K. What Are Data? \emph{Qualitative Health Research}. 2021;31(9):1567--1569. doi:\href{https://doi.org/10.1177/10497323211015960}{10.1177/10497323211015960}}

\bibitem[\citeproctext]{ref-van2022a}
\CSLLeftMargin{133. }%
\CSLRightInline{Smeden M van. A Very Short List of Common Pitfalls in Research Design, Data Analysis, and Reporting. \emph{PRiMER}. 2022;6. doi:\href{https://doi.org/10.22454/PRiMER.2022.511416}{10.22454/PRiMER.2022.511416}}

\bibitem[\citeproctext]{ref-Baillie2022}
\CSLLeftMargin{134. }%
\CSLRightInline{Baillie M, Cessie S le, Schmidt CO, Lusa L, Huebner M. Ten simple rules for initial data analysis. \emph{PLOS Computational Biology}. 2022;18(2):e1009819. doi:\href{https://doi.org/10.1371/journal.pcbi.1009819}{10.1371/journal.pcbi.1009819}}

\bibitem[\citeproctext]{ref-buttliere2021}
\CSLLeftMargin{135. }%
\CSLRightInline{Buttliere B. Adopting standard variable labels solves many of the problems with sharing and reusing data. \emph{Methodological Innovations}. 2021;14(2):205979912110266. doi:\href{https://doi.org/10.1177/20597991211026616}{10.1177/20597991211026616}}

\bibitem[\citeproctext]{ref-units}
\CSLLeftMargin{136. }%
\CSLRightInline{Pebesma E, Mailund T, Hiebert J. Measurement Units in {\textbraceleft}R{\textbraceright}. \emph{The R Journal}. 2016;8. doi:\href{https://doi.org/10.32614/RJ-2016-061}{10.32614/RJ-2016-061}}

\bibitem[\citeproctext]{ref-janitor}
\CSLLeftMargin{137. }%
\CSLRightInline{Firke S. \emph{janitor: Simple Tools for Examining and Cleaning Dirty Data}.; 2023. \href{https://CRAN.R-project.org/package=janitor}{https://CRAN.R-project.org/package=janitor.}}

\bibitem[\citeproctext]{ref-Hmisc}
\CSLLeftMargin{138. }%
\CSLRightInline{Harrell Jr FE. \emph{Hmisc: Harrell Miscellaneous}.; 2023. \href{https://CRAN.R-project.org/package=Hmisc}{https://CRAN.R-project.org/package=Hmisc.}}

\bibitem[\citeproctext]{ref-likert}
\CSLLeftMargin{139. }%
\CSLRightInline{Bryer J, Speerschneider K. \emph{likert: Analysis and Visualization Likert Items}.; 2016. \href{https://CRAN.R-project.org/package=likert}{https://CRAN.R-project.org/package=likert.}}

\bibitem[\citeproctext]{ref-ggstats}
\CSLLeftMargin{140. }%
\CSLRightInline{Larmarange J. \emph{ggstats: Extension to ggplot2 for Plotting Stats}.; 2025. doi:\href{https://doi.org/10.32614/CRAN.package.ggstats}{10.32614/CRAN.package.ggstats}}

\bibitem[\citeproctext]{ref-ferris2004}
\CSLLeftMargin{141. }%
\CSLRightInline{Ferris TLJ. A new definition of measurement. \emph{Measurement}. 2004;36(1):101--109. doi:\href{https://doi.org/10.1016/j.measurement.2004.03.001}{10.1016/j.measurement.2004.03.001}}

\bibitem[\citeproctext]{ref-healy1978}
\CSLLeftMargin{142. }%
\CSLRightInline{Healy MJR, Goldstein H. Regression to the mean. \emph{Annals of Human Biology}. 1978;5(3):277--280. doi:\href{https://doi.org/10.1080/03014467800002891}{10.1080/03014467800002891}}

\bibitem[\citeproctext]{ref-altman1983}
\CSLLeftMargin{143. }%
\CSLRightInline{Altman DG, Bland JM. Measurement in Medicine: The Analysis of Method Comparison Studies. \emph{The Statistician}. 1983;32(3):307. doi:\href{https://doi.org/10.2307/2987937}{10.2307/2987937}}

\bibitem[\citeproctext]{ref-menditto2006}
\CSLLeftMargin{144. }%
\CSLRightInline{Menditto A, Patriarca M, Magnusson B. Understanding the meaning of accuracy, trueness and precision. \emph{Accreditation and Quality Assurance}. 2006;12(1):45--47. doi:\href{https://doi.org/10.1007/s00769-006-0191-z}{10.1007/s00769-006-0191-z}}

\bibitem[\citeproctext]{ref-Streiner2006}
\CSLLeftMargin{145. }%
\CSLRightInline{Streiner DL, Norman GR. {``}Precision{''} and {``}Accuracy{''}: Two Terms That Are Neither. \emph{Journal of Clinical Epidemiology}. 2006;59(4):327--330. doi:\href{https://doi.org/10.1016/j.jclinepi.2005.09.005}{10.1016/j.jclinepi.2005.09.005}}

\bibitem[\citeproctext]{ref-tierney2023}
\CSLLeftMargin{146. }%
\CSLRightInline{Tierney N, Cook D. Expanding Tidy Data Principles to Facilitate Missing Data Exploration, Visualization and Assessment of Imputations. \emph{Journal of Statistical Software}. 2023;105(7). doi:\href{https://doi.org/10.18637/jss.v105.i07}{10.18637/jss.v105.i07}}

\bibitem[\citeproctext]{ref-DataEditR}
\CSLLeftMargin{147. }%
\CSLRightInline{Hammill D. \emph{DataEditR: An Interactive Editor for Viewing, Entering, Filtering \& Editing Data}.; 2022. \href{https://CRAN.R-project.org/package=DataEditR}{https://CRAN.R-project.org/package=DataEditR.}}

\bibitem[\citeproctext]{ref-broman2018}
\CSLLeftMargin{148. }%
\CSLRightInline{Broman KW, Woo KH. Data Organization in Spreadsheets. \emph{The American Statistician}. 2018;72(1):2--10. doi:\href{https://doi.org/10.1080/00031305.2017.1375989}{10.1080/00031305.2017.1375989}}

\bibitem[\citeproctext]{ref-Juluru2015}
\CSLLeftMargin{149. }%
\CSLRightInline{Juluru K, Eng J. Use of Spreadsheets for Research Data Collection and Preparation: \emph{Academic Radiology}. 2015;22(12):1592--1599. doi:\href{https://doi.org/10.1016/j.acra.2015.08.024}{10.1016/j.acra.2015.08.024}}

\bibitem[\citeproctext]{ref-data.table}
\CSLLeftMargin{150. }%
\CSLRightInline{Dowle M, Srinivasan A. \emph{data.table: Extension of `data.frame`}.; 2023. \href{https://CRAN.R-project.org/package=data.table}{https://CRAN.R-project.org/package=data.table.}}

\bibitem[\citeproctext]{ref-Altman2007}
\CSLLeftMargin{151. }%
\CSLRightInline{Altman DG, Bland JM. Missing data. \emph{BMJ}. 2007;334(7590):424--424. doi:\href{https://doi.org/10.1136/bmj.38977.682025.2c}{10.1136/bmj.38977.682025.2c}}

\bibitem[\citeproctext]{ref-Heymans2022}
\CSLLeftMargin{152. }%
\CSLRightInline{Heymans MW, Twisk JWR. Handling missing data in clinical research. \emph{Journal of Clinical Epidemiology}. setembro 2022. doi:\href{https://doi.org/10.1016/j.jclinepi.2022.08.016}{10.1016/j.jclinepi.2022.08.016}}

\bibitem[\citeproctext]{ref-carpenter2021}
\CSLLeftMargin{153. }%
\CSLRightInline{Carpenter JR, Smuk M. Missing data: A statistical framework for practice. \emph{Biometrical Journal}. 2021;63(5):915--947. doi:\href{https://doi.org/10.1002/bimj.202000196}{10.1002/bimj.202000196}}

\bibitem[\citeproctext]{ref-misty}
\CSLLeftMargin{154. }%
\CSLRightInline{Yanagida T. \emph{misty: Miscellaneous Functions}.; 2023. \href{https://CRAN.R-project.org/package=misty}{https://CRAN.R-project.org/package=misty.}}

\bibitem[\citeproctext]{ref-little1988}
\CSLLeftMargin{155. }%
\CSLRightInline{Little RJA. A Test of Missing Completely at Random for Multivariate Data with Missing Values. \emph{Journal of the American Statistical Association}. 1988;83(404):1198--1202. doi:\href{https://doi.org/10.1080/01621459.1988.10478722}{10.1080/01621459.1988.10478722}}

\bibitem[\citeproctext]{ref-naniar}
\CSLLeftMargin{156. }%
\CSLRightInline{Tierney N, Cook D. Expanding Tidy Data Principles to Facilitate Missing Data Exploration, Visualization and Assessment of Imputations. \emph{Journal of Statistical Software}. 2023;105(7):1--31. doi:\href{https://doi.org/10.18637/jss.v105.i07}{10.18637/jss.v105.i07}}

\bibitem[\citeproctext]{ref-Akl2015}
\CSLLeftMargin{157. }%
\CSLRightInline{Akl EA, Shawwa K, Kahale LA, et al. Reporting missing participant data in randomised trials: systematic survey of the methodological literature and a proposed guide. \emph{BMJ Open}. 2015;5(12):e008431. doi:\href{https://doi.org/10.1136/bmjopen-2015-008431}{10.1136/bmjopen-2015-008431}}

\bibitem[\citeproctext]{ref-austin2023}
\CSLLeftMargin{158. }%
\CSLRightInline{Austin PC, Buuren S van. Logistic regression vs. predictive mean matching for imputing binary covariates. \emph{Statistical Methods in Medical Research}. setembro 2023. doi:\href{https://doi.org/10.1177/09622802231198795}{10.1177/09622802231198795}}

\bibitem[\citeproctext]{ref-mice}
\CSLLeftMargin{159. }%
\CSLRightInline{Buuren S van, Groothuis-Oudshoorn K. {\textbraceleft}mice{\textbraceright}: Multivariate Imputation by Chained Equations in R. \emph{Journal of Statistical Software}. 2011;45:1--67. doi:\href{https://doi.org/10.18637/jss.v045.i03}{10.18637/jss.v045.i03}}

\bibitem[\citeproctext]{ref-rubin1986}
\CSLLeftMargin{160. }%
\CSLRightInline{Rubin DB. Statistical Matching Using File Concatenation with Adjusted Weights and Multiple Imputations. \emph{Journal of Business \& Economic Statistics}. 1986;4(1):87. doi:\href{https://doi.org/10.2307/1391390}{10.2307/1391390}}

\bibitem[\citeproctext]{ref-little1988a}
\CSLLeftMargin{161. }%
\CSLRightInline{Little RJA. Missing-Data Adjustments in Large Surveys. \emph{Journal of Business \& Economic Statistics}. 1988;6(3):287--296. doi:\href{https://doi.org/10.1080/07350015.1988.10509663}{10.1080/07350015.1988.10509663}}

\bibitem[\citeproctext]{ref-miceadds}
\CSLLeftMargin{162. }%
\CSLRightInline{Robitzsch A, Grund S. \emph{miceadds: Some Additional Multiple Imputation Functions, Especially for {\textbraceleft}mice{\textbraceright}}.; 2023. \href{https://CRAN.R-project.org/package=miceadds}{https://CRAN.R-project.org/package=miceadds.}}

\bibitem[\citeproctext]{ref-ids}
\CSLLeftMargin{163. }%
\CSLRightInline{FitzJohn R. \emph{ids: Generate Random Identifiers}.; 2017. \href{https://CRAN.R-project.org/package=ids}{https://CRAN.R-project.org/package=ids.}}

\bibitem[\citeproctext]{ref-hash}
\CSLLeftMargin{164. }%
\CSLRightInline{Brown C. \emph{hash: Full Featured Implementation of Hash Tables/Associative Arrays/Dictionaries}.; 2023. \href{https://CRAN.R-project.org/package=hash}{https://CRAN.R-project.org/package=hash.}}

\bibitem[\citeproctext]{ref-anonymizer}
\CSLLeftMargin{165. }%
\CSLRightInline{Hendricks P. \emph{anonymizer: Anonymize Data Containing Personally Identifiable Information}.; 2023. \href{https://github.com/paulhendricks/anonymizer}{https://github.com/paulhendricks/anonymizer.}}

\bibitem[\citeproctext]{ref-digest}
\CSLLeftMargin{166. }%
\CSLRightInline{Lucas DE with contributions by A, Tuszynski J, Bengtsson H, et al. \emph{digest: Create Compact Hash Digests of R Objects}.; 2023. \href{https://CRAN.R-project.org/package=digest}{https://CRAN.R-project.org/package=digest.}}

\bibitem[\citeproctext]{ref-synthpop}
\CSLLeftMargin{167. }%
\CSLRightInline{Nowok B, Raab GM, Dibben C. {\textbraceleft}synthpop{\textbraceright}: Bespoke Creation of Synthetic Data in {\textbraceleft}R{\textbraceright}. \emph{Journal of Statistical Software}. 2016;74. doi:\href{https://doi.org/10.18637/jss.v074.i11}{10.18637/jss.v074.i11}}

\bibitem[\citeproctext]{ref-s2011}
\CSLLeftMargin{168. }%
\CSLRightInline{S M. Frequency distribution. \emph{Journal of Pharmacology and Pharmacotherapeutics}. 2011;2(1):54--56. doi:\href{https://doi.org/10.4103/0976-500x.77120}{10.4103/0976-500x.77120}}

\bibitem[\citeproctext]{ref-sturges1926}
\CSLLeftMargin{169. }%
\CSLRightInline{Sturges HA. The Choice of a Class Interval. \emph{Journal of the American Statistical Association}. 1926;21(153):65--66. doi:\href{https://doi.org/10.1080/01621459.1926.10502161}{10.1080/01621459.1926.10502161}}

\bibitem[\citeproctext]{ref-scott1979}
\CSLLeftMargin{170. }%
\CSLRightInline{SCOTT DW. On optimal and data-based histograms. \emph{Biometrika}. 1979;66(3):605--610. doi:\href{https://doi.org/10.1093/biomet/66.3.605}{10.1093/biomet/66.3.605}}

\bibitem[\citeproctext]{ref-freedman1981}
\CSLLeftMargin{171. }%
\CSLRightInline{Freedman D, Diaconis P. On the histogram as a density estimator:L 2 theory. \emph{Zeitschrift für Wahrscheinlichkeitstheorie und Verwandte Gebiete}. 1981;57(4):453--476. doi:\href{https://doi.org/10.1007/bf01025868}{10.1007/bf01025868}}

\bibitem[\citeproctext]{ref-grDevices}
\CSLLeftMargin{172. }%
\CSLRightInline{R Core Team. \emph{R: A Language and Environment for Statistical Computing}.; 2023. \href{https://www.R-project.org/}{https://www.R-project.org/.}}

\bibitem[\citeproctext]{ref-ggplot2}
\CSLLeftMargin{173. }%
\CSLRightInline{Wickham H. \emph{ggplot2: Elegant Graphics for Data Analysis}. Springer; 2016. \href{https://ggplot2.tidyverse.org}{https://ggplot2.tidyverse.org.}}

\bibitem[\citeproctext]{ref-ggdist}
\CSLLeftMargin{174. }%
\CSLRightInline{Kay M. {\textbraceleft}ggdist{\textbraceright}: Visualizations of Distributions and Uncertainty in the Grammar of Graphics. \emph{IEEE Transactions on Visualization and Computer Graphics}. 2024;30(1):414--424. doi:\href{https://doi.org/10.1109/TVCG.2023.3327195}{10.1109/TVCG.2023.3327195}}

\bibitem[\citeproctext]{ref-ggfortify}
\CSLLeftMargin{175. }%
\CSLRightInline{Tang Y, Horikoshi M, Li W. \emph{ggfortify: Unified Interface to Visualize Statistical Result of Popular R Packages}. Vol 8.; 2016. doi:\href{https://doi.org/10.32614/RJ-2016-060}{10.32614/RJ-2016-060}}

\bibitem[\citeproctext]{ref-rochon2012}
\CSLLeftMargin{176. }%
\CSLRightInline{Rochon J, Gondan M, Kieser M. To test or not to test: Preliminary assessment of normality when comparing two independent samples. \emph{BMC Medical Research Methodology}. 2012;12(1). doi:\href{https://doi.org/10.1186/1471-2288-12-81}{10.1186/1471-2288-12-81}}

\bibitem[\citeproctext]{ref-greenhalgh1997}
\CSLLeftMargin{177. }%
\CSLRightInline{Greenhalgh T. How to read a paper: Statistics for the non-statistician. I: Different types of data need different statistical tests. \emph{BMJ}. 1997;315(7104):364--366. doi:\href{https://doi.org/10.1136/bmj.315.7104.364}{10.1136/bmj.315.7104.364}}

\bibitem[\citeproctext]{ref-schmider2010}
\CSLLeftMargin{178. }%
\CSLRightInline{Schmider E, Ziegler M, Danay E, Beyer L, Bühner M. Is It Really Robust? \emph{Methodology}. 2010;6(4):147--151. doi:\href{https://doi.org/10.1027/1614-2241/a000016}{10.1027/1614-2241/a000016}}

\bibitem[\citeproctext]{ref-kanji2006}
\CSLLeftMargin{179. }%
\CSLRightInline{Kanji G. \emph{100 Statistical Tests}. SAGE Publications Ltd; 2006. doi:\href{https://doi.org/10.4135/9781849208499}{10.4135/9781849208499}}

\bibitem[\citeproctext]{ref-Curran-Everett2008}
\CSLLeftMargin{180. }%
\CSLRightInline{Curran-Everett D. Explorations in statistics: standard deviations and standard errors. \emph{Advances in Physiology Education}. 2008;32(3):203--208. doi:\href{https://doi.org/10.1152/advan.90123.2008}{10.1152/advan.90123.2008}}

\bibitem[\citeproctext]{ref-Altman1994}
\CSLLeftMargin{181. }%
\CSLRightInline{Altman DG, Bland JM. Statistics Notes: Quartiles, quintiles, centiles, and other quantiles. \emph{BMJ}. 1994;309(6960):996--996. doi:\href{https://doi.org/10.1136/bmj.309.6960.996}{10.1136/bmj.309.6960.996}}

\bibitem[\citeproctext]{ref-krzywinski2013}
\CSLLeftMargin{182. }%
\CSLRightInline{Krzywinski M, Altman N. Error bars. \emph{Nature Methods}. 2013;10(10):921--922. doi:\href{https://doi.org/10.1038/nmeth.2659}{10.1038/nmeth.2659}}

\bibitem[\citeproctext]{ref-Cumming2007}
\CSLLeftMargin{183. }%
\CSLRightInline{Cumming G, Fidler F, Vaux DL. Error bars in experimental biology. \emph{The Journal of Cell Biology}. 2007;177(1):7--11. doi:\href{https://doi.org/10.1083/jcb.200611141}{10.1083/jcb.200611141}}

\bibitem[\citeproctext]{ref-s.2011a}
\CSLLeftMargin{184. }%
\CSLRightInline{S. M. Measures of central tendency: The mean. \emph{Journal of Pharmacology and Pharmacotherapeutics}. 2011;2(2):140--142. doi:\href{https://doi.org/10.4103/0976-500x.81920}{10.4103/0976-500x.81920}}

\bibitem[\citeproctext]{ref-s.2011}
\CSLLeftMargin{185. }%
\CSLRightInline{S. M. Measures of central tendency: Median and mode. \emph{Journal of Pharmacology and Pharmacotherapeutics}. 2011;2(3):214--215. doi:\href{https://doi.org/10.4103/0976-500x.83300}{10.4103/0976-500x.83300}}

\bibitem[\citeproctext]{ref-manikandan2011}
\CSLLeftMargin{186. }%
\CSLRightInline{Manikandan S. Measures of dispersion. \emph{Journal of Pharmacology and Pharmacotherapeutics}. 2011;2(4):315--316. doi:\href{https://doi.org/10.4103/0976-500x.85931}{10.4103/0976-500x.85931}}

\bibitem[\citeproctext]{ref-sahai1992}
\CSLLeftMargin{187. }%
\CSLRightInline{Sahai H, Misra S. Definitions of Sample Variance: Some Teaching Problems to be Overcome. \emph{The Statistician}. 1992;41(1):55. doi:\href{https://doi.org/10.2307/2348636}{10.2307/2348636}}

\bibitem[\citeproctext]{ref-leys2019}
\CSLLeftMargin{188. }%
\CSLRightInline{Leys C, Delacre M, Mora YL, Lakens D, Ley C. How to Classify, Detect, and Manage Univariate and Multivariate Outliers, With Emphasis on Pre-Registration. \emph{International Review of Social Psychology}. 2019;32(1). doi:\href{https://doi.org/10.5334/irsp.289}{10.5334/irsp.289}}

\bibitem[\citeproctext]{ref-rousseeuw2011}
\CSLLeftMargin{189. }%
\CSLRightInline{Rousseeuw PJ, Hubert M. Robust statistics for outlier detection. \emph{WIREs Data Mining and Knowledge Discovery}. 2011;1(1):73--79. doi:\href{https://doi.org/10.1002/widm.2}{10.1002/widm.2}}

\bibitem[\citeproctext]{ref-daszykowski2007}
\CSLLeftMargin{190. }%
\CSLRightInline{Daszykowski M, Kaczmarek K, Vander Heyden Y, Walczak B. Robust statistics in data analysis {\textemdash} A review. \emph{Chemometrics and Intelligent Laboratory Systems}. 2007;85(2):203--219. doi:\href{https://doi.org/10.1016/j.chemolab.2006.06.016}{10.1016/j.chemolab.2006.06.016}}

\bibitem[\citeproctext]{ref-chatfield1986}
\CSLLeftMargin{191. }%
\CSLRightInline{Chatfield C. Exploratory data analysis. \emph{European Journal of Operational Research}. 1986;23(1):5--13. doi:\href{https://doi.org/10.1016/0377-2217(86)90209-2}{10.1016/0377-2217(86)90209-2}}

\bibitem[\citeproctext]{ref-Ferketich1986}
\CSLLeftMargin{192. }%
\CSLRightInline{Ferketich S, Verran J. Technical Notes. \emph{Western Journal of Nursing Research}. 1986;8(4):464--466. doi:\href{https://doi.org/10.1177/019394598600800409}{10.1177/019394598600800409}}

\bibitem[\citeproctext]{ref-Landis2012}
\CSLLeftMargin{193. }%
\CSLRightInline{Landis SC, Amara SG, Asadullah K, et al. A call for transparent reporting to optimize the predictive value of preclinical research. \emph{Nature}. 2012;490(7419):187--191. doi:\href{https://doi.org/10.1038/nature11556}{10.1038/nature11556}}

\bibitem[\citeproctext]{ref-huebner2016}
\CSLLeftMargin{194. }%
\CSLRightInline{Huebner M, Vach W, Cessie S le. A systematic approach to initial data analysis is good research practice. \emph{The Journal of Thoracic and Cardiovascular Surgery}. 2016;151(1):25--27. doi:\href{https://doi.org/10.1016/j.jtcvs.2015.09.085}{10.1016/j.jtcvs.2015.09.085}}

\bibitem[\citeproctext]{ref-zuur2009}
\CSLLeftMargin{195. }%
\CSLRightInline{Zuur AF, Ieno EN, Elphick CS. A protocol for data exploration to avoid common statistical problems. \emph{Methods in Ecology and Evolution}. 2009;1(1):3--14. doi:\href{https://doi.org/10.1111/j.2041-210x.2009.00001.x}{10.1111/j.2041-210x.2009.00001.x}}

\bibitem[\citeproctext]{ref-explore}
\CSLLeftMargin{196. }%
\CSLRightInline{Krasser R. \emph{explore: Simplifies Exploratory Data Analysis}.; 2023. \href{https://CRAN.R-project.org/package=explore}{https://CRAN.R-project.org/package=explore.}}

\bibitem[\citeproctext]{ref-dataMaid}
\CSLLeftMargin{197. }%
\CSLRightInline{Petersen AH, Ekstrøm CT. {\textbraceleft}dataMaid{\textbraceright}: Your Assistant for Documenting Supervised Data Quality Screening in {\textbraceleft}R{\textbraceright}. \emph{Journal of Statistical Software}. 2019;90. doi:\href{https://doi.org/10.18637/jss.v090.i06}{10.18637/jss.v090.i06}}

\bibitem[\citeproctext]{ref-DataExplorer}
\CSLLeftMargin{198. }%
\CSLRightInline{Cui B. \emph{DataExplorer: Automate Data Exploration and Treatment}.; 2020. \href{https://CRAN.R-project.org/package=DataExplorer}{https://CRAN.R-project.org/package=DataExplorer.}}

\bibitem[\citeproctext]{ref-SmartEDA}
\CSLLeftMargin{199. }%
\CSLRightInline{Dayanand Ubrangala, R K, Prasad Kondapalli R, Putatunda S. \emph{SmartEDA: Summarize and Explore the Data}.; 2022. \href{https://CRAN.R-project.org/package=SmartEDA}{https://CRAN.R-project.org/package=SmartEDA.}}

\bibitem[\citeproctext]{ref-gtExtras}
\CSLLeftMargin{200. }%
\CSLRightInline{Mock T. \emph{gtExtras: Extending {\textbraceleft}gt{\textbraceright} for Beautiful HTML Tables}.; 2023. \href{https://CRAN.R-project.org/package=gtExtras}{https://CRAN.R-project.org/package=gtExtras.}}

\bibitem[\citeproctext]{ref-radiant}
\CSLLeftMargin{201. }%
\CSLRightInline{Nijs V. \emph{radiant: Business Analytics using R and Shiny}.; 2023. \href{https://CRAN.R-project.org/package=radiant}{https://CRAN.R-project.org/package=radiant.}}

\bibitem[\citeproctext]{ref-behrens1997}
\CSLLeftMargin{202. }%
\CSLRightInline{Behrens JT. Principles and procedures of exploratory data analysis. \emph{Psychological Methods}. 1997;2(2):131--160. doi:\href{https://doi.org/10.1037/1082-989x.2.2.131}{10.1037/1082-989x.2.2.131}}

\bibitem[\citeproctext]{ref-ggcleveland}
\CSLLeftMargin{203. }%
\CSLRightInline{Prunello M, Mari G. \emph{ggcleveland: Implementation of Plots from Cleveland's Visualizing Data Book}.; 2021. doi:\href{https://doi.org/10.32614/CRAN.package.ggcleveland}{10.32614/CRAN.package.ggcleveland}}

\bibitem[\citeproctext]{ref-gerring2012}
\CSLLeftMargin{204. }%
\CSLRightInline{Gerring J. Mere Description. \emph{British Journal of Political Science}. 2012;42(4):721--746. doi:\href{https://doi.org/10.1017/s0007123412000130}{10.1017/s0007123412000130}}

\bibitem[\citeproctext]{ref-Cummings2003}
\CSLLeftMargin{205. }%
\CSLRightInline{Cummings P, Rivara FP. Reporting Statistical Information in Medical Journal Articles. \emph{Archives of Pediatrics \& Adolescent Medicine}. 2003;157(4):321. doi:\href{https://doi.org/10.1001/archpedi.157.4.321}{10.1001/archpedi.157.4.321}}

\bibitem[\citeproctext]{ref-Cole2015a}
\CSLLeftMargin{206. }%
\CSLRightInline{Cole TJ. Setting number of decimal places for reporting risk ratios: rule of four. \emph{BMJ}. 2015;350(apr27 3):h1845--h1845. doi:\href{https://doi.org/10.1136/bmj.h1845}{10.1136/bmj.h1845}}

\bibitem[\citeproctext]{ref-cole2015b}
\CSLLeftMargin{207. }%
\CSLRightInline{Cole TJ. Too many digits: the presentation of numerical data. \emph{Archives of Disease in Childhood}. 2015;100(7):608--609. doi:\href{https://doi.org/10.1136/archdischild-2014-307149}{10.1136/archdischild-2014-307149}}

\bibitem[\citeproctext]{ref-Weissgerber2019}
\CSLLeftMargin{208. }%
\CSLRightInline{Weissgerber TL, Winham SJ, Heinzen EP, et al. Reveal, Don{'}t Conceal. \emph{Circulation}. 2019;140(18):1506--1518. doi:\href{https://doi.org/10.1161/circulationaha.118.037777}{10.1161/circulationaha.118.037777}}

\bibitem[\citeproctext]{ref-Inskip2017}
\CSLLeftMargin{209. }%
\CSLRightInline{Inskip H, Ntani G, Westbury L, et al. Getting started with tables. \emph{Archives of Public Health}. 2017;75(1). doi:\href{https://doi.org/10.1186/s13690-017-0180-1}{10.1186/s13690-017-0180-1}}

\bibitem[\citeproctext]{ref-Kwak2021}
\CSLLeftMargin{210. }%
\CSLRightInline{Kwak SG, Kang H, Kim JH, et al. The principles of presenting statistical results: Table. \emph{Korean Journal of Anesthesiology}. 2021;74(2):115--119. doi:\href{https://doi.org/10.4097/kja.20582}{10.4097/kja.20582}}

\bibitem[\citeproctext]{ref-gtsummary}
\CSLLeftMargin{211. }%
\CSLRightInline{Sjoberg DD, Whiting K, Curry M, Lavery JA, Larmarange J. Reproducible Summary Tables with the gtsummary Package. \emph{The R Journal}. 2021;13:570--580. doi:\href{https://doi.org/10.32614/RJ-2021-053}{10.32614/RJ-2021-053}}

\bibitem[\citeproctext]{ref-table1}
\CSLLeftMargin{212. }%
\CSLRightInline{Rich B. \emph{table1: Tables of Descriptive Statistics in HTML}.; 2023. \href{https://CRAN.R-project.org/package=table1}{https://CRAN.R-project.org/package=table1.}}

\bibitem[\citeproctext]{ref-flextable}
\CSLLeftMargin{213. }%
\CSLRightInline{Gohel D, Skintzos P. \emph{flextable: Functions for Tabular Reporting}.; 2023. \href{https://CRAN.R-project.org/package=flextable}{https://CRAN.R-project.org/package=flextable.}}

\bibitem[\citeproctext]{ref-rempsyc}
\CSLLeftMargin{214. }%
\CSLRightInline{Thériault R. {\textbraceleft}rempsyc{\textbraceright}: Convenience functions for psychology. \emph{Journal of Open Source Software}. 2023;8:5466. doi:\href{https://doi.org/10.21105/joss.05466}{10.21105/joss.05466}}

\bibitem[\citeproctext]{ref-barnett2023}
\CSLLeftMargin{215. }%
\CSLRightInline{Barnett A. Automated detection of over- and under-dispersion in baseline tables in randomised controlled trials. \emph{F1000Research}. 2023;11:783. doi:\href{https://doi.org/10.12688/f1000research.123002.2}{10.12688/f1000research.123002.2}}

\bibitem[\citeproctext]{ref-Westreich2013}
\CSLLeftMargin{216. }%
\CSLRightInline{Westreich D, Greenland S. The Table~2 Fallacy: Presenting and Interpreting Confounder and Modifier Coefficients. \emph{American Journal of Epidemiology}. 2013;177(4):292--298. doi:\href{https://doi.org/10.1093/aje/kws412}{10.1093/aje/kws412}}

\bibitem[\citeproctext]{ref-chen2020}
\CSLLeftMargin{217. }%
\CSLRightInline{Chen H, Lu Y, Slye N. Testing for baseline differences in clinical trials. \emph{International Journal of Clinical Trials}. 2020;7(2):150. doi:\href{https://doi.org/10.18203/2349-3259.ijct20201720}{10.18203/2349-3259.ijct20201720}}

\bibitem[\citeproctext]{ref-pijls2022}
\CSLLeftMargin{218. }%
\CSLRightInline{Pijls BG. The Table I Fallacy: P Values in Baseline Tables of Randomized Controlled Trials. \emph{Journal of Bone and Joint Surgery}. 2022;104(16):e71. doi:\href{https://doi.org/10.2106/jbjs.21.01166}{10.2106/jbjs.21.01166}}

\bibitem[\citeproctext]{ref-Hayes-Larson2019}
\CSLLeftMargin{219. }%
\CSLRightInline{Hayes-Larson E, Kezios KL, Mooney SJ, Lovasi G. Who is in this study, anyway? Guidelines for a useful Table~1. \emph{Journal of Clinical Epidemiology}. 2019;114:125--132. doi:\href{https://doi.org/10.1016/j.jclinepi.2019.06.011}{10.1016/j.jclinepi.2019.06.011}}

\bibitem[\citeproctext]{ref-bandoli2018}
\CSLLeftMargin{220. }%
\CSLRightInline{Bandoli G, Palmsten K, Chambers CD, Jelliffe-Pawlowski LL, Baer RJ, Thompson CA. Revisiting the Table~2 fallacy: A motivating example examining preeclampsia and preterm birth. \emph{Paediatric and Perinatal Epidemiology}. 2018;32(4):390--397. doi:\href{https://doi.org/10.1111/ppe.12474}{10.1111/ppe.12474}}

\bibitem[\citeproctext]{ref-midway2020}
\CSLLeftMargin{221. }%
\CSLRightInline{Midway SR. Principles of Effective Data Visualization. \emph{Patterns}. 2020;1(9):100141. doi:\href{https://doi.org/10.1016/j.patter.2020.100141}{10.1016/j.patter.2020.100141}}

\bibitem[\citeproctext]{ref-Park2022}
\CSLLeftMargin{222. }%
\CSLRightInline{Park JH, Lee DK, Kang H, et al. The principles of presenting statistical results using figures. \emph{Korean Journal of Anesthesiology}. 2022;75(2):139--150. doi:\href{https://doi.org/10.4097/kja.21508}{10.4097/kja.21508}}

\bibitem[\citeproctext]{ref-vandemeulebroecke2018}
\CSLLeftMargin{223. }%
\CSLRightInline{Vandemeulebroecke M, Baillie M, Carr D, et al. How can we make better graphs? An initiative to increase the graphical expertise and productivity of quantitative scientists. \emph{Pharmaceutical Statistics}. 2018;18(1):106--114. doi:\href{https://doi.org/10.1002/pst.1912}{10.1002/pst.1912}}

\bibitem[\citeproctext]{ref-plotly}
\CSLLeftMargin{224. }%
\CSLRightInline{Sievert C. \emph{Interactive Web-Based Data Visualization with R, plotly, and shiny}. Chapman; Hall/CRC; 2020. \href{https://plotly-r.com}{https://plotly-r.com.}}

\bibitem[\citeproctext]{ref-corrplot}
\CSLLeftMargin{225. }%
\CSLRightInline{Wei T, Simko V. \emph{R package corrplot: Visualization of a Correlation Matrix}.; 2024. \href{https://github.com/taiyun/corrplot}{https://github.com/taiyun/corrplot.}}

\bibitem[\citeproctext]{ref-ggsci}
\CSLLeftMargin{226. }%
\CSLRightInline{Xiao N. \emph{ggsci: Scientific Journal and Sci-Fi Themed Color Palettes for {\textbraceleft}ggplot2{\textbraceright}}.; 2023. \href{https://CRAN.R-project.org/package=ggsci}{https://CRAN.R-project.org/package=ggsci.}}

\bibitem[\citeproctext]{ref-tiff}
\CSLLeftMargin{227. }%
\CSLRightInline{Urbanek S, Johnson K. \emph{tiff: Read and Write TIFF Images}.; 2022. \href{https://CRAN.R-project.org/package=tiff}{https://CRAN.R-project.org/package=tiff.}}

\bibitem[\citeproctext]{ref-wiebels2023}
\CSLLeftMargin{228. }%
\CSLRightInline{Wiebels K, Moreau D. Dynamic Data Visualizations to Enhance Insight and Communication Across the Life Cycle of a Scientific Project. \emph{Advances in Methods and Practices in Psychological Science}. 2023;6(3). doi:\href{https://doi.org/10.1177/25152459231160103}{10.1177/25152459231160103}}

\bibitem[\citeproctext]{ref-gganimate}
\CSLLeftMargin{229. }%
\CSLRightInline{Pedersen TL, Robinson D. \emph{gganimate: A Grammar of Animated Graphics}.; 2025. doi:\href{https://doi.org/10.32614/CRAN.package.gganimate}{10.32614/CRAN.package.gganimate}}

\bibitem[\citeproctext]{ref-WRS2}
\CSLLeftMargin{230. }%
\CSLRightInline{Mair P, Wilcox R. Robust Statistical Methods in R Using the WRS2 Package. \emph{Behavior Research Methods}. 2020;52:464--488. doi:\href{https://doi.org/10.3758/s13428-019-01246-w}{10.3758/s13428-019-01246-w}}

\bibitem[\citeproctext]{ref-leys2013}
\CSLLeftMargin{231. }%
\CSLRightInline{Leys C, Ley C, Klein O, Bernard P, Licata L. Detecting outliers: Do not use standard deviation around the mean, use absolute deviation around the median. \emph{Journal of Experimental Social Psychology}. 2013;49(4):764--766. doi:\href{https://doi.org/10.1016/j.jesp.2013.03.013}{10.1016/j.jesp.2013.03.013}}

\bibitem[\citeproctext]{ref-leys2018}
\CSLLeftMargin{232. }%
\CSLRightInline{Leys C, Klein O, Dominicy Y, Ley C. Detecting multivariate outliers: Use a robust variant of the Mahalanobis distance. \emph{Journal of Experimental Social Psychology}. 2018;74:150--156. doi:\href{https://doi.org/10.1016/j.jesp.2017.09.011}{10.1016/j.jesp.2017.09.011}}

\bibitem[\citeproctext]{ref-Tukey1963}
\CSLLeftMargin{233. }%
\CSLRightInline{Tukey JW, McLaughlin DH. Less Vulnerable Confidence and Significance Procedures for Location Based on a Single Sample: Trimming/Winsorization 1. \emph{Sankhyā: The Indian Journal of Statistics, Series A (1961-2002)}. 1963;25(3):331--352. \url{http://www.jstor.org/stable/25049278}. Acessado abril 11, 2025.}

\bibitem[\citeproctext]{ref-outliers}
\CSLLeftMargin{234. }%
\CSLRightInline{Komsta L. \emph{outliers: Tests for Outliers}.; 2022. \href{https://CRAN.R-project.org/package=outliers}{https://CRAN.R-project.org/package=outliers.}}

\bibitem[\citeproctext]{ref-loh2025}
\CSLLeftMargin{235. }%
\CSLRightInline{Loh PL. A Theoretical Review of Modern Robust Statistics. \emph{Annual Review of Statistics and Its Application}. 2025;12(1):477--496. doi:\href{https://doi.org/10.1146/annurev-statistics-112723-034446}{10.1146/annurev-statistics-112723-034446}}

\bibitem[\citeproctext]{ref-WRS2-2}
\CSLLeftMargin{236. }%
\CSLRightInline{Mair P, Wilcox R, Indrajeet P. \emph{A Collection of Robust Statistical Methods}.; 2025. \href{https://CRAN.R-project.org/package=WRS2}{https://CRAN.R-project.org/package=WRS2.}}

\bibitem[\citeproctext]{ref-ggeffects}
\CSLLeftMargin{237. }%
\CSLRightInline{Lüdecke D. ggeffects: Tidy Data Frames of Marginal Effects from Regression Models. \emph{Journal of Open Source Software}. 2018;3:772. doi:\href{https://doi.org/10.21105/joss.00772}{10.21105/joss.00772}}

\bibitem[\citeproctext]{ref-Song2015}
\CSLLeftMargin{238. }%
\CSLRightInline{Song YY, Lu Y. Decision tree methods: applications for classification and prediction. \emph{Shanghai archives of psychiatry}. 2015;27(2):130--135. doi:\href{https://doi.org/10.11919/j.issn.1002-0829.215044}{10.11919/j.issn.1002-0829.215044}}

\bibitem[\citeproctext]{ref-hozo2023}
\CSLLeftMargin{239. }%
\CSLRightInline{Hozo I, Guyatt G, Djulbegovic B. Decision curve analysis based on summary data. \emph{Journal of Evaluation in Clinical Practice}. 2023;30(2):281--289. doi:\href{https://doi.org/10.1111/jep.13945}{10.1111/jep.13945}}

\bibitem[\citeproctext]{ref-vickers2019}
\CSLLeftMargin{240. }%
\CSLRightInline{Vickers AJ, Calster B van, Steyerberg EW. A simple, step-by-step guide to interpreting decision curve analysis. \emph{Diagnostic and Prognostic Research}. 2019;3(1). doi:\href{https://doi.org/10.1186/s41512-019-0064-7}{10.1186/s41512-019-0064-7}}

\bibitem[\citeproctext]{ref-aalen2007}
\CSLLeftMargin{241. }%
\CSLRightInline{AALEN OO, FRIGESSI A. What can Statistics Contribute to a Causal Understanding? \emph{Scandinavian Journal of Statistics}. 2007;34(1):155--168. doi:\href{https://doi.org/10.1111/j.1467-9469.2006.00549.x}{10.1111/j.1467-9469.2006.00549.x}}

\bibitem[\citeproctext]{ref-matute2015}
\CSLLeftMargin{242. }%
\CSLRightInline{Matute H, Blanco F, Yarritu I, Díaz-Lago M, Vadillo MA, Barberia I. Illusions of causality: how they bias our everyday thinking and how they could be reduced. \emph{Frontiers in Psychology}. 2015;6. doi:\href{https://doi.org/10.3389/fpsyg.2015.00888}{10.3389/fpsyg.2015.00888}}

\bibitem[\citeproctext]{ref-vickers2023}
\CSLLeftMargin{243. }%
\CSLRightInline{Vickers AJ, Assel M, Dunn RL, et al. Guidelines for Reporting Observational Research in Urology: The Importance of Clear Reference to Causality. \emph{European Urology}. 2023;84(2):147--151. doi:\href{https://doi.org/10.1016/j.eururo.2023.04.027}{10.1016/j.eururo.2023.04.027}}

\bibitem[\citeproctext]{ref-hill1965}
\CSLLeftMargin{244. }%
\CSLRightInline{Hill AB. The Environment and Disease: Association or Causation? \emph{Proceedings of the Royal Society of Medicine}. 1965;58(5):295--300. doi:\href{https://doi.org/10.1177/003591576505800503}{10.1177/003591576505800503}}

\bibitem[\citeproctext]{ref-rothman2005}
\CSLLeftMargin{245. }%
\CSLRightInline{Rothman KJ, Greenland S. H ill's Criteria for Causality. \emph{Encyclopedia of Biostatistics}. fevereiro 2005. doi:\href{https://doi.org/10.1002/0470011815.b2a03072}{10.1002/0470011815.b2a03072}}

\bibitem[\citeproctext]{ref-shimonovich2020}
\CSLLeftMargin{246. }%
\CSLRightInline{Shimonovich M, Pearce A, Thomson H, Keyes K, Katikireddi SV. Assessing causality in epidemiology: revisiting Bradford Hill to incorporate developments in causal thinking. \emph{European Journal of Epidemiology}. 2020;36(9):873--887. doi:\href{https://doi.org/10.1007/s10654-020-00703-7}{10.1007/s10654-020-00703-7}}

\bibitem[\citeproctext]{ref-dagitty}
\CSLLeftMargin{247. }%
\CSLRightInline{Textor J, Zander B van der, Gilthorpe MS, Liskiewicz M, Ellison GT. Robust causal inference using directed acyclic graphs: the R package {\textbraceleft}dagitty{\textbraceright}. \emph{International Journal of Epidemiology}. 2016;45:1887--1894. doi:\href{https://doi.org/10.1093/ije/dyw341}{10.1093/ije/dyw341}}

\bibitem[\citeproctext]{ref-ggdag}
\CSLLeftMargin{248. }%
\CSLRightInline{Barrett M. \emph{ggdag: Analyze and Create Elegant Directed Acyclic Graphs}.; 2024. \href{https://CRAN.R-project.org/package=ggdag}{https://CRAN.R-project.org/package=ggdag.}}

\bibitem[\citeproctext]{ref-performance}
\CSLLeftMargin{249. }%
\CSLRightInline{Lüdecke D, Ben-Shachar MS, Patil I, Waggoner P, Makowski D. {\textbraceleft}performance{\textbraceright}: An {\textbraceleft}R{\textbraceright} Package for Assessment, Comparison and Testing of Statistical Models. \emph{Journal of Open Source Software}. 2021;6:3139. doi:\href{https://doi.org/10.21105/joss.03139}{10.21105/joss.03139}}

\bibitem[\citeproctext]{ref-tidytext}
\CSLLeftMargin{250. }%
\CSLRightInline{Silge J, Robinson D. tidytext: Text Mining and Analysis Using Tidy Data Principles in R. \emph{The Journal of Open Source Software}. 2016;1. doi:\href{https://doi.org/10.21105/joss.00037}{10.21105/joss.00037}}

\bibitem[\citeproctext]{ref-Greenland1989}
\CSLLeftMargin{251. }%
\CSLRightInline{Greenland S. Modeling and variable selection in epidemiologic analysis. \emph{American Journal of Public Health}. 1989;79(3):340--349. doi:\href{https://doi.org/10.2105/ajph.79.3.340}{10.2105/ajph.79.3.340}}

\bibitem[\citeproctext]{ref-Breznau2022}
\CSLLeftMargin{252. }%
\CSLRightInline{Breznau N, Rinke EM, Wuttke A, et al. Observing many researchers using the same data and hypothesis reveals a hidden universe of uncertainty. \emph{Proceedings of the National Academy of Sciences}. 2022;(44):e2203150119. doi:\href{https://doi.org/10.1073/pnas.2203150119}{10.1073/pnas.2203150119}}

\bibitem[\citeproctext]{ref-dwivedi2019}
\CSLLeftMargin{253. }%
\CSLRightInline{Dwivedi AK, Shukla R. Evidence{-}based statistical analysis and methods in biomedical research (SAMBR) checklists according to design features. \emph{CANCER REPORTS}. 2019;3(4). doi:\href{https://doi.org/10.1002/cnr2.1211}{10.1002/cnr2.1211}}

\bibitem[\citeproctext]{ref-Dwivedi2022}
\CSLLeftMargin{254. }%
\CSLRightInline{Dwivedi AK. How to Write Statistical Analysis Section in Medical Research. \emph{Journal of Investigative Medicine}. 2022;70(8):1759--1770. doi:\href{https://doi.org/10.1136/jim-2022-002479}{10.1136/jim-2022-002479}}

\bibitem[\citeproctext]{ref-Kim2017}
\CSLLeftMargin{255. }%
\CSLRightInline{Kim N, Fischer AH, Dyring-Andersen B, Rosner B, Okoye GA. Research Techniques Made Simple: Choosing Appropriate Statistical Methods for Clinical Research. \emph{Journal of Investigative Dermatology}. 2017;137(10):e173--e178. doi:\href{https://doi.org/10.1016/j.jid.2017.08.007}{10.1016/j.jid.2017.08.007}}

\bibitem[\citeproctext]{ref-marusteri2010}
\CSLLeftMargin{256. }%
\CSLRightInline{Marusteri M, Bacarea V. Comparing groups for statistical differences: how to choose the right statistical test? \emph{Biochemia Medica}. 2010:15--32. doi:\href{https://doi.org/10.11613/bm.2010.004}{10.11613/bm.2010.004}}

\bibitem[\citeproctext]{ref-mishra2019}
\CSLLeftMargin{257. }%
\CSLRightInline{Mishra P, Pandey C, Singh U, Keshri A, Sabaretnam M. Selection of appropriate statistical methods for data analysis. \emph{Annals of Cardiac Anaesthesia}. 2019;22(3):297. doi:\href{https://doi.org/10.4103/aca.aca_248_18}{10.4103/aca.aca\_248\_18}}

\bibitem[\citeproctext]{ref-ray2021}
\CSLLeftMargin{258. }%
\CSLRightInline{Ray A, Najmi A, Sadasivam B. How to choose and interpret a statistical test? An update for budding researchers. \emph{Journal of Family Medicine and Primary Care}. 2021;10(8):2763. doi:\href{https://doi.org/10.4103/jfmpc.jfmpc_433_21}{10.4103/jfmpc.jfmpc\_433\_21}}

\bibitem[\citeproctext]{ref-nayak2011}
\CSLLeftMargin{259. }%
\CSLRightInline{Nayak B, Hazra A. How to choose the right statistical test? \emph{Indian Journal of Ophthalmology}. 2011;59(2):85. doi:\href{https://doi.org/10.4103/0301-4738.77005}{10.4103/0301-4738.77005}}

\bibitem[\citeproctext]{ref-shankar2014}
\CSLLeftMargin{260. }%
\CSLRightInline{Shankar S, Singh R. Demystifying statistics: How to choose a statistical test? \emph{Indian Journal of Rheumatology}. 2014;9(2):77--81. doi:\href{https://doi.org/10.1016/j.injr.2014.04.002}{10.1016/j.injr.2014.04.002}}

\bibitem[\citeproctext]{ref-Curran-Everett2009}
\CSLLeftMargin{261. }%
\CSLRightInline{Curran-Everett D. Explorations in statistics: hypothesis tests and {\emph{P}} values. \emph{Advances in Physiology Education}. 2009;33(2):81--86. doi:\href{https://doi.org/10.1152/advan.90218.2008}{10.1152/advan.90218.2008}}

\bibitem[\citeproctext]{ref-goodman1999}
\CSLLeftMargin{262. }%
\CSLRightInline{Goodman SN. Toward Evidence-Based Medical Statistics. 1: The P Value Fallacy. \emph{Annals of Internal Medicine}. 1999;130(12):995. doi:\href{https://doi.org/10.7326/0003-4819-130-12-199906150-00008}{10.7326/0003-4819-130-12-199906150-00008}}

\bibitem[\citeproctext]{ref-mccaskey2015}
\CSLLeftMargin{263. }%
\CSLRightInline{McCaskey K, Rainey C. Substantive Importance and the Veil of Statistical Significance. \emph{Statistics, Politics and Policy}. 2015;6(1-2). doi:\href{https://doi.org/10.1515/spp-2015-0001}{10.1515/spp-2015-0001}}

\bibitem[\citeproctext]{ref-uyguntunuxe72023}
\CSLLeftMargin{264. }%
\CSLRightInline{Uygun Tunç D, Tunç MN, Lakens D. The epistemic and pragmatic function of dichotomous claims based on statistical hypothesis tests. \emph{Theory \& Psychology}. 2023;33(3):403--423. doi:\href{https://doi.org/10.1177/09593543231160112}{10.1177/09593543231160112}}

\bibitem[\citeproctext]{ref-Vandenbroucke2018}
\CSLLeftMargin{265. }%
\CSLRightInline{Vandenbroucke JP, Pearce N. From ideas to studies: how to get ideas and sharpen them into research questions. \emph{Clinical Epidemiology}. 2018;Volume 10:253--264. doi:\href{https://doi.org/10.2147/clep.s142940}{10.2147/clep.s142940}}

\bibitem[\citeproctext]{ref-lakens2018}
\CSLLeftMargin{266. }%
\CSLRightInline{Lakens D, Scheel AM, Isager PM. Equivalence Testing for Psychological Research: A Tutorial. \emph{Advances in Methods and Practices in Psychological Science}. 2018;1(2):259--269. doi:\href{https://doi.org/10.1177/2515245918770963}{10.1177/2515245918770963}}

\bibitem[\citeproctext]{ref-Sullivan2012}
\CSLLeftMargin{267. }%
\CSLRightInline{Sullivan GM, Feinn R. Using Effect Size{\textemdash}or Why the {\emph{P}} Value Is Not Enough. \emph{Journal of Graduate Medical Education}. 2012;4(3):279--282. doi:\href{https://doi.org/10.4300/jgme-d-12-00156.1}{10.4300/jgme-d-12-00156.1}}

\bibitem[\citeproctext]{ref-neyman1937}
\CSLLeftMargin{268. }%
\CSLRightInline{Neyman J. Outline of a Theory of Statistical Estimation Based on the Classical Theory of Probability. \emph{Philosophical Transactions of the Royal Society of London Series A, Mathematical and Physical Sciences}. 1937;236(767):333--380. doi:\href{https://doi.org/10.1098/rsta.1937.0005}{10.1098/rsta.1937.0005}}

\bibitem[\citeproctext]{ref-goodman2016}
\CSLLeftMargin{269. }%
\CSLRightInline{Goodman SN. Aligning statistical and scientific reasoning. \emph{Science}. 2016;352(6290):1180--1181. doi:\href{https://doi.org/10.1126/science.aaf5406}{10.1126/science.aaf5406}}

\bibitem[\citeproctext]{ref-greenland2016}
\CSLLeftMargin{270. }%
\CSLRightInline{Greenland S, Senn SJ, Rothman KJ, et al. Statistical tests, P values, confidence intervals, and power: a guide to misinterpretations. \emph{European Journal of Epidemiology}. 2016;31(4):337--350. doi:\href{https://doi.org/10.1007/s10654-016-0149-3}{10.1007/s10654-016-0149-3}}

\bibitem[\citeproctext]{ref-cumming2005}
\CSLLeftMargin{271. }%
\CSLRightInline{Cumming G, Finch S. Inference by Eye: Confidence Intervals and How to Read Pictures of Data. \emph{American Psychologist}. 2005;60(2):170--180. doi:\href{https://doi.org/10.1037/0003-066x.60.2.170}{10.1037/0003-066x.60.2.170}}

\bibitem[\citeproctext]{ref-greenhalgh1997a}
\CSLLeftMargin{272. }%
\CSLRightInline{Greenhalgh T. How to read a paper: Statistics for the non-statistician. II: {̈}Significant{̈} relations and their pitfalls. \emph{BMJ}. 1997;315(7105):422--425. doi:\href{https://doi.org/10.1136/bmj.315.7105.422}{10.1136/bmj.315.7105.422}}

\bibitem[\citeproctext]{ref-weintraub2016}
\CSLLeftMargin{273. }%
\CSLRightInline{Weintraub PG. The Importance of Publishing Negative Results. \emph{Journal of Insect Science}. 2016;16(1):109. doi:\href{https://doi.org/10.1093/jisesa/iew092}{10.1093/jisesa/iew092}}

\bibitem[\citeproctext]{ref-altman1995}
\CSLLeftMargin{274. }%
\CSLRightInline{Altman DG, Bland JM. Statistics notes: Absence of evidence is not evidence of absence. \emph{BMJ}. 1995;311(7003):485--485. doi:\href{https://doi.org/10.1136/bmj.311.7003.485}{10.1136/bmj.311.7003.485}}

\bibitem[\citeproctext]{ref-gelman2014}
\CSLLeftMargin{275. }%
\CSLRightInline{Gelman A, Carlin J. Beyond Power Calculations. \emph{Perspectives on Psychological Science}. 2014;9(6):641--651. doi:\href{https://doi.org/10.1177/1745691614551642}{10.1177/1745691614551642}}

\bibitem[\citeproctext]{ref-lu2018}
\CSLLeftMargin{276. }%
\CSLRightInline{Lu J, Qiu Y, Deng A. A note on Type S/M errors in hypothesis testing. \emph{British Journal of Mathematical and Statistical Psychology}. 2018;72(1):1--17. doi:\href{https://doi.org/10.1111/bmsp.12132}{10.1111/bmsp.12132}}

\bibitem[\citeproctext]{ref-Kim2015}
\CSLLeftMargin{277. }%
\CSLRightInline{Kim HY. Statistical notes for clinical researchers: effect size. \emph{Restorative Dentistry \& Endodontics}. 2015;40(4):328. doi:\href{https://doi.org/10.5395/rde.2015.40.4.328}{10.5395/rde.2015.40.4.328}}

\bibitem[\citeproctext]{ref-epitools}
\CSLLeftMargin{278. }%
\CSLRightInline{Aragon TJ. \emph{epitools: Epidemiology Tools}.; 2020. doi:\href{https://doi.org/10.32614/CRAN.package.epitools}{10.32614/CRAN.package.epitools}}

\bibitem[\citeproctext]{ref-effectsize}
\CSLLeftMargin{279. }%
\CSLRightInline{Ben-Shachar MS, Lüdecke D, Makowski D. {\textbraceleft}e{\textbraceright}ffectsize: Estimation of Effect Size Indices and Standardized Parameters. \emph{Journal of Open Source Software}. 2020;5:2815. doi:\href{https://doi.org/10.21105/joss.02815}{10.21105/joss.02815}}

\bibitem[\citeproctext]{ref-pwr}
\CSLLeftMargin{280. }%
\CSLRightInline{Champely S. \emph{pwr: Basic Functions for Power Analysis}.; 2020. \href{https://CRAN.R-project.org/package=pwr}{https://CRAN.R-project.org/package=pwr.}}

\bibitem[\citeproctext]{ref-greenland1986}
\CSLLeftMargin{281. }%
\CSLRightInline{GREENLAND S, SCHLESSELMAN JJ, CRIQUI MH. THE FALLACY OF EMPLOYING STANDARDIZED REGRESSION COEFFICIENTS AND CORRELATIONS AS MEASURES OF EFFECT. \emph{American Journal of Epidemiology}. 1986;123(2):203--208. doi:\href{https://doi.org/10.1093/oxfordjournals.aje.a114229}{10.1093/oxfordjournals.aje.a114229}}

\bibitem[\citeproctext]{ref-greenland1991}
\CSLLeftMargin{282. }%
\CSLRightInline{Greenland S, Maclure M, Schlesselman JJ, Poole C, Morgenstern H. Standardized Regression Coefficients. \emph{Epidemiology}. 1991;2(5):387--392. doi:\href{https://doi.org/10.1097/00001648-199109000-00015}{10.1097/00001648-199109000-00015}}

\bibitem[\citeproctext]{ref-latter1902}
\CSLLeftMargin{283. }%
\CSLRightInline{LATTER OH. THE EGG OF CUCULUS CANORUS: AN ENQUIRY INTO THE DIMENSIONS OF THE CUCKOO'S EGO AND THE RELATION OF THE VARIATIONS TO THE SIZE OF THE EGGS OF THE FOSTER-PARENT, WITH NOTES ON COLORATION, \&c. \emph{Biometrika}. 1902;1(2):164--176. doi:\href{https://doi.org/10.1093/biomet/1.2.164}{10.1093/biomet/1.2.164}}

\bibitem[\citeproctext]{ref-aylmerfisher1926}
\CSLLeftMargin{284. }%
\CSLRightInline{Aylmer Fisher R. The arrangement of field experiments. \emph{Ministry of Agriculture and Fisheries}. 1926. doi:\href{https://doi.org/10.23637/ROTHAMSTED.8V61Q}{10.23637/ROTHAMSTED.8V61Q}}

\bibitem[\citeproctext]{ref-Superpower}
\CSLLeftMargin{285. }%
\CSLRightInline{Lakens D, Caldwell A. Simulation-Based Power Analysis for Factorial Analysis of Variance Designs. \emph{Advances in Methods and Practices in Psychological Science}. 2021;4:251524592095150. doi:\href{https://doi.org/10.1177/2515245920951503}{10.1177/2515245920951503}}

\bibitem[\citeproctext]{ref-wasserstein2016}
\CSLLeftMargin{286. }%
\CSLRightInline{Wasserstein RL, Lazar NA. The ASA Statement on {\emph{p}}-Values: Context, Process, and Purpose. \emph{The American Statistician}. 2016;70(2):129--133. doi:\href{https://doi.org/10.1080/00031305.2016.1154108}{10.1080/00031305.2016.1154108}}

\bibitem[\citeproctext]{ref-altman2017}
\CSLLeftMargin{287. }%
\CSLRightInline{Altman N, Krzywinski M. P values and the search for significance. \emph{Nature Methods}. 2017;14(1):3--4. doi:\href{https://doi.org/10.1038/nmeth.4120}{10.1038/nmeth.4120}}

\bibitem[\citeproctext]{ref-heinze2016}
\CSLLeftMargin{288. }%
\CSLRightInline{Heinze G, Dunkler D. Five myths about variable selection. \emph{Transplant International}. 2016;30(1):6--10. doi:\href{https://doi.org/10.1111/tri.12895}{10.1111/tri.12895}}

\bibitem[\citeproctext]{ref-blume2018}
\CSLLeftMargin{289. }%
\CSLRightInline{Blume JD, D'Agostino McGowan L, Dupont WD, Greevy RA. Second-generation p-values: Improved rigor, reproducibility, \& transparency in statistical analyses. Smalheiser NR, org. \emph{PLOS ONE}. 2018;13(3):e0188299. doi:\href{https://doi.org/10.1371/journal.pone.0188299}{10.1371/journal.pone.0188299}}

\bibitem[\citeproctext]{ref-lakens2020}
\CSLLeftMargin{290. }%
\CSLRightInline{Lakens D, Delacre M. Equivalence Testing and the Second Generation P-Value. \emph{Meta-Psychology}. 2020;4. doi:\href{https://doi.org/10.15626/mp.2018.933}{10.15626/mp.2018.933}}

\bibitem[\citeproctext]{ref-esquisse}
\CSLLeftMargin{291. }%
\CSLRightInline{Meyer F, Perrier V. \emph{esquisse: Explore and Visualize Your Data Interactively}.; 2022. \href{https://CRAN.R-project.org/package=esquisse}{https://CRAN.R-project.org/package=esquisse.}}

\bibitem[\citeproctext]{ref-cocor}
\CSLLeftMargin{292. }%
\CSLRightInline{Diedenhofen B, Musch J. cocor: A Comprehensive Solution for the Statistical Comparison of Correlations. \emph{PLOS ONE}. 2015;10:e0121945. doi:\href{https://doi.org/10.1371/journal.pone.0121945}{10.1371/journal.pone.0121945}}

\bibitem[\citeproctext]{ref-McHugh2013}
\CSLLeftMargin{293. }%
\CSLRightInline{McHugh ML. The Chi-square test of independence. \emph{Biochemia Medica}. 2013:143--149. doi:\href{https://doi.org/10.11613/bm.2013.018}{10.11613/bm.2013.018}}

\bibitem[\citeproctext]{ref-Kim2017a}
\CSLLeftMargin{294. }%
\CSLRightInline{Kim HY. Statistical notes for clinical researchers: Chi-squared test and Fisher's exact test. \emph{Restorative Dentistry \& Endodontics}. 2017;42(2):152. doi:\href{https://doi.org/10.5395/rde.2017.42.2.152}{10.5395/rde.2017.42.2.152}}

\bibitem[\citeproctext]{ref-khamis2008}
\CSLLeftMargin{295. }%
\CSLRightInline{Khamis H. Measures of Association: How to Choose? \emph{Journal of Diagnostic Medical Sonography}. 2008;24(3):155--162. doi:\href{https://doi.org/10.1177/8756479308317006}{10.1177/8756479308317006}}

\bibitem[\citeproctext]{ref-allison2022}
\CSLLeftMargin{296. }%
\CSLRightInline{Allison JS, Santana L, (Jaco) Visagie IJH. A primer on simple measures of association taught at undergraduate level. \emph{Teaching Statistics}. 2022;44(3):96--103. doi:\href{https://doi.org/10.1111/test.12307}{10.1111/test.12307}}

\bibitem[\citeproctext]{ref-psychmeta}
\CSLLeftMargin{297. }%
\CSLRightInline{Dahlke JA, Wiernik BM. {\textbraceleft}psychmeta{\textbraceright}: An R Package for Psychometric Meta-Analysis. \emph{Applied Psychological Measurement}. 2018;43(3):415--416. doi:\href{https://doi.org/10.1177/0146621618795933}{10.1177/0146621618795933}}

\bibitem[\citeproctext]{ref-anscombe1973}
\CSLLeftMargin{298. }%
\CSLRightInline{Anscombe FJ. Graphs in Statistical Analysis. \emph{The American Statistician}. 1973;27(1):17--21. doi:\href{https://doi.org/10.1080/00031305.1973.10478966}{10.1080/00031305.1973.10478966}}

\bibitem[\citeproctext]{ref-anscombiser}
\CSLLeftMargin{299. }%
\CSLRightInline{Northrop PJ. \emph{anscombiser: Create Datasets with Identical Summary Statistics}.; 2022. \href{https://CRAN.R-project.org/package=anscombiser}{https://CRAN.R-project.org/package=anscombiser.}}

\bibitem[\citeproctext]{ref-correlation}
\CSLLeftMargin{300. }%
\CSLRightInline{Makowski D, Wiernik BM, Patil I, Lüdecke D, Ben-Shachar MS. \emph{{\textbraceleft}{\textbraceleft}correlation{\textbraceright}{\textbraceright}: Methods for Correlation Analysis}.; 2022. \href{https://CRAN.R-project.org/package=correlation}{https://CRAN.R-project.org/package=correlation.}}

\bibitem[\citeproctext]{ref-easystats}
\CSLLeftMargin{301. }%
\CSLRightInline{Lüdecke D, Ben-Shachar MS, Patil I, et al. \emph{easystats: Framework for Easy Statistical Modeling, Visualization, and Reporting}.; 2022. \href{https://easystats.github.io/easystats/}{https://easystats.github.io/easystats/.}}

\bibitem[\citeproctext]{ref-Kim2019}
\CSLLeftMargin{302. }%
\CSLRightInline{Kim JH. Multicollinearity and misleading statistical results. \emph{Korean Journal of Anesthesiology}. 2019;72(6):558--569. doi:\href{https://doi.org/10.4097/kja.19087}{10.4097/kja.19087}}

\bibitem[\citeproctext]{ref-GGally}
\CSLLeftMargin{303. }%
\CSLRightInline{Schloerke B, Cook D, Larmarange J, et al. \emph{GGally: Extension to ggplot2}.; 2024. doi:\href{https://doi.org/10.32614/CRAN.package.GGally}{10.32614/CRAN.package.GGally}}

\bibitem[\citeproctext]{ref-modelsummary}
\CSLLeftMargin{304. }%
\CSLRightInline{Arel-Bundock V. {\textbraceleft}modelsummary{\textbraceright}: Data and Model Summaries in {\textbraceleft}R{\textbraceright}. \emph{Journal of Statistical Software}. 2022;103. doi:\href{https://doi.org/10.18637/jss.v103.i01}{10.18637/jss.v103.i01}}

\bibitem[\citeproctext]{ref-Hidalgo2013}
\CSLLeftMargin{305. }%
\CSLRightInline{Hidalgo B, Goodman M. Multivariate or Multivariable Regression? \emph{American Journal of Public Health}. 2013;103(1):39--40. doi:\href{https://doi.org/10.2105/ajph.2012.300897}{10.2105/ajph.2012.300897}}

\bibitem[\citeproctext]{ref-suits1957}
\CSLLeftMargin{306. }%
\CSLRightInline{Suits DB. Use of Dummy Variables in Regression Equations. \emph{Journal of the American Statistical Association}. 1957;52(280):548--551. doi:\href{https://doi.org/10.1080/01621459.1957.10501412}{10.1080/01621459.1957.10501412}}

\bibitem[\citeproctext]{ref-Healy1995}
\CSLLeftMargin{307. }%
\CSLRightInline{Healy MJ. Statistics from the inside. 16. Multiple regression (2). \emph{Archives of Disease in Childhood}. 1995;73(3):270--274. doi:\href{https://doi.org/10.1136/adc.73.3.270}{10.1136/adc.73.3.270}}

\bibitem[\citeproctext]{ref-fastDummies}
\CSLLeftMargin{308. }%
\CSLRightInline{Kaplan J. \emph{fastDummies: Fast Creation of Dummy (Binary) Columns and Rows from Categorical Variables}.; 2023. \href{https://CRAN.R-project.org/package=fastDummies}{https://CRAN.R-project.org/package=fastDummies.}}

\bibitem[\citeproctext]{ref-Sun1996}
\CSLLeftMargin{309. }%
\CSLRightInline{Sun GW, Shook TL, Kay GL. Inappropriate use of bivariable analysis to screen risk factors for use in multivariable analysis. \emph{Journal of Clinical Epidemiology}. 1996;49(8):907--916. doi:\href{https://doi.org/10.1016/0895-4356(96)00025-x}{10.1016/0895-4356(96)00025-x}}

\bibitem[\citeproctext]{ref-car}
\CSLLeftMargin{310. }%
\CSLRightInline{Fox J, Weisberg S. \emph{An {\textbraceleft}R{\textbraceright} Companion to Applied Regression}. Sage Publications, Inc.; 2019. \href{https://www.john-fox.ca/Companion/}{https://www.john-fox.ca/Companion/.}}

\bibitem[\citeproctext]{ref-Dales1978}
\CSLLeftMargin{311. }%
\CSLRightInline{DALES LG, URY HK. An Improper Use of Statistical Significance Testing in Studying Covariables. \emph{International Journal of Epidemiology}. 1978;7(4):373--376. doi:\href{https://doi.org/10.1093/ije/7.4.373}{10.1093/ije/7.4.373}}

\bibitem[\citeproctext]{ref-lindsey2011}
\CSLLeftMargin{312. }%
\CSLRightInline{Lindsey C, Sheather S. Variable Selection in Linear Regression. \emph{The Stata Journal: Promoting communications on statistics and Stata}. 2011;10(4):650--669. doi:\href{https://doi.org/10.1177/1536867x1001000407}{10.1177/1536867x1001000407}}

\bibitem[\citeproctext]{ref-leaps}
\CSLLeftMargin{313. }%
\CSLRightInline{Miller TL based on F code by A. \emph{leaps: Regression Subset Selection}.; 2024. doi:\href{https://doi.org/10.32614/CRAN.package.leaps}{10.32614/CRAN.package.leaps}}

\bibitem[\citeproctext]{ref-olsrr}
\CSLLeftMargin{314. }%
\CSLRightInline{Hebbali A. \emph{olsrr: Tools for Building OLS Regression Models}.; 2024. doi:\href{https://doi.org/10.32614/CRAN.package.olsrr}{10.32614/CRAN.package.olsrr}}

\bibitem[\citeproctext]{ref-box1976}
\CSLLeftMargin{315. }%
\CSLRightInline{Box GEP. Science and Statistics. \emph{Journal of the American Statistical Association}. 1976;71(356):791--799. doi:\href{https://doi.org/10.1080/01621459.1976.10480949}{10.1080/01621459.1976.10480949}}

\bibitem[\citeproctext]{ref-equatiomatic}
\CSLLeftMargin{316. }%
\CSLRightInline{Anderson D, Heiss A, Sumners J. \emph{equatiomatic: Transform Models into {\textbraceleft}LaTeX{\textbraceright} Equations}.; 2024. \href{https://CRAN.R-project.org/package=equatiomatic}{https://CRAN.R-project.org/package=equatiomatic.}}

\bibitem[\citeproctext]{ref-vanderploeg2014}
\CSLLeftMargin{317. }%
\CSLRightInline{Ploeg T van der, Austin PC, Steyerberg EW. Modern modelling techniques are data hungry: a simulation study for predicting dichotomous endpoints. \emph{BMC Medical Research Methodology}. 2014;14(1). doi:\href{https://doi.org/10.1186/1471-2288-14-137}{10.1186/1471-2288-14-137}}

\bibitem[\citeproctext]{ref-huxe4ggstruxf6m2007}
\CSLLeftMargin{318. }%
\CSLRightInline{HÄGGSTRÖM O. Problem Solving is Often a Matter of Cooking Up an Appropriate Markov Chain*. \emph{Scandinavian Journal of Statistics}. 2007;34(4):768--780. doi:\href{https://doi.org/10.1111/j.1467-9469.2007.00561.x}{10.1111/j.1467-9469.2007.00561.x}}

\bibitem[\citeproctext]{ref-markovchain}
\CSLLeftMargin{319. }%
\CSLRightInline{Spedicato GA. {Discrete Time Markov Chains with R}. \emph{{The R Journal}}. 2017;9(2):84--104. doi:\href{https://doi.org/10.32614/RJ-2017-036}{10.32614/RJ-2017-036}}

\bibitem[\citeproctext]{ref-Bours2023}
\CSLLeftMargin{320. }%
\CSLRightInline{Bours MJL. Using mediators to understand effect modification and interaction. \emph{Journal of Clinical Epidemiology}. setembro 2023. doi:\href{https://doi.org/10.1016/j.jclinepi.2023.09.005}{10.1016/j.jclinepi.2023.09.005}}

\bibitem[\citeproctext]{ref-Altman1996}
\CSLLeftMargin{321. }%
\CSLRightInline{Altman DG, Matthews JNS. Statistics Notes: Interaction 1: heterogeneity of effects. \emph{BMJ}. 1996;313(7055):486--486. doi:\href{https://doi.org/10.1136/bmj.313.7055.486}{10.1136/bmj.313.7055.486}}

\bibitem[\citeproctext]{ref-nlme}
\CSLLeftMargin{322. }%
\CSLRightInline{Pinheiro J, Bates D, R Core Team. \emph{nlme: Linear and Nonlinear Mixed Effects Models}.; 2023. \href{https://CRAN.R-project.org/package=nlme}{https://CRAN.R-project.org/package=nlme.}}

\bibitem[\citeproctext]{ref-mmrm}
\CSLLeftMargin{323. }%
\CSLRightInline{Sabanes Bove D, Dedic J, Kelkhoff D, et al. \emph{mmrm: Mixed Models for Repeated Measures}.; 2022. \href{https://CRAN.R-project.org/package=mmrm}{https://CRAN.R-project.org/package=mmrm.}}

\bibitem[\citeproctext]{ref-emmeans}
\CSLLeftMargin{324. }%
\CSLRightInline{Lenth RV. \emph{emmeans: Estimated Marginal Means, aka Least-Squares Means}.; 2023. \href{https://CRAN.R-project.org/package=emmeans}{https://CRAN.R-project.org/package=emmeans.}}

\bibitem[\citeproctext]{ref-Baron1986}
\CSLLeftMargin{325. }%
\CSLRightInline{Baron RM, Kenny DA. The moderator{\textendash}mediator variable distinction in social psychological research: Conceptual, strategic, and statistical considerations. \emph{Journal of Personality and Social Psychology}. 1986;51(6):1173--1182. doi:\href{https://doi.org/10.1037/0022-3514.51.6.1173}{10.1037/0022-3514.51.6.1173}}

\bibitem[\citeproctext]{ref-correctR}
\CSLLeftMargin{326. }%
\CSLRightInline{Henderson T. \emph{correctR: Corrected Test Statistics for Comparing Machine Learning Models on Correlated Samples}.; 2025. \href{https://CRAN.R-project.org/package=correctR}{https://CRAN.R-project.org/package=correctR.}}

\bibitem[\citeproctext]{ref-UComp}
\CSLLeftMargin{327. }%
\CSLRightInline{Diego J. Pedregal. \emph{UComp: Automatic Univariate Time Series Modelling of many Kinds}.; 2025. doi:\href{https://doi.org/10.32614/CRAN.package.UComp}{10.32614/CRAN.package.UComp}}

\bibitem[\citeproctext]{ref-sf}
\CSLLeftMargin{328. }%
\CSLRightInline{Pebesma E, Bivand R. \emph{{\textbraceleft}Spatial Data Science: With applications in R{\textbraceright}}. CRC Press, Taylor \& Francis Group; 2023. doi:\href{https://doi.org/10.1201/9780429459016}{10.1201/9780429459016}}

\bibitem[\citeproctext]{ref-leaflet}
\CSLLeftMargin{329. }%
\CSLRightInline{Cheng J, Schloerke B, Karambelkar B, Xie Y, Aden-Buie G. \emph{leaflet: Create Interactive Web Maps with the JavaScript 'Leaflet' Library}.; 2025. doi:\href{https://doi.org/10.32614/CRAN.package.leaflet}{10.32614/CRAN.package.leaflet}}

\bibitem[\citeproctext]{ref-survival}
\CSLLeftMargin{330. }%
\CSLRightInline{Therneau TM. \emph{A Package for Survival Analysis in R}.; 2024. \href{https://CRAN.R-project.org/package=survival}{https://CRAN.R-project.org/package=survival.}}

\bibitem[\citeproctext]{ref-tensorflow}
\CSLLeftMargin{331. }%
\CSLRightInline{Allaire J, Tang Y. \emph{tensorflow: R Interface to 'TensorFlow'}.; 2025. doi:\href{https://doi.org/10.32614/CRAN.package.tensorflow}{10.32614/CRAN.package.tensorflow}}

\bibitem[\citeproctext]{ref-torch}
\CSLLeftMargin{332. }%
\CSLRightInline{Falbel D, Luraschi J. \emph{torch: Tensors and Neural Networks with GPU Acceleration}.; 2025. doi:\href{https://doi.org/10.32614/CRAN.package.torch}{10.32614/CRAN.package.torch}}

\bibitem[\citeproctext]{ref-fastml}
\CSLLeftMargin{333. }%
\CSLRightInline{Korkmaz S, Goksuluk D, Karaismailoglu E. \emph{fastml: Guarded Resampling Workflows for Safe and Automated Machine Learning in R}.; 2026. doi:\href{https://doi.org/10.32614/CRAN.package.fastml}{10.32614/CRAN.package.fastml}}

\bibitem[\citeproctext]{ref-hand2006}
\CSLLeftMargin{334. }%
\CSLRightInline{Hand DJ. Classifier Technology and the Illusion of Progress. \emph{Statistical Science}. 2006;21(1). doi:\href{https://doi.org/10.1214/088342306000000060}{10.1214/088342306000000060}}

\bibitem[\citeproctext]{ref-andaurnavarro2023}
\CSLLeftMargin{335. }%
\CSLRightInline{Andaur Navarro CL, Damen JAA, Smeden M van, et al. Systematic review identifies the design and methodological conduct of studies on machine learning-based prediction models. \emph{Journal of Clinical Epidemiology}. 2023;154:8--22. doi:\href{https://doi.org/10.1016/j.jclinepi.2022.11.015}{10.1016/j.jclinepi.2022.11.015}}

\bibitem[\citeproctext]{ref-carriero2025}
\CSLLeftMargin{336. }%
\CSLRightInline{Carriero A, Luijken K, Hond A de, Moons KGM, Calster B van, Smeden M van. The Harms of Class Imbalance Corrections for Machine Learning Based Prediction Models: A Simulation Study. \emph{Statistics in Medicine}. 2025;44(3-4). doi:\href{https://doi.org/10.1002/sim.10320}{10.1002/sim.10320}}

\bibitem[\citeproctext]{ref-mcculloch1943}
\CSLLeftMargin{337. }%
\CSLRightInline{McCulloch WS, Pitts W. A logical calculus of the ideas immanent in nervous activity. \emph{The Bulletin of Mathematical Biophysics}. 1943;5(4):115--133. doi:\href{https://doi.org/10.1007/bf02478259}{10.1007/bf02478259}}

\bibitem[\citeproctext]{ref-rosenblatt1958}
\CSLLeftMargin{338. }%
\CSLRightInline{Rosenblatt F. The perceptron: A probabilistic model for information storage and organization in the brain. \emph{Psychological Review}. 1958;65(6):386--408. doi:\href{https://doi.org/10.1037/h0042519}{10.1037/h0042519}}

\bibitem[\citeproctext]{ref-rosenblatt1960}
\CSLLeftMargin{339. }%
\CSLRightInline{Rosenblatt F. Perceptron Simulation Experiments. \emph{Proceedings of the IRE}. 1960;48(3):301--309. doi:\href{https://doi.org/10.1109/jrproc.1960.287598}{10.1109/jrproc.1960.287598}}

\bibitem[\citeproctext]{ref-neuralnet}
\CSLLeftMargin{340. }%
\CSLRightInline{Fritsch S, Guenther F, Wright MN. \emph{neuralnet: Training of Neural Networks}.; 2019. doi:\href{https://doi.org/10.32614/CRAN.package.neuralnet}{10.32614/CRAN.package.neuralnet}}

\bibitem[\citeproctext]{ref-heckman2022}
\CSLLeftMargin{341. }%
\CSLRightInline{Heckman MG, Davis JM, Crowson CS. Post Hoc Power Calculations: An Inappropriate Method for Interpreting the Findings of a Research Study. \emph{The Journal of Rheumatology}. 2022;49(8):867--870. doi:\href{https://doi.org/10.3899/jrheum.211115}{10.3899/jrheum.211115}}

\bibitem[\citeproctext]{ref-longpower}
\CSLLeftMargin{342. }%
\CSLRightInline{Iddi S, Donohue MC. Power and Sample Size for Longitudinal Models in R -- The longpower Package and Shiny App. \emph{The R Journal}. 2022;14:264--282.}

\bibitem[\citeproctext]{ref-InteractionPoweR}
\CSLLeftMargin{343. }%
\CSLRightInline{Baranger DAA, Finsaas MC, Goldstein BL, Vize CE, Lynam DR, Olino TM. Tutorial: Power Analyses for Interaction Effects in Cross-Sectional Regressions. \emph{Advances in Methods and Practices in Psychological Science}. 2023;6(3):25152459231187531. doi:\href{https://doi.org/10.1177/25152459231187531}{10.1177/25152459231187531}}

\bibitem[\citeproctext]{ref-rodruxedguezdeluxe1guila2014}
\CSLLeftMargin{344. }%
\CSLRightInline{Rodríguez del Águila M, González-Ramírez A. Sample size calculation. \emph{Allergologia et Immunopathologia}. 2014;42(5):485--492. doi:\href{https://doi.org/10.1016/j.aller.2013.03.008}{10.1016/j.aller.2013.03.008}}

\bibitem[\citeproctext]{ref-Bacchetti2005}
\CSLLeftMargin{345. }%
\CSLRightInline{Bacchetti P. Ethics and Sample Size. \emph{American Journal of Epidemiology}. 2005;161(2):105--110. doi:\href{https://doi.org/10.1093/aje/kwi014}{10.1093/aje/kwi014}}

\bibitem[\citeproctext]{ref-ahmed2025}
\CSLLeftMargin{346. }%
\CSLRightInline{Ahmed SK. Sample size for saturation in qualitative research: Debates, definitions, and strategies. \emph{Journal of Medicine, Surgery, and Public Health}. 2025;5:100171. doi:\href{https://doi.org/10.1016/j.glmedi.2024.100171}{10.1016/j.glmedi.2024.100171}}

\bibitem[\citeproctext]{ref-hennink2022}
\CSLLeftMargin{347. }%
\CSLRightInline{Hennink M, Kaiser BN. Sample sizes for saturation in qualitative research: A systematic review of empirical tests. \emph{Social Science \& Medicine}. 2022;292:114523. doi:\href{https://doi.org/10.1016/j.socscimed.2021.114523}{10.1016/j.socscimed.2021.114523}}

\bibitem[\citeproctext]{ref-wutich2024}
\CSLLeftMargin{348. }%
\CSLRightInline{Wutich A, Beresford M, Bernard HR. Sample Sizes for 10 Types of Qualitative Data Analysis: An Integrative Review, Empirical Guidance, and Next Steps. \emph{International Journal of Qualitative Methods}. 2024;23. doi:\href{https://doi.org/10.1177/16094069241296206}{10.1177/16094069241296206}}

\bibitem[\citeproctext]{ref-vasileiou2018}
\CSLLeftMargin{349. }%
\CSLRightInline{Vasileiou K, Barnett J, Thorpe S, Young T. Characterising and justifying sample size sufficiency in interview-based studies: systematic analysis of qualitative health research over a 15-year period. \emph{BMC Medical Research Methodology}. 2018;18(1). doi:\href{https://doi.org/10.1186/s12874-018-0594-7}{10.1186/s12874-018-0594-7}}

\bibitem[\citeproctext]{ref-ying2023}
\CSLLeftMargin{350. }%
\CSLRightInline{Ying X, Robinson KA, Ehrhardt S. Re-evaluating the role of pilot trials in informing effect and sample size estimates for full-scale trials: a meta-epidemiological study. \emph{BMJ Evidence-Based Medicine}. 2023;28(6):383--391. doi:\href{https://doi.org/10.1136/bmjebm-2023-112358}{10.1136/bmjebm-2023-112358}}

\bibitem[\citeproctext]{ref-Andrade2020}
\CSLLeftMargin{351. }%
\CSLRightInline{Andrade C. Sample Size and its Importance in Research. \emph{Indian Journal of Psychological Medicine}. 2020;42(1):102--103. doi:\href{https://doi.org/10.4103/ijpsym.ijpsym_504_19}{10.4103/ijpsym.ijpsym\_504\_19}}

\bibitem[\citeproctext]{ref-Gamble2017}
\CSLLeftMargin{352. }%
\CSLRightInline{Gamble C, Krishan A, Stocken D, et al. Guidelines for the Content of Statistical Analysis Plans in Clinical Trials. \emph{JAMA}. 2017;318(23):2337. doi:\href{https://doi.org/10.1001/jama.2017.18556}{10.1001/jama.2017.18556}}

\bibitem[\citeproctext]{ref-Bland1994}
\CSLLeftMargin{353. }%
\CSLRightInline{Bland JM, Altman DG. Statistics notes: Matching. \emph{BMJ}. 1994;309(6962):1128--1128. doi:\href{https://doi.org/10.1136/bmj.309.6962.1128}{10.1136/bmj.309.6962.1128}}

\bibitem[\citeproctext]{ref-Grant2009}
\CSLLeftMargin{354. }%
\CSLRightInline{Grant MJ, Booth A. A typology of reviews: an analysis of 14 review types and associated methodologies. \emph{Health Information \& Libraries Journal}. 2009;26(2):91--108. doi:\href{https://doi.org/10.1111/j.1471-1842.2009.00848.x}{10.1111/j.1471-1842.2009.00848.x}}

\bibitem[\citeproctext]{ref-Suxfct2014}
\CSLLeftMargin{355. }%
\CSLRightInline{Sut N. Study Designs in Medicine. \emph{Balkan Medical Journal}. 2015;31(4):273--277. doi:\href{https://doi.org/10.5152/balkanmedj.2014.1408}{10.5152/balkanmedj.2014.1408}}

\bibitem[\citeproctext]{ref-Souza2017}
\CSLLeftMargin{356. }%
\CSLRightInline{Souza AC de, Alexandre NMC, Guirardello E de B, Souza AC de, Alexandre NMC, Guirardello E de B. Propriedades psicométricas na avaliação de instrumentos: avaliação da confiabilidade e da validade. \emph{Epidemiologia e Serviços de Saúde}. 2017;26(3):649--659. doi:\href{https://doi.org/10.5123/s1679-49742017000300022}{10.5123/s1679-49742017000300022}}

\bibitem[\citeproctext]{ref-reeves2017}
\CSLLeftMargin{357. }%
\CSLRightInline{Reeves BC, Wells GA, Waddington H. Quasi-experimental study designs series{\textemdash}paper 5: a checklist for~classifying studies evaluating the effects on health interventions{\textemdash}a taxonomy without labels. \emph{Journal of Clinical Epidemiology}. 2017;89:30--42. doi:\href{https://doi.org/10.1016/j.jclinepi.2017.02.016}{10.1016/j.jclinepi.2017.02.016}}

\bibitem[\citeproctext]{ref-echevarruxeda-guanilo2019}
\CSLLeftMargin{358. }%
\CSLRightInline{Echevarría-Guanilo ME, Gonçalves N, Romanoski PJ. PSYCHOMETRIC PROPERTIES OF MEASUREMENT INSTRUMENTS: CONCEPTUAL BASIS AND EVALUATION METHODS -- PART II. \emph{Texto \& Contexto -- Enfermagem}. 2019;28. doi:\href{https://doi.org/10.1590/1980-265x-tce-2017-0311}{10.1590/1980-265x-tce-2017-0311}}

\bibitem[\citeproctext]{ref-Chassuxe92019}
\CSLLeftMargin{359. }%
\CSLRightInline{Chassé M, Fergusson DA. Diagnostic Accuracy Studies. \emph{Seminars in Nuclear Medicine}. 2019;49(2):87--93. doi:\href{https://doi.org/10.1053/j.semnuclmed.2018.11.005}{10.1053/j.semnuclmed.2018.11.005}}

\bibitem[\citeproctext]{ref-Chidambaram2019}
\CSLLeftMargin{360. }%
\CSLRightInline{Chidambaram AG, Josephson M. Clinical research study designs: The essentials. \emph{PEDIATRIC INVESTIGATION}. 2019;3(4):245--252. doi:\href{https://doi.org/10.1002/ped4.12166}{10.1002/ped4.12166}}

\bibitem[\citeproctext]{ref-Erdemir2020}
\CSLLeftMargin{361. }%
\CSLRightInline{Erdemir A, Mulugeta L, Ku JP, et al. Credible practice of modeling and simulation in healthcare: ten rules from a multidisciplinary perspective. \emph{Journal of Translational Medicine}. 2020;18(1). doi:\href{https://doi.org/10.1186/s12967-020-02540-4}{10.1186/s12967-020-02540-4}}

\bibitem[\citeproctext]{ref-Yang2021}
\CSLLeftMargin{362. }%
\CSLRightInline{Yang B, Olsen M, Vali Y, et al. Study designs for comparative diagnostic test accuracy: A methodological review and classification scheme. \emph{Journal of Clinical Epidemiology}. 2021;138:128--138. doi:\href{https://doi.org/10.1016/j.jclinepi.2021.04.013}{10.1016/j.jclinepi.2021.04.013}}

\bibitem[\citeproctext]{ref-chipman2022}
\CSLLeftMargin{363. }%
\CSLRightInline{Chipman H, Bingham D. Let's practice what we preach: Planning and interpreting simulation studies with design and analysis of experiments. \emph{Canadian Journal of Statistics}. 2022;50(4):1228--1249. doi:\href{https://doi.org/10.1002/cjs.11719}{10.1002/cjs.11719}}

\bibitem[\citeproctext]{ref-donthu2021}
\CSLLeftMargin{364. }%
\CSLRightInline{Donthu N, Kumar S, Mukherjee D, Pandey N, Lim WM. How to conduct a bibliometric analysis: An overview and guidelines. \emph{Journal of Business Research}. 2021;133:285--296. doi:\href{https://doi.org/10.1016/j.jbusres.2021.04.070}{10.1016/j.jbusres.2021.04.070}}

\bibitem[\citeproctext]{ref-lim2023}
\CSLLeftMargin{365. }%
\CSLRightInline{Lim WM, Kumar S. Guidelines for interpreting the results of bibliometric analysis: A sensemaking approach. \emph{Global Business and Organizational Excellence}. agosto 2023. doi:\href{https://doi.org/10.1002/joe.22229}{10.1002/joe.22229}}

\bibitem[\citeproctext]{ref-trisovic2022a}
\CSLLeftMargin{366. }%
\CSLRightInline{Trisovic A, Lau MK, Pasquier T, Crosas M. A large-scale study on research code quality and execution. \emph{Scientific Data}. 2022;9(1). doi:\href{https://doi.org/10.1038/s41597-022-01143-6}{10.1038/s41597-022-01143-6}}

\bibitem[\citeproctext]{ref-metropolis1949b}
\CSLLeftMargin{367. }%
\CSLRightInline{Metropolis N, Ulam S. The Monte Carlo Method. \emph{Journal of the American Statistical Association}. 1949;44(247):335--341. doi:\href{https://doi.org/10.1080/01621459.1949.10483310}{10.1080/01621459.1949.10483310}}

\bibitem[\citeproctext]{ref-simstudy}
\CSLLeftMargin{368. }%
\CSLRightInline{Goldfeld K, Wujciak-Jens J. simstudy: Illuminating research methods through data generation. \emph{Journal of Open Source Software}. 2020;5:2763. doi:\href{https://doi.org/10.21105/joss.02763}{10.21105/joss.02763}}

\bibitem[\citeproctext]{ref-faux}
\CSLLeftMargin{369. }%
\CSLRightInline{DeBruine L. \emph{faux: Simulation for Factorial Designs}.; 2023. doi:\href{https://doi.org/10.5281/zenodo.2669586}{10.5281/zenodo.2669586}}

\bibitem[\citeproctext]{ref-monks2018}
\CSLLeftMargin{370. }%
\CSLRightInline{Monks T, Currie CSM, Onggo BS, Robinson S, Kunc M, Taylor SJE. Strengthening the reporting of empirical simulation studies: Introducing the STRESS guidelines. \emph{Journal of Simulation}. 2018;13(1):55--67. doi:\href{https://doi.org/10.1080/17477778.2018.1442155}{10.1080/17477778.2018.1442155}}

\bibitem[\citeproctext]{ref-vonelm2007}
\CSLLeftMargin{371. }%
\CSLRightInline{Elm E von, Altman DG, Egger M, Pocock SJ, Gøtzsche PC, Vandenbroucke JP. The Strengthening the Reporting of Observational Studies in Epidemiology (STROBE) Statement: Guidelines for Reporting Observational Studies. \emph{Annals of Internal Medicine}. 2007;147(8):573. doi:\href{https://doi.org/10.7326/0003-4819-147-8-200710160-00010}{10.7326/0003-4819-147-8-200710160-00010}}

\bibitem[\citeproctext]{ref-lavaan}
\CSLLeftMargin{372. }%
\CSLRightInline{Rosseel Y. {\textbraceleft}lavaan{\textbraceright}: An {\textbraceleft}R{\textbraceright} Package for Structural Equation Modeling. \emph{Journal of Statistical Software}. 2012;48. doi:\href{https://doi.org/10.18637/jss.v048.i02}{10.18637/jss.v048.i02}}

\bibitem[\citeproctext]{ref-findley2021}
\CSLLeftMargin{373. }%
\CSLRightInline{Findley MG, Kikuta K, Denly M. External Validity. \emph{Annual Review of Political Science}. 2021;24(1):365--393. doi:\href{https://doi.org/10.1146/annurev-polisci-041719-102556}{10.1146/annurev-polisci-041719-102556}}

\bibitem[\citeproctext]{ref-scott1955}
\CSLLeftMargin{374. }%
\CSLRightInline{Scott WA. Reliability of Content Analysis: The Case of Nominal Scale Coding. \emph{Public Opinion Quarterly}. 1955;19(3):321. doi:\href{https://doi.org/10.1086/266577}{10.1086/266577}}

\bibitem[\citeproctext]{ref-cohen1960}
\CSLLeftMargin{375. }%
\CSLRightInline{Cohen J. A Coefficient of Agreement for Nominal Scales. \emph{Educational and Psychological Measurement}. 1960;20(1):37--46. doi:\href{https://doi.org/10.1177/001316446002000104}{10.1177/001316446002000104}}

\bibitem[\citeproctext]{ref-irr}
\CSLLeftMargin{376. }%
\CSLRightInline{Gamer M, Lemon J, Ian Fellows Puspendra Singh. \emph{irr: Various Coefficients of Interrater Reliability and Agreement}.; 2019. doi:\href{https://doi.org/10.32614/CRAN.package.irr}{10.32614/CRAN.package.irr}}

\bibitem[\citeproctext]{ref-i.mathe1901}
\CSLLeftMargin{377. }%
\CSLRightInline{Mathews I, Pearson K. I. Mathematical contributions to the theory of evolution. {\textemdash}VII. On the correlation of characters not quantitatively measurable. \emph{Philosophical Transactions of the Royal Society of London Series A, Containing Papers of a Mathematical or Physical Character}. 1901;195(262-273):1--47. doi:\href{https://doi.org/10.1098/rsta.1900.0022}{10.1098/rsta.1900.0022}}

\bibitem[\citeproctext]{ref-banerjee1999}
\CSLLeftMargin{378. }%
\CSLRightInline{Banerjee M, Capozzoli M, McSweeney L, Sinha D. Beyond kappa: A review of interrater agreement measures. \emph{Canadian Journal of Statistics}. 1999;27(1):3--23. doi:\href{https://doi.org/10.2307/3315487}{10.2307/3315487}}

\bibitem[\citeproctext]{ref-psych}
\CSLLeftMargin{379. }%
\CSLRightInline{William Revelle. \emph{psych: Procedures for Psychological, Psychometric, and Personality Research}.; 2023. \href{https://CRAN.R-project.org/package=psych}{https://CRAN.R-project.org/package=psych.}}

\bibitem[\citeproctext]{ref-BlandAltmanLeh}
\CSLLeftMargin{380. }%
\CSLRightInline{Lehnert B. \emph{BlandAltmanLeh: Plots (Slightly Extended) Bland-Altman Plots}.; 2015. \href{https://CRAN.R-project.org/package=BlandAltmanLeh}{https://CRAN.R-project.org/package=BlandAltmanLeh.}}

\bibitem[\citeproctext]{ref-semTools}
\CSLLeftMargin{381. }%
\CSLRightInline{Contributors semTools. \emph{{semTools}: Useful tools for structural equation modeling}.; 2016. \href{https://CRAN.R-project.org/package=semTools}{https://CRAN.R-project.org/package=semTools.}}

\bibitem[\citeproctext]{ref-jomo}
\CSLLeftMargin{382. }%
\CSLRightInline{Quartagno M, Carpenter J. \emph{{\textbraceleft}jomo{\textbraceright}: A package for Multilevel Joint Modelling Multiple Imputation}.; 2023. \href{https://CRAN.R-project.org/package=jomo}{https://CRAN.R-project.org/package=jomo.}}

\bibitem[\citeproctext]{ref-gagnier2021}
\CSLLeftMargin{383. }%
\CSLRightInline{Gagnier JJ, Lai J, Mokkink LB, Terwee CB. COSMIN reporting guideline for studies on measurement properties of patient-reported outcome measures. \emph{Quality of Life Research}. 2021;30(8):2197--2218. doi:\href{https://doi.org/10.1007/s11136-021-02822-4}{10.1007/s11136-021-02822-4}}

\bibitem[\citeproctext]{ref-streiner2014}
\CSLLeftMargin{384. }%
\CSLRightInline{Streiner DL, Kottner J. Recommendations for reporting the results of studies of instrument and scale development and testing. \emph{Journal of Advanced Nursing}. 2014;70(9):1970--1979. doi:\href{https://doi.org/10.1111/jan.12402}{10.1111/jan.12402}}

\bibitem[\citeproctext]{ref-kottner2011}
\CSLLeftMargin{385. }%
\CSLRightInline{Kottner J, Audigé L, Brorson S, et al. Guidelines for Reporting Reliability and Agreement Studies (GRRAS) were proposed. \emph{Journal of Clinical Epidemiology}. 2011;64(1):96--106. doi:\href{https://doi.org/10.1016/j.jclinepi.2010.03.002}{10.1016/j.jclinepi.2010.03.002}}

\bibitem[\citeproctext]{ref-steckelberg2004}
\CSLLeftMargin{386. }%
\CSLRightInline{Steckelberg A, Balgenorth A, Berger J, Mühlhauser I. Explaining computation of predictive values: 2 × 2 table versus frequency tree. A randomized controlled trial {[}ISRCTN74278823{]}. \emph{BMC Medical Education}. 2004;4(1). doi:\href{https://doi.org/10.1186/1472-6920-4-13}{10.1186/1472-6920-4-13}}

\bibitem[\citeproctext]{ref-greenhalgh1997b}
\CSLLeftMargin{387. }%
\CSLRightInline{Greenhalgh T. How to read a paper: Papers that report diagnostic or screening tests. \emph{BMJ}. 1997;315(7107):540--543. doi:\href{https://doi.org/10.1136/bmj.315.7107.540}{10.1136/bmj.315.7107.540}}

\bibitem[\citeproctext]{ref-riskyr}
\CSLLeftMargin{388. }%
\CSLRightInline{Neth H, Gaisbauer F, Gradwohl N, Gaissmaier W. \emph{riskyr: Rendering Risk Literacy more Transparent}.; 2022. \href{https://CRAN.R-project.org/package=riskyr}{https://CRAN.R-project.org/package=riskyr.}}

\bibitem[\citeproctext]{ref-Glas2003}
\CSLLeftMargin{389. }%
\CSLRightInline{Glas AS, Lijmer JG, Prins MH, Bonsel GJ, Bossuyt PMM. The diagnostic odds ratio: a single indicator of test performance. \emph{Journal of Clinical Epidemiology}. 2003;56(11):1129--1135. doi:\href{https://doi.org/10.1016/s0895-4356(03)00177-x}{10.1016/s0895-4356(03)00177-x}}

\bibitem[\citeproctext]{ref-caret}
\CSLLeftMargin{390. }%
\CSLRightInline{Kuhn, Max. Building Predictive Models in R Using the caret Package. \emph{Journal of Statistical Software}. 2008;28(5):1--26. doi:\href{https://doi.org/10.18637/jss.v028.i05}{10.18637/jss.v028.i05}}

\bibitem[\citeproctext]{ref-xu2020}
\CSLLeftMargin{391. }%
\CSLRightInline{Xu J, Zhang Y, Miao D. Three-way confusion matrix for classification: A measure driven view. \emph{Information Sciences}. 2020;507:772--794. doi:\href{https://doi.org/10.1016/j.ins.2019.06.064}{10.1016/j.ins.2019.06.064}}

\bibitem[\citeproctext]{ref-he2024}
\CSLLeftMargin{392. }%
\CSLRightInline{He Z, Zhang Q, Song M, Tan X, Wang W. Four overlooked errors in ROC analysis: how to prevent and avoid. \emph{BMJ Evidence-Based Medicine}. 2024;30(3):208--211. doi:\href{https://doi.org/10.1136/bmjebm-2024-113078}{10.1136/bmjebm-2024-113078}}

\bibitem[\citeproctext]{ref-park2004}
\CSLLeftMargin{393. }%
\CSLRightInline{Park SH, Goo JM, Jo CH. Receiver Operating Characteristic (ROC) Curve: Practical Review for Radiologists. \emph{Korean Journal of Radiology}. 2004;5(1):11. doi:\href{https://doi.org/10.3348/kjr.2004.5.1.11}{10.3348/kjr.2004.5.1.11}}

\bibitem[\citeproctext]{ref-yarnold2014}
\CSLLeftMargin{394. }%
\CSLRightInline{Park SH, Goo JM, Jo CH. UniODA vs ROC Analysis: Computing the {``optimal''} cut-point. \emph{Optimal Data Analysis}. 2014;3(14):117--120. \href{https://odajournal.com/wp-content/uploads/2019/01/v3a29.pdf}{https://odajournal.com/wp-content/uploads/2019/01/v3a29.pdf.}}

\bibitem[\citeproctext]{ref-de2022}
\CSLLeftMargin{395. }%
\CSLRightInline{Hond AAH de, Steyerberg EW, Calster B van. Interpreting area under the receiver operating characteristic curve. \emph{The Lancet Digital Health}. 2022;4(12):e853--e855. doi:\href{https://doi.org/10.1016/s2589-7500(22)00188-1}{10.1016/s2589-7500(22)00188-1}}

\bibitem[\citeproctext]{ref-pROC}
\CSLLeftMargin{396. }%
\CSLRightInline{Robin X, Turck N, Hainard A, et al. pROC: an open-source package for R and S+ to analyze and compare ROC curves. \emph{BMC Bioinformatics}. 2011;12:77. doi:\href{https://doi.org/10.1186/1471-2105-12-77}{10.1186/1471-2105-12-77}}

\bibitem[\citeproctext]{ref-ferreira2021}
\CSLLeftMargin{397. }%
\CSLRightInline{Ferreira ADS, Meziat-Filho N, Ferreira APA. Double threshold receiver operating characteristic plot for three-modal continuous predictors. \emph{Computational Statistics}. 2021;36(3):2231--2245. doi:\href{https://doi.org/10.1007/s00180-021-01080-9}{10.1007/s00180-021-01080-9}}

\bibitem[\citeproctext]{ref-phillips2010}
\CSLLeftMargin{398. }%
\CSLRightInline{Phillips B, Stewart LA, Sutton AJ. {`}Cross hairs{'} plots for diagnostic meta{-}analysis. \emph{Research Synthesis Methods}. 2010;1(3-4):308--315. doi:\href{https://doi.org/10.1002/jrsm.26}{10.1002/jrsm.26}}

\bibitem[\citeproctext]{ref-mada}
\CSLLeftMargin{399. }%
\CSLRightInline{Sousa-Pinto PD with contributions from B. \emph{mada: Meta-Analysis of Diagnostic Accuracy}.; 2022. \href{https://CRAN.R-project.org/package=mada}{https://CRAN.R-project.org/package=mada.}}

\bibitem[\citeproctext]{ref-bossuyt2015}
\CSLLeftMargin{400. }%
\CSLRightInline{Bossuyt PM, Reitsma JB, Bruns DE, et al. STARD 2015: an updated list of essential items for reporting diagnostic accuracy studies. \emph{BMJ}. outubro 2015:h5527. doi:\href{https://doi.org/10.1136/bmj.h5527}{10.1136/bmj.h5527}}

\bibitem[\citeproctext]{ref-reeves2004}
\CSLLeftMargin{401. }%
\CSLRightInline{Reeves BC, Gaus W. Guidelines for Reporting Non-Randomised Studies. \emph{Complementary Medicine Research}. 2004;11(1):46--52. doi:\href{https://doi.org/10.1159/000080576}{10.1159/000080576}}

\bibitem[\citeproctext]{ref-bland2011}
\CSLLeftMargin{402. }%
\CSLRightInline{Bland JM, Altman DG. Comparisons within randomised groups can be very misleading. \emph{BMJ}. 2011;342(may06 2):d561--d561. doi:\href{https://doi.org/10.1136/bmj.d561}{10.1136/bmj.d561}}

\bibitem[\citeproctext]{ref-Bruce2022}
\CSLLeftMargin{403. }%
\CSLRightInline{Bruce CL, Juszczak E, Ogollah R, Partlett C, Montgomery A. A systematic review of randomisation method use in RCTs and association of trial design characteristics with method selection. \emph{BMC Medical Research Methodology}. 2022;22(1). doi:\href{https://doi.org/10.1186/s12874-022-01786-4}{10.1186/s12874-022-01786-4}}

\bibitem[\citeproctext]{ref-Vickers2001a}
\CSLLeftMargin{404. }%
\CSLRightInline{Vickers AJ, Altman DG. Statistics Notes: Analysing controlled trials with baseline and follow up measurements. \emph{BMJ}. 2001;323(7321):1123--1124. doi:\href{https://doi.org/10.1136/bmj.323.7321.1123}{10.1136/bmj.323.7321.1123}}

\bibitem[\citeproctext]{ref-OConnell2017}
\CSLLeftMargin{405. }%
\CSLRightInline{O Connell NS, Dai L, Jiang Y, et al. Methods for Analysis of Pre-Post Data in Clinical Research: A Comparison of Five Common Methods. \emph{Journal of Biometrics \& Biostatistics}. 2017;08(01). doi:\href{https://doi.org/10.4172/2155-6180.1000334}{10.4172/2155-6180.1000334}}

\bibitem[\citeproctext]{ref-laird1983}
\CSLLeftMargin{406. }%
\CSLRightInline{Laird N. Further Comparative Analyses of Pretest-Posttest Research Designs. \emph{The American Statistician}. 1983;37(4a):329--330. doi:\href{https://doi.org/10.1080/00031305.1983.10483133}{10.1080/00031305.1983.10483133}}

\bibitem[\citeproctext]{ref-Cnaan1997}
\CSLLeftMargin{407. }%
\CSLRightInline{Cnaan A, Laird NM, Slasor P. Using the general linear mixed model to analyse unbalanced repeated measures and longitudinal data. \emph{Statistics in Medicine}. 1997;16(20):2349--2380. doi:\href{https://doi.org/10.1002/(sici)1097-0258(19971030)16:20\%3C2349::aid-sim667\%3E3.0.co;2-e}{10.1002/(sici)1097-0258(19971030)16:20\textless2349::aid-sim667\textgreater3.0.co;2-e}}

\bibitem[\citeproctext]{ref-mallinckrodt2008}
\CSLLeftMargin{408. }%
\CSLRightInline{Mallinckrodt CH, Lane PW, Schnell D, Peng Y, Mancuso JP. Recommendations for the Primary Analysis of Continuous Endpoints in Longitudinal Clinical Trials. \emph{Drug Information Journal}. 2008;42(4):303--319. doi:\href{https://doi.org/10.1177/009286150804200402}{10.1177/009286150804200402}}

\bibitem[\citeproctext]{ref-Assmann2000}
\CSLLeftMargin{409. }%
\CSLRightInline{Assmann SF, Pocock SJ, Enos LE, Kasten LE. Subgroup analysis and other (mis)uses of baseline data in clinical trials. \emph{The Lancet}. 2000;355(9209):1064--1069. doi:\href{https://doi.org/10.1016/s0140-6736(00)02039-0}{10.1016/s0140-6736(00)02039-0}}

\bibitem[\citeproctext]{ref-Stang2018}
\CSLLeftMargin{410. }%
\CSLRightInline{Stang A, Baethge C. Imbalance \textless em\textgreater p\textless/em\textgreater{} values for baseline covariates in randomized controlled trials: a last resort for the use of \textless em\textgreater p\textless/em\textgreater{} values? A pro and contra debate. \emph{Clinical Epidemiology}. 2018;Volume 10:531--535. doi:\href{https://doi.org/10.2147/clep.s161508}{10.2147/clep.s161508}}

\bibitem[\citeproctext]{ref-Bolzern2019}
\CSLLeftMargin{411. }%
\CSLRightInline{Bolzern JE, Mitchell A, Torgerson DJ. Baseline testing in cluster randomised controlled trials: should this be done? \emph{BMC Medical Research Methodology}. 2019;19(1). doi:\href{https://doi.org/10.1186/s12874-019-0750-8}{10.1186/s12874-019-0750-8}}

\bibitem[\citeproctext]{ref-roberts1999}
\CSLLeftMargin{412. }%
\CSLRightInline{Roberts C, Torgerson DJ. Understanding controlled trials: Baseline imbalance in randomised controlled trials. \emph{BMJ}. 1999;319(7203):185--185. doi:\href{https://doi.org/10.1136/bmj.319.7203.185}{10.1136/bmj.319.7203.185}}

\bibitem[\citeproctext]{ref-gruijters2020}
\CSLLeftMargin{413. }%
\CSLRightInline{Gruijters SLK. Baseline comparisons and covariate fishing: Bad statistical habits we should have broken yesterday. julho 2020. \href{http://dx.doi.org/10.31234/osf.io/qftwg}{http://dx.doi.org/10.31234/osf.io/qftwg.}}

\bibitem[\citeproctext]{ref-vickers2001b}
\CSLLeftMargin{414. }%
\CSLRightInline{Vickers AJ. The use of percentage change from baseline as an outcome in a controlled trial is statistically inefficient: a simulation study. \emph{BMC Medical Research Methodology}. 2001;1(1). doi:\href{https://doi.org/10.1186/1471-2288-1-6}{10.1186/1471-2288-1-6}}

\bibitem[\citeproctext]{ref-Brookes2004}
\CSLLeftMargin{415. }%
\CSLRightInline{Brookes ST, Whitely E, Egger M, Smith GD, Mulheran PA, Peters TJ. Subgroup analyses in randomized trials: risks of subgroup-specific analyses; \emph{Journal of Clinical Epidemiology}. 2004;57(3):229--236. doi:\href{https://doi.org/10.1016/j.jclinepi.2003.08.009}{10.1016/j.jclinepi.2003.08.009}}

\bibitem[\citeproctext]{ref-Matthews1996}
\CSLLeftMargin{416. }%
\CSLRightInline{Matthews JNS, Altman DG. Statistics Notes: Interaction 2: compare effect sizes not P values. \emph{BMJ}. 1996;313(7060):808--808. doi:\href{https://doi.org/10.1136/bmj.313.7060.808}{10.1136/bmj.313.7060.808}}

\bibitem[\citeproctext]{ref-Altman2003}
\CSLLeftMargin{417. }%
\CSLRightInline{Altman DG. Statistics Notes: Interaction revisited: the difference between two estimates. \emph{BMJ}. 2003;326(7382):219--219. doi:\href{https://doi.org/10.1136/bmj.326.7382.219}{10.1136/bmj.326.7382.219}}

\bibitem[\citeproctext]{ref-Hauck1998}
\CSLLeftMargin{418. }%
\CSLRightInline{Hauck WW, Anderson S, Marcus SM. Should We Adjust for Covariates in Nonlinear Regression Analyses of Randomized Trials? \emph{Controlled Clinical Trials}. 1998;19(3):249--256. doi:\href{https://doi.org/10.1016/s0197-2456(97)00147-5}{10.1016/s0197-2456(97)00147-5}}

\bibitem[\citeproctext]{ref-Kahan2014}
\CSLLeftMargin{419. }%
\CSLRightInline{Kahan BC, Jairath V, Doré CJ, Morris TP. The risks and rewards of covariate adjustment in randomized trials: an assessment of 12 outcomes from 8 studies. \emph{Trials}. 2014;15(1). doi:\href{https://doi.org/10.1186/1745-6215-15-139}{10.1186/1745-6215-15-139}}

\bibitem[\citeproctext]{ref-Cao2022}
\CSLLeftMargin{420. }%
\CSLRightInline{Cao Y, Allore H, Vander Wyk B, Gutman R. Review and evaluation of imputation methods for multivariate longitudinal data with mixed{-}type incomplete variables. \emph{Statistics in Medicine}. outubro 2022. doi:\href{https://doi.org/10.1002/sim.9592}{10.1002/sim.9592}}

\bibitem[\citeproctext]{ref-schulz2010}
\CSLLeftMargin{421. }%
\CSLRightInline{Schulz KF. CONSORT 2010 Statement: Updated Guidelines for Reporting Parallel Group Randomized Trials. \emph{Annals of Internal Medicine}. 2010;152(11):726. doi:\href{https://doi.org/10.7326/0003-4819-152-11-201006010-00232}{10.7326/0003-4819-152-11-201006010-00232}}

\bibitem[\citeproctext]{ref-consort}
\CSLLeftMargin{422. }%
\CSLRightInline{Dayim A. \emph{consort: Create Consort Diagram}.; 2023. \href{https://CRAN.R-project.org/package=consort}{https://CRAN.R-project.org/package=consort.}}

\bibitem[\citeproctext]{ref-dwan2019}
\CSLLeftMargin{423. }%
\CSLRightInline{Dwan K, Li T, Altman DG, Elbourne D. CONSORT 2010 statement: extension to randomised crossover trials. \emph{BMJ}. julho 2019:l4378. doi:\href{https://doi.org/10.1136/bmj.l4378}{10.1136/bmj.l4378}}

\bibitem[\citeproctext]{ref-senn2024}
\CSLLeftMargin{424. }%
\CSLRightInline{Senn S. The analysis of continuous data from n-of-1 trials using paired cycles: a simple tutorial. \emph{Trials}. 2024;25(1). doi:\href{https://doi.org/10.1186/s13063-024-07964-7}{10.1186/s13063-024-07964-7}}

\bibitem[\citeproctext]{ref-baker2014}
\CSLLeftMargin{425. }%
\CSLRightInline{Baker KA, Weeks SM. An Overview of Systematic Review. \emph{Journal of PeriAnesthesia Nursing}. 2014;29(6):454--458. doi:\href{https://doi.org/10.1016/j.jopan.2014.07.002}{10.1016/j.jopan.2014.07.002}}

\bibitem[\citeproctext]{ref-easyPubMed}
\CSLLeftMargin{426. }%
\CSLRightInline{Fantini D. \emph{easyPubMed: Search and Retrieve Scientific Publication Records from PubMed}.; 2019. doi:\href{https://doi.org/10.32614/CRAN.package.easyPubMed}{10.32614/CRAN.package.easyPubMed}}

\bibitem[\citeproctext]{ref-rcrossref}
\CSLLeftMargin{427. }%
\CSLRightInline{Chamberlain S, Zhu H, Jahn N, Boettiger C, Ram K. \emph{rcrossref: Client for Various CrossRef APIs}.; 2022. doi:\href{https://doi.org/10.32614/CRAN.package.rcrossref}{10.32614/CRAN.package.rcrossref}}

\bibitem[\citeproctext]{ref-roadoi}
\CSLLeftMargin{428. }%
\CSLRightInline{Jahn N. \emph{roadoi: Find Free Versions of Scholarly Publications via Unpaywall}.; 2024. doi:\href{https://doi.org/10.32614/CRAN.package.roadoi}{10.32614/CRAN.package.roadoi}}

\bibitem[\citeproctext]{ref-silva2012}
\CSLLeftMargin{429. }%
\CSLRightInline{Silva V, Grande AJ, Martimbianco ALC, Riera R, Carvalho APV. Overview of systematic reviews - a new type of study: part I: why and for whom? \emph{Sao Paulo Medical Journal}. 2012;130(6):398--404. doi:\href{https://doi.org/10.1590/s1516-31802012000600007}{10.1590/s1516-31802012000600007}}

\bibitem[\citeproctext]{ref-silva2014}
\CSLLeftMargin{430. }%
\CSLRightInline{Silva V, Grande AJ, Carvalho APV de, Martimbianco ALC, Riera R. Overview of systematic reviews - a new type of study. Part II. \emph{Sao Paulo Medical Journal}. 2014;133(3):206--217. doi:\href{https://doi.org/10.1590/1516-3180.2013.8150015}{10.1590/1516-3180.2013.8150015}}

\bibitem[\citeproctext]{ref-stern2025}
\CSLLeftMargin{431. }%
\CSLRightInline{Stern C, Li J, Stone J, et al. Data analysis and presentation methods in umbrella reviews/overviews of reviews in health care: A cross-sectional study. \emph{Research Synthesis Methods}. outubro 2025:1--15. doi:\href{https://doi.org/10.1017/rsm.2025.10040}{10.1017/rsm.2025.10040}}

\bibitem[\citeproctext]{ref-snell2023}
\CSLLeftMargin{432. }%
\CSLRightInline{Snell KIE, Levis B, Damen JAA, et al. Transparent reporting of multivariable prediction models for individual prognosis or diagnosis: checklist for systematic reviews and meta-analyses (TRIPOD-SRMA). \emph{BMJ}. maio 2023:e073538. doi:\href{https://doi.org/10.1136/bmj-2022-073538}{10.1136/bmj-2022-073538}}

\bibitem[\citeproctext]{ref-moons2014}
\CSLLeftMargin{433. }%
\CSLRightInline{Moons KGM, Groot JAH de, Bouwmeester W, et al. Critical Appraisal and Data Extraction for Systematic Reviews of Prediction Modelling Studies: The CHARMS Checklist. \emph{PLoS Medicine}. 2014;11(10):e1001744. doi:\href{https://doi.org/10.1371/journal.pmed.1001744}{10.1371/journal.pmed.1001744}}

\bibitem[\citeproctext]{ref-borenstein2010}
\CSLLeftMargin{434. }%
\CSLRightInline{Borenstein M, Hedges LV, Higgins JPT, Rothstein HR. A basic introduction to fixed-effect and random-effects models for meta-analysis. \emph{Research Synthesis Methods}. 2010;1(2):97--111. doi:\href{https://doi.org/10.1002/jrsm.12}{10.1002/jrsm.12}}

\bibitem[\citeproctext]{ref-metafor}
\CSLLeftMargin{435. }%
\CSLRightInline{Viechtbauer W. \emph{Conducting meta-analyses in {\textbraceleft}R{\textbraceright} with the {\textbraceleft}metafor{\textbraceright} package}. Vol 36.; 2010. doi:\href{https://doi.org/10.18637/jss.v036.i03}{10.18637/jss.v036.i03}}

\bibitem[\citeproctext]{ref-netmeta}
\CSLLeftMargin{436. }%
\CSLRightInline{Balduzzi S, Rücker G, Nikolakopoulou A, et al. netmeta: An R Package for Network Meta-Analysis Using Frequentist Methods. \emph{Journal of Statistical Software}. 2023;106(2):1--40. doi:\href{https://doi.org/10.18637/jss.v106.i02}{10.18637/jss.v106.i02}}

\bibitem[\citeproctext]{ref-gemtc}
\CSLLeftMargin{437. }%
\CSLRightInline{Valkenhoef G van, Kuiper J. \emph{gemtc: Network Meta-Analysis Using Bayesian Methods}.; 2025. doi:\href{https://doi.org/10.32614/CRAN.package.gemtc}{10.32614/CRAN.package.gemtc}}

\bibitem[\citeproctext]{ref-hozo2005}
\CSLLeftMargin{438. }%
\CSLRightInline{Hozo SP, Djulbegovic B, Hozo I. Estimating the mean and variance from the median, range, and the size of a sample. \emph{BMC Medical Research Methodology}. 2005;5(1). doi:\href{https://doi.org/10.1186/1471-2288-5-13}{10.1186/1471-2288-5-13}}

\bibitem[\citeproctext]{ref-wan2014}
\CSLLeftMargin{439. }%
\CSLRightInline{Wan X, Wang W, Liu J, Tong T. Estimating the sample mean and standard deviation from the sample size, median, range and/or interquartile range. \emph{BMC Medical Research Methodology}. 2014;14(1). doi:\href{https://doi.org/10.1186/1471-2288-14-135}{10.1186/1471-2288-14-135}}

\bibitem[\citeproctext]{ref-Borenstein2022}
\CSLLeftMargin{440. }%
\CSLRightInline{Borenstein M. In a meta-analysis, the I-squared statistic does not tell us how much the effect size varies. \emph{Journal of Clinical Epidemiology}. outubro 2022. doi:\href{https://doi.org/10.1016/j.jclinepi.2022.10.003}{10.1016/j.jclinepi.2022.10.003}}

\bibitem[\citeproctext]{ref-Ruxfccker2008}
\CSLLeftMargin{441. }%
\CSLRightInline{Rücker G, Schwarzer G, Carpenter JR, Schumacher M. Undue reliance on I 2 in assessing heterogeneity may mislead. \emph{BMC Medical Research Methodology}. 2008;8(1). doi:\href{https://doi.org/10.1186/1471-2288-8-79}{10.1186/1471-2288-8-79}}

\bibitem[\citeproctext]{ref-degrooth2023}
\CSLLeftMargin{442. }%
\CSLRightInline{Grooth HJ de, Parienti JJ. Heterogeneity between studies can be explained more reliably with individual patient data. \emph{Intensive Care Medicine}. julho 2023. doi:\href{https://doi.org/10.1007/s00134-023-07163-z}{10.1007/s00134-023-07163-z}}

\bibitem[\citeproctext]{ref-dettori2021}
\CSLLeftMargin{443. }%
\CSLRightInline{Dettori JR, Norvell DC, Chapman JR. Seeing the Forest by Looking at the Trees: How to Interpret a Meta-Analysis Forest Plot. \emph{Global Spine Journal}. 2021;11(4):614--616. doi:\href{https://doi.org/10.1177/21925682211003889}{10.1177/21925682211003889}}

\bibitem[\citeproctext]{ref-song2000}
\CSLLeftMargin{444. }%
\CSLRightInline{Song, Eastwood, Gilbody, Duley, Sutton. Publication and related biases. \emph{Health Technology Assessment}. 2000;4(10). doi:\href{https://doi.org/10.3310/hta4100}{10.3310/hta4100}}

\bibitem[\citeproctext]{ref-egger1997}
\CSLLeftMargin{445. }%
\CSLRightInline{Egger M, Smith GD, Schneider M, Minder C. Bias in meta-analysis detected by a simple, graphical test. \emph{BMJ}. 1997;315(7109):629--634. doi:\href{https://doi.org/10.1136/bmj.315.7109.629}{10.1136/bmj.315.7109.629}}

\bibitem[\citeproctext]{ref-peters2006}
\CSLLeftMargin{446. }%
\CSLRightInline{Peters JL. Comparison of Two Methods to Detect Publication Bias in Meta-analysis. \emph{JAMA}. 2006;295(6):676. doi:\href{https://doi.org/10.1001/jama.295.6.676}{10.1001/jama.295.6.676}}

\bibitem[\citeproctext]{ref-sterne2011}
\CSLLeftMargin{447. }%
\CSLRightInline{Sterne JAC, Sutton AJ, Ioannidis JPA, et al. Recommendations for examining and interpreting funnel plot asymmetry in meta-analyses of randomised controlled trials. \emph{BMJ}. 2011;343(jul22 1):d4002--d4002. doi:\href{https://doi.org/10.1136/bmj.d4002}{10.1136/bmj.d4002}}

\bibitem[\citeproctext]{ref-duval2000}
\CSLLeftMargin{448. }%
\CSLRightInline{Duval S, Tweedie R. Trim and Fill: A Simple Funnel{-}Plot{\textendash}Based Method of Testing and Adjusting for Publication Bias in Meta{-}Analysis. \emph{Biometrics}. 2000;56(2):455--463. doi:\href{https://doi.org/10.1111/j.0006-341x.2000.00455.x}{10.1111/j.0006-341x.2000.00455.x}}

\bibitem[\citeproctext]{ref-page2021}
\CSLLeftMargin{449. }%
\CSLRightInline{Page MJ, McKenzie JE, Bossuyt PM, et al. The PRISMA 2020 statement: An updated guideline for reporting systematic reviews. \emph{PLOS Medicine}. 2021;18(3):e1003583. doi:\href{https://doi.org/10.1371/journal.pmed.1003583}{10.1371/journal.pmed.1003583}}

\bibitem[\citeproctext]{ref-metagear}
\CSLLeftMargin{450. }%
\CSLRightInline{Lajeunesse MJ. Facilitating systematic reviews, data extraction, and meta-analysis with the metagear package for R. \emph{Methods in Ecology and Evolution}. 2016;7(3):323--330. doi:\href{https://doi.org/10.1111/2041-210X.12472}{10.1111/2041-210X.12472}}

\bibitem[\citeproctext]{ref-Moher2015}
\CSLLeftMargin{451. }%
\CSLRightInline{Moher D, Shamseer L, Clarke M, et al. Preferred reporting items for systematic review and meta-analysis protocols (PRISMA-P) 2015 statement. \emph{Systematic Reviews}. 2015;4(1). doi:\href{https://doi.org/10.1186/2046-4053-4-1}{10.1186/2046-4053-4-1}}

\bibitem[\citeproctext]{ref-PRISMA2020}
\CSLLeftMargin{452. }%
\CSLRightInline{Haddaway NR, Page MJ, Pritchard CC, McGuinness LA. PRISMA2020: An R package and Shiny app for producing PRISMA 2020-compliant flow diagrams, with interactivity for optimised digital transparency and Open Synthesis. \emph{Campbell Systematic Reviews}. 2022;18:e1230. doi:\href{https://doi.org/10.1002/cl2.1230}{10.1002/cl2.1230}}

\bibitem[\citeproctext]{ref-gates2022}
\CSLLeftMargin{453. }%
\CSLRightInline{Gates M, Gates A, Pieper D, et al. Reporting guideline for overviews of reviews of healthcare interventions: development of the PRIOR statement. \emph{BMJ}. agosto 2022:e070849. doi:\href{https://doi.org/10.1136/bmj-2022-070849}{10.1136/bmj-2022-070849}}

\bibitem[\citeproctext]{ref-ocathain2013}
\CSLLeftMargin{454. }%
\CSLRightInline{O'Cathain A, Thomas KJ, Drabble SJ, Rudolph A, Hewison J. What can qualitative research do for randomised controlled trials? A systematic mapping review. \emph{BMJ Open}. 2013;3(6):e002889. doi:\href{https://doi.org/10.1136/bmjopen-2013-002889}{10.1136/bmjopen-2013-002889}}

\bibitem[\citeproctext]{ref-obrien2014}
\CSLLeftMargin{455. }%
\CSLRightInline{O'Brien BC, Harris IB, Beckman TJ, Reed DA, Cook DA. Standards for Reporting Qualitative Research. \emph{Academic Medicine}. 2014;89(9):1245--1251. doi:\href{https://doi.org/10.1097/acm.0000000000000388}{10.1097/acm.0000000000000388}}

\bibitem[\citeproctext]{ref-tong2012}
\CSLLeftMargin{456. }%
\CSLRightInline{Tong A, Flemming K, McInnes E, Oliver S, Craig J. Enhancing transparency in reporting the synthesis of qualitative research: ENTREQ. \emph{BMC Medical Research Methodology}. 2012;12(1). doi:\href{https://doi.org/10.1186/1471-2288-12-181}{10.1186/1471-2288-12-181}}

\bibitem[\citeproctext]{ref-tong2007}
\CSLLeftMargin{457. }%
\CSLRightInline{Tong A, Sainsbury P, Craig J. Consolidated criteria for reporting qualitative research (COREQ): a 32-item checklist for interviews and focus groups. \emph{International Journal for Quality in Health Care}. 2007;19(6):349--357. doi:\href{https://doi.org/10.1093/intqhc/mzm042}{10.1093/intqhc/mzm042}}

\bibitem[\citeproctext]{ref-report}
\CSLLeftMargin{458. }%
\CSLRightInline{Makowski D, Lüdecke D, Patil I, Thériault R, Ben-Shachar MS, Wiernik BM. \emph{Automated Results Reporting as a Practical Tool to Improve Reproducibility and Methodological Best Practices Adoption}.; 2023. \href{https://easystats.github.io/report/}{https://easystats.github.io/report/.}}

\bibitem[\citeproctext]{ref-statcheck}
\CSLLeftMargin{459. }%
\CSLRightInline{Nuijten MB, Epskamp S. \emph{statcheck: Extract Statistics from Articles and Recompute P-Values}.; 2024. doi:\href{https://doi.org/10.32614/CRAN.package.statcheck}{10.32614/CRAN.package.statcheck}}

\bibitem[\citeproctext]{ref-Wallisch2022}
\CSLLeftMargin{460. }%
\CSLRightInline{Wallisch C, Bach P, Hafermann L, et al. Review of guidance papers on regression modeling in statistical series of medical journals. Mathes T, org. \emph{PLOS ONE}. 2022;17(1):e0262918. doi:\href{https://doi.org/10.1371/journal.pone.0262918}{10.1371/journal.pone.0262918}}

\bibitem[\citeproctext]{ref-Lynggaard2022}
\CSLLeftMargin{461. }%
\CSLRightInline{Lynggaard H, Bell J, Lösch C, et al. Principles and recommendations for incorporating estimands into clinical study protocol templates. \emph{Trials}. 2022;23(1). doi:\href{https://doi.org/10.1186/s13063-022-06515-2}{10.1186/s13063-022-06515-2}}

\bibitem[\citeproctext]{ref-Althouse2021}
\CSLLeftMargin{462. }%
\CSLRightInline{Althouse AD, Below JE, Claggett BL, et al. Recommendations for Statistical Reporting in Cardiovascular Medicine: A Special Report From the American Heart Association. \emph{Circulation}. 2021;144(4). doi:\href{https://doi.org/10.1161/circulationaha.121.055393}{10.1161/circulationaha.121.055393}}

\bibitem[\citeproctext]{ref-Lee2021a}
\CSLLeftMargin{463. }%
\CSLRightInline{Lee KJ, Tilling KM, Cornish RP, et al. Framework for the treatment and reporting of missing data in observational studies: The Treatment And Reporting of Missing data in Observational Studies framework. \emph{Journal of Clinical Epidemiology}. 2021;134:79--88. doi:\href{https://doi.org/10.1016/j.jclinepi.2021.01.008}{10.1016/j.jclinepi.2021.01.008}}

\bibitem[\citeproctext]{ref-Vickers2020}
\CSLLeftMargin{464. }%
\CSLRightInline{Vickers AJ, Assel MJ, Sjoberg DD, et al. Guidelines for Reporting of Figures and Tables for Clinical Research in Urology. \emph{Urology}. 2020;142:1--13. doi:\href{https://doi.org/10.1016/j.urology.2020.05.002}{10.1016/j.urology.2020.05.002}}

\bibitem[\citeproctext]{ref-assel2019}
\CSLLeftMargin{465. }%
\CSLRightInline{Assel M, Sjoberg D, Elders A, et al. Guidelines for Reporting of Statistics for Clinical Research in Urology. \emph{Journal of Urology}. 2019;201(3):595--604. doi:\href{https://doi.org/10.1097/ju.0000000000000001}{10.1097/ju.0000000000000001}}

\bibitem[\citeproctext]{ref-Lang2015}
\CSLLeftMargin{466. }%
\CSLRightInline{Lang TA, Altman DG. Basic statistical reporting for articles published in Biomedical Journals: The {``}Statistical Analyses and Methods in the Published Literature{''} or the SAMPL Guidelines. \emph{International Journal of Nursing Studies}. 2015;52(1):5--9. doi:\href{https://doi.org/10.1016/j.ijnurstu.2014.09.006}{10.1016/j.ijnurstu.2014.09.006}}

\bibitem[\citeproctext]{ref-Weissgerber2015}
\CSLLeftMargin{467. }%
\CSLRightInline{Weissgerber TL, Milic NM, Winham SJ, Garovic VD. Beyond Bar and Line Graphs: Time for a New Data Presentation Paradigm. \emph{PLOS Biology}. 2015;13(4):e1002128. doi:\href{https://doi.org/10.1371/journal.pbio.1002128}{10.1371/journal.pbio.1002128}}

\bibitem[\citeproctext]{ref-Sauerbrei2014}
\CSLLeftMargin{468. }%
\CSLRightInline{Sauerbrei W, Abrahamowicz M, Altman DG, Cessie S, Carpenter J. STRengthening Analytical Thinking for Observational Studies: the STRATOS initiative. \emph{Statistics in Medicine}. 2014;33(30):5413--5432. doi:\href{https://doi.org/10.1002/sim.6265}{10.1002/sim.6265}}

\bibitem[\citeproctext]{ref-groves2008}
\CSLLeftMargin{469. }%
\CSLRightInline{Groves T. Research methods and reporting. \emph{BMJ}. 2008;337(oct22 1):a2201--a2201. doi:\href{https://doi.org/10.1136/bmj.a2201}{10.1136/bmj.a2201}}

\bibitem[\citeproctext]{ref-stratton2005}
\CSLLeftMargin{470. }%
\CSLRightInline{Stratton IM, Neil A. How to ensure your paper is rejected by the statistical reviewer. \emph{Diabetic Medicine}. 2005;22(4):371--373. doi:\href{https://doi.org/10.1111/j.1464-5491.2004.01443.x}{10.1111/j.1464-5491.2004.01443.x}}

\bibitem[\citeproctext]{ref-Mansournia2021}
\CSLLeftMargin{471. }%
\CSLRightInline{Mansournia MA, Collins GS, Nielsen RO, et al. A CHecklist for statistical Assessment of Medical Papers (the CHAMP statement): explanation and elaboration. \emph{British Journal of Sports Medicine}. 2021;55(18):1009--1017. doi:\href{https://doi.org/10.1136/bjsports-2020-103652}{10.1136/bjsports-2020-103652}}

\bibitem[\citeproctext]{ref-Gil-Sierra2020}
\CSLLeftMargin{472. }%
\CSLRightInline{Gil-Sierra MD, Fénix-Caballero S, Abdel kader-Martin L, et al. Checklist for clinical applicability of subgroup analysis. \emph{Journal of Clinical Pharmacy and Therapeutics}. 2019;45(3):530--538. doi:\href{https://doi.org/10.1111/jcpt.13102}{10.1111/jcpt.13102}}

\bibitem[\citeproctext]{ref-lee2021b}
\CSLLeftMargin{473. }%
\CSLRightInline{Lee H, Cashin AG, Lamb SE, et al. A Guideline for Reporting Mediation Analyses of Randomized Trials and Observational Studies. \emph{JAMA}. 2021;326(11):1045. doi:\href{https://doi.org/10.1001/jama.2021.14075}{10.1001/jama.2021.14075}}

\end{CSLReferences}

\includepdf[pages=-,noautoscale=false, fitpaper=true, width=8.27in, height=11.69in, frame=false, trim=0mm 0mm 0mm 0mm, clip, offset=0mm 0mm]{covers/Cover_4.pdf}

\end{document}
