% Options for packages loaded elsewhere
\PassOptionsToPackage{unicode}{hyperref}
\PassOptionsToPackage{hyphens}{url}
%
\documentclass[
  a4paper,
]{book}
\usepackage{amsmath,amssymb}
\usepackage{iftex}
\ifPDFTeX
  \usepackage[T1]{fontenc}
  \usepackage[utf8]{inputenc}
  \usepackage{textcomp} % provide euro and other symbols
\else % if luatex or xetex
  \usepackage{unicode-math} % this also loads fontspec
  \defaultfontfeatures{Scale=MatchLowercase}
  \defaultfontfeatures[\rmfamily]{Ligatures=TeX,Scale=1}
\fi
\usepackage{lmodern}
\ifPDFTeX\else
  % xetex/luatex font selection
\fi
% Use upquote if available, for straight quotes in verbatim environments
\IfFileExists{upquote.sty}{\usepackage{upquote}}{}
\IfFileExists{microtype.sty}{% use microtype if available
  \usepackage[]{microtype}
  \UseMicrotypeSet[protrusion]{basicmath} % disable protrusion for tt fonts
}{}
\makeatletter
\@ifundefined{KOMAClassName}{% if non-KOMA class
  \IfFileExists{parskip.sty}{%
    \usepackage{parskip}
  }{% else
    \setlength{\parindent}{0pt}
    \setlength{\parskip}{6pt plus 2pt minus 1pt}}
}{% if KOMA class
  \KOMAoptions{parskip=half}}
\makeatother
\usepackage{xcolor}
\usepackage[top=3cm,left=3cm,right=2cm,bottom=2cm]{geometry}
\usepackage{color}
\usepackage{fancyvrb}
\newcommand{\VerbBar}{|}
\newcommand{\VERB}{\Verb[commandchars=\\\{\}]}
\DefineVerbatimEnvironment{Highlighting}{Verbatim}{commandchars=\\\{\}}
% Add ',fontsize=\small' for more characters per line
\usepackage{framed}
\definecolor{shadecolor}{RGB}{248,248,248}
\newenvironment{Shaded}{\begin{snugshade}}{\end{snugshade}}
\newcommand{\AlertTok}[1]{\textcolor[rgb]{0.94,0.16,0.16}{#1}}
\newcommand{\AnnotationTok}[1]{\textcolor[rgb]{0.56,0.35,0.01}{\textbf{\textit{#1}}}}
\newcommand{\AttributeTok}[1]{\textcolor[rgb]{0.13,0.29,0.53}{#1}}
\newcommand{\BaseNTok}[1]{\textcolor[rgb]{0.00,0.00,0.81}{#1}}
\newcommand{\BuiltInTok}[1]{#1}
\newcommand{\CharTok}[1]{\textcolor[rgb]{0.31,0.60,0.02}{#1}}
\newcommand{\CommentTok}[1]{\textcolor[rgb]{0.56,0.35,0.01}{\textit{#1}}}
\newcommand{\CommentVarTok}[1]{\textcolor[rgb]{0.56,0.35,0.01}{\textbf{\textit{#1}}}}
\newcommand{\ConstantTok}[1]{\textcolor[rgb]{0.56,0.35,0.01}{#1}}
\newcommand{\ControlFlowTok}[1]{\textcolor[rgb]{0.13,0.29,0.53}{\textbf{#1}}}
\newcommand{\DataTypeTok}[1]{\textcolor[rgb]{0.13,0.29,0.53}{#1}}
\newcommand{\DecValTok}[1]{\textcolor[rgb]{0.00,0.00,0.81}{#1}}
\newcommand{\DocumentationTok}[1]{\textcolor[rgb]{0.56,0.35,0.01}{\textbf{\textit{#1}}}}
\newcommand{\ErrorTok}[1]{\textcolor[rgb]{0.64,0.00,0.00}{\textbf{#1}}}
\newcommand{\ExtensionTok}[1]{#1}
\newcommand{\FloatTok}[1]{\textcolor[rgb]{0.00,0.00,0.81}{#1}}
\newcommand{\FunctionTok}[1]{\textcolor[rgb]{0.13,0.29,0.53}{\textbf{#1}}}
\newcommand{\ImportTok}[1]{#1}
\newcommand{\InformationTok}[1]{\textcolor[rgb]{0.56,0.35,0.01}{\textbf{\textit{#1}}}}
\newcommand{\KeywordTok}[1]{\textcolor[rgb]{0.13,0.29,0.53}{\textbf{#1}}}
\newcommand{\NormalTok}[1]{#1}
\newcommand{\OperatorTok}[1]{\textcolor[rgb]{0.81,0.36,0.00}{\textbf{#1}}}
\newcommand{\OtherTok}[1]{\textcolor[rgb]{0.56,0.35,0.01}{#1}}
\newcommand{\PreprocessorTok}[1]{\textcolor[rgb]{0.56,0.35,0.01}{\textit{#1}}}
\newcommand{\RegionMarkerTok}[1]{#1}
\newcommand{\SpecialCharTok}[1]{\textcolor[rgb]{0.81,0.36,0.00}{\textbf{#1}}}
\newcommand{\SpecialStringTok}[1]{\textcolor[rgb]{0.31,0.60,0.02}{#1}}
\newcommand{\StringTok}[1]{\textcolor[rgb]{0.31,0.60,0.02}{#1}}
\newcommand{\VariableTok}[1]{\textcolor[rgb]{0.00,0.00,0.00}{#1}}
\newcommand{\VerbatimStringTok}[1]{\textcolor[rgb]{0.31,0.60,0.02}{#1}}
\newcommand{\WarningTok}[1]{\textcolor[rgb]{0.56,0.35,0.01}{\textbf{\textit{#1}}}}
\usepackage{longtable,booktabs,array}
\usepackage{calc} % for calculating minipage widths
% Correct order of tables after \paragraph or \subparagraph
\usepackage{etoolbox}
\makeatletter
\patchcmd\longtable{\par}{\if@noskipsec\mbox{}\fi\par}{}{}
\makeatother
% Allow footnotes in longtable head/foot
\IfFileExists{footnotehyper.sty}{\usepackage{footnotehyper}}{\usepackage{footnote}}
\makesavenoteenv{longtable}
\usepackage{graphicx}
\makeatletter
\def\maxwidth{\ifdim\Gin@nat@width>\linewidth\linewidth\else\Gin@nat@width\fi}
\def\maxheight{\ifdim\Gin@nat@height>\textheight\textheight\else\Gin@nat@height\fi}
\makeatother
% Scale images if necessary, so that they will not overflow the page
% margins by default, and it is still possible to overwrite the defaults
% using explicit options in \includegraphics[width, height, ...]{}
\setkeys{Gin}{width=\maxwidth,height=\maxheight,keepaspectratio}
% Set default figure placement to htbp
\makeatletter
\def\fps@figure{htbp}
\makeatother
\setlength{\emergencystretch}{3em} % prevent overfull lines
\providecommand{\tightlist}{%
  \setlength{\itemsep}{0pt}\setlength{\parskip}{0pt}}
\setcounter{secnumdepth}{5}
\newlength{\cslhangindent}
\setlength{\cslhangindent}{1.5em}
\newlength{\csllabelwidth}
\setlength{\csllabelwidth}{3em}
\newlength{\cslentryspacingunit} % times entry-spacing
\setlength{\cslentryspacingunit}{\parskip}
\newenvironment{CSLReferences}[2] % #1 hanging-ident, #2 entry spacing
 {% don't indent paragraphs
  \setlength{\parindent}{0pt}
  % turn on hanging indent if param 1 is 1
  \ifodd #1
  \let\oldpar\par
  \def\par{\hangindent=\cslhangindent\oldpar}
  \fi
  % set entry spacing
  \setlength{\parskip}{#2\cslentryspacingunit}
 }%
 {}
\usepackage{calc}
\newcommand{\CSLBlock}[1]{#1\hfill\break}
\newcommand{\CSLLeftMargin}[1]{\parbox[t]{\csllabelwidth}{#1}}
\newcommand{\CSLRightInline}[1]{\parbox[t]{\linewidth - \csllabelwidth}{#1}\break}
\newcommand{\CSLIndent}[1]{\hspace{\cslhangindent}#1}
% to use Portuguese-Brazil
\usepackage[brazilian]{babel}

% marca d'agua
\usepackage{draftwatermark}

% icons
\usepackage{academicons}

% fontawesome icons
\usepackage{fontawesome5}

% \usepackage{fontspec}
\usepackage[T1]{fontenc}
\usepackage[utf8]{inputenc}
\usepackage{lmodern}
\setmainfont{Times New Roman}

% to display and enumerate math functions
\usepackage{amsmath}

% figuras e tabelas
\usepackage{float}
% tables
\usepackage{booktabs}
% long tables across pages
\usepackage{longtable}
% include graphics
\usepackage{graphicx}
% work with SVG files
\usepackage[notransparent]{svg}

% page size and margin
\usepackage{geometry}

% create commands
\usepackage{letltxmacro}

% get PDF pages
\usepackage{pdfpages}

% using colors
\usepackage{xcolor}
% colored boxes
\usepackage{tcolorbox}

% add citation to the bibliography 
\usepackage{tocbibind}
% titles of toc, lof, lot
\usepackage[titles]{tocloft}

% hyperlinks
\usepackage[spaces,obeyspaces,hyphens]{url}
\usepackage[linktoc=all]{hyperref}

% commands
\makeatletter
\g@addto@macro{\UrlBreaks}{\UrlOrds}
\makeatother

\SetWatermarkLightness{0.9}
\SetWatermarkText{RASCUNHO}
\SetWatermarkScale{0.9}

\AtBeginDocument{\let\maketitle\relax}

\renewcommand{\href}[2]{#2\footnote{\url{#1}}}

\definecolor{lightgray}{gray}{0.95}

\newtcolorbox{blackbox}{
  colback=lightgray,
  colframe=orange,
  coltext=black,
  boxsep=5pt,
  arc=4pt}
\newenvironment{infobox}[1]
  {
  \begin{itemize}
  \renewcommand{\labelitemi}{
    \raisebox{-.7\height}[0pt][0pt]{
      {\setkeys{Gin}{width=3em,keepaspectratio}
        \includegraphics{#1}}
    }
  }
  \setlength{\fboxsep}{1em}
  \begin{blackbox}
  \item
  }
  {
  \end{blackbox}
  \end{itemize}
  }

\LetLtxMacro\Oldfootnote\footnote
\newcommand{\EnableFootNotes}{%
  \LetLtxMacro\footnote\Oldfootnote%
}
\newcommand{\DisableFootNotes}{%
  \renewcommand{\footnote}[2][]{\relax}
}

% unnumbered chapters and roman page numbering
\frontmatter
\usepackage{booktabs}
\usepackage{longtable}
\usepackage{array}
\usepackage{multirow}
\usepackage{wrapfig}
\usepackage{float}
\usepackage{colortbl}
\usepackage{pdflscape}
\usepackage{tabu}
\usepackage{threeparttable}
\usepackage{threeparttablex}
\usepackage[normalem]{ulem}
\usepackage{makecell}
\usepackage{xcolor}
\usepackage{caption}
\usepackage{graphicx}
\usepackage{siunitx}
\usepackage{hhline}
\usepackage{calc}
\usepackage{tabularx}
\usepackage{adjustbox}
\usepackage{hyperref}
\ifLuaTeX
  \usepackage{selnolig}  % disable illegal ligatures
\fi
\IfFileExists{bookmark.sty}{\usepackage{bookmark}}{\usepackage{hyperref}}
\IfFileExists{xurl.sty}{\usepackage{xurl}}{} % add URL line breaks if available
\urlstyle{same}
\hypersetup{
  pdftitle={Ciência com R},
  pdfauthor={© 2023 by Arthur de Sá Ferreira   https://orcid.org/0000-0001-7014-2002 },
  hidelinks,
  pdfcreator={LaTeX via pandoc}}

\title{\textbf{Ciência com R}}
\usepackage{etoolbox}
\makeatletter
\providecommand{\subtitle}[1]{% add subtitle to \maketitle
  \apptocmd{\@title}{\par {\large #1 \par}}{}{}
}
\makeatother
\subtitle{Perguntas e respostas para pesquisadores e analistas de dados}
\author{© 2023 by Arthur de Sá Ferreira https://orcid.org/0000-0001-7014-2002}
\date{}

\begin{document}
\maketitle

\includepdf[pages={1},noautoscale=false, fitpaper=true, width=8.27in, height=11.69in, frame=false, trim=0mm 0mm 0mm 0mm, clip, offset=0mm 0mm]{covers/Cover_1.pdf}
\newpage

\includepdf[pages={1},noautoscale=false, fitpaper=true, width=8.27in, height=11.69in, frame=false, trim=0mm 0mm 0mm 0mm, clip, offset=0mm 0mm]{covers/Cover_2.pdf}
\newpage

{
\setcounter{tocdepth}{1}
\tableofcontents
}
\listoffigures
\listoftables
\DisableFootNotes

\clearpage
\markboth{}{}

Ferreira, Arthur de Sá. \textbf{Ciência com R: Perguntas e respostas para pesquisadores e analistas de dados}. Rio de Janeiro: 1a edição, 2023. 165p. \href{https://doi.org/10.5281/zenodo.8320233}{doi: 10.5281/zenodo.8320233}.

\vspace*{\fill}

Copyright © 2023 Arthur de Sá Ferreira

Todos os direitos reservados. Nenhuma parte deste livro pode ser reproduzida ou usada de qualquer maneira sem a permissão prévia por escrito do proprietário dos direitos autorais, exceto para o uso de breves citações em uma resenha do livro.

Para solicitar permissões, entre em contato com \href{mailto:cienciacomr@gmail.com}{\nolinkurl{cienciacomr@gmail.com}}

Capa dura: ISBN

Brochura: ISBN

E-book: ISBN

\hypertarget{sobre-o-autor}{%
\chapter*{Sobre o autor}\label{sobre-o-autor}}
\addcontentsline{toc}{chapter}{Sobre o autor}

\markboth{}{}

\includegraphics{images/ASF.png}

\textbf{Arthur de Sá Ferreira, DSc}

Obtive minha Graduação em Fisioterapia pela Universidade Federal do Rio de Janeiro (UFRJ, 1999), Formação em Acupuntura pela Academia Brasileira de Arte e Ciência Oriental (ABACO, 2001), Mestrado em Engenharia Biomédica pela Universidade Federal do Rio de Janeiro (UFRJ, 2002) e Doutorado em Engenharia Biomédica pela Universidade Federal do Rio de Janeiro (UFRJ, 2006).

Tenho experiência em docência no ensino superior, atuando com professor da graduação em cursos de Fisioterapia, Enfermagem e Odontologia, entre outros (2001-atual); pós-graduação lato sensu em Fisioterapia (2001-atual) e stricto sensu em Ciências da Reabilitação (2010-atual).

Sou professor adjunto do Centro Universitário Augusto Motta (\href{https://www.unisuam.edu.br}{UNISUAM}), pesquisador dos Programas de Pós-graduação em Ciências da Reabilitação (\href{https://www.unisuam.edu.br/programa-pos-graduacao-ciencias-da-reabilitacao}{PPGCR}) e Desenvolvimento Local (\href{https://www.unisuam.edu.br/programa-pos-graduacao-desenvolvimento-local}{PPGDL}) e Coordenador do Comitê de Ética em Pesquisa (CEP) (2020-atual). Leciono as disciplinas Bioestatística I e II desde 2010 nesses Programas.

Fundei o Laboratório de Simulação Computacional e Modelagem em Reabilitação (LSCMR) em 2012, onde desenvolvo projetos de pesquisa principalmente nos seguintes temas: Bioestatística, Modelagem e simulação computacional, Processamento de sinais biomédicos, Movimento funcional humano, Medicina tradicional (chinesa), Distúrbios musculoesqueléticos, Doenças cardiovasculares e Doenças respiratórias.

Sou membro efetivo da Associação Brasileira de Pesquisa e Pós-Graduação em Fisioterapia (\href{https://abrapg-ft.org.br/portal/}{ABRAPG-FT}) (2007-atual), Consórcio Acadêmico Brasileiro de Saúde Integrativa (\href{https://cabsin.org.br}{CABSIN}) (2019-atual), Committee on Publication Ethics (\href{https://publicationethics.org}{COPE}) (2018-atual) e Royal Statistical Society (\href{https://rss.org.uk}{RSS}) (2021-atual).

Componho o corpo editorial e de revisores de periódicos nacionais e internacionais como \href{https://www.nature.com/srep/about/editors}{Scientific Reports}, \href{https://www.frontiersin.org/research-topics/26395/systemic-effects-and-disabilities-in-long-covid-syndrome-current-approaches-and-clinical-challenges}{Frontiers in Rehabilitation Sciences}, \href{https://www.hindawi.com/journals/ecam/editors/}{Evidence-Based Complementary and Alternative Medicine}, \href{https://www.springer.com/journal/11655/editors}{Chinese Journal of Integrative Medicine}, \href{https://www.journals.elsevier.com/journal-of-integrative-medicine/editorial-board}{Journal of Integrative Medicine}, \href{https://www.scielo.br/journal/fp/about/\#editors}{Fisioterapia e Pesquisa}.

\textbf{Currículos externos}

\href{http://lattes.cnpq.br/5432142731317894}{~5432142731317894}

0000-0001-7014-2002

\href{https://publons.com/researcher/F-6831-2012}{~F-6831-2012}

\hypertarget{dedicatuxf3ria}{%
\chapter*{Dedicatória}\label{dedicatuxf3ria}}
\addcontentsline{toc}{chapter}{Dedicatória}

\markboth{}{}

Esta obra é dedicada a todos que, em princípio, buscam conhecimento para melhorar a qualidade da pesquisa científica - seja a sua própria, a de colegas ou a de desconhecidos - mas, em última análise, desejam mesmo prover melhores condições de saúde e desenvolvimento da sociedade.

Dedico também ao leitor eventual que chegou aqui por acaso.

\hypertarget{agradecimentos}{%
\chapter*{Agradecimentos}\label{agradecimentos}}
\addcontentsline{toc}{chapter}{Agradecimentos}

\markboth{}{}

Este trabalho não seria possível sem o apoio e suporte da minha esposa Daniele, minha irmã Mônica, meu pai José Victorino e meus filhos Giovanna, Victor e Lucas.

\hypertarget{prefuxe1cio}{%
\chapter*{Prefácio}\label{prefuxe1cio}}
\addcontentsline{toc}{chapter}{Prefácio}

\markboth{}{}

No âmbito da análise estatística de dados, os processos envolvidos são marcados por uma série de escolhas críticas. Estas decisões abrangem considerações metodológicas e ações operacionais que moldam toda a jornada analítica. Deve-se selecionar, cuidadosamente, um delineamento de estudo para enfrentar os desafios únicos colocados por um projeto de pesquisa. Além disso, a escolha de métodos estatísticos adequados para lidar com os dados gerados pelo delineamento escolhido tem um peso importante. Estas decisões necessitam de uma base construída sobre as evidências mais convincentes da literatura existente e na adesão a práticas sólidas de investigação.

Interpretar os resultados destas análises não é uma tarefa simples. Confiar apenas na formação educacional convencional, no bom senso e na intuição para decifrar tabelas e gráficos pode revelar-se inadequado. Interpretações errôneas podem gerar consequências indesejáveis, incluindo a utilização de testes diagnósticos imprecisos ou o endosso de tratamentos ineficazes.

Este livro emerge do reconhecimento desses desafios.

A proposta gira em torno da organização de um compêndio abrangente de métodos e técnicas de ponta, para análise estatística de dados em pesquisa científica, apresentados em formato de perguntas e respostas. Esse formato promove um diálogo direto e objetivo com o leitor, respondendo a dúvidas comumente colocadas por alunos de graduação, pós-graduação, mestrado e doutorado, bem como por pesquisadores.

O objetivo geral de cada capítulo é elucidar as questões metodológicas fundamentais: \emph{``O que é?''}, \emph{``Por que usar?''}, \emph{``Quando usar?''}, \emph{``Quando não usar?''} e \emph{``Como fazer?''}. Em cada capítulo, diversas questões específicas são propostas e respondidas sistematicamente, permitindo ao leitor uma melhor elaboração do conteúdo e resultado do seu trabalho.

Os capítulos foram organizados para seguir uma progressão de conceitos e aplicações. Embora sejam fragmentados para maior clareza instrucional, as referências cruzadas ajudam a mitigar a fragmentação do conteúdo e reforçar a interconexão dos tópicos.

O público-alvo compreende pesquisadores, professores, analistas de dados, profissionais e estudantes que regularmente lidam com a tomada de decisões em pesquisa. Os estudantes de pós-graduação encontrarão aqui uma obra repleta de exemplos para adaptar na análise dos dados de seus projetos de pesquisa. Professores de graduação e pós-graduação terão acesso a uma obra didática de referência, direcionada para seus alunos. Pesquisadores e analistas de dados iniciantes descobrirão um valioso acervo de informações e referências para a construção de projetos e manuscritos. Pesquisadores e os cientistas mais experientes podem recorrer às referências e esclarecimentos mais atuais sobre vieses, paradoxos, mitos e mal práticas em pesquisa. E mesmo os leitores não familiarizados ainda com as técnicas de análise de dados em pesquisa terão a oportunidade de apreciar o papel fundamental de colocar e responder suas perguntas na busca do conhecimento científico.

Arthur de Sá Ferreira, DSc

% normal chapter numbering and arabic page numbering
\mainmatter

\cftaddtitleline{toc}{chapter}{\rule{\textwidth}{0.4pt}}{}

\hypertarget{parte-1---pensamento-cientuxedfico}{%
\chapter*{\texorpdfstring{\emph{Parte 1 - Pensamento Científico}}{Parte 1 - Pensamento Científico}}\label{parte-1---pensamento-cientuxedfico}}
\addcontentsline{toc}{chapter}{\emph{Parte 1 - Pensamento Científico}}

\markboth{}{}
\par\noindent\rule{\textwidth}{0.05in}

\EnableFootNotes

\hypertarget{computacao-ciencia}{%
\chapter{\texorpdfstring{\textbf{Computação e Ciência}}{Computação e Ciência}}\label{computacao-ciencia}}

\hypertarget{computacao-estatistica}{%
\section{Computação estatística}\label{computacao-estatistica}}

\hypertarget{o-que-uxe9-r}{%
\subsection{O que é R?}\label{o-que-uxe9-r}}

\begin{itemize}
\item
  R é um programa de computador com linguagem computacional direcionada para análise estatística.\textsuperscript{\protect\hyperlink{ref-ihaka1996}{1}}
\item
  R está disponível em \href{(https://cran.r-project.org)}{Comprehensive R Archive Network (CRAN)}.
\item
  R version 4.3.2 (2023-10-31).
\end{itemize}

\hypertarget{o-que-uxe9-rstudio}{%
\subsection{O que é RStudio?}\label{o-que-uxe9-rstudio}}

\begin{itemize}
\item
  RStudio é um ambiente de desenvolvimento integrado (\emph{integrated development environment}, IDE) desenvolvido visando a reprodutibilidade e a simplicidade para a criação e disseminação de conhecimento.\textsuperscript{\protect\hyperlink{ref-racine2011}{2}}
\item
  As principais características do RStudio incluem um ambiente de edição com abas para acesso rápido a arquivos, comandos e resultados; histórico de comandos previamente utilizados; ferramentas para visualização de bancos de dados e elaboração de scripts e gráficos e tabelas.\textsuperscript{\protect\hyperlink{ref-racine2011}{2}}
\item
  RStudio está disponível em \href{https://posit.co/download/rstudio-desktop/}{Posit}.
\end{itemize}

\hypertarget{por-que-usar-r}{%
\subsection{Por que usar R?}\label{por-que-usar-r}}

\begin{itemize}
\tightlist
\item
  R é o software de maior abrangência de métodos estatísticos, possui sintaxe que permite análises estatísticas reproduzíveis e está disponível gratuitamente no website CRAN (\url{http://cran.r-project.org/}).\textsuperscript{\protect\hyperlink{ref-mair2016}{3}}
\end{itemize}

\hypertarget{que-programas-de-computador-podem-ser-usados-para-anuxe1lise-estatuxedstica-com-r}{%
\subsection{Que programas de computador podem ser usados para análise estatística com R?}\label{que-programas-de-computador-podem-ser-usados-para-anuxe1lise-estatuxedstica-com-r}}

\begin{itemize}
\item
  \href{https://jasp-stats.org}{JASP}.\textsuperscript{\protect\hyperlink{ref-love2019}{4}}
\item
  \href{https://www.jamovi.org}{jamovi}.\textsuperscript{\protect\hyperlink{ref-sahin2020}{5}}
\item
  \href{https://www.blueskystatistics.com}{BlueSky}.
\end{itemize}

\begin{infobox}{images/Rlogo}
Os pacotes \emph{jmv}\textsuperscript{\protect\hyperlink{ref-jmv}{6}} e \emph{jmvconnect}\textsuperscript{\protect\hyperlink{ref-jmvconnect}{7}} fornecem funções para análise descritiva e inferencial com interface com jamovi.

\end{infobox}

\hypertarget{scripts-computacionais}{%
\section{Scripts computacionais}\label{scripts-computacionais}}

\hypertarget{o-que-suxe3o-scripts}{%
\subsection{O que são scripts?}\label{o-que-suxe3o-scripts}}

\begin{itemize}
\item
  ``Scripts são dados''.\textsuperscript{\protect\hyperlink{ref-hinsen2011}{8}}
\item
  Scripts permitem ao usuário se concentrar nas tarefas mais importantes da computação e utilizar pacotes ou bibliotecas para executar as funções mais básicas com maior eficiência.\textsuperscript{\protect\hyperlink{ref-hinsen2011}{8}}
\item
  Um script é um arquivo de texto contendo (quase) os mesmos comandos que você digitaria na linha de comando do R. (Quase) refere-se ao fato de que se você estiver usando \emph{sink()} para enviar a saída para um arquivo, você terá que incluir alguns comandos em \emph{print()} para obter a mesma saída da linha de comando (\href{https://cran.r-project.org/doc/contrib/Lemon-kickstart/kr_scrpt.html}{CRAN}).
\end{itemize}

\hypertarget{quais-pruxe1ticas-suxe3o-recomendadas-na-redauxe7uxe3o-de-scripts}{%
\subsection{Quais práticas são recomendadas na redação de scripts?}\label{quais-pruxe1ticas-suxe3o-recomendadas-na-redauxe7uxe3o-de-scripts}}

\begin{itemize}
\item
  Use nomes consistentes para as variáveis.\textsuperscript{\protect\hyperlink{ref-SchwabSimon2021}{9}}
\item
  Defina os tipos de variáveis adequadamente no banco de dados.\textsuperscript{\protect\hyperlink{ref-SchwabSimon2021}{9}}
\item
  Defina constantes - isto é, variáveis de valor fixo - ao invés de digitar valores.\textsuperscript{\protect\hyperlink{ref-SchwabSimon2021}{9}}
\item
  Use e cite os pacotes disponíveis para suas análises.\textsuperscript{\protect\hyperlink{ref-SchwabSimon2021}{9}}
\item
  Controle as versões do script.\textsuperscript{\protect\hyperlink{ref-SchwabSimon2021}{9},\protect\hyperlink{ref-Eglen2017}{10}}
\item
  Teste o script antes de sua utilização.\textsuperscript{\protect\hyperlink{ref-SchwabSimon2021}{9}}
\item
  Conduza revisão por pares do código durante a redação (digitação em dupla).\textsuperscript{\protect\hyperlink{ref-SchwabSimon2021}{9}}
\end{itemize}

\begin{infobox}{images/Rlogo}
O pacote \emph{formatR}\textsuperscript{\protect\hyperlink{ref-formatR}{11}} fornece a função \href{https://www.rdocumentation.org/packages/formatR/versions/1.14/topics/tidy_source}{\emph{tidy\_source}} para formatar um R script.

\end{infobox}

\begin{infobox}{images/Rlogo}
O pacote \emph{styler}\textsuperscript{\protect\hyperlink{ref-styler}{12}} fornece a função \href{https://www.rdocumentation.org/packages/styler/versions/1.10.1/topics/style_file}{\emph{style\_file}} para formatar um R script.

\end{infobox}

\hypertarget{relatuxf3rios-dinamicos}{%
\section{Manuscritos reprodutíveis}\label{relatuxf3rios-dinamicos}}

\hypertarget{o-que-suxe3o-manuscritos-reprodutuxedveis}{%
\subsection{O que são manuscritos reprodutíveis?}\label{o-que-suxe3o-manuscritos-reprodutuxedveis}}

\begin{itemize}
\item
  Manuscritos reprodutíveis - manuscritos executáveis ou relatórios dinâmicos - permitem a produção de um manuscrito completo a partir da integração do banco de dados da(s) amostra(s), do(s) script(s) de análise estatística (incluindo comentários para sua interpretação), dos pacotes ou bibliotecas utilizados, das fontes e referências bibliográficas citadas, além dos demais elementos textuais (tabelas, gráficos) - todos gerados dinamicamente.\textsuperscript{\protect\hyperlink{ref-hinsen2011}{8}}
\item
  O trabalho com RMarkdown\textsuperscript{\protect\hyperlink{ref-R-rmarkdown}{13}} permite um fluxo de dados totalmente transparente, desde o conjunto de dados coletados até o manuscrito finalizado. Todos os aspectos do fluxo de dados podem ser incorporados em blocos de R script (\emph{chunk}), exibindo tanto o R script quando o respectivo texto, tabelas e figuras formatadas no estilo científico de interesse.\textsuperscript{\protect\hyperlink{ref-holmes2021}{14}}
\item
  O RMarkdown\textsuperscript{\protect\hyperlink{ref-R-rmarkdown}{13}} foi projetado especificamente para relatórios dinâmicos onde a análise é realizada em R e oferece uma flexibilidade incrível por meio de uma linguagem de marcação.\textsuperscript{\protect\hyperlink{ref-mair2016}{3}}
\end{itemize}

\hypertarget{por-que-usar-manuscritos-reprodutuxedveis}{%
\subsection{Por que usar manuscritos reprodutíveis?}\label{por-que-usar-manuscritos-reprodutuxedveis}}

\begin{itemize}
\item
  No processo tradicional de redação científica há muitas etapas de copiar e colar não reproduzíveis envolvidas. Documentos dinâmicos combinam uma ferramenta de processamento de texto com o R script que produz o texto/tabela/figura a ser incorporado no manuscrito.\textsuperscript{\protect\hyperlink{ref-mair2016}{3}}
\item
  Ao trabalhar com relatórios dinâmicos, é possível extrair o mesmo script usado para análise estatística. Os documentos podem ser compilados em vários formatos de saída e salvos como DOCX, PPTX e PDF.\textsuperscript{\protect\hyperlink{ref-mair2016}{3}}
\end{itemize}

\begin{infobox}{images/Rlogo}
O pacote \emph{rmarkdown}\textsuperscript{\protect\hyperlink{ref-rmarkdown}{15}} fornece as funções {[}\emph{render}{]}\url{https://www.rdocumentation.org/packages/rmarkdown/versions/2.24/topics/render}) para criar manuscritos reprodutíveis a partir de arquivos .Rmd.

\end{infobox}

\begin{infobox}{images/Rlogo}
O pacote \emph{officedown}\textsuperscript{\protect\hyperlink{ref-officedown}{16}} fornece as funções \href{https://www.rdocumentation.org/packages/officedown/versions/0.3.0/topics/rdocx_document}{\emph{rdocx\_document}} e \href{https://www.rdocumentation.org/packages/officedown/versions/0.3.0/topics/rpptx_document}{\emph{rpptx\_document}} para criar arquivos DOCX e PPTX, respectivamente, com o conteúdo criado no manuscrito reprodutível.

\end{infobox}

\begin{infobox}{images/Rlogo}
O pacote \emph{bookdown}\textsuperscript{\protect\hyperlink{ref-bookdown}{17}} fornece as funções \href{https://www.rdocumentation.org/packages/bookdown/versions/0.35/topics/pdf_book}{\emph{pdf\_book}}, \href{https://www.rdocumentation.org/packages/bookdown/versions/0.35/topics/bs4_book}{\emph{bs4\_book}} e \href{https://www.rdocumentation.org/packages/bookdown/versions/0.35/topics/epub_book}{\emph{epub\_book}} para criar arquivos PDF, HTML e EPUB, respectivamente, com o conteúdo criado no manuscrito reprodutível.

\end{infobox}

\hypertarget{como-manuscritos-reprodutuxedveis-contribuem-para-a-ciuxeancia}{%
\subsection{Como manuscritos reprodutíveis contribuem para a ciência?}\label{como-manuscritos-reprodutuxedveis-contribuem-para-a-ciuxeancia}}

\begin{itemize}
\tightlist
\item
  O compartilhamento de bancos de dados e seus scripts de análise estatística permitem a adoção de práticas reprodutíveis, tais como a reanálise dos dados.\textsuperscript{\protect\hyperlink{ref-ioannidis2014}{18}}
\end{itemize}

\begin{infobox}{images/Rlogo}
O pacote \emph{projects}\textsuperscript{\protect\hyperlink{ref-projects}{19}} fornece a função \href{https://www.rdocumentation.org/packages/projects/versions/2.1.3/topics/setup_projects}{\emph{setup\_projects}} para criar um projeto com arquivos organizados em diretórios.

\end{infobox}

\hypertarget{pensamento-probabilistico}{%
\chapter{\texorpdfstring{\textbf{Pensamento probabilístico}}{Pensamento probabilístico}}\label{pensamento-probabilistico}}

\hypertarget{experimento}{%
\section{Experimento}\label{experimento}}

\hypertarget{o-que-uxe9-um-experimento}{%
\subsection{O que é um experimento?}\label{o-que-uxe9-um-experimento}}

\begin{itemize}
\item
  Um experimento é um processo de medição ou simulação cujo resultado é chamado de desfecho.\textsuperscript{\protect\hyperlink{ref-grami2023}{20}}
\item
  Em um experimento aleatório, o desfecho é imprevisível.\textsuperscript{\protect\hyperlink{ref-grami2023}{20}}
\item
  Tentativa se refere a uma repetição de um experimento aleatório.\textsuperscript{\protect\hyperlink{ref-grami2023}{20}}
\end{itemize}

\hypertarget{espacos-eventos-discretos}{%
\section{Espaços amostrais e eventos discretos}\label{espacos-eventos-discretos}}

\hypertarget{o-que-uxe9-espauxe7o-amostral-discreto}{%
\subsection{O que é espaço amostral discreto?}\label{o-que-uxe9-espauxe7o-amostral-discreto}}

\begin{itemize}
\item
  O espaço amostral \(S\) de um experimento aleatório é definido como o conjunto de todos os desfechos possíveis de um experimento.\textsuperscript{\protect\hyperlink{ref-grami2023}{20}}
\item
  Em probabilidade discreta, o espaço amostral \(S\) pode ser enumerado e contato.\textsuperscript{\protect\hyperlink{ref-grami2023}{20}}
\end{itemize}

\begin{figure}

{\centering \includegraphics{Ciencia-com-R_files/figure-latex/espaco-amostral-dado-1} 

}

\caption{Espaço amostral: Todas as faces de um dado.}\label{fig:espaco-amostral-dado}
\end{figure}

\hypertarget{o-que-uxe9-evento-discreto}{%
\subsection{O que é evento discreto?}\label{o-que-uxe9-evento-discreto}}

\begin{itemize}
\item
  Um evento \(E\) é um único desfecho ou uma coleção de desfechos.\textsuperscript{\protect\hyperlink{ref-grami2023}{20}}
\item
  Um evento \(E\) é um subconjunto do espaço amostral \(S\) de um experimento.\textsuperscript{\protect\hyperlink{ref-grami2023}{20}}
\end{itemize}

\begin{figure}

{\centering \includegraphics{Ciencia-com-R_files/figure-latex/evento-dado-1} 

}

\caption{Exemplo de evento do experimento de 1 lançamento de 1 dado.}\label{fig:evento-dado}
\end{figure}

\hypertarget{o-que-uxe9-espauxe7o-de-eventos-discretos}{%
\subsection{O que é espaço de eventos discretos?}\label{o-que-uxe9-espauxe7o-de-eventos-discretos}}

\begin{itemize}
\item
  Um espaço de eventos \(E_{s}\) também é um subconjunto do espaço amostral \(S\) de um experimento.\textsuperscript{\protect\hyperlink{ref-grami2023}{20}}
\item
  A união de dois eventos \(E_{1} \cup E_{2}\) é o conjunto de todos os desfechos que estão em ambos.\textsuperscript{\protect\hyperlink{ref-grami2023}{20}}
\item
  A intersecção de dois eventos \(E_{1} \cap E_{2}\), ou evento conjunto, é o conjunto de todos os desfechos que estão em ambos os eventos.\textsuperscript{\protect\hyperlink{ref-grami2023}{20}}
\item
  O complemento de um evento \(E^C\) consiste em todos os desfechos que não estão incluídos no evento \(E\).\textsuperscript{\protect\hyperlink{ref-grami2023}{20}}
\end{itemize}

\begin{figure}

{\centering \includegraphics{Ciencia-com-R_files/figure-latex/espaco-eventos-dado-1} 

}

\caption{Espaço de eventos: União dos eventos face = 3 e face = 4 de um dado.}\label{fig:espaco-eventos-dado}
\end{figure}

\hypertarget{espacos-eventos-continuos}{%
\section{Espaços amostrais e eventos contínuos}\label{espacos-eventos-continuos}}

\hypertarget{o-que-uxe9-espauxe7o-amostral-contuxednuo}{%
\subsection{O que é espaço amostral contínuo?}\label{o-que-uxe9-espauxe7o-amostral-contuxednuo}}

\begin{itemize}
\tightlist
\item
  .{[}REF{]}
\end{itemize}

\hypertarget{o-que-uxe9-evento-contuxednuo}{%
\subsection{O que é evento contínuo?}\label{o-que-uxe9-evento-contuxednuo}}

\begin{itemize}
\tightlist
\item
  .{[}REF{]}
\end{itemize}

\hypertarget{o-que-uxe9-espauxe7o-de-eventos-contuxednuo}{%
\subsection{O que é espaço de eventos contínuo?}\label{o-que-uxe9-espauxe7o-de-eventos-contuxednuo}}

\begin{itemize}
\tightlist
\item
  .{[}REF{]}
\end{itemize}

\hypertarget{probabilidade}{%
\section{Probabilidade}\label{probabilidade}}

\hypertarget{o-que-uxe9-probabilidade}{%
\subsection{O que é probabilidade?}\label{o-que-uxe9-probabilidade}}

\begin{itemize}
\item
  Com um espaço amostral \(S\) finito e não vazio de desfechos igualmente prováveis, a probabilidade de um evento \(E\) é a razão entre o número de desfechos no evento \(E\) e o número de desfechos no espaço amostral \(S\).\textsuperscript{\protect\hyperlink{ref-grami2023}{20}}
\item
  Um evento \(E\) impossível não contém um desfecho e, portanto, nunca ocorre: \(P(E)=0\).\textsuperscript{\protect\hyperlink{ref-grami2023}{20}}
\item
  Um evento \(E\) é certo consiste em qualquer um dos desfechos possíveis e, portanto, sempre ocorre: \(P(E)=1\).\textsuperscript{\protect\hyperlink{ref-grami2023}{20}}
\end{itemize}

\hypertarget{quais-suxe3o-os-axiomas-da-probabilidade}{%
\subsection{Quais são os axiomas da probabilidade?}\label{quais-suxe3o-os-axiomas-da-probabilidade}}

\begin{itemize}
\item
  A probabilidade de um evento é um número real que satisfaz os seguintes axiomas:\textsuperscript{\protect\hyperlink{ref-grami2023}{20}}

  \begin{itemize}
  \item
    Axioma I. Probabilidades de um evento \(E\) são números não-negativos: \(P(E)≥0\).
  \item
    Axioma II. Probabilidade de todos os eventos do espaço amostral \(A\) ocorrerem é 100\%: \(P(S)=1\).
  \item
    Axioma III. A probabilidade de um conjunto \emph{k} de eventos mutuamente exclusivos é igual a soma da probabilidade de cada evento: \(P(E_{1} \cup E_{2} \cup ... E_{k}) = P(E_{1}) + P(E_{2}) + ... + P(E_{k})\).
  \end{itemize}
\item
  Os axiomas possuem as seguintes consequências:\textsuperscript{\protect\hyperlink{ref-grami2023}{20}}

  \begin{itemize}
  \item
    A soma da probabilidade de dois eventos que dividem o espaço amostral é 100\%: \(P(E)+P(E)^C=1\).
  \item
    O valor máximo de probabilidade de um evento é 100\%: \(P(S)≤1\).
  \item
    A probabilidade é uma função não decrescente do número de desfechos de um evento.
  \end{itemize}
\end{itemize}

\hypertarget{independencia-probabilidade}{%
\section{Independência e probabilidade}\label{independencia-probabilidade}}

\hypertarget{o-que-uxe9-independuxeancia-em-estatuxedstica}{%
\subsection{O que é independência em estatística?}\label{o-que-uxe9-independuxeancia-em-estatuxedstica}}

\begin{itemize}
\item
  Em experimentos aleatórios, é comum assumir que os eventos de tentativas separadas são independentes devido a independência física de eventos e experimentos.\textsuperscript{\protect\hyperlink{ref-grami2023}{20}}
\item
  Se a ocorrência do evento \(E_{1}\) não tiver efeito na ocorrência do evento \(E_{2}\), os eventos \(E_{1}\) e \(E_{2}\) são considerados estatisticamente independentes.
\item
  Eventos são mutuamente exclusivos, ou disjuntos, se a ocorrência de um exclui a ocorrência dos outros.\textsuperscript{\protect\hyperlink{ref-grami2023}{20}}
\item
  Se dois eventos \(E_{1}\) e \(E_{2}\) são mutuamente exclusivos, então os eventos \(E_{1}\) e \(E_{2}\) não podem ocorrer ao mesmo tempo e, portanto, são eventos dependentes.
\item
  Em experimentos independentes, o desfecho de uma tentativa é independente dos desfechos de outras tentativas, passadas e/ou futuras. Uma tentativa em um experimento aleatório é independente se a probabilidade de cada desfecho possível não mudar de tentativa para tentativa.\textsuperscript{\protect\hyperlink{ref-grami2023}{20}}
\end{itemize}

\begin{figure}

{\centering \includegraphics{Ciencia-com-R_files/figure-latex/independencia-dado-1} 

}

\caption{Esquerda: Evento (face = 4). Direita: Experimentos de 1 lançamento de 1 dado (superior), 3 lançamentos de 1 dado (central), 10 lançamentos de 1 dado (inferior).}\label{fig:independencia-dado}
\end{figure}

\hypertarget{o-que-uxe9-probabilidade-marginal}{%
\subsection{O que é probabilidade marginal?}\label{o-que-uxe9-probabilidade-marginal}}

\begin{itemize}
\tightlist
\item
  Probabilidade marginal é a probabilidade de ocorrência de um evento \(E\) independentemente da(s) probabilidade(s) de outro(s) evento(s).\textsuperscript{\protect\hyperlink{ref-grami2023}{20}}
\end{itemize}

\hypertarget{o-que-uxe9-probabilidade-conjunta}{%
\subsection{O que é probabilidade conjunta?}\label{o-que-uxe9-probabilidade-conjunta}}

\begin{itemize}
\item
  Probabilidade conjunta é a probabilidade de ocorrência de dois ou mais eventos independentes \(E_{1}\), \(E_{2}\), \ldots, \(E_{k}\), independentemente da(s) probabilidade(s) de outro(s) evento(s).\textsuperscript{\protect\hyperlink{ref-grami2023}{20}}
\item
  Se a probabilidade conjunta dos eventos é nula (\(E_{1} \cup E_{2} = 0\)), esses dois eventos \(E_{1}\) e \(E_{2}\) são mutuamente exclusivos ou disjuntos.\textsuperscript{\protect\hyperlink{ref-grami2023}{20}}
\end{itemize}

\hypertarget{o-que-uxe9-probabilidade-condicional}{%
\subsection{O que é probabilidade condicional?}\label{o-que-uxe9-probabilidade-condicional}}

\begin{itemize}
\item
  Probabilidade condicional é a probabilidade de ocorrência do evento \(E_{2}\) quando se sabe que o evento \(E_{1}\) já ocorreu \(P(E_{2} | E_{1})\).\textsuperscript{\protect\hyperlink{ref-grami2023}{20}}
\item
  A probabilidade condicional \(P(E_{2} | E_{1})\) representa que a ocorrência do evento \(E_{1}\) fornece informação sobre a ocorrência do evento \(E_{2}\).\textsuperscript{\protect\hyperlink{ref-grami2023}{20}}
\item
  Se a ocorrência do evento \(E_{1}\) tiver alguma influência na ocorrência do evento \(E_{2}\), então a probabilidade condicional do evento \(E_{2}\) dado o evento \(E_{1}\) pode ser maior ou menor do que a probabilidade marginal.\textsuperscript{\protect\hyperlink{ref-grami2023}{20}}
\end{itemize}

\hypertarget{lei-grandes-numeros}{%
\section{Leis dos grandes números}\label{lei-grandes-numeros}}

\hypertarget{o-que-uxe9-a-lei-dos-grandes-nuxfameros}{%
\subsection{O que é a lei dos grandes números?}\label{o-que-uxe9-a-lei-dos-grandes-nuxfameros}}

\begin{itemize}
\tightlist
\item
  .{[}REF{]}
\end{itemize}

\hypertarget{lei-pequenos-numeros}{%
\section{Leis dos pequenos números}\label{lei-pequenos-numeros}}

\hypertarget{o-que-uxe9-a-lei-dos-pequenos-nuxfameros}{%
\subsection{O que é a lei dos pequenos números?}\label{o-que-uxe9-a-lei-dos-pequenos-nuxfameros}}

\begin{itemize}
\item
  A crença exagerada na probabilidade de replicar com sucesso os achados de um estudo, pela tendência de se considerar uma amostra como representativa da população.\textsuperscript{\protect\hyperlink{ref-tversky1971}{21}}
\item
  A crença na lei dos pequenos números se refere à tendência de superestimar a estabilidade das estimativas provenientes de estudos com amostras pequenas.\textsuperscript{\protect\hyperlink{ref-bishop2022}{22}}
\item
  Quando se percebe um padrão, pode não ser possível identificar se tal padrão é real.\textsuperscript{\protect\hyperlink{ref-guy1988}{23}}

  \begin{itemize}
  \item
    1a Lei Forte dos Pequenos Números: ``Não há pequenos números suficientes para atender às muitas demandas que lhes são feitas''.\textsuperscript{\protect\hyperlink{ref-guy1988}{23}}
  \item
    2a Lei Forte dos Pequenos Números: ``Quando dois números parecem iguais, não são necessariamente assim''.\textsuperscript{\protect\hyperlink{ref-guy1990}{24}}
  \end{itemize}
\end{itemize}

\hypertarget{lei-numeros-anomals}{%
\section{Leis dos números anômalos}\label{lei-numeros-anomals}}

\hypertarget{o-que-uxe9-a-lei-dos-nuxfameros-anuxf4malos}{%
\subsection{O que é a lei dos números anômalos?}\label{o-que-uxe9-a-lei-dos-nuxfameros-anuxf4malos}}

\begin{itemize}
\tightlist
\item
  .{[}REF{]}
\end{itemize}

\hypertarget{teorema-central-limite}{%
\section{Teorema central do limite}\label{teorema-central-limite}}

\hypertarget{o-que-uxe9-teorema-central-do-limite}{%
\subsection{O que é teorema central do limite?}\label{o-que-uxe9-teorema-central-do-limite}}

\begin{itemize}
\tightlist
\item
  .{[}REF{]}
\end{itemize}

\hypertarget{regressao-media}{%
\section{Regressão para a média}\label{regressao-media}}

\hypertarget{o-que-uxe9-regressuxe3o-para-a-muxe9dia}{%
\subsection{O que é regressão para a média?}\label{o-que-uxe9-regressuxe3o-para-a-muxe9dia}}

\begin{itemize}
\tightlist
\item
  .{[}REF{]}
\end{itemize}

\hypertarget{pensamento-estatistico}{%
\chapter{\texorpdfstring{\textbf{Pensamento estatístico}}{Pensamento estatístico}}\label{pensamento-estatistico}}

\hypertarget{metodologia-da-pesquisa}{%
\section{Metodologia da pesquisa}\label{metodologia-da-pesquisa}}

\hypertarget{qual-a-relauxe7uxe3o-entre-estatuxedstica-e-metodologia-da-pesquisa-cientuxedfica}{%
\subsection{Qual a relação entre estatística e metodologia da pesquisa científica?}\label{qual-a-relauxe7uxe3o-entre-estatuxedstica-e-metodologia-da-pesquisa-cientuxedfica}}

\begin{itemize}
\tightlist
\item
  .\textsuperscript{\protect\hyperlink{ref-munafuxf22017}{25}}
\end{itemize}

\begin{figure}

{\centering \includegraphics{Ciencia-com-R_files/figure-latex/unnamed-chunk-2-1} 

}

\caption{Mapa mental da relação entre o pensamento estatístico e o pensamento metodológico.}\label{fig:unnamed-chunk-2}
\end{figure}

\hypertarget{populacao-amostra}{%
\section{População e amostra}\label{populacao-amostra}}

\hypertarget{o-que-uxe9-populauxe7uxe3o}{%
\subsection{O que é população?}\label{o-que-uxe9-populauxe7uxe3o}}

\begin{itemize}
\item
  População - ou população-alvo - refere-se ao conjunto completo sobre o qual se pretende obter informações.\textsuperscript{\protect\hyperlink{ref-Banerjee2010}{26}}
\item
  População é metodologicamente delimitada pelos critérios de inclusão e exclusão do estudo.\textsuperscript{\protect\hyperlink{ref-Banerjee2010}{26}}
\item
  Em estudos observacionais, inicialmente as características geográficas e/ou demográficas, por exemplo, definem a população a ser estudada.\textsuperscript{\protect\hyperlink{ref-Banerjee2010}{26}}
\item
  Em estudos analíticos, a população é inicialmente definida pelos objetivos da pesquisa e posteriormente as observações são realizadas na amostra.\textsuperscript{\protect\hyperlink{ref-Banerjee2010}{26}}
\end{itemize}

\hypertarget{o-que-uxe9-amostra}{%
\subsection{O que é amostra?}\label{o-que-uxe9-amostra}}

\begin{itemize}
\item
  Amostra é uma parte da população do estudo.\textsuperscript{\protect\hyperlink{ref-Banerjee2010}{26}}
\item
  Em pesquisa científica, utilizam-se dados de uma amostra de participantes (ou outras unidades de análise) para realizar inferências sobre a população.\textsuperscript{\protect\hyperlink{ref-Bland2015}{27}}
\end{itemize}

\hypertarget{unidade-analise}{%
\section{Unidade de análise}\label{unidade-analise}}

\hypertarget{o-que-uxe9-unidade-de-anuxe1lise}{%
\subsection{O que é unidade de análise?}\label{o-que-uxe9-unidade-de-anuxe1lise}}

\begin{itemize}
\item
  A unidade de análise (ou unidade experimental) de pesquisas na área de saúde geralmente é o indivíduo.\textsuperscript{\protect\hyperlink{ref-Altman1997}{28}}
\item
  A unidade de análise também pode ser a instituição em estudos multicêntricos (ex.: hospitais, clínicas) ou um estudo publicado em meta-análise (ex.: ensaios clínicos).\textsuperscript{\protect\hyperlink{ref-Altman1997}{28}}
\end{itemize}

\hypertarget{por-que-identificar-a-unidade-de-anuxe1lise-de-um-estudo}{%
\subsection{Por que identificar a unidade de análise de um estudo?}\label{por-que-identificar-a-unidade-de-anuxe1lise-de-um-estudo}}

\begin{itemize}
\tightlist
\item
  É fundamental identificar corretamente a unidade de análise para evitar inflação do tamanho da amostra (ex.: medidas bilaterais resultando em o dobro de participantes), violações de suposições dos testes de hipótese (ex.: independência entre medidas e/ou unidade de análise) e resultados espúrios em testes de hipótese (ex.: P-valores menores que aqueles observados se a amostra não estivesse inflada).\textsuperscript{\protect\hyperlink{ref-Altman1997}{28},\protect\hyperlink{ref-Matthews1990}{29}}
\end{itemize}

\hypertarget{que-medidas-podem-ser-obtidas-da-unidade-de-anuxe1lise-de-um-estudo}{%
\subsection{Que medidas podem ser obtidas da unidade de análise de um estudo?}\label{que-medidas-podem-ser-obtidas-da-unidade-de-anuxe1lise-de-um-estudo}}

\begin{itemize}
\tightlist
\item
  Da unidade de análise podem ser coletadas informações em medidas únicas, repetidas, seriadas ou múltiplas.
\end{itemize}

\hypertarget{amostragem}{%
\section{Amostragem}\label{amostragem}}

\hypertarget{o-que-uxe9-amostragem}{%
\subsection{O que é amostragem?}\label{o-que-uxe9-amostragem}}

\begin{itemize}
\tightlist
\item
  .{[}REF{]}
\end{itemize}

\hypertarget{quais-muxe9todos-de-amostragem-suxe3o-usados-para-obter-uma-amostra-da-populauxe7uxe3o}{%
\subsection{Quais métodos de amostragem são usados para obter uma amostra da população?}\label{quais-muxe9todos-de-amostragem-suxe3o-usados-para-obter-uma-amostra-da-populauxe7uxe3o}}

\begin{itemize}
\item
  O método de amostragem é geralmente definido pelas condições de viabilidade do estudo, no que diz respeito a acesso aos participantes, ao tempo de execução e aos custos envolvidos, entre outras.\textsuperscript{\protect\hyperlink{ref-Banerjee2010}{26}}
\item
  Não-probabilísticas ou intencionais:\textsuperscript{\protect\hyperlink{ref-Banerjee2010}{26}}

  \begin{itemize}
  \item
    Bola de neve.
  \item
    Conveniência.
  \item
    Participantes encaminhados
  \end{itemize}
\item
  Probabilísticas:\textsuperscript{\protect\hyperlink{ref-Banerjee2010}{26}}

  \begin{itemize}
  \item
    Simples.
  \item
    Sistemática.
  \item
    Multiestágio.
  \item
    Estratificada.
  \item
    Agregada.
  \end{itemize}
\end{itemize}

\hypertarget{o-que-uxe9-erro-de-amostragem}{%
\subsection{O que é erro de amostragem?}\label{o-que-uxe9-erro-de-amostragem}}

\begin{itemize}
\tightlist
\item
  .{[}REF{]}
\end{itemize}

\hypertarget{reamostragem}{%
\section{Reamostragem}\label{reamostragem}}

\hypertarget{o-que-uxe9-reamostragem}{%
\subsection{O que é reamostragem?}\label{o-que-uxe9-reamostragem}}

\begin{itemize}
\item
  Reamostragem é um procedimento que cria vários conjuntos de dados sorteados a partir de um conjunto de dados real - a amostra da população - sem a necessidade de fazer suposições sobre os dados e suas distribuições.\textsuperscript{\protect\hyperlink{ref-Bland2015}{27}}
\item
  O procedimento é repetido várias vezes para usar a variabilidade dos resultados para obter um intervalo de confiança do parâmetro.\textsuperscript{\protect\hyperlink{ref-Bland2015}{27}}
\end{itemize}

\hypertarget{por-que-utilizar-reamostragem}{%
\subsection{Por que utilizar reamostragem?}\label{por-que-utilizar-reamostragem}}

\begin{itemize}
\item
  Quando se dispõe de dados de apenas 1 amostra, as diversas suposições que são feitas podem não ser atingidas.\textsuperscript{\protect\hyperlink{ref-Bland2015}{27}}
\item
  Procedimentos de reamostragem produzem um conjunto de observações escolhidas aleatoriamente da amostra, igualmente representativo da população original.\textsuperscript{\protect\hyperlink{ref-Bland2015}{27}}
\item
  Procedimentos de reamostragem permitem estimar o erro-padrão e intervalos de confiança sem a necessidade de tais suposições, sendo, portanto, um conjunto de procedimentos não-paramétricos.\textsuperscript{\protect\hyperlink{ref-Bland2015}{27}}
\end{itemize}

\hypertarget{quais-procedimentos-de-reamostragem-podem-ser-realizados}{%
\subsection{Quais procedimentos de reamostragem podem ser realizados?}\label{quais-procedimentos-de-reamostragem-podem-ser-realizados}}

\begin{itemize}
\tightlist
\item
  \emph{Bootstrap}: Cada iteração gera uma amostra \emph{bootstrap} do mesmo tamanho do conjunto de dados original escolhendo aleatoriamente observações reais, uma de cada vez. Cada observação tem chances iguais de ser escolhida a cada vez, portanto, algumas observações serão escolhidas mais de uma vez e outras nem serão escolhidas.\textsuperscript{\protect\hyperlink{ref-Bland2015}{27}}
\end{itemize}

\hypertarget{pensamento-metodologico}{%
\chapter{\texorpdfstring{\textbf{Pensamento metodológico}}{Pensamento metodológico}}\label{pensamento-metodologico}}

\hypertarget{reprodutibilidade}{%
\section{Reprodutibilidade}\label{reprodutibilidade}}

\hypertarget{o-que-uxe9-reprodutibilidade}{%
\subsection{O que é reprodutibilidade?}\label{o-que-uxe9-reprodutibilidade}}

\begin{itemize}
\tightlist
\item
  Reprodutibilidade é a habilidade de se obter resultados iguais ou similares quando uma análise ou teste estatístico é repetido.\textsuperscript{\protect\hyperlink{ref-mair2016}{3},\protect\hyperlink{ref-resnik2016}{30},\protect\hyperlink{ref-hofner2015}{31}}
\end{itemize}

\hypertarget{por-que-reprodutibilidade-uxe9-importante}{%
\subsection{Por que reprodutibilidade é importante?}\label{por-que-reprodutibilidade-uxe9-importante}}

\begin{itemize}
\item
  Analisar a reprodutibilidade pode fornecer evidências a respeito da objetividade e confiabilidade dos achados, em detrimento de terem sido obtidos devido a vieses ou ao acaso.\textsuperscript{\protect\hyperlink{ref-resnik2016}{30}}
\item
  A reprodutibilidade não é apenas uma questão metodológica, mas também ética, uma vez que pode envolver mal práticas científicas como fabricação e/ou falsificação de dados.\textsuperscript{\protect\hyperlink{ref-resnik2016}{30}}
\item
  Reprodutibilidade pode ser considerada um padrão mínimo em pesquisa científica.\textsuperscript{\protect\hyperlink{ref-hofner2015}{31}}{]}
\end{itemize}

\hypertarget{como-contribuir-para-a-reprodutibilidade}{%
\subsection{Como contribuir para a reprodutibilidade?}\label{como-contribuir-para-a-reprodutibilidade}}

\begin{itemize}
\tightlist
\item
  Disponibilize publicamente os bancos de dados, respeitando as considerações éticas vigentes (ex.: autorização dos participantes e do Comitê de Ética em Pesquisa) e internacionalmente.\textsuperscript{\protect\hyperlink{ref-mair2016}{3}}
\end{itemize}

\begin{infobox}{images/Rlogo}
O pacote \emph{base}\textsuperscript{\protect\hyperlink{ref-base-5}{32}} fornece a função \href{https://www.rdocumentation.org/packages/base/versions/3.6.2/topics/Random}{\emph{set.seed}} para especificar uma semente para reprodutibilidade de computações que envolvem números aleatórios.

\end{infobox}

\begin{itemize}
\tightlist
\item
  Produza manuscritos reprodutíveis - manuscritos executáveis ou relatórios dinâmicos - que permitem a integração do banco de dados da(s) amostra(s), do(s) script(s) de análise estatística (incluindo comentários para sua interpretação), dos pacotes ou bibliotecas utilizados, das fontes e referências bibliográficas citadas, além dos demais elementos textuais (tabelas, gráficos) - todos gerados dinamicamente.\textsuperscript{\protect\hyperlink{ref-hinsen2011}{8}}
\end{itemize}

\begin{infobox}{images/Rlogo}
O pacote \emph{rmarkdown}\textsuperscript{\protect\hyperlink{ref-rmarkdown}{15}} fornece a função \href{https://www.rdocumentation.org/packages/rmarkdown/versions/2.24/topics/render}{\emph{render}} para criar manuscritos reprodutíveis a partir de arquivos .Rmd.

\end{infobox}

\begin{infobox}{images/Rlogo}
O pacote \emph{bookdown}\textsuperscript{\protect\hyperlink{ref-bookdown-2}{33}} fornece as funções \href{https://www.rdocumentation.org/packages/bookdown/versions/0.35/topics/gitbook}{\emph{gitbook}}, \href{https://www.rdocumentation.org/packages/bookdown/versions/0.35/topics/pdf_book}{\emph{pdf\_book}}, \href{https://www.rdocumentation.org/packages/bookdown/versions/0.35/topics/epub_book}{\emph{epub\_book}} e \href{https://www.rdocumentation.org/packages/bookdown/versions/0.35/topics/html_document2}{\emph{html\_document2}} para criar documentos reprodutíveis em diversos formatos (Git, PDF, EPUB e HTML, respectivamente).

\end{infobox}

\hypertarget{robustez}{%
\section{Robustez}\label{robustez}}

\hypertarget{o-que-uxe9-robustez}{%
\subsection{O que é robustez?}\label{o-que-uxe9-robustez}}

\begin{itemize}
\tightlist
\item
  .{[}REF{]}
\end{itemize}

\hypertarget{replicabilidade}{%
\section{Replicabilidade}\label{replicabilidade}}

\hypertarget{o-que-uxe9-replicabilidade}{%
\subsection{O que é replicabilidade?}\label{o-que-uxe9-replicabilidade}}

\begin{itemize}
\tightlist
\item
  Replicabilidade é a habilidade de se obter conclusões iguais ou similares quando um experimento é repetido.\textsuperscript{\protect\hyperlink{ref-mair2016}{3},\protect\hyperlink{ref-hofner2015}{31}}
\end{itemize}

\hypertarget{compartilhamento}{%
\section{Compartilhamento}\label{compartilhamento}}

\hypertarget{o-que-pode-ser-compartilhado}{%
\subsection{O que pode ser compartilhado?}\label{o-que-pode-ser-compartilhado}}

\begin{itemize}
\item
  Idealmente, todos os scripts, pacotes/bibliotecas e dados necessários para outros reproduzirem seus dados.\textsuperscript{\protect\hyperlink{ref-Eglen2017}{10}}
\item
  Minimamente, partes importantes incluindo implementações de novos algoritmos e dados que permitam reproduzir um resultado importante.\textsuperscript{\protect\hyperlink{ref-Eglen2017}{10}}
\end{itemize}

\hypertarget{como-preparar-dados-para-compartilhamento}{%
\subsection{Como preparar dados para compartilhamento?}\label{como-preparar-dados-para-compartilhamento}}

\begin{itemize}
\tightlist
\item
  .{[}REF{]}
\end{itemize}

\begin{infobox}{images/Rlogo}
O pacote \emph{synthpop}\textsuperscript{\protect\hyperlink{ref-synthpop}{34}} fornece a função \href{https://www.rdocumentation.org/packages/synthpop/versions/1.8-0/topics/syn}{\emph{syn}} para criar bancos de dados sintéticos a partir de um banco de dados real.

\end{infobox}

\begin{infobox}{images/Rlogo}
O pacote \emph{grateful}\textsuperscript{\protect\hyperlink{ref-grateful}{35}} fornece a função \href{https://www.rdocumentation.org/packages/grateful/versions/0.2.0/topics/cite_packages}{\emph{cite\_packages}} para citar os pacotes utilizados em um projeto R.

\end{infobox}

\hypertarget{como-preparar-scripts-para-compartilhamento}{%
\subsection{Como preparar scripts para compartilhamento?}\label{como-preparar-scripts-para-compartilhamento}}

\begin{itemize}
\item
  Documente em um arquivo README os arquivos disponíveis e os pré-requisitos necessários para executar o código (ex.: pacotes e respectivas versões). Uma lista de configurações (hardware e software) que foram usadas para rodar o código pode ajudar na reprodução dos resultados.\textsuperscript{\protect\hyperlink{ref-hofner2015}{31}}
\item
  Crie links persistentes para versões do seu script.\textsuperscript{\protect\hyperlink{ref-Eglen2017}{10}}
\item
  Defina uma semente para o gerador de números aleatórios em scripts com métodos computacionais que dependem da geração de números pseudoaleatórios.\textsuperscript{\protect\hyperlink{ref-hofner2015}{31}}
\item
  Escolha uma licença apropriada para garantir como outros usarão seus scripts.\textsuperscript{\protect\hyperlink{ref-Eglen2017}{10}}
\item
  Compartilhe todos os pacotes relacionados à sua análise.\textsuperscript{\protect\hyperlink{ref-Zhao2023}{36}}
\item
  Providencie a documentação sobre seu script (ex.: arquivo README).\textsuperscript{\protect\hyperlink{ref-Eglen2017}{10}}
\end{itemize}

\begin{infobox}{images/Rlogo}
O pacote \emph{base}\textsuperscript{\protect\hyperlink{ref-base-5}{32}} fornece a função \href{https://www.rdocumentation.org/packages/base/versions/3.6.2/topics/Random}{\emph{set.seed}} para especificar uma semente para reprodutibilidade de computações que envolvem números aleatórios.

\end{infobox}

\hypertarget{o-que-incluir-no-arquivo-readme}{%
\subsection{O que incluir no arquivo README?}\label{o-que-incluir-no-arquivo-readme}}

\begin{itemize}
\item
  Título do manuscrito.\textsuperscript{\protect\hyperlink{ref-hofner2015}{31}}
\item
  Autores do manuscrito.\textsuperscript{\protect\hyperlink{ref-hofner2015}{31}}
\item
  Principais responsáveis pela escrita do script e quaisquer outras pessoas que fizeram contribuições substanciais para o desenvolvimento do script.\textsuperscript{\protect\hyperlink{ref-hofner2015}{31}}
\item
  Endereço de e-mail do autor ou contribuidor a quem devem ser direcionadas dúvidas, comentários, sugestões e bugs sobre o script.\textsuperscript{\protect\hyperlink{ref-hofner2015}{31}}
\item
  Lista de configurações nas quais o script foi testado, tais com nome e versão do programa, pacotes e plataforma.\textsuperscript{\protect\hyperlink{ref-hofner2015}{31}}
\end{itemize}

\begin{infobox}{images/Rlogo}
O pacote \emph{utils}\textsuperscript{\protect\hyperlink{ref-utils}{37}} fornece a função \href{https://www.rdocumentation.org/packages/utils/versions/3.6.2/topics/sessionInfo}{\emph{sessionInfo}} para descrever as características do programa, pacotes e plataforma da sessão atual.

\end{infobox}

\hypertarget{vieses-paradoxos-estatuxedsticos}{%
\chapter{\texorpdfstring{\textbf{Vieses e paradoxos estatísticos}}{Vieses e paradoxos estatísticos}}\label{vieses-paradoxos-estatuxedsticos}}

\hypertarget{vieses-estatisticos}{%
\section{Vieses estatísticos}\label{vieses-estatisticos}}

\hypertarget{o-que-suxe3o-vieses-estatuxedsticos}{%
\subsection{O que são vieses estatísticos?}\label{o-que-suxe3o-vieses-estatuxedsticos}}

\begin{itemize}
\tightlist
\item
  .{[}REF{]}
\end{itemize}

\hypertarget{paradoxos}{%
\section{Paradoxos estatísticos}\label{paradoxos}}

\hypertarget{o-que-suxe3o-paradoxos-estatuxedsticos}{%
\subsection{O que são paradoxos estatísticos?}\label{o-que-suxe3o-paradoxos-estatuxedsticos}}

\begin{itemize}
\tightlist
\item
  .{[}REF{]}
\end{itemize}

\hypertarget{abelson}{%
\subsection{O que é o paradoxo de Abelson?}\label{abelson}}

\begin{itemize}
\tightlist
\item
  .\textsuperscript{\protect\hyperlink{ref-abelson1985}{38}}
\end{itemize}

\hypertarget{Berkson}{%
\subsection{O que é o paradoxo de Berkson?}\label{Berkson}}

\begin{itemize}
\tightlist
\item
  .\textsuperscript{\protect\hyperlink{ref-berkson1946}{39}}
\end{itemize}

\hypertarget{ellsberg}{%
\subsection{O que é o paradoxo de Ellsberg?}\label{ellsberg}}

\begin{itemize}
\tightlist
\item
  .\textsuperscript{\protect\hyperlink{ref-ellsberg1961}{40}}
\end{itemize}

\hypertarget{freedman}{%
\subsection{O que é o paradoxo de Freedman?}\label{freedman}}

\begin{itemize}
\tightlist
\item
  .\textsuperscript{\protect\hyperlink{ref-freedman1983}{41},\protect\hyperlink{ref-freedman1989}{42}}
\end{itemize}

\hypertarget{hand}{%
\subsection{O que é o paradoxo de Hand?}\label{hand}}

\begin{itemize}
\tightlist
\item
  .\textsuperscript{\protect\hyperlink{ref-hand1992}{43}}
\end{itemize}

\hypertarget{lindley}{%
\subsection{O que é o paradoxo de Lindley?}\label{lindley}}

\begin{itemize}
\tightlist
\item
  .\textsuperscript{\protect\hyperlink{ref-lindley1957}{44}}
\end{itemize}

\hypertarget{lord}{%
\subsection{O que é o paradoxo de Lord?}\label{lord}}

\begin{itemize}
\tightlist
\item
  .\textsuperscript{\protect\hyperlink{ref-lord1967}{45},\protect\hyperlink{ref-lord1969}{46}}
\end{itemize}

\hypertarget{proebsting}{%
\subsection{O que é o paradoxo de Proebsting?}\label{proebsting}}

\begin{itemize}
\tightlist
\item
  .{[}REF{]}
\end{itemize}

\hypertarget{simpson}{%
\subsection{O que é o paradoxo de Simpson?}\label{simpson}}

\begin{itemize}
\tightlist
\item
  .\textsuperscript{\protect\hyperlink{ref-simpson1951}{47},\protect\hyperlink{ref-blyth1972}{48}}
\end{itemize}

\hypertarget{stein}{%
\subsection{O que é o paradoxo de Stein?}\label{stein}}

\begin{itemize}
\tightlist
\item
  .\textsuperscript{\protect\hyperlink{ref-stein1956}{49}}
\end{itemize}

\hypertarget{okie}{%
\subsection{O que é o paradoxo de Okie?}\label{okie}}

\begin{itemize}
\tightlist
\item
  .{[}REF{]}
\end{itemize}

\hypertarget{acuracia}{%
\subsection{O que é o paradoxo da acurácia?}\label{acuracia}}

\begin{itemize}
\tightlist
\item
  .{[}REF{]}
\end{itemize}

\hypertarget{elevador}{%
\subsection{O que é o paradoxo do elevador?}\label{elevador}}

\begin{itemize}
\tightlist
\item
  .\textsuperscript{\protect\hyperlink{ref-de1996}{50}}
\end{itemize}

\hypertarget{falso-positivo}{%
\subsection{O que é o paradoxo do falso positivo?}\label{falso-positivo}}

\begin{itemize}
\tightlist
\item
  .{[}REF{]}
\end{itemize}

\hypertarget{amizade}{%
\subsection{O que é o paradoxo da amizade?}\label{amizade}}

\begin{itemize}
\tightlist
\item
  .\textsuperscript{\protect\hyperlink{ref-feld1991}{51}}
\end{itemize}

\hypertarget{vieses-metodologicos}{%
\chapter{\texorpdfstring{\textbf{Vieses metodológicos}}{Vieses metodológicos}}\label{vieses-metodologicos}}

\hypertarget{vieses-metodologicos}{%
\section{Vieses metodológicos}\label{vieses-metodologicos}}

\hypertarget{o-que-suxe3o-vieses-metodoluxf3gicos}{%
\subsection{O que são vieses metodológicos?}\label{o-que-suxe3o-vieses-metodoluxf3gicos}}

\begin{itemize}
\tightlist
\item
  .{[}REF{]}
\end{itemize}

\cftaddtitleline{toc}{chapter}{\rule{\textwidth}{0.4pt}}{}

\hypertarget{parte-2---estatuxedstica-buxe1sica}{%
\chapter*{\texorpdfstring{\emph{Parte 2 - Estatística Básica}}{Parte 2 - Estatística Básica}}\label{parte-2---estatuxedstica-buxe1sica}}
\addcontentsline{toc}{chapter}{\emph{Parte 2 - Estatística Básica}}

\markboth{}{}
\par\noindent\rule{\textwidth}{0.05in}

\hypertarget{medidas-instrumentos}{%
\chapter{\texorpdfstring{\textbf{Medidas e instrumentos}}{Medidas e instrumentos}}\label{medidas-instrumentos}}

\hypertarget{medidas}{%
\section{Medidas}\label{medidas}}

\hypertarget{o-que-suxe3o-medidas-diretas}{%
\subsection{O que são medidas diretas?}\label{o-que-suxe3o-medidas-diretas}}

\begin{itemize}
\tightlist
\item
  .{[}REF{]}
\end{itemize}

\hypertarget{o-que-suxe3o-medidas-derivadas}{%
\subsection{O que são medidas derivadas?}\label{o-que-suxe3o-medidas-derivadas}}

\begin{itemize}
\tightlist
\item
  .{[}REF{]}
\end{itemize}

\hypertarget{o-que-suxe3o-medidas-por-teoria}{%
\subsection{O que são medidas por teoria?}\label{o-que-suxe3o-medidas-por-teoria}}

\begin{itemize}
\tightlist
\item
  .{[}REF{]}
\end{itemize}

\hypertarget{o-que-suxe3o-medidas-uxfanicas}{%
\subsection{O que são medidas únicas?}\label{o-que-suxe3o-medidas-uxfanicas}}

\begin{itemize}
\item
  A medida única da pressão arterial sistólica no braço esquerdo resulta em um valor pontual.{[}REF{]}
\item
  Medidas únicas obtidas de diferentes unidades de análise podem ser consideradas independentes se observadas outras condições na coleta de dados.{[}REF{]}
\item
  O valor pontual será considerado representativo da variável para a unidade de análise (ex.: \textbf{120 mmHg} para o participante \textbf{\#9}).
\end{itemize}

\begin{table}

\caption{\label{tab:medidas-unicas}Tabela de dados brutos com medidas únicas.}
\centering
\begin{tabu} to \linewidth {>{\centering}X>{\centering}X}
\toprule
\textbf{Unidade de análise} & \textbf{Pressão arterial, braço esquerdo (mmHg)}\\
\midrule
\cellcolor{gray!6}{1} & \cellcolor{gray!6}{118}\\
2 & 113\\
\cellcolor{gray!6}{3} & \cellcolor{gray!6}{116}\\
4 & 110\\
\cellcolor{gray!6}{5} & \cellcolor{gray!6}{111}\\
6 & 116\\
\cellcolor{gray!6}{7} & \cellcolor{gray!6}{120}\\
8 & 111\\
\cellcolor[HTML]{E6E6E6}{\textbf{\cellcolor{gray!6}{9}}} & \cellcolor[HTML]{E6E6E6}{\textbf{\cellcolor{gray!6}{120}}}\\
10 & 112\\
\bottomrule
\end{tabu}
\end{table}

\hypertarget{o-que-suxe3o-medidas-repetidas}{%
\subsection{O que são medidas repetidas?}\label{o-que-suxe3o-medidas-repetidas}}

\begin{itemize}
\item
  As medidas repetidas podem ser tabuladas separadamente, por exemplo para análise da confiabilidade de obtenção dessa medida.{[}REF{]}
\item
  A medida repetida da pressão arterial no braço esquerdo resulta em um conjunto de valores pontuais (ex.: \textbf{110 mmHg}, \textbf{118 mmHg} e \textbf{116 mmHg} para o participante \textbf{\#5}).
\end{itemize}

\begin{table}

\caption{\label{tab:medidas-repetidas-separadas}Tabela de dados brutos com medidas repetidas.}
\centering
\begin{tabu} to \linewidth {>{\centering}X>{\centering}X>{\centering}X>{\centering}X}
\toprule
\textbf{Unidade de análise} & \textbf{Pressão arterial, braço esquerdo (mmHg) \#1} & \textbf{Pressão arterial, braço esquerdo (mmHg) \#2} & \textbf{Pressão arterial, braço esquerdo (mmHg) \#3}\\
\midrule
\cellcolor{gray!6}{1} & \cellcolor{gray!6}{114} & \cellcolor{gray!6}{112} & \cellcolor{gray!6}{112}\\
2 & 115 & 120 & 113\\
\cellcolor{gray!6}{3} & \cellcolor{gray!6}{115} & \cellcolor{gray!6}{110} & \cellcolor{gray!6}{120}\\
4 & 117 & 116 & 114\\
\cellcolor[HTML]{E6E6E6}{\textbf{\cellcolor{gray!6}{5}}} & \cellcolor[HTML]{E6E6E6}{\textbf{\cellcolor{gray!6}{110}}} & \cellcolor[HTML]{E6E6E6}{\textbf{\cellcolor{gray!6}{118}}} & \cellcolor[HTML]{E6E6E6}{\textbf{\cellcolor{gray!6}{116}}}\\
6 & 110 & 120 & 113\\
\cellcolor{gray!6}{7} & \cellcolor{gray!6}{118} & \cellcolor{gray!6}{114} & \cellcolor{gray!6}{117}\\
8 & 111 & 112 & 119\\
\cellcolor{gray!6}{9} & \cellcolor{gray!6}{120} & \cellcolor{gray!6}{112} & \cellcolor{gray!6}{117}\\
10 & 110 & 115 & 115\\
\bottomrule
\end{tabu}
\end{table}

\begin{itemize}
\item
  As medidas repetidas podem ser agregadas por algum parâmetro --- ex.: média, mediana, máximo, mínimo, entre outros ---, observando-se a relevância biológica, clínica e/ou metodológica desta escolha.{[}REF{]}
\item
  Medidas agregadas obtidas de diferentes unidades de análise podem ser consideradas independentes se observadas outras condições na coleta de dados.{[}REF{]}
\item
  O valor agregado será considerado representativo da variável para a unidade de análise (ex.: média = \textbf{115 mmHg} para o participante \textbf{\#5}).{[}REF{]}
\end{itemize}

\begin{table}

\caption{\label{tab:medidas-repetidas-agregadas}Tabela de dados brutos com medidas repetidas agregadas.}
\centering
\begin{tabu} to \linewidth {>{\centering}X>{\centering}X}
\toprule
\textbf{Unidade de análise} & \textbf{Pressão arterial, braço esquerdo (mmHg) média}\\
\midrule
\cellcolor{gray!6}{1} & \cellcolor{gray!6}{112.6667}\\
2 & 116.0000\\
\cellcolor{gray!6}{3} & \cellcolor{gray!6}{115.0000}\\
4 & 115.6667\\
\cellcolor[HTML]{E6E6E6}{\textbf{\cellcolor{gray!6}{5}}} & \cellcolor[HTML]{E6E6E6}{\textbf{\cellcolor{gray!6}{114.6667}}}\\
6 & 114.3333\\
\cellcolor{gray!6}{7} & \cellcolor{gray!6}{116.3333}\\
8 & 114.0000\\
\cellcolor{gray!6}{9} & \cellcolor{gray!6}{116.3333}\\
10 & 113.3333\\
\bottomrule
\end{tabu}
\end{table}

\begin{infobox}{images/Rlogo}
O pacote \emph{stats}\textsuperscript{\protect\hyperlink{ref-stats-2}{52}} fornece a função \href{https://www.rdocumentation.org/packages/stats/versions/3.6.2/topics/aggregate}{\emph{aggregate}} para agregar medidas repetidas utilizando uma função personalizada.

\end{infobox}

\hypertarget{o-que-suxe3o-medidas-seriadas}{%
\subsection{O que são medidas seriadas?}\label{o-que-suxe3o-medidas-seriadas}}

\begin{itemize}
\item
  Medidas seriadas são possivelmente relacionadas e, portanto, dependentes na mesma unidade de análise.{[}REF{]}
\item
  Por exemplo, a medida seriada da pressão arterial no braço esquerdo, em intervalos tipicamente regulares (ex.: \textbf{114 mmHg}, \textbf{120 mmHg} e \textbf{110 mmHg} em \textbf{1 min}, \textbf{2 min} e \textbf{3 min}, respectivamente, para o participante \textbf{\#1}).
\end{itemize}

\begin{table}

\caption{\label{tab:medidas-seriadas-separadas}Tabela de dados brutos com medidas seriadas não agregadas.}
\centering
\begin{tabu} to \linewidth {>{\centering}X>{\centering}X>{\centering}X}
\toprule
\textbf{Unidade de análise} & \textbf{Tempo (min)} & \textbf{Pressão arterial, braço esquerdo (mmHg)}\\
\midrule
\cellcolor[HTML]{E6E6E6}{\textbf{\cellcolor{gray!6}{1}}} & \cellcolor[HTML]{E6E6E6}{\textbf{\cellcolor{gray!6}{1}}} & \cellcolor[HTML]{E6E6E6}{\textbf{\cellcolor{gray!6}{114}}}\\
\cellcolor[HTML]{E6E6E6}{\textbf{1}} & \cellcolor[HTML]{E6E6E6}{\textbf{2}} & \cellcolor[HTML]{E6E6E6}{\textbf{120}}\\
\cellcolor[HTML]{E6E6E6}{\textbf{\cellcolor{gray!6}{1}}} & \cellcolor[HTML]{E6E6E6}{\textbf{\cellcolor{gray!6}{3}}} & \cellcolor[HTML]{E6E6E6}{\textbf{\cellcolor{gray!6}{110}}}\\
2 & 1 & 119\\
\cellcolor{gray!6}{2} & \cellcolor{gray!6}{2} & \cellcolor{gray!6}{120}\\
2 & 3 & 114\\
\cellcolor{gray!6}{3} & \cellcolor{gray!6}{1} & \cellcolor{gray!6}{116}\\
3 & 2 & 114\\
\cellcolor{gray!6}{3} & \cellcolor{gray!6}{3} & \cellcolor{gray!6}{116}\\
4 & 1 & 113\\
\bottomrule
\end{tabu}
\end{table}

\begin{itemize}
\tightlist
\item
  Medidas seriadas também agregadas por parâmetros --- ex.: máximo, mínimo, amplitude --- são consideradas representativas da variação temporal ou de uma característica de interesse (ex.: amplitude = \textbf{10 mmHg} para o participante \textbf{\#1}).
\end{itemize}

\begin{table}

\caption{\label{tab:medidas-seriadas-agregadas}Tabela de dados brutos com medidas seriadas não agregadas.}
\centering
\begin{tabu} to \linewidth {>{\centering}X>{\centering}X}
\toprule
\textbf{Unidade de análise} & \textbf{Pressão arterial, braço esquerdo (mmHg) amplitude}\\
\midrule
\cellcolor[HTML]{E6E6E6}{\textbf{\cellcolor{gray!6}{1}}} & \cellcolor[HTML]{E6E6E6}{\textbf{\cellcolor{gray!6}{10}}}\\
2 & 6\\
\cellcolor{gray!6}{3} & \cellcolor{gray!6}{2}\\
4 & 6\\
\cellcolor{gray!6}{5} & \cellcolor{gray!6}{1}\\
6 & 8\\
\cellcolor{gray!6}{7} & \cellcolor{gray!6}{9}\\
8 & 10\\
\cellcolor{gray!6}{9} & \cellcolor{gray!6}{7}\\
10 & 5\\
\bottomrule
\end{tabu}
\end{table}

\begin{infobox}{images/Rlogo}
O pacote \emph{stats}\textsuperscript{\protect\hyperlink{ref-stats-2}{52}} fornece a função \href{https://www.rdocumentation.org/packages/stats/versions/3.6.2/topics/aggregate}{\emph{aggregate}} para agregar medidas repetidas utilizando uma função personalizada.

\end{infobox}

\hypertarget{o-que-suxe3o-medidas-muxfaltiplas}{%
\subsection{O que são medidas múltiplas?}\label{o-que-suxe3o-medidas-muxfaltiplas}}

\begin{itemize}
\item
  Medidas múltiplas também são possivelmente relacionadas e, portanto, são dependentes na mesma unidade de análise. Medidas múltiplas podem ser obtidas de modo repetido para análise agregada ou seriada.{[}REF{]}
\item
  A medida de pressão arterial bilateral resulta em um conjunto de valores pontuais (ex.: braço esquerdo = \textbf{114 mmHg}, braço direito = \textbf{118 mmHg} para o participante \textbf{\#8}). Neste caso, ambos os valores pontuais são considerados representativos daquela unidade de análise.
\end{itemize}

\begin{table}

\caption{\label{tab:medidas-multiplas}Tabela de dados brutos com medidas múltiplas.}
\centering
\begin{tabu} to \linewidth {>{\centering}X>{\centering}X>{\centering}X}
\toprule
\textbf{Unidade de análise} & \textbf{Pressão arterial, braço esquerdo (mmHg)} & \textbf{Pressão arterial, braço direito (mmHg)}\\
\midrule
\cellcolor{gray!6}{1} & \cellcolor{gray!6}{117} & \cellcolor{gray!6}{115}\\
2 & 120 & 118\\
\cellcolor{gray!6}{3} & \cellcolor{gray!6}{112} & \cellcolor{gray!6}{118}\\
4 & 112 & 112\\
\cellcolor{gray!6}{5} & \cellcolor{gray!6}{116} & \cellcolor{gray!6}{112}\\
6 & 112 & 118\\
\cellcolor{gray!6}{7} & \cellcolor{gray!6}{115} & \cellcolor{gray!6}{113}\\
\cellcolor[HTML]{E6E6E6}{\textbf{8}} & \cellcolor[HTML]{E6E6E6}{\textbf{114}} & \cellcolor[HTML]{E6E6E6}{\textbf{118}}\\
\cellcolor{gray!6}{9} & \cellcolor{gray!6}{119} & \cellcolor{gray!6}{114}\\
10 & 112 & 116\\
\bottomrule
\end{tabu}
\end{table}

\begin{infobox}{images/Rlogo}
O pacote \emph{stats}\textsuperscript{\protect\hyperlink{ref-stats-2}{52}} fornece a função \href{https://www.rdocumentation.org/packages/stats/versions/3.6.2/topics/aggregate}{\emph{aggregate}} para agregar medidas repetidas utilizando uma função personalizada.

\end{infobox}

\hypertarget{erro-medida}{%
\section{Erros de medida}\label{erro-medida}}

\hypertarget{o-que-suxe3o-erros-de-medida}{%
\subsection{O que são erros de medida?}\label{o-que-suxe3o-erros-de-medida}}

\begin{itemize}
\tightlist
\item
  .{[}REF{]}
\end{itemize}

\hypertarget{instrumentos}{%
\section{Instrumentos}\label{instrumentos}}

\hypertarget{o-que-suxe3o-instrumentos}{%
\subsection{O que são instrumentos?}\label{o-que-suxe3o-instrumentos}}

\begin{itemize}
\tightlist
\item
  .{[}REF{]}
\end{itemize}

\hypertarget{acuruxe1cia-e-precisuxe3o}{%
\section{Acurácia e precisão}\label{acuruxe1cia-e-precisuxe3o}}

\hypertarget{o-que-uxe9-acuruxe1cia}{%
\subsection{O que é acurácia?}\label{o-que-uxe9-acuruxe1cia}}

\begin{itemize}
\tightlist
\item
  .{[}REF{]}
\end{itemize}

\hypertarget{o-que-uxe9-precisuxe3o}{%
\subsection{O que é precisão?}\label{o-que-uxe9-precisuxe3o}}

\begin{itemize}
\tightlist
\item
  .{[}REF{]}
\end{itemize}

\begin{figure}

{\centering \includegraphics{Ciencia-com-R_files/figure-latex/acuracia-precisao-1} 

}

\caption{Acurácia e precisão como propriedades de uma medida.}\label{fig:acuracia-precisao}
\end{figure}

\hypertarget{vies-variabilidade}{%
\section{Viés e variabilidade}\label{vies-variabilidade}}

\hypertarget{qual-uxe9-a-relauxe7uxe3o-entre-viuxe9s-e-variabilidade}{%
\subsection{Qual é a relação entre viés e variabilidade?}\label{qual-uxe9-a-relauxe7uxe3o-entre-viuxe9s-e-variabilidade}}

\begin{itemize}
\tightlist
\item
  .{[}REF{]}
\end{itemize}

\begin{figure}

{\centering \includegraphics{Ciencia-com-R_files/figure-latex/vies-variabilidade-1} 

}

\caption{Acurácia e precisão como propriedades de uma medida.}\label{fig:vies-variabilidade}
\end{figure}

\hypertarget{dados-metadados}{%
\chapter{\texorpdfstring{\textbf{Dados e metadados}}{Dados e metadados}}\label{dados-metadados}}

\hypertarget{dados}{%
\section{Dados}\label{dados}}

\hypertarget{o-que-suxe3o-dados}{%
\subsection{O que são dados?}\label{o-que-suxe3o-dados}}

\begin{itemize}
\item
  ``Tudo são dados''.\textsuperscript{\protect\hyperlink{ref-Olson2021}{53}}
\item
  Dados coletados em um estudo geralmente contêm erros de mensuração e/ou classificação, dados perdidos e são agrupados por alguma unidade de análise.\textsuperscript{\protect\hyperlink{ref-van2022a}{54}}
\end{itemize}

\hypertarget{o-que-suxe3o-dados-primuxe1rios-e-secunduxe1rios}{%
\subsection{O que são dados primários e secundários?}\label{o-que-suxe3o-dados-primuxe1rios-e-secunduxe1rios}}

\begin{itemize}
\item
  Dados primários são dados originais coletados intencionalmente para uma determinada análise exploratória ou inferencial planejada a priori.\textsuperscript{\protect\hyperlink{ref-vetter2017}{55}}
\item
  Dados secundários compreendem dados coletados inicialmente para análises de um estudo, e são subsequentemente utilizados para outras análises.\textsuperscript{\protect\hyperlink{ref-vetter2017}{55}}
\end{itemize}

\hypertarget{dados-perdidos}{%
\section{Dados perdidos}\label{dados-perdidos}}

\hypertarget{o-que-suxe3o-dados-perdidos}{%
\subsection{O que são dados perdidos?}\label{o-que-suxe3o-dados-perdidos}}

\begin{itemize}
\tightlist
\item
  Dados perdidos são dados não coletados de um ou mais participantes, para uma ou mais variáveis.\textsuperscript{\protect\hyperlink{ref-Altman2007}{56}}
\end{itemize}

\begin{infobox}{images/Rlogo}
O pacote \emph{base}\textsuperscript{\protect\hyperlink{ref-base-2}{57}} fornece a função \href{https://www.rdocumentation.org/packages/base/versions/3.6.2/topics/na}{\emph{is.na}} para identificar que elementos de um objeto são dados perdidos.

\end{infobox}

\hypertarget{qual-o-problema-de-um-estudo-ter-dados-perdidos}{%
\subsection{Qual o problema de um estudo ter dados perdidos?}\label{qual-o-problema-de-um-estudo-ter-dados-perdidos}}

\begin{itemize}
\item
  Uma grande quantidade de dados perdidos pode comprometer a integridade científica do estudo, considerando-se que o tamanho da amostra foi estimado para observar um determinado tamanho de efeito mínimo.\textsuperscript{\protect\hyperlink{ref-Altman2007}{56}}
\item
  Perda de participantes no estudo por dados perdidos pode reduzir o poder estatístico (erro tipo II).\textsuperscript{\protect\hyperlink{ref-Altman2007}{56}}
\item
  Não existe solução globalmente satisfatória para o problema de dados perdidos.\textsuperscript{\protect\hyperlink{ref-Altman2007}{56}}
\end{itemize}

\hypertarget{quais-os-mecanismos-geradores-de-dados-perdidos}{%
\subsection{Quais os mecanismos geradores de dados perdidos?}\label{quais-os-mecanismos-geradores-de-dados-perdidos}}

\begin{itemize}
\item
  Dados perdidos completamente ao acaso (\emph{missing completely at random}, MCAR), em que os dados perdidos estão distribuídos aleatoriamente nos dados da amostra.\textsuperscript{\protect\hyperlink{ref-Heymans2022}{58},\protect\hyperlink{ref-carpenter2021}{59}}
\item
  Dados perdidos ao acaso (\emph{missing at random}, MAR), em que a probabilidade de ocorrência de dados perdidos é relacionada a outras variáveis medidas.\textsuperscript{\protect\hyperlink{ref-Heymans2022}{58},\protect\hyperlink{ref-carpenter2021}{59}}
\item
  Dados perdidos não ao acaso (\emph{missing not at random}, MNAR), em que a probabilidade da ocorrência de dados perdidos é relacionada com a própria variável.\textsuperscript{\protect\hyperlink{ref-Heymans2022}{58},\protect\hyperlink{ref-carpenter2021}{59}}
\end{itemize}

\hypertarget{como-identificar-o-mecanismo-gerador-de-dados-perdidos-em-um-banco-de-dados}{%
\subsection{Como identificar o mecanismo gerador de dados perdidos em um banco de dados?}\label{como-identificar-o-mecanismo-gerador-de-dados-perdidos-em-um-banco-de-dados}}

\begin{itemize}
\item
  Por definição, não é possível avaliar se os dados foram perdidos ao acaso (MAR) ou não (MNAR).\textsuperscript{\protect\hyperlink{ref-Heymans2022}{58}}
\item
  Testes t e regressões logísticas podem ser aplicados para identificar relações entre variáveis com e sem dados perdidos, criando um fator de análise (`dado perdido' = 1, `dado observado' = 0).\textsuperscript{\protect\hyperlink{ref-Heymans2022}{58}}
\end{itemize}

\begin{infobox}{images/Rlogo}
O pacote \emph{misty}\textsuperscript{\protect\hyperlink{ref-misty}{60}} fornece a função \href{https://www.rdocumentation.org/packages/misty/versions/0.5.0/topics/na.test}{\emph{na.test}} para executar o Little's Missing Completely at Random (MCAR) test\textsuperscript{\protect\hyperlink{ref-little1988}{61}}.

\end{infobox}

\hypertarget{que-estratuxe9gias-podem-ser-utilizadas-na-coleta-de-dados-quando-huxe1-expectativa-de-perda-amostral}{%
\subsection{Que estratégias podem ser utilizadas na coleta de dados quando há expectativa de perda amostral?}\label{que-estratuxe9gias-podem-ser-utilizadas-na-coleta-de-dados-quando-huxe1-expectativa-de-perda-amostral}}

\begin{itemize}
\tightlist
\item
  Na expectativa de ocorrência de perda amostral, com consequente ocorrência de dados perdidos, recomenda-se ampliar o tamanho da amostra com um percentual correspondente a tal estimativa (ex.: 10\%), embora ainda não corrija potenciais vieses pela perda.\textsuperscript{\protect\hyperlink{ref-Altman2007}{56}}
\end{itemize}

\hypertarget{que-estratuxe9gias-podem-ser-utilizadas-na-anuxe1lise-quando-huxe1-dados-perdidos}{%
\subsection{Que estratégias podem ser utilizadas na análise quando há dados perdidos?}\label{que-estratuxe9gias-podem-ser-utilizadas-na-anuxe1lise-quando-huxe1-dados-perdidos}}

\begin{itemize}
\item
  Na ocorrência de dados perdidos, a análise mais comum compreende apenas os `casos completos', com exclusão de participantes com algum dado perdido nas variáveis do estudo. Em casos de grande quantidade de dados perdidos, pode-se perder muito poder estatístico (erro tipo II elevado).\textsuperscript{\protect\hyperlink{ref-Altman2007}{56}}
\item
  A análise de dados completos é válida quando pode-se argumentar que que a probabilidade de o participante ter dados completos depende apenas das covariáveis e não dos desfechos.\textsuperscript{\protect\hyperlink{ref-carpenter2021}{59}}
\item
  A análise de dados completos é eficiente quando todos os dados perdidos estão no desfecho, ou quando cada participante com dados perdidos nas covariáveis também possui dados perdidos nos desfechos.\textsuperscript{\protect\hyperlink{ref-carpenter2021}{59}}
\end{itemize}

\begin{infobox}{images/Rlogo}
O pacote \emph{base}\textsuperscript{\protect\hyperlink{ref-base-2}{57}} fornece a função \href{https://www.rdocumentation.org/packages/base/versions/3.6.2/topics/na.fail}{\emph{na.omit}} para remover dados perdidos de um objeto em um banco de dados.

\end{infobox}

\begin{infobox}{images/Rlogo}
O pacote \emph{stats}\textsuperscript{\protect\hyperlink{ref-stats}{62}} fornece a função \href{https://www.rdocumentation.org/packages/stats/versions/3.6.2/topics/complete.cases}{\emph{complete.cases}} para identificar os casos completos - isto é, sem dados perdidos - em um banco de dados.

\end{infobox}

\begin{itemize}
\item
  A análise com imputação de dados pode ser útil quando pode-se argumentar que os dados foram perdidos ao acaso (MAR); quando o desfecho foi observado e os dados perdidos estão nas covariáveis; e variáveis auxiliares --- preditoras do desfecho e não dos dados perdidos --- estão disponíveis.\textsuperscript{\protect\hyperlink{ref-carpenter2021}{59}}
\item
  Na ocorrência de dados perdidos, a imputação de dados (substituição por dados simulados plausíveis preditos pelos dados presentes) pode ser uma alternativa para manter o erro tipo II estipulado no plano de análise.\textsuperscript{\protect\hyperlink{ref-Altman2007}{56}}
\item
  Modelos lineares e logísticos podem ser utilizados para imputar dados perdidos em variáveis contínuas e dicotômicas, respectivamente.\textsuperscript{\protect\hyperlink{ref-austin2023}{63}}
\item
  Os métodos de imputação de dados mais robustos incluem a imputação multivariada por equações encadeadas (\emph{multivariate imputation by chained equations}, MICE)\textsuperscript{\protect\hyperlink{ref-mice}{64}} e a correspondência média preditiva (\emph{predictive mean matching}, PMM)\textsuperscript{\protect\hyperlink{ref-rubin1986}{65},\protect\hyperlink{ref-little1988a}{66}}.
\end{itemize}

\begin{infobox}{images/Rlogo}
Os pacotes \emph{mice}\textsuperscript{\protect\hyperlink{ref-mice}{64}} e \emph{miceadds}\textsuperscript{\protect\hyperlink{ref-miceadds}{67}} fornecem funções \href{https://www.rdocumentation.org/packages/mice/versions/3.16.0/topics/mice}{\emph{mice}} e \href{https://www.rdocumentation.org/packages/miceadds/versions/3.16-18/topics/mi.anova}{\emph{mi.anova}} para imputação multivariada por equações encadeadas, respectivamente, para imputação de dados.

\end{infobox}

\hypertarget{que-estratuxe9gias-podem-ser-utilizadas-na-redauxe7uxe3o-de-estudos-em-que-huxe1-dados-perdidos}{%
\subsection{Que estratégias podem ser utilizadas na redação de estudos em que há dados perdidos?}\label{que-estratuxe9gias-podem-ser-utilizadas-na-redauxe7uxe3o-de-estudos-em-que-huxe1-dados-perdidos}}

\begin{itemize}
\tightlist
\item
  Informar: o número de participantes com dados perdidos; diferenças nas taxas de dados perdidos entre os braços do estudo; os motivos dos dados perdidos; o fluxo de participantes; quaisquer diferenças entre os participantes com e sem dados perdidos; o padrão de ausência (por exemplo, se é aleatória); os métodos para tratamento de dados perdidos das variáveis em análise; os resultados de quaisquer análises de sensibilidade; as implicações dos dados perdidos na interpretação do resultados.\textsuperscript{\protect\hyperlink{ref-Akl2015}{68}}
\end{itemize}

\hypertarget{dados-anonimizados}{%
\section{Dados anonimizados}\label{dados-anonimizados}}

\hypertarget{o-que-suxe3o-dados-anonimizados}{%
\subsection{O que são dados anonimizados?}\label{o-que-suxe3o-dados-anonimizados}}

\begin{itemize}
\tightlist
\item
  .{[}REF{]}
\end{itemize}

\hypertarget{com-anonimizar-os-dados-de-um-banco}{%
\subsection{Com anonimizar os dados de um banco?}\label{com-anonimizar-os-dados-de-um-banco}}

\begin{itemize}
\tightlist
\item
  .{[}REF{]}
\end{itemize}

\begin{infobox}{images/Rlogo}
O pacote \emph{ids}\textsuperscript{\protect\hyperlink{ref-ids}{69}} fornece a função \href{https://www.rdocumentation.org/packages/ids/versions/1.0.1/topics/random_id}{\emph{random\_id}} para criar identificadores aleatórios por criptografia.

\end{infobox}

\begin{infobox}{images/Rlogo}
O pacote \emph{hash}\textsuperscript{\protect\hyperlink{ref-hash}{70}} fornece a função \href{https://www.rdocumentation.org/packages/hash/versions/3.0.1/topics/hash}{\emph{hash}} para criar identificadores por objetos \emph{hash}.

\end{infobox}

\begin{infobox}{images/Rlogo}
O pacote \emph{anonimizer}\textsuperscript{\protect\hyperlink{ref-anonymizer}{71}} fornece a função \href{https://www.rdocumentation.org/packages/anonymizer/versions/0.2.0/topics/anonymize}{\emph{anonymize}} para criar uma versão anônima de variáveis em um banco de dados.

\end{infobox}

\begin{infobox}{images/Rlogo}
O pacote \emph{digest}\textsuperscript{\protect\hyperlink{ref-digest}{72}} fornece a função \href{https://www.rdocumentation.org/packages/digest/versions/0.6.33/topics/digest}{\emph{digest}} para criar identificadores por objetos \emph{hash} criptografados ou não.

\end{infobox}

\hypertarget{metadados}{%
\section{Metadados}\label{metadados}}

\hypertarget{o-que-suxe3o-metadados}{%
\subsection{O que são metadados?}\label{o-que-suxe3o-metadados}}

\begin{itemize}
\item
  Metadados são informações técnicas relacionadas às variáveis do estudo, tais como rótulos, limites de valores plausíveis, códigos para dados perdidos e unidades de medida.\textsuperscript{\protect\hyperlink{ref-Baillie2022}{73}}
\item
  Metadados também são informações relacionadas ao delineamento e/ou protocolo do estudo, recrutamento dos participantes, e métodos para realização das medidas.\textsuperscript{\protect\hyperlink{ref-Baillie2022}{73}}
\end{itemize}

\hypertarget{quais-suxe3o-as-recomendauxe7uxf5es-para-os-metadados-de-um-banco-de-dados}{%
\subsection{Quais são as recomendações para os metadados de um banco de dados?}\label{quais-suxe3o-as-recomendauxe7uxf5es-para-os-metadados-de-um-banco-de-dados}}

\begin{itemize}
\item
  Utilize rótulos padronizados para variáveis e fatores para facilitar o reuso (reprodutibilidade) do conjuntos de dados e scripts de análise.\textsuperscript{\protect\hyperlink{ref-buttliere2021}{74}}
\item
  Crie rótulos de variáveis concisos, claros e mutuamente exclusivos.\textsuperscript{\protect\hyperlink{ref-buttliere2021}{74}}
\item
  Evite muitas letras maiúsculas ou outros caracteres especiais que usam a \emph{shift}.\textsuperscript{\protect\hyperlink{ref-buttliere2021}{74}}
\item
  Na existência de versões de instrumentos publicadas em diferentes anos, use o ano de publicação das escalas no rótulo.\textsuperscript{\protect\hyperlink{ref-buttliere2021}{74}}
\item
  Divida o rótulo da variável ou fator em partes e ordene-as do mais geral para o mais particular geral (ex.: experimento -\textgreater{} repetição -\textgreater{} escala -\textgreater{} item).\textsuperscript{\protect\hyperlink{ref-buttliere2021}{74}}
\end{itemize}

\begin{infobox}{images/Rlogo}
O pacote \emph{base}\textsuperscript{\protect\hyperlink{ref-base-3}{75}} fornece a função \href{https://www.rdocumentation.org/packages/base/versions/3.6.2/topics/names}{\emph{names}} para declarar o nome de uma variável.

\end{infobox}

\begin{infobox}{images/Rlogo}
O pacote \emph{base}\textsuperscript{\protect\hyperlink{ref-base-3}{75}} fornece a função \href{https://www.rdocumentation.org/packages/base/versions/3.6.2/topics/labels}{\emph{labels}} para declarar o rótulo de uma variável.

\end{infobox}

\begin{infobox}{images/Rlogo}
O pacote \emph{units}\textsuperscript{\protect\hyperlink{ref-units}{76}} fornece a função \href{https://www.rdocumentation.org/packages/units/versions/0.8-3/topics/units}{\emph{units}} para declarar as unidades de medida de uma variável.

\end{infobox}

\begin{infobox}{images/Rlogo}
O pacote \emph{janitor}\textsuperscript{\protect\hyperlink{ref-janitor}{77}} fornece a função \href{https://www.rdocumentation.org/packages/janitor/versions/2.2.0/topics/clean_names}{\emph{clean\_names}} para formatar de modo padronizado o nome das variáveis utilizando apenas caracteres, números e o símbolo '\_'.

\end{infobox}

\begin{infobox}{images/Rlogo}
O pacote \emph{Hmisc}\textsuperscript{\protect\hyperlink{ref-Hmisc}{78}} fornece a função \href{https://www.rdocumentation.org/packages/Hmisc/versions/5.1-0/topics/contents}{\emph{contents}} para criar um objeto com os metadados (nomes, rótulos, unidades, quantidade e níveis das variáveis categóricas, e quantidade de dados perdidos) de um dataframe.

\end{infobox}

\hypertarget{tabulacao-dados}{%
\chapter{\texorpdfstring{\textbf{Tabulação de dados}}{Tabulação de dados}}\label{tabulacao-dados}}

\hypertarget{planilhas}{%
\section{Planilhas eletrônicas}\label{planilhas}}

\hypertarget{qual-a-organizauxe7uxe3o-de-uma-tabela-de-dados}{%
\subsection{Qual a organização de uma tabela de dados?}\label{qual-a-organizauxe7uxe3o-de-uma-tabela-de-dados}}

\begin{itemize}
\item
  As informações podem ser organizadas em formato de dados retangulares (ex.: matrizes, tabelas, quadro de dados) ou não retangulares (ex.: listas).{[}REF{]}
\item
  Cada variável possui sua própria coluna (vertical).\textsuperscript{\protect\hyperlink{ref-tierney2023}{79}}
\item
  Cada observação possui sua própria linha (horizontal).\textsuperscript{\protect\hyperlink{ref-tierney2023}{79}}
\item
  Cada valor possui sua própria célula especificada em um par (linha, coluna).\textsuperscript{\protect\hyperlink{ref-tierney2023}{79}}
\item
  Cada célula possui seu próprio dado.\textsuperscript{\protect\hyperlink{ref-tierney2023}{79}}
\end{itemize}

\begin{infobox}{images/Rlogo}
O pacote \emph{DataEditR}\textsuperscript{\protect\hyperlink{ref-DataEditR}{80}} fornece a função \href{https://www.rdocumentation.org/packages/DataEditR/versions/0.1.5/topics/dataInput}{\emph{data\_edit}} para interativamente criar, editar e salvar a tabela de dados.

\end{infobox}

\hypertarget{qual-a-estrutura-buxe1sica-de-uma-tabela-para-anuxe1lise-estatuxedstica}{%
\subsection{Qual a estrutura básica de uma tabela para análise estatística?}\label{qual-a-estrutura-buxe1sica-de-uma-tabela-para-anuxe1lise-estatuxedstica}}

\begin{itemize}
\item
  Use apenas 1 (uma) planilha eletrônica para conter todas as informações coletadas. Evite múltiplas abas no mesmo arquivo, assim como múltiplos arquivos quando possível.\textsuperscript{\protect\hyperlink{ref-broman2018}{81}}
\item
  Use apenas 1 (uma) linha de cabeçalho para nomear os fatores e variáveis do seu estudo.\textsuperscript{\protect\hyperlink{ref-broman2018}{81}}
\item
  Tipicamente, cada linha representa um participante e cada coluna representa uma variável ou fator do estudo. Estudos com medidas repetidas dos participantes podem conter múltiplas linhas para o mesmo participante (repetindo os dados na mesma coluna, conhecido como \emph{formato curto}) ou só uma linha para o participante (repetindo os dados em colunas separadas, conhecido como \emph{formato longo} ).\textsuperscript{\protect\hyperlink{ref-Juluru2015}{82}}
\end{itemize}

\begin{table}

\caption{\label{tab:tabela-0}Estrutura básica de uma tabela de dados.}
\centering
\begin{tabu} to \linewidth {>{\centering}X>{\centering}X>{\centering}X>{\centering}X}
\toprule
\textbf{V1} & \textbf{V2} & \textbf{V3} & \textbf{V4}\\
\midrule
\cellcolor{gray!6}{$x_{1,1}$} & \cellcolor{gray!6}{$x_{1,2}$} & \cellcolor{gray!6}{$x_{1,3}$} & \cellcolor{gray!6}{$x_{1,4}$}\\
$x_{2,1}$ & $x_{2,2}$ & $x_{2,3}$ & $x_{2,4}$\\
\cellcolor{gray!6}{$x_{3,1}$} & \cellcolor{gray!6}{$x_{3,2}$} & \cellcolor{gray!6}{$x_{3,3}$} & \cellcolor{gray!6}{$x_{3,4}$}\\
$x_{4,1}$ & $x_{4,2}$ & $x_{4,3}$ & $x_{4,4}$\\
\cellcolor{gray!6}{$x_{5,1}$} & \cellcolor{gray!6}{$x_{5,2}$} & \cellcolor{gray!6}{$x_{5,3}$} & \cellcolor{gray!6}{$x_{5,4}$}\\
\bottomrule
\end{tabu}
\end{table}

\hypertarget{o-que-usar-para-organizar-tabelas-para-anuxe1lise-computadorizada}{%
\subsection{O que usar para organizar tabelas para análise computadorizada?}\label{o-que-usar-para-organizar-tabelas-para-anuxe1lise-computadorizada}}

\begin{itemize}
\item
  Seja consistente em: códigos para as variáveis categóricas; códigos para dados perdidos; nomes das variáveis; identificadores de participantes; nome dos arquivos; formato de datas; uso de caracteres de espaço.\textsuperscript{\protect\hyperlink{ref-broman2018}{81},\protect\hyperlink{ref-Juluru2015}{82}}
\item
  Crie um dicionário de dados (metadados) em um arquivo separado contendo: nome da variável, descrição da variável, unidades de medida e valores extremos possíveis.\textsuperscript{\protect\hyperlink{ref-broman2018}{81}}
\item
  Use recursos para validação de dados antes e durante a digitação de dados.\textsuperscript{\protect\hyperlink{ref-broman2018}{81},\protect\hyperlink{ref-Juluru2015}{82}}
\end{itemize}

\begin{infobox}{images/Rlogo}
O pacote \emph{data.table}\textsuperscript{\protect\hyperlink{ref-data.table}{83}} fornece a função \href{https://www.rdocumentation.org/packages/data.table/versions/1.14.8/topics/melt.data.table}{\emph{melt.data.table}} para reorganizar a tabela em diferentes formatos.

\end{infobox}

\hypertarget{o-que-nuxe3o-usar-para-organizar-tabelas-para-anuxe1lise-computadorizada}{%
\subsection{O que não usar para organizar tabelas para análise computadorizada?}\label{o-que-nuxe3o-usar-para-organizar-tabelas-para-anuxe1lise-computadorizada}}

\begin{itemize}
\item
  Não deixe células em branco: substitua dados perdidos por um código sistemático (ex.: NA {[}\emph{not available}{]}).\textsuperscript{\protect\hyperlink{ref-broman2018}{81}}
\item
  Não inclua análises estatísticas ou gráficos nas tabelas de dados brutos.\textsuperscript{\protect\hyperlink{ref-broman2018}{81}}
\item
  Não utilize cores como informação. Se necessário, crie colunas adicionais - variáveis instrumentais ou auxiliares - para identificar a informação de modo que possa ser analisada.\textsuperscript{\protect\hyperlink{ref-broman2018}{81}}
\item
  Não use células mescladas.
\item
  Delete linhas e/ou colunas totalmente em branco (sem unidades de análise e/ou sem variáveis).
\end{itemize}

\hypertarget{o-que-uxe9-recomendado-e-o-que-deve-ser-evitado-na-organizauxe7uxe3o-das-tabelas-para-anuxe1lise}{%
\subsection{O que é recomendado e o que deve ser evitado na organização das tabelas para análise?}\label{o-que-uxe9-recomendado-e-o-que-deve-ser-evitado-na-organizauxe7uxe3o-das-tabelas-para-anuxe1lise}}

\begin{table}

\caption{\label{tab:tabela-recomendada}Formatação recomendada para tabela de dados.}
\centering
\begin{tabu} to \linewidth {>{\raggedright}X>{\raggedright}X>{\raggedright}X>{\raggedright}X}
\toprule
\textbf{ID} & \textbf{Data.Coleta} & \textbf{Estado.Civil} & \textbf{Numero.Filhos}\\
\midrule
\cellcolor{gray!6}{1} & \cellcolor{gray!6}{13-11-2023} & \cellcolor{gray!6}{casado} & \cellcolor{gray!6}{NA}\\
2 & 14-11-2023 & casado & 1\\
\cellcolor{gray!6}{3} & \cellcolor{gray!6}{15-11-2023} & \cellcolor{gray!6}{casado} & \cellcolor{gray!6}{NA}\\
4 & 16-11-2023 & solteiro & NA\\
\cellcolor{gray!6}{5} & \cellcolor{gray!6}{17-11-2023} & \cellcolor{gray!6}{casado} & \cellcolor{gray!6}{NA}\\
6 & 18-11-2023 & solteiro & 0\\
\cellcolor{gray!6}{7} & \cellcolor{gray!6}{19-11-2023} & \cellcolor{gray!6}{solteiro} & \cellcolor{gray!6}{NA}\\
8 & 20-11-2023 & solteiro & NA\\
\cellcolor{gray!6}{9} & \cellcolor{gray!6}{21-11-2023} & \cellcolor{gray!6}{casado} & \cellcolor{gray!6}{NA}\\
10 & 22-11-2023 & solteiro & NA\\
\bottomrule
\end{tabu}
\end{table}

\begin{table}

\caption{\label{tab:tabela-evite}Formatação não recomendada para tabela de dados.}
\centering
\begin{tabu} to \linewidth {>{\raggedright}X>{\raggedright}X>{\raggedright}X>{\raggedright}X}
\toprule
\textbf{ID} & \textbf{Data de Coleta} & \textbf{Estado Civil} & \textbf{Número de Filhos}\\
\midrule
\cellcolor{gray!6}{1} & \cellcolor{gray!6}{13-11-2023} & \cellcolor{gray!6}{casado} & \cellcolor{gray!6}{NA}\\
2 & 14-11-2023 & Casado & 1\\
\cellcolor{gray!6}{3} & \cellcolor{gray!6}{15-11-2023} & \cellcolor{gray!6}{casado} & \cellcolor{gray!6}{NaN}\\
4 & 16-11-2023 & Solteiro & N/A\\
\cellcolor{gray!6}{5} & \cellcolor{gray!6}{17-11-2023} & \cellcolor{gray!6}{Casado} & \cellcolor{gray!6}{N.A.}\\
6 & 18-11-2023 & solteiro & 0\\
\cellcolor{gray!6}{7} & \cellcolor{gray!6}{19-11-2023} & \cellcolor{gray!6}{solteiro} & \cellcolor{gray!6}{}\\
8 & 20-11-2023 & Solteiro & na\\
\cellcolor{gray!6}{9} & \cellcolor{gray!6}{21-11-2023} & \cellcolor{gray!6}{casado} & \cellcolor{gray!6}{n.a.}\\
10 & 22-11-2023 & Solteiro & 999\\
\bottomrule
\end{tabu}
\end{table}

\hypertarget{variaveis-fatores}{%
\chapter{\texorpdfstring{\textbf{Variáveis e fatores}}{Variáveis e fatores}}\label{variaveis-fatores}}

\hypertarget{variaveis}{%
\section{Variáveis}\label{variaveis}}

\hypertarget{o-que-suxe3o-variuxe1veis}{%
\subsection{O que são variáveis?}\label{o-que-suxe3o-variuxe1veis}}

\begin{itemize}
\item
  Variáveis são informações que podem variar entre medidas em diferentes indivíduos e/ou repetições.\textsuperscript{\protect\hyperlink{ref-Altman1999}{84}}
\item
  Variáveis definem características de uma amostra extraída da população, tipicamente observados por aplicação de métodos de amostragem (isto é, seleção) da população de interesse.\textsuperscript{\protect\hyperlink{ref-vetter2017}{55}}
\end{itemize}

\hypertarget{como-suxe3o-classificadas-as-variuxe1veis}{%
\subsection{Como são classificadas as variáveis?}\label{como-suxe3o-classificadas-as-variuxe1veis}}

\begin{itemize}
\item
  Quanto à informação:\textsuperscript{\protect\hyperlink{ref-vetter2017}{55},\protect\hyperlink{ref-Ali2016}{85}--\protect\hyperlink{ref-kaliyadan2019}{87}}

  \begin{itemize}
  \item
    Quantitativa
  \item
    Qualitativa
  \end{itemize}
\item
  Quanto ao conteúdo:\textsuperscript{\protect\hyperlink{ref-vetter2017}{55},\protect\hyperlink{ref-Ali2016}{85}--\protect\hyperlink{ref-barkan2015}{88}}

  \begin{itemize}
  \item
    Contínua: representam ordem e magnitude entre valores.

    \begin{itemize}
    \item
      Contínua (números inteiros) vs.~Discreta (números racionais).
    \item
      Intervalo (valor `0' é arbitrário) vs.~Razão (valor `0' verdadeiro).
    \end{itemize}
  \item
    Categórica ordinal (numérica ou nominal): representam ordem, mas não magnitude entre valores.
  \item
    Categórica nominal (multinominal ou dicotômica): não representam ordem ou magnitude, apenas categorias.
  \end{itemize}
\item
  Quanto à interpretação:\textsuperscript{\protect\hyperlink{ref-vetter2017}{55},\protect\hyperlink{ref-Ali2016}{85}--\protect\hyperlink{ref-kaliyadan2019}{87}}

  \begin{itemize}
  \item
    Dependente (desfecho)
  \item
    Independente (preditora, covariável, confundidora, controle)
  \item
    Mediadora
  \item
    Moderadora
  \item
    Modificadora
  \item
    Auxiliar
  \item
    Indicadora
  \end{itemize}
\end{itemize}

\begin{infobox}{images/Rlogo}
O pacote \emph{base}\textsuperscript{\protect\hyperlink{ref-base-2}{57}} fornece a função \href{https://www.rdocumentation.org/packages/base/versions/3.6.2/topics/class}{\emph{class}} para identificar qual é o tipo do objeto.

\end{infobox}

\begin{infobox}{images/Rlogo}
O pacote \emph{base}\textsuperscript{\protect\hyperlink{ref-base-2}{57}} fornece as funções \href{https://www.rdocumentation.org/packages/base/versions/3.6.2/topics/numeric}{\emph{as.numeric}} e \href{https://www.rdocumentation.org/packages/base/versions/3.6.2/topics/character}{\emph{as.character}} para criar objetos numéricos e categóricos, respectivamente.

\end{infobox}

\begin{infobox}{images/Rlogo}
O pacote \emph{base}\textsuperscript{\protect\hyperlink{ref-base-2}{57}} fornece as funções \href{https://www.rdocumentation.org/packages/base/versions/3.6.2/topics/as.Date}{\emph{as.Date}} e \href{https://www.rdocumentation.org/packages/base/versions/3.6.2/topics/logical}{\emph{as.logical}} para criar objetos em formato de data e lógicos (VERDADEIRO, FALSO), respectivamente.

\end{infobox}

\hypertarget{por-que-uxe9-importante-classificar-as-variuxe1veis}{%
\subsection{Por que é importante classificar as variáveis?}\label{por-que-uxe9-importante-classificar-as-variuxe1veis}}

\begin{itemize}
\tightlist
\item
  Identificar corretamente os tipos de variáveis da pesquisa é uma das etapas da escolha dos métodos estatísticos adequados para as análises e representações no texto, tabelas e gráficos.\textsuperscript{\protect\hyperlink{ref-Dettori2018}{86}}
\end{itemize}

\hypertarget{transformacao}{%
\section{Transformação de variáveis}\label{transformacao}}

\hypertarget{o-que-uxe9-transformauxe7uxe3o-de-variuxe1veis}{%
\subsection{O que é transformação de variáveis?}\label{o-que-uxe9-transformauxe7uxe3o-de-variuxe1veis}}

\begin{itemize}
\item
  Transformação significa aplicar uma função matemática à variável medida em sua unidade original.\textsuperscript{\protect\hyperlink{ref-Bland1996}{89}}
\item
  A transformação visa atender aos pressupostos dos modelos estatísticos quanto à distribuição da variável, em geral a distribuição gaussiana.\textsuperscript{\protect\hyperlink{ref-vetter2017}{55},\protect\hyperlink{ref-Bland1996}{89}}
\item
  A dicotomização pode ser interpretada como um caso particular de agrupamento.\textsuperscript{\protect\hyperlink{ref-Fedorov2009}{90}}
\end{itemize}

\hypertarget{por-que-transformar-variuxe1veis}{%
\subsection{Por que transformar variáveis?}\label{por-que-transformar-variuxe1veis}}

\begin{itemize}
\item
  Muitos procedimentos estatísticos supõem que as variáveis - ou seus termos de erro, mais especificamente - são normalmente distribuídas. A violação dessa suposição pode aumentar suas chances de cometer um erro do tipo I ou II.\textsuperscript{\protect\hyperlink{ref-osborne2010}{91}}
\item
  Mesmo quando se está usando análises consideradas robustas para violações dessas suposições ou testes não paramétricos (que não assumem explicitamente termos de erro normalmente distribuídos), atender a essas questões pode melhorar os resultados das análises (por exemplo, Zimmerman, 1995).\textsuperscript{\protect\hyperlink{ref-osborne2010}{91}}
\end{itemize}

\hypertarget{quais-transformauxe7uxf5es-podem-ser-aplicadas}{%
\subsection{Quais transformações podem ser aplicadas?}\label{quais-transformauxe7uxf5es-podem-ser-aplicadas}}

\begin{itemize}
\item
  Distribuições com assimetria à direita:\textsuperscript{\protect\hyperlink{ref-osborne2010}{91}}

  \begin{itemize}
  \item
    Raiz quadrada
  \item
    Logaritmo natural
  \item
    Logaritmo base 10
  \item
    Transformação inversa
  \end{itemize}
\item
  Distribuições com assimetria à esquerda:\textsuperscript{\protect\hyperlink{ref-osborne2010}{91}}

  \begin{itemize}
  \item
    Reflexão e raiz quadrada
  \item
    Reflexão e logaritmo natural
  \item
    Reflexão e logaritmo base 10
  \item
    Reflexão e transformação inversa
  \end{itemize}
\item
  Transformação arco-seno.\textsuperscript{\protect\hyperlink{ref-osborne2010}{91}}
\item
  Transformação de Box-Cox.\textsuperscript{\protect\hyperlink{ref-box1964}{92}}
\item
  Dicotomização.
\end{itemize}

\begin{infobox}{images/Rlogo}
O pacote \emph{MASS}\textsuperscript{\protect\hyperlink{ref-MASS}{93}} fornece a função \href{https://www.rdocumentation.org/packages/MASS/versions/7.3-58.3/topics/boxcox}{\emph{boxcox}} para executar a transformação de Box-Cox.\textsuperscript{\protect\hyperlink{ref-box1964}{92}}

\end{infobox}

\hypertarget{categorizacao}{%
\section{Categorização de variáveis contínuas}\label{categorizacao}}

\hypertarget{o-que-uxe9-categorizauxe7uxe3o-de-uma-variuxe1vel}{%
\subsection{O que é categorização de uma variável?}\label{o-que-uxe9-categorizauxe7uxe3o-de-uma-variuxe1vel}}

\begin{itemize}
\tightlist
\item
  .{[}REF{]}
\end{itemize}

\hypertarget{por-que-nuxe3o-uxe9-recomendado-categorizar-variuxe1veis-contuxednuas}{%
\subsection{Por que não é recomendado categorizar variáveis contínuas?}\label{por-que-nuxe3o-uxe9-recomendado-categorizar-variuxe1veis-contuxednuas}}

\begin{itemize}
\item
  Nenhum dos argumentos usados para defender a categorização de variáveis se sustenta sob uma análise técnica rigorosa.\textsuperscript{\protect\hyperlink{ref-MacCallum2002}{94}}
\item
  Categorizar variáveis não é necessário para conduzir análises estatísticas. Ao invés de categorizar, priorize as variáveis contínuas.\textsuperscript{\protect\hyperlink{ref-Altman2006}{95}--\protect\hyperlink{ref-Collins2016}{97}}
\item
  Em geral, não existe uma justificativa racional (plausibilidade biológica) para assumir que as categorias artificiais subjacentes existam.\textsuperscript{\protect\hyperlink{ref-Altman2006}{95}--\protect\hyperlink{ref-Collins2016}{97}}
\item
  Caso exista um ponto de corte ou limiar verdadeiro que discrimine três ou mais grupos independentes, identificar tal ponto de corte ainda é um desafio.\textsuperscript{\protect\hyperlink{ref-Prince2017}{98}}
\item
  Categorização de variáveis contínuas aumenta a quantidade de testes de hipótese para comparações pareadas entre os quantis, inflando, portanto, o erro tipo I.\textsuperscript{\protect\hyperlink{ref-Bennette2012}{99}}
\item
  Categorização de variáveis contínuas requer uma função teórica que pressupõe a homogeneidade da variável dentro dos grupos, levando tanto a uma perda de poder como a uma estimativa imprecisa.\textsuperscript{\protect\hyperlink{ref-Bennette2012}{99}}
\item
  Categorização de variáveis contínuas pode dificultar a comparação de resultados entre estudos devido aos pontos de corte baseados em dados de um banco usados para definir as categorias.\textsuperscript{\protect\hyperlink{ref-Bennette2012}{99}}
\end{itemize}

\hypertarget{quais-suxe3o-as-alternativas-uxe0-categorizauxe7uxe3o-de-variuxe1veis-contuxednuas}{%
\subsection{Quais são as alternativas à categorização de variáveis contínuas?}\label{quais-suxe3o-as-alternativas-uxe0-categorizauxe7uxe3o-de-variuxe1veis-contuxednuas}}

\begin{itemize}
\item
  Análise com os dados das variáveis na escala de medida original.\textsuperscript{\protect\hyperlink{ref-MacCallum2002}{94}}
\item
  Análise com modelos de regressão com pesos locais (\emph{lowess}) tais como \emph{splines} e polinômios fracionais.\textsuperscript{\protect\hyperlink{ref-MacCallum2002}{94}}
\end{itemize}

\hypertarget{dicotomizacao}{%
\section{Dicotomização de variáveis contínuas}\label{dicotomizacao}}

\hypertarget{o-que-suxe3o-variuxe1veis-dicotuxf4micas}{%
\subsection{O que são variáveis dicotômicas?}\label{o-que-suxe3o-variuxe1veis-dicotuxf4micas}}

\begin{itemize}
\item
  Variáveis dicotômicas (ou binárias) podem representar categorias naturais tipo ``presente/ausente'', ``sim/não''.{[}REF{]}
\item
  Variáveis dicotômicas podem representar categorias fictícias, criadas a partir de variáveis multinominais, em que cada nível é convertido em uma variável dicotômica \emph{dummy}.{[}REF{]}
\item
  Geralmente são representadas por ``1'' e ``0''.{[}REF{]}
\end{itemize}

\hypertarget{quais-argumentos-suxe3o-usados-para-defender-a-categorizauxe7uxe3o-ou-dicotomizauxe7uxe3o-de-variuxe1veis-contuxednuas}{%
\subsection{Quais argumentos são usados para defender a categorização ou dicotomização de variáveis contínuas?}\label{quais-argumentos-suxe3o-usados-para-defender-a-categorizauxe7uxe3o-ou-dicotomizauxe7uxe3o-de-variuxe1veis-contuxednuas}}

\begin{itemize}
\item
  O argumento principal para dicotomização de variáveis é que tal procedimento facilita e simplifica a apresentação dos resultados, principalmente para o público em geral.\textsuperscript{\protect\hyperlink{ref-Fedorov2009}{90}}
\item
  Os pesquisadores não conhecem as consequências estatísticas da dicotomização.\textsuperscript{\protect\hyperlink{ref-MacCallum2002}{94}}
\item
  Os pesquisadores não conhecem os métodos adequados de análise não-paramétrica, não-linear e robusta.\textsuperscript{\protect\hyperlink{ref-MacCallum2002}{94}}
\item
  As categorias representam características existentes dos participantes da pesquisa, de modo que as análises devam ser feitas por grupos e não por indivíduos.\textsuperscript{\protect\hyperlink{ref-MacCallum2002}{94}}
\item
  A confiabilidade da(s) variável(eis) medida(s) é baixa e, portanto, categorizar os participantes resultaria em uma medida mais confiável.\textsuperscript{\protect\hyperlink{ref-MacCallum2002}{94}}
\end{itemize}

\hypertarget{por-que-nuxe3o-uxe9-recomendado-dicotomizar-variuxe1veis-contuxednuas}{%
\subsection{Por que não é recomendado dicotomizar variáveis contínuas?}\label{por-que-nuxe3o-uxe9-recomendado-dicotomizar-variuxe1veis-contuxednuas}}

\begin{itemize}
\item
  Nenhum dos argumentos usados para defender a dicotomização de variáveis se sustenta sob uma análise técnica rigorosa.\textsuperscript{\protect\hyperlink{ref-MacCallum2002}{94}}
\item
  Dicotomizar variáveis não é necessário para conduzir análises estatísticas. Ao invés de dicotomizar, priorize as variáveis contínuas.\textsuperscript{\protect\hyperlink{ref-Altman2006}{95}--\protect\hyperlink{ref-Collins2016}{97}}
\item
  Em geral, não existe uma justificativa racional (plausibilidade biológica) para assumir que as categorias artificiais subjacentes existam.\textsuperscript{\protect\hyperlink{ref-Altman2006}{95}--\protect\hyperlink{ref-Collins2016}{97}}
\item
  Dicotomização causa perda de informação e consequentemente perda de poder estatístico para detectar efeitos.\textsuperscript{\protect\hyperlink{ref-MacCallum2002}{94},\protect\hyperlink{ref-Altman2006}{95}}
\item
  Dicotomização também classifica indivíduos com valores próximos na variável contínua como indivíduos em pontos opostos e extremos, artificialmente sugerindo que são muito diferentes.\textsuperscript{\protect\hyperlink{ref-Altman2006}{95}}
\item
  Dicotomização pode diminuir a variabilidade das variáveis.\textsuperscript{\protect\hyperlink{ref-Altman2006}{95}}
\item
  Dicotomização pode ocultar não-linearidades presentes na variável contínua.\textsuperscript{\protect\hyperlink{ref-MacCallum2002}{94},\protect\hyperlink{ref-Altman2006}{95}}
\item
  A média ou a mediana, embora amplamente utilizadas, não são bons parâmetros para dicotomizar variáveis.\textsuperscript{\protect\hyperlink{ref-Fedorov2009}{90},\protect\hyperlink{ref-Altman2006}{95}}
\item
  Caso exista um ponto de corte ou limiar verdadeiro que discrimine dois grupos independentes, identificar tal ponto de corte ainda é um desafio.\textsuperscript{\protect\hyperlink{ref-Prince2017}{98}}
\end{itemize}

\hypertarget{quais-cenuxe1rios-legitimam-a-dicotomizauxe7uxe3o-das-variuxe1veis-contuxednuas}{%
\subsection{Quais cenários legitimam a dicotomização das variáveis contínuas?}\label{quais-cenuxe1rios-legitimam-a-dicotomizauxe7uxe3o-das-variuxe1veis-contuxednuas}}

\begin{itemize}
\item
  Quando existem dados e/ou análises que suportem a existência - não apenas a suposição ou teorização - de categorias com um ponto de corte claro e com significado entre elas.\textsuperscript{\protect\hyperlink{ref-MacCallum2002}{94}}
\item
  Quando a distribuição da variável contínua é muito assimétrica, de modo que uma grande quantidade de observações está em um dos extremos da escala.\textsuperscript{\protect\hyperlink{ref-MacCallum2002}{94}}
\end{itemize}

\hypertarget{quais-muxe9todos-suxe3o-usados-para-dicotomizar-variuxe1veis-contuxednuas}{%
\subsection{Quais métodos são usados para dicotomizar variáveis contínuas?}\label{quais-muxe9todos-suxe3o-usados-para-dicotomizar-variuxe1veis-contuxednuas}}

\begin{itemize}
\item
  Em termos de tabelas de contingência 2x2, os seguintes métodos permitem\textsuperscript{\protect\hyperlink{ref-Prince2017}{98}} a identificação do limiar verdadeiro:

  \begin{itemize}
  \item
    Youden.\textsuperscript{\protect\hyperlink{ref-YOUDEN1950}{100}}
  \item
    Gini Index.\textsuperscript{\protect\hyperlink{ref-strobl2007}{101}}
  \item
    Estatística qui-quadrado (\(\chi^2\)).\textsuperscript{\protect\hyperlink{ref-pearson1900}{102}}
  \item
    Risco relativo (\(RR\)).\textsuperscript{\protect\hyperlink{ref-Greiner2000}{103}}
  \item
    Kappa (\(\kappa\)).\textsuperscript{\protect\hyperlink{ref-fleiss1971}{104}}.
  \end{itemize}
\end{itemize}

\hypertarget{fatores}{%
\section{Fatores}\label{fatores}}

\hypertarget{o-que-suxe3o-fatores}{%
\subsection{O que são fatores?}\label{o-que-suxe3o-fatores}}

\begin{itemize}
\item
  Fator é um sinônimo de variável categórica.{[}REF{]}
\item
  Na modelagem, fator é sinônimo de variável preditora, em particular quando se refere à modelagem de efeitos fixos e aleatórios -- os fatores (variáveis) são fatores fixos ou fatores aleatórios.{[}REF{]}
\item
  Fatores são variáveis controladas pelos pesquisadores em um experimento para determinar seu efeito na(s) variável(ies) de resposta. Um fator pode assumir apenas um pequeno número de valores, conhecidos como níveis. Os fatores podem ser uma variável categórica ou baseados em uma variável contínua, mas usam apenas um número limitado de valores escolhidos pelos experimentadores.{[}REF{]}
\end{itemize}

\begin{infobox}{images/Rlogo}
O pacote \emph{base}\textsuperscript{\protect\hyperlink{ref-base-2}{57}} fornece a função \href{https://www.rdocumentation.org/packages/base/versions/3.6.2/topics/factor}{\emph{as.factor}} para converter uma variável em fator.

\end{infobox}

\hypertarget{o-que-suxe3o-nuxedveis-de-um-fator}{%
\subsection{O que são níveis de um fator?}\label{o-que-suxe3o-nuxedveis-de-um-fator}}

\begin{itemize}
\tightlist
\item
  Níveis de um fator são as possíveis categorias que descrevem um fator.{[}REF{]}
\end{itemize}

\begin{infobox}{images/Rlogo}
O pacote \emph{base}\textsuperscript{\protect\hyperlink{ref-base-2}{57}} fornece as funções \href{https://www.rdocumentation.org/packages/base/versions/3.6.2/topics/levels}{\emph{levels}} e \href{https://www.rdocumentation.org/packages/base/versions/3.6.2/topics/nlevels}{\emph{nlevels}} para listar os níveis e a quantidade deles em um fator.

\end{infobox}

\hypertarget{distribuicoes-parametros}{%
\chapter{\texorpdfstring{\textbf{Distribuições e parâmetros}}{Distribuições e parâmetros}}\label{distribuicoes-parametros}}

\hypertarget{distribuicoes}{%
\section{Distribuições de probabilidade}\label{distribuicoes}}

\hypertarget{o-que-suxe3o-distribuiuxe7uxf5es-de-probabilidade}{%
\subsection{O que são distribuições de probabilidade?}\label{o-que-suxe3o-distribuiuxe7uxf5es-de-probabilidade}}

\begin{itemize}
\tightlist
\item
  Uma distribuição de probabilidade é uma função que descreve os valores possíveis ou o intervalo de valores de uma variável (eixo horizontal) e a frequência com que cada valor é observado (eixo vertical).\textsuperscript{\protect\hyperlink{ref-vetter2017}{55}}
\end{itemize}

\hypertarget{quais-caracteruxedsticas-definem-uma-distribuiuxe7uxe3o}{%
\subsection{Quais características definem uma distribuição?}\label{quais-caracteruxedsticas-definem-uma-distribuiuxe7uxe3o}}

\begin{itemize}
\tightlist
\item
  Uma distribuição pode ser definida por modelos matemáticos e caracterizada por sua tendência central, dispersão, simetria e curtose.
\end{itemize}

\hypertarget{quais-suxe3o-as-funuxe7uxf5es-de-uma-distribuiuxe7uxe3o}{%
\subsection{Quais são as funções de uma distribuição?}\label{quais-suxe3o-as-funuxe7uxf5es-de-uma-distribuiuxe7uxe3o}}

\begin{itemize}
\item
  Função de massa de probabilidade (\emph{probability mass function}, pmf).{[}REF{]}
\item
  Função de distribuição cumulativa (\emph{cumulative distribution function}, cdf).{[}REF{]}
\item
  Função quantílicas (\emph{quantile function}, qf).{[}REF{]}
\item
  Função geradora de números aleatórios (\emph{random function}, rf).{[}REF{]}
\end{itemize}

\begin{infobox}{images/Rlogo}
O pacote \emph{stats}\textsuperscript{\protect\hyperlink{ref-stats-3}{105}} fornece funções de distribuição de probabilidade (p), funções de densidade (d), funções quantílicas (q) e funções geradores de números aleatórios (r) para as distribuições \href{https://www.rdocumentation.org/packages/stats/versions/3.6.2/topics/Normal}{normal}, \href{https://www.rdocumentation.org/packages/stats/versions/3.6.2/topics/TDist}{Student t}, \href{https://www.rdocumentation.org/packages/stats/versions/3.6.2/topics/Binomial}{binomial}, \href{https://www.rdocumentation.org/packages/stats/versions/3.6.2/topics/Chisquare}{qui-quadrado}, \href{https://www.rdocumentation.org/packages/stats/versions/3.6.2/topics/Uniform}{uniforme}, dentre outras.

\end{infobox}

\hypertarget{o-que-uxe9-a-distribuiuxe7uxe3o-normal}{%
\subsection{O que é a distribuição normal?}\label{o-que-uxe9-a-distribuiuxe7uxe3o-normal}}

\begin{itemize}
\item
  A distribuição normal (ou gaussiana) é uma distribuição com desvios simétricos positivos e negativos em torno de um valor central.\textsuperscript{\protect\hyperlink{ref-Ali2016}{85}}
\item
  Em uma distribuição normal, o intervalo de 1 desvio-padrão (±1DP) inclui cerca de 68\% dos dados; de 2 desvios-padrão (±2DP) cerca de 95\% dos dados; e no intervalo de 3 desvios-padrão (±3DP) cerca de 99\% dos dados.\textsuperscript{\protect\hyperlink{ref-Ali2016}{85}}
\end{itemize}

\hypertarget{o-que-suxe3o-distribuiuxe7uxf5es-nuxe3o-normais}{%
\subsection{O que são distribuições não-normais?}\label{o-que-suxe3o-distribuiuxe7uxf5es-nuxe3o-normais}}

\begin{itemize}
\tightlist
\item
  .{[}REF{]}
\end{itemize}

\hypertarget{que-muxe9todos-podem-ser-utilizados-para-identificar-a-normalidade-da-distribuiuxe7uxe3o}{%
\subsection{Que métodos podem ser utilizados para identificar a normalidade da distribuição?}\label{que-muxe9todos-podem-ser-utilizados-para-identificar-a-normalidade-da-distribuiuxe7uxe3o}}

\begin{itemize}
\item
  Histogramas.\textsuperscript{\protect\hyperlink{ref-vetter2017}{55}}
\item
  Gráficos Q-Q.\textsuperscript{\protect\hyperlink{ref-vetter2017}{55}}
\item
  Testes de hipótese nula:\textsuperscript{\protect\hyperlink{ref-vetter2017}{55}}

  \begin{itemize}
  \item
    Kolmogorov-Smirnov
  \item
    Shapiro-Wilk
  \item
    Anderson-Darling
  \end{itemize}
\end{itemize}

\hypertarget{parametros}{%
\section{Parâmetros}\label{parametros}}

\hypertarget{o-que-suxe3o-paruxe2metros}{%
\subsection{O que são parâmetros?}\label{o-que-suxe3o-paruxe2metros}}

\begin{itemize}
\item
  Parâmetros são informações que definem um modelo teórico, como propriedades de uma coleção de indivíduos.\textsuperscript{\protect\hyperlink{ref-Altman1999}{84}}
\item
  Parâmetros definem características de uma população inteira, tipicamente não observados por ser inviável ter acesso a todos os indivíduos que constituem tal população.\textsuperscript{\protect\hyperlink{ref-vetter2017}{55}}
\end{itemize}

\hypertarget{que-paruxe2metros-podem-ser-estimados}{%
\subsection{Que parâmetros podem ser estimados?}\label{que-paruxe2metros-podem-ser-estimados}}

\begin{itemize}
\item
  Parâmetros de tendência central.\textsuperscript{\protect\hyperlink{ref-Ali2016}{85},\protect\hyperlink{ref-kanji2006}{106}}
\item
  Parâmetros de dispersão.\textsuperscript{\protect\hyperlink{ref-Ali2016}{85},\protect\hyperlink{ref-kanji2006}{106},\protect\hyperlink{ref-Curran-Everett2008}{107}}
\item
  Parâmetros de proporção.\textsuperscript{\protect\hyperlink{ref-Ali2016}{85},\protect\hyperlink{ref-kanji2006}{106},\protect\hyperlink{ref-Altman1994}{108},\protect\hyperlink{ref-Altman1994}{108}}
\item
  Parâmetros de distribuição.\textsuperscript{\protect\hyperlink{ref-kanji2006}{106}}
\item
  Parâmetros de extremos.\textsuperscript{\protect\hyperlink{ref-Ali2016}{85}}
\end{itemize}

\hypertarget{o-que-uxe9-uma-anuxe1lise-paramuxe9trica}{%
\subsection{O que é uma análise paramétrica?}\label{o-que-uxe9-uma-anuxe1lise-paramuxe9trica}}

\begin{itemize}
\item
  Testes paramétricos possuem suposições sobre as características e/ou parâmetros da distribuição dos dados na população.\textsuperscript{\protect\hyperlink{ref-vetter2017}{55}}
\item
  Testes paramétricos assumem que: a variável é quantitativa numérica (contínua); os dados foram amostrados de uma população com distribuição normal; a variância da(S) amostra(s) é igual à da população; as amostras foram selecionadas de modo aleatório na população; os valores de cada amostra são independentes entre si.\textsuperscript{\protect\hyperlink{ref-vetter2017}{55},\protect\hyperlink{ref-Ali2016}{85}}
\end{itemize}

\hypertarget{o-que-uxe9-uma-anuxe1lise-nuxe3o-paramuxe9trica}{%
\subsection{O que é uma análise não paramétrica?}\label{o-que-uxe9-uma-anuxe1lise-nuxe3o-paramuxe9trica}}

\begin{itemize}
\item
  Testes não-paramétricos fazem poucas suposições, ou menos rigorosas, sobre as características e/ou parâmetros da distribuição dos dados na população.\textsuperscript{\protect\hyperlink{ref-vetter2017}{55},\protect\hyperlink{ref-Ali2016}{85}}
\item
  Testes não-paramétricos são úteis quando as suposições de normalidade não podem ser sustentadas.\textsuperscript{\protect\hyperlink{ref-Ali2016}{85}}
\end{itemize}

\hypertarget{por-que-as-anuxe1lises-paramuxe9tricas-suxe3o-preferidas}{%
\subsection{Por que as análises paramétricas são preferidas?}\label{por-que-as-anuxe1lises-paramuxe9tricas-suxe3o-preferidas}}

\begin{itemize}
\item
  Em geral, testes paramétricos são mais robustos (isto é, possuem menores erros tipo I e II) que seus testes não-paramétricos correspondentes.\textsuperscript{\protect\hyperlink{ref-vetter2017}{55},\protect\hyperlink{ref-greenhalgh1997}{109}}
\item
  Testes não-paramétricos apresentam menor poder estatístico (maior erro tipo II) comparados aos testes paramétricos correspondentes.\textsuperscript{\protect\hyperlink{ref-Ali2016}{85}}
\end{itemize}

\#\# Análise inferencial \{\#inferencial\}

\hypertarget{valores-esperados}{%
\section{Valores esperados}\label{valores-esperados}}

\hypertarget{que-paruxe2metros-de-tenduxeancia-central-podem-ser-estimados}{%
\subsection{Que parâmetros de tendência central podem ser estimados?}\label{que-paruxe2metros-de-tenduxeancia-central-podem-ser-estimados}}

\begin{itemize}
\item
  \emph{Média}.\textsuperscript{\protect\hyperlink{ref-Ali2016}{85},\protect\hyperlink{ref-kanji2006}{106}}
\item
  \emph{Mediana}.\textsuperscript{\protect\hyperlink{ref-Ali2016}{85},\protect\hyperlink{ref-kanji2006}{106}}
\item
  \emph{Moda}.\textsuperscript{\protect\hyperlink{ref-Ali2016}{85},\protect\hyperlink{ref-kanji2006}{106}}
\end{itemize}

\hypertarget{que-paruxe2metros-de-dispersuxe3o-podem-ser-estimados}{%
\subsection{Que parâmetros de dispersão podem ser estimados?}\label{que-paruxe2metros-de-dispersuxe3o-podem-ser-estimados}}

\begin{itemize}
\item
  \emph{Variância}.\textsuperscript{\protect\hyperlink{ref-Ali2016}{85},\protect\hyperlink{ref-kanji2006}{106}}
\item
  \emph{Desvio-padrão}: Estima a variabilidade entre as observações e a média amostra, e estima a variabilidade na população.\textsuperscript{\protect\hyperlink{ref-Curran-Everett2008}{107}}
\item
  \emph{Erro-padrão}: Estima a variabilidade teórica entre médias amostrais.\textsuperscript{\protect\hyperlink{ref-Curran-Everett2008}{107}}
\item
  \emph{Amplitude}.\textsuperscript{\protect\hyperlink{ref-Ali2016}{85},\protect\hyperlink{ref-kanji2006}{106}}
\item
  \emph{Intervalo interquartil}.\textsuperscript{\protect\hyperlink{ref-Ali2016}{85},\protect\hyperlink{ref-kanji2006}{106}}
\item
  \emph{Intervalo de confiança}.\textsuperscript{\protect\hyperlink{ref-Ali2016}{85},\protect\hyperlink{ref-kanji2006}{106}}
\end{itemize}

\hypertarget{que-paruxe2metros-de-proporuxe7uxe3o-podem-ser-estimados}{%
\subsection{Que parâmetros de proporção podem ser estimados?}\label{que-paruxe2metros-de-proporuxe7uxe3o-podem-ser-estimados}}

\begin{itemize}
\item
  \emph{Frequência absoluta}.\textsuperscript{\protect\hyperlink{ref-Ali2016}{85},\protect\hyperlink{ref-kanji2006}{106},\protect\hyperlink{ref-Altman1994}{108}}
\item
  \emph{Frequência relativa}.\textsuperscript{\protect\hyperlink{ref-Ali2016}{85},\protect\hyperlink{ref-kanji2006}{106},\protect\hyperlink{ref-Altman1994}{108}}
\item
  \emph{Percentil}.\textsuperscript{\protect\hyperlink{ref-Ali2016}{85},\protect\hyperlink{ref-kanji2006}{106},\protect\hyperlink{ref-Altman1994}{108}}
\item
  \emph{Quantil}: é o ponto de corte que define a divisão da amostra em grupos de tamanhos iguais. Portanto, não se referem aos grupos em si, mas aos valores que os dividem.\textsuperscript{\protect\hyperlink{ref-Altman1994}{108}}
\end{itemize}

\begin{infobox}{images/Rlogo}
O pacote \emph{stats}\textsuperscript{\protect\hyperlink{ref-base}{110}} fornece a função \href{https://www.rdocumentation.org/packages/stats/versions/3.6.2/topics/quantile}{\emph{quantile}} para executar análise de percentis.

\end{infobox}

\hypertarget{que-paruxe2metros-de-distribuiuxe7uxe3o-podem-ser-estimados}{%
\subsection{Que parâmetros de distribuição podem ser estimados?}\label{que-paruxe2metros-de-distribuiuxe7uxe3o-podem-ser-estimados}}

\begin{itemize}
\item
  \emph{Assimetria}.\textsuperscript{\protect\hyperlink{ref-kanji2006}{106}}
\item
  \emph{Curtose}.\textsuperscript{\protect\hyperlink{ref-kanji2006}{106}}
\end{itemize}

\hypertarget{que-paruxe2metros-extremos-podem-ser-estimados}{%
\subsection{Que parâmetros extremos podem ser estimados?}\label{que-paruxe2metros-extremos-podem-ser-estimados}}

\begin{itemize}
\item
  \emph{Mínimo}.\textsuperscript{\protect\hyperlink{ref-Ali2016}{85}}
\item
  \emph{Máximo}.\textsuperscript{\protect\hyperlink{ref-Ali2016}{85}}
\end{itemize}

\begin{infobox}{images/Rlogo}
O pacote \emph{stats}\textsuperscript{\protect\hyperlink{ref-base}{110}} fornece a função \href{https://www.rdocumentation.org/packages/stats/versions/3.6.2/topics/quantile}{\emph{quantile}} para executar análise de percentis.

\end{infobox}

\hypertarget{outliers}{%
\section{Valores discrepantes}\label{outliers}}

\hypertarget{o-que-suxe3o-valores-discrepantes-outliers}{%
\subsection{\texorpdfstring{O que são valores discrepantes (\emph{outliers})?}{O que são valores discrepantes (outliers)?}}\label{o-que-suxe3o-valores-discrepantes-outliers}}

\begin{itemize}
\item
  Em termos gerais, um valor discrepante - ``fora da curva'' ou \emph{outlier} - é uma observação que possui um valor relativamente grande ou pequeno em comparação com a maioria das observações.\textsuperscript{\protect\hyperlink{ref-zuur2009}{111}}
\item
  Mais especificamente, um valor discrepante é uma observação incomum que exerce influência indevida em uma análise.\textsuperscript{\protect\hyperlink{ref-zuur2009}{111}}
\end{itemize}

\hypertarget{como-conduzir-anuxe1lises-com-valores-discrepantes}{%
\subsection{Como conduzir análises com valores discrepantes?}\label{como-conduzir-anuxe1lises-com-valores-discrepantes}}

\begin{itemize}
\item
  Erros de observação e de medição são uma justificativa válida para descartar observações discrepantes.\textsuperscript{\protect\hyperlink{ref-zuur2009}{111}}
\item
  Valores discrepantes na variável de desfecho podem exigir uma abordagem mais refinada, especialmente quando representam uma variação real na variável que está sendo medida.\textsuperscript{\protect\hyperlink{ref-zuur2009}{111}}
\item
  Valores discrepantes em uma (co)variável podem surgir devido a um projeto experimental inadequado; nesse caso, abandonar a observação ou transformar a covariável são opções adequadas.\textsuperscript{\protect\hyperlink{ref-zuur2009}{111}}
\item
  É importante reportar se existem valores discrepantes e como foram tratados.\textsuperscript{\protect\hyperlink{ref-zuur2009}{111}}
\end{itemize}

\begin{infobox}{images/Rlogo}
O pacote \emph{outliers}\textsuperscript{\protect\hyperlink{ref-outliers}{112}} fornece a função \href{https://www.rdocumentation.org/packages/outliers/versions/0.15/topics/outlier}{\emph{outlier}} para identificar os valores mais distantes da média.

\end{infobox}

\begin{infobox}{images/Rlogo}
O pacote \emph{outliers}\textsuperscript{\protect\hyperlink{ref-outliers}{112}} fornece a função \href{https://www.rdocumentation.org/packages/outliers/versions/0.15/topics/rm.outlier}{\emph{rm.outlier}} para remover os valores mais distantes da média detectados por testes de hipótese e/ou substitui-los pela média ou mediana.

\end{infobox}

\hypertarget{analise-inicial-dados}{%
\chapter{\texorpdfstring{\textbf{Análise inicial de dados}}{Análise inicial de dados}}\label{analise-inicial-dados}}

\hypertarget{analise-inicial}{%
\section{Análise inicial de dados}\label{analise-inicial}}

\hypertarget{o-que-uxe9-anuxe1lise-inicial-de-dados}{%
\subsection{O que é análise inicial de dados?}\label{o-que-uxe9-anuxe1lise-inicial-de-dados}}

\begin{itemize}
\item
  Análise inicial de dados\textsuperscript{\protect\hyperlink{ref-chatfield1986}{113}} é uma sequência de procedimentos que visam principalmente a transparência e integridade das pré-condições do estudo para conduzir a análise estatística apropriada de modo responsável para responder aos problemas da pesquisa.\textsuperscript{\protect\hyperlink{ref-Baillie2022}{73}}
\item
  O objetivo da análise inicial de dados é propiciar dados prontos para análise estatística, incluindo informações confiáveis sobre as propriedades dos dados.\textsuperscript{\protect\hyperlink{ref-Baillie2022}{73}}
\item
  A análise inicial de dados pode ser dividida nas seguintes etapas:\textsuperscript{\protect\hyperlink{ref-Baillie2022}{73}}

  \begin{itemize}
  \item
    Configuração dos metadados
  \item
    Limpeza dos dados
  \item
    Verificação dos dados
  \item
    Relatório inicial dos dados
  \item
    Refinamento e atualização do plano de análise estatística
  \item
    Documentação e relatório da análise inicial de dados
  \end{itemize}
\item
  A análise inicial de dados não deve ser confundida com análise exploratória\textsuperscript{\protect\hyperlink{ref-Ferketich1986}{114}}, nem deve ser utilizada para hipotetizar após os dados serem coletados (conhecido como \emph{Hypothesizing After Results are Known}, HARKing)\textsuperscript{\protect\hyperlink{ref-Kerr1998}{115}}.
\end{itemize}

\hypertarget{como-conduzir-uma-anuxe1lise-inicial-de-dados}{%
\subsection{Como conduzir uma análise inicial de dados?}\label{como-conduzir-uma-anuxe1lise-inicial-de-dados}}

\begin{itemize}
\item
  Desenvolva um plano de análise inicial de dados consistente com os objetivos da pesquisa. Por exemplo, verifique a distribuição e escala das variáveis, procure por observações não-usuais ou improváveis, avalie possíveis padrões de dados perdidos.\textsuperscript{\protect\hyperlink{ref-Baillie2022}{73}}
\item
  Não altere diretamente os dados de uma tabela obtida de uma fonte. Use scripts para implementar eventuais alterações, de modo a manter o registro de todas as modificações realizadas no banco de dados.\textsuperscript{\protect\hyperlink{ref-Baillie2022}{73}}
\item
  Use os metadados do estudo para guiar a análise inicial dos dados e compartilhe com os dados para maior transparência e reprodutibilidade.\textsuperscript{\protect\hyperlink{ref-Baillie2022}{73}}
\item
  Representação gráfica dos dados pode ajudar a identificar características e padrões no banco de dados, tais como suposições e tendências.\textsuperscript{\protect\hyperlink{ref-Baillie2022}{73}}
\item
  Verifique a frequência e proporção de dados perdidos em cada variável, e depois examine por padrões de dados perdidos simultaneamente por duas ou mais variáveis.\textsuperscript{\protect\hyperlink{ref-Baillie2022}{73}}
\item
  Verifique a frequência e proporção de dados perdidos em cada variável, e depois examine por padrões de dados perdidos simultaneamente por duas ou mais variáveis.\textsuperscript{\protect\hyperlink{ref-Baillie2022}{73}}
\item
  Exclusão de dados \emph{ad hoc} baseada no desfecho pode influenciar os resultados do estudo, portanto os critérios de exclusão de dados antes da análise estatística (descritiva e/ou inferencial) devem ser reportados.\textsuperscript{\protect\hyperlink{ref-Landis2012}{116}}
\end{itemize}

\hypertarget{quais-problemas-podem-ser-detectados-na-anuxe1lise-inicial-de-dados}{%
\subsection{Quais problemas podem ser detectados na análise inicial de dados?}\label{quais-problemas-podem-ser-detectados-na-anuxe1lise-inicial-de-dados}}

\begin{itemize}
\item
  Registros duplicados, que devem ser excluídos para não inflar a amostra.\textsuperscript{\protect\hyperlink{ref-huebner2016}{117}}
\item
  Codificação 0 ou 1 para variáveis dicotômicas para representar a direção esperada da associação entre elas.\textsuperscript{\protect\hyperlink{ref-huebner2016}{117}}
\item
  Ordenação cronológica de variáveis com registros temporais (retrospectivos ou prospectivos).\textsuperscript{\protect\hyperlink{ref-huebner2016}{117}}
\item
  A distribuição das variáveis para verificação das suposições das análises planejadas.\textsuperscript{\protect\hyperlink{ref-huebner2016}{117}}
\item
  Ocorrência de efeitos teto e piso nas variáveis.\textsuperscript{\protect\hyperlink{ref-huebner2016}{117}}
\end{itemize}

\hypertarget{analise-exploratoria-dados}{%
\chapter{\texorpdfstring{\textbf{Análise exploratória de dados}}{Análise exploratória de dados}}\label{analise-exploratoria-dados}}

\hypertarget{analise-exploratoria}{%
\section{Análise exploratória de dados}\label{analise-exploratoria}}

\hypertarget{o-que-uxe9-anuxe1lise-exploratuxf3ria-de-dados}{%
\subsection{O que é análise exploratória de dados?}\label{o-que-uxe9-anuxe1lise-exploratuxf3ria-de-dados}}

\begin{itemize}
\item
  Análise exploratória de dados consiste em um processo iterativo de elaboração e interpretação da síntese de dados, tabelas e gráficos, considerando os aspectos teóricos do estudo.\textsuperscript{\protect\hyperlink{ref-Ferketich1986}{114}}
\item
  Análise exploratória deve ser separada da análise inferencial de testes de hipóteses; a decisão sobre os modelos a testar deve ser feita \emph{a priori}.\textsuperscript{\protect\hyperlink{ref-zuur2009}{111}}
\end{itemize}

\hypertarget{por-que-conduzir-a-anuxe1lise-exploratuxf3ria-de-dados}{%
\subsection{Por que conduzir a análise exploratória de dados?}\label{por-que-conduzir-a-anuxe1lise-exploratuxf3ria-de-dados}}

\begin{itemize}
\item
  A condução de análise exploratória de dados pode ajudar a identificar padrões e pode orientar trabalhos futuros, mas os resultados não devem ser interpretados como inferências sobre uma população.\textsuperscript{\protect\hyperlink{ref-zuur2009}{111}}
\item
  A análise exploratória não deve ser usada para definir as questões e hipóteses científicas do estudo.\textsuperscript{\protect\hyperlink{ref-zuur2009}{111}}
\end{itemize}

\begin{infobox}{images/Rlogo}
O pacote \emph{explore}\textsuperscript{\protect\hyperlink{ref-explore}{118}} fornece a função \href{https://www.rdocumentation.org/packages/explore/versions/1.0.2/topics/explore}{\emph{explore}} para análise exploratória de um banco de dados.

\end{infobox}

\begin{infobox}{images/Rlogo}
O pacote \emph{dataMaid}\textsuperscript{\protect\hyperlink{ref-dataMaid}{119}} fornece a função \href{https://www.rdocumentation.org/packages/dataMaid/versions/1.4.1/topics/makeDataReport}{\emph{makeDataReport}} para criar um relatório de análise exploratória de um banco de dados.

\end{infobox}

\begin{infobox}{images/Rlogo}
O pacote \emph{DataExplorer}\textsuperscript{\protect\hyperlink{ref-DataExplorer-2}{120}} fornece a função \href{https://www.rdocumentation.org/packages/DataExplorer/versions/0.8.2/topics/create_report}{\emph{create\_report}} para criar um relatório de análise exploratória de um banco de dados.

\end{infobox}

\begin{infobox}{images/Rlogo}
O pacote \emph{SmartEDA}\textsuperscript{\protect\hyperlink{ref-SmartEDA}{121}} fornece a função \href{https://www.rdocumentation.org/packages/SmartEDA/versions/0.3.9/topics/ExpReport}{\emph{ExpReport}} para criar um relatório de análise exploratória de um banco de dados.

\end{infobox}

\hypertarget{quais-etapas-constituem-a-anuxe1lise-exploratuxf3ria-de-dados}{%
\subsection{Quais etapas constituem a análise exploratória de dados?}\label{quais-etapas-constituem-a-anuxe1lise-exploratuxf3ria-de-dados}}

\begin{itemize}
\item
  Cada combinação de problema de pesquisa e delineamento de estudo pode demandar um plano de análise exploratório distinto.\textsuperscript{\protect\hyperlink{ref-zuur2009}{111}}
\item
  Verifique a existência e/ou influência de valores discrepantes (``fora da curva'' ou \emph{outliers}):\textsuperscript{\protect\hyperlink{ref-zuur2009}{111},\protect\hyperlink{ref-chatfield1986}{113},\protect\hyperlink{ref-Ferketich1986}{114}}

  \begin{itemize}
  \item
    Boxplots
  \item
    Gráficos quantil-quantil (Q-Q)
  \end{itemize}
\end{itemize}

\begin{infobox}{images/Rlogo}
O pacote \emph{graphics}\textsuperscript{\protect\hyperlink{ref-graphics}{122}} fornece a função \href{https://www.rdocumentation.org/packages/graphics/versions/3.6.2/topics/boxplot}{\emph{boxplot}} para construção de gráficos \emph{boxplot}.

\end{infobox}

\begin{itemize}
\item
  Verifique a homocedasticidade (homogeneidade da variância):\textsuperscript{\protect\hyperlink{ref-zuur2009}{111}}

  \begin{itemize}
  \item
    Boxplots condicionais (por fator de análise)
  \item
    Análise dos resíduos do modelo de regressão
  \item
    Gráfico resíduos vs.~valores ajustados
  \end{itemize}
\end{itemize}

\begin{itemize}
\item
  Verifique a normalidade da distribuição dos dados:\textsuperscript{\protect\hyperlink{ref-zuur2009}{111},\protect\hyperlink{ref-chatfield1986}{113}}

  \begin{itemize}
  \item
    Histograma das variáveis (por fator de análise)
  \item
    Histograma dos resíduos da regressão
  \end{itemize}
\end{itemize}

\begin{itemize}
\item
  Verifique a existência de grande quantidade de valores 0:\textsuperscript{\protect\hyperlink{ref-zuur2009}{111}}

  \begin{itemize}
  \tightlist
  \item
    Histograma das variáveis (por fator de análise)
  \end{itemize}
\end{itemize}

\begin{itemize}
\item
  Verifique a existência de colinearidade entre variáveis independentes de um modelo de regressão:\textsuperscript{\protect\hyperlink{ref-zuur2009}{111}}

  \begin{itemize}
  \item
    Fator de inflação de variância (\emph{variance inflation factor}, VIF)
  \item
    Coeficiente de correlação de Pearson (\(r\))
  \item
    Gráfico de dispersão entre variáveis
  \end{itemize}
\end{itemize}

\begin{itemize}
\item
  Verifique possíveis relações entre as variáveis dependente(s) e independente(s) de um modelo de regressão:\textsuperscript{\protect\hyperlink{ref-zuur2009}{111}}

  \begin{itemize}
  \tightlist
  \item
    Gráfico de dispersão entre variáveis independente e dependente
  \end{itemize}
\end{itemize}

\begin{itemize}
\item
  Verifique possíveis interações entre as variáveis dependente(s) de um modelo de regressão:\textsuperscript{\protect\hyperlink{ref-zuur2009}{111}}

  \begin{itemize}
  \tightlist
  \item
    Gráfico \emph{coplot} de dispersão entre variáveis dependentes
  \end{itemize}
\end{itemize}

\begin{infobox}{images/Rlogo}
O pacote \emph{graphics}\textsuperscript{\protect\hyperlink{ref-graphics}{122}} fornece a função \href{https://www.rdocumentation.org/packages/graphics/versions/3.6.2/topics/coplot}{\emph{coplot}} para construção de gráficos \emph{boxplot} condicionais.

\end{infobox}

\begin{itemize}
\item
  Verifique por dependência entre variáveis de um modelo de regressão:\textsuperscript{\protect\hyperlink{ref-zuur2009}{111}}

  \begin{itemize}
  \item
    Gráfico de série temporal das variáveis
  \item
    Gráfico de autocorrelação entre as variáveis
  \end{itemize}
\end{itemize}

\hypertarget{analise-descritiva}{%
\chapter{\texorpdfstring{\textbf{Análise descritiva}}{Análise descritiva}}\label{analise-descritiva}}

\hypertarget{descritiva}{%
\section{Análise descritiva}\label{descritiva}}

\hypertarget{o-que-uxe9-anuxe1lise-descritiva}{%
\subsection{O que é análise descritiva?}\label{o-que-uxe9-anuxe1lise-descritiva}}

\begin{itemize}
\item
  A análise descritiva utiliza métodos para calcular, descrever e resumir os dados coletados da(s) amostra(s) de modo que sejam interpretadas adequadamente.\textsuperscript{\protect\hyperlink{ref-vetter2017}{55}}
\item
  As análises descritivas geralmente compreendem a apresentação quantitativa (numérica) em tabelas e/ou gráficos.\textsuperscript{\protect\hyperlink{ref-vetter2017}{55}}
\end{itemize}

\begin{infobox}{images/Rlogo}
O pacote \emph{explore}\textsuperscript{\protect\hyperlink{ref-explore}{118}} fornece a função \href{https://www.rdocumentation.org/packages/explore/versions/1.0.2/topics/explore}{\emph{explore}} para análise exploratória de um banco de dados.

\end{infobox}

\begin{infobox}{images/Rlogo}
O pacote \emph{dataMaid}\textsuperscript{\protect\hyperlink{ref-dataMaid}{119}} fornece a função \href{https://www.rdocumentation.org/packages/dataMaid/versions/1.4.1/topics/makeDataReport}{\emph{makeDataReport}} para criar um relatório de análise exploratória de um banco de dados.

\end{infobox}

\begin{infobox}{images/Rlogo}
O pacote \emph{DataExplorer}\textsuperscript{\protect\hyperlink{ref-DataExplorer-2}{120}} fornece a função \href{https://www.rdocumentation.org/packages/DataExplorer/versions/0.8.2/topics/create_report}{\emph{create\_report}} para criar um relatório de análise exploratória de um banco de dados.

\end{infobox}

\begin{infobox}{images/Rlogo}
O pacote \emph{SmartEDA}\textsuperscript{\protect\hyperlink{ref-SmartEDA}{121}} fornece a função \href{https://www.rdocumentation.org/packages/SmartEDA/versions/0.3.9/topics/ExpReport}{\emph{ExpReport}} para criar um relatório de análise exploratória de um banco de dados.

\end{infobox}

\hypertarget{como-apresentar-os-resultados-descritivos}{%
\subsection{Como apresentar os resultados descritivos?}\label{como-apresentar-os-resultados-descritivos}}

\begin{itemize}
\item
  Variáveis categóricas: Reporte valores de frequência absoluta e relativa (n, \%).\textsuperscript{\protect\hyperlink{ref-Cummings2003}{123}}
\item
  Organização das tabelas: as variáveis são exibidas em linhas e os grupos são exibidos em colunas.\textsuperscript{\protect\hyperlink{ref-Cummings2003}{123}}
\item
  Calcule percentagens para as colunas (isto é, entre grupos) e não entre linhas.\textsuperscript{\protect\hyperlink{ref-Cummings2003}{123}}
\item
  Em caso de dados perdidos, não inclua uma linha com total de dados perdidos, pois distorce as proporções entre colunas e as análises de tabela de contingência. Neste caso, indique no texto ou em uma coluna separada o total de dados perdidos por variável.\textsuperscript{\protect\hyperlink{ref-Cummings2003}{123}}
\end{itemize}

\hypertarget{tabelas}{%
\section{Tabelas}\label{tabelas}}

\hypertarget{por-que-usar-tabelas}{%
\subsection{Por que usar tabelas?}\label{por-que-usar-tabelas}}

\begin{itemize}
\tightlist
\item
  Tabelas complementam o texto (e vice-versa), e podem apresentar os dados de modo mais acessível e informativo.\textsuperscript{\protect\hyperlink{ref-Inskip2017}{124}}
\end{itemize}

\hypertarget{que-informauxe7uxf5es-incluir-nas-tabelas}{%
\subsection{Que informações incluir nas tabelas?}\label{que-informauxe7uxf5es-incluir-nas-tabelas}}

\begin{itemize}
\tightlist
\item
  Título ou legenda, uma síntese descritiva (geralmente por meio de parâmetros descritivos), intervalos de confiança e/ou P-valores conforme necessário para adequada interpretação.\textsuperscript{\protect\hyperlink{ref-Inskip2017}{124},\protect\hyperlink{ref-Kwak2021}{125}}
\end{itemize}

\hypertarget{quais-suxe3o-os-erros-mais-comuns-de-preenchimento-de-tabelas}{%
\subsection{Quais são os erros mais comuns de preenchimento de tabelas?}\label{quais-suxe3o-os-erros-mais-comuns-de-preenchimento-de-tabelas}}

\begin{itemize}
\item
  Erros tipográficos.\textsuperscript{\protect\hyperlink{ref-barnett2023}{126}}
\item
  Ausência de rótulos ou unidades nas variáveis.\textsuperscript{\protect\hyperlink{ref-barnett2023}{126}}
\item
  Relatar estatísticas incorretamente, tais como rotular variáveis contínuas como porcentagens.\textsuperscript{\protect\hyperlink{ref-barnett2023}{126}}
\item
  Estatísticas descritivas de tendência central (ex.: médias) relatadas sem a estatística de dispersão correspondente (ex.: desvio-padrão).\textsuperscript{\protect\hyperlink{ref-barnett2023}{126}}
\item
  Desvio-padrão nulo (\(\sigma=0\)).\textsuperscript{\protect\hyperlink{ref-barnett2023}{126}}
\item
  Valores porcentuais que não correspondem ao numerador dividido pelo denominador.\textsuperscript{\protect\hyperlink{ref-barnett2023}{126}}
\end{itemize}

\hypertarget{como-exportar-tabelas-em-formato-docx}{%
\subsection{Como exportar tabelas em formato DOCX?}\label{como-exportar-tabelas-em-formato-docx}}

\begin{infobox}{images/Rlogo}
O pacote \emph{table1}\textsuperscript{\protect\hyperlink{ref-flextable}{127}} fornece as funções \href{https://www.rdocumentation.org/packages/flextable/versions/0.9.2/topics/as_flextable}{\emph{as\_flextable}} e \href{https://www.rdocumentation.org/packages/flextable/versions/0.9.2/topics/save_as_docx}{\emph{save\_as\_docx}} para salvar tabelas em formato DOCX.

\end{infobox}

\hypertarget{tabela-1}{%
\section{Tabela 1}\label{tabela-1}}

\hypertarget{o-que-uxe9-a-tabela-1}{%
\subsection{O que é a `Tabela 1'?}\label{o-que-uxe9-a-tabela-1}}

\begin{itemize}
\tightlist
\item
  A `Tabela 1' descreve as características demográficas, sociais e clínicas da amostra, completa ou agrupada por algum fator, geralmente por meio de parâmetros de tendência central e dispersão.\textsuperscript{\protect\hyperlink{ref-Westreich2013}{128},\protect\hyperlink{ref-chen2020}{129}}
\end{itemize}

\hypertarget{qual-a-utilidade-da-tabela-1}{%
\subsection{Qual a utilidade da `Tabela 1'?}\label{qual-a-utilidade-da-tabela-1}}

\begin{itemize}
\item
  Descrever (conhecer) as características da amostra e dos grupos sendo comparados, quando aplicável.\textsuperscript{\protect\hyperlink{ref-chen2020}{129}}
\item
  Verificar aderência ao protocolo do estudo, incluindo critérios de inclusão/exclusão, tamanho da amostra e perdas amostrais.\textsuperscript{\protect\hyperlink{ref-chen2020}{129}}
\item
  Permitir a replicação do estudo.\textsuperscript{\protect\hyperlink{ref-chen2020}{129}}
\item
  Meta-analisar os dados junto a estudos similares.\textsuperscript{\protect\hyperlink{ref-chen2020}{129}}
\item
  Avaliar a generalização (validade externa) das conclusões do estudo.\textsuperscript{\protect\hyperlink{ref-chen2020}{129}}
\end{itemize}

\hypertarget{o-que-uxe9-a-faluxe1cia-da-tabela-1}{%
\subsection{O que é a falácia da `Tabela 1'?}\label{o-que-uxe9-a-faluxe1cia-da-tabela-1}}

\begin{itemize}
\tightlist
\item
  Falácia da Tabela 1 ocorre pela interpretação errônea dos P-valores na comparação entre grupos, na linha de base, de um ensaio clínico aleatorizado.\textsuperscript{\protect\hyperlink{ref-pijls2022}{130}}
\end{itemize}

\hypertarget{como-construir-a-tabela-1}{%
\subsection{Como construir a `Tabela 1'?}\label{como-construir-a-tabela-1}}

\begin{itemize}
\tightlist
\item
  A Tabela 1 geralmente é utilizada para descrever as características da amostra estudada, possibilitando a análise de ameaças à validade interna e/ou externa ao estudo.\textsuperscript{\protect\hyperlink{ref-greenhalgh1997}{109},\protect\hyperlink{ref-Hayes-Larson2019}{131}}
\end{itemize}

\begin{infobox}{images/Rlogo}
O pacote \emph{table1}\textsuperscript{\protect\hyperlink{ref-table1}{132}} fornece a função \href{https://www.rdocumentation.org/packages/table1/versions/1.4.3/topics/table1}{\emph{table1}} para construção de tabelas.

\end{infobox}

\begin{infobox}{images/Rlogo}
O pacote \emph{gtsummary}\textsuperscript{\protect\hyperlink{ref-gtsummary-2}{133}} fornece a função \href{https://www.rdocumentation.org/packages/gtsummary/versions/1.6.3/topics/tbl_summary}{\emph{tbl\_summary}} para construção da `Tabela 1' com dados descritivos.

\end{infobox}

\hypertarget{tabela-2}{%
\section{Tabela 2}\label{tabela-2}}

\hypertarget{o-que-uxe9-a-tabela-2}{%
\subsection{O que é a `Tabela 2'?}\label{o-que-uxe9-a-tabela-2}}

\begin{itemize}
\tightlist
\item
  .{[}REF{]}
\end{itemize}

\hypertarget{qual-a-utilidade-da-tabela-2}{%
\subsection{Qual a utilidade da `Tabela 2'?}\label{qual-a-utilidade-da-tabela-2}}

\begin{itemize}
\tightlist
\item
  A Tabela 2 mostra associações ajustadas multivariadas com o resultado para variáveis resumidas na Tabela 1.\textsuperscript{\protect\hyperlink{ref-Westreich2013}{128}}
\end{itemize}

\hypertarget{o-que-uxe9-a-faluxe1cia-da-tabela-2}{%
\subsection{O que é a falácia da `Tabela 2'?}\label{o-que-uxe9-a-faluxe1cia-da-tabela-2}}

\begin{itemize}
\item
  A Tabela 2 pode induzir ao erro de interpretação pelas estimativas de efeitos para covariáveis do modelo também serem utilizados para controlar a confusão da exposição.\textsuperscript{\protect\hyperlink{ref-Westreich2013}{128},\protect\hyperlink{ref-bandoli2018}{134}}
\item
  Ao apresentar estimativas de efeito ajustadas para covariáveis juntamente com a estimativa de efeito ajustada para a exposição primária, a Tabela 2 sugere implicitamente que todas estas estimativas podem ser interpretadas de forma semelhante, se não de forma idêntica, como estimativa do efeito total.\textsuperscript{\protect\hyperlink{ref-Westreich2013}{128},\protect\hyperlink{ref-bandoli2018}{134}}
\item
  A falácia da Tabela 2 pode ser evitada limitando-se a tabela a estimativas das medidas primárias do efeito de exposição nos diferentes modelos, com as covariáveis secundárias de ``ajuste'' relatadas em uma nota de rodapé, juntamente com a forma como foram categorizadas ou modeladas.\textsuperscript{\protect\hyperlink{ref-Westreich2013}{128}}
\end{itemize}

\hypertarget{como-construir-a-tabela-2}{%
\subsection{Como construir a `Tabela 2'?}\label{como-construir-a-tabela-2}}

\begin{itemize}
\tightlist
\item
  A Tabela 2 pode ser utilizada para apresentar estimativas de múltiplos efeitos ajustados de um mesmo modelo estatístico.\textsuperscript{\protect\hyperlink{ref-Westreich2013}{128}}
\end{itemize}

\begin{infobox}{images/Rlogo}
O pacote \emph{table1}\textsuperscript{\protect\hyperlink{ref-table1}{132}} fornece a função \href{https://www.rdocumentation.org/packages/table1/versions/1.4.3/topics/table1}{\emph{table1}} para construção de tabelas.

\end{infobox}

\begin{infobox}{images/Rlogo}
O pacote \emph{gtsummary}\textsuperscript{\protect\hyperlink{ref-gtsummary-2}{133}} fornece a função \href{https://www.rdocumentation.org/packages/gtsummary/versions/1.6.3/topics/tbl_summary}{\emph{tbl\_summary}} para construção da `Tabela 2' com análises inferenciais.

\end{infobox}

\hypertarget{graficos}{%
\section{Gráficos}\label{graficos}}

\hypertarget{o-que-suxe3o-gruxe1ficos}{%
\subsection{O que são gráficos?}\label{o-que-suxe3o-gruxe1ficos}}

\begin{itemize}
\tightlist
\item
  Gráficos são utilizados para apresentar dados (geralmente em grande quantidade) de modo mais intuitivo e fácil de compreender.\textsuperscript{\protect\hyperlink{ref-Park2022}{135}}
\end{itemize}

\hypertarget{qual-a-utilidade-dos-gruxe1ficos}{%
\subsection{Qual a utilidade dos gráficos?}\label{qual-a-utilidade-dos-gruxe1ficos}}

\begin{itemize}
\tightlist
\item
  .{[}REF{]}
\end{itemize}

\hypertarget{que-elementos-incluir-em-gruxe1ficos}{%
\subsection{Que elementos incluir em gráficos?}\label{que-elementos-incluir-em-gruxe1ficos}}

\begin{itemize}
\tightlist
\item
  Título, eixos horizontal e vertical com respectivas unidades, escalas em intervalos representativos das variáveis, legenda com símbolos, síntese descritiva dos valores e respectiva margem de erro, conforme necessário para adequada interpretação.\textsuperscript{\protect\hyperlink{ref-Park2022}{135}}
\end{itemize}

\begin{infobox}{images/Rlogo}
Os pacotes \emph{ggplot2}\textsuperscript{\protect\hyperlink{ref-ggplot2}{136}}, \emph{plotly}\textsuperscript{\protect\hyperlink{ref-plotly}{137}} e \emph{corrplot}\textsuperscript{\protect\hyperlink{ref-corrplot}{138}} fornecem diversas funções para construção de gráficos tais como \href{https://www.rdocumentation.org/packages/ggplot2/versions/3.4.3/topics/ggplot}{\emph{ggplot}}, \href{https://www.rdocumentation.org/packages/plotly/versions/4.10.2/topics/plot_ly}{\emph{plot\_ly}} e \href{https://www.rdocumentation.org/packages/corrplot/versions/0.92/topics/corrplot}{\emph{corrplot}} respectivamente.

\end{infobox}

\hypertarget{para-que-servem-as-barras-de-erro-em-gruxe1ficos}{%
\subsection{Para que servem as barras de erro em gráficos?}\label{para-que-servem-as-barras-de-erro-em-gruxe1ficos}}

\begin{itemize}
\item
  Barras de erro ajudam ao autor a apresentar as informações que descrevem os dados (por exemplo, em uma análise descritiva) ou sobre as inferências ou conclusões tomadas a partir de dados.\textsuperscript{\protect\hyperlink{ref-Cumming2007}{139}}
\item
  Barras de erro mais longas representam mais imprecisão (maiores erros), enquanto barras mais curtas representam mais precisão na estimativa.\textsuperscript{\protect\hyperlink{ref-Cumming2007}{139}}
\item
  Barras de erro descritivas geralmente apresentam a amplitude (mínimo-máximo) ou desvio-padrão.\textsuperscript{\protect\hyperlink{ref-Cumming2007}{139}}
\item
  Barras de erro inferenciais geralmente apresentam o erro-padrão ou intervalo de confiança (por exemplo, de 95\%).\textsuperscript{\protect\hyperlink{ref-Cumming2007}{139}}
\item
  O comprimento das barras de erro sugere graficamente a imprecisão dos dados do estudo, uma vez que o valor verdadeiro da população pode estar em qualquer nível do intervalo da barra.\textsuperscript{\protect\hyperlink{ref-Cumming2007}{139}}
\end{itemize}

\hypertarget{quais-suxe3o-as-boas-pruxe1ticas-na-elaborauxe7uxe3o-de-gruxe1ficos}{%
\subsection{Quais são as boas práticas na elaboração de gráficos?}\label{quais-suxe3o-as-boas-pruxe1ticas-na-elaborauxe7uxe3o-de-gruxe1ficos}}

\begin{itemize}
\item
  O tamanho da amostra total e subgrupos, se houver, deve estar descrito na figura ou na sua legenda.\textsuperscript{\protect\hyperlink{ref-Cumming2007}{139}}
\item
  Para análise inferencial de figuras, as barras de erro representadas por erro-padrão ou intervalo de confiança são preferíveis à amplitude ou desvio-padrão.\textsuperscript{\protect\hyperlink{ref-Cumming2007}{139}}
\item
  Evite gráficos de barra e mostre a distribuição dos dados sempre que possível.\textsuperscript{\protect\hyperlink{ref-Weissgerber2019}{140}}
\item
  Exiba os pontos de dados em boxplots.\textsuperscript{\protect\hyperlink{ref-Weissgerber2019}{140}}
\item
  Use \emph{jitter} simétrico em gráficos de pontos para permitir a visualização de todos os dados.\textsuperscript{\protect\hyperlink{ref-Weissgerber2019}{140}}
\item
  Prefira palhetas de cor adaptadas para daltônicos.\textsuperscript{\protect\hyperlink{ref-Weissgerber2019}{140}}
\end{itemize}

\begin{infobox}{images/Rlogo}
O pacote \emph{ggsci}\textsuperscript{\protect\hyperlink{ref-ggsci}{141}} fornece palhetas de cores tais como \href{https://www.rdocumentation.org/packages/ggsci/versions/3.0.0/topics/pal_lancet}{\emph{pal\_lancet}}, \href{https://www.rdocumentation.org/packages/ggsci/versions/3.0.0/topics/pal_nejm}{\emph{pal\_nejm}} e \href{https://www.rdocumentation.org/packages/ggsci/versions/3.0.0/topics/pal_npg}{\emph{pal\_npg}} inspiradas em publicações científicas para uso em gráficos.

\end{infobox}

\begin{infobox}{images/Rlogo}
O pacote \emph{grDevices}\textsuperscript{\protect\hyperlink{ref-grDevices}{142}} fornece a função \href{https://www.rdocumentation.org/packages/grDevices/versions/3.6.2/topics/dev}{\emph{dev.new}} para controlar diversos aspectos do gráfico, tais como tamanho e resolução.

\end{infobox}

\hypertarget{como-exportar-figuras-em-formato-tiff}{%
\subsection{Como exportar figuras em formato TIFF?}\label{como-exportar-figuras-em-formato-tiff}}

\begin{infobox}{images/Rlogo}
O pacote \emph{tiff}\textsuperscript{\protect\hyperlink{ref-tiff}{143}} fornece a função \href{https://www.rdocumentation.org/packages/tiff/versions/0.1-11/topics/writeTIFF}{\emph{writeTIFF}} para exportar gráficos em formato TIFF.

\end{infobox}

\hypertarget{analise-inferencial}{%
\chapter{\texorpdfstring{\textbf{Análise inferencial}}{Análise inferencial}}\label{analise-inferencial}}

\hypertarget{raciocinio-inferencial}{%
\section{Raciocínio inferencial}\label{raciocinio-inferencial}}

\hypertarget{o-que-uxe9-anuxe1lise-inferencial}{%
\subsection{O que é análise inferencial?}\label{o-que-uxe9-anuxe1lise-inferencial}}

\begin{itemize}
\item
  Na análise inferencial são utilizados dados da(s) amostra(s) para fazer uma inferência válida (isto é, estimativa) sobre os parâmetros populacionais desconhecidos.\textsuperscript{\protect\hyperlink{ref-vetter2017}{55}}
\item
  No paradigma de Jerzy Neyman e Egon Pearson, um teste de hipótese científica envolve a tomada de decisão sobre hipóteses nulas (\(H_{0}\)) e alternativa (\(H_{1}\)) concorrentes e mutuamente exclusivas.\textsuperscript{\protect\hyperlink{ref-Curran-Everett2009}{144}}
\end{itemize}

\hypertarget{quais-suxe3o-os-tipos-de-raciocuxednio-inferencial}{%
\subsection{Quais são os tipos de raciocínio inferencial?}\label{quais-suxe3o-os-tipos-de-raciocuxednio-inferencial}}

\begin{itemize}
\item
  Inferência dedutiva: Uma dada hipótese inicial é utilizada para prever o que seria observado caso tal hipótese fosse verdadeira.\textsuperscript{\protect\hyperlink{ref-goodman1999}{145}}
\item
  Inferência indutiva: Com base nos dados observados, avalia-se qual hipótese é mais defensável (isto é, mais provável).\textsuperscript{\protect\hyperlink{ref-goodman1999}{145}}
\end{itemize}

\hypertarget{o-que-uxe9-generalizauxe7uxe3o-de-uma-populauxe7uxe3o}{%
\subsection{O que é generalização de uma população?}\label{o-que-uxe9-generalizauxe7uxe3o-de-uma-populauxe7uxe3o}}

\begin{itemize}
\tightlist
\item
  Generalização de uma população refere-se à extrapolação das conclusões do estudo, observados na amostra, para a população.\textsuperscript{\protect\hyperlink{ref-Banerjee2010}{26}}
\end{itemize}

\hypertarget{ideias-hipoteses}{%
\section{Hipóteses científicas}\label{ideias-hipoteses}}

\hypertarget{o-que-uxe9-hipuxf3tese-cientuxedfica}{%
\subsection{O que é hipótese científica?}\label{o-que-uxe9-hipuxf3tese-cientuxedfica}}

\begin{itemize}
\item
  Hipótese científica é uma ideia que pode ser testada.\textsuperscript{\protect\hyperlink{ref-Curran-Everett2009}{144}}
\item
  Definir claramente os problemas e os objetivos da pesquisa são o ponto de partida de todos os estudos científicos.\textsuperscript{\protect\hyperlink{ref-van2022a}{54}}
\end{itemize}

\hypertarget{quais-suxe3o-as-principais-fontes-de-ideias-para-gerar-hipuxf3teses-cientuxedficas}{%
\subsection{Quais são as principais fontes de ideias para gerar hipóteses científicas?}\label{quais-suxe3o-as-principais-fontes-de-ideias-para-gerar-hipuxf3teses-cientuxedficas}}

\begin{itemize}
\item
  Revisão das práticas atuais.\textsuperscript{\protect\hyperlink{ref-Vandenbroucke2018}{146}}
\item
  Desafio a ideias aceitas.\textsuperscript{\protect\hyperlink{ref-Vandenbroucke2018}{146}}
\item
  Conflito entre ideias divergentes.\textsuperscript{\protect\hyperlink{ref-Vandenbroucke2018}{146}}
\item
  Variações regionais, temporais e populacionais.\textsuperscript{\protect\hyperlink{ref-Vandenbroucke2018}{146}}
\item
  Experiências dos próprios pesquisadores.\textsuperscript{\protect\hyperlink{ref-Vandenbroucke2018}{146}}
\item
  Imaginação sem fronteiras ou limites convencionais.\textsuperscript{\protect\hyperlink{ref-Vandenbroucke2018}{146}}
\end{itemize}

\hypertarget{erros-inferencia}{%
\section{Testes de hipóteses}\label{erros-inferencia}}

\hypertarget{o-que-uxe9-hipuxf3tese-nula}{%
\subsection{O que é hipótese nula?}\label{o-que-uxe9-hipuxf3tese-nula}}

\begin{itemize}
\tightlist
\item
  A hipótese nula (\(H_{0}\)) é uma expressão que representa o estado atual do conhecimento (\emph{status quo}), em geral a não existência de um determinado efeito.\textsuperscript{\protect\hyperlink{ref-kanji2006}{106}}
\end{itemize}

\hypertarget{o-que-uxe9-hipuxf3tese-alternativa}{%
\subsection{O que é hipótese alternativa?}\label{o-que-uxe9-hipuxf3tese-alternativa}}

\begin{itemize}
\tightlist
\item
  A hipótese alternativa (\(H_{1}\)) é uma expressão que contém as situações que serão testadas, de modo que um resultado positivo indique alguma ação a ser conduzida.\textsuperscript{\protect\hyperlink{ref-kanji2006}{106}}
\end{itemize}

\hypertarget{qual-hipuxf3tese-estuxe1-sendo-testada}{%
\subsection{Qual hipótese está sendo testada?}\label{qual-hipuxf3tese-estuxe1-sendo-testada}}

\begin{itemize}
\item
  A hipótese nula (\(H_{0}\)) é a hipótese sob teste em análises inferenciais.\textsuperscript{\protect\hyperlink{ref-Ali2016}{85}}
\item
  Pode-se concluir sobre rejeitar ou não rejeitar a hipótese nula (\(H_{0}\)).\textsuperscript{\protect\hyperlink{ref-Ali2016}{85}}
\item
  Não se conclui sobre a hipótese alternativa (\(H_{1}\)).\textsuperscript{\protect\hyperlink{ref-kanji2006}{106}}
\item
  Para testar a hipótese nula, deve-se selecionar o nível de significância crítica (P-valor de corte); a probabilidade de rejeitarmos uma hipótese nula verdadeira (\(\alpha\)); e a probabilidade de não rejeitarmos uma hipótese nula falsa (\(\beta\)).\textsuperscript{\protect\hyperlink{ref-Curran-Everett2009}{144}}
\end{itemize}

\hypertarget{quais-suxe3o-os-tipos-de-teste-de-hipuxf3teses}{%
\subsection{Quais são os tipos de teste de hipóteses?}\label{quais-suxe3o-os-tipos-de-teste-de-hipuxf3teses}}

\begin{itemize}
\item
  Teste (clássico) de significância da hipótese nula.\textsuperscript{\protect\hyperlink{ref-lakens2018}{147}}
\item
  Teste de mínimos efeitos.\textsuperscript{\protect\hyperlink{ref-lakens2018}{147}}
\item
  Teste de equivalência.\textsuperscript{\protect\hyperlink{ref-lakens2018}{147}}
\item
  Teste de inferioridade.\textsuperscript{\protect\hyperlink{ref-lakens2018}{147}}
\item
  Teste de não-inferioridade.{[}REF{]}
\item
  Teste de superioridade.{[}REF{]}
\end{itemize}

\hypertarget{o-que-uxe9-uma-famuxedlia-de-hipuxf3teses}{%
\subsection{O que é uma família de hipóteses?}\label{o-que-uxe9-uma-famuxedlia-de-hipuxf3teses}}

\begin{itemize}
\tightlist
\item
  .{[}REF{]}
\end{itemize}

\hypertarget{o-que-suxe3o-testes-ad-hoc-e-post-hoc}{%
\subsection{\texorpdfstring{O que são testes \emph{ad hoc} e \emph{post hoc}?}{O que são testes ad hoc e post hoc?}}\label{o-que-suxe3o-testes-ad-hoc-e-post-hoc}}

\begin{itemize}
\tightlist
\item
  .{[}REF{]}
\end{itemize}

\hypertarget{como-ajustar-a-anuxe1lise-inferencial-para-hipuxf3teses-muxfaltiplas}{%
\subsection{Como ajustar a análise inferencial para hipóteses múltiplas?}\label{como-ajustar-a-anuxe1lise-inferencial-para-hipuxf3teses-muxfaltiplas}}

\begin{itemize}
\tightlist
\item
  .{[}REF{]}
\end{itemize}

\begin{infobox}{images/Rlogo}
O pacote \emph{stats}\textsuperscript{\protect\hyperlink{ref-stats-2}{52}} fornece a função \href{https://www.rdocumentation.org/packages/stats/versions/3.6.2/topics/p.adjust}{\emph{p.adjust}} para ajustar o P-valor utilizando diversos métodos.

\end{infobox}

\hypertarget{o-que-suxe3o-testes-unicaudais-e-bicaudais}{%
\subsection{O que são testes unicaudais e bicaudais?}\label{o-que-suxe3o-testes-unicaudais-e-bicaudais}}

\begin{itemize}
\tightlist
\item
  .{[}REF{]}
\end{itemize}

\hypertarget{o-que-reportar-apuxf3s-um-teste-de-hipuxf3tese}{%
\subsection{O que reportar após um teste de hipótese?}\label{o-que-reportar-apuxf3s-um-teste-de-hipuxf3tese}}

\begin{itemize}
\item
  P-valores, como estimativa da significância estatística.\textsuperscript{\protect\hyperlink{ref-Sullivan2012}{148}}
\item
  Tamanho do efeito, como estimativa de significância substantiva (clínica).\textsuperscript{\protect\hyperlink{ref-Sullivan2012}{148}}
\end{itemize}

\hypertarget{significancia-estatistica}{%
\section{Significância estatística}\label{significancia-estatistica}}

\hypertarget{o-que-uxe9-significuxe2ncia-estatuxedstica}{%
\subsection{O que é significância estatística?}\label{o-que-uxe9-significuxe2ncia-estatuxedstica}}

\begin{itemize}
\tightlist
\item
  A expressão ``significância estatística''\textsuperscript{\protect\hyperlink{ref-latter1902}{149}} ou ``evidência estatística de significância'' sugere apenas que um experimento merece ser repetido, uma vez que um baixo P-valor (calculado a partir dos dados, modelos e demais suposições do estudo) sugere ser improvável que os dados coletados sejam coletados no contexto de que a hipótese nula \(H_{0}\) assumida é verdadeira.\textsuperscript{\protect\hyperlink{ref-aylmerfisher1926}{150}}
\end{itemize}

\hypertarget{como-justificar-o-nuxedvel-de-significuxe2ncia-estatuxedstica-de-um-teste}{%
\subsection{Como justificar o nível de significância estatística de um teste?}\label{como-justificar-o-nuxedvel-de-significuxe2ncia-estatuxedstica-de-um-teste}}

\begin{itemize}
\tightlist
\item
  .{[}REF{]}
\end{itemize}

\begin{infobox}{images/Rlogo}
O pacote \emph{Superpower}\textsuperscript{\protect\hyperlink{ref-Superpower}{151}} fornece a função \href{https://www.rdocumentation.org/packages/Superpower/versions/0.2.0/topics/optimal_alpha}{\emph{optimal\_alpha}} para calcular justificar o nível de significância \(\alpha\) por balanço dos erros tipo I e II.

\end{infobox}

\begin{infobox}{images/Rlogo}
O pacote \emph{Superpower}\textsuperscript{\protect\hyperlink{ref-Superpower}{151}} fornece a função \href{https://www.rdocumentation.org/packages/Superpower/versions/0.2.0/topics/ANOVA_compromise}{\emph{ANOVA\_compromise}} para calcular justificar o nível de significância \(\alpha\) por balanço dos erros tipo I e II em análise de variância (ANOVA).

\end{infobox}

\hypertarget{p-valor}{%
\section{P-valor}\label{p-valor}}

\hypertarget{o-que-uxe9-o-p-valor}{%
\subsection{O que é o P-valor?}\label{o-que-uxe9-o-p-valor}}

\begin{itemize}
\item
  P-valor é a probabilidade, assumindo-se um dado modelo estatístico, de que um efeito calculado a partir dos dados seria igual ou mais extremo do que o seu valor observado.\textsuperscript{\protect\hyperlink{ref-wasserstein2016}{152}}
\item
  P-valor é uma variável aleatória que possui distribuição uniforme quando a hipótese nula \(H_{0}\) é verdadeira.\textsuperscript{\protect\hyperlink{ref-altman2017}{153}}
\end{itemize}

\hypertarget{como-interpretar-o-p-valor}{%
\subsection{Como interpretar o P-valor?}\label{como-interpretar-o-p-valor}}

\begin{itemize}
\item
  P-valores podem indicar quantitativamente a incompatibilidade entre os dados obtidos e o modelo estatístico especificado a priori (geralmente constituído pela hipótese nula \(H_{0}\)).\textsuperscript{\protect\hyperlink{ref-wasserstein2016}{152}}
\item
  P-valores menores/maiores do que o nível de significância estatístico pré-estabelecido não devem ser utilizados como única fonte de informação para tomada de decisão em ciência.\textsuperscript{\protect\hyperlink{ref-wasserstein2016}{152}}
\item
  P-valor resulta da coleta e análise de dados, e assim quantifica a plausibilidade dos dados observados sob a hipótese nula \(H_{0}\).\textsuperscript{\protect\hyperlink{ref-heinze2016}{154}}
\item
  P-valores abaixo de um nível de significância estatística pré-especificado representam que um experimento merece ser repetido, com a rejeição da hipótese nula \(H_{0}\)) justificada apenas quando experimentos adicionais frequentemente reportem igualmente resultados positivos (rejeição da hipótese nula \(H_{0}\)).\textsuperscript{\protect\hyperlink{ref-goodman2016}{155}}
\end{itemize}

\hypertarget{o-que-o-p-valor-nuxe3o-uxe9}{%
\subsection{O que o P-valor não é?}\label{o-que-o-p-valor-nuxe3o-uxe9}}

\begin{itemize}
\item
  P-valor não representa a probabilidade de que a hipótese nula \(H_{0}\)) seja verdadeira, nem a probabilidade de que os dados tenham sido produzidos pelo acaso.\textsuperscript{\protect\hyperlink{ref-wasserstein2016}{152}}
\item
  P-valor não mede o tamanho do efeito ou a relevância da sua observação.\textsuperscript{\protect\hyperlink{ref-wasserstein2016}{152}}
\item
  P-valor sozinho não provê informação suficiente sobre a evidência sobre um modelo teórico. A sua interpretação correta requer uma descrição ampla sobre o delineamento, métodos e análises estatísticas aplicados no estudo.\textsuperscript{\protect\hyperlink{ref-wasserstein2016}{152}}
\item
  Evidência estatística de significância não provê informação sobre a magnitude do efeito observado e não necessariamente implica que o efeito é robusto.\textsuperscript{\protect\hyperlink{ref-Landis2012}{116},\protect\hyperlink{ref-altman2017}{153}}
\end{itemize}

\hypertarget{qual-a-origem-do-p005}{%
\subsection{Qual a origem do `P\textless0,05'?}\label{qual-a-origem-do-p005}}

\begin{itemize}
\tightlist
\item
  .{[}REF{]}
\end{itemize}

\hypertarget{quais-suxe3o-os-complementos-ou-alternativas-ao-p-valor}{%
\subsection{Quais são os complementos ou alternativas ao P-valor?}\label{quais-suxe3o-os-complementos-ou-alternativas-ao-p-valor}}

\begin{itemize}
\item
  Intervalos de confiança, credibilidade ou predição.\textsuperscript{\protect\hyperlink{ref-wasserstein2016}{152}}
\item
  Razão de verossimilhança.\textsuperscript{\protect\hyperlink{ref-wasserstein2016}{152}}
\item
  Métodos Bayesianos, fator Bayes.\textsuperscript{\protect\hyperlink{ref-wasserstein2016}{152}}
\end{itemize}

\hypertarget{tamanho-efeito}{%
\section{Tamanho do efeito}\label{tamanho-efeito}}

\hypertarget{o-que-uxe9-o-tamanho-do-efeito}{%
\subsection{O que é o tamanho do efeito?}\label{o-que-uxe9-o-tamanho-do-efeito}}

\begin{itemize}
\tightlist
\item
  Tamanho do efeito quantifica a magnitude de um efeito real da análise, expressando uma importância descritiva dos resultados.\textsuperscript{\protect\hyperlink{ref-Kim2015}{156}}
\end{itemize}

\hypertarget{quais-suxe3o-os-tipos-de-tamanho-do-efeito}{%
\subsection{Quais são os tipos de tamanho do efeito?}\label{quais-suxe3o-os-tipos-de-tamanho-do-efeito}}

\begin{itemize}
\item
  Diferenças padronizadas entre grupos:\textsuperscript{\protect\hyperlink{ref-Sullivan2012}{148},\protect\hyperlink{ref-Kim2015}{156}}

  \begin{itemize}
  \item
    Cohen's d
  \item
    Glass's \(\Delta\)
  \item
    Razão de chances (\(RC\) ou \(OR\))
  \item
    Risco relativo ou razão de risco (\(RR\))
  \end{itemize}
\item
  Medidas de associação:\textsuperscript{\protect\hyperlink{ref-Sullivan2012}{148},\protect\hyperlink{ref-Kim2015}{156}}

  \begin{itemize}
  \item
    Coeficiente de correlação de Pearson (\(r\))
  \item
    Coeficiente de determinação (\(R^2\))
  \end{itemize}
\end{itemize}

\hypertarget{como-converter-um-tamanho-de-efeito-em-outro}{%
\subsection{Como converter um tamanho de efeito em outro?}\label{como-converter-um-tamanho-de-efeito-em-outro}}

\begin{itemize}
\tightlist
\item
  .\textsuperscript{\protect\hyperlink{ref-Kim2015}{156}}
\end{itemize}

\begin{infobox}{images/Rlogo}
O pacote \emph{effectsize}\textsuperscript{\protect\hyperlink{ref-effectsize}{157}} fornece diversas funções para conversão de diferentes estimativas de tamanhos de efeito.

\end{infobox}

\hypertarget{como-interpretar-um-tamanho-do-efeito}{%
\subsection{Como interpretar um tamanho do efeito?}\label{como-interpretar-um-tamanho-do-efeito}}

\begin{itemize}
\tightlist
\item
  Tamanhos de efeito podem ser comparadores entre diferentes estudos.\textsuperscript{\protect\hyperlink{ref-Sullivan2012}{148}}
\end{itemize}

\begin{infobox}{images/Rlogo}
O pacote \emph{effectsize}\textsuperscript{\protect\hyperlink{ref-effectsize}{157}} fornece a função \href{https://www.rdocumentation.org/packages/effectsize/versions/0.8.3/topics/rules}{\emph{rules}} para criar regras de interpretação de tamanhos de efeito.

\end{infobox}

\begin{infobox}{images/Rlogo}
O pacote \emph{effectsize}\textsuperscript{\protect\hyperlink{ref-effectsize}{157}} fornece a função \href{https://www.rdocumentation.org/packages/effectsize/versions/0.8.3/topics/interpret}{\emph{interpret}} para interpretar os tamanhos de efeito com base em uma lista de regras pré-definidas.

\end{infobox}

\begin{infobox}{images/Rlogo}
O pacote \emph{pwr}\textsuperscript{\protect\hyperlink{ref-pwr}{158}} fornece a função \href{https://www.rdocumentation.org/packages/pwr/versions/1.3-0/topics/cohen.ES}{\emph{cohen.ES}} para obter os tamanhos de efeito ``pequeno'', ``médio'' e ``grande'' para diversos testes de hipóteses.

\end{infobox}

\hypertarget{poder-teste}{%
\section{Poder do teste}\label{poder-teste}}

\hypertarget{o-que-uxe9-poder-do-teste}{%
\subsection{O que é poder do teste?}\label{o-que-uxe9-poder-do-teste}}

\begin{itemize}
\item
  Poder do teste é a probabilidade de rejeitar corretamente a hipótese nula (\(H_{0}\)) quando esta é falsa.\textsuperscript{\protect\hyperlink{ref-Curran-Everett2009}{144}}
\item
  Poder do teste pode ser calculado como (\(1 - \beta\)).\textsuperscript{\protect\hyperlink{ref-Curran-Everett2009}{144}}
\end{itemize}

\hypertarget{o-que-uxe9-anuxe1lise-de-poder-do-teste}{%
\subsection{O que é análise de poder do teste?}\label{o-que-uxe9-anuxe1lise-de-poder-do-teste}}

\begin{itemize}
\item
  Poder é a probabilidade de que um dado tamanho de efeito será observado em um experimento futuro sob um conjunto de hipóteses - tamanho de efeito real e erro tipo I - para um dado tamanho de amostra.\textsuperscript{\protect\hyperlink{ref-heckman2022}{159}}
\item
  O objetivo geral da análise de poder ao projetar um estudo é escolher um tamanho de amostra que controle os 2 tipos de erros de inferência estatística: tipo I (\(\alpha\), resultado falso-positivo) e tipo II (\(\beta\), resultado falso-negativo).\textsuperscript{\protect\hyperlink{ref-heckman2022}{159}}
\item
  Numericamente, o poder de um estudo é calculado como \(1-\beta\) e reportado em valor percentual.\textsuperscript{\protect\hyperlink{ref-heckman2022}{159}}
\end{itemize}

\hypertarget{quando-realizar-a-anuxe1lise-de-poder-do-teste}{%
\subsection{Quando realizar a análise de poder do teste?}\label{quando-realizar-a-anuxe1lise-de-poder-do-teste}}

\begin{itemize}
\item
  Na fase de projeto de pesquisa: a análise de poder para determinar o tamanho da amostra objetiva que o tamanho da amostra permita uma probabilidade razoável de detectar um efeito significativo pré-especificado.\textsuperscript{\protect\hyperlink{ref-heckman2022}{159}}
\item
  Após a coleta de dados: a análise de poder objetiva informar estudos futuros a respeito do tamanho da amostra necessário para a detecção de um efeito significativo pré-especificado.\textsuperscript{\protect\hyperlink{ref-heckman2022}{159}}
\end{itemize}

\begin{infobox}{images/Rlogo}
O pacote \emph{pwr}\textsuperscript{\protect\hyperlink{ref-pwr}{158}} fornece a função \href{https://www.rdocumentation.org/packages/pwr/versions/1.3-0/topics/pwr.2p.test}{\emph{pwr.2p.test}} para cálculo do poder do teste de proporção balanceado (2 amostras com mesmo número de participantes).

\end{infobox}

\begin{infobox}{images/Rlogo}
O pacote \emph{pwr}\textsuperscript{\protect\hyperlink{ref-pwr}{158}} fornece a função \href{https://www.rdocumentation.org/packages/pwr/versions/1.3-0/topics/pwr.2p.test}{\emph{pwr.2p2n.test}} para cálculo do do poder do teste de proporção não balanceado (2 amostras com diferente número de participantes).

\end{infobox}

\begin{infobox}{images/Rlogo}
O pacote \emph{pwr}\textsuperscript{\protect\hyperlink{ref-pwr}{158}} fornece a função \href{https://www.rdocumentation.org/packages/pwr/versions/1.3-0/topics/pwr.anova.test}{\emph{pwr.anova.test}} para cálculo do poder do teste de análise de variância balanceado (3 ou mais amostras com mesmo número de participantes).

\end{infobox}

\begin{infobox}{images/Rlogo}
O pacote \emph{pwr}\textsuperscript{\protect\hyperlink{ref-pwr}{158}} fornece a função \href{https://www.rdocumentation.org/packages/pwr/versions/1.3-0/topics/pwr.chisq.test}{\emph{pwr.chisq.test}} para cálculo do poder do teste de qui-quadrado \(\chi^2\).

\end{infobox}

\begin{infobox}{images/Rlogo}
O pacote \emph{pwr}\textsuperscript{\protect\hyperlink{ref-pwr}{158}} fornece a função \href{https://www.rdocumentation.org/packages/pwr/versions/1.3-0/topics/pwr.f2.test}{\emph{pwr.f2.test}} para cálculo do poder do teste com modelo linear geral.

\end{infobox}

\begin{infobox}{images/Rlogo}
O pacote \emph{pwr}\textsuperscript{\protect\hyperlink{ref-pwr}{158}} fornece a função \href{https://www.rdocumentation.org/packages/pwr/versions/1.3-0/topics/pwr.norm.test}{\emph{pwr.norm.test}} para cálculo do poder do teste de média de uma distribuição normal com variância conhecida.

\end{infobox}

\begin{infobox}{images/Rlogo}
O pacote \emph{pwr}\textsuperscript{\protect\hyperlink{ref-pwr}{158}} fornece a função \href{https://www.rdocumentation.org/packages/pwr/versions/1.3-0/topics/pwr.p.test}{\emph{pwr.p.test}} para cálculo do poder do teste de proporção (1 amostra).

\end{infobox}

\begin{infobox}{images/Rlogo}
O pacote \emph{pwr}\textsuperscript{\protect\hyperlink{ref-pwr}{158}} fornece a função \href{https://www.rdocumentation.org/packages/pwr/versions/1.3-0/topics/pwr.r.test}{\emph{pwr.r.test}} para cálculo do do poder to teste de correlação (1 amostra).

\end{infobox}

\begin{infobox}{images/Rlogo}
O pacote \emph{pwr}\textsuperscript{\protect\hyperlink{ref-pwr}{158}} fornece a função \href{https://www.rdocumentation.org/packages/pwr/versions/1.3-0/topics/pwr.t.test}{\emph{pwr.t.test}} para cálculo do poder do teste \emph{t} de diferença de 1 amostra, 2 amostras dependentes ou 2 amostras independentes (grupos balanceados).

\end{infobox}

\begin{infobox}{images/Rlogo}
O pacote \emph{pwr}\textsuperscript{\protect\hyperlink{ref-pwr}{158}} fornece a função \href{https://www.rdocumentation.org/packages/pwr/versions/1.3-0/topics/pwr.t2n.test}{\emph{pwr.t2n.test}} para cálculo do poder do teste \emph{t} de diferença de 2 amostras independentes (grupos não balanceados).

\end{infobox}

\begin{infobox}{images/Rlogo}
O pacote \emph{longpower}\textsuperscript{\protect\hyperlink{ref-longpower}{160}} fornece a função \href{https://www.rdocumentation.org/packages/longpower/versions/1.0.24/topics/power.mmrm}{\emph{power.mmrm}} para calcular o poder de testes com análises por modelo de regressão linear misto.

\end{infobox}

\begin{infobox}{images/Rlogo}
O pacote \emph{Superpower}\textsuperscript{\protect\hyperlink{ref-Superpower}{151}} fornece a função \href{https://www.rdocumentation.org/packages/Superpower/versions/0.2.0/topics/power.ftest}{\emph{power.ftest}} para calcular o poder do teste por análise de testes F.

\end{infobox}

\begin{infobox}{images/Rlogo}
O pacote \emph{Superpower}\textsuperscript{\protect\hyperlink{ref-Superpower}{151}} fornece a função \href{https://www.rdocumentation.org/packages/Superpower/versions/0.2.0/topics/power_oneway_between}{\emph{power\_oneway\_between}} para calcular o poder do teste por análise de variância (ANOVA) de 1 fator entre-sujeitos.

\end{infobox}

\begin{infobox}{images/Rlogo}
O pacote \emph{Superpower}\textsuperscript{\protect\hyperlink{ref-Superpower}{151}} fornece a função \href{https://www.rdocumentation.org/packages/Superpower/versions/0.2.0/topics/power_oneway_within}{\emph{power\_oneway\_within}} para calcular o poder do teste por análise de variância (ANOVA) de 1 fator intra-sujeitos.

\end{infobox}

\begin{infobox}{images/Rlogo}
O pacote \emph{Superpower}\textsuperscript{\protect\hyperlink{ref-Superpower}{151}} fornece a função \href{https://www.rdocumentation.org/packages/Superpower/versions/0.2.0/topics/power_oneway_ancova}{\emph{power\_oneway\_ancova}} para calcular o poder do teste por análise de covariância (ANCOVA).

\end{infobox}

\begin{infobox}{images/Rlogo}
O pacote \emph{Superpower}\textsuperscript{\protect\hyperlink{ref-Superpower}{151}} fornece a função \href{https://www.rdocumentation.org/packages/Superpower/versions/0.2.0/topics/power_twoway_between}{\emph{power\_twoway\_between}} para calcular o poder do teste por análise de covariância (ANOVA) de 2 fatores entre-sujeitos.

\end{infobox}

\begin{infobox}{images/Rlogo}
O pacote \emph{Superpower}\textsuperscript{\protect\hyperlink{ref-Superpower}{151}} fornece a função \href{https://www.rdocumentation.org/packages/Superpower/versions/0.2.0/topics/power_threeway_between}{\emph{power\_threeway\_between}} para calcular o poder do teste por análise de covariância (ANOVA) de 3 fatores entre-sujeitos.

\end{infobox}

\hypertarget{por-que-a-anuxe1lise-de-poder-do-teste-post-hoc-uxe9-inadequada}{%
\subsection{\texorpdfstring{Por que a análise de poder do teste \emph{post hoc} é inadequada?}{Por que a análise de poder do teste post hoc é inadequada?}}\label{por-que-a-anuxe1lise-de-poder-do-teste-post-hoc-uxe9-inadequada}}

\begin{itemize}
\tightlist
\item
  A análise do poder é teoricamente incorreta, uma vez que a probabilidade calculada \(1-\beta\) expressa a probabilidade de um evento futuro, o que não é mais relevante quando o evento de interesse já ocorreu.\textsuperscript{\protect\hyperlink{ref-Cummings2003}{123},\protect\hyperlink{ref-heckman2022}{159}}
\end{itemize}

\hypertarget{o-que-pode-ser-realizado-ao-invuxe9s-da-anuxe1lise-de-poder}{%
\subsection{O que pode ser realizado ao invés da análise de poder?}\label{o-que-pode-ser-realizado-ao-invuxe9s-da-anuxe1lise-de-poder}}

\begin{itemize}
\tightlist
\item
  Após a coleta e análise de dados, recomenda-se realizar a análise e interpretação dos resultados a partir do tamanho do efeito e do seu intervalo de confiança no nível de significância \(\alpha\) pré-estabelecido.\textsuperscript{\protect\hyperlink{ref-heckman2022}{159}}
\end{itemize}

\hypertarget{erros-de-inferuxeancia}{%
\section{Erros de inferência}\label{erros-de-inferuxeancia}}

\hypertarget{o-que-suxe3o-erros-de-inferuxeancia-estatuxedstica}{%
\subsection{O que são erros de inferência estatística?}\label{o-que-suxe3o-erros-de-inferuxeancia-estatuxedstica}}

\begin{itemize}
\tightlist
\item
  Um erro de inferência é a tomada de decisão incorreta, seja a favor ou contra a hipótese nula \(H_{0}\).\textsuperscript{\protect\hyperlink{ref-Curran-Everett2009}{144}}
\end{itemize}

\hypertarget{o-que-suxe3o-erros-tipo-i-e-tipo-ii}{%
\subsection{O que são erros tipo I e tipo II?}\label{o-que-suxe3o-erros-tipo-i-e-tipo-ii}}

\begin{itemize}
\item
  Erro tipo I significa a rejeição de uma hipótese nula (\(H_{0}\)) quando esta é verdadeira.\textsuperscript{\protect\hyperlink{ref-Curran-Everett2009}{144}}
\item
  Erro tipo II significa a não rejeição de uma hipótese nula (\(H_{0}\)) quando esta é falsa.\textsuperscript{\protect\hyperlink{ref-Curran-Everett2009}{144}}
\end{itemize}

\begin{table}

\caption{\label{tab:erros-inferencia-I-II}Tabela de erros de inferência estatística.}
\centering
\begin{tabu} to \linewidth {>{\raggedright}X>{\centering}X>{\centering}X}
\toprule
  & Hipótese nula $H_{0}$ 
 é falsa & Hipótese nula $H_{0}$ 
\textbf{ é verdadeira}\\
\midrule
Hipótese nula $H_{0}$ 
 foi rejeitada & Decisão correta & Decisão incorreta 
 (erro tipo I)\\
Hipótese nula $H_{0}$ 
 não foi rejeitada & Decisão incorreta 
 (erro tipo II) & Decisão correta\\
\bottomrule
\end{tabu}
\end{table}

\hypertarget{o-que-suxe3o-erros-tipo-s-e-tipo-m}{%
\subsection{O que são erros tipo S e tipo M?}\label{o-que-suxe3o-erros-tipo-s-e-tipo-m}}

\begin{itemize}
\item
  .{[}REF{]}
\item
  .{[}REF{]}
\end{itemize}

\hypertarget{interpretauxe7uxe3o-de-anuxe1lise-inferencial}{%
\section{Interpretação de análise inferencial}\label{interpretauxe7uxe3o-de-anuxe1lise-inferencial}}

\hypertarget{como-interpretar-uma-anuxe1lise-inferencial}{%
\subsection{Como interpretar uma análise inferencial?}\label{como-interpretar-uma-anuxe1lise-inferencial}}

\begin{itemize}
\item
  Testes de hipótese nula (\(H_{0}\)) vs.~alternativa (\(H_{1}\)) a partir de um nível de significância (\(\alpha\)) pré-especificado.\textsuperscript{\protect\hyperlink{ref-goodman2016}{155}}
\item
  P-valor como evidência estatística sobre (\(H_{0}\)).\textsuperscript{\protect\hyperlink{ref-goodman2016}{155}}
\item
  Estimação de intervalos de confiança de um nível de significância (\(\alpha\)) pré-especificado bicaudal (\(IC_{1-\alpha/2}\)) ou unicaudal (\(IC_{1-\alpha}\)).\textsuperscript{\protect\hyperlink{ref-goodman2016}{155}}
\item
  Análise Bayesiana.\textsuperscript{\protect\hyperlink{ref-goodman2016}{155}}
\end{itemize}

\hypertarget{o-que-suxe3o-resultados-positivos-e-negativos-em-teste-de-hipuxf3tese}{%
\subsection{O que são resultados `positivos' e `negativos' em teste de hipótese?}\label{o-que-suxe3o-resultados-positivos-e-negativos-em-teste-de-hipuxf3tese}}

\begin{itemize}
\item
  Resultados `positivos' compreendem um P-valor dentro da zona crítica estatisticamente significativa (ex.: P \textless{} 0,05 ou outro ponto de corte) e sugerem que os autores rejeitem a hipótese nula \(H_{0}\), confirmando assim sua hipótese científica.\textsuperscript{\protect\hyperlink{ref-greenhalgh1997a}{161}}
\item
  Resultados `negativos' compreendem um P-valor fora da zona crítica estatisticamente significativa (ex.: P ≥ 0,05 ou outro ponto de corte) e sugerem que os autores não rejeitem a hipótese nula \(H_{0}\) porque o efeito observado é nulo, ou porque o estudo não possui poder suficiente para detectá-lo, não permitindo portanto afirmar a hipótese científica.\textsuperscript{\protect\hyperlink{ref-greenhalgh1997a}{161}}
\end{itemize}

\hypertarget{qual-a-importuxe2ncia-de-resultados-negativos}{%
\subsection{Qual a importância de resultados `negativos'?}\label{qual-a-importuxe2ncia-de-resultados-negativos}}

\begin{itemize}
\item
  Conhecer resultados negativos contribui com uma visão mais ampla do campo de estudo junto aos resultados positivos.\textsuperscript{\protect\hyperlink{ref-weintraub2016}{162}}
\item
  Resultados negativos permitem um melhor planejamento das pesquisas futuras e pode aumentar suas chances de sucesso.\textsuperscript{\protect\hyperlink{ref-weintraub2016}{162}}
\end{itemize}

\hypertarget{o-que-suxe3o-resultados-inconclusivos}{%
\subsection{O que são resultados inconclusivos?}\label{o-que-suxe3o-resultados-inconclusivos}}

\begin{itemize}
\tightlist
\item
  {[}REF{]}
\end{itemize}

\hypertarget{ausuxeancia-de-eviduxeancia-ou-eviduxeancia-de-ausuxeancia}{%
\subsection{Ausência de evidência ou evidência de ausência?}\label{ausuxeancia-de-eviduxeancia-ou-eviduxeancia-de-ausuxeancia}}

\begin{itemize}
\item
  Em estudos (geralmente com amostras grandes), resultados estatisticamente significativos (com P-valores menores do limiar pré-estabelecido, \(P<\alpha\)) podem não ser clinicamente relevantes.\textsuperscript{\protect\hyperlink{ref-altman1995}{163}}
\item
  Em estudos (geralmente com amostras pequenas), resultados estatisticamente não significativos (com P-valores iguais ou maiores do limiar pré-estabelecido, \(P≥\alpha\)) não devem ser interpretados como evidência de inexistência do efeito.\textsuperscript{\protect\hyperlink{ref-altman1995}{163}}
\item
  Geralmente é razoável aceitar uma nova conclusão apenas quando há dados a seu favor (`resultados positivos'). Também é razoável questionar se apenas a ausência de dados a seu favor (`resultados negativos') justifica suficientemente a rejeição de tal conclusão.\textsuperscript{\protect\hyperlink{ref-altman1995}{163}}
\end{itemize}

\hypertarget{selecao-testes}{%
\chapter{\texorpdfstring{\textbf{Seleção de testes}}{Seleção de testes}}\label{selecao-testes}}

\hypertarget{multiverso-estatistica}{%
\section{Multiverso de análises estatísticas}\label{multiverso-estatistica}}

\hypertarget{por-que-escolher-o-teste-uxe9-um-problema}{%
\subsection{Por que escolher o teste é um problema?}\label{por-que-escolher-o-teste-uxe9-um-problema}}

\begin{itemize}
\item
  Analisar a mesma hipótese com o mesmo banco de dados pode resultar em diferenças substanciais nas estimativas estatísticas e nas conclusões.\textsuperscript{\protect\hyperlink{ref-Breznau2022}{164}}
\item
  As decisões para especificação das análises estatísticas podem ser tão minuciosas que muitas vezes nem sequer são registadas como decisões e, assim, podem impactar na reprodutibilidade do estudo.\textsuperscript{\protect\hyperlink{ref-Breznau2022}{164}}
\end{itemize}

\hypertarget{escolha-analise-inferencial}{%
\section{Escolha de testes para análise inferencial}\label{escolha-analise-inferencial}}

\hypertarget{como-selecionar-os-testes-para-a-anuxe1lise-estatuxedstica-inferencial}{%
\subsection{Como selecionar os testes para a análise estatística inferencial?}\label{como-selecionar-os-testes-para-a-anuxe1lise-estatuxedstica-inferencial}}

\begin{itemize}
\item
  .\textsuperscript{\protect\hyperlink{ref-dwivedi2019}{165}}
\item
  .\textsuperscript{\protect\hyperlink{ref-Dwivedi2022}{166}}
\item
  .\textsuperscript{\protect\hyperlink{ref-Kim2017}{167}}
\item
  .\textsuperscript{\protect\hyperlink{ref-marusteri2010}{168}}
\item
  .\textsuperscript{\protect\hyperlink{ref-mishra2019}{169}}
\item
  .\textsuperscript{\protect\hyperlink{ref-ray2021}{170}}
\item
  .\textsuperscript{\protect\hyperlink{ref-nayak2011}{171}}
\item
  .\textsuperscript{\protect\hyperlink{ref-shankar2014}{172}}
\end{itemize}

\hypertarget{testes-estatisticos}{%
\chapter{\texorpdfstring{\textbf{Testes estatísticos}}{Testes estatísticos}}\label{testes-estatisticos}}

\hypertarget{scripts-compartilhados}{%
\section{Scripts compartilhados}\label{scripts-compartilhados}}

\hypertarget{concordancia-e-confiabilidade}{%
\subsection{Concordancia e Confiabilidade}\label{concordancia-e-confiabilidade}}

\begin{itemize}
\tightlist
\item
  \href{https://github.com/FerreiraAS/Ciencia-com-R/blob/main/R/Concordancia\%20e\%20Confiabilidade/reliability-kappa-icc.R}{reliability-kappa-icc.R}
\end{itemize}

\hypertarget{descricao}{%
\subsection{Descricao}\label{descricao}}

\begin{itemize}
\item
  \href{https://github.com/FerreiraAS/Ciencia-com-R/blob/main/R/Descricao/extracolumn-es.R}{extracolumn-es.R}
\item
  \href{https://github.com/FerreiraAS/Ciencia-com-R/blob/main/R/Descricao/extracolumn-N.R}{extracolumn-N.R}
\item
  \href{https://github.com/FerreiraAS/Ciencia-com-R/blob/main/R/Descricao/extracolumn-p.R}{extracolumn-p.R}
\item
  \href{https://github.com/FerreiraAS/Ciencia-com-R/blob/main/R/Descricao/pilotdata_gopal.R}{pilotdata\_gopal.R}
\end{itemize}

\hypertarget{desempenho-diagnostico}{%
\subsection{Desempenho diagnostico}\label{desempenho-diagnostico}}

\begin{itemize}
\item
  \href{https://github.com/FerreiraAS/Ciencia-com-R/blob/main/R/Desempenho\%20diagnostico/diag-stats.R}{diag-stats.R}
\item
  \href{https://github.com/FerreiraAS/Ciencia-com-R/blob/main/R/Desempenho\%20diagnostico/dtROC.R}{dtROC.R}
\item
  \href{https://github.com/FerreiraAS/Ciencia-com-R/blob/main/R/Desempenho\%20diagnostico/stROC.R}{stROC.R}
\end{itemize}

\hypertarget{ensaio-clinico-aleatorizado}{%
\subsection{Ensaio clinico aleatorizado}\label{ensaio-clinico-aleatorizado}}

\begin{itemize}
\item
  \href{https://github.com/FerreiraAS/Ciencia-com-R/blob/main/R/Ensaio\%20clinico\%20aleatorizado/RCT-Figure1.R}{RCT-Figure1.R}
\item
  \href{https://github.com/FerreiraAS/Ciencia-com-R/blob/main/R/Ensaio\%20clinico\%20aleatorizado/RCT-Missingness.R}{RCT-Missingness.R}
\item
  \href{https://github.com/FerreiraAS/Ciencia-com-R/blob/main/R/Ensaio\%20clinico\%20aleatorizado/RCT-Table1.R}{RCT-Table1.R}
\item
  \href{https://github.com/FerreiraAS/Ciencia-com-R/blob/main/R/Ensaio\%20clinico\%20aleatorizado/RCT-Table2a.R}{RCT-Table2a.R}
\item
  \href{https://github.com/FerreiraAS/Ciencia-com-R/blob/main/R/Ensaio\%20clinico\%20aleatorizado/RCT-Table2b.R}{RCT-Table2b.R}
\item
  \href{https://github.com/FerreiraAS/Ciencia-com-R/blob/main/R/Ensaio\%20clinico\%20aleatorizado/RCT-Table3.R}{RCT-Table3.R}
\end{itemize}

\hypertarget{ensaio-cruzado}{%
\subsection{Ensaio cruzado}\label{ensaio-cruzado}}

\begin{itemize}
\item
  \href{https://github.com/FerreiraAS/Ciencia-com-R/blob/main/R/Ensaio\%20cruzado/crossover.R}{crossover.R}
\item
  \href{https://github.com/FerreiraAS/Ciencia-com-R/blob/main/R/Ensaio\%20cruzado/RSTR-crossover-trial.R}{RSTR-crossover-trial.R}
\end{itemize}

\hypertarget{regressao}{%
\subsection{Regressao}\label{regressao}}

\begin{itemize}
\item
  \href{https://github.com/FerreiraAS/Ciencia-com-R/blob/main/R/Regressao/mediation-analysis.R}{mediation-analysis.R}
\item
  \href{https://github.com/FerreiraAS/Ciencia-com-R/blob/main/R/Regressao/regression-diagnosis.R}{regression-diagnosis.R}
\end{itemize}

\begin{infobox}{images/Rlogo}
O pacote \emph{base}\textsuperscript{\protect\hyperlink{ref-base-2}{57}} fornece a função \href{https://www.rdocumentation.org/packages/base/versions/3.6.2/topics/source}{\emph{source}} para abrir um arquivo .R com script e executar seus comandos.

\end{infobox}

\hypertarget{testes-de-qui-quadrado-chi2}{%
\section{\texorpdfstring{Testes de Qui-quadrado (\(\chi^2\))}{Testes de Qui-quadrado (\textbackslash chi\^{}2)}}\label{testes-de-qui-quadrado-chi2}}

\begin{Shaded}
\begin{Highlighting}[]
\CommentTok{\# carrega os pacotes}
\FunctionTok{library}\NormalTok{(}\StringTok{"dplyr"}\NormalTok{)}
\FunctionTok{library}\NormalTok{(}\StringTok{"gtsummary"}\NormalTok{)}

\CommentTok{\# tabela 2x2}
\NormalTok{tbl\_cross }\OtherTok{\textless{}{-}}
  \CommentTok{\# banco de dados}
\NormalTok{  trial }\SpecialCharTok{\%\textgreater{}\%}
  \CommentTok{\# cria a tabela de contingência}
\NormalTok{  gtsummary}\SpecialCharTok{::}\FunctionTok{tbl\_cross}\NormalTok{(}
    \AttributeTok{row =}\NormalTok{ trt,}
    \AttributeTok{col =}\NormalTok{ response,}
    \AttributeTok{statistic =} \StringTok{"\{n\}"}\NormalTok{,}
    \AttributeTok{digits =} \DecValTok{0}\NormalTok{,}
    \AttributeTok{percent =} \StringTok{"cell"}\NormalTok{,}
    \AttributeTok{margin =} \FunctionTok{c}\NormalTok{(}\StringTok{"row"}\NormalTok{, }\StringTok{"column"}\NormalTok{),}
    \AttributeTok{missing =} \StringTok{"no"}\NormalTok{,}
    \AttributeTok{missing\_text =} \StringTok{"Dados perdidos"}\NormalTok{,}
    \AttributeTok{margin\_text =} \StringTok{"Total"}
\NormalTok{  ) }\SpecialCharTok{\%\textgreater{}\%}
  \CommentTok{\# calcula o p{-}valor do teste}
\NormalTok{  gtsummary}\SpecialCharTok{::}\FunctionTok{add\_p}\NormalTok{(}
    \AttributeTok{test =} \StringTok{"chisq.test"}\NormalTok{,}
    \AttributeTok{pvalue\_fun =} \ControlFlowTok{function}\NormalTok{(x) }\FunctionTok{style\_pvalue}\NormalTok{(x, }\AttributeTok{digits =} \DecValTok{3}\NormalTok{)}
\NormalTok{  ) }\SpecialCharTok{\%\textgreater{}\%}
\NormalTok{  gtsummary}\SpecialCharTok{::}\FunctionTok{modify\_header}\NormalTok{(}
    \AttributeTok{p.value =} \StringTok{"**P{-}valor**"}
\NormalTok{  ) }\SpecialCharTok{\%\textgreater{}\%}
  \CommentTok{\# calcula o tamanho do efeito}
\NormalTok{  gtsummary}\SpecialCharTok{::}\FunctionTok{modify\_table\_styling}\NormalTok{(}
    \AttributeTok{rows =} \ConstantTok{NULL}\NormalTok{,}
    \AttributeTok{footnote =} \FunctionTok{as.character}\NormalTok{(rstatix}\SpecialCharTok{::}\FunctionTok{cramer\_v}\NormalTok{(trt, response))}
\NormalTok{  ) }\SpecialCharTok{\%\textgreater{}\%}
  \CommentTok{\# formata o título em negrito}
\NormalTok{  gtsummary}\SpecialCharTok{::}\FunctionTok{bold\_labels}\NormalTok{() }\SpecialCharTok{\%\textgreater{}\%}
  \CommentTok{\# cria título da tabela}
\NormalTok{  gtsummary}\SpecialCharTok{::}\FunctionTok{modify\_caption}\NormalTok{(}
    \StringTok{"Teste Qui{-}quadrado (com correção de Yates)"}
\NormalTok{  )}

\CommentTok{\# exibe a tabela}
\NormalTok{tbl\_cross }\SpecialCharTok{\%\textgreater{}\%}
\NormalTok{  gtsummary}\SpecialCharTok{::}\FunctionTok{as\_hux\_table}\NormalTok{()}
\end{Highlighting}
\end{Shaded}

 
  \providecommand{\huxb}[2]{\arrayrulecolor[RGB]{#1}\global\arrayrulewidth=#2pt}
  \providecommand{\huxvb}[2]{\color[RGB]{#1}\vrule width #2pt}
  \providecommand{\huxtpad}[1]{\rule{0pt}{#1}}
  \providecommand{\huxbpad}[1]{\rule[-#1]{0pt}{#1}}

\begin{table}[ht]
\begin{centerbox}
\begin{threeparttable}
\captionsetup{justification=centering,singlelinecheck=off}
\caption{\label{tab:unnamed-chunk-3} Teste Qui-quadrado (com correção de Yates)}
 \setlength{\tabcolsep}{0pt}
\begin{tabular}{l l l l l}


\hhline{}
\arrayrulecolor{black}

\multicolumn{1}{!{\huxvb{0, 0, 0}{0}}l!{\huxvb{0, 0, 0}{0}}}{\huxtpad{6pt + 1em}\raggedright \hspace{6pt} 
 \hspace{6pt}\huxbpad{6pt}} &
\multicolumn{2}{c!{\huxvb{0, 0, 0}{0}}}{\huxtpad{6pt + 1em}\centering \hspace{6pt} \textbf{Tumor Response}
 \hspace{6pt}\huxbpad{6pt}} &
\multicolumn{1}{c!{\huxvb{0, 0, 0}{0}}}{\huxtpad{6pt + 1em}\centering \hspace{6pt} 
 \hspace{6pt}\huxbpad{6pt}} &
\multicolumn{1}{c!{\huxvb{0, 0, 0}{0}}}{\huxtpad{6pt + 1em}\centering \hspace{6pt} 
 \hspace{6pt}\huxbpad{6pt}} \tabularnewline[-0.5pt]


\hhline{}
\arrayrulecolor{black}

\multicolumn{1}{!{\huxvb{0, 0, 0}{0}}l!{\huxvb{0, 0, 0}{0}}}{\huxtpad{6pt + 1em}\raggedright \hspace{6pt} 
 \hspace{6pt}\huxbpad{6pt}} &
\multicolumn{1}{c!{\huxvb{0, 0, 0}{0}}}{\huxtpad{6pt + 1em}\centering \hspace{6pt} 0
 \hspace{6pt}\huxbpad{6pt}} &
\multicolumn{1}{c!{\huxvb{0, 0, 0}{0}}}{\huxtpad{6pt + 1em}\centering \hspace{6pt} 1
 \hspace{6pt}\huxbpad{6pt}} &
\multicolumn{1}{c!{\huxvb{0, 0, 0}{0}}}{\huxtpad{6pt + 1em}\centering \hspace{6pt} \textbf{Total}
 \hspace{6pt}\huxbpad{6pt}} &
\multicolumn{1}{c!{\huxvb{0, 0, 0}{0}}}{\huxtpad{6pt + 1em}\centering \hspace{6pt} \textbf{\textbf{P-valor}}
 \hspace{6pt}\huxbpad{6pt}} \tabularnewline[-0.5pt]


\hhline{>{\huxb{0, 0, 0}{0.4}}->{\huxb{0, 0, 0}{0.4}}->{\huxb{0, 0, 0}{0.4}}->{\huxb{0, 0, 0}{0.4}}->{\huxb{0, 0, 0}{0.4}}-}
\arrayrulecolor{black}

\multicolumn{1}{!{\huxvb{0, 0, 0}{0}}l!{\huxvb{0, 0, 0}{0}}}{\huxtpad{6pt + 1em}\raggedright \hspace{6pt} \textbf{Chemotherapy Treatment} \hspace{6pt}\huxbpad{6pt}} &
\multicolumn{1}{c!{\huxvb{0, 0, 0}{0}}}{\huxtpad{6pt + 1em}\centering \hspace{6pt}  \hspace{6pt}\huxbpad{6pt}} &
\multicolumn{1}{c!{\huxvb{0, 0, 0}{0}}}{\huxtpad{6pt + 1em}\centering \hspace{6pt}  \hspace{6pt}\huxbpad{6pt}} &
\multicolumn{1}{c!{\huxvb{0, 0, 0}{0}}}{\huxtpad{6pt + 1em}\centering \hspace{6pt}  \hspace{6pt}\huxbpad{6pt}} &
\multicolumn{1}{c!{\huxvb{0, 0, 0}{0}}}{\huxtpad{6pt + 1em}\centering \hspace{6pt} 0.637 \hspace{6pt}\huxbpad{6pt}} \tabularnewline[-0.5pt]


\hhline{}
\arrayrulecolor{black}

\multicolumn{1}{!{\huxvb{0, 0, 0}{0}}l!{\huxvb{0, 0, 0}{0}}}{\huxtpad{6pt + 1em}\raggedright \hspace{15pt} Drug A \hspace{6pt}\huxbpad{6pt}} &
\multicolumn{1}{c!{\huxvb{0, 0, 0}{0}}}{\huxtpad{6pt + 1em}\centering \hspace{6pt} 67 \hspace{6pt}\huxbpad{6pt}} &
\multicolumn{1}{c!{\huxvb{0, 0, 0}{0}}}{\huxtpad{6pt + 1em}\centering \hspace{6pt} 28 \hspace{6pt}\huxbpad{6pt}} &
\multicolumn{1}{c!{\huxvb{0, 0, 0}{0}}}{\huxtpad{6pt + 1em}\centering \hspace{6pt} 95 \hspace{6pt}\huxbpad{6pt}} &
\multicolumn{1}{c!{\huxvb{0, 0, 0}{0}}}{\huxtpad{6pt + 1em}\centering \hspace{6pt}  \hspace{6pt}\huxbpad{6pt}} \tabularnewline[-0.5pt]


\hhline{}
\arrayrulecolor{black}

\multicolumn{1}{!{\huxvb{0, 0, 0}{0}}l!{\huxvb{0, 0, 0}{0}}}{\huxtpad{6pt + 1em}\raggedright \hspace{15pt} Drug B \hspace{6pt}\huxbpad{6pt}} &
\multicolumn{1}{c!{\huxvb{0, 0, 0}{0}}}{\huxtpad{6pt + 1em}\centering \hspace{6pt} 65 \hspace{6pt}\huxbpad{6pt}} &
\multicolumn{1}{c!{\huxvb{0, 0, 0}{0}}}{\huxtpad{6pt + 1em}\centering \hspace{6pt} 33 \hspace{6pt}\huxbpad{6pt}} &
\multicolumn{1}{c!{\huxvb{0, 0, 0}{0}}}{\huxtpad{6pt + 1em}\centering \hspace{6pt} 98 \hspace{6pt}\huxbpad{6pt}} &
\multicolumn{1}{c!{\huxvb{0, 0, 0}{0}}}{\huxtpad{6pt + 1em}\centering \hspace{6pt}  \hspace{6pt}\huxbpad{6pt}} \tabularnewline[-0.5pt]


\hhline{}
\arrayrulecolor{black}

\multicolumn{1}{!{\huxvb{0, 0, 0}{0}}l!{\huxvb{0, 0, 0}{0}}}{\huxtpad{6pt + 1em}\raggedright \hspace{6pt} \textbf{Total} \hspace{6pt}\huxbpad{6pt}} &
\multicolumn{1}{c!{\huxvb{0, 0, 0}{0}}}{\huxtpad{6pt + 1em}\centering \hspace{6pt} 132 \hspace{6pt}\huxbpad{6pt}} &
\multicolumn{1}{c!{\huxvb{0, 0, 0}{0}}}{\huxtpad{6pt + 1em}\centering \hspace{6pt} 61 \hspace{6pt}\huxbpad{6pt}} &
\multicolumn{1}{c!{\huxvb{0, 0, 0}{0}}}{\huxtpad{6pt + 1em}\centering \hspace{6pt} 193 \hspace{6pt}\huxbpad{6pt}} &
\multicolumn{1}{c!{\huxvb{0, 0, 0}{0}}}{\huxtpad{6pt + 1em}\centering \hspace{6pt}  \hspace{6pt}\huxbpad{6pt}} \tabularnewline[-0.5pt]


\hhline{>{\huxb{0, 0, 0}{0.8}}->{\huxb{0, 0, 0}{0.8}}->{\huxb{0, 0, 0}{0.8}}->{\huxb{0, 0, 0}{0.8}}->{\huxb{0, 0, 0}{0.8}}-}
\arrayrulecolor{black}

\multicolumn{5}{!{\huxvb{0, 0, 0}{0}}l!{\huxvb{0, 0, 0}{0}}}{\huxtpad{6pt + 1em}\raggedright \hspace{6pt} Pearson's Chi-squared test \hspace{6pt}\huxbpad{6pt}} \tabularnewline[-0.5pt]


\hhline{}
\arrayrulecolor{black}
\end{tabular}
\end{threeparttable}\par\end{centerbox}

\end{table}
 

\begin{Shaded}
\begin{Highlighting}[]
\CommentTok{\# carrega os pacotes}
\FunctionTok{library}\NormalTok{(}\StringTok{"dplyr"}\NormalTok{)}
\FunctionTok{library}\NormalTok{(}\StringTok{"gtsummary"}\NormalTok{)}

\CommentTok{\# tabela 2x2}
\NormalTok{tbl\_cross }\OtherTok{\textless{}{-}}
  \CommentTok{\# banco de dados}
\NormalTok{  trial }\SpecialCharTok{\%\textgreater{}\%}
  \CommentTok{\# cria a tabela de contingência}
\NormalTok{  gtsummary}\SpecialCharTok{::}\FunctionTok{tbl\_cross}\NormalTok{(}
    \AttributeTok{row =}\NormalTok{ trt,}
    \AttributeTok{col =}\NormalTok{ response,}
    \AttributeTok{statistic =} \StringTok{"\{n\}"}\NormalTok{,}
    \AttributeTok{digits =} \DecValTok{0}\NormalTok{,}
    \AttributeTok{percent =} \StringTok{"cell"}\NormalTok{,}
    \AttributeTok{margin =} \FunctionTok{c}\NormalTok{(}\StringTok{"row"}\NormalTok{, }\StringTok{"column"}\NormalTok{),}
    \AttributeTok{missing =} \StringTok{"no"}\NormalTok{,}
    \AttributeTok{missing\_text =} \StringTok{"Dados perdidos"}\NormalTok{,}
    \AttributeTok{margin\_text =} \StringTok{"Total"}
\NormalTok{  ) }\SpecialCharTok{\%\textgreater{}\%}
  \CommentTok{\# calcula o p{-}valor do teste}
\NormalTok{  gtsummary}\SpecialCharTok{::}\FunctionTok{add\_p}\NormalTok{(}
    \AttributeTok{test =} \StringTok{"chisq.test.no.correct"}\NormalTok{,}
    \AttributeTok{pvalue\_fun =} \ControlFlowTok{function}\NormalTok{(x) }\FunctionTok{style\_pvalue}\NormalTok{(x, }\AttributeTok{digits =} \DecValTok{3}\NormalTok{)}
\NormalTok{  ) }\SpecialCharTok{\%\textgreater{}\%}
\NormalTok{  gtsummary}\SpecialCharTok{::}\FunctionTok{modify\_header}\NormalTok{(}
    \AttributeTok{p.value =} \StringTok{"**P{-}valor**"}
\NormalTok{  ) }\SpecialCharTok{\%\textgreater{}\%}
  \CommentTok{\# calcula o tamanho do efeito}
\NormalTok{  gtsummary}\SpecialCharTok{::}\FunctionTok{modify\_table\_styling}\NormalTok{(}
    \AttributeTok{rows =} \ConstantTok{NULL}\NormalTok{,}
    \AttributeTok{footnote =} \FunctionTok{as.character}\NormalTok{(rstatix}\SpecialCharTok{::}\FunctionTok{cramer\_v}\NormalTok{(trt, response))}
\NormalTok{  ) }\SpecialCharTok{\%\textgreater{}\%}
  \CommentTok{\# formata o título em negrito}
\NormalTok{  gtsummary}\SpecialCharTok{::}\FunctionTok{bold\_labels}\NormalTok{() }\SpecialCharTok{\%\textgreater{}\%}
  \CommentTok{\# cria título da tabela}
\NormalTok{  gtsummary}\SpecialCharTok{::}\FunctionTok{modify\_caption}\NormalTok{(}
    \StringTok{"Teste Qui{-}quadrado (sem correção de Yates)"}
\NormalTok{  )}

\CommentTok{\# exibe a tabela}
\NormalTok{tbl\_cross }\SpecialCharTok{\%\textgreater{}\%}
\NormalTok{  gtsummary}\SpecialCharTok{::}\FunctionTok{as\_hux\_table}\NormalTok{()}
\end{Highlighting}
\end{Shaded}

 
  \providecommand{\huxb}[2]{\arrayrulecolor[RGB]{#1}\global\arrayrulewidth=#2pt}
  \providecommand{\huxvb}[2]{\color[RGB]{#1}\vrule width #2pt}
  \providecommand{\huxtpad}[1]{\rule{0pt}{#1}}
  \providecommand{\huxbpad}[1]{\rule[-#1]{0pt}{#1}}

\begin{table}[ht]
\begin{centerbox}
\begin{threeparttable}
\captionsetup{justification=centering,singlelinecheck=off}
\caption{\label{tab:unnamed-chunk-4} Teste Qui-quadrado (sem correção de Yates)}
 \setlength{\tabcolsep}{0pt}
\begin{tabular}{l l l l l}


\hhline{}
\arrayrulecolor{black}

\multicolumn{1}{!{\huxvb{0, 0, 0}{0}}l!{\huxvb{0, 0, 0}{0}}}{\huxtpad{6pt + 1em}\raggedright \hspace{6pt} 
 \hspace{6pt}\huxbpad{6pt}} &
\multicolumn{2}{c!{\huxvb{0, 0, 0}{0}}}{\huxtpad{6pt + 1em}\centering \hspace{6pt} \textbf{Tumor Response}
 \hspace{6pt}\huxbpad{6pt}} &
\multicolumn{1}{c!{\huxvb{0, 0, 0}{0}}}{\huxtpad{6pt + 1em}\centering \hspace{6pt} 
 \hspace{6pt}\huxbpad{6pt}} &
\multicolumn{1}{c!{\huxvb{0, 0, 0}{0}}}{\huxtpad{6pt + 1em}\centering \hspace{6pt} 
 \hspace{6pt}\huxbpad{6pt}} \tabularnewline[-0.5pt]


\hhline{}
\arrayrulecolor{black}

\multicolumn{1}{!{\huxvb{0, 0, 0}{0}}l!{\huxvb{0, 0, 0}{0}}}{\huxtpad{6pt + 1em}\raggedright \hspace{6pt} 
 \hspace{6pt}\huxbpad{6pt}} &
\multicolumn{1}{c!{\huxvb{0, 0, 0}{0}}}{\huxtpad{6pt + 1em}\centering \hspace{6pt} 0
 \hspace{6pt}\huxbpad{6pt}} &
\multicolumn{1}{c!{\huxvb{0, 0, 0}{0}}}{\huxtpad{6pt + 1em}\centering \hspace{6pt} 1
 \hspace{6pt}\huxbpad{6pt}} &
\multicolumn{1}{c!{\huxvb{0, 0, 0}{0}}}{\huxtpad{6pt + 1em}\centering \hspace{6pt} \textbf{Total}
 \hspace{6pt}\huxbpad{6pt}} &
\multicolumn{1}{c!{\huxvb{0, 0, 0}{0}}}{\huxtpad{6pt + 1em}\centering \hspace{6pt} \textbf{\textbf{P-valor}}
 \hspace{6pt}\huxbpad{6pt}} \tabularnewline[-0.5pt]


\hhline{>{\huxb{0, 0, 0}{0.4}}->{\huxb{0, 0, 0}{0.4}}->{\huxb{0, 0, 0}{0.4}}->{\huxb{0, 0, 0}{0.4}}->{\huxb{0, 0, 0}{0.4}}-}
\arrayrulecolor{black}

\multicolumn{1}{!{\huxvb{0, 0, 0}{0}}l!{\huxvb{0, 0, 0}{0}}}{\huxtpad{6pt + 1em}\raggedright \hspace{6pt} \textbf{Chemotherapy Treatment} \hspace{6pt}\huxbpad{6pt}} &
\multicolumn{1}{c!{\huxvb{0, 0, 0}{0}}}{\huxtpad{6pt + 1em}\centering \hspace{6pt}  \hspace{6pt}\huxbpad{6pt}} &
\multicolumn{1}{c!{\huxvb{0, 0, 0}{0}}}{\huxtpad{6pt + 1em}\centering \hspace{6pt}  \hspace{6pt}\huxbpad{6pt}} &
\multicolumn{1}{c!{\huxvb{0, 0, 0}{0}}}{\huxtpad{6pt + 1em}\centering \hspace{6pt}  \hspace{6pt}\huxbpad{6pt}} &
\multicolumn{1}{c!{\huxvb{0, 0, 0}{0}}}{\huxtpad{6pt + 1em}\centering \hspace{6pt} 0.530 \hspace{6pt}\huxbpad{6pt}} \tabularnewline[-0.5pt]


\hhline{}
\arrayrulecolor{black}

\multicolumn{1}{!{\huxvb{0, 0, 0}{0}}l!{\huxvb{0, 0, 0}{0}}}{\huxtpad{6pt + 1em}\raggedright \hspace{15pt} Drug A \hspace{6pt}\huxbpad{6pt}} &
\multicolumn{1}{c!{\huxvb{0, 0, 0}{0}}}{\huxtpad{6pt + 1em}\centering \hspace{6pt} 67 \hspace{6pt}\huxbpad{6pt}} &
\multicolumn{1}{c!{\huxvb{0, 0, 0}{0}}}{\huxtpad{6pt + 1em}\centering \hspace{6pt} 28 \hspace{6pt}\huxbpad{6pt}} &
\multicolumn{1}{c!{\huxvb{0, 0, 0}{0}}}{\huxtpad{6pt + 1em}\centering \hspace{6pt} 95 \hspace{6pt}\huxbpad{6pt}} &
\multicolumn{1}{c!{\huxvb{0, 0, 0}{0}}}{\huxtpad{6pt + 1em}\centering \hspace{6pt}  \hspace{6pt}\huxbpad{6pt}} \tabularnewline[-0.5pt]


\hhline{}
\arrayrulecolor{black}

\multicolumn{1}{!{\huxvb{0, 0, 0}{0}}l!{\huxvb{0, 0, 0}{0}}}{\huxtpad{6pt + 1em}\raggedright \hspace{15pt} Drug B \hspace{6pt}\huxbpad{6pt}} &
\multicolumn{1}{c!{\huxvb{0, 0, 0}{0}}}{\huxtpad{6pt + 1em}\centering \hspace{6pt} 65 \hspace{6pt}\huxbpad{6pt}} &
\multicolumn{1}{c!{\huxvb{0, 0, 0}{0}}}{\huxtpad{6pt + 1em}\centering \hspace{6pt} 33 \hspace{6pt}\huxbpad{6pt}} &
\multicolumn{1}{c!{\huxvb{0, 0, 0}{0}}}{\huxtpad{6pt + 1em}\centering \hspace{6pt} 98 \hspace{6pt}\huxbpad{6pt}} &
\multicolumn{1}{c!{\huxvb{0, 0, 0}{0}}}{\huxtpad{6pt + 1em}\centering \hspace{6pt}  \hspace{6pt}\huxbpad{6pt}} \tabularnewline[-0.5pt]


\hhline{}
\arrayrulecolor{black}

\multicolumn{1}{!{\huxvb{0, 0, 0}{0}}l!{\huxvb{0, 0, 0}{0}}}{\huxtpad{6pt + 1em}\raggedright \hspace{6pt} \textbf{Total} \hspace{6pt}\huxbpad{6pt}} &
\multicolumn{1}{c!{\huxvb{0, 0, 0}{0}}}{\huxtpad{6pt + 1em}\centering \hspace{6pt} 132 \hspace{6pt}\huxbpad{6pt}} &
\multicolumn{1}{c!{\huxvb{0, 0, 0}{0}}}{\huxtpad{6pt + 1em}\centering \hspace{6pt} 61 \hspace{6pt}\huxbpad{6pt}} &
\multicolumn{1}{c!{\huxvb{0, 0, 0}{0}}}{\huxtpad{6pt + 1em}\centering \hspace{6pt} 193 \hspace{6pt}\huxbpad{6pt}} &
\multicolumn{1}{c!{\huxvb{0, 0, 0}{0}}}{\huxtpad{6pt + 1em}\centering \hspace{6pt}  \hspace{6pt}\huxbpad{6pt}} \tabularnewline[-0.5pt]


\hhline{>{\huxb{0, 0, 0}{0.8}}->{\huxb{0, 0, 0}{0.8}}->{\huxb{0, 0, 0}{0.8}}->{\huxb{0, 0, 0}{0.8}}->{\huxb{0, 0, 0}{0.8}}-}
\arrayrulecolor{black}

\multicolumn{5}{!{\huxvb{0, 0, 0}{0}}l!{\huxvb{0, 0, 0}{0}}}{\huxtpad{6pt + 1em}\raggedright \hspace{6pt} Pearson's Chi-squared test \hspace{6pt}\huxbpad{6pt}} \tabularnewline[-0.5pt]


\hhline{}
\arrayrulecolor{black}
\end{tabular}
\end{threeparttable}\par\end{centerbox}

\end{table}
 

\hypertarget{teste-exato-de-fisher}{%
\section{Teste exato de Fisher}\label{teste-exato-de-fisher}}

\begin{Shaded}
\begin{Highlighting}[]
\CommentTok{\# carrega os pacotes}
\FunctionTok{library}\NormalTok{(}\StringTok{"dplyr"}\NormalTok{)}
\FunctionTok{library}\NormalTok{(}\StringTok{"gtsummary"}\NormalTok{)}

\CommentTok{\# tabela 2x2}
\NormalTok{tbl\_cross }\OtherTok{\textless{}{-}}
  \CommentTok{\# banco de dados}
\NormalTok{  trial }\SpecialCharTok{\%\textgreater{}\%}
  \CommentTok{\# cria a tabela de contingência}
\NormalTok{  gtsummary}\SpecialCharTok{::}\FunctionTok{tbl\_cross}\NormalTok{(}
    \AttributeTok{row =}\NormalTok{ trt,}
    \AttributeTok{col =}\NormalTok{ response,}
    \AttributeTok{statistic =} \StringTok{"\{n\}"}\NormalTok{,}
    \AttributeTok{digits =} \DecValTok{0}\NormalTok{,}
    \AttributeTok{percent =} \StringTok{"cell"}\NormalTok{,}
    \AttributeTok{margin =} \FunctionTok{c}\NormalTok{(}\StringTok{"row"}\NormalTok{, }\StringTok{"column"}\NormalTok{),}
    \AttributeTok{missing =} \StringTok{"no"}\NormalTok{,}
    \AttributeTok{missing\_text =} \StringTok{"Dados perdidos"}\NormalTok{,}
    \AttributeTok{margin\_text =} \StringTok{"Total"}
\NormalTok{  ) }\SpecialCharTok{\%\textgreater{}\%}
  \CommentTok{\# calcula o p{-}valor do teste}
\NormalTok{  gtsummary}\SpecialCharTok{::}\FunctionTok{add\_p}\NormalTok{(}
    \AttributeTok{test =} \StringTok{"fisher.test"}\NormalTok{,}
    \AttributeTok{pvalue\_fun =} \ControlFlowTok{function}\NormalTok{(x) }\FunctionTok{style\_pvalue}\NormalTok{(x, }\AttributeTok{digits =} \DecValTok{3}\NormalTok{)}
\NormalTok{  ) }\SpecialCharTok{\%\textgreater{}\%}
\NormalTok{  gtsummary}\SpecialCharTok{::}\FunctionTok{modify\_header}\NormalTok{(}
    \AttributeTok{p.value =} \StringTok{"**P{-}valor**"}
\NormalTok{  ) }\SpecialCharTok{\%\textgreater{}\%}
  \CommentTok{\# calcula o tamanho do efeito}
\NormalTok{  gtsummary}\SpecialCharTok{::}\FunctionTok{modify\_table\_styling}\NormalTok{(}
    \AttributeTok{rows =} \ConstantTok{NULL}\NormalTok{,}
    \AttributeTok{footnote =} \FunctionTok{as.character}\NormalTok{(rstatix}\SpecialCharTok{::}\FunctionTok{cramer\_v}\NormalTok{(trt, response))}
\NormalTok{  ) }\SpecialCharTok{\%\textgreater{}\%}
  \CommentTok{\# formata o título em negrito}
\NormalTok{  gtsummary}\SpecialCharTok{::}\FunctionTok{bold\_labels}\NormalTok{() }\SpecialCharTok{\%\textgreater{}\%}
  \CommentTok{\# cria título da tabela}
\NormalTok{  gtsummary}\SpecialCharTok{::}\FunctionTok{modify\_caption}\NormalTok{(}
    \StringTok{"Teste exato de Fisher"}
\NormalTok{  )}

\CommentTok{\# exibe a tabela}
\NormalTok{tbl\_cross }\SpecialCharTok{\%\textgreater{}\%}
\NormalTok{  gtsummary}\SpecialCharTok{::}\FunctionTok{as\_hux\_table}\NormalTok{()}
\end{Highlighting}
\end{Shaded}

 
  \providecommand{\huxb}[2]{\arrayrulecolor[RGB]{#1}\global\arrayrulewidth=#2pt}
  \providecommand{\huxvb}[2]{\color[RGB]{#1}\vrule width #2pt}
  \providecommand{\huxtpad}[1]{\rule{0pt}{#1}}
  \providecommand{\huxbpad}[1]{\rule[-#1]{0pt}{#1}}

\begin{table}[ht]
\begin{centerbox}
\begin{threeparttable}
\captionsetup{justification=centering,singlelinecheck=off}
\caption{\label{tab:unnamed-chunk-5} Teste exato de Fisher}
 \setlength{\tabcolsep}{0pt}
\begin{tabular}{l l l l l}


\hhline{}
\arrayrulecolor{black}

\multicolumn{1}{!{\huxvb{0, 0, 0}{0}}l!{\huxvb{0, 0, 0}{0}}}{\huxtpad{6pt + 1em}\raggedright \hspace{6pt} 
 \hspace{6pt}\huxbpad{6pt}} &
\multicolumn{2}{c!{\huxvb{0, 0, 0}{0}}}{\huxtpad{6pt + 1em}\centering \hspace{6pt} \textbf{Tumor Response}
 \hspace{6pt}\huxbpad{6pt}} &
\multicolumn{1}{c!{\huxvb{0, 0, 0}{0}}}{\huxtpad{6pt + 1em}\centering \hspace{6pt} 
 \hspace{6pt}\huxbpad{6pt}} &
\multicolumn{1}{c!{\huxvb{0, 0, 0}{0}}}{\huxtpad{6pt + 1em}\centering \hspace{6pt} 
 \hspace{6pt}\huxbpad{6pt}} \tabularnewline[-0.5pt]


\hhline{}
\arrayrulecolor{black}

\multicolumn{1}{!{\huxvb{0, 0, 0}{0}}l!{\huxvb{0, 0, 0}{0}}}{\huxtpad{6pt + 1em}\raggedright \hspace{6pt} 
 \hspace{6pt}\huxbpad{6pt}} &
\multicolumn{1}{c!{\huxvb{0, 0, 0}{0}}}{\huxtpad{6pt + 1em}\centering \hspace{6pt} 0
 \hspace{6pt}\huxbpad{6pt}} &
\multicolumn{1}{c!{\huxvb{0, 0, 0}{0}}}{\huxtpad{6pt + 1em}\centering \hspace{6pt} 1
 \hspace{6pt}\huxbpad{6pt}} &
\multicolumn{1}{c!{\huxvb{0, 0, 0}{0}}}{\huxtpad{6pt + 1em}\centering \hspace{6pt} \textbf{Total}
 \hspace{6pt}\huxbpad{6pt}} &
\multicolumn{1}{c!{\huxvb{0, 0, 0}{0}}}{\huxtpad{6pt + 1em}\centering \hspace{6pt} \textbf{\textbf{P-valor}}
 \hspace{6pt}\huxbpad{6pt}} \tabularnewline[-0.5pt]


\hhline{>{\huxb{0, 0, 0}{0.4}}->{\huxb{0, 0, 0}{0.4}}->{\huxb{0, 0, 0}{0.4}}->{\huxb{0, 0, 0}{0.4}}->{\huxb{0, 0, 0}{0.4}}-}
\arrayrulecolor{black}

\multicolumn{1}{!{\huxvb{0, 0, 0}{0}}l!{\huxvb{0, 0, 0}{0}}}{\huxtpad{6pt + 1em}\raggedright \hspace{6pt} \textbf{Chemotherapy Treatment} \hspace{6pt}\huxbpad{6pt}} &
\multicolumn{1}{c!{\huxvb{0, 0, 0}{0}}}{\huxtpad{6pt + 1em}\centering \hspace{6pt}  \hspace{6pt}\huxbpad{6pt}} &
\multicolumn{1}{c!{\huxvb{0, 0, 0}{0}}}{\huxtpad{6pt + 1em}\centering \hspace{6pt}  \hspace{6pt}\huxbpad{6pt}} &
\multicolumn{1}{c!{\huxvb{0, 0, 0}{0}}}{\huxtpad{6pt + 1em}\centering \hspace{6pt}  \hspace{6pt}\huxbpad{6pt}} &
\multicolumn{1}{c!{\huxvb{0, 0, 0}{0}}}{\huxtpad{6pt + 1em}\centering \hspace{6pt} 0.540 \hspace{6pt}\huxbpad{6pt}} \tabularnewline[-0.5pt]


\hhline{}
\arrayrulecolor{black}

\multicolumn{1}{!{\huxvb{0, 0, 0}{0}}l!{\huxvb{0, 0, 0}{0}}}{\huxtpad{6pt + 1em}\raggedright \hspace{15pt} Drug A \hspace{6pt}\huxbpad{6pt}} &
\multicolumn{1}{c!{\huxvb{0, 0, 0}{0}}}{\huxtpad{6pt + 1em}\centering \hspace{6pt} 67 \hspace{6pt}\huxbpad{6pt}} &
\multicolumn{1}{c!{\huxvb{0, 0, 0}{0}}}{\huxtpad{6pt + 1em}\centering \hspace{6pt} 28 \hspace{6pt}\huxbpad{6pt}} &
\multicolumn{1}{c!{\huxvb{0, 0, 0}{0}}}{\huxtpad{6pt + 1em}\centering \hspace{6pt} 95 \hspace{6pt}\huxbpad{6pt}} &
\multicolumn{1}{c!{\huxvb{0, 0, 0}{0}}}{\huxtpad{6pt + 1em}\centering \hspace{6pt}  \hspace{6pt}\huxbpad{6pt}} \tabularnewline[-0.5pt]


\hhline{}
\arrayrulecolor{black}

\multicolumn{1}{!{\huxvb{0, 0, 0}{0}}l!{\huxvb{0, 0, 0}{0}}}{\huxtpad{6pt + 1em}\raggedright \hspace{15pt} Drug B \hspace{6pt}\huxbpad{6pt}} &
\multicolumn{1}{c!{\huxvb{0, 0, 0}{0}}}{\huxtpad{6pt + 1em}\centering \hspace{6pt} 65 \hspace{6pt}\huxbpad{6pt}} &
\multicolumn{1}{c!{\huxvb{0, 0, 0}{0}}}{\huxtpad{6pt + 1em}\centering \hspace{6pt} 33 \hspace{6pt}\huxbpad{6pt}} &
\multicolumn{1}{c!{\huxvb{0, 0, 0}{0}}}{\huxtpad{6pt + 1em}\centering \hspace{6pt} 98 \hspace{6pt}\huxbpad{6pt}} &
\multicolumn{1}{c!{\huxvb{0, 0, 0}{0}}}{\huxtpad{6pt + 1em}\centering \hspace{6pt}  \hspace{6pt}\huxbpad{6pt}} \tabularnewline[-0.5pt]


\hhline{}
\arrayrulecolor{black}

\multicolumn{1}{!{\huxvb{0, 0, 0}{0}}l!{\huxvb{0, 0, 0}{0}}}{\huxtpad{6pt + 1em}\raggedright \hspace{6pt} \textbf{Total} \hspace{6pt}\huxbpad{6pt}} &
\multicolumn{1}{c!{\huxvb{0, 0, 0}{0}}}{\huxtpad{6pt + 1em}\centering \hspace{6pt} 132 \hspace{6pt}\huxbpad{6pt}} &
\multicolumn{1}{c!{\huxvb{0, 0, 0}{0}}}{\huxtpad{6pt + 1em}\centering \hspace{6pt} 61 \hspace{6pt}\huxbpad{6pt}} &
\multicolumn{1}{c!{\huxvb{0, 0, 0}{0}}}{\huxtpad{6pt + 1em}\centering \hspace{6pt} 193 \hspace{6pt}\huxbpad{6pt}} &
\multicolumn{1}{c!{\huxvb{0, 0, 0}{0}}}{\huxtpad{6pt + 1em}\centering \hspace{6pt}  \hspace{6pt}\huxbpad{6pt}} \tabularnewline[-0.5pt]


\hhline{>{\huxb{0, 0, 0}{0.8}}->{\huxb{0, 0, 0}{0.8}}->{\huxb{0, 0, 0}{0.8}}->{\huxb{0, 0, 0}{0.8}}->{\huxb{0, 0, 0}{0.8}}-}
\arrayrulecolor{black}

\multicolumn{5}{!{\huxvb{0, 0, 0}{0}}l!{\huxvb{0, 0, 0}{0}}}{\huxtpad{6pt + 1em}\raggedright \hspace{6pt} Fisher's exact test \hspace{6pt}\huxbpad{6pt}} \tabularnewline[-0.5pt]


\hhline{}
\arrayrulecolor{black}
\end{tabular}
\end{threeparttable}\par\end{centerbox}

\end{table}
 

\cftaddtitleline{toc}{chapter}{\rule{\textwidth}{0.4pt}}{}

\hypertarget{parte-3---estatuxedstica-aplicada}{%
\chapter*{\texorpdfstring{\emph{Parte 3 - Estatística Aplicada}}{Parte 3 - Estatística Aplicada}}\label{parte-3---estatuxedstica-aplicada}}
\addcontentsline{toc}{chapter}{\emph{Parte 3 - Estatística Aplicada}}

\markboth{}{}
\par\noindent\rule{\textwidth}{0.05in}

\hypertarget{analise-descricao}{%
\chapter{\texorpdfstring{\textbf{Descrição}}{Descrição}}\label{analise-descricao}}

\hypertarget{analise-descricao}{%
\section{Análise de descrição}\label{analise-descricao}}

\hypertarget{o-que-uxe9-anuxe1lise-de-descriuxe7uxe3o-de-dados}{%
\subsection{O que é análise de descrição de dados?}\label{o-que-uxe9-anuxe1lise-de-descriuxe7uxe3o-de-dados}}

\begin{itemize}
\tightlist
\item
  .{[}REF{]}
\end{itemize}

\hypertarget{analise-comparacao}{%
\chapter{\texorpdfstring{\textbf{Comparação}}{Comparação}}\label{analise-comparacao}}

\hypertarget{analise-inferencial-comparacao}{%
\section{Análise inferencial de comparação}\label{analise-inferencial-comparacao}}

\hypertarget{o-que-uxe9-anuxe1lise-de-comparauxe7uxe3o-de-dados}{%
\subsection{O que é análise de comparação de dados?}\label{o-que-uxe9-anuxe1lise-de-comparauxe7uxe3o-de-dados}}

\begin{itemize}
\tightlist
\item
  .{[}REF{]}
\end{itemize}

\begin{infobox}{images/Rlogo}
O pacote \emph{cocor}\textsuperscript{\protect\hyperlink{ref-cocor-4}{173}} fornece as funções \href{https://www.rdocumentation.org/packages/stats/versions/3.6.2/topics/cor.test}{cocor.indep.groups}, \href{https://www.rdocumentation.org/packages/stats/versions/3.6.2/topics/cor.test}{cocor.dep.groups.overlap} e \href{https://www.rdocumentation.org/packages/stats/versions/3.6.2/topics/cor.test}{cocor.dep.groups.nonoverlap} para comparar 2 coeficientes de correlação entre grupos independentes, grupos sobrepostos ou independentes, respectivamente.\textsuperscript{\protect\hyperlink{ref-cocor}{174}}

\end{infobox}

\hypertarget{analise-inferencial-correlacao}{%
\chapter{\texorpdfstring{\textbf{Correlação}}{Correlação}}\label{analise-inferencial-correlacao}}

\hypertarget{analise-correlacao}{%
\section{Análise de correlação}\label{analise-correlacao}}

\hypertarget{o-que-uxe9-anuxe1lise-de-correlauxe7uxe3o}{%
\subsection{O que é análise de correlação?}\label{o-que-uxe9-anuxe1lise-de-correlauxe7uxe3o}}

\begin{itemize}
\tightlist
\item
  .{[}REF{]}
\end{itemize}

\hypertarget{qual-uxe9-a-interpretauxe7uxe3o-das-medidas-de-correlauxe7uxe3o}{%
\subsection{Qual é a interpretação das medidas de correlação?}\label{qual-uxe9-a-interpretauxe7uxe3o-das-medidas-de-correlauxe7uxe3o}}

\begin{itemize}
\item
  Os valores de correlação estão no intervalo \([-1; 1]\).\textsuperscript{\protect\hyperlink{ref-barkan2015}{88},\protect\hyperlink{ref-khamis2008}{175},\protect\hyperlink{ref-allison2022}{176}}
\item
  Valores de correlação positivos representam uma relação direta entre as variáveis, tal que valores maiores de uma variável estão associados a valores maiores de outra variável.\textsuperscript{\protect\hyperlink{ref-khamis2008}{175},\protect\hyperlink{ref-allison2022}{176}}
\item
  Valores de correlação negativos representam uma relação indireta (ou inversa) entre as variáveis, tal que valores maiores (menores) de uma variável estão associados a valores maiores (menores) de outra variável.\textsuperscript{\protect\hyperlink{ref-khamis2008}{175},\protect\hyperlink{ref-allison2022}{176}}
\item
  Valores de correlação próximos de \(0\) representam a inexistência de relação entre as variáveis.\textsuperscript{\protect\hyperlink{ref-khamis2008}{175},\protect\hyperlink{ref-allison2022}{176}}
\end{itemize}

\hypertarget{quais-precauuxe7uxf5es-devem-ser-tomadas-na-interpretauxe7uxe3o-de-medidas-de-correlauxe7uxe3o}{%
\subsection{Quais precauções devem ser tomadas na interpretação de medidas de correlação?}\label{quais-precauuxe7uxf5es-devem-ser-tomadas-na-interpretauxe7uxe3o-de-medidas-de-correlauxe7uxe3o}}

\begin{itemize}
\item
  Tamanhos de efeito grande (ou qualquer outro) não representam necessariamente uma relação causa-efeito entre as variáveis.\textsuperscript{\protect\hyperlink{ref-khamis2008}{175}}
\item
  Tamanhos de efeito grande (ou qualquer outro) não representam necessariamente uma relação de concordância ou confiabilidade entre as variáveis.\textsuperscript{\protect\hyperlink{ref-khamis2008}{175}}
\end{itemize}

\hypertarget{quais-testes-podem-ser-usados-para-anuxe1lises-de-correlauxe7uxe3o}{%
\subsection{Quais testes podem ser usados para análises de correlação?}\label{quais-testes-podem-ser-usados-para-anuxe1lises-de-correlauxe7uxe3o}}

\begin{itemize}
\item
  Coeficiente de correlação de Pearson (\(r\)).\textsuperscript{\protect\hyperlink{ref-khamis2008}{175},\protect\hyperlink{ref-allison2022}{176}}

  \begin{itemize}
  \item
    O coeficiente de correlação de Pearson (\(r\)) avalia a força e direção da relação linear entre duas variáveis quantitativas.\textsuperscript{\protect\hyperlink{ref-khamis2008}{175},\protect\hyperlink{ref-allison2022}{176}}
  \item
    Tipo: paramétrico.\textsuperscript{\protect\hyperlink{ref-khamis2008}{175},\protect\hyperlink{ref-allison2022}{176}}
  \item
    Hipóteses:\textsuperscript{\protect\hyperlink{ref-allison2022}{176}}

    \begin{itemize}
    \item
      Nula (\(H_{0}\)): \(r=0\)
    \item
      Alternativa (\(H_{1}\)): \(r≠0\)
    \end{itemize}
  \item
    Tamanho do efeito:\textsuperscript{\protect\hyperlink{ref-khamis2008}{175},\protect\hyperlink{ref-allison2022}{176}}

    \begin{itemize}
    \tightlist
    \item
      Coeficiente de correlação de Pearson (\(r\))
    \end{itemize}
  \end{itemize}
\end{itemize}

\begin{infobox}{images/Rlogo}
O pacote \emph{stats}\textsuperscript{\protect\hyperlink{ref-stats-2}{52}} fornece a função \href{https://www.rdocumentation.org/packages/stats/versions/3.6.2/topics/cor.test}{\emph{cor.test}} para calcular o coeficiente de correlação de Pearson (\(r\)).

\end{infobox}

\begin{itemize}
\item
  Coeficiente de correlação ponto-bisserial (\(r_{s}\)).\textsuperscript{\protect\hyperlink{ref-khamis2008}{175}}

  \begin{itemize}
  \item
    O coeficiente de correlação ponto-bisserial (\(r_{s}\)) avalia a força e direção da relação linear entre uma variável quantitativa e outra dicotômica.\textsuperscript{\protect\hyperlink{ref-khamis2008}{175}}
  \item
    Tipo: paramétrico.\textsuperscript{\protect\hyperlink{ref-khamis2008}{175}}
  \item
    Hipóteses:\textsuperscript{\protect\hyperlink{ref-khamis2008}{175}}

    \begin{itemize}
    \item
      Nula (\(H_{0}\)): \(r_{s}=0\)
    \item
      Alternativa (\(H_{1}\)): \(r_{s}≠0\)
    \end{itemize}
  \item
    Tamanho do efeito:\textsuperscript{\protect\hyperlink{ref-khamis2008}{175}}

    \begin{itemize}
    \tightlist
    \item
      Coeficiente de correlação ponto-bisserial (\(r_{s}\))
    \end{itemize}
  \end{itemize}
\end{itemize}

\begin{infobox}{images/Rlogo}
O pacote \emph{stats}\textsuperscript{\protect\hyperlink{ref-stats-2}{52}} fornece a função \href{https://www.rdocumentation.org/packages/stats/versions/3.6.2/topics/cor.test}{\emph{cor.test}} para calcular o coeficiente de correlação ponto-bisserial (\(r_{s}\)).

\end{infobox}

\begin{itemize}
\item
  Coeficiente de correlação de Spearman (\(\rho\)).\textsuperscript{\protect\hyperlink{ref-khamis2008}{175},\protect\hyperlink{ref-allison2022}{176}}

  \begin{itemize}
  \item
    O coeficiente de correlação de Spearman (\(\rho\)) avalia a força e direção da relação monotônica entre duas variáveis quantitativas.\textsuperscript{\protect\hyperlink{ref-khamis2008}{175},\protect\hyperlink{ref-allison2022}{176}}
  \item
    O coeficiente de correlação de Spearman (\(\rho\)) pode ser também definida como a correlação de Pearson (\(r\)) entre as classificações (\emph{ranks}) das duas variáveis quantitativas.\textsuperscript{\protect\hyperlink{ref-khamis2008}{175},\protect\hyperlink{ref-allison2022}{176}}
  \item
    Tipo: não-paramétrico.\textsuperscript{\protect\hyperlink{ref-khamis2008}{175},\protect\hyperlink{ref-allison2022}{176}}
  \item
    Hipóteses:\textsuperscript{\protect\hyperlink{ref-khamis2008}{175},\protect\hyperlink{ref-allison2022}{176}}

    \begin{itemize}
    \item
      Nula (\(H_{0}\)): \(\rho=0\)
    \item
      Alternativa (\(H_{1}\)): \(\rho≠0\)
    \end{itemize}
  \item
    Tamanho do efeito:\textsuperscript{\protect\hyperlink{ref-khamis2008}{175},\protect\hyperlink{ref-allison2022}{176}}

    \begin{itemize}
    \tightlist
    \item
      Coeficiente de correlação de Spearman (\(\rho\))
    \end{itemize}
  \end{itemize}
\end{itemize}

\begin{infobox}{images/Rlogo}
O pacote \emph{stats}\textsuperscript{\protect\hyperlink{ref-stats-2}{52}} fornece a função \href{https://www.rdocumentation.org/packages/stats/versions/3.6.2/topics/cor.test}{\emph{cor.test}} para calcular o coeficiente de correlação de Spearman (\(\rho\)).

\end{infobox}

\begin{infobox}{images/Rlogo}
O pacote \emph{corrplot}\textsuperscript{\protect\hyperlink{ref-corrplot-2}{177}} fornece a função \href{https://www.rdocumentation.org/packages/corrplot/versions/0.92/topics/cor.mtest}{\emph{cor.mtest}} para calcular os P-valores e intervalos de confiança da matriz de correlação.

\end{infobox}

\begin{infobox}{images/Rlogo}
O pacote \emph{corrplot}\textsuperscript{\protect\hyperlink{ref-corrplot-2}{177}} fornece a função \href{https://www.rdocumentation.org/packages/corrplot/versions/0.92/topics/corrplot}{\emph{corrplot}} para visualização da matriz de correlação.

\end{infobox}

\hypertarget{analise-redes}{%
\chapter{\texorpdfstring{\textbf{Redes}}{Redes}}\label{analise-redes}}

\hypertarget{redes}{%
\section{Análise de redes}\label{redes}}

\hypertarget{o-que-uxe9-anuxe1lise-de-rede}{%
\subsection{O que é análise de rede?}\label{o-que-uxe9-anuxe1lise-de-rede}}

\begin{itemize}
\tightlist
\item
  .{[}REF{]}
\end{itemize}

\begin{infobox}{images/Rlogo}
O pacote \emph{cooccur}\textsuperscript{\protect\hyperlink{ref-cooccur}{178}} fornece a função \href{https://www.rdocumentation.org/packages/cooccur/versions/1.3/topics/cooccur}{\emph{cooccur}} para criar calcular a coocorrência de objetos em um banco de dados.

\end{infobox}

\hypertarget{analise-inferencial-associacao}{%
\chapter{\texorpdfstring{\textbf{Associação}}{Associação}}\label{analise-inferencial-associacao}}

\hypertarget{analise-associacao}{%
\section{Análise de associação}\label{analise-associacao}}

\hypertarget{o-que-uxe9-anuxe1lise-de-associauxe7uxe3o}{%
\subsection{O que é análise de associação?}\label{o-que-uxe9-anuxe1lise-de-associauxe7uxe3o}}

\begin{itemize}
\tightlist
\item
  .{[}REF{]}
\end{itemize}

\hypertarget{bivariada}{%
\section{Associação bivariada}\label{bivariada}}

\hypertarget{o-que-suxe3o-anuxe1lises-de-associauxe7uxe3o-bivariada}{%
\subsection{O que são análises de associação bivariada?}\label{o-que-suxe3o-anuxe1lises-de-associauxe7uxe3o-bivariada}}

\begin{itemize}
\tightlist
\item
  .{[}REF{]}
\end{itemize}

\hypertarget{quais-testes-podem-ser-usados-para-anuxe1lises-de-associauxe7uxe3o-bivariada}{%
\subsection{Quais testes podem ser usados para análises de associação bivariada?}\label{quais-testes-podem-ser-usados-para-anuxe1lises-de-associauxe7uxe3o-bivariada}}

\begin{itemize}
\item
  Teste Qui-quadrado (\(\chi^2\)).\textsuperscript{\protect\hyperlink{ref-McHugh2013}{179},\protect\hyperlink{ref-Kim2017a}{180}}

  \begin{itemize}
  \item
    O teste qui-quadrado (\(\chi^2\)) avalia uma hipótese global se a relação entre duas variáveis e/ou fatores é independente ou associada.\textsuperscript{\protect\hyperlink{ref-Kim2017a}{180}}
  \item
    O teste qui-quadrado é utilizado para comparar a distribuição de uma variável categórica em uma amostra ou grupo com a distribuição em outro. Se a distribuição da variável categórica não for muito diferente nos diferentes grupos, pode-se concluir que a distribuição da variável categórica não está relacionada com a variável dos grupos. Pode-se também concluir que a variável categórica e os grupos são independentes.\textsuperscript{\protect\hyperlink{ref-Kim2017a}{180}}
  \item
    Tipo: não paramétrico.\textsuperscript{\protect\hyperlink{ref-McHugh2013}{179},\protect\hyperlink{ref-Kim2017a}{180}}
  \item
    Suposições:\textsuperscript{\protect\hyperlink{ref-McHugh2013}{179},\protect\hyperlink{ref-Kim2017a}{180}}

    \begin{itemize}
    \item
      As variáveis são ordinais ou categóricas nominais, de modo que as células representem frequência.
    \item
      Os níveis dos fatores (variáveis categóricas) são mutuamente exclusivos.
    \item
      Tamanho de amostra grande e adequado porque é baseado em uma abordagem de aproximação.
    \item
      Menos de 20\% das células com frequências esperadas \textless{} 5
    \item
      Nenhuma célula com frequência esperada \textless{} 1.
    \end{itemize}
  \item
    Hipóteses:\textsuperscript{\protect\hyperlink{ref-Kim2017a}{180}}

    \begin{itemize}
    \item
      Nula (\(H_{0}\)): independente (sem associação)
    \item
      Alternativa (\(H_{1}\)): não independente (associação)
    \end{itemize}
  \item
    Tamanho do efeito:\textsuperscript{\protect\hyperlink{ref-Kim2017a}{180}}

    \begin{itemize}
    \item
      Phi (\(\phi\)), para tabelas de contingência 2x2
    \item
      Razão de chances (\(RC\) ou \(OR\)), para tabelas de contingência 2x2
    \item
      Cramer V (\(V\)), para tabelas de contingência NxM
    \end{itemize}
  \end{itemize}
\end{itemize}

\begin{infobox}{images/Rlogo}
O pacote \emph{gtsummary}\textsuperscript{\protect\hyperlink{ref-gtsummary}{181}} fornece a função \href{https://www.rdocumentation.org/packages/gtsummary/versions/1.6.3/topics/tbl_cross}{\emph{tbl\_cross}} para criar uma tabela NxM.

\end{infobox}

\begin{itemize}
\item
  Teste Exato de Fisher (\(\chi^2\)).\textsuperscript{\protect\hyperlink{ref-McHugh2013}{179},\protect\hyperlink{ref-Kim2017a}{180}}

  \begin{itemize}
  \item
    O teste exato de Fisher avalia a hipótese nula de independência aplicando a distribuição hipergeométrica dos números nas células da tabela.\textsuperscript{\protect\hyperlink{ref-Kim2017a}{180}}
  \item
    Hipóteses:\textsuperscript{\protect\hyperlink{ref-McHugh2013}{179},\protect\hyperlink{ref-Kim2017a}{180}}

    \begin{itemize}
    \item
      Nula (\(H_{0}\)): independente (sem associação)
    \item
      Alternativa (\(H_{1}\)): não independente (associação)
    \end{itemize}
  \item
    Tamanho do efeito:\textsuperscript{\protect\hyperlink{ref-McHugh2013}{179},\protect\hyperlink{ref-Kim2017a}{180}}

    \begin{itemize}
    \item
      Phi (\(\phi\)), para tabelas de contingência 2x2
    \item
      Razão de chances (\(RC\) ou \(OR\)), para tabelas de contingência 2x2
    \item
      Cramer V (\(V\)), para tabelas de contingência NxM
    \end{itemize}
  \end{itemize}
\end{itemize}

\begin{infobox}{images/Rlogo}
O pacote \emph{gtsummary}\textsuperscript{\protect\hyperlink{ref-gtsummary}{181}} fornece a função \href{https://www.rdocumentation.org/packages/gtsummary/versions/1.6.3/topics/tbl_cross}{\emph{tbl\_cross}} para criar uma tabela NxM.

\end{infobox}

\begin{itemize}
\item
  Kendall \(\tau\).\textsuperscript{\protect\hyperlink{ref-khamis2008}{175},\protect\hyperlink{ref-allison2022}{176}}

  \begin{itemize}
  \item
    O coeficiente Kendall \(\tau\) avalia a força e direção da relação monotônica entre duas variáveis quantitativas ou qualitativas.\textsuperscript{\protect\hyperlink{ref-khamis2008}{175},\protect\hyperlink{ref-allison2022}{176}}
  \item
    O coeficiente Kendall \(\tau\) é definido como a proporção de todos os pares concordantes menos a proporção de todos os pares discordantes.\textsuperscript{\protect\hyperlink{ref-khamis2008}{175},\protect\hyperlink{ref-allison2022}{176}}
  \item
    Tipo: não-paramétrico.\textsuperscript{\protect\hyperlink{ref-khamis2008}{175},\protect\hyperlink{ref-allison2022}{176}}
  \item
    Hipóteses:\textsuperscript{\protect\hyperlink{ref-khamis2008}{175},\protect\hyperlink{ref-allison2022}{176}}

    \begin{itemize}
    \item
      Nula (\(H_{0}\)): \(\tau=0\)
    \item
      Alternativa (\(H_{1}\)): \(\tau≠0\)
    \end{itemize}
  \item
    Tamanho do efeito:\textsuperscript{\protect\hyperlink{ref-khamis2008}{175},\protect\hyperlink{ref-allison2022}{176}}

    \begin{itemize}
    \tightlist
    \item
      Kendall \(\tau\)
    \end{itemize}
  \end{itemize}
\end{itemize}

\begin{infobox}{images/Rlogo}
O pacote \emph{stats}\textsuperscript{\protect\hyperlink{ref-stats-2}{52}} fornece a função \href{https://www.rdocumentation.org/packages/stats/versions/3.6.2/topics/cor.test}{\emph{cor.test}} para calcular o coeficiente Kendall \(\tau\).

\end{infobox}

\hypertarget{analise-inferencial-regressao}{%
\chapter{\texorpdfstring{\textbf{Regressão}}{Regressão}}\label{analise-inferencial-regressao}}

\hypertarget{analise-regressao}{%
\section{Análise de regressão}\label{analise-regressao}}

\hypertarget{o-que-uxe9-anuxe1lise-de-regressuxe3o}{%
\subsection{O que é análise de regressão?}\label{o-que-uxe9-anuxe1lise-de-regressuxe3o}}

\begin{itemize}
\item
  Regressão refere-se a uma equação matemática que permite que uma ou mais variável(is) de desfecho (dependentes) seja(m) prevista(s) a partir de uma ou mais variável(is) independente(s). A regressão implica em uma direção de efeito, mas não garante causalidade.\textsuperscript{\protect\hyperlink{ref-greenhalgh1997a}{161}}
\item
  Para estimar os efeitos imparciais de um fator de exposição primária sobre uma variável de desfecho, frequentemente constroem-se modelos estatísticos de regressão.\textsuperscript{\protect\hyperlink{ref-bandoli2018}{134}}
\end{itemize}

\hypertarget{o-que-suxe3o-as-anuxe1lises-de-regressuxe3o-simples-multivariuxe1vel-e-multivariada}{%
\subsection{O que são as análises de regressão simples, multivariável e multivariada?}\label{o-que-suxe3o-as-anuxe1lises-de-regressuxe3o-simples-multivariuxe1vel-e-multivariada}}

\begin{itemize}
\item
  A análise de regressão simples consiste em modelos estatísticos com 1 variável dependente (desfecho) e 1 variável independente (preditor).\textsuperscript{\protect\hyperlink{ref-Hidalgo2013}{182}}
\item
  A análise multivariável (ou múltiplo) consiste em modelos estatísticos com 1 variável dependente (desfecho) e duas ou mais variáveis independentes.\textsuperscript{\protect\hyperlink{ref-Hidalgo2013}{182}}
\item
  A análise multivariada consiste em modelos estatísticos com 2 ou mais variáveis dependente (desfechos) e duas ou mais variáveis independentes.\textsuperscript{\protect\hyperlink{ref-Hidalgo2013}{182}}
\end{itemize}

\begin{infobox}{images/Rlogo}
O pacote \emph{modelsummary}\textsuperscript{\protect\hyperlink{ref-modelsummary}{183}} fornece as funções \href{https://www.rdocumentation.org/packages/modelsummary/versions/1.4.1/topics/modelsummary}{\emph{modelsummary}} e \href{https://www.rdocumentation.org/packages/modelsummary/versions/1.4.1/topics/modelplot}{\emph{modelplot}} para gerar tabelas e gráficos de coeficientes de regressão.

\end{infobox}

\begin{infobox}{images/Rlogo}
O pacote \emph{gtsummary}\textsuperscript{\protect\hyperlink{ref-gtsummary-2}{133}} fornece a função \href{https://www.rdocumentation.org/packages/gtsummary/versions/1.6.3/topics/tbl_regression}{\emph{tbl\_regression}} para construção da `Tabela 2' com dados do modelo de regressão.

\end{infobox}

\hypertarget{selecao}{%
\section{Seleção de variáveis do modelo}\label{selecao}}

\hypertarget{como-preparar-as-variuxe1veis-categuxf3ricas-para-anuxe1lise-de-regressuxe3o}{%
\subsection{Como preparar as variáveis categóricas para análise de regressão?}\label{como-preparar-as-variuxe1veis-categuxf3ricas-para-anuxe1lise-de-regressuxe3o}}

\begin{itemize}
\item
  Variáveis fictícias (\emph{dummy}) compreendem variáveis criadas para introduzir, nos modelos de regressão, informações contidas em outras variáveis que não podem ser medidas em escala numérica.\textsuperscript{\protect\hyperlink{ref-suits1957}{184}}
\item
  Variáveis categóricas nominais, com 2 ou mais níveis, devem ser subdivididas em variáveis fictícias dicotômicas para ser usada em modelos de regressão.\textsuperscript{\protect\hyperlink{ref-Healy1995}{185}}
\item
  Cada nível da variável categórica nominal será convertido em uma nova variável fictícias dicotômica, tal que a nova variável dicotômica assume valor 1 para a presença do nível correspondente e 0 em qualquer outro caso.\textsuperscript{\protect\hyperlink{ref-Healy1995}{185}}
\end{itemize}

\begin{infobox}{images/Rlogo}
O pacote \emph{fastDummies}\textsuperscript{\protect\hyperlink{ref-fastDummies}{186}} fornece a função \href{https://www.rdocumentation.org/packages/fastDummies/versions/1.7.3/topics/dummy_columns}{\emph{dummy\_cols}} para preparar as variáveis categóricas fictícias para análise de regressão.

\end{infobox}

\hypertarget{correlauxe7uxe3o-bivariada-pode-ser-usada-para-seleuxe7uxe3o-de-variuxe1veis-em-modelos-de-regressuxe3o-multivariuxe1vel}{%
\subsection{Correlação bivariada pode ser usada para seleção de variáveis em modelos de regressão multivariável?}\label{correlauxe7uxe3o-bivariada-pode-ser-usada-para-seleuxe7uxe3o-de-variuxe1veis-em-modelos-de-regressuxe3o-multivariuxe1vel}}

\begin{itemize}
\item
  Seleção bivariada de variáveis - isto é, aplicação de testes de correlação em pares de variáveis candidatas e variável de desfecho afim de selecionar quais serão incluídas no modelo multivariável - é um dos erros mais comuns na literatura.\textsuperscript{\protect\hyperlink{ref-heinze2016}{154},\protect\hyperlink{ref-Dales1978}{187},\protect\hyperlink{ref-Sun1996}{188}}
\item
  A seleção bivariada de variáveis torna o modelo mais suscetível a otimismo no ajuste se as variáveis de confundimento não são adequadamente controladas.\textsuperscript{\protect\hyperlink{ref-Dales1978}{187},\protect\hyperlink{ref-Sun1996}{188}}
\end{itemize}

\hypertarget{variuxe1veis-sem-significuxe2ncia-estatuxedstica-devem-ser-excluuxeddas-do-modelo-final}{%
\subsection{Variáveis sem significância estatística devem ser excluídas do modelo final?}\label{variuxe1veis-sem-significuxe2ncia-estatuxedstica-devem-ser-excluuxeddas-do-modelo-final}}

\begin{itemize}
\item
  Eliminar uma variável de um modelo significa anular o seu coeficiente de regressão (\(\beta = 0\)), mesmo que o valor estimado pelos dados seja outro. Desta forma, os resultados se afasTAM de uma solução de máxima verossimilhança (que tem fundamento teórico) e o modelo resultante é intencionalmente subótimo.\textsuperscript{\protect\hyperlink{ref-heinze2016}{154}}
\item
  Os coeficientes de regressão geralmente dependem do conjunto de variáveis do modelo e, portanto, podem mudam de valor (``mudança na estimativa'' positiva ou negativa) se uma (ou mais) variável(is) for(em) eliminada(s) do modelo.\textsuperscript{\protect\hyperlink{ref-heinze2016}{154}}
\end{itemize}

\hypertarget{por-que-muxe9todos-de-regressuxe3o-gradual-nuxe3o-suxe3o-recomendados-para-seleuxe7uxe3o-de-variuxe1veis-em-modelos-de-regressuxe3o-multivariuxe1vel}{%
\subsection{Por que métodos de regressão gradual não são recomendados para seleção de variáveis em modelos de regressão multivariável?}\label{por-que-muxe9todos-de-regressuxe3o-gradual-nuxe3o-suxe3o-recomendados-para-seleuxe7uxe3o-de-variuxe1veis-em-modelos-de-regressuxe3o-multivariuxe1vel}}

\begin{itemize}
\item
  Métodos diferentes de regressão gradual podem produzir diferentes seleções de variáveis de um mesmo banco de dados.\textsuperscript{\protect\hyperlink{ref-Healy1995}{185}}
\item
  Nenhum método de regressão gradual garante a seleção ótima de variáveis de um banco de dados.\textsuperscript{\protect\hyperlink{ref-Healy1995}{185}}
\item
  As regras de término da regressão baseadas em P-valor tendem a ser arbitrárias.\textsuperscript{\protect\hyperlink{ref-Healy1995}{185}}
\end{itemize}

\hypertarget{o-que-pode-ser-feito-para-reduzir-o-nuxfamero-de-variuxe1veis-candidatas-em-modelos-de-regressuxe3o-multivariuxe1vel}{%
\subsection{O que pode ser feito para reduzir o número de variáveis candidatas em modelos de regressão multivariável?}\label{o-que-pode-ser-feito-para-reduzir-o-nuxfamero-de-variuxe1veis-candidatas-em-modelos-de-regressuxe3o-multivariuxe1vel}}

\begin{itemize}
\item
  Verifique a existência de multicolinearidade entre as variáveis candidatas.\textsuperscript{\protect\hyperlink{ref-Sun1996}{188}}
\item
  Em caso de uma proporção baixa entre o número de participantes e de variáveis, use o conhecimento prévio da literatura para selecionar um pequeno conjunto de variáveis candidatas.\textsuperscript{\protect\hyperlink{ref-Sun1996}{188}}
\item
  Colapse categorias com contagem nula (células com valor igual a 0) de variáveis candidatas.\textsuperscript{\protect\hyperlink{ref-Sun1996}{188}}
\item
  Use simulações de dados para identificar qual(is) variável(is) está(ão) causando problemas de convergência do ajuste do modelo.\textsuperscript{\protect\hyperlink{ref-Sun1996}{188}}
\item
  A eliminação retroativa tem sido recomendada como a abordagem de regressão gradual mais confiável entre aquelas que podem ser facilmente alcançadas com programas de computador.\textsuperscript{\protect\hyperlink{ref-heinze2016}{154}}
\end{itemize}

\#\# Efeito principal \{\#efeito-principal\}

\hypertarget{o-que-uxe9-efeito-principal}{%
\subsection{O que é efeito principal?}\label{o-que-uxe9-efeito-principal}}

\begin{itemize}
\tightlist
\item
  .\textsuperscript{\protect\hyperlink{ref-Bours2023}{189}}
\end{itemize}

\hypertarget{modificacao}{%
\section{Efeito de modificação}\label{modificacao}}

\hypertarget{o-que-uxe9-um-modificador-de-efeito}{%
\subsection{O que é um modificador de efeito?}\label{o-que-uxe9-um-modificador-de-efeito}}

\begin{itemize}
\tightlist
\item
  .\textsuperscript{\protect\hyperlink{ref-Bours2023}{189}}
\end{itemize}

\hypertarget{o-que-uxe9-efeito-de-modificauxe7uxe3o}{%
\subsection{O que é efeito de modificação?}\label{o-que-uxe9-efeito-de-modificauxe7uxe3o}}

\begin{itemize}
\tightlist
\item
  .\textsuperscript{\protect\hyperlink{ref-Bours2023}{189}}
\end{itemize}

\hypertarget{interacao}{%
\section{Efeito de interação}\label{interacao}}

\hypertarget{o-que-uxe9-efeito-de-interauxe7uxe3o}{%
\subsection{O que é efeito de interação?}\label{o-que-uxe9-efeito-de-interauxe7uxe3o}}

\begin{itemize}
\item
  A interação - representada pelo símbolo `*' - é o termo estatístico empregado para representar a heterogeneidade de um determinado efeito.\textsuperscript{\protect\hyperlink{ref-Altman1996}{190}}
\item
  .\textsuperscript{\protect\hyperlink{ref-Bours2023}{189}}
\end{itemize}

\begin{infobox}{images/Rlogo}
O pacote \emph{nlme}\textsuperscript{\protect\hyperlink{ref-nlme}{191}} fornece a função \href{https://www.rdocumentation.org/packages/nlme/versions/3.1-163/topics/nlme}{\emph{nlme}} para ajustar um modelo de regressão misto não linear.

\end{infobox}

\begin{infobox}{images/Rlogo}
O pacote \emph{mmrm}\textsuperscript{\protect\hyperlink{ref-mmrm}{192}} fornece a função \href{https://rdrr.io/cran/mmrm/man/mmrm.html}{\emph{mmrm}} para ajuste de um modelo de regressão misto linear.

\end{infobox}

\begin{infobox}{images/Rlogo}
O pacote \emph{emmeans}\textsuperscript{\protect\hyperlink{ref-emmeans}{193}} fornece a função \href{https://www.rdocumentation.org/packages/emmeans/versions/1.8.7/topics/emmeans}{\emph{emmeans}} para calcular as médias marginais dos fatores e suas combinações de um modelo de regressão misto linear.

\end{infobox}

\hypertarget{mediacao}{%
\section{Efeito de mediação}\label{mediacao}}

\hypertarget{o-que-uxe9-um-mediador-de-efeito}{%
\subsection{O que é um mediador de efeito?}\label{o-que-uxe9-um-mediador-de-efeito}}

\begin{itemize}
\item
  .\textsuperscript{\protect\hyperlink{ref-Baron1986}{194}}
\item
  .\textsuperscript{\protect\hyperlink{ref-Bours2023}{189}}
\end{itemize}

\hypertarget{o-que-uxe9-efeito-de-mediauxe7uxe3o}{%
\subsection{O que é efeito de mediação?}\label{o-que-uxe9-efeito-de-mediauxe7uxe3o}}

\begin{itemize}
\item
  .\textsuperscript{\protect\hyperlink{ref-Baron1986}{194}}
\item
  .\textsuperscript{\protect\hyperlink{ref-Bours2023}{189}}
\end{itemize}

\hypertarget{o-que-uxe9-efeito-direto}{%
\subsection{O que é efeito direto?}\label{o-que-uxe9-efeito-direto}}

\begin{itemize}
\item
  .\textsuperscript{\protect\hyperlink{ref-Baron1986}{194}}
\item
  .\textsuperscript{\protect\hyperlink{ref-Bours2023}{189}}
\end{itemize}

\hypertarget{o-que-uxe9-efeito-indireto}{%
\subsection{O que é efeito indireto?}\label{o-que-uxe9-efeito-indireto}}

\begin{itemize}
\item
  .\textsuperscript{\protect\hyperlink{ref-Baron1986}{194}}
\item
  .\textsuperscript{\protect\hyperlink{ref-Bours2023}{189}}
\end{itemize}

\hypertarget{o-que-uxe9-efeito-total}{%
\subsection{O que é efeito total?}\label{o-que-uxe9-efeito-total}}

\begin{itemize}
\item
  .\textsuperscript{\protect\hyperlink{ref-Baron1986}{194}}
\item
  .\textsuperscript{\protect\hyperlink{ref-Bours2023}{189}}
\end{itemize}

\hypertarget{suposicoes-modelos}{%
\section{Efeitos brutos ou padronizados}\label{suposicoes-modelos}}

\hypertarget{o-que-uxe9-efeito-bruto}{%
\subsection{O que é efeito bruto?}\label{o-que-uxe9-efeito-bruto}}

\begin{itemize}
\item
  .\textsuperscript{\protect\hyperlink{ref-greenland1986}{195}}
\item
  .\textsuperscript{\protect\hyperlink{ref-greenland1991}{196}}
\end{itemize}

\hypertarget{o-que-uxe9-efeito-padronizado}{%
\subsection{O que é efeito padronizado?}\label{o-que-uxe9-efeito-padronizado}}

\begin{itemize}
\item
  .\textsuperscript{\protect\hyperlink{ref-greenland1986}{195}}
\item
  .\textsuperscript{\protect\hyperlink{ref-greenland1991}{196}}
\end{itemize}

\hypertarget{suposiuxe7uxf5es-dos-modelos}{%
\section{Suposições dos modelos}\label{suposiuxe7uxf5es-dos-modelos}}

\hypertarget{quais-suposiuxe7uxf5es-suxe3o-feitas-para-modelagem-de-regressuxe3o}{%
\subsection{Quais suposições são feitas para modelagem de regressão?}\label{quais-suposiuxe7uxf5es-suxe3o-feitas-para-modelagem-de-regressuxe3o}}

\begin{itemize}
\tightlist
\item
  .{[}REF{]}
\end{itemize}

\hypertarget{como-avaliar-as-suposiuxe7uxf5es-do-modelo}{%
\subsection{Como avaliar as suposições do modelo?}\label{como-avaliar-as-suposiuxe7uxf5es-do-modelo}}

\begin{itemize}
\tightlist
\item
  .{[}REF{]}
\end{itemize}

\begin{infobox}{images/Rlogo}
O pacote \emph{performance}\textsuperscript{\protect\hyperlink{ref-performance}{197}} fornece a função \href{https://www.rdocumentation.org/packages/performance/versions/0.10.4/topics/check_model}{\emph{check\_model}} para analisar a colinearidade entre variáveis, a normalidade da distribuição das variáveis e a heteroscedasticidade.

\end{infobox}

\hypertarget{qualidade-ajuste-modelo}{%
\section{Qualidade do ajuste do modelo}\label{qualidade-ajuste-modelo}}

\hypertarget{o-que-uxe9-qualidade-de-ajuste}{%
\subsection{O que é qualidade de ajuste?}\label{o-que-uxe9-qualidade-de-ajuste}}

\begin{itemize}
\tightlist
\item
  .{[}REF{]}
\end{itemize}

\hypertarget{como-avaliar-a-qualidade-de-ajuste}{%
\subsection{Como avaliar a qualidade de ajuste?}\label{como-avaliar-a-qualidade-de-ajuste}}

\begin{itemize}
\tightlist
\item
  .{[}REF{]}
\end{itemize}

\begin{infobox}{images/Rlogo}
O pacote \emph{performance}\textsuperscript{\protect\hyperlink{ref-performance}{197}} fornece a função \href{https://www.rdocumentation.org/packages/performance/versions/0.10.4/topics/model_performance}{\emph{model\_performance}} para calcular as métricas de ajuste da regressão adequadas ao modelo pré-especificado.

\end{infobox}

\begin{infobox}{images/Rlogo}
O pacote \emph{performance}\textsuperscript{\protect\hyperlink{ref-performance}{197}} fornece a função \href{https://www.rdocumentation.org/packages/performance/versions/0.10.4/topics/compare_performance}{\emph{compare\_performance}} para comparar o desempenho e a qualidade do ajuste de diversos modelos de regressão pré-especificados.

\end{infobox}

\hypertarget{arvore-decisao}{%
\chapter{\texorpdfstring{\textbf{Árvore de decisão}}{Árvore de decisão}}\label{arvore-decisao}}

\hypertarget{arvore-decisao}{%
\section{Árvore de decisão}\label{arvore-decisao}}

\hypertarget{o-que-uxe9-uxe1rvore-de-decisuxe3o}{%
\subsection{O que é árvore de decisão?}\label{o-que-uxe9-uxe1rvore-de-decisuxe3o}}

\begin{itemize}
\tightlist
\item
  .{[}REF{]}
\end{itemize}

\cftaddtitleline{toc}{chapter}{\rule{\textwidth}{0.4pt}}{}

\hypertarget{parte-4---metodologia-aplicada}{%
\chapter*{\texorpdfstring{\emph{Parte 4 - Metodologia Aplicada}}{Parte 4 - Metodologia Aplicada}}\label{parte-4---metodologia-aplicada}}
\addcontentsline{toc}{chapter}{\emph{Parte 4 - Metodologia Aplicada}}

\markboth{}{}
\par\noindent\rule{\textwidth}{0.05in}

\hypertarget{delineamento-estudos}{%
\chapter{\texorpdfstring{\textbf{Delineamento de estudos}}{Delineamento de estudos}}\label{delineamento-estudos}}

\hypertarget{delineamento-estudos}{%
\section{Delineamento de estudos}\label{delineamento-estudos}}

\hypertarget{como-podem-ser-classificados-os-estudos-cientuxedficos}{%
\subsection{Como podem ser classificados os estudos científicos?}\label{como-podem-ser-classificados-os-estudos-cientuxedficos}}

\begin{itemize}
\item
  Estudos científicos podem ser classificados em \emph{básicos}, \emph{observacionais}, \emph{experimentais}, \emph{acurácia diagnóstica}, \emph{propriedades psicométricas}, \emph{avaliação econômica} e \emph{revisões de literatura}:\textsuperscript{\protect\hyperlink{ref-Grant2009}{198}--\protect\hyperlink{ref-chipman2022}{207}}
\item
  \emph{Estudos básicos}\textsuperscript{\protect\hyperlink{ref-Suxfct2014}{199},\protect\hyperlink{ref-Chidambaram2019}{204}}

  \begin{itemize}
  \item
    Genética
  \item
    Celular
  \item
    Experimentos com animais
  \item
    Desenvolvimento de métodos
  \end{itemize}
\item
  \emph{Estudos de simulação computacional}\textsuperscript{\protect\hyperlink{ref-Erdemir2020}{205},\protect\hyperlink{ref-chipman2022}{207}}
\item
  \emph{Estudos observacionais}\textsuperscript{\protect\hyperlink{ref-Suxfct2014}{199},\protect\hyperlink{ref-Chidambaram2019}{204}}

  \begin{itemize}
  \item
    Descritivo

    \begin{itemize}
    \item
      Estudo de caso
    \item
      Série de casos
    \item
      Transversal
    \end{itemize}
  \item
    Analítico

    \begin{itemize}
    \item
      Transversal
    \item
      Caso-Controle

      \begin{itemize}
      \item
        Caso-Controle aninhado
      \item
        Caso-Coorte
      \end{itemize}
    \end{itemize}
  \item
    Coorte prospectiva ou retrospectiva
  \end{itemize}
\item
  \emph{Estudos de desempenho diagnóstico}\textsuperscript{\protect\hyperlink{ref-Chassuxe92019}{203},\protect\hyperlink{ref-Yang2021}{206}}

  \begin{itemize}
  \item
    Transversal
  \item
    Caso-Controle
  \item
    Comparativo
  \item
    Totalmente pareado
  \item
    Parcialmente pareado com subgrupo aleatório
  \item
    Parcialmente pareado com subgrupo não aleatório
  \item
    Não pareado aleatório
  \item
    Não pareado não aleatório
  \end{itemize}
\item
  \emph{Estudos de propriedades psicométricas}\textsuperscript{\protect\hyperlink{ref-Souza2017}{200},\protect\hyperlink{ref-echevarruxeda-guanilo2019}{202}}

  \begin{itemize}
  \item
    Validade
  \item
    Confiabilidade
  \item
    Concordância
  \end{itemize}
\item
  \emph{Estudos quase-experimentais}\textsuperscript{\protect\hyperlink{ref-reeves2017}{201}}

  \begin{itemize}
  \item
    Quase-aleatorizado controlado
  \item
    Estimação de variável instrumental
  \item
    Descontinuidade de regressão
  \item
    Série temporal interrompida controlada
  \item
    Série temporal interrompida
  \item
    Diferença
  \end{itemize}
\item
  \emph{Estudos experimentais}\textsuperscript{\protect\hyperlink{ref-Suxfct2014}{199},\protect\hyperlink{ref-Chidambaram2019}{204}}

  \begin{itemize}
  \item
    Fases I a IV

    \begin{itemize}
    \item
      Aleatorizado controlado
    \item
      Não-aleatorizado controlado
    \item
      Autocontrolado
    \item
      Cruzado
    \item
      Fatorial
    \end{itemize}
  \item
    Campo
  \item
    Comunitário
  \end{itemize}
\item
  \emph{Estudos de avaliação econômica}\textsuperscript{\protect\hyperlink{ref-Suxfct2014}{199}}

  \begin{itemize}
  \item
    Análise de custo
  \item
    Análise de minimização de custo
  \item
    Análise de custo-utilidade
  \item
    Análise de custo-efetividade
  \item
    Análise de custo-benefício
  \end{itemize}
\item
  \emph{Estudos de revisão}\textsuperscript{\protect\hyperlink{ref-Grant2009}{198}}

  \begin{itemize}
  \item
    Estado-da-arte
  \item
    Narrativa
  \item
    Crítica
  \item
    Mapeamento
  \item
    Escopo
  \item
    Busca e revisão sistemática
  \item
    Sistematizada
  \item
    Sistemática

    \begin{itemize}
    \item
      Meta-análise
    \item
      Bibliométrica.\textsuperscript{\protect\hyperlink{ref-donthu2021}{208},\protect\hyperlink{ref-lim2023}{209}}
    \end{itemize}
  \item
    Sistemática qualitativa
  \item
    Mista
  \item
    Visão geral
  \item
    Rápida
  \item
    Guarda-chuva
  \end{itemize}
\end{itemize}

\hypertarget{pareamento}{%
\section{Pareamento}\label{pareamento}}

\hypertarget{o-que-uxe9-pareamento}{%
\subsection{O que é pareamento?}\label{o-que-uxe9-pareamento}}

\begin{itemize}
\item
  Pareamento significa que para cada participante de um grupo (por exemplo, com alguma condição clínica), existe um (ou mais) participantes (por exemplo, grupo controle) que possui características iguais ou similares relativas a algumas variáveis de interesse.\textsuperscript{\protect\hyperlink{ref-Bland1994}{210}}
\item
  As variáveis escolhidas para pareamento devem ter relação com as variáveis de desfecho, mas não são de interesse elas mesmas.\textsuperscript{\protect\hyperlink{ref-Bland1994}{210}}
\item
  O ajuste por pareamento deve ser incluído nas análises estatísticas mesmo que as variáveis de pareamento não sejam consideradas prognósticas ou confundidores na amostra estudada.\textsuperscript{\protect\hyperlink{ref-Bland1994}{210}}
\item
  A ausência de evidência estatística de diferença entre grupos não é considerada pareamento.\textsuperscript{\protect\hyperlink{ref-Bland1994}{210}}
\end{itemize}

\hypertarget{aleatorizauxe7uxe3o}{%
\section{Aleatorização}\label{aleatorizauxe7uxe3o}}

\begin{itemize}
\tightlist
\item
  .{[}REF{]}
\end{itemize}

\hypertarget{o-que-uxe9-aleatorizauxe7uxe3o}{%
\subsection{O que é aleatorização?}\label{o-que-uxe9-aleatorizauxe7uxe3o}}

\hypertarget{alocacao}{%
\section{Alocação}\label{alocacao}}

\hypertarget{o-que-uxe9-alocauxe7uxe3o}{%
\subsection{O que é alocação?}\label{o-que-uxe9-alocauxe7uxe3o}}

\begin{itemize}
\tightlist
\item
  .{[}REF{]}
\end{itemize}

\hypertarget{cegamento}{%
\section{Cegamento}\label{cegamento}}

\begin{itemize}
\tightlist
\item
  .{[}REF{]}
\end{itemize}

\hypertarget{o-que-uxe9-cegamento}{%
\subsection{O que é cegamento?}\label{o-que-uxe9-cegamento}}

\hypertarget{tamanho-amostral}{%
\chapter{\texorpdfstring{\textbf{Tamanho da amostra}}{Tamanho da amostra}}\label{tamanho-amostral}}

\hypertarget{tamanho-da-amostra}{%
\section{Tamanho da amostra}\label{tamanho-da-amostra}}

\hypertarget{o-que-uxe9-tamanho-da-amostra}{%
\subsection{O que é tamanho da amostra?}\label{o-que-uxe9-tamanho-da-amostra}}

\begin{itemize}
\item
  Tamanho da amostra \(n\) é a quantidade de participantes (ou unidades de análise) necessárias para conduzir um estudo a fim de testar uma hipótese.\textsuperscript{\protect\hyperlink{ref-rodruxedguezdeluxe1guila2014}{211}}
\item
  .\textsuperscript{\protect\hyperlink{ref-Banerjee2010}{26}}
\end{itemize}

\hypertarget{por-que-determinar-o-tamanho-da-amostra-uxe9-importante}{%
\subsection{Por que determinar o tamanho da amostra é importante?}\label{por-que-determinar-o-tamanho-da-amostra-uxe9-importante}}

\begin{itemize}
\tightlist
\item
  Uma amostra muito pequena para o estudo pode resultar em ajuste exagerado, imprecisão e baixo poder do teste.\textsuperscript{\protect\hyperlink{ref-van2022a}{54}}
\end{itemize}

\hypertarget{quais-fatores-devem-ser-considerados-para-determinar-o-tamanho-da-amostra}{%
\subsection{Quais fatores devem ser considerados para determinar o tamanho da amostra?}\label{quais-fatores-devem-ser-considerados-para-determinar-o-tamanho-da-amostra}}

\begin{itemize}
\item
  Tamanho da população (\(N\)): O tamanho da amostra depende parcialmente do tamanho da população de origem. Geralmente assume-se que a população tem tamanho desconhecido ou infinito. Em alguns estudos serão amostradas populações de tamanho finito (inferior a 100.000 indivíduos), geralmente em pesquisas descritivas, em que esse tamanho deve ser incorporado nos cálculos.\textsuperscript{\protect\hyperlink{ref-rodruxedguezdeluxe1guila2014}{211}}
\item
  Delineamento do estudo.\textsuperscript{\protect\hyperlink{ref-rodruxedguezdeluxe1guila2014}{211}}
\item
  Quantidade e características (dependente vs.~independente) dos grupos de participantes do estudo.\textsuperscript{\protect\hyperlink{ref-rodruxedguezdeluxe1guila2014}{211}}
\item
  Erros tipo I (\(\alpha\)) e tipo II (\(\beta\)).\textsuperscript{\protect\hyperlink{ref-rodruxedguezdeluxe1guila2014}{211}}
\item
  Tipo de variável a ser observada (contínua, intervalo, ordinal, nominal, dicotômica).\textsuperscript{\protect\hyperlink{ref-rodruxedguezdeluxe1guila2014}{211}}
\item
  Tamanho de efeito mínimo a ser observado.\textsuperscript{\protect\hyperlink{ref-rodruxedguezdeluxe1guila2014}{211}}
\item
  Variabilidade da(s) variável(eis) coletada(s).\textsuperscript{\protect\hyperlink{ref-rodruxedguezdeluxe1guila2014}{211}}
\item
  Lateralidade do teste de hipótese (uni- ou bicaudais).\textsuperscript{\protect\hyperlink{ref-rodruxedguezdeluxe1guila2014}{211}}
\item
  Perdas de dados durante a coleta e/ou acompanhamento dos participantes do estudo.\textsuperscript{\protect\hyperlink{ref-rodruxedguezdeluxe1guila2014}{211}}
\end{itemize}

\begin{infobox}{images/Rlogo}
O pacote \emph{pwr}\textsuperscript{\protect\hyperlink{ref-pwr}{158}} fornece a função \href{https://www.rdocumentation.org/packages/pwr/versions/1.3-0/topics/plot.power.htest}{\emph{plot.power.htest}} para apresentar graficamente a relação entre o tamanho da amostra e o poder de testes de hipóteses.

\end{infobox}

\hypertarget{quais-aspectos-uxe9ticos-estuxe3o-envolvidos-no-tamanho-da-amostra}{%
\subsection{Quais aspectos éticos estão envolvidos no tamanho da amostra?}\label{quais-aspectos-uxe9ticos-estuxe3o-envolvidos-no-tamanho-da-amostra}}

\begin{itemize}
\item
  Determinar a priori o tamanho da amostra pode diminuir o risco de realizar testes ou intervenções desnecessários, de desperdício de recursos (tempo e dinheiro) associados e, por outro lado, de coletar dados insuficientes para testar as hipóteses do estudo.\textsuperscript{\protect\hyperlink{ref-rodruxedguezdeluxe1guila2014}{211}}
\item
  O tratamento ético dos participantes do estudo, portanto, não exige que se considere se o poder do estudo é inferior à meta convencional de 80\% ou 90\%.\textsuperscript{\protect\hyperlink{ref-Bacchetti2005}{212}}
\item
  Estudos com poder \textless80\% não são necessariamente antiéticos.\textsuperscript{\protect\hyperlink{ref-Bacchetti2005}{212}}
\item
  Grandes estudos podem ser desejáveis por outras razões que não as éticas.\textsuperscript{\protect\hyperlink{ref-Bacchetti2005}{212}}
\end{itemize}

\hypertarget{como-uxe9-calculado-o-tamanho-da-amostra-de-um-estudo}{%
\subsection{Como é calculado o tamanho da amostra de um estudo?}\label{como-uxe9-calculado-o-tamanho-da-amostra-de-um-estudo}}

\begin{itemize}
\item
  O tamanho amostral pode ser calculado por meio de fórmulas matemáticas que tendem a assegurar margens de erros tipos I (\(\alpha\)) e II (\(\beta\)) para a estimação dos parâmetros populacionais a partir dos dados amostrais.\textsuperscript{\protect\hyperlink{ref-rodruxedguezdeluxe1guila2014}{211}}
\item
  Estudo-piloto --- realizados nas mesmas condições do estudo, mas envolvendo um tamanho de amostra limitado --- pode ser útil na estimativa do tamanho da amostra a partir do tamanho do efeito estimado.\textsuperscript{\protect\hyperlink{ref-rodruxedguezdeluxe1guila2014}{211}}
\item
  O tamanho da amostra deve ser calculado para cada um dos objetivos primários e/ou secundários, sendo escolhido o maior tamanho de amostra calculado para o estudo.\textsuperscript{\protect\hyperlink{ref-rodruxedguezdeluxe1guila2014}{211}}
\item
  Geralmente é recomendado ser cético em relação às regras práticas para o tamanho da amostra, tais como a proporção entre o número de variáveis (ou eventos) e de participantes.\textsuperscript{\protect\hyperlink{ref-van2022a}{54}}
\end{itemize}

\begin{infobox}{images/Rlogo}
O pacote \emph{pwr}\textsuperscript{\protect\hyperlink{ref-pwr}{158}} fornece a função \href{https://www.rdocumentation.org/packages/pwr/versions/1.3-0/topics/cohen.ES}{\emph{cohen.ES}} para cálculo do tamanho da amostra baseado em diferentes testes de hipóteses.

\end{infobox}

\hypertarget{cuxe1lculo-do-tamanho-da-amostra}{%
\section{Cálculo do tamanho da amostra}\label{cuxe1lculo-do-tamanho-da-amostra}}

\hypertarget{como-calcular-o-tamanho-da-amostra}{%
\subsection{Como calcular o tamanho da amostra?}\label{como-calcular-o-tamanho-da-amostra}}

\begin{itemize}
\tightlist
\item
\end{itemize}

\begin{infobox}{images/Rlogo}
O pacote \emph{pwr}\textsuperscript{\protect\hyperlink{ref-pwr}{158}} fornece a função \href{https://www.rdocumentation.org/packages/pwr/versions/1.3-0/topics/pwr.2p.test}{\emph{pwr.2p.test}} para cálculo do tamanho da amostra para testes de proporção balanceados (2 amostras com mesmo número de participantes).

\end{infobox}

\begin{infobox}{images/Rlogo}
O pacote \emph{pwr}\textsuperscript{\protect\hyperlink{ref-pwr}{158}} fornece a função \href{https://www.rdocumentation.org/packages/pwr/versions/1.3-0/topics/pwr.2p.test}{\emph{pwr.2p2n.test}} para cálculo do tamanho da amostra para testes de proporção não balanceados (2 amostras com diferente número de participantes).

\end{infobox}

\begin{infobox}{images/Rlogo}
O pacote \emph{pwr}\textsuperscript{\protect\hyperlink{ref-pwr}{158}} fornece a função \href{https://www.rdocumentation.org/packages/pwr/versions/1.3-0/topics/pwr.anova.test}{\emph{pwr.anova.test}} para cálculo do tamanho da amostra para testes de análise de variância balanceados (3 ou mais amostras com mesmo número de participantes).

\end{infobox}

\begin{infobox}{images/Rlogo}
O pacote \emph{pwr}\textsuperscript{\protect\hyperlink{ref-pwr}{158}} fornece a função \href{https://www.rdocumentation.org/packages/pwr/versions/1.3-0/topics/pwr.chisq.test}{\emph{pwr.chisq.test}} para cálculo do tamanho da amostra para testes de qui-quadrado \(\chi^2\).

\end{infobox}

\begin{infobox}{images/Rlogo}
O pacote \emph{pwr}\textsuperscript{\protect\hyperlink{ref-pwr}{158}} fornece a função \href{https://www.rdocumentation.org/packages/pwr/versions/1.3-0/topics/pwr.f2.test}{\emph{pwr.f2.test}} para cálculo do tamanho da amostra para testes com modelo linear geral.

\end{infobox}

\begin{infobox}{images/Rlogo}
O pacote \emph{pwr}\textsuperscript{\protect\hyperlink{ref-pwr}{158}} fornece a função \href{https://www.rdocumentation.org/packages/pwr/versions/1.3-0/topics/pwr.norm.test}{\emph{pwr.norm.test}} para cálculo do tamanho da amostra para a média de uma distribuição normal com variância conhecida.

\end{infobox}

\begin{infobox}{images/Rlogo}
O pacote \emph{pwr}\textsuperscript{\protect\hyperlink{ref-pwr}{158}} fornece a função \href{https://www.rdocumentation.org/packages/pwr/versions/1.3-0/topics/pwr.p.test}{\emph{pwr.p.test}} para cálculo do tamanho da amostra para testes de proporção (1 amostra).

\end{infobox}

\begin{infobox}{images/Rlogo}
O pacote \emph{pwr}\textsuperscript{\protect\hyperlink{ref-pwr}{158}} fornece a função \href{https://www.rdocumentation.org/packages/pwr/versions/1.3-0/topics/pwr.r.test}{\emph{pwr.r.test}} para cálculo do tamanho da amostra para testes de correlação (1 amostra).

\end{infobox}

\begin{infobox}{images/Rlogo}
O pacote \emph{pwr}\textsuperscript{\protect\hyperlink{ref-pwr}{158}} fornece a função \href{https://www.rdocumentation.org/packages/pwr/versions/1.3-0/topics/pwr.t.test}{\emph{pwr.t.test}} para cálculo do tamanho da amostra para testes \emph{t} de diferença de 1 amostra, 2 amostras dependentes ou 2 amostras independentes (grupos balanceados).

\end{infobox}

\begin{infobox}{images/Rlogo}
O pacote \emph{pwr}\textsuperscript{\protect\hyperlink{ref-pwr}{158}} fornece a função \href{https://www.rdocumentation.org/packages/pwr/versions/1.3-0/topics/pwr.t2n.test}{\emph{pwr.t2n.test}} para cálculo do tamanho da amostra para testes \emph{t} de diferença de 2 amostras independentes (grupos não balanceados).

\end{infobox}

\begin{infobox}{images/Rlogo}
O pacote \emph{longpower}\textsuperscript{\protect\hyperlink{ref-longpower}{160}} fornece a função \href{https://www.rdocumentation.org/packages/longpower/versions/1.0.24/topics/power.mmrm}{\emph{power.mmrm}} para calcular o tamanho da amostra para estudos com análises por modelo de regressão linear misto.

\end{infobox}

\hypertarget{perdas-de-amostra}{%
\section{Perdas de amostra}\label{perdas-de-amostra}}

\hypertarget{o-que-uxe9-perda-de-amostra}{%
\subsection{O que é perda de amostra?}\label{o-que-uxe9-perda-de-amostra}}

\begin{itemize}
\item
  Perda de amostra(s) --- isto é, participante(s) ou unidade(s) de análise --- pode ocorrer durante a coleta e/ou acompanhamento dos participantes do estudo.\textsuperscript{\protect\hyperlink{ref-rodruxedguezdeluxe1guila2014}{211}}
\item
  Perda amostral pode ocorrer por: abandono ou desistência do participante, perda de contato com o participante, perda de informação, ocorrência de eventos adversos, morte do participante, entre outros.\textsuperscript{\protect\hyperlink{ref-rodruxedguezdeluxe1guila2014}{211}}
\end{itemize}

\hypertarget{por-que-a-perda-de-amostra-uxe9-um-problema}{%
\subsection{Por que a perda de amostra é um problema?}\label{por-que-a-perda-de-amostra-uxe9-um-problema}}

\begin{itemize}
\item
  A perda de amostra pode levar a uma redução do poder do estudo, aumentando a probabilidade de erro tipo II (\(\beta\)).{[}REF{]}
\item
  A perda de amostra pode levar a um viés de seleção, pois os participantes que permanecem no estudo podem ser diferentes daqueles que foram perdidos.{[}REF{]}
\end{itemize}

\hypertarget{como-evitar-perda-de-amostra}{%
\subsection{Como evitar perda de amostra?}\label{como-evitar-perda-de-amostra}}

\begin{itemize}
\item
  A perda de amostra pode ser evitada por meio de um planejamento cuidadoso do estudo, incluindo a definição de critérios de inclusão e exclusão claros e apropriados, bem como a definição de estratégias para minimizar a perda de amostra.{[}REF{]}
\item
  A perda de amostra pode ser compensada pelo aumento do tamanho da amostra, desde que o aumento seja suficiente para manter o poder do estudo.\textsuperscript{\protect\hyperlink{ref-rodruxedguezdeluxe1guila2014}{211}}
\end{itemize}

\hypertarget{ajustes-no-tamanho-da-amostra}{%
\section{Ajustes no tamanho da amostra}\label{ajustes-no-tamanho-da-amostra}}

\hypertarget{por-que-ajustar-o-tamanho-da-amostra}{%
\subsection{Por que ajustar o tamanho da amostra?}\label{por-que-ajustar-o-tamanho-da-amostra}}

\begin{itemize}
\tightlist
\item
  O tamanho da amostra pode ser ajustado durante o estudo para compensar a perda de amostra, desde que o aumento seja suficiente para manter o poder do estudo.\textsuperscript{\protect\hyperlink{ref-rodruxedguezdeluxe1guila2014}{211}}
\end{itemize}

\hypertarget{como-ajustar-para-perda-amostral}{%
\subsection{Como ajustar para perda amostral?}\label{como-ajustar-para-perda-amostral}}

\begin{itemize}
\tightlist
\item
  Aumentar o tamanho da amostra estimada \(n\) pela porcentagem \(d\) de perdas esperada ou prevista, para obter o tamanho da amostra efetiva \(n'\) com base na equação \eqref{eq:samplesizeadj1}:\textsuperscript{\protect\hyperlink{ref-rodruxedguezdeluxe1guila2014}{211}}
\end{itemize}

\begin{equation}
\label{eq:samplesizeadj1}
n' = \dfrac{n}{1-d}
\end{equation}

\hypertarget{como-ajustar-para-confiabilidade}{%
\subsection{Como ajustar para confiabilidade?}\label{como-ajustar-para-confiabilidade}}

\begin{itemize}
\tightlist
\item
  .{[}REF{]}
\end{itemize}

\hypertarget{justificativa-do-tamanho-da-amostra}{%
\section{Justificativa do tamanho da amostra}\label{justificativa-do-tamanho-da-amostra}}

\hypertarget{como-justificar-o-tamanho-da-amostra-de-um-estudo}{%
\subsection{Como justificar o tamanho da amostra de um estudo?}\label{como-justificar-o-tamanho-da-amostra-de-um-estudo}}

\begin{itemize}
\item
  Em estudos que envolvem condições raras, pode ser difícil recrutar o número necessário de participantes devido à limitada disponibilidade de casos da população. Mesmo assim, é aconselhável determinar o tamanho da amostra.\textsuperscript{\protect\hyperlink{ref-rodruxedguezdeluxe1guila2014}{211}}
\item
  Quando um estudo deste tipo não é possível, as considerações referentes ao tamanho da amostra são justificadas de acordo com o número máximo de pacientes que podem ser recrutados no decorrer do estudo.\textsuperscript{\protect\hyperlink{ref-rodruxedguezdeluxe1guila2014}{211}}
\end{itemize}

\hypertarget{simulacao-computacional}{%
\chapter{\texorpdfstring{\textbf{Simulação computacional}}{Simulação computacional}}\label{simulacao-computacional}}

\hypertarget{simulacao-computacional-dados}{%
\section{Simulação computacional de dados}\label{simulacao-computacional-dados}}

\hypertarget{o-que-uxe9-simulauxe7uxe3o-computacional-de-dados}{%
\subsection{O que é simulação computacional de dados?}\label{o-que-uxe9-simulauxe7uxe3o-computacional-de-dados}}

\begin{itemize}
\tightlist
\item
  .{[}REF{]}
\end{itemize}

\hypertarget{monte-carlo}{%
\section{Método de Monte Carlo}\label{monte-carlo}}

\hypertarget{o-que-uxe9-o-muxe9todo-de-monte-carlo}{%
\subsection{O que é o método de Monte Carlo?}\label{o-que-uxe9-o-muxe9todo-de-monte-carlo}}

\begin{itemize}
\tightlist
\item
  .{[}REF{]}
\end{itemize}

\begin{infobox}{images/Rlogo}
O pacote \emph{base}\textsuperscript{\protect\hyperlink{ref-base-5}{32}} fornece a função \href{https://www.rdocumentation.org/packages/base/versions/3.6.2/topics/Random}{\emph{set.seed}} para especificar uma semente para reprodutibilidade de computações que envolvem números aleatórios.

\end{infobox}

\begin{infobox}{images/Rlogo}
O pacote \emph{simstudy}\textsuperscript{\protect\hyperlink{ref-simstudy}{213}} fornece as funções \href{https://www.rdocumentation.org/packages/simstudy/versions/0.7.0/topics/defData}{\emph{defData}} e \href{https://www.rdocumentation.org/packages/simstudy/versions/0.7.0/topics/genData}{\emph{genData}} para criar variáveis e simular um banco de dados de acordo com o delineamento pré-especificado, respectivamente.

\end{infobox}

\hypertarget{estudos-observacionais}{%
\chapter{\texorpdfstring{\textbf{Estudos observacionais}}{Estudos observacionais}}\label{estudos-observacionais}}

\hypertarget{observacionais}{%
\section{Estudos observacionais}\label{observacionais}}

\hypertarget{ensaio-cluxednico-aleatorizado}{%
\chapter{\texorpdfstring{\textbf{Ensaio clínico aleatorizado}}{Ensaio clínico aleatorizado}}\label{ensaio-cluxednico-aleatorizado}}

\hypertarget{caracteristicas}{%
\section{Características}\label{caracteristicas}}

\hypertarget{quais-suxe3o-as-caracteruxedsticas-dos-ensaios-cluxednicos-aleatorizados}{%
\subsection{Quais são as características dos ensaios clínicos aleatorizados?}\label{quais-suxe3o-as-caracteruxedsticas-dos-ensaios-cluxednicos-aleatorizados}}

\begin{itemize}
\item
  A característica essencial de um ensaio clínico aleatorizado é a comparação entre grupos.\textsuperscript{\protect\hyperlink{ref-bland2011}{214}}
\item
  Quanto à unidade de alocação:\textsuperscript{\protect\hyperlink{ref-Bruce2022}{215}}

  \begin{itemize}
  \item
    Individual
  \item
    Agrupado
  \end{itemize}
\item
  Quanto ao número de braços:\textsuperscript{\protect\hyperlink{ref-Bruce2022}{215}}

  \begin{itemize}
  \item
    Único*
  \item
    Múltiplos
  \end{itemize}
\item
  Quanto ao número de centros:\textsuperscript{\protect\hyperlink{ref-Bruce2022}{215}}

  \begin{itemize}
  \item
    Único
  \item
    Múltiplos
  \end{itemize}
\item
  Quanto ao cegamento:\textsuperscript{\protect\hyperlink{ref-Bruce2022}{215}}

  \begin{itemize}
  \item
    Aberto*
  \item
    Simples-cego
  \item
    Duplo-cego
  \item
    Tripo-cego
  \item
    Quádruplo-cego
  \end{itemize}
\item
  Quanto à alocação:\textsuperscript{\protect\hyperlink{ref-Bruce2022}{215}}

  \begin{itemize}
  \item
    Sem sorteio
  \item
    Estratificada (centro apenas)
  \item
    Estratificada
  \item
    Minimizada
  \item
    Estratificada e minimizada
  \end{itemize}
\end{itemize}

\hypertarget{metodos-comparacao}{%
\section{Modelos de análise de comparação}\label{metodos-comparacao}}

\hypertarget{que-modelos-podem-ser-utilizados-para-comparauxe7uxf5es}{%
\subsection{Que modelos podem ser utilizados para comparações?}\label{que-modelos-podem-ser-utilizados-para-comparauxe7uxf5es}}

\begin{itemize}
\item
  As abordagens compreendem a comparação da variável de desfecho medida entre os momentos antes e depois ou da sua mudança (pré - pós) entre os momentos.\textsuperscript{\protect\hyperlink{ref-Vickers2001}{216}}
\item
  Se a média da variável é igual entre grupos no início do acompanhamento, ambas abordagens estimam o mesmo efeito. Caso contrário, o efeito será influenciado pela correlação entre as medidas antes e depois. A análise da mudança não controla para desbalanços no início do estudo.\textsuperscript{\protect\hyperlink{ref-Vickers2001}{216}}
\item
  Uma abordagem recomendada é a análise de covariância (ANCOVA), pois ajusta os valores pós-intervenção aos valores pré-intervenção para cada participante, e não é afetada pelas diferenças entre grupos no início do estudo.\textsuperscript{\protect\hyperlink{ref-Vickers2001}{216}}
\item
  A análise de covariância (ANCOVA) modelando seja a mudança (pré - pós) quando o desfecho no pós-tratamento parece ser o método mais efetivo considerando-se o viés de estimação dos parâmetros, a precisão das estimativas, a cobertura nominal (isto é, intervalo de confiança) e o poder do teste.\textsuperscript{\protect\hyperlink{ref-OConnell2017}{217}}
\item
  Análise de variância (ANOVA) e modelos lineares mistos (MLM) são outras opções de métodos, embora apresentem maior variância, menor poder, e cobertura nominal comparados à ANCOVA.\textsuperscript{\protect\hyperlink{ref-OConnell2017}{217}}
\item
  .\textsuperscript{\protect\hyperlink{ref-Cnaan1997}{218}}
\item
  .\textsuperscript{\protect\hyperlink{ref-mallinckrodt2008}{219}}
\end{itemize}

\hypertarget{imputacao-dados}{%
\section{Imputação de dados perdidos}\label{imputacao-dados}}

\hypertarget{como-lidar-com-os-dados-perdidos-em-desfechos}{%
\subsection{Como lidar com os dados perdidos em desfechos?}\label{como-lidar-com-os-dados-perdidos-em-desfechos}}

\begin{itemize}
\item
  Em dados longitudinais com um pequeno número de `ondas' (medidas repetidas) e poucas variáveis, para análise com modelos de regressão univariados, a imputação paramétrica via especificação condicional completa - também conhecido como imputação multivariada por equações encadeadas (\emph{multivariate imputation by chained equations}, MICE) - é eficiente do ponto de vista computacional e possui acurácia e precisão para estimação de parâmetros.\textsuperscript{\protect\hyperlink{ref-Heymans2022}{58},\protect\hyperlink{ref-Cao2022}{220}}
\item
  Para dados perdidos em desfechos dicotômicos, o desempenho dos métodos de imputação multivariada por equações encadeadas (\emph{multivariate imputation by chained equations}, MICE)\textsuperscript{\protect\hyperlink{ref-mice}{64}} e por correspondência média preditiva (\emph{predictive mean matching}, PMM)\textsuperscript{\protect\hyperlink{ref-rubin1986}{65},\protect\hyperlink{ref-little1988a}{66}} é similar.\textsuperscript{\protect\hyperlink{ref-austin2023}{63}}
\end{itemize}

\hypertarget{como-lidar-com-os-dados-perdidos-em-covariuxe1veis}{%
\subsection{Como lidar com os dados perdidos em covariáveis?}\label{como-lidar-com-os-dados-perdidos-em-covariuxe1veis}}

\begin{itemize}
\tightlist
\item
  Imputação de dados perdidos de uma covariável pela média dos dados do respectivo grupo permite estimativas não enviesadas dos efeitos do tratamento, preserva o erro tipo I e aumenta o poder estatístico comparado à análise de dados completos.\textsuperscript{\protect\hyperlink{ref-Kahan2014}{221}}
\end{itemize}

\begin{infobox}{images/Rlogo}
Os pacotes \emph{mice}\textsuperscript{\protect\hyperlink{ref-mice}{64}} e \emph{miceadds}\textsuperscript{\protect\hyperlink{ref-miceadds}{67}} fornecem funções \href{https://www.rdocumentation.org/packages/mice/versions/3.16.0/topics/mice}{\emph{mice}} e \href{https://www.rdocumentation.org/packages/miceadds/versions/3.16-18/topics/mi.anova}{\emph{mi.anova}} para imputação multivariada por equações encadeadas, respectivamente, para imputação de dados.

\end{infobox}

\hypertarget{ajuste-de-covariaveis}{%
\section{Ajuste de covariáveis}\label{ajuste-de-covariaveis}}

\hypertarget{quais-variuxe1veis-devem-ser-utilizadas-no-ajuste-de-covariuxe1veis}{%
\subsection{Quais variáveis devem ser utilizadas no ajuste de covariáveis?}\label{quais-variuxe1veis-devem-ser-utilizadas-no-ajuste-de-covariuxe1veis}}

\begin{itemize}
\tightlist
\item
  A escolha das características de linha de base pelas quais uma análise é ajustada deve ser determinada pelo conhecimento prévio de uma possível influência no resultado, em vez da evidência de desequilíbrio entre os grupos de tratamento no estudo.\textsuperscript{\protect\hyperlink{ref-roberts1999}{222}}
\end{itemize}

\hypertarget{quais-os-benefuxedcios-do-ajuste-de-covariuxe1veis}{%
\subsection{Quais os benefícios do ajuste de covariáveis?}\label{quais-os-benefuxedcios-do-ajuste-de-covariuxe1veis}}

\begin{itemize}
\item
  Ajustar por covariáveis ajuda a estimar os efeitos do tratamento para o indivíduo, assim como aumenta a eficiência dos testes para hipótese nula e a validade externa do estudo.\textsuperscript{\protect\hyperlink{ref-Hauck1998}{223}}
\item
  Incluir a variável de desfecho medida na linha de base como covariável - independentemente de a análise ser realizada com a medida pós-tratamento da mesma variável ou a diferença para a linha de base - pode aumentar o poder estatístico do estudo.\textsuperscript{\protect\hyperlink{ref-Kahan2014}{221}}
\item
  Incluir outras variáveis medidas na linha de base, com potencial para serem desbalanceadas entre grupos após a aleatorização, diminui a chance de afetar as estimativas de efeito dos tratamentos.\textsuperscript{\protect\hyperlink{ref-Kahan2014}{221}}
\end{itemize}

\hypertarget{quais-os-riscos-do-ajuste-de-covariuxe1veis}{%
\subsection{Quais os riscos do ajuste de covariáveis?}\label{quais-os-riscos-do-ajuste-de-covariuxe1veis}}

\begin{itemize}
\item
  Incluir covariáveis que não são prognósticas do desfecho pode reduzir o poder estatístico do estudo.\textsuperscript{\protect\hyperlink{ref-Kahan2014}{221}}
\item
  Incluir covariáveis com dados perdidos pode reduzir o tamanho amostral e consequentemente o poder estatístico do estudo (análise de casos completos) ou levar a desvios do plano de análise por exclusão de covariáveis prognósticas.\textsuperscript{\protect\hyperlink{ref-Kahan2014}{221}}
\end{itemize}

\hypertarget{comparacao-linha-de-base}{%
\section{Comparação na linha de base}\label{comparacao-linha-de-base}}

\hypertarget{o-que-uxe9-comparauxe7uxe3o-entre-grupos-na-linha-de-base-em-ensaios-cluxednicos-aleatorizados}{%
\subsection{O que é comparação entre grupos na linha de base em ensaios clínicos aleatorizados?}\label{o-que-uxe9-comparauxe7uxe3o-entre-grupos-na-linha-de-base-em-ensaios-cluxednicos-aleatorizados}}

\begin{itemize}
\item
  A comparação se refere ao teste de hipótese nula de não haver diferença (`balanço' ou `equilíbrio') entre grupos de tratamento nas (co)variáveis na linha de base, geralmente apresentadas apenas de modo descritivo na `Tabela 1'.\textsuperscript{\protect\hyperlink{ref-Stang2018}{224}}
\item
  A interpretação isolada do P-valor da comparação entre grupos na linha de base não permite identificar as razões para eventuais diferenças.\textsuperscript{\protect\hyperlink{ref-Stang2018}{224}}
\end{itemize}

\hypertarget{para-quuxea-comparar-grupos-na-linha-de-base-em-ensaios-cluxednicos-aleatorizados}{%
\subsection{Para quê comparar grupos na linha de base em ensaios clínicos aleatorizados?}\label{para-quuxea-comparar-grupos-na-linha-de-base-em-ensaios-cluxednicos-aleatorizados}}

\begin{itemize}
\item
  Os P-valores estão relacionados à aleatorização dos participantes em grupos.\textsuperscript{\protect\hyperlink{ref-Bolzern2019}{225}}
\item
  Em ensaios clínicos aleatorizados, a comparação de (co)variáveis na linha de base é usada para avaliar se aleatorização foi `bem sucedida'.\textsuperscript{\protect\hyperlink{ref-Bolzern2019}{225}}
\end{itemize}

\hypertarget{quais-suxe3o-as-razuxf5es-para-diferenuxe7as-entre-grupos-de-tratamento-nas-covariuxe1veis-na-linha-de-base}{%
\subsection{Quais são as razões para diferenças entre grupos de tratamento nas (co)variáveis na linha de base?}\label{quais-suxe3o-as-razuxf5es-para-diferenuxe7as-entre-grupos-de-tratamento-nas-covariuxe1veis-na-linha-de-base}}

\begin{itemize}
\item
  Acaso.\textsuperscript{\protect\hyperlink{ref-chen2020}{129},\protect\hyperlink{ref-Stang2018}{224}}
\item
  Viés.\textsuperscript{\protect\hyperlink{ref-chen2020}{129},\protect\hyperlink{ref-Stang2018}{224}}
\item
  Tamanho da amostra.\textsuperscript{\protect\hyperlink{ref-chen2020}{129},\protect\hyperlink{ref-Stang2018}{224}}
\item
  Má conduta científica.\textsuperscript{\protect\hyperlink{ref-chen2020}{129}}
\end{itemize}

\hypertarget{quais-cenuxe1rios-permitem-a-comparauxe7uxe3o-entre-grupos-na-linha-de-base-em-ensaios-cluxednicos-aleatorizados}{%
\subsection{Quais cenários permitem a comparação entre grupos na linha de base em ensaios clínicos aleatorizados?}\label{quais-cenuxe1rios-permitem-a-comparauxe7uxe3o-entre-grupos-na-linha-de-base-em-ensaios-cluxednicos-aleatorizados}}

\begin{itemize}
\item
  Em ensaios clínicos aleatorizados agregados, os P-valores possuem interpretação diferente de estudos aleatorizados individualmente.\textsuperscript{\protect\hyperlink{ref-Bolzern2019}{225}}
\item
  Em ensaios clínicos com agrupamento, nos quais o recrutamento ocorreu após a aleatorização, os P-valores já não estão inteiramente relacionados ao processo de aleatorização, mas sim ao método de recrutamento, o que pode resultar na comparação de amostras não aleatórias.\textsuperscript{\protect\hyperlink{ref-Bolzern2019}{225}}
\end{itemize}

\hypertarget{por-que-nuxe3o-se-deve-comparar-grupos-na-linha-de-base-em-ensaios-cluxednicos-aleatorizados}{%
\subsection{Por que não se deve comparar grupos na linha de base em ensaios clínicos aleatorizados?}\label{por-que-nuxe3o-se-deve-comparar-grupos-na-linha-de-base-em-ensaios-cluxednicos-aleatorizados}}

\begin{itemize}
\item
  Quando a aleatorização é bem-sucedida, a hipótese nula de diferença entre grupos na linha de base é verdadeira.\textsuperscript{\protect\hyperlink{ref-roberts1999}{222}}
\item
  Testes de significância estatística na linha de base avaliam a probabilidade de que as diferenças observadas possam ter ocorrido por acaso; no entanto, já sabemos - pelo delineamento do experimento - que quaisquer diferenças são causadas pelo acaso.\textsuperscript{\protect\hyperlink{ref-gruijters2020}{226}}
\end{itemize}

\hypertarget{quais-estratuxe9gias-podem-ser-adotadas-para-substituir-a-comparauxe7uxe3o-entre-grupos-na-linha-de-base-em-ensaios-cluxednicos-aleatorizados}{%
\subsection{Quais estratégias podem ser adotadas para substituir a comparação entre grupos na linha de base em ensaios clínicos aleatorizados?}\label{quais-estratuxe9gias-podem-ser-adotadas-para-substituir-a-comparauxe7uxe3o-entre-grupos-na-linha-de-base-em-ensaios-cluxednicos-aleatorizados}}

\begin{itemize}
\item
  Na fase de projeto: identifique as variáveis prognósticas do desfecho de acordo com a literatura.\textsuperscript{\protect\hyperlink{ref-roberts1999}{222}}
\item
  Na fase de análise: inclua as variáveis prognósticas nos modelos para ajuste.\textsuperscript{\protect\hyperlink{ref-roberts1999}{222}}
\end{itemize}

\hypertarget{comparacao-intragrupos}{%
\section{Comparação intragrupos}\label{comparacao-intragrupos}}

\hypertarget{por-que-nuxe3o-se-deve-comparar-intragrupos-pruxe9---puxf3s-em-ensaios-cluxednicos-aleatorizados}{%
\subsection{Por que não se deve comparar intragrupos (pré - pós) em ensaios clínicos aleatorizados?}\label{por-que-nuxe3o-se-deve-comparar-intragrupos-pruxe9---puxf3s-em-ensaios-cluxednicos-aleatorizados}}

\begin{itemize}
\tightlist
\item
  Testar por mudanças a partir da linha de base separadamente em cada grupos aleatorizados não permite concluir sobre diferenças entre grupos; não se pode fazer inferências a partir da comparação de P-valores.\textsuperscript{\protect\hyperlink{ref-bland2011}{214}}
\end{itemize}

\hypertarget{comparacao-entre-grupos}{%
\section{Comparação entre grupos}\label{comparacao-entre-grupos}}

\hypertarget{o-que-uxe9-comparauxe7uxe3o-entre-grupos-em-ensaios-cluxednicos-aleatorizados}{%
\subsection{O que é comparação entre grupos em ensaios clínicos aleatorizados?}\label{o-que-uxe9-comparauxe7uxe3o-entre-grupos-em-ensaios-cluxednicos-aleatorizados}}

\begin{itemize}
\tightlist
\item
  A comparação se refere ao teste de hipótese nula de não haver diferença (`alteração' ou `mudança') pós-tratamento entre grupos de tratamento.\textsuperscript{\protect\hyperlink{ref-bland2011}{214}}
\end{itemize}

\hypertarget{interacao}{%
\section{Efeito de interação}\label{interacao}}

\hypertarget{por-que-analisar-o-efeito-de-interauxe7uxe3o}{%
\subsection{Por que analisar o efeito de interação?}\label{por-que-analisar-o-efeito-de-interauxe7uxe3o}}

\begin{itemize}
\item
  Em ensaios clínicos aleatorizados, o principal problema de pesquisa é se há uma diferença pré - pós maior em um grupo do que em outro(s).\textsuperscript{\protect\hyperlink{ref-bland2011}{214}}
\item
  A comparação de subgrupos por meio de testes de significância de hipótese nula separados é enganosa por não testar (comparar) diretamente os tamanhos dos efeitos dos tratamentos.\textsuperscript{\protect\hyperlink{ref-Matthews1996}{227}}
\item
  .\textsuperscript{\protect\hyperlink{ref-Bours2023}{189}}
\end{itemize}

\hypertarget{quando-usar-o-termo-de-interauxe7uxe3o}{%
\subsection{Quando usar o termo de interação?}\label{quando-usar-o-termo-de-interauxe7uxe3o}}

\begin{itemize}
\item
  Análise de efeito de interação pode ser usada para testar se o efeito de um tratamento varia entre dois ou mais subgrupos de indivíduos, ou seja, se um efeito é modificado pelo(s) outros(s) efeito(s).\textsuperscript{\protect\hyperlink{ref-Altman1996}{190}}
\item
  A interação entre duas (ou mais) variáveis pode ser utilizada para comparar efeitos do tratamento em subgrupos de ensaios clínicos.\textsuperscript{\protect\hyperlink{ref-Altman2003}{228}}
\item
  O poder estatístico para detectar efeitos de interação é limitado.\textsuperscript{\protect\hyperlink{ref-Altman2003}{228}}
\end{itemize}

\hypertarget{analise-desempenho-diagnostico}{%
\chapter{\texorpdfstring{\textbf{Desempenho diagnóstico}}{Desempenho diagnóstico}}\label{analise-desempenho-diagnostico}}

\hypertarget{tabelas-2x2}{%
\section{Tabelas 2x2}\label{tabelas-2x2}}

\hypertarget{o-que-uxe9-uma-tabela-de-confusuxe3o-2x2}{%
\subsection{O que é uma tabela de confusão 2x2?}\label{o-que-uxe9-uma-tabela-de-confusuxe3o-2x2}}

\begin{itemize}
\tightlist
\item
  Tabela de confusão é uma matriz de 2 linhas por 2 colunas que permite analisar o desempenho de classificação de uma variável dicotômica (padrão-ouro ou referência) versus outra variável dicotômica (novo teste).\textsuperscript{\protect\hyperlink{ref-steckelberg2004}{229}}
\end{itemize}

\hypertarget{como-analisar-o-desempenho-diagnuxf3stico-em-tabelas-2x2}{%
\subsection{Como analisar o desempenho diagnóstico em tabelas 2x2?}\label{como-analisar-o-desempenho-diagnuxf3stico-em-tabelas-2x2}}

\begin{itemize}
\item
  Verdadeiro-positivo (\(VP\)): caso com a condição presente e corretamente identificado como tal.\textsuperscript{\protect\hyperlink{ref-greenhalgh1997b}{230}}
\item
  Falso-negativo (\(FN\)): caso com a condição presente e erroneamente identificado como ausente.\textsuperscript{\protect\hyperlink{ref-greenhalgh1997b}{230}}
\item
  Verdadeiro-negativo (\(VN\)): controle sem a condição presente e corretamente identificados como tal.\textsuperscript{\protect\hyperlink{ref-greenhalgh1997b}{230}}
\item
  Falso-positivo (\(FP\)): controle sem a condição presente e erroneamente identificado como presente.\textsuperscript{\protect\hyperlink{ref-greenhalgh1997b}{230}}
\end{itemize}

\begin{table}

\caption{\label{tab:crosstable}Tabela de confusão 2x2 para análise de desempenho diagnóstico de testes e variáveis dicotômicas.}
\centering
\begin{tabu} to \linewidth {>{\raggedright}X>{\centering}X>{\centering}X>{\centering}X}
\toprule
\textbf{ } & \textbf{Condição presente} & \textbf{Condição ausente} & \textbf{Total}\\
\midrule
Teste positivo & $VP$ & $FP$ & $VP+FP$\\
Teste negativo & $FN$ & $VN$ & $FN+VN$\\
Total & $VP+FN$ & $FP+VN$ & $N=VP+VN+FP+FN$\\
\bottomrule
\end{tabu}
\end{table}

\begin{itemize}
\tightlist
\item
  Tabelas de confusão também podem ser visualizadas em formato de árvores de frequência.\textsuperscript{\protect\hyperlink{ref-steckelberg2004}{229}}
\end{itemize}

\begin{figure}

{\centering \includegraphics{Ciencia-com-R_files/figure-latex/frequency-tree-1} 

}

\caption{Árvore de frequência do desempenho diagnóstico de uma tabela de confusão 2x2 representando um método novo (dicotômico) comparado ao método padrão-ouro ou referência (dicotômico).}\label{fig:frequency-tree}
\end{figure}

\begin{infobox}{images/Rlogo}
O pacote \emph{riskyr}\textsuperscript{\protect\hyperlink{ref-riskyr}{231}} fornece a função \href{https://www.rdocumentation.org/packages/riskyr/versions/0.4.0/topics/plot_prism}{\emph{plot\_prism}} para construir árvores de frequência a partir de diferentes cenários.

\end{infobox}

\hypertarget{quais-probabilidades-caracterizam-o-desempenho-diagnuxf3stico-de-um-teste-em-tabelas-2x2}{%
\subsection{Quais probabilidades caracterizam o desempenho diagnóstico de um teste em tabelas 2x2?}\label{quais-probabilidades-caracterizam-o-desempenho-diagnuxf3stico-de-um-teste-em-tabelas-2x2}}

\begin{itemize}
\tightlist
\item
  Sensibilidade (\(SEN\)), equação \eqref{eq:sen}: Proporção de verdadeiro-positivos dentre aqueles com a condição.\textsuperscript{\protect\hyperlink{ref-greenhalgh1997b}{230}}
\end{itemize}

\begin{equation}
\label{eq:sen}
SEN = \dfrac{VP}{VP+FN}
\end{equation}

\begin{itemize}
\tightlist
\item
  Especificidade (\(ESP\)), equação \eqref{eq:esp}: Proporção de verdadeiro-negativos dentre aqueles sem a condição.\textsuperscript{\protect\hyperlink{ref-greenhalgh1997b}{230}}
\end{itemize}

\begin{equation}
\label{eq:esp}
ESP = \dfrac{VN}{VN+FP}
\end{equation}

\begin{itemize}
\tightlist
\item
  Acurácia (\(ACU\)), equação \eqref{eq:acu}: Proporção de casos e controle corretamente identificados.\textsuperscript{\protect\hyperlink{ref-greenhalgh1997b}{230}}
\end{itemize}

\begin{equation}
\label{eq:acu}
ACU = \dfrac{VP+VN}{VP+VN+FP+FN}
\end{equation}

\begin{itemize}
\tightlist
\item
  Valor preditivo positivo (\(VPP\)), equação \eqref{eq:vpp}: Proporção de casos corretamente identificados como verdadeiro-positivos.\textsuperscript{\protect\hyperlink{ref-greenhalgh1997b}{230}}
\end{itemize}

\begin{equation}
\label{eq:vpp}
VPP = \dfrac{VP}{VP+FP}
\end{equation}

\begin{itemize}
\tightlist
\item
  Valor preditivo negativo (\(VPN\)), equação \eqref{eq:vpn}: Proporção de controles corretamente identificados como verdadeiro-negativos.\textsuperscript{\protect\hyperlink{ref-greenhalgh1997b}{230}}
\end{itemize}

\begin{equation}
\label{eq:vpn}
VPN = \dfrac{VN}{VN+FN}
\end{equation}

\begin{table}

\caption{\label{tab:crosstable-prob}Probabilidades calculados a partir da tabela de confusão 2x2 para análise de desempenho diagnóstico de testes e variáveis dicotômicas.}
\centering
\begin{tabu} to \linewidth {>{\raggedright}X>{\centering}X>{\centering}X>{\centering}X>{\centering}X}
\toprule
\textbf{ } & \textbf{Condição presente} & \textbf{Condição ausente} & \textbf{Total} & \textbf{Probabilidades}\\
\midrule
Teste positivo & $VP$ & $FP$ & $VP+FP$ & $VPP = \frac{VP}{VP+FP}$\\
Teste negativo & $FN$ & $VN$ & $FN+VN$ & $VPN = \frac{VN}{VN+FN}$\\
Total & $VP+FN$ & $FP+VN$ & $N=VP+VN+FP+FN$ & \\
Probabilidades & $SEN = \frac{VP}{VP+FN}$ & $ESP = \frac{VN}{VN+FP}$ &  & $ACU = \frac{VP+VN}{VP+VN+FP+FN}$\\
\bottomrule
\end{tabu}
\end{table}

\begin{infobox}{images/Rlogo}
O pacote \emph{riskyr}\textsuperscript{\protect\hyperlink{ref-riskyr}{231}} fornece a função \href{https://www.rdocumentation.org/packages/riskyr/versions/0.4.0/topics/comp_prob}{\emph{comp\_prob}} para estimar 13 probabilidades relacionadas ao desempenho diagnóstico em tabelas 2x2.

\end{infobox}

\begin{infobox}{images/Rlogo}
O pacote \emph{caret}\textsuperscript{\protect\hyperlink{ref-caret}{232}} fornece a função \href{https://www.rdocumentation.org/packages/caret/versions/3.45/topics/confusionMatrix}{\emph{confusionMatrix}} para estimar 11 probabilidades relacionadas ao desempenho diagnóstico em tabelas 2x2.

\end{infobox}

\hypertarget{graficos-crosshair}{%
\section{\texorpdfstring{Gráficos \emph{crosshair}}{Gráficos crosshair}}\label{graficos-crosshair}}

\hypertarget{o-que-um-gruxe1fico-crosshair}{%
\subsection{\texorpdfstring{O que um gráfico \emph{crosshair}?}{O que um gráfico crosshair?}}\label{o-que-um-gruxe1fico-crosshair}}

\begin{itemize}
\tightlist
\item
  .\textsuperscript{\protect\hyperlink{ref-phillips2010}{233}}
\end{itemize}

\begin{infobox}{images/Rlogo}
O pacote \emph{mada}\textsuperscript{\protect\hyperlink{ref-mada}{234}} fornece a função \href{https://www.rdocumentation.org/packages/mada/versions/0.5.11/topics/crosshair}{\emph{crosshair}} para criar um gráfico \emph{crosshair}\textsuperscript{\protect\hyperlink{ref-phillips2010}{233}} a partir de dados de verdadeiro-positivo, falso-positivo, verdadeiro-negativo e verdadeiro-positivo de tabelas de confusão 2x2.

\end{infobox}

\hypertarget{curvas-roc}{%
\section{Curvas ROC}\label{curvas-roc}}

\hypertarget{o-que-uxe9-a-uxe1rea-sob-a-curva-auroc}{%
\subsection{O que é a área sob a curva (AUROC)?}\label{o-que-uxe9-a-uxe1rea-sob-a-curva-auroc}}

\begin{itemize}
\tightlist
\item
  A área sob a curva ROC (AUC ou AUROC) quantifica o poder de discriminação ou desempenho diagnóstico na classificação de uma variável dicotômica.\textsuperscript{\protect\hyperlink{ref-de2022}{235}}
\end{itemize}

\begin{infobox}{images/Rlogo}
O pacote \emph{proc}\textsuperscript{\protect\hyperlink{ref-pROC}{236}} fornece a função \href{https://www.rdocumentation.org/packages/pROC/versions/1.18.4/topics/plot.roc}{\emph{plot.roc}} para criar uma curva ROC.

\end{infobox}

\hypertarget{como-interpretar-a-uxe1rea-sob-a-curva-roc}{%
\subsection{Como interpretar a área sob a curva (ROC)?}\label{como-interpretar-a-uxe1rea-sob-a-curva-roc}}

\begin{itemize}
\item
  A área sob a curva AUC varia no intervalo \([0.5; 1]\), com valores mais elevados indicando melhor discriminação ou desempenho do modelo de classificação.\textsuperscript{\protect\hyperlink{ref-de2022}{235}}
\item
  As interpretações qualitativas (isto é: pobre/fraca/baixa, moderada/razoável/aceitável, boa ou muito boa/alta/excelente) dos valores de área sob a curva são arbitrários e não devem ser considerados isoladamente.\textsuperscript{\protect\hyperlink{ref-de2022}{235}}
\item
  Modelos de classificação com valores altos de área sob a curva podem ser enganosos se os valores preditos por esses modelos não estiverem adequadamente calibrados.\textsuperscript{\protect\hyperlink{ref-de2022}{235}}
\end{itemize}

\hypertarget{como-analisar-o-desempenho-diagnuxf3stico-em-desfechos-com-distribuiuxe7uxe3o-trimodal-na-populauxe7uxe3o}{%
\subsection{Como analisar o desempenho diagnóstico em desfechos com distribuição trimodal na população?}\label{como-analisar-o-desempenho-diagnuxf3stico-em-desfechos-com-distribuiuxe7uxe3o-trimodal-na-populauxe7uxe3o}}

\begin{itemize}
\tightlist
\item
  Limiares duplos podem ser utilizados para análise de desempenho diagnóstico de testes com distribuição trimodal.\textsuperscript{\protect\hyperlink{ref-ferreira2021}{237}}
\end{itemize}

\hypertarget{interpretacao-desempenho}{%
\section{Interpretação da validade de um teste}\label{interpretacao-desempenho}}

\hypertarget{que-itens-devem-ser-verificados-na-interpretauxe7uxe3o-de-um-estudo-de-validade}{%
\subsection{Que itens devem ser verificados na interpretação de um estudo de validade?}\label{que-itens-devem-ser-verificados-na-interpretauxe7uxe3o-de-um-estudo-de-validade}}

\begin{itemize}
\item
  O novo teste foi comparado junto ao método padrão-ouro.\textsuperscript{\protect\hyperlink{ref-greenhalgh1997b}{230}}
\item
  As probabilidades pontuais estimadas que caracterizam o desempenho diagnóstico do novo teste são altas e adequadas para sua aplicação clínica.\textsuperscript{\protect\hyperlink{ref-greenhalgh1997b}{230}}
\item
  Os intervalos de confiança estimados para as probabilidades do novo teste são estreitos e adequadas para sua aplicação clínica.\textsuperscript{\protect\hyperlink{ref-greenhalgh1997b}{230}}
\item
  O novo teste possui adequada confiabilidade intra/inter examinadores.\textsuperscript{\protect\hyperlink{ref-greenhalgh1997b}{230}}
\item
  O estudo de validação incluiu um espectro adequado da amostra.\textsuperscript{\protect\hyperlink{ref-greenhalgh1997b}{230}}
\item
  Todos os participantes realizaram ambos o novo teste e o padrão-ouro no estudo de validação.\textsuperscript{\protect\hyperlink{ref-greenhalgh1997b}{230}}
\item
  Os examinadores do novo teste estavam cegados para o resultado do teste padrão-ouro.\textsuperscript{\protect\hyperlink{ref-greenhalgh1997b}{230}}
\end{itemize}

\hypertarget{propriedades-psicometricas}{%
\chapter{\texorpdfstring{\textbf{Propriedades psicométricas}}{Propriedades psicométricas}}\label{propriedades-psicometricas}}

\hypertarget{propriedades-psicomuxe9tricas}{%
\section{Propriedades psicométricas}\label{propriedades-psicomuxe9tricas}}

\hypertarget{o-que-suxe3o-propriedades-psicomuxe9tricas}{%
\subsection{O que são propriedades psicométricas?}\label{o-que-suxe3o-propriedades-psicomuxe9tricas}}

\begin{itemize}
\tightlist
\item
  .{[}REF{]}
\end{itemize}

\begin{table}

\caption{\label{tab:crosstable-psicometria}Tabela de confusão sobre propriedades psicométricas de instrumentos.}
\centering
\begin{tabu} to \linewidth {>{\raggedright}X>{\centering}X>{\centering}X}
\toprule
\textbf{ } & \textbf{Concordância alta} & \textbf{Concordância baixa}\\
\midrule
Validade alta & Adequado & Inadequado\\
Validade baixa & Inadequado & Inadequado\\
\bottomrule
\end{tabu}
\end{table}

\begin{infobox}{images/Rlogo}
O pacote \emph{lavaan}\textsuperscript{\protect\hyperlink{ref-lavaan}{238}} fornece a função \href{https://www.rdocumentation.org/packages/lavaan/versions/0.6-16/topics/cfa}{\emph{cfa}} para implementar modelos de análise fatorial confirmatória.

\end{infobox}

\begin{infobox}{images/Rlogo}
O pacote \emph{lavaan}\textsuperscript{\protect\hyperlink{ref-lavaan}{238}} fornece a função \href{https://www.rdocumentation.org/packages/lavaan/versions/0.6-16/topics/modificationIndices}{\emph{modificationIndices}} para calcular os índices de modificação.

\end{infobox}

\begin{infobox}{images/Rlogo}
O pacote \emph{semTools}\textsuperscript{\protect\hyperlink{ref-semTools}{239}} fornece a função \href{https://www.rdocumentation.org/packages/semTools/versions/0.5-6/topics/reliability-deprecated}{\emph{reliability}} para analisar a confiabilidade de um instrumento.

\end{infobox}

\begin{infobox}{images/Rlogo}
O pacote \emph{psych}\textsuperscript{\protect\hyperlink{ref-psych-2}{240}} fornece a função \href{https://www.rdocumentation.org/packages/psych/versions/2.3.6/topics/ICC}{\emph{icc}} para calcular a confiabilidade utilizando coeficientes de correlação intraclasse.

\end{infobox}

\hypertarget{analise-fatorial-exploratoria}{%
\section{Análise fatorial exploratória}\label{analise-fatorial-exploratoria}}

\hypertarget{o-que-uxe9-anuxe1lise-fatorial-exploratuxf3ria}{%
\subsection{O que é análise fatorial exploratória?}\label{o-que-uxe9-anuxe1lise-fatorial-exploratuxf3ria}}

\begin{itemize}
\tightlist
\item
  .{[}REF{]}
\end{itemize}

\hypertarget{analise-fatorial-ocnfirmatoria}{%
\section{Análise fatorial confirmatória}\label{analise-fatorial-ocnfirmatoria}}

\hypertarget{o-que-uxe9-anuxe1lise-fatorial-confirmatuxf3ria}{%
\subsection{O que é análise fatorial confirmatória?}\label{o-que-uxe9-anuxe1lise-fatorial-confirmatuxf3ria}}

\begin{itemize}
\tightlist
\item
  .{[}REF{]}
\end{itemize}

\hypertarget{validade}{%
\section{Validade}\label{validade}}

\hypertarget{quais-suxe3o-os-tipos-de-validade}{%
\subsection{Quais são os tipos de validade?}\label{quais-suxe3o-os-tipos-de-validade}}

\begin{itemize}
\item
  Conteúdo.{[}REF{]}
\item
  Construto.{[}REF{]}
\item
  Critério.{[}REF{]}
\end{itemize}

\hypertarget{validade-conteuxfado}{%
\section{Validade de conteúdo}\label{validade-conteuxfado}}

\hypertarget{o-que-uxe9-validade-interna}{%
\subsection{O que é validade interna?}\label{o-que-uxe9-validade-interna}}

\begin{itemize}
\tightlist
\item
  .\textsuperscript{\protect\hyperlink{ref-findley2021}{241}}
\end{itemize}

\hypertarget{o-que-uxe9-validade-externa}{%
\subsection{O que é validade externa?}\label{o-que-uxe9-validade-externa}}

\begin{itemize}
\tightlist
\item
  .\textsuperscript{\protect\hyperlink{ref-findley2021}{241}}
\end{itemize}

\hypertarget{que-fatores-afetam-a-validade}{%
\subsection{Que fatores afetam a validade?}\label{que-fatores-afetam-a-validade}}

\begin{itemize}
\item
  A amostragem não probabilística pode dificultar a generalização dos achados da amostra para a população, diminuindo assim a validade externa do estudo.\textsuperscript{\protect\hyperlink{ref-Banerjee2010}{26}}
\item
  Quando as características da amostra obtida por seleção não probabilística forem similares às da população, a validade externa pode ser maior.\textsuperscript{\protect\hyperlink{ref-Banerjee2010}{26}}
\end{itemize}

\hypertarget{como-avaliar-a-validade-de-um-estudo}{%
\subsection{Como avaliar a validade de um estudo?}\label{como-avaliar-a-validade-de-um-estudo}}

\begin{itemize}
\tightlist
\item
  As características da amostra apresentadas na Tabela 1 são úteis para interpretação da validade interna e externa dos achados do estudo.\textsuperscript{\protect\hyperlink{ref-Westreich2013}{128}}
\end{itemize}

\hypertarget{validade-face}{%
\section{Validade de face}\label{validade-face}}

\hypertarget{o-que-uxe9-validade-de-face}{%
\subsection{O que é validade de face?}\label{o-que-uxe9-validade-de-face}}

\begin{itemize}
\tightlist
\item
  .{[}RF{]}
\end{itemize}

\hypertarget{validade-constructo}{%
\section{Validade do construto}\label{validade-constructo}}

\hypertarget{o-que-uxe9-construto}{%
\subsection{O que é construto?}\label{o-que-uxe9-construto}}

\begin{itemize}
\tightlist
\item
  .{[}RF{]}
\end{itemize}

\hypertarget{validade-fatorial}{%
\section{Validade fatorial}\label{validade-fatorial}}

\hypertarget{o-que-uxe9-validade-fatorial}{%
\subsection{O que é validade fatorial?}\label{o-que-uxe9-validade-fatorial}}

\begin{itemize}
\tightlist
\item
  .{[}RF{]}
\end{itemize}

\hypertarget{validade-convergente}{%
\section{Validade convergente}\label{validade-convergente}}

\hypertarget{o-que-uxe9-validade-convergente}{%
\subsection{O que é validade convergente?}\label{o-que-uxe9-validade-convergente}}

\begin{itemize}
\tightlist
\item
  .{[}RF{]}
\end{itemize}

\hypertarget{validade-discriminante}{%
\section{Validade discriminante}\label{validade-discriminante}}

\hypertarget{o-que-uxe9-validade-discriminante}{%
\subsection{O que é validade discriminante?}\label{o-que-uxe9-validade-discriminante}}

\begin{itemize}
\tightlist
\item
  .{[}RF{]}
\end{itemize}

\hypertarget{validade-criterio}{%
\section{Validade de critério}\label{validade-criterio}}

\hypertarget{o-que-uxe9-validade-de-crituxe9rio}{%
\subsection{O que é validade de critério?}\label{o-que-uxe9-validade-de-crituxe9rio}}

\begin{itemize}
\tightlist
\item
  .{[}RF{]}
\end{itemize}

\hypertarget{validade-concorrente}{%
\section{Validade concorrente}\label{validade-concorrente}}

\hypertarget{o-que-uxe9-concorrente}{%
\subsection{O que é concorrente?}\label{o-que-uxe9-concorrente}}

\begin{itemize}
\tightlist
\item
  .{[}RF{]}
\end{itemize}

\hypertarget{o-que-uxe9-validade-concorrente}{%
\subsection{O que é validade concorrente?}\label{o-que-uxe9-validade-concorrente}}

\begin{itemize}
\tightlist
\item
  .{[}RF{]}
\end{itemize}

\hypertarget{o-que-uxe9-validade-preditiva}{%
\subsection{O que é validade preditiva?}\label{o-que-uxe9-validade-preditiva}}

\begin{itemize}
\tightlist
\item
  .{[}RF{]}
\end{itemize}

\hypertarget{responsividade}{%
\section{Responsividade}\label{responsividade}}

\hypertarget{o-que-uxe9-responsividade}{%
\subsection{O que é responsividade?}\label{o-que-uxe9-responsividade}}

\begin{itemize}
\tightlist
\item
  .{[}REF{]}
\end{itemize}

\hypertarget{analise-concordancia-confiabilidade}{%
\chapter{\texorpdfstring{\textbf{Concordância e confiabilidade}}{Concordância e confiabilidade}}\label{analise-concordancia-confiabilidade}}

\hypertarget{problemas}{%
\section{Problemas de pesquisa}\label{problemas}}

\hypertarget{quais-problemas-de-pesquisa-suxe3o-investigados-com-estudos-de-concorduxe2ncia-e-confiabilidade}{%
\subsection{Quais problemas de pesquisa são investigados com estudos de concordância e confiabilidade?}\label{quais-problemas-de-pesquisa-suxe3o-investigados-com-estudos-de-concorduxe2ncia-e-confiabilidade}}

\begin{itemize}
\item
  Quão reprodutíveis são as mensurações?\textsuperscript{\protect\hyperlink{ref-altman1983}{242}}
\item
  Os diferentes métodos medem a mesma coisa em média?\textsuperscript{\protect\hyperlink{ref-altman1983}{242}}
\item
  Existe viés entre as medidas de diferentes métodos (isto é, medem a mesma coisa em média)?\textsuperscript{\protect\hyperlink{ref-altman1983}{242}}
\item
  Um método pode substituir o outro?\textsuperscript{\protect\hyperlink{ref-altman1983}{242}}
\end{itemize}

\hypertarget{quais-fontes-de-variabilidade-suxe3o-comumente-investigadas}{%
\subsection{Quais fontes de variabilidade são comumente investigadas?}\label{quais-fontes-de-variabilidade-suxe3o-comumente-investigadas}}

\begin{itemize}
\item
  Intra/Entre sujeitos.\textsuperscript{\protect\hyperlink{ref-altman1983}{242}}
\item
  Intra/Entre repetições.\textsuperscript{\protect\hyperlink{ref-altman1983}{242}}
\item
  Intra/Entre observadores.\textsuperscript{\protect\hyperlink{ref-altman1983}{242}}
\end{itemize}

\hypertarget{concordancia}{%
\section{Concordância}\label{concordancia}}

\hypertarget{o-que-uxe9-concorduxe2ncia}{%
\subsection{O que é concordância?}\label{o-que-uxe9-concorduxe2ncia}}

\begin{itemize}
\tightlist
\item
  .{[}REF{]}
\end{itemize}

\hypertarget{quais-muxe9todos-suxe3o-adequados-para-anuxe1lise-de-concorduxe2ncia-de-variuxe1veis-dicotuxf4micas}{%
\subsection{Quais métodos são adequados para análise de concordância de variáveis dicotômicas?}\label{quais-muxe9todos-suxe3o-adequados-para-anuxe1lise-de-concorduxe2ncia-de-variuxe1veis-dicotuxf4micas}}

\begin{itemize}
\tightlist
\item
  Coeficiente de Cohen \(\kappa\): mede a concordância corrigida pelo acaso.\textsuperscript{\protect\hyperlink{ref-scott1955}{243},\protect\hyperlink{ref-cohen1960}{244}}
\end{itemize}

\begin{table}

\caption{\label{tab:crosstable-kappa-2x2}Tabela de confusão 2x2 para análise de concordância de testes e variáveis dicotômicas.}
\centering
\begin{tabu} to \linewidth {>{\raggedright}X>{\centering}X>{\centering}X>{\centering}X}
\toprule
\textbf{ } & \textbf{Teste positivo} & \textbf{Teste negativo} & \textbf{Total}\\
\midrule
Teste positivo & $a$ & $b$ & $g=a+b$\\
Teste negativo & $c$ & $d$ & $h=c+d$\\
Total & $e=a+c$ & $f=b+d$ & $N=a+b+c+d$\\
\bottomrule
\end{tabu}
\end{table}

\begin{itemize}
\tightlist
\item
  Coeficiente de correlação tetracórica \(r_{tet}\).\textsuperscript{\protect\hyperlink{ref-i.mathe1901}{245},\protect\hyperlink{ref-banerjee1999}{246}}
\end{itemize}

\begin{infobox}{images/Rlogo}
O pacote \emph{psych}\textsuperscript{\protect\hyperlink{ref-psych}{247}} fornece a função \href{https://www.rdocumentation.org/packages/psych/versions/2.3.6/topics/tetrachoric}{\emph{tetrachoric}} para calcular o coeficiente de correlação tetracórica (\(r_{tet}\)).

\end{infobox}

\hypertarget{quais-muxe9todos-nuxe3o-suxe3o-adequados-para-anuxe1lise-de-concorduxe2ncia-de-variuxe1veis-dicotuxf4micas}{%
\subsection{Quais métodos não são adequados para análise de concordância de variáveis dicotômicas?}\label{quais-muxe9todos-nuxe3o-suxe3o-adequados-para-anuxe1lise-de-concorduxe2ncia-de-variuxe1veis-dicotuxf4micas}}

\begin{itemize}
\item
  Concordância absoluta \(C_{A}\) - quantidade de casos em que examinadores concordam - não é recomendada porque não corrige a estimativa para possíveis concordâncias ao acaso.\textsuperscript{\protect\hyperlink{ref-banerjee1999}{246}}
\item
  Concordância percentual \(C_{\%}\) - proporção de casos em que examinadores concordam pela quantidade total de casos - não é recomendada porque não corrige a estimativa para possíveis concordâncias ao acaso.\textsuperscript{\protect\hyperlink{ref-banerjee1999}{246}}
\item
  Qui-quadrado \(\chi^2\) a partir da tabela de contigência não é recomendado porque tal teste analisa associação.\textsuperscript{\protect\hyperlink{ref-banerjee1999}{246}}
\item
  A família de coeficientes de Cohen \(\kappa\) não é adequada para analisar concordância quando as variáveis são aparentemente (e não originalmente) dicotômicas.\textsuperscript{\protect\hyperlink{ref-banerjee1999}{246}}
\end{itemize}

\hypertarget{quais-muxe9todos-suxe3o-adequados-para-anuxe1lise-de-concorduxe2ncia-de-variuxe1veis-categuxf3ricas}{%
\subsection{Quais métodos são adequados para análise de concordância de variáveis categóricas?}\label{quais-muxe9todos-suxe3o-adequados-para-anuxe1lise-de-concorduxe2ncia-de-variuxe1veis-categuxf3ricas}}

\begin{itemize}
\item
  Coeficiente de Cohen \(\kappa\): mede a concordância corrigida pelo acaso.\textsuperscript{\protect\hyperlink{ref-scott1955}{243},\protect\hyperlink{ref-cohen1960}{244}}
\item
  Coeficiente de Cohen ponderado \(\kappa_{w}\): mede a concordância corrigida pelo acaso.\textsuperscript{\protect\hyperlink{ref-scott1955}{243},\protect\hyperlink{ref-cohen1960}{244}}
\end{itemize}

\begin{table}

\caption{\label{tab:crosstable-kappa-3x3}Tabela de confusão 3x3 para análise de concordância de testes e variáveis dicotômicas.}
\centering
\begin{tabu} to \linewidth {>{\raggedright}X>{\centering}X>{\centering}X>{\centering}X>{\centering}X}
\toprule
\textbf{ } & \textbf{Grave} & \textbf{Moderado} & \textbf{Leve} & \textbf{Total}\\
\midrule
Grave & $a$ & $b$ & $c$ & $j=a+b+c$\\
Moderado & $d$ & $e$ & $f$ & $k=d+e+f$\\
Leve & $g$ & $h$ & $i$ & $l=g+h+i$\\
Total & $j=a+d+g$ & $k=b+e+h$ & $l=c+f+i$ & $N=a+b+c+d+e+f+g+h+i$\\
\bottomrule
\end{tabu}
\end{table}

\begin{itemize}
\tightlist
\item
  Coeficiente de correlação policórica \(r_{pol}\).\textsuperscript{\protect\hyperlink{ref-banerjee1999}{246}}
\end{itemize}

\begin{infobox}{images/Rlogo}
O pacote \emph{psych}\textsuperscript{\protect\hyperlink{ref-psych}{247}} fornece a função \href{https://www.rdocumentation.org/packages/psych/versions/2.3.6/topics/tetrachoric}{\emph{tetrachoric}} para calcular o coeficiente de correlação policórica (\(r_{pol}\)).

\end{infobox}

\hypertarget{quais-muxe9todos-suxe3o-adequados-para-anuxe1lise-de-concorduxe2ncia-de-variuxe1veis-categuxf3ricas-e-contuxednuas}{%
\subsection{Quais métodos são adequados para análise de concordância de variáveis categóricas e contínuas?}\label{quais-muxe9todos-suxe3o-adequados-para-anuxe1lise-de-concorduxe2ncia-de-variuxe1veis-categuxf3ricas-e-contuxednuas}}

\begin{itemize}
\tightlist
\item
  Coeficiente de correlação bisserial \(r_{s}\).\textsuperscript{\protect\hyperlink{ref-banerjee1999}{246}}
\end{itemize}

\begin{infobox}{images/Rlogo}
O pacote \emph{psych}\textsuperscript{\protect\hyperlink{ref-psych}{247}} fornece a função \href{https://www.rdocumentation.org/packages/psych/versions/2.3.6/topics/tetrachoric}{\emph{tetrachoric}} para calcular o coeficiente de correlação bisserial (\(r_{s}\)).

\end{infobox}

\hypertarget{quais-muxe9todos-suxe3o-adequados-para-anuxe1lise-de-concorduxe2ncia-de-variuxe1veis-ordinais}{%
\subsection{Quais métodos são adequados para análise de concordância de variáveis ordinais?}\label{quais-muxe9todos-suxe3o-adequados-para-anuxe1lise-de-concorduxe2ncia-de-variuxe1veis-ordinais}}

\begin{itemize}
\tightlist
\item
  Coeficiente de Cohen ponderado \(\kappa_{w}\): mede a concordância corrigida pelo acaso.\textsuperscript{\protect\hyperlink{ref-scott1955}{243},\protect\hyperlink{ref-cohen1960}{244}}
\end{itemize}

\hypertarget{quais-muxe9todos-suxe3o-adequados-para-anuxe1lise-de-concorduxe2ncia-de-variuxe1veis-contuxednuas}{%
\subsection{Quais métodos são adequados para análise de concordância de variáveis contínuas?}\label{quais-muxe9todos-suxe3o-adequados-para-anuxe1lise-de-concorduxe2ncia-de-variuxe1veis-contuxednuas}}

\begin{itemize}
\item
  Gráfico de dispersão com a reta de regressão.\textsuperscript{\protect\hyperlink{ref-altman1983}{242}}
\item
  Gráfico de limites de concordância (média dos testes vs.~diferença entre testes) com a reta de regressão do viés e respectivo intervalo de confiança.\textsuperscript{\protect\hyperlink{ref-altman1983}{242}}
\end{itemize}

\hypertarget{quais-muxe9todos-nuxe3o-suxe3o-adequados-para-anuxe1lise-de-concorduxe2ncia-de-variuxe1veis-contuxednuas}{%
\subsection{Quais métodos não são adequados para análise de concordância de variáveis contínuas?}\label{quais-muxe9todos-nuxe3o-suxe3o-adequados-para-anuxe1lise-de-concorduxe2ncia-de-variuxe1veis-contuxednuas}}

\begin{itemize}
\item
  Comparação de médias: dois métodos apresentarem médias similares - isto é, `sem diferença estatística' após um teste inferencial de hipótese nula \(H_{0}:\mu_{1} = \mu_{2}\) - não informa sobre a concordância deles. Métodos com maior erro de medida tendem a ter menos chance de rejeição da hipótese nula.\textsuperscript{\protect\hyperlink{ref-altman1983}{242}}
\item
  Correlação bivariada: o coeficiente de correlação dependente tanto da variação entre indivíduos (isto é, entre os valores verdadeiros) quanto da variação intraindividual (isto é, erro de medida). Se a variância dos erros de medida de ambos os métodos não for pequena comparadas à variância dos valores verdadeiros, o tamanho do efeito da correlação será pequeno mesmo que os métodos possuam boa concordância.\textsuperscript{\protect\hyperlink{ref-altman1983}{242}}
\item
  Regressão linear: o teste da hipótese nula da inclinação da reta de regressão (\(H_{0}:\beta = 0\)) é equivalente a testar a correlação bivariada (\(H_{0}:\rho = 0\)).\textsuperscript{\protect\hyperlink{ref-altman1983}{242}}
\end{itemize}

\hypertarget{quais-muxe9todos-suxe3o-adequados-para-modelagem-de-concorduxe2ncia}{%
\subsection{Quais métodos são adequados para modelagem de concordância?}\label{quais-muxe9todos-suxe3o-adequados-para-modelagem-de-concorduxe2ncia}}

\begin{itemize}
\tightlist
\item
  Modelo log-linear.\textsuperscript{\protect\hyperlink{ref-banerjee1999}{246}}
\end{itemize}

\hypertarget{confiabilidade}{%
\section{Confiabilidade}\label{confiabilidade}}

\hypertarget{o-que-uxe9-confiabilidade}{%
\subsection{O que é confiabilidade?}\label{o-que-uxe9-confiabilidade}}

\begin{itemize}
\tightlist
\item
  .{[}REF{]}
\end{itemize}

\hypertarget{quais-muxe9todos-suxe3o-adequados-para-anuxe1lise-de-confiabilidade}{%
\subsection{Quais métodos são adequados para análise de confiabilidade?}\label{quais-muxe9todos-suxe3o-adequados-para-anuxe1lise-de-confiabilidade}}

\begin{itemize}
\tightlist
\item
  .{[}REF{]}
\end{itemize}

\hypertarget{meta-analise}{%
\chapter{\texorpdfstring{\textbf{Meta-análise}}{Meta-análise}}\label{meta-analise}}

\hypertarget{meta-analise}{%
\section{Meta-análise}\label{meta-analise}}

\hypertarget{o-que-uxe9-meta-anuxe1lise}{%
\subsection{O que é meta-análise?}\label{o-que-uxe9-meta-anuxe1lise}}

\begin{itemize}
\tightlist
\item
  .{[}{]}
\end{itemize}

\hypertarget{interpretacao}{%
\section{Interpretação de efeitos em meta-análise}\label{interpretacao}}

\hypertarget{como-avaliar-a-variauxe7uxe3o-do-tamanho-do-efeito}{%
\subsection{Como avaliar a variação do tamanho do efeito?}\label{como-avaliar-a-variauxe7uxe3o-do-tamanho-do-efeito}}

\begin{itemize}
\item
  O intervalo de predição contém informação sobre a variação do tamanho do efeito.\textsuperscript{\protect\hyperlink{ref-Borenstein2022}{248}}
\item
  Se o intervalo de predição não contém a hipótese nula (\(H_{0}\)), podemos concluir que (a) o tratamento funciona igualmente bem em todas as populações, ou que ele funciona melhor em algumas populações do que em outras.\textsuperscript{\protect\hyperlink{ref-Borenstein2022}{248}}
\item
  Se o intervalo de predição contém a hipótese nula (\(H_{0}\)), podemos concluir que o tratamento pode ser benéfico em algumas populações, mas prejudicial em outras, de modo que a estimativa pontual (geralmente a média) torna-se amplamente irrelevante. Nesse caso, é recomendado investigar em que populações o tratamento seria benéfico e em quais causaria danos.\textsuperscript{\protect\hyperlink{ref-Borenstein2022}{248}}
\end{itemize}

\hypertarget{como-avaliar-a-heterogeneidade-entre-os-estudos}{%
\subsection{Como avaliar a heterogeneidade entre os estudos?}\label{como-avaliar-a-heterogeneidade-entre-os-estudos}}

\begin{itemize}
\item
  A heterogeneidade - variação não-aleatória - no efeito do tratamento entre os estudos incluídos em uma meta-análise pode ser avaliada pelo \(I^{2}\).\textsuperscript{\protect\hyperlink{ref-Borenstein2022}{248},\protect\hyperlink{ref-Ruxfccker2008}{249}}
\item
  \(I^{2}\) representa qual proporção da variância observada reflete a variância nos efeitos verdadeiros em vez do erro de amostragem.\textsuperscript{\protect\hyperlink{ref-Borenstein2022}{248}}
\item
  \(I^{2}\) não depende da quantidade de estudos incluídos na meta-análise. Entretanto, \(I^{2}\) aumenta com a quantidade de participantes incluídos nos estudos meta-analisados.\textsuperscript{\protect\hyperlink{ref-Ruxfccker2008}{249}}
\item
  A heterogeneidade entre estudos é explicada de modo mais confiável utilizando dados de pacientes individuais, uma vez que a direção verdadeira da modificação de efeito não pode ser observada a partir de dados agregados no estudo.\textsuperscript{\protect\hyperlink{ref-degrooth2023}{250}}
\end{itemize}

\begin{infobox}{images/Rlogo}
O pacote \emph{metagear}\textsuperscript{\protect\hyperlink{ref-metagear}{251}} fornece funções para condução e análise de revisões sistemáticas.

\end{infobox}

\begin{infobox}{images/Rlogo}
O pacote \emph{metagear}\textsuperscript{\protect\hyperlink{ref-metagear}{251}} fornece a função \href{https://www.rdocumentation.org/packages/metagear/versions/0.7/topics/plot_PRISMA}{\emph{plot\_PRISMA}} para gerar o fluxograma de uma revisão sistemática de acordo com o \emph{Preferred Reporting Items for Systematic Reviews and Meta-Analyses}\textsuperscript{\protect\hyperlink{ref-Moher2015}{252}}.

\end{infobox}

\begin{infobox}{images/Rlogo}
O pacote \emph{PRISMA2020}\textsuperscript{\protect\hyperlink{ref-PRISMA2020-2}{253},\protect\hyperlink{ref-PRISMA2020}{254}} fornece a função \href{https://www.rdocumentation.org/packages/PRISMA2020/versions/1.1.1/topics/PRISMA_flowdiagram}{\emph{PRISMA\_flowdiagram}} para elaboração do fluxograma de revisões sistemáticas no formato padrão.

\end{infobox}

\cftaddtitleline{toc}{chapter}{\rule{\textwidth}{0.4pt}}{}

\hypertarget{parte-5---produuxe7uxe3o-cientuxedfica}{%
\chapter*{\texorpdfstring{\emph{Parte 5 - Produção Científica}}{Parte 5 - Produção Científica}}\label{parte-5---produuxe7uxe3o-cientuxedfica}}
\addcontentsline{toc}{chapter}{\emph{Parte 5 - Produção Científica}}

\markboth{}{}
\par\noindent\rule{\textwidth}{0.05in}

\hypertarget{redacao}{%
\chapter{\texorpdfstring{\textbf{Redação estatística}}{Redação estatística}}\label{redacao}}

\hypertarget{diretrizes}{%
\section{Diretrizes}\label{diretrizes}}

\begin{itemize}
\item
  \emph{Review of guidance papers on regression modeling in statistical series of medical journals}.\textsuperscript{\protect\hyperlink{ref-Wallisch2022}{255}}
\item
  \emph{Principles and recommendations for incorporating estimands into clinical study protocol templates}.\textsuperscript{\protect\hyperlink{ref-Lynggaard2022}{256}}
\item
  \emph{How to write statistical analysis section in medical research}.\textsuperscript{\protect\hyperlink{ref-Dwivedi2022}{166}}
\item
  \emph{Recommendations for Statistical Reporting in Cardiovascular Medicine: A Special Report From the American Heart Association}.\textsuperscript{\protect\hyperlink{ref-Althouse2021}{257}}
\item
  \emph{Framework for the treatment and reporting of missing data in observational studies: The Treatment And Reporting of Missing data in Observational Studies framework}.\textsuperscript{\protect\hyperlink{ref-Lee2021}{258}}
\item
  \emph{Guidelines for reporting of figures and tables for clinical research in urology}.\textsuperscript{\protect\hyperlink{ref-Vickers2020}{259}}
\item
  \emph{Who is in this study, anyway? Guidelines for a useful Table 1}.\textsuperscript{\protect\hyperlink{ref-Hayes-Larson2019}{131}}
\item
  \emph{Guidelines for Reporting of Statistics for Clinical Research in Urology}.\textsuperscript{\protect\hyperlink{ref-assel2019}{260}}
\item
  \emph{Reveal, Don't Conceal: Transforming Data Visualization to Improve Transparency}.\textsuperscript{\protect\hyperlink{ref-Weissgerber2019}{140}}
\item
  \emph{Guidelines for the Content of Statistical Analysis Plans in Clinical Trials}.\textsuperscript{\protect\hyperlink{ref-Gamble2017}{261}}
\item
  \emph{Basic statistical reporting for articles published in Biomedical Journals: The `'Statistical Analyses and Methods in the Published Literature'' or the SAMPL Guidelines}.\textsuperscript{\protect\hyperlink{ref-Lang2015}{262}}
\item
  \emph{Beyond Bar and Line Graphs: Time for a New Data Presentation Paradigm}.\textsuperscript{\protect\hyperlink{ref-Weissgerber2015}{263}}
\item
  \emph{STRengthening analytical thinking for observational studies: the STRATOS initiative}.\textsuperscript{\protect\hyperlink{ref-Sauerbrei2014}{264}}
\item
  \emph{Research methods and reporting}.\textsuperscript{\protect\hyperlink{ref-groves2008}{265}}
\item
  \emph{How to ensure your paper is rejected by the statistical reviewer}.\textsuperscript{\protect\hyperlink{ref-stratton2005}{266}}
\end{itemize}

\hypertarget{checklists}{%
\section{Listas de verificação}\label{checklists}}

\begin{itemize}
\item
  \emph{A CHecklist for statistical Assessment of Medical Papers (the CHAMP statement): explanation and elaboration}.\textsuperscript{\protect\hyperlink{ref-Mansournia2021}{267}}
\item
  \emph{Checklist for clinical applicability of subgroup analysis}.\textsuperscript{\protect\hyperlink{ref-Gil-Sierra2020}{268}}
\item
  \emph{Evidence-based statistical analysis and methods in biomedical research (SAMBR) checklists according to design features}.\textsuperscript{\protect\hyperlink{ref-dwivedi2019}{165}}
\end{itemize}

\hypertarget{plano-analise-estatistica}{%
\section{Plano de análise estatística}\label{plano-analise-estatistica}}

\hypertarget{o-que-uxe9-plano-de-anuxe1lise-estatuxedstica}{%
\subsection{O que é plano de análise estatística?}\label{o-que-uxe9-plano-de-anuxe1lise-estatuxedstica}}

\begin{itemize}
\tightlist
\item
  .{[}REF{]}
\end{itemize}

\hypertarget{resultados-analise-estatistica}{%
\section{Resultados da análise estatística}\label{resultados-analise-estatistica}}

\hypertarget{como-redigir-os-resultados-da-anuxe1lise-estatuxedstica}{%
\subsection{Como redigir os resultados da análise estatística?}\label{como-redigir-os-resultados-da-anuxe1lise-estatuxedstica}}

\begin{itemize}
\tightlist
\item
  .{[}REF{]}
\end{itemize}

\cftaddtitleline{toc}{chapter}{\rule{\textwidth}{0.4pt}}{}

\hypertarget{bibliografia}{%
\chapter*{\texorpdfstring{\emph{Bibliografia}}{Bibliografia}}\label{bibliografia}}
\addcontentsline{toc}{chapter}{\emph{Bibliografia}}

\markboth{}{}
\par\noindent\rule{\textwidth}{0.05in}

\hypertarget{fontes-externas}{%
\chapter*{\texorpdfstring{\textbf{Fontes externas}}{Fontes externas}}\label{fontes-externas}}
\addcontentsline{toc}{chapter}{\textbf{Fontes externas}}

\hypertarget{american-heart-association}{%
\section*{American Heart Association}\label{american-heart-association}}
\addcontentsline{toc}{section}{American Heart Association}

\begin{itemize}
\tightlist
\item
  \href{https://www.ahajournals.org/statistical-recommendations}{\emph{Statistical Reporting Recommendations - AHA/ASA journals}}
\end{itemize}

\hypertarget{american-physiological-society}{%
\section*{American Physiological Society}\label{american-physiological-society}}
\addcontentsline{toc}{section}{American Physiological Society}

\begin{itemize}
\item
  \href{https://journals.physiology.org/topic/advances-collections/statistics?seriesKey=\&tagCode=\&}{\emph{Statistics}}
\item
  \href{https://journals.physiology.org/topic/advances-collections/explorations-in-statistics?seriesKey=\&tagCode=\&}{\emph{Exploration in Statistics}}
\item
  \href{https://journals.physiology.org/topic/advances-collections/general-statistics?seriesKey=\&tagCode=\&}{\emph{General Statistics}}
\item
  \href{https://journals.physiology.org/topic/advances-collections/reporting-statistics?seriesKey=\&tagCode=\&}{\emph{Reporting Statistics}}
\end{itemize}

\hypertarget{american-statistical-association}{%
\section*{American Statistical Association}\label{american-statistical-association}}
\addcontentsline{toc}{section}{American Statistical Association}

\begin{itemize}
\tightlist
\item
  \href{https://www.tandfonline.com/toc/utas20/73/sup1?nav=tocList}{\emph{Statistical Inference in the 21st Century: A World Beyond p \textless{} 0.05 - The American Statistical Association}}
\end{itemize}

\hypertarget{british-medicine-journal}{%
\section*{British Medicine Journal}\label{british-medicine-journal}}
\addcontentsline{toc}{section}{British Medicine Journal}

\begin{itemize}
\item
  \href{https://www.bmj.com/specialties/statistics}{\emph{Statistics - Latest from The BMJ}}
\item
  \href{https://www.bmj.com/specialties/statistics-notes}{\emph{Statistics notes - Latest from The BMJ}}
\item
  \href{https://www.bmj.com/specialties/statistics-and-research-methods}{\emph{Statistics and research methods - Latest from The BMJ}}
\item
  \href{https://www.bmj.com/about-bmj/resources-readers/publications/statistics-square-one}{\emph{Statistics at Square One}}
\item
  \href{https://www.bmj.com/research/research-methods-and-reporting}{\emph{Research methods \& reporting}}
\end{itemize}

\hypertarget{enhancing-the-quality-and-transparency-of-health-research-network}{%
\section*{Enhancing the QUality And Transparency Of health Research Network}\label{enhancing-the-quality-and-transparency-of-health-research-network}}
\addcontentsline{toc}{section}{Enhancing the QUality And Transparency Of health Research Network}

\begin{itemize}
\tightlist
\item
  \emph{Enhancing the Quality and Transparency of health research} \href{https://www.equator-network.org}{EQUATOR Network}.\textsuperscript{\protect\hyperlink{ref-Altman2008}{269}}
\end{itemize}

\hypertarget{journal-of-the-amercan-medical-association}{%
\section*{Journal of the Amercan Medical Association}\label{journal-of-the-amercan-medical-association}}
\addcontentsline{toc}{section}{Journal of the Amercan Medical Association}

\begin{itemize}
\tightlist
\item
  \href{https://jamanetwork.com/collections/44042/jama-guide-to-statistics-and-methods}{\emph{JAMA Guide to Statistics and Methods - JAMA}}
\end{itemize}

\hypertarget{nature-publishing-group}{%
\section*{Nature Publishing Group}\label{nature-publishing-group}}
\addcontentsline{toc}{section}{Nature Publishing Group}

\begin{itemize}
\tightlist
\item
  \href{https://www.nature.com/collections/qghhqm}{\emph{Statistics for Biologists - Nature Publising Group}}
\end{itemize}

\hypertarget{royal-statistical-society}{%
\section*{Royal Statistical Society}\label{royal-statistical-society}}
\addcontentsline{toc}{section}{Royal Statistical Society}

\begin{itemize}
\tightlist
\item
  \href{https://royal-statistical-society.github.io/datavisguide/}{\emph{Best Practices for Data Visualisation - Royal Statistical Society}}
\end{itemize}

\hypertarget{statistics-in-medicine}{%
\section*{Statistics in Medicine}\label{statistics-in-medicine}}
\addcontentsline{toc}{section}{Statistics in Medicine}

\begin{itemize}
\tightlist
\item
  \href{https://onlinelibrary.wiley.com/page/journal/10970258/homepage/tutorials.htm}{\emph{Tutorials in Biostatistics Papers}}
\end{itemize}

\hypertarget{the-journal-of-applied-statistics-in-the-pharmaceutical-industry}{%
\section*{The Journal of Applied Statistics in the Pharmaceutical Industry}\label{the-journal-of-applied-statistics-in-the-pharmaceutical-industry}}
\addcontentsline{toc}{section}{The Journal of Applied Statistics in the Pharmaceutical Industry}

\begin{itemize}
\tightlist
\item
  \href{https://onlinelibrary.wiley.com/page/journal/15391612/homepage/tutorial_papers.htm}{\emph{Tutorial Papers}}
\end{itemize}

\hypertarget{referuxeancias}{%
\chapter*{\texorpdfstring{\textbf{Referências}}{Referências}}\label{referuxeancias}}
\addcontentsline{toc}{chapter}{\textbf{Referências}}

\DisableFootNotes

\hypertarget{refs}{}
\begin{CSLReferences}{0}{0}
\leavevmode\vadjust pre{\hypertarget{ref-ihaka1996}{}}%
\CSLLeftMargin{1. }%
\CSLRightInline{Ihaka R, Gentleman R. R: A language for data analysis and graphics. \emph{Journal of Computational and Graphical Statistics}. 1996;5(3):299. doi:\href{https://doi.org/10.2307/1390807}{10.2307/1390807}}

\leavevmode\vadjust pre{\hypertarget{ref-racine2011}{}}%
\CSLLeftMargin{2. }%
\CSLRightInline{Racine JS. RStudio: A Platform{-}Independent IDE for R and Sweave. \emph{Journal of Applied Econometrics}. 2011;27(1):167-172. doi:\href{https://doi.org/10.1002/jae.1278}{10.1002/jae.1278}}

\leavevmode\vadjust pre{\hypertarget{ref-mair2016}{}}%
\CSLLeftMargin{3. }%
\CSLRightInline{Mair P. Thou shalt be reproducible! A technology perspective. \emph{Frontiers in Psychology}. 2016;7. doi:\href{https://doi.org/10.3389/fpsyg.2016.01079}{10.3389/fpsyg.2016.01079}}

\leavevmode\vadjust pre{\hypertarget{ref-love2019}{}}%
\CSLLeftMargin{4. }%
\CSLRightInline{Love J, Selker R, Marsman M, et al. {\textbf{JASP}}: Graphical Statistical Software for Common Statistical Designs. \emph{Journal of Statistical Software}. 2019;88(2). doi:\href{https://doi.org/10.18637/jss.v088.i02}{10.18637/jss.v088.i02}}

\leavevmode\vadjust pre{\hypertarget{ref-sahin2020}{}}%
\CSLLeftMargin{5. }%
\CSLRightInline{ŞAHİN M, AYBEK E. Jamovi: An easy to use statistical software for the social scientists. \emph{International Journal of Assessment Tools in Education}. 2020;6(4):670-692. doi:\href{https://doi.org/10.21449/ijate.661803}{10.21449/ijate.661803}}

\leavevmode\vadjust pre{\hypertarget{ref-jmv}{}}%
\CSLLeftMargin{6. }%
\CSLRightInline{Selker R, Love J, Dropmann D. Jmv: The 'jamovi' analyses. 2023. \href{https://CRAN.R-project.org/package=jmv}{https://CRAN.R-project.org/package=jmv.}}

\leavevmode\vadjust pre{\hypertarget{ref-jmvconnect}{}}%
\CSLLeftMargin{7. }%
\CSLRightInline{Love J. Jmvconnect: Connect to the 'jamovi' statistical spreadsheet. 2022. \href{https://CRAN.R-project.org/package=jmvconnect}{https://CRAN.R-project.org/package=jmvconnect.}}

\leavevmode\vadjust pre{\hypertarget{ref-hinsen2011}{}}%
\CSLLeftMargin{8. }%
\CSLRightInline{Hinsen K. A data and code model for reproducible research and executable papers. \emph{Procedia Computer Science}. 2011;4:579-588. doi:\href{https://doi.org/10.1016/j.procs.2011.04.061}{10.1016/j.procs.2011.04.061}}

\leavevmode\vadjust pre{\hypertarget{ref-SchwabSimon2021}{}}%
\CSLLeftMargin{9. }%
\CSLRightInline{Schwab, Simon, Held, Leonhard. Statistical programming: Small mistakes, big impacts. \emph{Wiley-Blackwell Publishing, Inc}. 2021. doi:\href{https://doi.org/10.5167/UZH-205154}{10.5167/UZH-205154}}

\leavevmode\vadjust pre{\hypertarget{ref-Eglen2017}{}}%
\CSLLeftMargin{10. }%
\CSLRightInline{Eglen SJ, Marwick B, Halchenko YO, et al. Toward standard practices for sharing computer code and programs in neuroscience. \emph{Nature Neuroscience}. 2017;20(6):770-773. doi:\href{https://doi.org/10.1038/nn.4550}{10.1038/nn.4550}}

\leavevmode\vadjust pre{\hypertarget{ref-formatR}{}}%
\CSLLeftMargin{11. }%
\CSLRightInline{Xie Y. formatR: Format r code automatically. 2022. \href{https://CRAN.R-project.org/package=formatR}{https://CRAN.R-project.org/package=formatR.}}

\leavevmode\vadjust pre{\hypertarget{ref-styler}{}}%
\CSLLeftMargin{12. }%
\CSLRightInline{Müller K, Walthert L. Styler: Non-invasive pretty printing of r code. 2023. \href{https://CRAN.R-project.org/package=styler}{https://CRAN.R-project.org/package=styler.}}

\leavevmode\vadjust pre{\hypertarget{ref-R-rmarkdown}{}}%
\CSLLeftMargin{13. }%
\CSLRightInline{Allaire J, Xie Y, Dervieux C, et al. \emph{Rmarkdown: Dynamic Documents for r}.; 2023. \href{https://CRAN.R-project.org/package=rmarkdown}{https://CRAN.R-project.org/package=rmarkdown.}}

\leavevmode\vadjust pre{\hypertarget{ref-holmes2021}{}}%
\CSLLeftMargin{14. }%
\CSLRightInline{Holmes DT, Mobini M, McCudden CR. Reproducible manuscript preparation with RMarkdown application to JMSACL and other Elsevier Journals. \emph{Journal of Mass Spectrometry and Advances in the Clinical Lab}. 2021;22:8-16. doi:\href{https://doi.org/10.1016/j.jmsacl.2021.09.002}{10.1016/j.jmsacl.2021.09.002}}

\leavevmode\vadjust pre{\hypertarget{ref-rmarkdown}{}}%
\CSLLeftMargin{15. }%
\CSLRightInline{Allaire J, Xie Y, Dervieux C, et al. Rmarkdown: Dynamic documents for r. 2023. \href{https://github.com/rstudio/rmarkdown}{https://github.com/rstudio/rmarkdown.}}

\leavevmode\vadjust pre{\hypertarget{ref-officedown}{}}%
\CSLLeftMargin{16. }%
\CSLRightInline{Gohel D, Ross N. Officedown: Enhanced 'r markdown' format for 'word' and 'PowerPoint'. 2023. \href{https://CRAN.R-project.org/package=officedown}{https://CRAN.R-project.org/package=officedown.}}

\leavevmode\vadjust pre{\hypertarget{ref-bookdown}{}}%
\CSLLeftMargin{17. }%
\CSLRightInline{Xie Y. Bookdown: Authoring books and technical documents with r markdown. 2023. \href{https://github.com/rstudio/bookdown}{https://github.com/rstudio/bookdown.}}

\leavevmode\vadjust pre{\hypertarget{ref-ioannidis2014}{}}%
\CSLLeftMargin{18. }%
\CSLRightInline{Ioannidis JPA. How to Make More Published Research True. \emph{PLoS Medicine}. 2014;11(10):e1001747. doi:\href{https://doi.org/10.1371/journal.pmed.1001747}{10.1371/journal.pmed.1001747}}

\leavevmode\vadjust pre{\hypertarget{ref-projects}{}}%
\CSLLeftMargin{19. }%
\CSLRightInline{Krieger N, Perzynski A, Dalton J. Projects: A project infrastructure for researchers. 2021. \href{https://CRAN.R-project.org/package=projects}{https://CRAN.R-project.org/package=projects.}}

\leavevmode\vadjust pre{\hypertarget{ref-grami2023}{}}%
\CSLLeftMargin{20. }%
\CSLRightInline{Grami A. Discrete probability. In: Elsevier; 2023:285-305. doi:\href{https://doi.org/10.1016/b978-0-12-820656-0.00016-2}{10.1016/b978-0-12-820656-0.00016-2}}

\leavevmode\vadjust pre{\hypertarget{ref-tversky1971}{}}%
\CSLLeftMargin{21. }%
\CSLRightInline{Tversky A, Kahneman D. Belief in the law of small numbers. \emph{Psychological Bulletin}. 1971;76(2):105-110. doi:\href{https://doi.org/10.1037/h0031322}{10.1037/h0031322}}

\leavevmode\vadjust pre{\hypertarget{ref-bishop2022}{}}%
\CSLLeftMargin{22. }%
\CSLRightInline{Bishop DVM, Thompson J, Parker AJ. Can we shift belief in the {`}Law of Small Numbers{'}? \emph{Royal Society Open Science}. 2022;9(3). doi:\href{https://doi.org/10.1098/rsos.211028}{10.1098/rsos.211028}}

\leavevmode\vadjust pre{\hypertarget{ref-guy1988}{}}%
\CSLLeftMargin{23. }%
\CSLRightInline{Guy RK. The strong law of small numbers. \emph{The American Mathematical Monthly}. 1988;95(8):697. doi:\href{https://doi.org/10.2307/2322249}{10.2307/2322249}}

\leavevmode\vadjust pre{\hypertarget{ref-guy1990}{}}%
\CSLLeftMargin{24. }%
\CSLRightInline{Guy RK. The Second Strong Law of Small Numbers. \emph{Mathematics Magazine}. 1990;63(1):3-20. doi:\href{https://doi.org/10.1080/0025570x.1990.11977475}{10.1080/0025570x.1990.11977475}}

\leavevmode\vadjust pre{\hypertarget{ref-munafuxf22017}{}}%
\CSLLeftMargin{25. }%
\CSLRightInline{Munafò MR, Nosek BA, Bishop DVM, et al. A manifesto for reproducible science. \emph{Nature Human Behaviour}. 2017;1(1). doi:\href{https://doi.org/10.1038/s41562-016-0021}{10.1038/s41562-016-0021}}

\leavevmode\vadjust pre{\hypertarget{ref-Banerjee2010}{}}%
\CSLLeftMargin{26. }%
\CSLRightInline{Banerjee A, Chaudhury S. Statistics without tears: Populations and samples. \emph{Industrial Psychiatry Journal}. 2010;19(1):60. doi:\href{https://doi.org/10.4103/0972-6748.77642}{10.4103/0972-6748.77642}}

\leavevmode\vadjust pre{\hypertarget{ref-Bland2015}{}}%
\CSLLeftMargin{27. }%
\CSLRightInline{Bland JM, Altman DG. Statistics Notes: Bootstrap resampling methods. \emph{BMJ}. 2015;350(jun02 13):h2622-h2622. doi:\href{https://doi.org/10.1136/bmj.h2622}{10.1136/bmj.h2622}}

\leavevmode\vadjust pre{\hypertarget{ref-Altman1997}{}}%
\CSLLeftMargin{28. }%
\CSLRightInline{Altman DG, Bland JM. Statistics Notes: Units of analysis. \emph{BMJ}. 1997;314(7098):1874-1874. doi:\href{https://doi.org/10.1136/bmj.314.7098.1874}{10.1136/bmj.314.7098.1874}}

\leavevmode\vadjust pre{\hypertarget{ref-Matthews1990}{}}%
\CSLLeftMargin{29. }%
\CSLRightInline{Matthews JN, Altman DG, Campbell MJ, Royston P. Analysis of serial measurements in medical research. \emph{BMJ}. 1990;300(6719):230-235. doi:\href{https://doi.org/10.1136/bmj.300.6719.230}{10.1136/bmj.300.6719.230}}

\leavevmode\vadjust pre{\hypertarget{ref-resnik2016}{}}%
\CSLLeftMargin{30. }%
\CSLRightInline{Resnik DB, Shamoo AE. Reproducibility and Research Integrity. \emph{Accountability in Research}. 2016;24(2):116-123. doi:\href{https://doi.org/10.1080/08989621.2016.1257387}{10.1080/08989621.2016.1257387}}

\leavevmode\vadjust pre{\hypertarget{ref-hofner2015}{}}%
\CSLLeftMargin{31. }%
\CSLRightInline{Hofner B, Schmid M, Edler L. Reproducible research in statistics: A review and guidelines for the {\emph{Biometrical Journal}}. \emph{Biometrical Journal}. 2015;58(2):416-427. doi:\href{https://doi.org/10.1002/bimj.201500156}{10.1002/bimj.201500156}}

\leavevmode\vadjust pre{\hypertarget{ref-base-5}{}}%
\CSLLeftMargin{32. }%
\CSLRightInline{R Core Team. R: A language and environment for statistical computing. 2023. \href{https://www.R-project.org/}{https://www.R-project.org/.}}

\leavevmode\vadjust pre{\hypertarget{ref-bookdown-2}{}}%
\CSLLeftMargin{33. }%
\CSLRightInline{Xie Y. Bookdown: Authoring books and technical documents with r markdown. 2023. \href{https://github.com/rstudio/bookdown}{https://github.com/rstudio/bookdown.}}

\leavevmode\vadjust pre{\hypertarget{ref-synthpop}{}}%
\CSLLeftMargin{34. }%
\CSLRightInline{Nowok B, Raab GM, Dibben C. {\textbraceleft}Synthpop{\textbraceright}: Bespoke creation of synthetic data in {\textbraceleft}r{\textbraceright}. 2016;74. doi:\href{https://doi.org/10.18637/jss.v074.i11}{10.18637/jss.v074.i11}}

\leavevmode\vadjust pre{\hypertarget{ref-grateful}{}}%
\CSLLeftMargin{35. }%
\CSLRightInline{Francisco Rodríguez-Sánchez, Connor P. Jackson, Shaurita D. Hutchins. Grateful: Facilitate citation of r packages. 2023. \href{https://github.com/Pakillo/grateful}{https://github.com/Pakillo/grateful.}}

\leavevmode\vadjust pre{\hypertarget{ref-Zhao2023}{}}%
\CSLLeftMargin{36. }%
\CSLRightInline{Zhao Y, Xiao N, Anderson K, Zhang Y. Electronic common technical document submission with analysis using R. \emph{Clinical Trials}. 2022;20(1):89-92. doi:\href{https://doi.org/10.1177/17407745221123244}{10.1177/17407745221123244}}

\leavevmode\vadjust pre{\hypertarget{ref-utils}{}}%
\CSLLeftMargin{37. }%
\CSLRightInline{R Core Team. R: A language and environment for statistical computing. 2023. \href{https://www.R-project.org/}{https://www.R-project.org/.}}

\leavevmode\vadjust pre{\hypertarget{ref-abelson1985}{}}%
\CSLLeftMargin{38. }%
\CSLRightInline{Abelson RP. A variance explanation paradox: When a little is a lot. \emph{Psychological Bulletin}. 1985;97(1):129-133. doi:\href{https://doi.org/10.1037/0033-2909.97.1.129}{10.1037/0033-2909.97.1.129}}

\leavevmode\vadjust pre{\hypertarget{ref-berkson1946}{}}%
\CSLLeftMargin{39. }%
\CSLRightInline{Berkson J. Limitations of the application of fourfold table analysis to hospital data. \emph{Biometrics Bulletin}. 1946;2(3):47. doi:\href{https://doi.org/10.2307/3002000}{10.2307/3002000}}

\leavevmode\vadjust pre{\hypertarget{ref-ellsberg1961}{}}%
\CSLLeftMargin{40. }%
\CSLRightInline{Ellsberg D. Risk, ambiguity, and the savage axioms. \emph{The Quarterly Journal of Economics}. 1961;75(4):643. doi:\href{https://doi.org/10.2307/1884324}{10.2307/1884324}}

\leavevmode\vadjust pre{\hypertarget{ref-freedman1983}{}}%
\CSLLeftMargin{41. }%
\CSLRightInline{Freedman DA, Freedman DA. A Note on Screening Regression Equations. \emph{The American Statistician}. 1983;37(2):152-155. doi:\href{https://doi.org/10.1080/00031305.1983.10482729}{10.1080/00031305.1983.10482729}}

\leavevmode\vadjust pre{\hypertarget{ref-freedman1989}{}}%
\CSLLeftMargin{42. }%
\CSLRightInline{Freedman LS, Pee D. Return to a note on screening regression equations. \emph{The American Statistician}. 1989;43(4):279. doi:\href{https://doi.org/10.2307/2685389}{10.2307/2685389}}

\leavevmode\vadjust pre{\hypertarget{ref-hand1992}{}}%
\CSLLeftMargin{43. }%
\CSLRightInline{Hand DJ. On Comparing Two Treatments. \emph{The American Statistician}. 1992;46(3):190-192. doi:\href{https://doi.org/10.1080/00031305.1992.10475881}{10.1080/00031305.1992.10475881}}

\leavevmode\vadjust pre{\hypertarget{ref-lindley1957}{}}%
\CSLLeftMargin{44. }%
\CSLRightInline{LINDLEY DV. A STATISTICAL PARADOX. \emph{Biometrika}. 1957;44(1-2):187-192. doi:\href{https://doi.org/10.1093/biomet/44.1-2.187}{10.1093/biomet/44.1-2.187}}

\leavevmode\vadjust pre{\hypertarget{ref-lord1967}{}}%
\CSLLeftMargin{45. }%
\CSLRightInline{Lord FM. A paradox in the interpretation of group comparisons. \emph{Psychological Bulletin}. 1967;68(5):304-305. doi:\href{https://doi.org/10.1037/h0025105}{10.1037/h0025105}}

\leavevmode\vadjust pre{\hypertarget{ref-lord1969}{}}%
\CSLLeftMargin{46. }%
\CSLRightInline{Lord FM. Statistical adjustments when comparing preexisting groups. \emph{Psychological Bulletin}. 1969;72(5):336-337. doi:\href{https://doi.org/10.1037/h0028108}{10.1037/h0028108}}

\leavevmode\vadjust pre{\hypertarget{ref-simpson1951}{}}%
\CSLLeftMargin{47. }%
\CSLRightInline{Simpson EH. The Interpretation of Interaction in Contingency Tables. \emph{Journal of the Royal Statistical Society: Series B (Methodological)}. 1951;13(2):238-241. doi:\href{https://doi.org/10.1111/j.2517-6161.1951.tb00088.x}{10.1111/j.2517-6161.1951.tb00088.x}}

\leavevmode\vadjust pre{\hypertarget{ref-blyth1972}{}}%
\CSLLeftMargin{48. }%
\CSLRightInline{Blyth CR. On Simpson's Paradox and the Sure-Thing Principle. \emph{Journal of the American Statistical Association}. 1972;67(338):364-366. doi:\href{https://doi.org/10.1080/01621459.1972.10482387}{10.1080/01621459.1972.10482387}}

\leavevmode\vadjust pre{\hypertarget{ref-stein1956}{}}%
\CSLLeftMargin{49. }%
\CSLRightInline{Stein C. INADMISSIBILITY OF THE USUAL ESTIMATOR FOR THE MEAN OF a MULTIVARIATE NORMAL DISTRIBUTION. In: University of California Press; 1956:197-206. doi:\href{https://doi.org/10.1525/9780520313880-018}{10.1525/9780520313880-018}}

\leavevmode\vadjust pre{\hypertarget{ref-de1996}{}}%
\CSLLeftMargin{50. }%
\CSLRightInline{De S, Sen A. The generalised Gamow-Stern problem. \emph{The Mathematical Gazette}. 1996;80(488):345-348. doi:\href{https://doi.org/10.2307/3619568}{10.2307/3619568}}

\leavevmode\vadjust pre{\hypertarget{ref-feld1991}{}}%
\CSLLeftMargin{51. }%
\CSLRightInline{Feld SL. Why Your Friends Have More Friends Than You Do. \emph{American Journal of Sociology}. 1991;96(6):1464-1477. doi:\href{https://doi.org/10.1086/229693}{10.1086/229693}}

\leavevmode\vadjust pre{\hypertarget{ref-stats-2}{}}%
\CSLLeftMargin{52. }%
\CSLRightInline{R Core Team. R: A language and environment for statistical computing. 2023. \href{https://www.R-project.org/}{https://www.R-project.org/.}}

\leavevmode\vadjust pre{\hypertarget{ref-Olson2021}{}}%
\CSLLeftMargin{53. }%
\CSLRightInline{Olson K. What Are Data? \emph{Qualitative Health Research}. 2021;31(9):1567-1569. doi:\href{https://doi.org/10.1177/10497323211015960}{10.1177/10497323211015960}}

\leavevmode\vadjust pre{\hypertarget{ref-van2022a}{}}%
\CSLLeftMargin{54. }%
\CSLRightInline{Smeden M van. A very short list of common pitfalls in research design, data analysis, and reporting. \emph{PRiMER}. 2022;6. doi:\href{https://doi.org/10.22454/PRiMER.2022.511416}{10.22454/PRiMER.2022.511416}}

\leavevmode\vadjust pre{\hypertarget{ref-vetter2017}{}}%
\CSLLeftMargin{55. }%
\CSLRightInline{Vetter TR. Fundamentals of Research Data and Variables. \emph{Anesthesia \& Analgesia}. 2017;125(4):1375-1380. doi:\href{https://doi.org/10.1213/ane.0000000000002370}{10.1213/ane.0000000000002370}}

\leavevmode\vadjust pre{\hypertarget{ref-Altman2007}{}}%
\CSLLeftMargin{56. }%
\CSLRightInline{Altman DG, Bland JM. Missing data. \emph{BMJ}. 2007;334(7590):424-424. doi:\href{https://doi.org/10.1136/bmj.38977.682025.2c}{10.1136/bmj.38977.682025.2c}}

\leavevmode\vadjust pre{\hypertarget{ref-base-2}{}}%
\CSLLeftMargin{57. }%
\CSLRightInline{R Core Team. R: A language and environment for statistical computing. 2023. \href{https://www.R-project.org/}{https://www.R-project.org/.}}

\leavevmode\vadjust pre{\hypertarget{ref-Heymans2022}{}}%
\CSLLeftMargin{58. }%
\CSLRightInline{Heymans MW, Twisk JWR. Handling missing data in clinical research. \emph{Journal of Clinical Epidemiology}. September 2022. doi:\href{https://doi.org/10.1016/j.jclinepi.2022.08.016}{10.1016/j.jclinepi.2022.08.016}}

\leavevmode\vadjust pre{\hypertarget{ref-carpenter2021}{}}%
\CSLLeftMargin{59. }%
\CSLRightInline{Carpenter JR, Smuk M. Missing data: A statistical framework for practice. \emph{Biometrical Journal}. 2021;63(5):915-947. doi:\href{https://doi.org/10.1002/bimj.202000196}{10.1002/bimj.202000196}}

\leavevmode\vadjust pre{\hypertarget{ref-misty}{}}%
\CSLLeftMargin{60. }%
\CSLRightInline{Yanagida T. Misty: Miscellaneous functions 't. yanagida'. 2023. \href{https://CRAN.R-project.org/package=misty}{https://CRAN.R-project.org/package=misty.}}

\leavevmode\vadjust pre{\hypertarget{ref-little1988}{}}%
\CSLLeftMargin{61. }%
\CSLRightInline{Little RJA. A Test of Missing Completely at Random for Multivariate Data with Missing Values. \emph{Journal of the American Statistical Association}. 1988;83(404):1198-1202. doi:\href{https://doi.org/10.1080/01621459.1988.10478722}{10.1080/01621459.1988.10478722}}

\leavevmode\vadjust pre{\hypertarget{ref-stats}{}}%
\CSLLeftMargin{62. }%
\CSLRightInline{R Core Team. R: A language and environment for statistical computing. 2022. \href{https://www.R-project.org/}{https://www.R-project.org/.}}

\leavevmode\vadjust pre{\hypertarget{ref-austin2023}{}}%
\CSLLeftMargin{63. }%
\CSLRightInline{Austin PC, Buuren S van. Logistic regression vs. predictive mean matching for imputing binary covariates. \emph{Statistical Methods in Medical Research}. September 2023. doi:\href{https://doi.org/10.1177/09622802231198795}{10.1177/09622802231198795}}

\leavevmode\vadjust pre{\hypertarget{ref-mice}{}}%
\CSLLeftMargin{64. }%
\CSLRightInline{Buuren S van, Groothuis-Oudshoorn K. {\textbraceleft}Mice{\textbraceright}: Multivariate imputation by chained equations in r. 2011;45:1-67. doi:\href{https://doi.org/10.18637/jss.v045.i03}{10.18637/jss.v045.i03}}

\leavevmode\vadjust pre{\hypertarget{ref-rubin1986}{}}%
\CSLLeftMargin{65. }%
\CSLRightInline{Rubin DB. Statistical matching using file concatenation with adjusted weights and multiple imputations. \emph{Journal of Business \& Economic Statistics}. 1986;4(1):87. doi:\href{https://doi.org/10.2307/1391390}{10.2307/1391390}}

\leavevmode\vadjust pre{\hypertarget{ref-little1988a}{}}%
\CSLLeftMargin{66. }%
\CSLRightInline{Little RJA. Missing-Data Adjustments in Large Surveys. \emph{Journal of Business \& Economic Statistics}. 1988;6(3):287-296. doi:\href{https://doi.org/10.1080/07350015.1988.10509663}{10.1080/07350015.1988.10509663}}

\leavevmode\vadjust pre{\hypertarget{ref-miceadds}{}}%
\CSLLeftMargin{67. }%
\CSLRightInline{Robitzsch A, Grund S. Miceadds: Some additional multiple imputation functions, especially for 'mice'. 2023. \href{https://CRAN.R-project.org/package=miceadds}{https://CRAN.R-project.org/package=miceadds.}}

\leavevmode\vadjust pre{\hypertarget{ref-Akl2015}{}}%
\CSLLeftMargin{68. }%
\CSLRightInline{Akl EA, Shawwa K, Kahale LA, et al. Reporting missing participant data in randomised trials: systematic survey of the methodological literature and a proposed guide. \emph{BMJ Open}. 2015;5(12):e008431. doi:\href{https://doi.org/10.1136/bmjopen-2015-008431}{10.1136/bmjopen-2015-008431}}

\leavevmode\vadjust pre{\hypertarget{ref-ids}{}}%
\CSLLeftMargin{69. }%
\CSLRightInline{FitzJohn R. Ids: Generate random identifiers. 2017. \href{https://CRAN.R-project.org/package=ids}{https://CRAN.R-project.org/package=ids.}}

\leavevmode\vadjust pre{\hypertarget{ref-hash}{}}%
\CSLLeftMargin{70. }%
\CSLRightInline{Brown C. Hash: Full featured implementation of hash tables/associative arrays/dictionaries. 2023. \href{https://CRAN.R-project.org/package=hash}{https://CRAN.R-project.org/package=hash.}}

\leavevmode\vadjust pre{\hypertarget{ref-anonymizer}{}}%
\CSLLeftMargin{71. }%
\CSLRightInline{Hendricks P. Anonymizer: Anonymize data containing personally identifiable information. 2023. \href{https://github.com/paulhendricks/anonymizer}{https://github.com/paulhendricks/anonymizer.}}

\leavevmode\vadjust pre{\hypertarget{ref-digest}{}}%
\CSLLeftMargin{72. }%
\CSLRightInline{Lucas DE with contributions by A, Tuszynski J, Bengtsson H, et al. Digest: Create compact hash digests of r objects. 2023. \href{https://CRAN.R-project.org/package=digest}{https://CRAN.R-project.org/package=digest.}}

\leavevmode\vadjust pre{\hypertarget{ref-Baillie2022}{}}%
\CSLLeftMargin{73. }%
\CSLRightInline{Baillie M, Cessie S le, Schmidt CO, Lusa L, Huebner M. Ten simple rules for initial data analysis. \emph{PLOS Computational Biology}. 2022;18(2):e1009819. doi:\href{https://doi.org/10.1371/journal.pcbi.1009819}{10.1371/journal.pcbi.1009819}}

\leavevmode\vadjust pre{\hypertarget{ref-buttliere2021}{}}%
\CSLLeftMargin{74. }%
\CSLRightInline{Buttliere B. Adopting standard variable labels solves many of the problems with sharing and reusing data. \emph{Methodological Innovations}. 2021;14(2):205979912110266. doi:\href{https://doi.org/10.1177/20597991211026616}{10.1177/20597991211026616}}

\leavevmode\vadjust pre{\hypertarget{ref-base-3}{}}%
\CSLLeftMargin{75. }%
\CSLRightInline{R Core Team. R: A language and environment for statistical computing. 2023. \href{https://www.R-project.org/}{https://www.R-project.org/.}}

\leavevmode\vadjust pre{\hypertarget{ref-units}{}}%
\CSLLeftMargin{76. }%
\CSLRightInline{Pebesma E, Mailund T, Hiebert J. Measurement units in {\textbraceleft}r{\textbraceright}. 2016;8. doi:\href{https://doi.org/10.32614/RJ-2016-061}{10.32614/RJ-2016-061}}

\leavevmode\vadjust pre{\hypertarget{ref-janitor}{}}%
\CSLLeftMargin{77. }%
\CSLRightInline{Firke S. Janitor: Simple tools for examining and cleaning dirty data. 2023. \href{https://CRAN.R-project.org/package=janitor}{https://CRAN.R-project.org/package=janitor.}}

\leavevmode\vadjust pre{\hypertarget{ref-Hmisc}{}}%
\CSLLeftMargin{78. }%
\CSLRightInline{Harrell Jr FE. Hmisc: Harrell miscellaneous. 2023. \href{https://CRAN.R-project.org/package=Hmisc}{https://CRAN.R-project.org/package=Hmisc.}}

\leavevmode\vadjust pre{\hypertarget{ref-tierney2023}{}}%
\CSLLeftMargin{79. }%
\CSLRightInline{Tierney N, Cook D. Expanding Tidy Data Principles to Facilitate Missing Data Exploration, Visualization and Assessment of Imputations. \emph{Journal of Statistical Software}. 2023;105(7). doi:\href{https://doi.org/10.18637/jss.v105.i07}{10.18637/jss.v105.i07}}

\leavevmode\vadjust pre{\hypertarget{ref-DataEditR}{}}%
\CSLLeftMargin{80. }%
\CSLRightInline{Hammill D. DataEditR: An interactive editor for viewing, entering, filtering \& editing data. 2022. \href{https://CRAN.R-project.org/package=DataEditR}{https://CRAN.R-project.org/package=DataEditR.}}

\leavevmode\vadjust pre{\hypertarget{ref-broman2018}{}}%
\CSLLeftMargin{81. }%
\CSLRightInline{Broman KW, Woo KH. Data Organization in Spreadsheets. \emph{The American Statistician}. 2018;72(1):2-10. doi:\href{https://doi.org/10.1080/00031305.2017.1375989}{10.1080/00031305.2017.1375989}}

\leavevmode\vadjust pre{\hypertarget{ref-Juluru2015}{}}%
\CSLLeftMargin{82. }%
\CSLRightInline{Juluru K, Eng J. Use of Spreadsheets for Research Data Collection and Preparation: \emph{Academic Radiology}. 2015;22(12):1592-1599. doi:\href{https://doi.org/10.1016/j.acra.2015.08.024}{10.1016/j.acra.2015.08.024}}

\leavevmode\vadjust pre{\hypertarget{ref-data.table}{}}%
\CSLLeftMargin{83. }%
\CSLRightInline{Dowle M, Srinivasan A. Data.table: Extension of `data.frame`. 2023. \href{https://CRAN.R-project.org/package=data.table}{https://CRAN.R-project.org/package=data.table.}}

\leavevmode\vadjust pre{\hypertarget{ref-Altman1999}{}}%
\CSLLeftMargin{84. }%
\CSLRightInline{Altman DG, Bland JM. Statistics notes Variables and parameters. \emph{BMJ}. 1999;318(7199):1667-1667. doi:\href{https://doi.org/10.1136/bmj.318.7199.1667}{10.1136/bmj.318.7199.1667}}

\leavevmode\vadjust pre{\hypertarget{ref-Ali2016}{}}%
\CSLLeftMargin{85. }%
\CSLRightInline{Ali Z, Bhaskar Sb. Basic statistical tools in research and data analysis. \emph{Indian Journal of Anaesthesia}. 2016;60(9):662. doi:\href{https://doi.org/10.4103/0019-5049.190623}{10.4103/0019-5049.190623}}

\leavevmode\vadjust pre{\hypertarget{ref-Dettori2018}{}}%
\CSLLeftMargin{86. }%
\CSLRightInline{Dettori JR, Norvell DC. The Anatomy of Data. \emph{Global Spine Journal}. 2018;8(3):311-313. doi:\href{https://doi.org/10.1177/2192568217746998}{10.1177/2192568217746998}}

\leavevmode\vadjust pre{\hypertarget{ref-kaliyadan2019}{}}%
\CSLLeftMargin{87. }%
\CSLRightInline{Kaliyadan F, Kulkarni V. Types of variables, descriptive statistics, and sample size. \emph{Indian Dermatology Online Journal}. 2019;10(1):82. doi:\href{https://doi.org/10.4103/idoj.idoj_468_18}{10.4103/idoj.idoj\_468\_18}}

\leavevmode\vadjust pre{\hypertarget{ref-barkan2015}{}}%
\CSLLeftMargin{88. }%
\CSLRightInline{Barkan H. Statistics in clinical research: Important considerations. \emph{Annals of Cardiac Anaesthesia}. 2015;18(1):74. doi:\href{https://doi.org/10.4103/0971-9784.148325}{10.4103/0971-9784.148325}}

\leavevmode\vadjust pre{\hypertarget{ref-Bland1996}{}}%
\CSLLeftMargin{89. }%
\CSLRightInline{Bland JM, Altman DG. Statistics Notes: Transforming data. \emph{BMJ}. 1996;312(7033):770-770. doi:\href{https://doi.org/10.1136/bmj.312.7033.770}{10.1136/bmj.312.7033.770}}

\leavevmode\vadjust pre{\hypertarget{ref-Fedorov2009}{}}%
\CSLLeftMargin{90. }%
\CSLRightInline{Fedorov V, Mannino F, Zhang R. Consequences of dichotomization. \emph{Pharmaceutical Statistics}. 2009;8(1):50-61. doi:\href{https://doi.org/10.1002/pst.331}{10.1002/pst.331}}

\leavevmode\vadjust pre{\hypertarget{ref-osborne2010}{}}%
\CSLLeftMargin{91. }%
\CSLRightInline{Osborne J. Improving your data transformations: Applying the box-cox transformation. \emph{University of Massachusetts Amherst}. 2010. doi:\href{https://doi.org/10.7275/QBPC-GK17}{10.7275/QBPC-GK17}}

\leavevmode\vadjust pre{\hypertarget{ref-box1964}{}}%
\CSLLeftMargin{92. }%
\CSLRightInline{Box GEP, Cox DR. An Analysis of Transformations. \emph{Journal of the Royal Statistical Society: Series B (Methodological)}. 1964;26(2):211-243. doi:\href{https://doi.org/10.1111/j.2517-6161.1964.tb00553.x}{10.1111/j.2517-6161.1964.tb00553.x}}

\leavevmode\vadjust pre{\hypertarget{ref-MASS}{}}%
\CSLLeftMargin{93. }%
\CSLRightInline{Venables WN, Ripley BD. Modern applied statistics with s. 2002. \href{https://www.stats.ox.ac.uk/pub/MASS4/}{https://www.stats.ox.ac.uk/pub/MASS4/.}}

\leavevmode\vadjust pre{\hypertarget{ref-MacCallum2002}{}}%
\CSLLeftMargin{94. }%
\CSLRightInline{MacCallum RC, Zhang S, Preacher KJ, Rucker DD. On the practice of dichotomization of quantitative variables. \emph{Psychological Methods}. 2002;7(1):19-40. doi:\href{https://doi.org/10.1037/1082-989x.7.1.19}{10.1037/1082-989x.7.1.19}}

\leavevmode\vadjust pre{\hypertarget{ref-Altman2006}{}}%
\CSLLeftMargin{95. }%
\CSLRightInline{Altman DG, Royston P. The cost of dichotomising continuous variables. \emph{BMJ}. 2006;332(7549):1080.1. doi:\href{https://doi.org/10.1136/bmj.332.7549.1080}{10.1136/bmj.332.7549.1080}}

\leavevmode\vadjust pre{\hypertarget{ref-Royston2006}{}}%
\CSLLeftMargin{96. }%
\CSLRightInline{Royston P, Altman DG, Sauerbrei W. Dichotomizing continuous predictors in multiple regression: a bad idea. \emph{Statistics in Medicine}. 2005;25(1):127-141. doi:\href{https://doi.org/10.1002/sim.2331}{10.1002/sim.2331}}

\leavevmode\vadjust pre{\hypertarget{ref-Collins2016}{}}%
\CSLLeftMargin{97. }%
\CSLRightInline{Collins GS, Ogundimu EO, Cook JA, Manach YL, Altman DG. Quantifying the impact of different approaches for handling continuous predictors on the performance of a prognostic model. \emph{Statistics in Medicine}. 2016;35(23):4124-4135. doi:\href{https://doi.org/10.1002/sim.6986}{10.1002/sim.6986}}

\leavevmode\vadjust pre{\hypertarget{ref-Prince2017}{}}%
\CSLLeftMargin{98. }%
\CSLRightInline{Nelson SLP, Ramakrishnan V, Nietert PJ, Kamen DL, Ramos PS, Wolf BJ. An evaluation of common methods for dichotomization of continuous variables to discriminate disease status. \emph{Communications in Statistics - Theory and Methods}. 2017;46(21):10823-10834. doi:\href{https://doi.org/10.1080/03610926.2016.1248783}{10.1080/03610926.2016.1248783}}

\leavevmode\vadjust pre{\hypertarget{ref-Bennette2012}{}}%
\CSLLeftMargin{99. }%
\CSLRightInline{Bennette C, Vickers A. Against quantiles: categorization of continuous variables in epidemiologic research, and its discontents. \emph{BMC Medical Research Methodology}. 2012;12(1). doi:\href{https://doi.org/10.1186/1471-2288-12-21}{10.1186/1471-2288-12-21}}

\leavevmode\vadjust pre{\hypertarget{ref-YOUDEN1950}{}}%
\CSLLeftMargin{100. }%
\CSLRightInline{Youden WJ. Index for rating diagnostic tests. \emph{Cancer}. 1950;3(1):32-35. doi:\href{https://doi.org/10.1002/1097-0142(1950)3:1\%3C32::aid-cncr2820030106\%3E3.0.co;2-3}{10.1002/1097-0142(1950)3:1\textless32::aid-cncr2820030106\textgreater3.0.co;2-3}}

\leavevmode\vadjust pre{\hypertarget{ref-strobl2007}{}}%
\CSLLeftMargin{101. }%
\CSLRightInline{Strobl C, Boulesteix AL, Augustin T. Unbiased split selection for classification trees based on the Gini Index. \emph{Computational Statistics \& Data Analysis}. 2007;52(1):483-501. doi:\href{https://doi.org/10.1016/j.csda.2006.12.030}{10.1016/j.csda.2006.12.030}}

\leavevmode\vadjust pre{\hypertarget{ref-pearson1900}{}}%
\CSLLeftMargin{102. }%
\CSLRightInline{Pearson K. X. {\emph{On the criterion that a given system of deviations from the probable in the case of a correlated system of variables is such that it can be reasonably supposed to have arisen from random sampling}}. \emph{The London, Edinburgh, and Dublin Philosophical Magazine and Journal of Science}. 1900;50(302):157-175. doi:\href{https://doi.org/10.1080/14786440009463897}{10.1080/14786440009463897}}

\leavevmode\vadjust pre{\hypertarget{ref-Greiner2000}{}}%
\CSLLeftMargin{103. }%
\CSLRightInline{Greiner M, Pfeiffer D, Smith RD. Principles and practical application of the receiver-operating characteristic analysis for diagnostic tests. \emph{Preventive Veterinary Medicine}. 2000;45(1-2):23-41. doi:\href{https://doi.org/10.1016/s0167-5877(00)00115-x}{10.1016/s0167-5877(00)00115-x}}

\leavevmode\vadjust pre{\hypertarget{ref-fleiss1971}{}}%
\CSLLeftMargin{104. }%
\CSLRightInline{Fleiss JL. Measuring nominal scale agreement among many raters. \emph{Psychological Bulletin}. 1971;76(5):378-382. doi:\href{https://doi.org/10.1037/h0031619}{10.1037/h0031619}}

\leavevmode\vadjust pre{\hypertarget{ref-stats-3}{}}%
\CSLLeftMargin{105. }%
\CSLRightInline{R Core Team. R: A language and environment for statistical computing. 2023. \href{https://www.R-project.org/}{https://www.R-project.org/.}}

\leavevmode\vadjust pre{\hypertarget{ref-kanji2006}{}}%
\CSLLeftMargin{106. }%
\CSLRightInline{Kanji G. 100 statistical tests. 2006. doi:\href{https://doi.org/10.4135/9781849208499}{10.4135/9781849208499}}

\leavevmode\vadjust pre{\hypertarget{ref-Curran-Everett2008}{}}%
\CSLLeftMargin{107. }%
\CSLRightInline{Curran-Everett D. Explorations in statistics: standard deviations and standard errors. \emph{Advances in Physiology Education}. 2008;32(3):203-208. doi:\href{https://doi.org/10.1152/advan.90123.2008}{10.1152/advan.90123.2008}}

\leavevmode\vadjust pre{\hypertarget{ref-Altman1994}{}}%
\CSLLeftMargin{108. }%
\CSLRightInline{Altman DG, Bland JM. Statistics Notes: Quartiles, quintiles, centiles, and other quantiles. \emph{BMJ}. 1994;309(6960):996-996. doi:\href{https://doi.org/10.1136/bmj.309.6960.996}{10.1136/bmj.309.6960.996}}

\leavevmode\vadjust pre{\hypertarget{ref-greenhalgh1997}{}}%
\CSLLeftMargin{109. }%
\CSLRightInline{Greenhalgh T. How to read a paper: Statistics for the non-statistician. I: Different types of data need different statistical tests. \emph{BMJ}. 1997;315(7104):364-366. doi:\href{https://doi.org/10.1136/bmj.315.7104.364}{10.1136/bmj.315.7104.364}}

\leavevmode\vadjust pre{\hypertarget{ref-base}{}}%
\CSLLeftMargin{110. }%
\CSLRightInline{R Core Team. \emph{{R}: A Language and Environment for Statistical Computing}. Vienna, Austria: R Foundation for Statistical Computing; 2023. \href{https://www.R-project.org/}{https://www.R-project.org/.}}

\leavevmode\vadjust pre{\hypertarget{ref-zuur2009}{}}%
\CSLLeftMargin{111. }%
\CSLRightInline{Zuur AF, Ieno EN, Elphick CS. A protocol for data exploration to avoid common statistical problems. \emph{Methods in Ecology and Evolution}. 2009;1(1):3-14. doi:\href{https://doi.org/10.1111/j.2041-210x.2009.00001.x}{10.1111/j.2041-210x.2009.00001.x}}

\leavevmode\vadjust pre{\hypertarget{ref-outliers}{}}%
\CSLLeftMargin{112. }%
\CSLRightInline{Komsta L. Outliers: Tests for outliers. 2022. \href{https://CRAN.R-project.org/package=outliers}{https://CRAN.R-project.org/package=outliers.}}

\leavevmode\vadjust pre{\hypertarget{ref-chatfield1986}{}}%
\CSLLeftMargin{113. }%
\CSLRightInline{Chatfield C. Exploratory data analysis. \emph{European Journal of Operational Research}. 1986;23(1):5-13. doi:\href{https://doi.org/10.1016/0377-2217(86)90209-2}{10.1016/0377-2217(86)90209-2}}

\leavevmode\vadjust pre{\hypertarget{ref-Ferketich1986}{}}%
\CSLLeftMargin{114. }%
\CSLRightInline{Ferketich S, Verran J. Technical Notes. \emph{Western Journal of Nursing Research}. 1986;8(4):464-466. doi:\href{https://doi.org/10.1177/019394598600800409}{10.1177/019394598600800409}}

\leavevmode\vadjust pre{\hypertarget{ref-Kerr1998}{}}%
\CSLLeftMargin{115. }%
\CSLRightInline{Kerr NL. HARKing: Hypothesizing After the Results are Known. \emph{Personality and Social Psychology Review}. 1998;2(3):196-217. doi:\href{https://doi.org/10.1207/s15327957pspr0203_4}{10.1207/s15327957pspr0203\_4}}

\leavevmode\vadjust pre{\hypertarget{ref-Landis2012}{}}%
\CSLLeftMargin{116. }%
\CSLRightInline{Landis SC, Amara SG, Asadullah K, et al. A call for transparent reporting to optimize the predictive value of preclinical research. \emph{Nature}. 2012;490(7419):187-191. doi:\href{https://doi.org/10.1038/nature11556}{10.1038/nature11556}}

\leavevmode\vadjust pre{\hypertarget{ref-huebner2016}{}}%
\CSLLeftMargin{117. }%
\CSLRightInline{Huebner M, Vach W, Cessie S le. A systematic approach to initial data analysis is good research practice. \emph{The Journal of Thoracic and Cardiovascular Surgery}. 2016;151(1):25-27. doi:\href{https://doi.org/10.1016/j.jtcvs.2015.09.085}{10.1016/j.jtcvs.2015.09.085}}

\leavevmode\vadjust pre{\hypertarget{ref-explore}{}}%
\CSLLeftMargin{118. }%
\CSLRightInline{Krasser R. Explore: Simplifies exploratory data analysis. 2023. \href{https://CRAN.R-project.org/package=explore}{https://CRAN.R-project.org/package=explore.}}

\leavevmode\vadjust pre{\hypertarget{ref-dataMaid}{}}%
\CSLLeftMargin{119. }%
\CSLRightInline{Petersen AH, Ekstrøm CT. {\textbraceleft}dataMaid{\textbraceright}: Your assistant for documenting supervised data quality screening in {\textbraceleft}r{\textbraceright}. 2019;90. doi:\href{https://doi.org/10.18637/jss.v090.i06}{10.18637/jss.v090.i06}}

\leavevmode\vadjust pre{\hypertarget{ref-DataExplorer-2}{}}%
\CSLLeftMargin{120. }%
\CSLRightInline{Cui B. DataExplorer: Automate data exploration and treatment. 2020. \href{https://CRAN.R-project.org/package=DataExplorer}{https://CRAN.R-project.org/package=DataExplorer.}}

\leavevmode\vadjust pre{\hypertarget{ref-SmartEDA}{}}%
\CSLLeftMargin{121. }%
\CSLRightInline{Dayanand Ubrangala, R K, Prasad Kondapalli R, Putatunda S. SmartEDA: Summarize and explore the data. 2022. \href{https://CRAN.R-project.org/package=SmartEDA}{https://CRAN.R-project.org/package=SmartEDA.}}

\leavevmode\vadjust pre{\hypertarget{ref-graphics}{}}%
\CSLLeftMargin{122. }%
\CSLRightInline{R Core Team. R: A language and environment for statistical computing. 2023. \href{https://www.R-project.org/}{https://www.R-project.org/.}}

\leavevmode\vadjust pre{\hypertarget{ref-Cummings2003}{}}%
\CSLLeftMargin{123. }%
\CSLRightInline{Cummings P, Rivara FP. Reporting Statistical Information in Medical Journal Articles. \emph{Archives of Pediatrics \& Adolescent Medicine}. 2003;157(4):321. doi:\href{https://doi.org/10.1001/archpedi.157.4.321}{10.1001/archpedi.157.4.321}}

\leavevmode\vadjust pre{\hypertarget{ref-Inskip2017}{}}%
\CSLLeftMargin{124. }%
\CSLRightInline{Inskip H, Ntani G, Westbury L, et al. Getting started with tables. \emph{Archives of Public Health}. 2017;75(1). doi:\href{https://doi.org/10.1186/s13690-017-0180-1}{10.1186/s13690-017-0180-1}}

\leavevmode\vadjust pre{\hypertarget{ref-Kwak2021}{}}%
\CSLLeftMargin{125. }%
\CSLRightInline{Kwak SG, Kang H, Kim JH, et al. The principles of presenting statistical results: Table. \emph{Korean Journal of Anesthesiology}. 2021;74(2):115-119. doi:\href{https://doi.org/10.4097/kja.20582}{10.4097/kja.20582}}

\leavevmode\vadjust pre{\hypertarget{ref-barnett2023}{}}%
\CSLLeftMargin{126. }%
\CSLRightInline{Barnett A. Automated detection of over- and under-dispersion in baseline tables in randomised controlled trials. \emph{F1000Research}. 2023;11:783. doi:\href{https://doi.org/10.12688/f1000research.123002.2}{10.12688/f1000research.123002.2}}

\leavevmode\vadjust pre{\hypertarget{ref-flextable}{}}%
\CSLLeftMargin{127. }%
\CSLRightInline{Gohel D, Skintzos P. Flextable: Functions for tabular reporting. 2023. \href{https://CRAN.R-project.org/package=flextable}{https://CRAN.R-project.org/package=flextable.}}

\leavevmode\vadjust pre{\hypertarget{ref-Westreich2013}{}}%
\CSLLeftMargin{128. }%
\CSLRightInline{Westreich D, Greenland S. The Table 2 Fallacy: Presenting and Interpreting Confounder and Modifier Coefficients. \emph{American Journal of Epidemiology}. 2013;177(4):292-298. doi:\href{https://doi.org/10.1093/aje/kws412}{10.1093/aje/kws412}}

\leavevmode\vadjust pre{\hypertarget{ref-chen2020}{}}%
\CSLLeftMargin{129. }%
\CSLRightInline{Chen H, Lu Y, Slye N. Testing for baseline differences in clinical trials. \emph{International Journal of Clinical Trials}. 2020;7(2):150. doi:\href{https://doi.org/10.18203/2349-3259.ijct20201720}{10.18203/2349-3259.ijct20201720}}

\leavevmode\vadjust pre{\hypertarget{ref-pijls2022}{}}%
\CSLLeftMargin{130. }%
\CSLRightInline{Pijls BG. The Table I Fallacy: P Values in Baseline Tables of Randomized Controlled Trials. \emph{Journal of Bone and Joint Surgery}. 2022;104(16):e71. doi:\href{https://doi.org/10.2106/jbjs.21.01166}{10.2106/jbjs.21.01166}}

\leavevmode\vadjust pre{\hypertarget{ref-Hayes-Larson2019}{}}%
\CSLLeftMargin{131. }%
\CSLRightInline{Hayes-Larson E, Kezios KL, Mooney SJ, Lovasi G. Who is in this study, anyway? Guidelines for a useful Table~1. \emph{Journal of Clinical Epidemiology}. 2019;114:125-132. doi:\href{https://doi.org/10.1016/j.jclinepi.2019.06.011}{10.1016/j.jclinepi.2019.06.011}}

\leavevmode\vadjust pre{\hypertarget{ref-table1}{}}%
\CSLLeftMargin{132. }%
\CSLRightInline{Rich B. table1: Tables of descriptive statistics in HTML. 2023. \href{https://CRAN.R-project.org/package=table1}{https://CRAN.R-project.org/package=table1.}}

\leavevmode\vadjust pre{\hypertarget{ref-gtsummary-2}{}}%
\CSLLeftMargin{133. }%
\CSLRightInline{Sjoberg DD, Whiting K, Curry M, Lavery JA, Larmarange J. Reproducible summary tables with the gtsummary package. 2021;13:570-580. doi:\href{https://doi.org/10.32614/RJ-2021-053}{10.32614/RJ-2021-053}}

\leavevmode\vadjust pre{\hypertarget{ref-bandoli2018}{}}%
\CSLLeftMargin{134. }%
\CSLRightInline{Bandoli G, Palmsten K, Chambers CD, Jelliffe-Pawlowski LL, Baer RJ, Thompson CA. Revisiting the Table 2 fallacy: A motivating example examining preeclampsia and preterm birth. \emph{Paediatric and Perinatal Epidemiology}. 2018;32(4):390-397. doi:\href{https://doi.org/10.1111/ppe.12474}{10.1111/ppe.12474}}

\leavevmode\vadjust pre{\hypertarget{ref-Park2022}{}}%
\CSLLeftMargin{135. }%
\CSLRightInline{Park JH, Lee DK, Kang H, et al. The principles of presenting statistical results using figures. \emph{Korean Journal of Anesthesiology}. 2022;75(2):139-150. doi:\href{https://doi.org/10.4097/kja.21508}{10.4097/kja.21508}}

\leavevmode\vadjust pre{\hypertarget{ref-ggplot2}{}}%
\CSLLeftMargin{136. }%
\CSLRightInline{Wickham H. ggplot2: Elegant graphics for data analysis. 2016. \href{https://ggplot2.tidyverse.org}{https://ggplot2.tidyverse.org.}}

\leavevmode\vadjust pre{\hypertarget{ref-plotly}{}}%
\CSLLeftMargin{137. }%
\CSLRightInline{Sievert C. Interactive web-based data visualization with r, plotly, and shiny. 2020. \href{https://plotly-r.com}{https://plotly-r.com.}}

\leavevmode\vadjust pre{\hypertarget{ref-corrplot}{}}%
\CSLLeftMargin{138. }%
\CSLRightInline{Wei T, Simko V. R package 'corrplot': Visualization of a correlation matrix. 2021. \href{https://github.com/taiyun/corrplot}{https://github.com/taiyun/corrplot.}}

\leavevmode\vadjust pre{\hypertarget{ref-Cumming2007}{}}%
\CSLLeftMargin{139. }%
\CSLRightInline{Cumming G, Fidler F, Vaux DL. Error bars in experimental biology. \emph{The Journal of Cell Biology}. 2007;177(1):7-11. doi:\href{https://doi.org/10.1083/jcb.200611141}{10.1083/jcb.200611141}}

\leavevmode\vadjust pre{\hypertarget{ref-Weissgerber2019}{}}%
\CSLLeftMargin{140. }%
\CSLRightInline{Weissgerber TL, Winham SJ, Heinzen EP, et al. Reveal, Don{'}t Conceal. \emph{Circulation}. 2019;140(18):1506-1518. doi:\href{https://doi.org/10.1161/circulationaha.118.037777}{10.1161/circulationaha.118.037777}}

\leavevmode\vadjust pre{\hypertarget{ref-ggsci}{}}%
\CSLLeftMargin{141. }%
\CSLRightInline{Xiao N. Ggsci: Scientific journal and sci-fi themed color palettes for 'ggplot2'. 2023. \href{https://CRAN.R-project.org/package=ggsci}{https://CRAN.R-project.org/package=ggsci.}}

\leavevmode\vadjust pre{\hypertarget{ref-grDevices}{}}%
\CSLLeftMargin{142. }%
\CSLRightInline{R Core Team. R: A language and environment for statistical computing. 2023. \href{https://www.R-project.org/}{https://www.R-project.org/.}}

\leavevmode\vadjust pre{\hypertarget{ref-tiff}{}}%
\CSLLeftMargin{143. }%
\CSLRightInline{Urbanek S, Johnson K. Tiff: Read and write TIFF images. 2022. \href{https://CRAN.R-project.org/package=tiff}{https://CRAN.R-project.org/package=tiff.}}

\leavevmode\vadjust pre{\hypertarget{ref-Curran-Everett2009}{}}%
\CSLLeftMargin{144. }%
\CSLRightInline{Curran-Everett D. Explorations in statistics: hypothesis tests and {\emph{P}} values. \emph{Advances in Physiology Education}. 2009;33(2):81-86. doi:\href{https://doi.org/10.1152/advan.90218.2008}{10.1152/advan.90218.2008}}

\leavevmode\vadjust pre{\hypertarget{ref-goodman1999}{}}%
\CSLLeftMargin{145. }%
\CSLRightInline{Goodman SN. Toward Evidence-Based Medical Statistics. 1: The P Value Fallacy. \emph{Annals of Internal Medicine}. 1999;130(12):995. doi:\href{https://doi.org/10.7326/0003-4819-130-12-199906150-00008}{10.7326/0003-4819-130-12-199906150-00008}}

\leavevmode\vadjust pre{\hypertarget{ref-Vandenbroucke2018}{}}%
\CSLLeftMargin{146. }%
\CSLRightInline{Vandenbroucke JP, Pearce N. From ideas to studies: how to get ideas and sharpen them into research questions. \emph{Clinical Epidemiology}. 2018;Volume 10:253-264. doi:\href{https://doi.org/10.2147/clep.s142940}{10.2147/clep.s142940}}

\leavevmode\vadjust pre{\hypertarget{ref-lakens2018}{}}%
\CSLLeftMargin{147. }%
\CSLRightInline{Lakens D, Scheel AM, Isager PM. Equivalence Testing for Psychological Research: A Tutorial. \emph{Advances in Methods and Practices in Psychological Science}. 2018;1(2):259-269. doi:\href{https://doi.org/10.1177/2515245918770963}{10.1177/2515245918770963}}

\leavevmode\vadjust pre{\hypertarget{ref-Sullivan2012}{}}%
\CSLLeftMargin{148. }%
\CSLRightInline{Sullivan GM, Feinn R. Using Effect Size{\textemdash}or Why the {\emph{P}} Value Is Not Enough. \emph{Journal of Graduate Medical Education}. 2012;4(3):279-282. doi:\href{https://doi.org/10.4300/jgme-d-12-00156.1}{10.4300/jgme-d-12-00156.1}}

\leavevmode\vadjust pre{\hypertarget{ref-latter1902}{}}%
\CSLLeftMargin{149. }%
\CSLRightInline{LATTER OH. THE EGG OF CUCULUS CANORUS: AN ENQUIRY INTO THE DIMENSIONS OF THE CUCKOO'S EGO AND THE RELATION OF THE VARIATIONS TO THE SIZE OF THE EGGS OF THE FOSTER-PARENT, WITH NOTES ON COLORATION, \&c. \emph{Biometrika}. 1902;1(2):164-176. doi:\href{https://doi.org/10.1093/biomet/1.2.164}{10.1093/biomet/1.2.164}}

\leavevmode\vadjust pre{\hypertarget{ref-aylmerfisher1926}{}}%
\CSLLeftMargin{150. }%
\CSLRightInline{Aylmer Fisher R. The arrangement of field experiments. \emph{Ministry of Agriculture and Fisheries}. 1926. doi:\href{https://doi.org/10.23637/ROTHAMSTED.8V61Q}{10.23637/ROTHAMSTED.8V61Q}}

\leavevmode\vadjust pre{\hypertarget{ref-Superpower}{}}%
\CSLLeftMargin{151. }%
\CSLRightInline{Lakens D, Caldwell A. Simulation-based power analysis for factorial analysis of variance designs. 2021;4:251524592095150. doi:\href{https://doi.org/10.1177/2515245920951503}{10.1177/2515245920951503}}

\leavevmode\vadjust pre{\hypertarget{ref-wasserstein2016}{}}%
\CSLLeftMargin{152. }%
\CSLRightInline{Wasserstein RL, Lazar NA. The ASA Statement on {\emph{p}}-Values: Context, Process, and Purpose. \emph{The American Statistician}. 2016;70(2):129-133. doi:\href{https://doi.org/10.1080/00031305.2016.1154108}{10.1080/00031305.2016.1154108}}

\leavevmode\vadjust pre{\hypertarget{ref-altman2017}{}}%
\CSLLeftMargin{153. }%
\CSLRightInline{Altman N, Krzywinski M. P values and the search for significance. \emph{Nature Methods}. 2017;14(1):3-4. doi:\href{https://doi.org/10.1038/nmeth.4120}{10.1038/nmeth.4120}}

\leavevmode\vadjust pre{\hypertarget{ref-heinze2016}{}}%
\CSLLeftMargin{154. }%
\CSLRightInline{Heinze G, Dunkler D. Five myths about variable selection. \emph{Transplant International}. 2016;30(1):6-10. doi:\href{https://doi.org/10.1111/tri.12895}{10.1111/tri.12895}}

\leavevmode\vadjust pre{\hypertarget{ref-goodman2016}{}}%
\CSLLeftMargin{155. }%
\CSLRightInline{Goodman SN. Aligning statistical and scientific reasoning. \emph{Science}. 2016;352(6290):1180-1181. doi:\href{https://doi.org/10.1126/science.aaf5406}{10.1126/science.aaf5406}}

\leavevmode\vadjust pre{\hypertarget{ref-Kim2015}{}}%
\CSLLeftMargin{156. }%
\CSLRightInline{Kim HY. Statistical notes for clinical researchers: effect size. \emph{Restorative Dentistry \& Endodontics}. 2015;40(4):328. doi:\href{https://doi.org/10.5395/rde.2015.40.4.328}{10.5395/rde.2015.40.4.328}}

\leavevmode\vadjust pre{\hypertarget{ref-effectsize}{}}%
\CSLLeftMargin{157. }%
\CSLRightInline{Ben-Shachar MS, Lüdecke D, Makowski D. {\textbraceleft}E{\textbraceright}ffectsize: Estimation of effect size indices and standardized parameters. 2020;5:2815. doi:\href{https://doi.org/10.21105/joss.02815}{10.21105/joss.02815}}

\leavevmode\vadjust pre{\hypertarget{ref-pwr}{}}%
\CSLLeftMargin{158. }%
\CSLRightInline{Champely S. Pwr: Basic functions for power analysis. 2020. \href{https://CRAN.R-project.org/package=pwr}{https://CRAN.R-project.org/package=pwr.}}

\leavevmode\vadjust pre{\hypertarget{ref-heckman2022}{}}%
\CSLLeftMargin{159. }%
\CSLRightInline{Heckman MG, Davis JM, Crowson CS. Post Hoc Power Calculations: An Inappropriate Method for Interpreting the Findings of a Research Study. \emph{The Journal of Rheumatology}. 2022;49(8):867-870. doi:\href{https://doi.org/10.3899/jrheum.211115}{10.3899/jrheum.211115}}

\leavevmode\vadjust pre{\hypertarget{ref-longpower}{}}%
\CSLLeftMargin{160. }%
\CSLRightInline{Iddi S, Donohue MC. Power and sample size for longitudinal models in r-the longpower package and shiny app. 2022;14:264-281.}

\leavevmode\vadjust pre{\hypertarget{ref-greenhalgh1997a}{}}%
\CSLLeftMargin{161. }%
\CSLRightInline{Greenhalgh T. How to read a paper: Statistics for the non-statistician. II: {̈}Significant{̈} relations and their pitfalls. \emph{BMJ}. 1997;315(7105):422-425. doi:\href{https://doi.org/10.1136/bmj.315.7105.422}{10.1136/bmj.315.7105.422}}

\leavevmode\vadjust pre{\hypertarget{ref-weintraub2016}{}}%
\CSLLeftMargin{162. }%
\CSLRightInline{Weintraub PG. The Importance of Publishing Negative Results. \emph{Journal of Insect Science}. 2016;16(1):109. doi:\href{https://doi.org/10.1093/jisesa/iew092}{10.1093/jisesa/iew092}}

\leavevmode\vadjust pre{\hypertarget{ref-altman1995}{}}%
\CSLLeftMargin{163. }%
\CSLRightInline{Altman DG, Bland JM. Statistics notes: Absence of evidence is not evidence of absence. \emph{BMJ}. 1995;311(7003):485-485. doi:\href{https://doi.org/10.1136/bmj.311.7003.485}{10.1136/bmj.311.7003.485}}

\leavevmode\vadjust pre{\hypertarget{ref-Breznau2022}{}}%
\CSLLeftMargin{164. }%
\CSLRightInline{Breznau N, Rinke EM, Wuttke A, et al. Observing many researchers using the same data and hypothesis reveals a hidden universe of uncertainty. \emph{Proceedings of the National Academy of Sciences}. 2022;(44):e2203150119. doi:\href{https://doi.org/10.1073/pnas.2203150119}{10.1073/pnas.2203150119}}

\leavevmode\vadjust pre{\hypertarget{ref-dwivedi2019}{}}%
\CSLLeftMargin{165. }%
\CSLRightInline{Dwivedi AK, Shukla R. Evidence{-}based statistical analysis and methods in biomedical research (SAMBR) checklists according to design features. \emph{CANCER REPORTS}. 2019;3(4). doi:\href{https://doi.org/10.1002/cnr2.1211}{10.1002/cnr2.1211}}

\leavevmode\vadjust pre{\hypertarget{ref-Dwivedi2022}{}}%
\CSLLeftMargin{166. }%
\CSLRightInline{Dwivedi AK. How to Write Statistical Analysis Section in Medical Research. \emph{Journal of Investigative Medicine}. 2022;70(8):1759-1770. doi:\href{https://doi.org/10.1136/jim-2022-002479}{10.1136/jim-2022-002479}}

\leavevmode\vadjust pre{\hypertarget{ref-Kim2017}{}}%
\CSLLeftMargin{167. }%
\CSLRightInline{Kim N, Fischer AH, Dyring-Andersen B, Rosner B, Okoye GA. Research Techniques Made Simple: Choosing Appropriate Statistical Methods for Clinical Research. \emph{Journal of Investigative Dermatology}. 2017;137(10):e173-e178. doi:\href{https://doi.org/10.1016/j.jid.2017.08.007}{10.1016/j.jid.2017.08.007}}

\leavevmode\vadjust pre{\hypertarget{ref-marusteri2010}{}}%
\CSLLeftMargin{168. }%
\CSLRightInline{Marusteri M, Bacarea V. Comparing groups for statistical differences: How to choose the right statistical test? \emph{Biochemia Medica}. 2010:15-32. doi:\href{https://doi.org/10.11613/bm.2010.004}{10.11613/bm.2010.004}}

\leavevmode\vadjust pre{\hypertarget{ref-mishra2019}{}}%
\CSLLeftMargin{169. }%
\CSLRightInline{Mishra P, Pandey C, Singh U, Keshri A, Sabaretnam M. Selection of appropriate statistical methods for data analysis. \emph{Annals of Cardiac Anaesthesia}. 2019;22(3):297. doi:\href{https://doi.org/10.4103/aca.aca_248_18}{10.4103/aca.aca\_248\_18}}

\leavevmode\vadjust pre{\hypertarget{ref-ray2021}{}}%
\CSLLeftMargin{170. }%
\CSLRightInline{Ray A, Najmi A, Sadasivam B. How to choose and interpret a statistical test? An update for budding researchers. \emph{Journal of Family Medicine and Primary Care}. 2021;10(8):2763. doi:\href{https://doi.org/10.4103/jfmpc.jfmpc_433_21}{10.4103/jfmpc.jfmpc\_433\_21}}

\leavevmode\vadjust pre{\hypertarget{ref-nayak2011}{}}%
\CSLLeftMargin{171. }%
\CSLRightInline{Nayak B, Hazra A. How to choose the right statistical test? \emph{Indian Journal of Ophthalmology}. 2011;59(2):85. doi:\href{https://doi.org/10.4103/0301-4738.77005}{10.4103/0301-4738.77005}}

\leavevmode\vadjust pre{\hypertarget{ref-shankar2014}{}}%
\CSLLeftMargin{172. }%
\CSLRightInline{Shankar S, Singh R. Demystifying statistics: How to choose a statistical test? \emph{Indian Journal of Rheumatology}. 2014;9(2):77-81. doi:\href{https://doi.org/10.1016/j.injr.2014.04.002}{10.1016/j.injr.2014.04.002}}

\leavevmode\vadjust pre{\hypertarget{ref-cocor-4}{}}%
\CSLLeftMargin{173. }%
\CSLRightInline{Diedenhofen B, Musch J. Cocor: A comprehensive solution for the statistical comparison of correlations. 2015;10:e0121945. doi:\href{https://doi.org/10.1371/journal.pone.0121945}{10.1371/journal.pone.0121945}}

\leavevmode\vadjust pre{\hypertarget{ref-cocor}{}}%
\CSLLeftMargin{174. }%
\CSLRightInline{Diedenhofen B, Musch J. Cocor: A comprehensive solution for the statistical comparison of correlations. 2015;10:e0121945. doi:\href{https://doi.org/10.1371/journal.pone.0121945}{10.1371/journal.pone.0121945}}

\leavevmode\vadjust pre{\hypertarget{ref-khamis2008}{}}%
\CSLLeftMargin{175. }%
\CSLRightInline{Khamis H. Measures of Association: How to Choose? \emph{Journal of Diagnostic Medical Sonography}. 2008;24(3):155-162. doi:\href{https://doi.org/10.1177/8756479308317006}{10.1177/8756479308317006}}

\leavevmode\vadjust pre{\hypertarget{ref-allison2022}{}}%
\CSLLeftMargin{176. }%
\CSLRightInline{Allison JS, Santana L, (Jaco) Visagie IJH. A primer on simple measures of association taught at undergraduate level. \emph{Teaching Statistics}. 2022;44(3):96-103. doi:\href{https://doi.org/10.1111/test.12307}{10.1111/test.12307}}

\leavevmode\vadjust pre{\hypertarget{ref-corrplot-2}{}}%
\CSLLeftMargin{177. }%
\CSLRightInline{Wei T, Simko V. R package 'corrplot': Visualization of a correlation matrix. 2021. \href{https://github.com/taiyun/corrplot}{https://github.com/taiyun/corrplot.}}

\leavevmode\vadjust pre{\hypertarget{ref-cooccur}{}}%
\CSLLeftMargin{178. }%
\CSLRightInline{Griffith DM, Veech JA, Marsh CJ. {\textbraceleft}Cooccur{\textbraceright}: Probabilistic species co-occurrence analysis in {\textbraceleft}r{\textbraceright}. 2016;69. doi:\href{https://doi.org/10.18637/jss.v069.c02}{10.18637/jss.v069.c02}}

\leavevmode\vadjust pre{\hypertarget{ref-McHugh2013}{}}%
\CSLLeftMargin{179. }%
\CSLRightInline{McHugh ML. The chi-square test of independence. \emph{Biochemia Medica}. 2013:143-149. doi:\href{https://doi.org/10.11613/bm.2013.018}{10.11613/bm.2013.018}}

\leavevmode\vadjust pre{\hypertarget{ref-Kim2017a}{}}%
\CSLLeftMargin{180. }%
\CSLRightInline{Kim HY. Statistical notes for clinical researchers: Chi-squared test and Fisher's exact test. \emph{Restorative Dentistry \& Endodontics}. 2017;42(2):152. doi:\href{https://doi.org/10.5395/rde.2017.42.2.152}{10.5395/rde.2017.42.2.152}}

\leavevmode\vadjust pre{\hypertarget{ref-gtsummary}{}}%
\CSLLeftMargin{181. }%
\CSLRightInline{Sjoberg DD, Whiting K, Curry M, Lavery JA, Larmarange J. Reproducible summary tables with the gtsummary package. 2021;13:570-580. doi:\href{https://doi.org/10.32614/RJ-2021-053}{10.32614/RJ-2021-053}}

\leavevmode\vadjust pre{\hypertarget{ref-Hidalgo2013}{}}%
\CSLLeftMargin{182. }%
\CSLRightInline{Hidalgo B, Goodman M. Multivariate or Multivariable Regression? \emph{American Journal of Public Health}. 2013;103(1):39-40. doi:\href{https://doi.org/10.2105/ajph.2012.300897}{10.2105/ajph.2012.300897}}

\leavevmode\vadjust pre{\hypertarget{ref-modelsummary}{}}%
\CSLLeftMargin{183. }%
\CSLRightInline{Arel-Bundock V. {\textbraceleft}Modelsummary{\textbraceright}: Data and model summaries in {\textbraceleft}r{\textbraceright}. 2022;103. doi:\href{https://doi.org/10.18637/jss.v103.i01}{10.18637/jss.v103.i01}}

\leavevmode\vadjust pre{\hypertarget{ref-suits1957}{}}%
\CSLLeftMargin{184. }%
\CSLRightInline{Suits DB. Use of Dummy Variables in Regression Equations. \emph{Journal of the American Statistical Association}. 1957;52(280):548-551. doi:\href{https://doi.org/10.1080/01621459.1957.10501412}{10.1080/01621459.1957.10501412}}

\leavevmode\vadjust pre{\hypertarget{ref-Healy1995}{}}%
\CSLLeftMargin{185. }%
\CSLRightInline{Healy MJ. Statistics from the inside. 16. Multiple regression (2). \emph{Archives of Disease in Childhood}. 1995;73(3):270-274. doi:\href{https://doi.org/10.1136/adc.73.3.270}{10.1136/adc.73.3.270}}

\leavevmode\vadjust pre{\hypertarget{ref-fastDummies}{}}%
\CSLLeftMargin{186. }%
\CSLRightInline{Kaplan J. fastDummies: Fast creation of dummy (binary) columns and rows from categorical variables. 2023. \href{https://CRAN.R-project.org/package=fastDummies}{https://CRAN.R-project.org/package=fastDummies.}}

\leavevmode\vadjust pre{\hypertarget{ref-Dales1978}{}}%
\CSLLeftMargin{187. }%
\CSLRightInline{DALES LG, URY HK. An Improper Use of Statistical Significance Testing in Studying Covariables. \emph{International Journal of Epidemiology}. 1978;7(4):373-376. doi:\href{https://doi.org/10.1093/ije/7.4.373}{10.1093/ije/7.4.373}}

\leavevmode\vadjust pre{\hypertarget{ref-Sun1996}{}}%
\CSLLeftMargin{188. }%
\CSLRightInline{Sun GW, Shook TL, Kay GL. Inappropriate use of bivariable analysis to screen risk factors for use in multivariable analysis. \emph{Journal of Clinical Epidemiology}. 1996;49(8):907-916. doi:\href{https://doi.org/10.1016/0895-4356(96)00025-x}{10.1016/0895-4356(96)00025-x}}

\leavevmode\vadjust pre{\hypertarget{ref-Bours2023}{}}%
\CSLLeftMargin{189. }%
\CSLRightInline{Bours MJL. Using mediators to understand effect modification and interaction. \emph{Journal of Clinical Epidemiology}. September 2023. doi:\href{https://doi.org/10.1016/j.jclinepi.2023.09.005}{10.1016/j.jclinepi.2023.09.005}}

\leavevmode\vadjust pre{\hypertarget{ref-Altman1996}{}}%
\CSLLeftMargin{190. }%
\CSLRightInline{Altman DG, Matthews JNS. Statistics Notes: Interaction 1: heterogeneity of effects. \emph{BMJ}. 1996;313(7055):486-486. doi:\href{https://doi.org/10.1136/bmj.313.7055.486}{10.1136/bmj.313.7055.486}}

\leavevmode\vadjust pre{\hypertarget{ref-nlme}{}}%
\CSLLeftMargin{191. }%
\CSLRightInline{Pinheiro J, Bates D, R Core Team. Nlme: Linear and nonlinear mixed effects models. 2023. \href{https://CRAN.R-project.org/package=nlme}{https://CRAN.R-project.org/package=nlme.}}

\leavevmode\vadjust pre{\hypertarget{ref-mmrm}{}}%
\CSLLeftMargin{192. }%
\CSLRightInline{Sabanes Bove D, Dedic J, Kelkhoff D, et al. Mmrm: Mixed models for repeated measures. 2022. \href{https://CRAN.R-project.org/package=mmrm}{https://CRAN.R-project.org/package=mmrm.}}

\leavevmode\vadjust pre{\hypertarget{ref-emmeans}{}}%
\CSLLeftMargin{193. }%
\CSLRightInline{Lenth RV. Emmeans: Estimated marginal means, aka least-squares means. 2023. \href{https://CRAN.R-project.org/package=emmeans}{https://CRAN.R-project.org/package=emmeans.}}

\leavevmode\vadjust pre{\hypertarget{ref-Baron1986}{}}%
\CSLLeftMargin{194. }%
\CSLRightInline{Baron RM, Kenny DA. The moderator{\textendash}mediator variable distinction in social psychological research: Conceptual, strategic, and statistical considerations. \emph{Journal of Personality and Social Psychology}. 1986;51(6):1173-1182. doi:\href{https://doi.org/10.1037/0022-3514.51.6.1173}{10.1037/0022-3514.51.6.1173}}

\leavevmode\vadjust pre{\hypertarget{ref-greenland1986}{}}%
\CSLLeftMargin{195. }%
\CSLRightInline{GREENLAND S, SCHLESSELMAN JJ, CRIQUI MH. THE FALLACY OF EMPLOYING STANDARDIZED REGRESSION COEFFICIENTS AND CORRELATIONS AS MEASURES OF EFFECT. \emph{American Journal of Epidemiology}. 1986;123(2):203-208. doi:\href{https://doi.org/10.1093/oxfordjournals.aje.a114229}{10.1093/oxfordjournals.aje.a114229}}

\leavevmode\vadjust pre{\hypertarget{ref-greenland1991}{}}%
\CSLLeftMargin{196. }%
\CSLRightInline{Greenland S, Maclure M, Schlesselman JJ, Poole C, Morgenstern H. Standardized Regression Coefficients. \emph{Epidemiology}. 1991;2(5):387-392. doi:\href{https://doi.org/10.1097/00001648-199109000-00015}{10.1097/00001648-199109000-00015}}

\leavevmode\vadjust pre{\hypertarget{ref-performance}{}}%
\CSLLeftMargin{197. }%
\CSLRightInline{Lüdecke D, Ben-Shachar MS, Patil I, Waggoner P, Makowski D. {\textbraceleft}Performance{\textbraceright}: An {\textbraceleft}r{\textbraceright} package for assessment, comparison and testing of statistical models. 2021;6:3139. doi:\href{https://doi.org/10.21105/joss.03139}{10.21105/joss.03139}}

\leavevmode\vadjust pre{\hypertarget{ref-Grant2009}{}}%
\CSLLeftMargin{198. }%
\CSLRightInline{Grant MJ, Booth A. A typology of reviews: an analysis of 14 review types and associated methodologies. \emph{Health Information \& Libraries Journal}. 2009;26(2):91-108. doi:\href{https://doi.org/10.1111/j.1471-1842.2009.00848.x}{10.1111/j.1471-1842.2009.00848.x}}

\leavevmode\vadjust pre{\hypertarget{ref-Suxfct2014}{}}%
\CSLLeftMargin{199. }%
\CSLRightInline{Sut N. Study designs in medicine. \emph{Balkan Medical Journal}. 2015;31(4):273-277. doi:\href{https://doi.org/10.5152/balkanmedj.2014.1408}{10.5152/balkanmedj.2014.1408}}

\leavevmode\vadjust pre{\hypertarget{ref-Souza2017}{}}%
\CSLLeftMargin{200. }%
\CSLRightInline{Souza AC de, Alexandre NMC, Guirardello E de B, Souza AC de, Alexandre NMC, Guirardello E de B. Propriedades psicométricas na avaliação de instrumentos: avaliação da confiabilidade e da validade. \emph{Epidemiologia e Serviços de Saúde}. 2017;26(3):649-659. doi:\href{https://doi.org/10.5123/s1679-49742017000300022}{10.5123/s1679-49742017000300022}}

\leavevmode\vadjust pre{\hypertarget{ref-reeves2017}{}}%
\CSLLeftMargin{201. }%
\CSLRightInline{Reeves BC, Wells GA, Waddington H. Quasi-experimental study designs series{\textemdash}paper 5: a checklist for~classifying studies evaluating the effects on health interventions{\textemdash}a taxonomy without labels. \emph{Journal of Clinical Epidemiology}. 2017;89:30-42. doi:\href{https://doi.org/10.1016/j.jclinepi.2017.02.016}{10.1016/j.jclinepi.2017.02.016}}

\leavevmode\vadjust pre{\hypertarget{ref-echevarruxeda-guanilo2019}{}}%
\CSLLeftMargin{202. }%
\CSLRightInline{Echevarría-Guanilo ME, Gonçalves N, Romanoski PJ. PSYCHOMETRIC PROPERTIES OF MEASUREMENT INSTRUMENTS: CONCEPTUAL BASIS AND EVALUATION METHODS - PART II. \emph{Texto \& Contexto - Enfermagem}. 2019;28. doi:\href{https://doi.org/10.1590/1980-265x-tce-2017-0311}{10.1590/1980-265x-tce-2017-0311}}

\leavevmode\vadjust pre{\hypertarget{ref-Chassuxe92019}{}}%
\CSLLeftMargin{203. }%
\CSLRightInline{Chassé M, Fergusson DA. Diagnostic Accuracy Studies. \emph{Seminars in Nuclear Medicine}. 2019;49(2):87-93. doi:\href{https://doi.org/10.1053/j.semnuclmed.2018.11.005}{10.1053/j.semnuclmed.2018.11.005}}

\leavevmode\vadjust pre{\hypertarget{ref-Chidambaram2019}{}}%
\CSLLeftMargin{204. }%
\CSLRightInline{Chidambaram AG, Josephson M. Clinical research study designs: The essentials. \emph{PEDIATRIC INVESTIGATION}. 2019;3(4):245-252. doi:\href{https://doi.org/10.1002/ped4.12166}{10.1002/ped4.12166}}

\leavevmode\vadjust pre{\hypertarget{ref-Erdemir2020}{}}%
\CSLLeftMargin{205. }%
\CSLRightInline{Erdemir A, Mulugeta L, Ku JP, et al. Credible practice of modeling and simulation in healthcare: ten rules from a multidisciplinary perspective. \emph{Journal of Translational Medicine}. 2020;18(1). doi:\href{https://doi.org/10.1186/s12967-020-02540-4}{10.1186/s12967-020-02540-4}}

\leavevmode\vadjust pre{\hypertarget{ref-Yang2021}{}}%
\CSLLeftMargin{206. }%
\CSLRightInline{Yang B, Olsen M, Vali Y, et al. Study designs for comparative diagnostic test accuracy: A methodological review and classification scheme. \emph{Journal of Clinical Epidemiology}. 2021;138:128-138. doi:\href{https://doi.org/10.1016/j.jclinepi.2021.04.013}{10.1016/j.jclinepi.2021.04.013}}

\leavevmode\vadjust pre{\hypertarget{ref-chipman2022}{}}%
\CSLLeftMargin{207. }%
\CSLRightInline{Chipman H, Bingham D. Let's practice what we preach: Planning and interpreting simulation studies with design and analysis of experiments. \emph{Canadian Journal of Statistics}. 2022;50(4):1228-1249. doi:\href{https://doi.org/10.1002/cjs.11719}{10.1002/cjs.11719}}

\leavevmode\vadjust pre{\hypertarget{ref-donthu2021}{}}%
\CSLLeftMargin{208. }%
\CSLRightInline{Donthu N, Kumar S, Mukherjee D, Pandey N, Lim WM. How to conduct a bibliometric analysis: An overview and guidelines. \emph{Journal of Business Research}. 2021;133:285-296. doi:\href{https://doi.org/10.1016/j.jbusres.2021.04.070}{10.1016/j.jbusres.2021.04.070}}

\leavevmode\vadjust pre{\hypertarget{ref-lim2023}{}}%
\CSLLeftMargin{209. }%
\CSLRightInline{Lim WM, Kumar S. Guidelines for interpreting the results of bibliometric analysis: A sensemaking approach. \emph{Global Business and Organizational Excellence}. August 2023. doi:\href{https://doi.org/10.1002/joe.22229}{10.1002/joe.22229}}

\leavevmode\vadjust pre{\hypertarget{ref-Bland1994}{}}%
\CSLLeftMargin{210. }%
\CSLRightInline{Bland JM, Altman DG. Statistics notes: Matching. \emph{BMJ}. 1994;309(6962):1128-1128. doi:\href{https://doi.org/10.1136/bmj.309.6962.1128}{10.1136/bmj.309.6962.1128}}

\leavevmode\vadjust pre{\hypertarget{ref-rodruxedguezdeluxe1guila2014}{}}%
\CSLLeftMargin{211. }%
\CSLRightInline{Rodríguez del Águila M, González-Ramírez A. Sample size calculation. \emph{Allergologia et Immunopathologia}. 2014;42(5):485-492. doi:\href{https://doi.org/10.1016/j.aller.2013.03.008}{10.1016/j.aller.2013.03.008}}

\leavevmode\vadjust pre{\hypertarget{ref-Bacchetti2005}{}}%
\CSLLeftMargin{212. }%
\CSLRightInline{Bacchetti P. Ethics and Sample Size. \emph{American Journal of Epidemiology}. 2005;161(2):105-110. doi:\href{https://doi.org/10.1093/aje/kwi014}{10.1093/aje/kwi014}}

\leavevmode\vadjust pre{\hypertarget{ref-simstudy}{}}%
\CSLLeftMargin{213. }%
\CSLRightInline{Goldfeld K, Wujciak-Jens J. Simstudy: Illuminating research methods through data generation. 2020;5:2763. doi:\href{https://doi.org/10.21105/joss.02763}{10.21105/joss.02763}}

\leavevmode\vadjust pre{\hypertarget{ref-bland2011}{}}%
\CSLLeftMargin{214. }%
\CSLRightInline{Bland JM, Altman DG. Comparisons within randomised groups can be very misleading. \emph{BMJ}. 2011;342(may06 2):d561-d561. doi:\href{https://doi.org/10.1136/bmj.d561}{10.1136/bmj.d561}}

\leavevmode\vadjust pre{\hypertarget{ref-Bruce2022}{}}%
\CSLLeftMargin{215. }%
\CSLRightInline{Bruce CL, Juszczak E, Ogollah R, Partlett C, Montgomery A. A systematic review of randomisation method use in RCTs and association of trial design characteristics with method selection. \emph{BMC Medical Research Methodology}. 2022;22(1). doi:\href{https://doi.org/10.1186/s12874-022-01786-4}{10.1186/s12874-022-01786-4}}

\leavevmode\vadjust pre{\hypertarget{ref-Vickers2001}{}}%
\CSLLeftMargin{216. }%
\CSLRightInline{Vickers AJ, Altman DG. Statistics Notes: Analysing controlled trials with baseline and follow up measurements. \emph{BMJ}. 2001;323(7321):1123-1124. doi:\href{https://doi.org/10.1136/bmj.323.7321.1123}{10.1136/bmj.323.7321.1123}}

\leavevmode\vadjust pre{\hypertarget{ref-OConnell2017}{}}%
\CSLLeftMargin{217. }%
\CSLRightInline{O Connell NS, Dai L, Jiang Y, et al. Methods for analysis of pre-post data in clinical research: A comparison of five common methods. \emph{Journal of Biometrics \& Biostatistics}. 2017;08(01). doi:\href{https://doi.org/10.4172/2155-6180.1000334}{10.4172/2155-6180.1000334}}

\leavevmode\vadjust pre{\hypertarget{ref-Cnaan1997}{}}%
\CSLLeftMargin{218. }%
\CSLRightInline{Cnaan A, Laird NM, Slasor P. Using the general linear mixed model to analyse unbalanced repeated measures and longitudinal data. \emph{Statistics in Medicine}. 1997;16(20):2349-2380. doi:\href{https://doi.org/10.1002/(sici)1097-0258(19971030)16:20\%3C2349::aid-sim667\%3E3.0.co;2-e}{10.1002/(sici)1097-0258(19971030)16:20\textless2349::aid-sim667\textgreater3.0.co;2-e}}

\leavevmode\vadjust pre{\hypertarget{ref-mallinckrodt2008}{}}%
\CSLLeftMargin{219. }%
\CSLRightInline{Mallinckrodt CH, Lane PW, Schnell D, Peng Y, Mancuso JP. Recommendations for the Primary Analysis of Continuous Endpoints in Longitudinal Clinical Trials. \emph{Drug Information Journal}. 2008;42(4):303-319. doi:\href{https://doi.org/10.1177/009286150804200402}{10.1177/009286150804200402}}

\leavevmode\vadjust pre{\hypertarget{ref-Cao2022}{}}%
\CSLLeftMargin{220. }%
\CSLRightInline{Cao Y, Allore H, Vander Wyk B, Gutman R. Review and evaluation of imputation methods for multivariate longitudinal data with mixed{-}type incomplete variables. \emph{Statistics in Medicine}. October 2022. doi:\href{https://doi.org/10.1002/sim.9592}{10.1002/sim.9592}}

\leavevmode\vadjust pre{\hypertarget{ref-Kahan2014}{}}%
\CSLLeftMargin{221. }%
\CSLRightInline{Kahan BC, Jairath V, Doré CJ, Morris TP. The risks and rewards of covariate adjustment in randomized trials: an assessment of 12 outcomes from 8 studies. \emph{Trials}. 2014;15(1). doi:\href{https://doi.org/10.1186/1745-6215-15-139}{10.1186/1745-6215-15-139}}

\leavevmode\vadjust pre{\hypertarget{ref-roberts1999}{}}%
\CSLLeftMargin{222. }%
\CSLRightInline{Roberts C, Torgerson DJ. Understanding controlled trials: Baseline imbalance in randomised controlled trials. \emph{BMJ}. 1999;319(7203):185-185. doi:\href{https://doi.org/10.1136/bmj.319.7203.185}{10.1136/bmj.319.7203.185}}

\leavevmode\vadjust pre{\hypertarget{ref-Hauck1998}{}}%
\CSLLeftMargin{223. }%
\CSLRightInline{Hauck WW, Anderson S, Marcus SM. Should We Adjust for Covariates in Nonlinear Regression Analyses of Randomized Trials? \emph{Controlled Clinical Trials}. 1998;19(3):249-256. doi:\href{https://doi.org/10.1016/s0197-2456(97)00147-5}{10.1016/s0197-2456(97)00147-5}}

\leavevmode\vadjust pre{\hypertarget{ref-Stang2018}{}}%
\CSLLeftMargin{224. }%
\CSLRightInline{Stang A, Baethge C. Imbalance \textless em\textgreater p\textless/em\textgreater{} values for baseline covariates in randomized controlled trials: a last resort for the use of \textless em\textgreater p\textless/em\textgreater{} values? A pro and contra debate. \emph{Clinical Epidemiology}. 2018;Volume 10:531-535. doi:\href{https://doi.org/10.2147/clep.s161508}{10.2147/clep.s161508}}

\leavevmode\vadjust pre{\hypertarget{ref-Bolzern2019}{}}%
\CSLLeftMargin{225. }%
\CSLRightInline{Bolzern JE, Mitchell A, Torgerson DJ. Baseline testing in cluster randomised controlled trials: should this be done? \emph{BMC Medical Research Methodology}. 2019;19(1). doi:\href{https://doi.org/10.1186/s12874-019-0750-8}{10.1186/s12874-019-0750-8}}

\leavevmode\vadjust pre{\hypertarget{ref-gruijters2020}{}}%
\CSLLeftMargin{226. }%
\CSLRightInline{Gruijters SLK. Baseline comparisons and covariate fishing: Bad statistical habits we should have broken yesterday. July 2020. \href{http://dx.doi.org/10.31234/osf.io/qftwg}{http://dx.doi.org/10.31234/osf.io/qftwg.}}

\leavevmode\vadjust pre{\hypertarget{ref-Matthews1996}{}}%
\CSLLeftMargin{227. }%
\CSLRightInline{Matthews JNS, Altman DG. Statistics Notes: Interaction 2: compare effect sizes not P values. \emph{BMJ}. 1996;313(7060):808-808. doi:\href{https://doi.org/10.1136/bmj.313.7060.808}{10.1136/bmj.313.7060.808}}

\leavevmode\vadjust pre{\hypertarget{ref-Altman2003}{}}%
\CSLLeftMargin{228. }%
\CSLRightInline{Altman DG. Statistics notes: Interaction revisited: The difference between two estimates. \emph{BMJ}. 2003;326(7382):219-219. doi:\href{https://doi.org/10.1136/bmj.326.7382.219}{10.1136/bmj.326.7382.219}}

\leavevmode\vadjust pre{\hypertarget{ref-steckelberg2004}{}}%
\CSLLeftMargin{229. }%
\CSLRightInline{Steckelberg A, Balgenorth A, Berger J, Mühlhauser I. Explaining computation of predictive values: 2 × 2 table versus frequency tree. A randomized controlled trial {[}ISRCTN74278823{]}. \emph{BMC Medical Education}. 2004;4(1). doi:\href{https://doi.org/10.1186/1472-6920-4-13}{10.1186/1472-6920-4-13}}

\leavevmode\vadjust pre{\hypertarget{ref-greenhalgh1997b}{}}%
\CSLLeftMargin{230. }%
\CSLRightInline{Greenhalgh T. How to read a paper: Papers that report diagnostic or screening tests. \emph{BMJ}. 1997;315(7107):540-543. doi:\href{https://doi.org/10.1136/bmj.315.7107.540}{10.1136/bmj.315.7107.540}}

\leavevmode\vadjust pre{\hypertarget{ref-riskyr}{}}%
\CSLLeftMargin{231. }%
\CSLRightInline{Neth H, Gaisbauer F, Gradwohl N, Gaissmaier W. Riskyr: Rendering risk literacy more transparent. 2022. \href{https://CRAN.R-project.org/package=riskyr}{https://CRAN.R-project.org/package=riskyr.}}

\leavevmode\vadjust pre{\hypertarget{ref-caret}{}}%
\CSLLeftMargin{232. }%
\CSLRightInline{Kuhn, Max. Building predictive models in r using the caret package. \emph{Journal of Statistical Software}. 2008;28(5):1-26. doi:\href{https://doi.org/10.18637/jss.v028.i05}{10.18637/jss.v028.i05}}

\leavevmode\vadjust pre{\hypertarget{ref-phillips2010}{}}%
\CSLLeftMargin{233. }%
\CSLRightInline{Phillips B, Stewart LA, Sutton AJ. {`}Cross hairs{'} plots for diagnostic meta{-}analysis. \emph{Research Synthesis Methods}. 2010;1(3-4):308-315. doi:\href{https://doi.org/10.1002/jrsm.26}{10.1002/jrsm.26}}

\leavevmode\vadjust pre{\hypertarget{ref-mada}{}}%
\CSLLeftMargin{234. }%
\CSLRightInline{Sousa-Pinto PD with contributions from B. Mada: Meta-analysis of diagnostic accuracy. 2022. \href{https://CRAN.R-project.org/package=mada}{https://CRAN.R-project.org/package=mada.}}

\leavevmode\vadjust pre{\hypertarget{ref-de2022}{}}%
\CSLLeftMargin{235. }%
\CSLRightInline{Hond AAH de, Steyerberg EW, Calster B van. Interpreting area under the receiver operating characteristic curve. \emph{The Lancet Digital Health}. 2022;4(12):e853-e855. doi:\href{https://doi.org/10.1016/s2589-7500(22)00188-1}{10.1016/s2589-7500(22)00188-1}}

\leavevmode\vadjust pre{\hypertarget{ref-pROC}{}}%
\CSLLeftMargin{236. }%
\CSLRightInline{Robin X, Turck N, Hainard A, et al. pROC: An open-source package for r and s+ to analyze and compare ROC curves. 2011;12:77.}

\leavevmode\vadjust pre{\hypertarget{ref-ferreira2021}{}}%
\CSLLeftMargin{237. }%
\CSLRightInline{Ferreira ADS, Meziat-Filho N, Ferreira APA. Double threshold receiver operating characteristic plot for three-modal continuous predictors. \emph{Computational Statistics}. 2021;36(3):2231-2245. doi:\href{https://doi.org/10.1007/s00180-021-01080-9}{10.1007/s00180-021-01080-9}}

\leavevmode\vadjust pre{\hypertarget{ref-lavaan}{}}%
\CSLLeftMargin{238. }%
\CSLRightInline{Rosseel Y. {\textbraceleft}Lavaan{\textbraceright}: An {\textbraceleft}r{\textbraceright} package for structural equation modeling. 2012;48. doi:\href{https://doi.org/10.18637/jss.v048.i02}{10.18637/jss.v048.i02}}

\leavevmode\vadjust pre{\hypertarget{ref-semTools}{}}%
\CSLLeftMargin{239. }%
\CSLRightInline{Contributors semTools. \emph{{semTools}: Useful Tools for Structural Equation Modeling}.; 2016. \href{https://CRAN.R-project.org/package=semTools}{https://CRAN.R-project.org/package=semTools.}}

\leavevmode\vadjust pre{\hypertarget{ref-psych-2}{}}%
\CSLLeftMargin{240. }%
\CSLRightInline{William Revelle. Psych: Procedures for psychological, psychometric, and personality research. 2023. \href{https://CRAN.R-project.org/package=psych}{https://CRAN.R-project.org/package=psych.}}

\leavevmode\vadjust pre{\hypertarget{ref-findley2021}{}}%
\CSLLeftMargin{241. }%
\CSLRightInline{Findley MG, Kikuta K, Denly M. External Validity. \emph{Annual Review of Political Science}. 2021;24(1):365-393. doi:\href{https://doi.org/10.1146/annurev-polisci-041719-102556}{10.1146/annurev-polisci-041719-102556}}

\leavevmode\vadjust pre{\hypertarget{ref-altman1983}{}}%
\CSLLeftMargin{242. }%
\CSLRightInline{Altman DG, Bland JM. Measurement in medicine: The analysis of method comparison studies. \emph{The Statistician}. 1983;32(3):307. doi:\href{https://doi.org/10.2307/2987937}{10.2307/2987937}}

\leavevmode\vadjust pre{\hypertarget{ref-scott1955}{}}%
\CSLLeftMargin{243. }%
\CSLRightInline{Scott WA. Reliability of content analysis: The case of nominal scale coding. \emph{Public Opinion Quarterly}. 1955;19(3):321. doi:\href{https://doi.org/10.1086/266577}{10.1086/266577}}

\leavevmode\vadjust pre{\hypertarget{ref-cohen1960}{}}%
\CSLLeftMargin{244. }%
\CSLRightInline{Cohen J. A Coefficient of Agreement for Nominal Scales. \emph{Educational and Psychological Measurement}. 1960;20(1):37-46. doi:\href{https://doi.org/10.1177/001316446002000104}{10.1177/001316446002000104}}

\leavevmode\vadjust pre{\hypertarget{ref-i.mathe1901}{}}%
\CSLLeftMargin{245. }%
\CSLRightInline{I. Mathematical contributions to the theory of evolution. {\textemdash}VII. On the correlation of characters not quantitatively measurable. \emph{Philosophical Transactions of the Royal Society of London Series A, Containing Papers of a Mathematical or Physical Character}. 1901;195(262-273):1-47. doi:\href{https://doi.org/10.1098/rsta.1900.0022}{10.1098/rsta.1900.0022}}

\leavevmode\vadjust pre{\hypertarget{ref-banerjee1999}{}}%
\CSLLeftMargin{246. }%
\CSLRightInline{Banerjee M, Capozzoli M, McSweeney L, Sinha D. Beyond kappa: A review of interrater agreement measures. \emph{Canadian Journal of Statistics}. 1999;27(1):3-23. doi:\href{https://doi.org/10.2307/3315487}{10.2307/3315487}}

\leavevmode\vadjust pre{\hypertarget{ref-psych}{}}%
\CSLLeftMargin{247. }%
\CSLRightInline{William Revelle. Psych: Procedures for psychological, psychometric, and personality research. 2023. \href{https://CRAN.R-project.org/package=psych}{https://CRAN.R-project.org/package=psych.}}

\leavevmode\vadjust pre{\hypertarget{ref-Borenstein2022}{}}%
\CSLLeftMargin{248. }%
\CSLRightInline{Borenstein M. In a meta-analysis, the I-squared statistic does not tell us how much the effect size varies. \emph{Journal of Clinical Epidemiology}. October 2022. doi:\href{https://doi.org/10.1016/j.jclinepi.2022.10.003}{10.1016/j.jclinepi.2022.10.003}}

\leavevmode\vadjust pre{\hypertarget{ref-Ruxfccker2008}{}}%
\CSLLeftMargin{249. }%
\CSLRightInline{Rücker G, Schwarzer G, Carpenter JR, Schumacher M. Undue reliance on I 2 in assessing heterogeneity may mislead. \emph{BMC Medical Research Methodology}. 2008;8(1). doi:\href{https://doi.org/10.1186/1471-2288-8-79}{10.1186/1471-2288-8-79}}

\leavevmode\vadjust pre{\hypertarget{ref-degrooth2023}{}}%
\CSLLeftMargin{250. }%
\CSLRightInline{Grooth HJ de, Parienti JJ. Heterogeneity between studies can be explained more reliably with individual patient data. \emph{Intensive Care Medicine}. July 2023. doi:\href{https://doi.org/10.1007/s00134-023-07163-z}{10.1007/s00134-023-07163-z}}

\leavevmode\vadjust pre{\hypertarget{ref-metagear}{}}%
\CSLLeftMargin{251. }%
\CSLRightInline{Lajeunesse MJ. Facilitating systematic reviews, data extraction, and meta-analysis with the metagear package for r. 2016;7:323-330.}

\leavevmode\vadjust pre{\hypertarget{ref-Moher2015}{}}%
\CSLLeftMargin{252. }%
\CSLRightInline{Moher D, Shamseer L, Clarke M, et al. Preferred reporting items for systematic review and meta-analysis protocols (PRISMA-P) 2015 statement. \emph{Systematic Reviews}. 2015;4(1). doi:\href{https://doi.org/10.1186/2046-4053-4-1}{10.1186/2046-4053-4-1}}

\leavevmode\vadjust pre{\hypertarget{ref-PRISMA2020-2}{}}%
\CSLLeftMargin{253. }%
\CSLRightInline{Haddaway NR, Page MJ, Pritchard CC, McGuinness LA. PRISMA2020: An r package and shiny app for producing PRISMA 2020-compliant flow diagrams, with interactivity for optimised digital transparency and open synthesis. 2022;18:e1230. doi:\href{https://doi.org/10.1002/cl2.1230}{10.1002/cl2.1230}}

\leavevmode\vadjust pre{\hypertarget{ref-PRISMA2020}{}}%
\CSLLeftMargin{254. }%
\CSLRightInline{Haddaway NR, Page MJ, Pritchard CC, McGuinness LA. PRISMA2020: An r package and shiny app for producing PRISMA 2020-compliant flow diagrams, with interactivity for optimised digital transparency and open synthesis. 2022;18:e1230. doi:\href{https://doi.org/10.1002/cl2.1230}{10.1002/cl2.1230}}

\leavevmode\vadjust pre{\hypertarget{ref-Wallisch2022}{}}%
\CSLLeftMargin{255. }%
\CSLRightInline{Wallisch C, Bach P, Hafermann L, et al. Review of guidance papers on regression modeling in statistical series of medical journals. Mathes T, ed. \emph{PLOS ONE}. 2022;17(1):e0262918. doi:\href{https://doi.org/10.1371/journal.pone.0262918}{10.1371/journal.pone.0262918}}

\leavevmode\vadjust pre{\hypertarget{ref-Lynggaard2022}{}}%
\CSLLeftMargin{256. }%
\CSLRightInline{Lynggaard H, Bell J, Lösch C, et al. Principles and recommendations for incorporating estimands into clinical study protocol templates. \emph{Trials}. 2022;23(1). doi:\href{https://doi.org/10.1186/s13063-022-06515-2}{10.1186/s13063-022-06515-2}}

\leavevmode\vadjust pre{\hypertarget{ref-Althouse2021}{}}%
\CSLLeftMargin{257. }%
\CSLRightInline{Althouse AD, Below JE, Claggett BL, et al. Recommendations for Statistical Reporting in Cardiovascular Medicine: A Special Report From the American Heart Association. \emph{Circulation}. 2021;144(4). doi:\href{https://doi.org/10.1161/circulationaha.121.055393}{10.1161/circulationaha.121.055393}}

\leavevmode\vadjust pre{\hypertarget{ref-Lee2021}{}}%
\CSLLeftMargin{258. }%
\CSLRightInline{Lee KJ, Tilling KM, Cornish RP, et al. Framework for the treatment and reporting of missing data in observational studies: The Treatment And Reporting of Missing data in Observational Studies framework. \emph{Journal of Clinical Epidemiology}. 2021;134:79-88. doi:\href{https://doi.org/10.1016/j.jclinepi.2021.01.008}{10.1016/j.jclinepi.2021.01.008}}

\leavevmode\vadjust pre{\hypertarget{ref-Vickers2020}{}}%
\CSLLeftMargin{259. }%
\CSLRightInline{Vickers AJ, Assel MJ, Sjoberg DD, et al. Guidelines for Reporting of Figures and Tables for Clinical Research in Urology. \emph{Urology}. 2020;142:1-13. doi:\href{https://doi.org/10.1016/j.urology.2020.05.002}{10.1016/j.urology.2020.05.002}}

\leavevmode\vadjust pre{\hypertarget{ref-assel2019}{}}%
\CSLLeftMargin{260. }%
\CSLRightInline{Assel M, Sjoberg D, Elders A, et al. Guidelines for Reporting of Statistics for Clinical Research in Urology. \emph{Journal of Urology}. 2019;201(3):595-604. doi:\href{https://doi.org/10.1097/ju.0000000000000001}{10.1097/ju.0000000000000001}}

\leavevmode\vadjust pre{\hypertarget{ref-Gamble2017}{}}%
\CSLLeftMargin{261. }%
\CSLRightInline{Gamble C, Krishan A, Stocken D, et al. Guidelines for the Content of Statistical Analysis Plans in Clinical Trials. \emph{JAMA}. 2017;318(23):2337. doi:\href{https://doi.org/10.1001/jama.2017.18556}{10.1001/jama.2017.18556}}

\leavevmode\vadjust pre{\hypertarget{ref-Lang2015}{}}%
\CSLLeftMargin{262. }%
\CSLRightInline{Lang TA, Altman DG. Basic statistical reporting for articles published in Biomedical Journals: The {``}Statistical Analyses and Methods in the Published Literature{''} or the SAMPL Guidelines. \emph{International Journal of Nursing Studies}. 2015;52(1):5-9. doi:\href{https://doi.org/10.1016/j.ijnurstu.2014.09.006}{10.1016/j.ijnurstu.2014.09.006}}

\leavevmode\vadjust pre{\hypertarget{ref-Weissgerber2015}{}}%
\CSLLeftMargin{263. }%
\CSLRightInline{Weissgerber TL, Milic NM, Winham SJ, Garovic VD. Beyond Bar and Line Graphs: Time for a New Data Presentation Paradigm. \emph{PLOS Biology}. 2015;13(4):e1002128. doi:\href{https://doi.org/10.1371/journal.pbio.1002128}{10.1371/journal.pbio.1002128}}

\leavevmode\vadjust pre{\hypertarget{ref-Sauerbrei2014}{}}%
\CSLLeftMargin{264. }%
\CSLRightInline{Sauerbrei W, Abrahamowicz M, Altman DG, Cessie S, Carpenter J. STRengthening Analytical Thinking for Observational Studies: the STRATOS initiative. \emph{Statistics in Medicine}. 2014;33(30):5413-5432. doi:\href{https://doi.org/10.1002/sim.6265}{10.1002/sim.6265}}

\leavevmode\vadjust pre{\hypertarget{ref-groves2008}{}}%
\CSLLeftMargin{265. }%
\CSLRightInline{Groves T. Research methods and reporting. \emph{BMJ}. 2008;337(oct22 1):a2201-a2201. doi:\href{https://doi.org/10.1136/bmj.a2201}{10.1136/bmj.a2201}}

\leavevmode\vadjust pre{\hypertarget{ref-stratton2005}{}}%
\CSLLeftMargin{266. }%
\CSLRightInline{Stratton IM, Neil A. How to ensure your paper is rejected by the statistical reviewer. \emph{Diabetic Medicine}. 2005;22(4):371-373. doi:\href{https://doi.org/10.1111/j.1464-5491.2004.01443.x}{10.1111/j.1464-5491.2004.01443.x}}

\leavevmode\vadjust pre{\hypertarget{ref-Mansournia2021}{}}%
\CSLLeftMargin{267. }%
\CSLRightInline{Mansournia MA, Collins GS, Nielsen RO, et al. A CHecklist for statistical Assessment of Medical Papers (the CHAMP statement): explanation and elaboration. \emph{British Journal of Sports Medicine}. 2021;55(18):1009-1017. doi:\href{https://doi.org/10.1136/bjsports-2020-103652}{10.1136/bjsports-2020-103652}}

\leavevmode\vadjust pre{\hypertarget{ref-Gil-Sierra2020}{}}%
\CSLLeftMargin{268. }%
\CSLRightInline{Gil-Sierra MD, Fénix-Caballero S, Abdel kader-Martin L, et al. Checklist for clinical applicability of subgroup analysis. \emph{Journal of Clinical Pharmacy and Therapeutics}. 2019;45(3):530-538. doi:\href{https://doi.org/10.1111/jcpt.13102}{10.1111/jcpt.13102}}

\leavevmode\vadjust pre{\hypertarget{ref-Altman2008}{}}%
\CSLLeftMargin{269. }%
\CSLRightInline{Altman DG, Simera I, Hoey J, Moher D, Schulz K. EQUATOR: reporting guidelines for health research. \emph{The Lancet}. 2008;371(9619):1149-1150. doi:\href{https://doi.org/10.1016/s0140-6736(08)60505-x}{10.1016/s0140-6736(08)60505-x}}

\end{CSLReferences}

\includepdf[pages=-,noautoscale=false, fitpaper=true, width=8.27in, height=11.69in, frame=false, trim=0mm 0mm 0mm 0mm, clip, offset=0mm 0mm]{covers/Cover_4.pdf}

\end{document}
